
\chapter{Gear}
\label{cha:gear}

The accelerated technological levels of \emph{Eclipse Phase}
enable a number of devices for personal enhancement,
survival, and other uses.


\section{Equipment rules}
\label{sec:equipment-rules}

The following rules apply to all technological items in
\emph{Eclipse Phase}.


\subsection{Acquiring gear}
\label{sec:acquiring-gear}

During character creation, players purchase gear for
their characters using the credits they have during
the character creation process. Once play begins,
however, characters must obtain any equipment they
need the usual way: by buying, borrowing, making,
or stealing it.

In the inner system, hypercorp, and Jovian Republic
settlements—and other places where capitalism still
reigns—gear acquisition is simply a matter of fi nding
a seller and buying it. Each item has a listed cost, from
Trivial to Expensive, as noted on the Gear Costs table.
Due to local availability of resources, supply and
demand, and legalities, these listed costs are meant
to be approximations. When no other factors apply,
the listed Average Cost for that category can be used.
Otherwise the gamemaster should modify the item’s
worth as they see fi t, according to local economic factors,
while still keeping it within that cost category
range. The Cost Modifi ers table lists out some suggested
changes to an item’s cost, but these are simply
recommendations, and can be ignored or followed as
the gamemaster deems fi t. The exact local conditions
are largely up to the gamemaster to determine, as best
fi ts their game.

In some circumstances, characters may attempt to
haggle over gear prices. This is best handled as roleplaying,
but the gamemaster may also call for an
Opposed Persuasion Test (or possibly an Intimidation
Test). The character who wins may increase or reduce
the price by 10\% per 10 points of MoS.

In the outer system, anarchist, Titanian, scum, and
other habitats that use the reputation economy, characters
must rely on their rep scores to acquire the goods
and services they need. The mechanics for this are covered
under \emph{Reputation and Social Networks}, p. 285.

Characters are of course free to get their hands
on equipment by any other means they devise—con
schemes, borrowing from friends, and outright
theft, with all of the appropriate tests and consequences.
In some cases, acquiring gear may be an
adventure unto itself.

\begin{table}
\begin{tabular}{|l|r|r|}
\hline
\multicolumn{3}{|c|}{\textbf{Gear costs}}			\\
\hline
\textbf{Category}	& \textbf{Range (in credits)}	& \textbf{Average (in credits)} \\
\hline
Trivial			& 1–99					& 50 \\
\hline
Low				& 100–499					& 250 \\
\hline
Moderate			& 500–1499				& 1000 \\
\hline
High				& 1500–9999				& 5000 \\
\hline
Expensive			& 10000+					& 20000 \\
\hline
\end{tabular}
\label{tab:gear-costs}
\end{table}

\begin{table}
\begin{tabular}{|l|r|}
\hline
\multicolumn{2}{|c|}{\textbf{Gear cost modifiers}}			\\
\hline
\textbf{Economic factor}	& \textbf{Suggested cost modifier} \\
\hline
Item Stolen			& –50\% \\
\hline
Item Used				& –25\% \\
\hline
Item Restricted		& +25\% \\
\hline
Item Illegal			& +50\% \\
\hline
Item Scarce			& +25\% \\
\hline
Item Extremely Rare		& +50\% \\
\hline
Item Common			& –25\% \\
\hline
\end{tabular}
\label{tab:gear-cost-modifiers}
\end{table}

\subsubsection{Fabricating gear}

Thanks to nanofabrication technology, characters may
also create their own equipment using cornucopia
machines and similar nanofab devices (p. 327). The
character must have the appropriate blueprints to do
so, whether they come with the fabber, are bought
legitimately or on the black market, acquired with rep,
or found online. Characters may also code their own
blueprint desires, using the Programming: Nanofabrication
skill.


\subsection{Gear modifiers}
\label{sec:gear-modifiers}

In the technological future, gear is a necessity. In many
cases, use of equipment provides no bonuses, it simply
allows a character to perform a task they would otherwise
be unable to do. For example, it is impossible
to pick a mechanical lock without lockpick or some
sort of tool.

In other cases, however, gear provides a bonus to
the task at hand. Climbing a wall may be possible
without tools, but if you happen to have gecko gloves
or other climbing gear, it’s going to be a lot easier. The
specifi c modifi er applied is usually noted in the gear
item’s description, typically ranging from +10 to +30.

\subsubsection{Gear quality}

In both of the situations above, it is possible to have
items that are of either exceptional or inferior quality,
with corresponding positive or negative modifi ers. The
gear may be well-crafted, state-of-the-art, cutting-edge
experimental, or simply top-of-the-line, applying an
additional +10 to +30. Or it may be outdated, shoddy,
or in disrepair, infl icting a –10 to –30 modifi er (in
some cases canceling out the basic gear bonus).

\subsubsection{Gear sizes}

On occasion, you’ll need to know how small or large
a certain piece of equipment is. Though this is largely
something the gamemaster can wing on the fl y using
common sense, we’ve listed sizes for many gear items
that are unusual or so futuristic that the average player
may not have a feel for what dimensions the tech
likely is. These size categories are listed on the Gear
Sizes table (p. 297). These sizes should be considered
approximations, as depending on the manufacturer
and process, some items may be smaller or larger than
similar items. It is also important to keep in mind that
as technology advances, the size and components of
various equipment items shrink, so when in doubt, go
with smaller.


\begin{table}
\begin{tabularx}{\textwidth}{|l|X|}
\hline
\multicolumn{2}{|c|}{\textbf{Gear sizes}}			\\
\hline
\textbf{Size category}	& \textbf{General dimensions and notes} \\
Nano					& So small that the item cannot be seen without the aid of a microscope or nanoscopic vision (p. 311), and may not be manipulated without fractal digits (p. 311) or similar tools. \\
\hline
Micro				& Anything ranging from the size of a barely visible small dot to an average insect. \\
\hline
Mini					& Mini items may be concealed within someone’s palm or small pockets. \\
\hline
Small				& Small items may be held in one hand and concealed in normal pockets.\\
\hline
Medium				& Medium size items are cumbersome to hold with one hand, ranging from the size of a 2-liter bottle to the size of a medium dog. They do not fi t in pockets, but they may be concealed by larger coverings. \\
\hline
Large				& Roughly human-sized. \\
\hline
Huge					& Vehicles and other more massive objects. \\
\hline
\end{tabularx}
\label{tab:gear-sizes}
\end{table}


\subsubsection{Mass and encumbrance}

A character who is carrying too much gear should
be slowed down, suffering negative modifi ers both
to their movement rates and their skill tests. Rather
than micromanaging the weights of individual pieces
of equipment, however, this matter is largely left to
the gamemaster’s discretion, using common sense. If a
character loads up beyond reason, apply modifi ers as
seem appropriate. The gamemaster should, however,
keep in mind that many of the manufacturing materials
used in \emph{Eclipse Phase} allow for items that are
much lighter than current standards without any loss
of durability or function (see \emph{Future Materials}, p. 298).
Likewise, characters in low or microgravity environments
can carry much larger loads.

\subsubsection{Concealing gear}

Characters may attempt to conceal items on their
person, hoping at least to hide them from casual
notice if not an intensive search. To determine how
effectively the character conceals the equipment, make
a Palming Test and note the MoS (the gamemaster
may wish to roll this secretly). Whenever another
character has a chance to notice the concealed item,
they must succeed in a Perception Test and achieve
a higher MoS than was scored on the Palming Test.
The gamemaster should apply modifi ers to both tests
as appropriate. For example, concealing a large item
like a sword would be diffi cult (–30), whereas wearing
concealing clothing like a longcoat or multi-pocketed
jumpsuit would help (+20). Likewise, a character
who is not actively looking is less likely to notice the
hidden gear (–30), whereas someone who conducts a
physical search (+30) or who has enhanced vision to
pierce protective layers will fare better.


\subsection{Design and fashion}
\label{sec:design-fashion}

Many objects in \emph{Eclipse Phase} closely resemble their
early 21st century equivalents—a bottle of soda is still
a transparent container holding a brightly colored
liquid, clothing is obviously something you wear, and
a knife still consists of a blade and a handle. The materials,
processes, and mindsets that go into making
them, however, are quite different. To start, very few
items look have a uniform, mass-produced look, even
if they were. The procedures of minifacturing and
nanofabrication allow every individual item to be
manufactured with a unique (or at least different)
look. In areas with anarchist/reputation economies, in
fact, where personal possessions have very little intrinsic
value, expression and creativity are favored and so
many items are artistically personalized (and actual
hand-crafted items are rare and prized). Likewise,
almost all equipment is designed with ergonomics
and ease-of-use prioritized, so gear with soft curves,
pleasing colors, and form-fi tting shapes are common.
Many items of personal technology, such as fl ashlights
or small tools, are made in the form of ovoids that
fi t comfortably in the user’s hand or in similar forms
that can be easily worn or attached to clothing. To
someone from the 20th century, many common
devices look like oddly colored rocks or decorative
pieces of plastic or ceramic (in fact, many such items
are referred to as “blobjects” by older transhumans).

The materials used to create everyday items are
also advanced, ranging from aerogel and graphene
to smart materials (p. 298) and exotic metamaterials
with unusual physical properties. In practice, this
means that most items are light, durable (with both
tensile strength and/or fl exibility, as needed), waterproof,
dirt-repellent, and self-cleaning. Most gear is
also designed with zero-G or microgravity functionality
in mind, and can easily be clipped, tethered, or
stuck to a surface with grip pads.

Almost all gear available in \emph{Eclipse Phase} is also
available in forms that are wearable/usable by uplifted
animals and non-humanoid morphs, such
as novacrabs, slitheroids, and so on. Even if such
customized gear is not immediately available, it is
usually not diffi cult to nanofabricate. Smart materials
(p. 298) also make interoperability between different
morphs easy.

\subsubsection{Interface}

It is not uncommon for everyday devices to have no
visible controls as they are designed to be operated via
radio broadcasts from the user’s ecto or mesh inserts.
Any items crafted for use in emergency, combat, survival,
or exploration situations, however, will feature
basic physical controls, just in case. Physical interfaces
are typically controlled by touch pads that are nothing
more than colored spots on the device’s surface,
though some may also project a holographic interface
display. Most equipment of this sort can can also be
voice-activated and controlled.

Almost all devices are loaded with a complete set
of help fi les and tutorials. Most electronics are also
mesh-capable and equipped with specialized AIs (see
\emph{Meshed Gear}, next page).

\subsubsection{Smart materials}

Many common items of technology are made from
so-called smart materials. These devices contain—or
sometimes consist entirely of—many small nanomachines
that can both move and reshape themselves
to alter the object’s shape, color, and texture. For
example, smart clothing can transform from a suit of
specialized cold weather clothing suitable for the Martian
poles in winter to a fashionable suit in the latest
style due to hundreds of thousands of tiny nanomachines
in the clothing that shift and move to reshape
the garment. Similarly, a tool made of smart materials
can switch from a powered screwdriver to a wrench
or a hammer, as the nanomachines move around and
completely reshape the tool. Smart materials all contain
specialized advanced nanomachine generators (p.
328) that keep them in perfect repair as long as they
are regularly recharged.


\subsection{Future materials}
\label{sec:future-materials}

Many materials are available and commonly used in
\emph{Eclipse Phase} that are rare, theorized, or unheardof
today. The following entries note some of the
more interesting.

\subsubsection{Aerogel}

Low-density, solid-state “Frozen smoke” is made by
carefully foaming various materials, typically glasses
or ceramics, to an ultra-low density state. Aerogel
is semi-transparent and light-weight, feels like styrofoam,
but acts as an incredible insulator against
heat and cold. It is commonly used in habitats.

\subsubsection{Diamond}

Artifi cial diamond is lightweight and super-strong,
has an extremely high melting point, and has nearperfect
thermal conductivity. This makes it an ideal
substance for hardening coated surfaces (armor)
and creating super-tough diamond machinery.

\subsubsection{Fullerenes/Fullerites}

Fullerenes are molecular carbon structures (known
as buckyballs, carbon nanotubes, and graphene)
that are extremely strong (vastly stronger by weight
than steel), heat-resistant, and can be either insulative
or superconductive. This makes them useful in
equipment as diverse as armor, electronics, sensor
systems, or the cables of space elevators.

\subsubsection{Metallic foam}

Metal foam is created by adding foaming agents
to liquid metals, resulting in extremely lightweight
metallic structures—light enough to fl oat on water.
Ideal for habitat construction and fl oating cities.

\subsubsection{Metallic glass}

Metallic glass are metals highly alloyed to possess a
disordered (rather than crystalline) atomic structure
with unique combinations of stiffness and strength,
making it a good wear surface and alternative to
ceramics in armor. It is also useful for its unusual
(for a metal) electrical resistance properties.

\subsubsection{Metamaterials}

Metamaterials have unusual physical properties
(usually electromagnetic) due to their structure,
such as having a negative refractive index. Metamaterials
are used to create invisibility cloaks (p. 316),
superlenses, phased array optics, and impressive
2-D holograms.

\subsubsection{Refactory metals}

These metallic alloys have extremely high melting
points, making them ideal for extremely hot
engine systems, atmospheric entry vehicles, and
hypersonic craft.

\subsubsection{Transparent alumina}

In transparent form, this ceramic is often known as
sapphire. Transparent alumina is harder than steel
and zero-g casting techniques allow for intriguing
transparent construction designs, so long as its poor
tensile strength is respected.


\section{Meshed gear}
\label{sec:meshed-gear}

Almost all technology in \emph{Eclipse Phase} is designed
to be operated via radio signals from the user’s basic
implant, although models usable by characters without
basic implants are also available. In addition all
devices contain a nearly microscopic computer and
radio link (known as a “voice”) that allows the user to
easily locate the object and that reports on the condition
of the object or device, how to properly use and
care for it, as well as telling the user when it needs
to be repaired and how. Most are discrete and highly
useful, but cheaply made goods sometimes have overly
annoying voices.

This means that almost all devices can be accessed
via the mesh or directly if within radio range. This
makes them vulnerable to hacking and intrusion attempts
(p. 254) as well as radio jamming (p. 262).
Many devices are, however, publicly accessible (see
\emph{Spimes}, p. 238). Meshed gear may also be tracked
through the mesh (p. 251). For privacy and security,
these devices are often slaved to other systems (see
\emph{Slaving Devices}, p. 248); devices worn/carried by
characters are usually made part of the personal area
network and slaved to the character’s mesh inserts/
ecto. For more info on meshed devices, see the \emph{Mesh
chapter}, p. 234.

Many devices come equipped with AIs, who are
equipped with skillsofts that enable them to operate
the device on their own, as according to voiced
instructions or commands issued through the net. AIs
are described on p. 264 and p. 331.


\subsection{Radio and sensor ranges}
\label{sec:radio-sensor-ranges}

In \emph{Eclipse Phase}, almost all devices are equipped with
small radios so that they may be meshed. Likewise,
many pieces of gear are equipped with sensors such as
cameras, microphones, or other detectors. The Radio
and Sensor Ranges table notes what range these devices
operate at.

\begin{table}
\begin{tabularx}{\textwidth}{|l|l|l|X|}
\hline
\multicolumn{4}{|c|}{\textbf{Radio and sensor ranges}}			\\
\hline
\textbf{Size category}	& \textbf{Urban range}	& \textbf{Urban range}	& \textbf{Examples} \\
\hline
Nano 				& 20 meters 			& 100 meters			& Smart Dust, Nanobot/Microbot Swarms \\
Micro				& 50 meters			& 500 meters			& Microbugs \\
Mini					& 1 kilometer			& 20 kilometers		& Mesh Inserts \\
Small				& 5 kilometers			& 50 kilometers		& Ectos, Miniature Radio Farcasters, Portable Sensors \\
Medium				& 25 kilometers		& 250 kilometers		& Radio Boosters, Vehicle Sensors \\
Large				& 500 kilometers		& 5000 kilometers		& \\
\hline
\end{tabularx}
\label{tab:radio-sensor-ranges}
\end{table}


\subsection{Power}
\label{sec:power}

All of the powered devices in \emph{Eclipse Phase} require
electricity to function. With rare exceptions, most of
them rely on either solar cells or powerful batteries.
These batteries are high-density, room-temperature
superconductors with 25 times the capacity of the
best batteries in common use in the early 21st century.
Such batteries may also be constructed so that they
are fl exible, printed on devices, or woven into fabric.
They are good for 100–500 hours of use, and will alert
the user when they start running low. More powerful
radio-isotope nuclear batteries are also available,
heavily shielded so they do not emit radiation and
good for 3 years or more of use.

In short, power should rarely be an issue in \emph{Eclipse
Phase} games, unless it happens to fi t the plot. Power
failure could also result from a critical failure roll.


\section{Personal augmentation}
\label{sec:personal-augmentation}

\subsection{Standard augmentations}
\label{sec:std-augmentations}

\subsection{Bioware}
\label{sec:bioware}

\subsection{Using enhanced senses}
\label{sec:using-enhanced-senses}

\subsection{Cyberware}
\label{sec:cyberware}

\subsection{Synthmorphs and bioware}
\label{sec:synthmorphs-bioware}

\subsection{Nanoware}
\label{sec:nanoware}

\subsection{Cosmetic mods}
\label{sec:cosmetic-mods}

\subsection{Robotic enhancements}
\label{sec:robotic-enhancements}

\subsection{Armor}
\label{sec:armor}

\subsection{Armor mods}
\label{sec:armor-mods}

\subsection{Communications}
\label{sec:communications}

\subsection{Neutrino communicators}
\label{sec:neutrino-communicators}

\subsection{Quantum farcasters}
\label{sec:quantum-farcasters}

\subsection{Quantum entanglement communication}
\label{sec:quantum-entanglement-communication}

\subsection{Covert and espionage technologies}
\label{sec:covert-espionage-tech}

\subsection{Bugs and surveilance}
\label{sec:bugs-surveilance}

\section{Drugs, chemicals and toxins}
\label{sec:drugs-chemicals-toxins}

\subsection{Substance rules}
\label{sec:substance-rules}

\subsection{Drugs}
\label{sec:drugs}

\subsection{Nanodrugs}
\label{sec:nanodrugs}

\subsection{Other nanodrugs}
\label{sec:other-nanodrugs}

\subsection{Sample petals}
\label{sec:sample-petals}

\subsection{Narcoalgorithms}
\label{sec:narcoalgorithms}

\subsection{Chemicals}
\label{sec:chemicals}

\subsection{Toxins}
\label{sec:toxins}

\subsection{Nanotoxins}
\label{sec:nanotoxins}

\subsection{Pathogens}
\label{sec:pathogens}

\subsection{Psi drugs}
\label{sec:psi-drugs}

\section{Everyday technology}
\label{sec:everyday-tech}

\section{Nanotechnology}
\label{sec:nanotech}

\subsection{Basic nanotechnology}
\label{sec:basic-nanotech}

\subsection{Advanced nanotechnology}
\label{sec:advanced-nanotech}

\subsection{Pets}
\label{sec:pets}

\subsection{Scavenger tech}
\label{sec:scavenger-tech}

\subsection{Services}
\label{sec:services}

\subsection{Software}
\label{sec:software}

\subsection{Survival gear}
\label{sec:survival-gear}

\section{Weapons}
\label{sec:weapons}

\subsection{Malee weapons}
\label{sec:melee-weapons}

\subsection{Kinetic weapons}
\label{sec:kinetic-weapons}

\subsection{Brand name weapons and combined arms}
\label{sec:brand-weapons-combined}

\subsection{Beam weapons}
\label{sec:beam-weapons}

\subsection{Seekers}
\label{sec:seekers}

\subsection{Spray weapons}
\label{sec:spray-weapons}

\subsection{Grenades and seekers}
\label{sec:grenades-seekers}

\subsection{Exotic ranged weapons}
\label{sec:exotic-ranged-weapons}

\subsection{Weapon accessories}
\label{sec:weapon-accessories}

\section{Robots and vehicles}
\label{sec:robots-vehicles}

\subsection{Aircraft}
\label{sec:aircraft}

\subsection{Exoskeletons}
\label{sec:exoskeletons}

\subsection{Groundcraft}
\label{sec:groundcraft}

\subsection{Personal vehicles}
\label{sec:personal-vehicles}

\subsection{Robots}
\label{sec:robots}

\subsection{Spacecraft}
\label{sec:spacecraft}

%%% Local Variables: 
%%% mode: latex
%%% TeX-master: "ep"
%%% End: 
