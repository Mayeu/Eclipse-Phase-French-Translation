\chapter{Skills}
\label{cha:skills}

In a setting where physical looks and capabilities are easily changed at the push of a button, who you are and what you know is more important than any inborn ability. Skills represent the knowledge your character has, the accumulated set of experience, education, and inherent know-how possessed by each and every sentient transhuman in Eclipse Phase. They are what allow you to sneak into a hypercorp station, disable the security systems, hack the mesh hub, and then impersonate security personnel to make your escape. Your skills represent the one thing you have no matter what you look like or where you find yourself. When your characters explore what they can do, their skills, or lack thereof, often determine the margin between success and failure.

Having a well-rounded set of skills is vital to survival and success in Eclipse Phase. The skills below encompass a wide selection of talents, enough so that each character can be unique in their abilities and knowledge.


\section{Skills Overview}
\label{sec:skills-overview}

Skills are divided into aptitudes and learned skills (see Character Skills, p. 123). Most (but not all) learned skills are built on and linked to an aptitude. If a character lacks the specifi c skill needed in a situation, they may default to the linked aptitude. You may also
choose to specialize in certain skills (see Specializations, p. 123), refl ecting an enhanced knowledge of a particular aspect of a certain skill.


\subsection{Core skills: Aptitudes}
\label{sec:core-skills-aptitudes}

Aptitudes represent inherent skills and abilities acquired at birth or during the course of growing up. Aptitudes are sometimes used for tests, but their primary use is determining the starting point at which learned skills are developed. Aptitudes determine the starting value of their linked skills. For example, a character with Somatics aptitude 10 who wishes to purchase points in the Freerunning skill (which is linked to Somatics) would start with a Freerunning rank of 10 and then buy additionally points in that skill. Aptitudes are also used when a character doesn’t posses knowledge of a needed skill (see Defaulting, p.
116). Aptitudes represent the basic knowledge that a character has acquired regarding rudimentary use of that skill. They may not have ever received any formal training with the skill, but they can still attempt to use it. Aptitudes range in value from 1 to 30, with 10 being the unaugmented human average and 15 representing the average of most genetically modifi ed transhumans. Since aptitudes represent untrained ability, they are capped at a maximum rating of 30. There are seven different aptitudes that all players possess. These aptitudes are purchased during character creation (p. 128), but depending on the morph the character is currently inhabiting, they may find their aptitudes capped by the quality of the morph (see p. 124).


\subsection{Learned skills}
\label{sec:skills:learned-skills}

A player’s learned skills are the most important part of their character, representing the acquired knowledge they carry with them from morph to morph, knowledge that plays a fundamental role in helping define the person’s ego. Learned skills encompass nearly any skill that you might need to use in Eclipse Phase, and they range in value from 0 to 99. All learned skills have a linked aptitude that is used to calculate their initial value, and which is also defaulted to if the player does not have that particular skill.


\subsubsection{Skill categories}
\label{sec:skills:skill-categories}

Each learned skill is classified as either an Active skill or a Knowledge skill. Active skills represent skills that typically require physical actions and are used in action scenes within game play. Knowledge skills are more knowledge-based and intellectual, representing ideas and facts. Knowledge skills may play a less dramatic role in certain action-oriented game play moments, but they flesh out the character’s background and interests and are integral to roleplaying interactions. Active and Knowledge skills are purchased separately during character creation. Active skills are further divided into Combat, Mental, Physical, Psi, Social, Technical, and Vehicle skills. Certain traits and abilities may apply to specific categories.


\subsubsection{Field skills}
\label{sec:skills:field-skills}

Some learned skills are fi eld skills, meaning that when this skill is chosen a particular fi eld of emphasis must also be selected. For example, the skill of Academics requires the character to specify a specific academic discipline in which they are knowledgeable, such as Biology, Chemistry, or Xenosociology. Field skills are written as “[skill]: [fi eld];” for example: “Art: Painting.” Field skills can be taken multiple times, choosing a different area of emphasis each time, reflecting skills in different fi elds; that is to say, each fi eld is a separate skill. Several suggested fi elds are listed for each field
skill, but gamemasters and players may also cooperate to create others that fit their games. Field skills may also have specializations; for example, Professional: Accounting (Money Laundering).


\subsubsection{Psi skills}
\label{sec:skills:psi-skills}

Psi refers to the ability to perceive and manipulate biological minds via psi waves and/or other inexplicable phenomena. Due to the uniqueness of this ability, characters that wish to wield psi must acquire the Psi trait (p. 147). Psi use also requires a number of specialized skills (Control, Psi Assault, and Sense) that reflect special training characters acquire to tap into their psi powers. Psi skills may not be defaulted on; the only way to use a psi skill is to possess the trait
along with training in that skill. For more details, see Psi, p. 220.


\subsection{Specializations}
\label{sec:skills:specializations}
Any character may opt to specialize in a given skill (see Specializations, p. 123). This specialization reflects increased knowledge in one particular aspect of the skill. Many of the skills offered below include sample specializations. Gamemasters and players are encouraged to develop other specialization ideas together for their campaigns. Specialization provides a +10 modifier when using that skill in a situation appropriate to that specialization.


\section{Using skills}
\label{sec:skills:using-skills}

Whenever a character wants to do something using a skill, they must succeed at a skill test (see Making Tests, p. 115). The difficulty of the action is applied as a modifi er, as are any other extenuating circumstances that may affect the test (see Difficulty and Modifiers, p. 115). As with other types of tests, all skill tests are successful when the character rolls less than or equal to the test’s target number after any modifi ers have been applied. In the case of skill tests, the target number is the character’s skill rating with that particular skill. Modifiers representing difficulty and other factors are applied directly to the target number (see Difficulty and Modifiers, p. 115). A roll of a 00 is always a success, regardless of modifiers, and a result of 99 is always a failure, again despite any modifiers that may increase a character’s target number over 100. Standard critical success and failure rules apply to skill tests (see Criticals: Rolling Doubles, p. 116), so any time a character rolls a double (i.e. 00, 11, 22, 33, etc.) they score a critical success or failure.


\subsection{Defaulting}
\label{sec:skills:defaulting}

Sometimes you lack the skill needed in a certain situation. In these instances, characters may default their skill test to the linked aptitude. This refl ects the fact that most learned skills are developed from some sort of baseline physical ability. Even though you may not know how to do something, you’ve likely seen how it’s done at some point or have some idea of how to do it, or can at least take a shot at it. Naturally, you’re not as good as someone who has training in that skill, but it still allows you to make an attempt. Not all skills can be defaulted. Some skills are simply too complex or obscure, or demand special knowledge or ability, for someone to attempt their use untrained. For example, brain surgery or most psi skills are simply beyond anyone who doesn’t have that ability or the knowledge of what they’re attempting.


\subsubsection{Defaulting to field skills}
\label{sec:skills:defaulting-to-field-skills}

In some cases, a character may not possess the particular field skill that a test calls for, but they may be skilled in another related field. For example, a test to conduct an alien autopsy might call for an Academics: Xenobiology roll, but a character who doesn’t have that skill may be allowed to default to Academics: Biology instead. The gamemaster decides if and when to allow this, perhaps applying a modifi er to the test based on the difference between fi elds.


\subsubsection{Defaulting to related skills}
\label{sec:skills:defaulting-to-related-skills}

If the gamemaster allows it, characters may default to a related skill that also has some relevance to the test at hand. For example, a character skilled in Kinetic Weapons might not be trained in the use of a laser, but they know enough to point at the target and pull the trigger. Likewise, a character might not be skilled in Investigation, but the gamemaster could still allow them to use their Perception skill instead in order to realize that a body had been moved from the place where it had been shot. In situations like this, when the gamemaster allows defaulting to a related skill, a –30 modifi er should be applied to the test.

\begin{quotation}
Srit is wandering through a black market souk on Mars, trying to find a particular piece of sensory equipment. The gamemaster calls for a Scrounging Test, but Srit does not have that skill. She could default her INT of 22, but instead she asks the gamemaster if she can default to the related skill of Perception, which she has at 82. The gamemaster agrees, and so Srit rolls against a target number of 52 (82 - 30).
\end{quotation}



\subsection{Complimentary skills}
\label{sec:skills:complimentary-skills}

Sometimes more than one skill may apply to a particular test, or knowledge in one area can aid your skill in another. In this case, the gamemaster may apply a modifier to the skill test based on the strength of the complementing skill, as noted on the Complementary
Skill Bonus table.

\begin{quotation}
Dav is hoping to persuade a brinker pilot to take him to an isolated habitat that doesn’t welcome visitors. To impress upon the pilot that he is a friend of these particular isolates, he calls on his knowledge of their particular cultural practices (Interests: Religious Cults skill at 45). The gamemaster allows this and applies a +20 modifier to Dav’s Persuasion Test.
\end{quotation}

\begin{tabular}{|l|l|}
\hline
\multicolumn{2}{|c|}{Complimentary skill bonus} \\
\hline
Skill rating & Modifier \\
\hline
0-30		& +10 \\
31-50	& +20 \\
61+		& +30 \\
\hline
\end{tabular}


\subsection{Skill ranges}
\label{sec:skills:skill-ranges}

What is the difference between being a clumsy neophyte
wobbling in zero gravity and being a veteran
gliding effortlessly through space as though you were
dancing? The answer is training and skill. The greater
your skill, the more likely you are to not only succeed
at what you want to do, but succeed well.
Aptitudes in \emph{Eclipse Phase} range from 1 to 30,
while learned skills range from 0 to 99. These numbers
are an abstraction of the range of transhuman
abilities and traits. The Aptitude Range table provides
a breakdown of different aptitude levels and how they
relate to each other. Likewise, the Learned Skill Range
table provides an interpretation for the capabilities at
different skill levels.

\begin{tabular}{|l|l|l|l|l|l|l|l|l|}
\hline
\multicolumn{5}{|c|}{Aptitude range} \\
\hline
Rating & Assessment         & Somatics & Coordination & Reflexes  \\
\hline
1–5    & child average      & inept    & clumsy       & slow      \\
6–10   & adult average      & weak     & able         & paced     \\
11–15  & transhuman average & fit      & coordinated  & swift     \\
16–20  & enhanced           & enhanced & agile        & fast      \\
21–25  & superhuman         & gifted   & nimble       & lightning \\
26-30  & posthuman          & elite    & unerring     & synaptic  \\
\hline
Rating & Cognition   & Intuition       & Savvy       & Willpower \\
\hline
1–5    & limited     & aware           & awkward     & distracted \\
6–10   & intelligent & perceptive      & personable  & controlled \\
11–15  & bright      & sharp           & charismatic & focused    \\
16–20  & learned     & uncanny         & dazzling    & resolute   \\
21–25  & brilliant   & prescient       & mesmerizing & unwavering \\
26-30  & genius      & near omniscient & hypnotic    & unshakable \\
\hline
\end{tabular}

\begin{tabular}{|l|l|}
\hline
\multicolumn{2}{|c|}{Learned skill ranges} \\
\hline
Skill & Equivalence \\
\hline
00    & No exposure or familiarity, completely unskilled \\
10    & Very rudimentary knowledge \\
20    & Basic operator’s proficiency (driver’s license, gun permit, high school diploma) \\
30    & Hands-on experience, some professional training \\
40    & Basic professional certifi cation (police driving, army rifle certified, college diploma) \\
50    & Experience from professional-level work, some advanced training \\
60    & Expert competence (competitive driver, marksman, PhD) \\
70    & Experience from expert-level work, has had unique innovations or insights \\
80    & Worthy of being a system-renowned authority on the subject \\
90    & Nobel/Olympic/grandmaster \\
99    & Pinnacle of current understanding and innovation \\
\hline
\end{tabular}

\section{Aptitudes}
\label{sec:skills:aptitudes}

There are 7 aptitudes in Eclipse Phase, described on p.123. Each character has these aptitudes at a minimum rating of 1.

\subsection{Aptitude-only tests}
\label{sec:skills:aptitude-only-tests}

In rare cases, a test may call for using an aptitude only, rather than a learned skill. This should only occur when no learned skills are appropriate to the test, and these circumstances are usually noted in the rules. Aptitude-only tests must be handled carefully, as the range of aptitude ratings (1–30) is typically much smaller than the rating of learned skills (0–99). For this reason, most aptitude tests should use a target number equal to the aptitude x 3. In rare cases where the test is more difficult, the gamemaster may simply use an aptitude x 2, or just the straight aptitude rating. In some cases, more than one aptitude may be relevant to the test, and so they may be added together to derive the target number. What follows are a few examples where an aptitudeonly test might be appropriate. Gamemasters may call for similar tests in other situations, but learned skills should be used whenever possible.


\subsubsection{Brute strength}
\label{sec:skills:brute-strength}

Any test that involves simple brute strength can be handled as an SOM x 3 Test. Use this when smashing down a door, breaking an item in half, engaging in a tug-of-war, or lifting and carrying a heavy item.


\subsubsection{Catching thrown objects}
\label{sec:skills:catching-thrown-objects}

Use REF + (COO x 2) any time you need to catch a thrown or dropped object, such as catching a baseball, saving a priceless vase from shattering, or throwing back a grenade (see p. 200).


\subsubsection{Compusure and resolve}
\label{sec:skills:composure-and-resolve}

Various game situations may frighten your character, turn their stomach, horrify them, or rattle them to the core of their being. Use WIL x 3 to determine if your character can hold their ground, keep it down, and
pull themselves together.


\subsubsection{Escape artist}
\label{sec:skills:escape-artist}

If a character wants to slip free of physical bonds (such as ropes or handcuffs) or otherwise contort themselves (such as wriggling out from under a collapsed wall or an overturned vehicle), an Escape Artist Test may be
called for using the character’s COO + SOM. Apply modifi ers appropriate to the diffi culty of the situation. At the gamemaster’s discretion, escaping from some restraining situations may be considered a Task Action with an appropriate timeframe.


\subsubsection{Having an idea}
\label{sec:skills:having-an-idea}

Sometimes the players miss the obvious, or their personal mindset or biases cause them to misinterpret a situation or understand events in a way different from how the actual character would. In cases like this, the gamemaster can call for an INT x 3 or COG x 3 roll (whichever is more appropriate) to determine if the character gets an idea that will help them along. This test should be used sparingly, and only for assessing the character’s interpretation of obvious and known facts and details.


\subsubsection{Memorizing and remembering}
\label{sec:skills:memorizing-and-remembering}

Memories are what egos use to maintain continuity of self from morph to morph, but humans are notorious for remembering things incorrectly. Whenever characters attempt to recall a memory or memorize some piece of information, use COG x 3 to determine how well they succeed. Note that characters with eidetic memory (p. 146 or 301) or mnemonic augmentation (p. 307) have perfect memory, so no test is required.

%%% Local Variables: 
%%% mode: latex
%%% TeX-master: "ep"
%%% End: 
