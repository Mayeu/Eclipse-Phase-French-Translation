\chapter{Action et Combat} \label{chap:action-combat} 

les jeux de rôles s'articulent autour du drame et de l'aventure, et cela signifie générallement de 'laction et du combat. Les scènes d'action et de combat sont les moments où le niveau d'adrénaline explose et où la vie et la mission des personnages sont en jeu. 

Les scénarios d'action et de combat peuvent être confus a  jouer, particulièrement si le maître de jeu doit également garder la trace des actions de nombreux PNJ. Pour ces raisons, il est important que le maître de jeu détaille ces actions d'une manière que tout le monde peut visualier, que cela signifie utiliser des cartes et des figurines, un logiciel, un tableau effaçable ou de rapide schémas sur une feuille de papier. Bien que la plupart des règles pour gérer les actiosn et le combat soient abstraites - permettant à l'interprétation et et au bidouillage des résultats de s'adapter à l'histoire - de nombreux facteurs tatciques sont également inclut, et même les plus petits détails peuvent faire la différence. Il est aussi utilse d'avoir les capacités des PNJ prédétermiénes et de les gérer en tant que groupe lorsque c'est possible, pour réduire le fardeau du maître de jeu au milieu d'une situation mouvementée. 

\section{Tour d'action} \label{sec:combat-action-turns} 

Les scènes d'action dans Eclipse Phase sont gérées en segments atomiques appelés Tour d'Action, chacun durant approximativement 3 secondes. Nou parlons "d'aproximativement" 3 secondes, car le système méthodoligue et par étape utilisé pour résoudere les actions ne se treanspose pas nécessairement de manière réaliste dans la vraie vie, où les gens prennent souvent des pauses, s'interrompent pour analyser la situation, reprennent leur souffle et ainsi de suite. Un combat qui commence et se termine en 5 Tour d'Action (15 secondes) dans Eclipse Phase pourrait de 30 seconde à plusierus minutes dans la vraie vie. D'un autre cœté, les personnages peuvent être dans une situation où leur environnement est en train de décompresser et sera vide d'air en 15 secondes, chaque seconde peut donc compter. Le maître de jeu devrait se tenir à la règle de 3 secondes par tour, mais ne devraient aps être effrayés de modifier le temps à chaque fois que la situation le demande. 

Les Tour d'Actions devraient être utilisé pour le combat et d'autres situations où le timing et l'ordre d'action est important. Si il n'est pas nécessaire de garder une trace de qui fait quoi de manière si précise, vous pouvez abandonner la structure en Tour d'Action et revenir à la forme de jeu libre "normale". 

Chaque Tour d'Action est divisé en plusieurs étapes distinctes, décrites ci-dessous. 

\subsection{Étape 1 : Déterminer l'Initiative} \label{sec:roll-initiative} 

AU débit de chaque Tour d'Action, chaque JOUEUR impliqué dans la scène lance son Initiative pour déterminer l'ordre dans lequel chaque personnage agît. Pour plus de détails, voir \emph{ Initiative }. 

\subsection{Étape 2 : Commencer la première phase d'action} \label{sec:begin-first-phase} 

Une fois que l'Initiative a été déterminée, la première Phase d'Action commence. Tout le monde peu agir dans la Première Phase d'Action (puisque tout le monde à une Vitesse minimale de 1), sauf si ils sont inconscients/mort/mis hors-jeu, en commençant par le personnage ayant le plus haut résultat d'Initiative réussi. 

\subsection{Étape 3 : Déclarer et résoudre les actions} \label{sec:declare-resolve} 

le personnage qui agît en premier déclare puis résout maintenant les actions qu'il accomplira pendant sa première Phase d'Action. Puisque certaines actions que le personnage peut faire peuvent dépendre de l'issue d'autres actions, il n'est pas nécessaire de toutes les déclarer d'abord - elles doivent être annoncé et réglée les unes après les autres. 

Tel que décrit au chapitre Actions (p. 189), chaque personnage peut effectuer un nombre variable d'Action Rapide et/ou une seule Action Complexe pendant leur tour. Un personnage peut également commencer ou continuer une Actin de Tâche, ou retarder son action en attendant que les choses évoluent (voir Actions Retardée, p. 189). 

Un personnage qui a retardé son action peut interrompre un autre personnage à n'importe quel moment de cette étape. Ce personnage interrupteru doit accomplir entièrement son action, puis l'action revient au personnage interrompu qui finti alors le reste de son étape. 

\subsection{Étape 4 : Continuer et recommencer} \label{sec:rotate-repeat} 

Une fois que le personnage a terminé ses actions pour cette phase, le personnage suivant dans l'ordre d'Initiative prend le relai, exécutant l'Étape 3 pour son compte. 

Si tous els personnages ont terminés leurs actions de cette phase, retournez à l'Étape 2 et démarrez la deuxième Phase d'Action. Chaque personnage avec une Vitesse de 2 ou plus peut repasser par l'Étape 3, avec el même ordre d'Initiative (modifié par les modificateurs de blessures). Une fois que la deuxième Phase d'Action est terminée, retournez à l'Étape 2 pour la troisième Phase d'Action, où tous les personnages qui ont une Vitesse de 3 ou plus peuvent agir. Finallement, une fois que toutes les personnegs pouvant agir pendant la troisième Phase d'Action ont agit, retournez à l"Étape 2 pour la quatrième et dernière Phase d'Action, où tout les personnages ayant une Vitesse de 4 peuvent agir une dernière fois. 

Arrivé à la fin de la quatrième Phase d'Action, retournez à l'Étape 1 et relancez l'Initiative pour le prochain Tour d'Action. 

\section{Initiative} \label{sec:initiative} 

Le minutage lors des Tours d'Action peut être critique - cela peut être la différence entre la vie et la mort pour un personnage qui a besoin de se mettre à couvert avant qu'un opposant ne dégaine et tire. La détermination de l'Inititative permet de savoir dans quel ordre agissent les protagonistes. 

\subsection{Ordre d'Initiative} \label{sec:initiative-order} 

La stat d'Initiative d'un personnage est égale à la somme de Intuition et Réflexes divisé par 5. Ce score peut être ensuite modifié par le type de morph, les implants, la drogue, le psi ou les blessures. 

Lors de la première étape de chaque Tour d'Action, chaque personnage fait un Test d'Initiative, lançant 1d10 et y ajoutant sa stat d'Initiative. Celui qui obtiens le meilleru score agît en premier, suivi par les autres personnages en ordre décraoissanr, du plus élevé au plus bas. En cas dégalité, les personnages agissent en simultané. 

\begin{quotation} Adam, Bob, et Cami lancent leur Initiative. La stat d'Initiative d'Adam est de 8, celle de Bob est de 11 et celle de Cami est de 6. Adam obtient un 3, Bob un 2 et Cami un 8 Le score d'Initiative total d'Adam est de 11 (8 + 3), celui de Bob est de 13 (11 + 2) et celui de Cami est de 14 (6 + 8). Cami a le meilleur score, elle agît donc en premier, suivit de Bob et finalement d'Adam. Si Cami \& Bob avaient étés à égalité, ils auraient tous les deux agît en même temps. \end{quotation} 

\subsubsection{Initiative et dommages} 

Les personnages qui ont des blessures voient leur score d'Initiative temporairement réduit (voir Blessures, p. 207). Ce modificateur ets appliqué immédiatement lorsque la blessure est reçue, ce qui signifie qu'il peut modifier un score d'Initiative au milieu d'un tour d'Action. Si un personnage est blessé avant de pouvoir démarrer une Phase d'Action, son Initiative est réduite en fonction, ce qui peut signifier qu'il va maintenant agir après quelqu'un avant qui il aurait agit dans l'ordre d'Initiative. 

\begin{quotation} Avat la Phase d'Action de Bob, il prend deux blessures, réduisant son Initiative de 134 à 114. Cela signifie qu'Adam, avec une Initiative de 114, peut maintenant agir avant lui. \end{quotation} 

\subsubsection{Initiative, Moxie et Critiques} 

Un personnage peut dépenser un point de Moxie pour agir en premier lors d'une Phase d'Action, peu importe son jet d'Initiative (voir Moxie, p. 122). Si plsu d'un personnage choisit cette option, l'ordre est déterminé normalement parmi ceux qui ont dépensés du Moxie, suivit par ceux qui ne l'ont pas cfait. 

De manière similaire, tout personnage qui obtient un critique sur un jet d'Initiative agit automatiquement en premier, même avant ceux qui ont dépensés du Moxie. Si deux personnage ont obtenus des critiques, déterminez l'ordre entre eux de manière normale. 

\subsection{Simplifier l'initiative} \label{sec:simplifying-init} 

Pour des résolutions plsu rapide, demandez au personnage de ne faire un jet d'Initiative qu'une fois pour la scène entière. Ce résultat d'Initiative reste le leur pour tous les Tour d'Actions jusqu'à la fin du combat ou que le scénario soit terminé. De manière similaire, ignorez les modificateurs d'Initiative liés aux blessures. 

\subsection{Vitesse} \label{sec:speed} 

La Vitesse détermine le nombre de fois qu'un personnage peut agir pendant un Tour d'Action. Chaque personnage démarre avec une VItesse apr défaut de 1, ce qui signifie qu'ils peuvent agir que lors de la première Phase d'Action de chaque tout. Certaines morphs, implants, drogues, psi ou d'autres facteurs peuvent augmenter cumulativement la VItesse à 2, 3 ou même 4 (le maximum), leur permettant d'agir lors d'autres Phases d'Actiosn également. Par exemple, un personnage avec une Vitesse de 2 peut agi lors de la première et de la deuxième Phase d'Action, et un personnage avec une Vitesse de 3 peut agir de la première à la troisième Phase d'Action. Un peronnage avec une Vitesse de 4 peut agir dans toutes les Phases d'Actions. Cela représente les réflexes et la neurologie améliorés du personnage, lui permettant de penser et d'agir bien plsu vite que les personnages non améliorés. 

Si la Vitesse d'un personnage ne lui permet pas d'agir pendant une Phase d'Action, ile ne peuvent faire aucune action pendant la phase - ils doivent simplement attendre leur tour. le persqonnage peut cependant toujours se défendre et toute acton automatique reste "active." Notez que tout mouvement démarré par le personnage est toujours considéré comme en cours même durant les Phases d'Actions auquelles les personnages ne participents pas (coir \emph{Mouvement }, p. 190). 

\subsection{Actions retardées} \label{sec:delayed-actions} 

Lorsque vent votre tour d'agir pendant une Phase d'Action, vous pouvez décider que vous n'êtes pas encore prêt à agir. Vous pouvez être en train d'attendre le résultat d'une autre actiond 'un personnage, espérer interrompre l'action de quelqu'un d'autre ou simplement ne pas savoir quoi faire. Dans ces cas là, vous pouvez décider de retarder votre action. 

Quand vous retardez votre action, vous vous mettez en attente. À un moment ultérieur de la Phase d'Action, vous pouvez annoncer que vous effectuez votre action maintenant - même si vous interrompez l'actiond 'un autre personnage. Dans ce cas, toutes les autres activités sont mises en attentes jusqu'à ce que votre action soit résolue. Une fois que votre action s'est déroulée, l'ordre d'Initiative reprend là où vous l'avez interrompu. 

Vous pouvez retarder votre action jusque dans la prochaien phase d'Action, ou même jusqu'au prochain Tour d'Action, mais si vous ne l'avez prise au moment où arrive votre prochaine action dans l'ordre d'Initiative, alors vous la perdez. De plus, si vous délayez votre action dans une autre phase ou tour, vous perdez toute action que vous auriez eu dans cette Phase d'Action au moment où vous la prenez. 

\section{Actions} \label{sec:actions} 

Lorsque c'est à votre tour d'agir lors d'une Phase d'Action, vous avez de nombreuses options pour faire ce que vous voulez - bien trop pour toutes les lister ici. Il y a cependant une limite à ce que vous pouvez accomplir en 3 secondes et certaines limitations doivent être respéctées. La première chose est de détermienr quel type d'action vous voulez faire. À Eclipse Phase, les actiosn sont catégorisées comme AUtomatique, Rapide, Complexe ou de Tâche, en fonction de la quantité de temps et d'effort ils entraînent. 

\subsection{Actions automatiques} \label{sec:combat-automatic-actions} 

Les Atcions Automatiques ne nécessitent aucun effort. Ce sont les capacités ou les activités qui sont "toujours active" (du moment que vous êtes conscients) ou qui sont d'un du niveau des réflexes (elles arrivent automatiquement en réponse à certaines conditions, sans effort de votre part). Respirez, apr exemple, est une action automatique - votre corpos le fait sans effort conscient ou réflexion de votre part. 

Dans la plupart des cas, les Actions Automatiques ne sont pas quelque chsoe que vous initiez - elles sont toujours active, ou au moins en attente. Certaine circonstance, cependant, peuvent demander de s'occuper des Actiosn Automatiques. De telles Actions Automatiques sont gérées immédiatement quand elles entrent en action, sans nécessiter d'effort de votre personnage. 

\subsubsection{Résistance} 

Résister aux dégats - qu'ils viennent du combat, d'un poison ou d'une attaque psi - est l'un des exemples d'Actiosn Automatiques qui se décelncehnt en répponse à quelque chose d'autre. 

\subsubsection{Perception basique} 

Vos sens sont continuellement actifs, accmuluant des données sur le modne autour de vous. La perception basique est considérée comme une Action Automatique, et le maître de jeu peut vous demander de faire un Test de Perception lorsque vous recevez une entrée sensorielle que votre cerveau peut vouloir prendre en compte (voir perception, p. 182). De amnière similaire, vous pouvez demander au maître de jeu à n'importe quel moment - même pendant les actiosn d'autres personnages - de faire un Test de Perception basique juste pouv savoir ce que votre personnage remarque dans son environnement immédiat. 

En raison de la nature autoimatique, subconsciente, de la perception basique, vous subirez un modificateur de -20 à cause de la distraction - votre attention est concentrée ailleurs. Afin d'éviter le modificateur de distraction vous devez vous concentrer pleinement dans une perception détaillée ou utilisez un oracle implanté (p. 308). 

\subsection{Actions rapide} \label{sec:combat-quick-actions} 

Les actions rapide sont simples, elles peuvent donc être exécutées rapidement et simultanément à d'autres. Elles nécessitent un effort et une concentration minimales. Vous pouvez effectuer plusieurs Actions Rapide lors de chacun de vos Phases d'Actions, limitées uniquement par le jugement du maître de jeu. Si vous ne faites que des Actiosn Rapides durant une Phase d'Actions, vous devriez avoir le doit de faire au moins 3 Actiosn Rapides distinctes. Si vous effectuez également une Action Complexe ou une Action de Tâche pendant la même Phase d'Action, vous devriez avoir droit à un minimum d'une Action Rapide. C'est le maîþre de jeu qui a le dernier mot par rapport à ce que vous pouvez ou ne pouvez pas faire tenir dans une seule Phase d'Action. 

Quelques exemples d'Actions Rapides incluent: parler, basculer un cran de sureté, activer un implant, se relever, se jetter au sol, communiquer par geste, dégainer/préparer uen arme, manipuler un objet ou utiliser un appareil simple. 

\subsubsection{Ajuster} 

Ajuster est un cas particulier car c'est une Action Rapide mais qu'elle nécessite un certain niveau de concentration qui fait exceptions aux autres actions mineures. Si vous souhaiter ajuster avant d'effectuer une attaque dans la même Phase d'Action, ajuster est la seule ActioN Rapide que vous pouvez prendre pendant cette Phase d'Action (voir Tirs Ajustés, p. 193). 

\subsubsection{Perception détaillée} 

La perception détaillée implique de prendre un moment pour utiliser activement vos sens à la recherche d'information et analyser ce que vous percevez (voir Perception, p. 182). Cela recquiert un peu plus d'effort et d'effort cérébral (ou de puissance de calcul) que la perception basique, qui est une automatique. En tant qu'Action Rapide, vous ne pouvez vous livrer à une perception détaillée que lors d'une de vos Phases d'Actions, mais vous ne souffrez d'aucun modificateur pour la distraction (à moins que vous ne soyez dans un environnement extrêmement distractif, tels qu'une fusillade ou une foule agitée). 

\subsection{Actions complexe} \label{sec:combat-complex-actions} 

Les Actions Complexe nécessitent plus de concentration et d'effort que les Actions Rapides - elles monopolisent effectivement votre attention. Vous ne pouvez prendre qu'une Action Complexe à chacune de vos Phases d'Actions. Vous ne pouvez pas non plus effectuer une Action Complexe et une Action de Tâche pendant la même Phase d'Action. 

Des exemples d'Actions Complexes incluent: attaquer, tirer, faire des acrobaties, défense totale, désamorcer une bombe, utilsier un appareil complexe ou recharger une arme. 

\subsection{Actions de Tâche} \label{sec:combat-task-actions} 

Les actions de Tâches sont des actions qui nécessitent plus de temps qu'un Tour d'Action pour se réaliser. Chaque action de Tâche possède un intervalle pour déterminer le temps que mets l'action à s'accoplir. Cet intervalle peut varier de 2 Tours d'Actions à 2 ans. Lorsque vous effectuez une Action de Tâche, vous ne pouvez pas effectuer d'Action Complexe, même si dans certains cas vous pouvez mettre la tâche en pause et y revenir plus tard. Pour plus d'information, voir Action de Tâche, p. 120. 

Des exemples d'Actions de Tâche incluent: réparer un appareil, programmer, mener une analyse scientifique, fouiller une pièce, esacalader un mur ou cuisiner un repas. 

\section{Mouvement} \label{sec:combat-movement} 

Le mouvement dans \emph{Eclipse Phase} est géré comme n'importe quelel autre action, et peut changer d'une Pahse d'Action à une autre. Marcher et courrir comptent tout les deux comme des Actiosn Rapide, car elle ne nécessitent pas tout votre concentration. La même règle s'applique pour le déplacement en ondulant, en rampant, en flottant, ou en glisant. Courrir peut cependant infliger un modificateur de -10 aux autres actions qui peuvent être affecté par votre mouvement de course. Le sprint est, plus que tout, une course tout aziumts et recquière donc uen Action Complexe (voir Sprinter, p. 191). 

À la discrétion du maître de jeu, d'autres mouvements peuvent peuvent nécessiter une Action Complexe. Franchir une clotûre, sauter à la perche, sauter d'un aplomb, nager ou traverser un habitat en zéro-g avec le parkour nécessitent un peu de finesse et d'attention, et devraient compter comme une Action Complese et devraient recevoir les mêmes modificateur que la course simple. Voler compte générallement comme une Action Rapide, les manœuvres plus délicates nécessitant une Action Complexe. 

\subsection{Allure} \label{sec:movement-rates} 

Parfois il est important de savoir non seulement comment un personnage se déplace, mais à quelle vitesse. Pour la plupart de la transhumanité, cette allure est la mrme: 4 mètres par Tour d'Action en marchant, 20 en courant. Pour déterminer quelle distance un personnage peut franchir en une Phase d'Action, divisés cette allure par le nombre total de Phase d'Actions dans ce tour. Dans un tour avec 4 Phases d'Actions, cela se divise en 1 mètre en marchant par Phase d'Action, 5 mètres en courant. 

Les déplacement en nageant ou en rampant partnet sur une base d'1 mètre par Tour d'Action, soit 0,25 mètres par Phase d'Action. Vous pouvez aussi Sprinter pour augmenter votre allure (voir Sprinter). Les véhicules, robots, créatures et morphs inhabituelles peuvent avoir des allures particulières, listées sous la forme vitesse de marche/de course en mètres par tour. 

Ces allures sont valables sur la gravité standard Terrestre. Si vous vous déplacez en environnement à faible gravité, en microgravité ou en haut-egravité, les choses changent. Voisr \emph{ Gravité}, p. 198. 

\subsubsection{Sauter} 

Les personnages effectuant un saut avec élan peuvent franchir SOM $\div$ 5 (arrondissez au suéprieur); mètres utilisez SOM $\div$ 20 (arronissez au supérieur) mètre pour les sauts sans élan. La hauteur de saut verticales est de 1 mètre. Les personnages faisant un test de Parkour peuvent augmenter la distance de saut de 1 mètre (saut avec élan) ou 0,25 mètres (saust sans élan/saut verticaux) par tranche de 10 points de MdR. 

\subsubsection{Sprinter} 

Vous pouvez utiliser Parkour pour augmenter votre distance de déplacement lors d'une Phase d'Action. Vous devez dépenser une Action Complexe pour sprinter et effectuer un test de Parkour. Chaque tranceh de 10 points de MdR augmente votre course de distance de cette Phase d'Action d'1 mètre, jusqu'à un maximum de +5 mètres. 

\section{Combat} \label{sec:combat} 

Parfois, la discussion échoue, et c'est le moment où les lames et les déchiqueteuse entrent en jeu. Tous les combats dans Eclipse Phase respectent la même mécanique de base, qu'ils soient gérés avec des griffes, des poings, des armes, des armes à feu ou du psi: c'estun Test en Oppoistion entre l'attaquant et le(s) défenseur(s). 

\subsection{Résoudre le combat} 

Utilisez la séquence d'étape suivante pour déterminer l'issue d'une attaque. 

\subsubsection{Étape 1: Déclarer une attaque} 

l'attaquant initie son action en prenant une Action Complexe pour attaquer lors de sa Phase d'Action. La compéternce utilisée dépend de la méthode utilisée pour attaquer. Si le personnage ne possède pas la compétence de Combat appropriée, il peut défausser sur l'aptitude liée. 

\subsubsection{Étape 2: Déclarer la défense} 

Une fois l'attaque déclarée, le défenseur choisit comment répondre. Se défendre est toujours considéré comme une Action AUtomatique sauf si le défenseur est surpis (voir Surprise, p. 204) ou rendu incapable de se défendre d'une manière ou d'une autre. 

\textbf{Contact:} Un personnage se défendant contre une attaque au contact utilsie sa compétence Esquive, représentant le fait d'esquiver les coups (si le personnage n'as pas cette compétence, il peut se défausser sur Réflexes). Le personnage peut aussi utiliser une compétence de combat au contact pour se défendre, représentant les blocages et les parades au lieu de l'esquive. 

\textbf{À distance:} Contre les attaques à distance, le personnage se défendant ne peut utiliser que la moitié de sa compétence Esquive (arrondie à l'inférieur). 

\textbf{Défense Totale:} Les personnages ayant utilisé une Action Complexe pour se mettre en défense totale (p. 198) reçoivent un modificateur de +30 à leur jet de défense. 

\textbf{Psi:} Un personange se défendant contre une attaque psi lance VOL $\times$ 2 (p. 222). UNe sorte de défense mentale totale peut également être utilisée contre les attaques psi. 

\subsubsection{Étape 3: Appliquer les modificateurs} 

Tout modificateur approprié est maintenant appliqué à la compétence de l'attaquant et du défenseur. Voir la table des Modificateurs de Combat (p. 193) pour des modificateurs situationnels communs. 

\subsubsection{Étape 4: Faire le test en opposition} 

L'attaquant et le défenseur lancent tous les deux 1d100 et comparent le résultat à leur seuil modifié. 

\subsubsection{Étape 5: Déterminez l'issue} 

Si l'attaquant réussit et que le défenseur échoue, l'attaque touche. Si lattaquant échoue, l'attaque rate complètement sa cible. 

Si l'attaquant et le défenseur réussissent tous les deux, comaprez leurs résultats. Si l'attaquant obtiens un résultat plsu élevé, l'attaque touche en dépit d'une défense inspirée; sinon l'attaque rate. 

\textbf{Réussite Execeptionnelle:} Si l'attaquant obtient un Succès Exceptionnel (MdR de 30+), une frappe solide est infligée. Augmentez la Valeur de Dommage (VD) infligée de +5. Si la MdR est de 6°+, augmentez la VD de +10. 

\textbf{Critique:} Si l'attaquant obtient un succès critique, l'attaque surclasse l'armure, signifiant que l'armure du défensuer est complètement ignorée - un défaut ou une faille a été exploitée, permettant à l'attaque de la traverser complètement. 

Si le défenseur obtient un succès critique, il esquive avec brio, atteint un couvert qui le protège des attaques à venir, manœuvre vers une position supérieure ou obtient un bénéfice quelconque. 

\subsubsection{Étape 6: Modifiez l'armure} 

Si la cible est touchée, son armure va l'aider à la protéger contre l'attaque (sauf si l'attaquant a obtenu un critique, voir au-dessus). Déterminez quelle type d'armure est approprié pour se défendre contre cette attaque (voir Armure, p. 194). La valeur de pénétration d'Armure (PA) de l'attaque réduit la valeur de l'armure représentant la capacité de l'arme à transpercer les mesures de protection. 

\subsubsection{Étape 7: Déterminez les dégâts} 

Chaque arme et tout type d'attaque possède une Valeur de Dégats (VD, voir p. 207). Ce total est réduit par la valeur de l'armure réduite par la PA de l'attaque. Si les dommages sont réduits à 0 ou moins, l'armure est efficace et l'attaque ne aprvient pas à blesser la cible. Sinon, les dommages supplémentaires sont appliquées au défenseur. Si les dégâts accumulés dépasse la Solidité du défenseur, ils sont rendus incapables de bouger et peuvent mourir (voir Solidité et Santé, p. 207). 

Notez que certaines attaques psi infligent du stress mentak au lieu de dommage physique (voir Santé Mentale, p. 209). Dans ce cas, la Valeur de Stress (VS) et gérée de la même manière que la VD. 

\subsubsection{Étape 8: Déterminez les blessures} 

Les dommages infligés par une seule attaque sont ensuite compérés au Seuild e Blessure de la victime. Si la VD modifée par l'armure égale ou dépasse le Seuil de Blessure, le personnage reçoit une blessure. Des blessurs multiples peuvent être appliquée avec uen seule attaque si la VD modifiée est un multiple du Seuil de Blessure. Les blessures représentent des blessures plus sérieuses et infligent des modificateurs et d'autres effets au personnage (voir Blessures, p. 207). 

\begin{quotation} Stoya essayes de se sortir rapidement de la station, mais l'assassin du  Night Cartel l'a repérée et l'a surprise dans la zone en microgravité de l'habitat. L'INIT de l'assassin est de 63, plus un jet de 23, pour une Initiative de 86. L'INIT de Stoya est de 55, plus un jet de 27, pour une Initiative de 82. 

L'assassin commence, dépensant une Action Rapide pour dégainer un déchiquetteur. Cette arme à fléchette est une arme à tir en rafale, avec une Action Complexe l'assassin peut donc tirer deux fois. Sa compétence d'Armes à Spray est de 65, il a une interface d'arme (+10), et ils sont à portée courte (+0), il a donc besoin de 75 ou moins. Stoya se défend avec sa compétence Esquive (60) divisée par 2, soit 30. 

L'assassin obtien un 08 avec son premier tir. Étonnament, Stoya obtient un 28. Ils réussissent tous les deux, mais Stoya a obtenu le meilleur jet, elle esquive donc le premier tir. 

l'assassin obtient un 20 pour son deuxième tir, un autre succès, et cette fois Stoya obtient un 83, un échec. L'assassin a également obtenu une Réussite Exceptionnelle avec une MdR de 55, augmentant sa VD de +5. 

Les dégats de base de l'assassin sont de 2d10 + 5, mais il tir en rafale contre uen seule cible pour +1d10 et c'est également une arme à zone d'effet conique à courte distance pour un +d10 additionel, pour une VD totale de 4d10 + 5. L'assassin lance 4d10 et obtient 16, il y ajoute ensuite les +5 pour une VD totale de 21. 

Stoya porte une armure corporelle légère (VA 10/10) mais la pénétration d'Armure du déchiqueteur est de -10, son armure est donc entièrement annulée. Elle encaisse donc un dévastateur score de 21 de VD, dépassant son Seuil de Blessure de 10. Deux fois. Cela signifie que Stoya souffre de 2 blessures suite au tir, subissant un -20 à toutes ses actions. De plus, elle doit fair deux Tests de SOM $\times$ 3, l'un pour éviter d'être jettée au sol et l'autre pour éviter de sombrer dans l'incosncience. Sa SOM est de 30, signifiant qu'elle a besoin d'un 70 (30 $\times$ 3 = 90, 90 $-$ 20 de modificateurs de blessure = 70) sur ses deux jets. Elle obtient 40 et 27 les réussissant tous les deux. 

C'est maintenant à Stoya de joueur. Elle prend un Action Rapide pour attraper son arme: un assomeur. Sa compétence d'Armes à Rayon est de 47, modifiée par ses vlessure s(-20) et une interface d'arme (+10) pour 37. La compétence d'Esquive de l'assassin est de 48, divisée par 2 pour un 24 contre les attaques à distances. Stoya obtient un 22 - un succès critique - et l'assassin obtiens un 68. L'assomeur n'inflige qu'1d10 $\div$ 2 de VD, mais puisque l'attaque est un succès critique, elle surclasse l'armure. Stoya obtient un 8, pour 4 point de VD, en-dessous du Seuil de Blessure de l'assasin de 7. 

Les assomeurs sont cependat des armes à choc, l'assassin doit donc fair un Test de SOL + Armure Énergétique. Sa SOL est de 35 et il porte une veste blindée (VA 6/6), son seuil est donc de 41. Il obtient un 71 - Une MdE de 30, signifiant qu'il est immédiatement incapable d'agir pour 3 Tours d'Action. 

Ayant vaincu son opposant, Stoya en profite pour s'échapper rapidement. \end{quotation} 

\subsection{Résumé de combat} 

\begin{quotation} \textbf{Stats d'ego} \begin{itemize} \item L'attaquant lance sa compétence d'attaque +/- les modificateurs. \item Contact: Le défenseur lance Esquive ou sa compétence de combat au contact +/- les modificateurs. \item A distance: Le défenseur lance (Esquive $\div$ 2, arrondi à l'inférieur) +/- les modificateurs. \item Si l'attaquant réussit et obtient un meilleur score que le défenseur, l'atatque touche. \item Les résultats critique surclassent l'armure (l'armure n'est pas prise en compte). \item L'armure est réduite par la valeur de Pénétration d'Armure (PA) de l'attaque. \item Les dégats de l'arme sont réduit par la valeur d'Armure modifiée de la cible (sauf si l'attaque surclasse l'armure). \item Si les dégats dépassent le Seuil de Blessure de la cible, une blessure est également notée. (Si les dégats dépassent le Seuil de Blessure de plusieurs facteurs, de multipls blessures sont infligées.) \end{itemize} \end{quotation} 

\begin{table} \begin{tabularx}{\textwidth}{|X|l|} \hline

\multicolumn{2}{|c|}{\textbf{Modificateurs de Combat}} \\ \hline

\textbf{Situation} &\textbf{Modificateur}	\\ \hline

Le personnage utilsie sa mauvaise main	&-20	\\ \hline

Le personnage ets blessé/traumatisé	&-10 par blessure/trauma	\\ \hline

Le personnage a une position supérieure	&+20	\\ \hline

Attaque de toucher	&+20	\\ \hline

Tir visé	&-10	\\ \hline

Le personnage manie une armes à deux mains d'une seule main &-20	\\ \hline

Petite cible (taille d'un enfant)	&-10	\\ \hline

Trés petite cible 'souris ou insecte)	&-30	\\ \hline

Grosse cible (taille d'une voiture) &+10	\\ \hline

Trés grosse cible (taille d'une grange) &+30	\\ \hline

Visibilité génée (mineur: reflet, fumée légère, faible lumière) &-10	\\ \hline

Visibilité génée (majeur: fumée épaisse, obscurité) &-20	\\ \hline

Attaque aveugle &-30	\\ \hline

\textbf{Combat au contact} &\textbf{Modificateur}	\\ \hline

Le personnage a une meilleure allonge &+10	\\ \hline

Le personnage charge &-10	\\ \hline

Le personnage réceptionne une charge &+20	\\ \hline

\textbf{Combat à distance (attaquant)} &\textbf{Modificateur}	\\ \hline

L'attaquant utilise une interface d'arme ou un viseur laser	&+10	\\ \hline

L'attaquant est derrière un couvert &-10	\\ \hline

L'attaquant court &-20	\\ \hline

L'attaquant est en combat au contact &-30	\\ \hline

Le défenseur a un couvert mineur &-10	\\ \hline

Le défenseur a un couvert modéré &-20	\\ \hline

Le défenseur a un couvert majeur &-30	\\ \hline

Le défenseur est à plat ventre et loin (10+ mètres) &-10	\\ \hline

Le défenseur est caché &-60	\\ \hline

Tir ajusté (action rapide) &+10	\\ \hline

Tir ajusté (actoin complexe) &+30	\\ \hline

Tir de balayage avec une arme à rayon &+10 sur le deuxième tir	\\ \hline

Cibles multiples en une seule Phase d'Action &-20 par cible additionnelle \\ \hline

Tir indirect &-30	\\ \hline

Tire à bout portant (2 mètres ou moins) &+10	\\ \hline

Portée courte &--	\\ \hline

Portée moyenne &-10	\\ \hline

Portée longue &-20	\\ \hline

Portée extrême &-30	\\ \hline

\end{tabularx} \label{tab:combat-modifiers} \end{table} 

\section{Complication d'action et de combat} \label{sec:action-combat-comp} 

Le combat n'est pas aussi simple que de décider qi vous touchez ou ratez votre cible. Les armes, armures, munitions et de nombreux autres facteurs peuvent influer l'issue d'une attaque. De même, de nombreux facteurs peuvent avoir un impact sur une scène d'action, telles que le feu ou l'effet de la microgravité. 

\subsection{Tirs ajustés} \label{sec:aimed-shots} 

Comme noté au paragraphe AJuster, p. 190, un personnage peut sacrifier leur Action Rapide pour se concentrer sur la visée d'une attaque à distance et recevoir un modificateur de +10 à l'attaque. Vous pouvez également sacrifier une Action Complexe complète pour verrouiller votre mire sur la cible. Dans ce cas, et tant que la cible reste en vue jusqu'à la prochaine Phase d'Action, vous recevez un modificateur de +30 pour toucher. 

\subsection{Munitions et recharger} \label{sec:ammunition-reloading} 

Chaque armes possède une capacité de munitions qui indique combien de tir l'arme peut effectuer. Lorsque ces munitions sont épuisés, une nouveau stock doit être chargé. Les jouerus devraient garder une trace des tirs qu'ils effectuent. 

Recharger nécessite presque toujours une Action Complexe, que vous eneclenchier un nouveau chargeur de cartouches ou une batterie neuve pour laser. A la discrétion du maître de jeu, une recharge qui est immédiatement accessible (tel qu'un nouveau chargeur scotché à l'envers au chargeur enclenché, afin que la recharge n'écessite juste de retourner le chargeur et d'enclencher le plein) peut ne nécessiter qu'une Action Rapide. Les armes archaîques telles que les fusil à magasins peuvent nécessiter plus de temps pour être complètement rechargés. 

\subsection{Armes à aire d'effet} \label{sec:area-effect-weapons} 

Certaines attaques à distances sont conçue pour affecter plus d'une cible à la fois. Ces armes se répartissent en trois catégories: souffle, souffle réparti et cône 

\subsubsection{Effet de souffle} 

Les armes à effet de souffle telles que les grenades, les mines et les autres explosifs qui s'étendent autur d'un point de déflagration central. La plupart des attaques à souffle se développent en sphère, bien que certaines charge peuvent diriger une explosion dans une seule direction. La force d'explosion ets plus forte près de l'épicentre et s'affaiblit près des bors de la sphère. Poru chaque mètre de distance entre une cible et le centre, réduisez les dommages d'une armes à souffle de -2. 

\subsubsection{Souffle uniforme} 

Les attaques à souffle uniforme distibuent leur puissance de manière équitable sur toute la surface de l'effet. Cela inclut par exemple les bombes à vide et les arems thermobariques qui dispersent une mixture explosive dans un nuage de vapeur puis le fait détonner en une fois. Toutes els cibles dans le rayon d'explosion souffrent des même dommages. Les dommages contre les cibles hors de la sphère principale d'explosion sont réduits de -2 par mètres. 

\subsubsection{Cône} 

Les armes à effet de cône ont une aire d'effet qui commence au bout de l'arme et s'étend vers l'extérieur en un cône. Â courte portée, cette attaque affecte 1 cible; a portée moyenne elle en affecte 2 distante de moins d'un mètre; et à portée longue ou extrême elle affecte 3 cible ayant un écart maximal d'un mètre entre deux d'entre elles. Les attaque à effet de cône font +1d10 de dégat à courte portée et -1d10 à portée longue et extrème. 

\label{sec:combat-armor} 

Les armures ont rpogressé au même rythme technologique que les armes, permettant d'atteindre des niveaux de protectiosn sans précédents. Comme décrit à l'Étape 7: Déterminez les Dégats (voir p. 192), la valeur d'armure réduit les points de dégats d'une attaque. Pour une liste complète des types d'armures et de leur valeur, voir p. 311. 

\subsubsection{Énergétique contre Cinétique} 

Chaque type d'armure à une Valeur d'Armure (VA) avec deux valeurs - Énergétique et Cinétique - représentant la protection qu'elle fournit contre chaque type d'attaque. Elles sont listées sous la forme "Armure Énergétique/Armure Cinétique." Par exemple, un objet ayant une valeur d'armure "5/10" fournit 5 points d'armure contre les attaques éngertique et 10 points contre les attaques cinétiques. 

Les dégats énergétiques incluent ceux qui sont causés par les armes à rayons (laser, micro-onde, faisceau de particule, plasma, etc) ainsi que le feu et les explosifs à haute énergie. Les armures protégeant contre ce type de dégats sont fait de matériaux qui reflètent ou diffusent une telle énergie, dissipent et transfèrent la chaleur et les matériaux ablatifs. 

Les dégats cinétiques sont le transfert d'une énergie dévastatrice lorsqu'un objet en mouvement (tel qu'un point, un couteau, une massue ou une balle par exemple) entre en collision avec un autre objet(la cible). La pluaprt des armes de mélée et des armes à feu infligent des dégats cinétqiues, comme le font les rochers, les pendules ou les fragments propulsés par une explosion. Les armures cinétiques incluent des plaques résistante à l'impact, des fluides réoépaississant et des gels qui se durcissent à l'impact, ainsi que les fibres ballistique et renforcées. 

\subsubsection{Pénétration d'armure} 

Certaines armes disposent d'une valeur de Pénétration d'Armure (PA). Cette valeur représente la capacité d'une attaque à percer les différentes couches protectrices. La valeur d'AP réduit la valeur de l'armure utilisée pour se défendre contre l'attaque (voir Étape 6: Modifier l'Armure, p. 192). 

\subsubsection{Armures superposée} 

Si deux tyeps d'armure ou plus sont portées, les valeurs d'armures sont additionnées les unes aux autres. Cependant, porter de multiples couches d'armure est peu pratique et encombrant. Appliquez un modificateur de -20 aux actions du personnage pour chaque couche d'armure additionnelle portée. La valeur maximal d'Armure ne peut pas dépasser la valeur de SOL du personnage.

Notez que les objest spécifiquement notés comme étant des accessoires d'armures - casques, boucliers, etc - n'infligent pas de la pénalité d'armure superposés, ils ajoutent simplement leur bonus d'armur au total. Notez également que l'armure inhérente à une synthmorph ou à la structure d'un bot ne constituent pas une couche d'armure (càd que vous pouvez porter une armure sur une coque synthétique sans pénalité). 

\subsection{Asphyxie} \label{sec:asphyxiation} 

Le transshumain moyen peut retenir sa repsiration pour deux minutes avant de s'évanouir. Une activité fatiguante réduit dce total de temps. Pour chaque intervalle de 30 seconde après la première minute où une biomorph est empéchée de respirer, elle doit faire un Test de SOL. Appliquez un modificateur cumulatif de -10 à chaque fois que ce test est effectué. Si le personnage échoue au test, il tombe immédiatement inconscient et commence à souffrir des dommages de l'asphycie, au rythme de 10 oints par minutes, jusqu'à ce qu'il décède ou qu'il puisse respirer de nouveau. Ces dégats ne causent pas de blessure. 

L'asphyxie est un processus horrible, menant souvent à la panique. Les personnages qui ont été asphyxié doivent faire un Test de VOL $\times$ 3. Si ils échouent, ils souffrent de 1d10 $\div$ 2 (arrondissez au supérieur) point de stress mental et ne peuvent effectuer aucune action pour s'aider ce Tour d'Action. Un personnage qui réussit peuvent essayer de s'aider, et ils doivent en fait réussir un Test de VOL $\times$ pour effectuer toute action non directement reliée à leur survie immédiate (attaquer un autre personnage, une créature ou un objet maintenant le personnage sous l'eau sont exemptées de cette règle). 

\subsection{Armes à Rayons} \label{sec:combat-beam-weapons} 

Les arems à rayons sont plus facile à "pointer" sur la cible, en raison du rayon d'énergie continu qu'elles émettent en lieu et place des porjectiles. Cela signifie que l'une des deux règles suivantes peut-être utilisée lorsque vous attaquez avec une arme à rayons. Étant donné que la plupart des armes à rayons ne sont pas visible avec une vue standart, l'attaquant doit avoir une augmentation visuelle pour voir le rayon, ou activer un systéme de viser laser visible pour avoir les avantages conféré par ces régles.

\subsubsection{Tir de balayage} 

Un attaquant effectuant deux attaques semi-automatiques (p. 198) avec une arme à rayon dans la même Action Complexe et qui rate sa première attaque peu traiter cette attaque comme une Action Rapide AJuster (p. 190), recevant un modificateur de +10 sur sa deuxième attaque. En d'autres terme, même si la première attaque rate sa cible, le personnage profite de l'opportunité pour balayer le fasiccau pour le rapprocher de la cible lors de la deuxième atatque. Cela ne s'applique que lorsque les deux attaques sont effectéues contre la même cible. 

\subsubsection{Tir concentré} 

Un personnage utilisant une arme à rayon semi-automatqiue et qui touche avec sa première atatque peut choisir de maintenir le rayon et de concentrer le faisceau, cuisant la cible. Dans ce cas, si le personnage abandonne sa deuxième attaque semi-atutomatique de cette Action Complexe, mais augmente automatique la VD de sa première atatque par 2. Cette décision doit être prise avant que les dés de dégâts ne soient lancés. 

\subsection{Attaques aveugle} \label{sec:blind-attacks} 

Attaquer une cible que vous ne pouvez aps voir est, au mieux, difficile et est une questiond e chance dans le pire des cas. Si vous ne pouvez pas voir, vous devez faire un test de Perception en utilisant un autre sens disponible pour détecter votre cible. Si ce test est un succès, vous subissez un modificateur de -30 à votre attaque. Si votre Test de Perception échoue, votre ataque est essentiellement basée sur la chance - le seuil de votre attaque est égal à votre stat Moxie (sans aucun autre modificateur). 

\subsubsection{Tir indirect} 

Avec l'aide d'un désignateur, vous pouvez cibler un ennemi que vous ne pouvez pas voir en utilisant le tir indirect. Dans ce cas vous devez être meshé avec un personnage, un bot ou un système de cpateur qui a la cible en vue et qui vous pousse les donéns de ciblage (le maître de jeu peu demander un Test de Perception pour le désignateur). Les attaques indirectes souffrent d'un modificateur de -30. 

Les missiles chercheurs (p. 340) peuvent atterir sur une cible qui est "éclairée" par l'énergie réfélchie d'un viseur laser (p. 342) ou d'un système de désignation de cible similaire. Une "atatque" doit d'abord être effectuée pour élcairer la cible avec le viseur laser en utilisant une compétence appropriée. Si l'attaque réussit, cela annule le modificateur de -30 du tir indirect pour le test d'attaque du lanceur. La ible doit rester en vue du désignateur (nécessitant une Action Complexe à chaque Phase d'Action) jusqu'à ce que le chercheur touche. 

\subsection{Bots, synthmorphs et véhciules} \label{sec:bots-synthmorphs-vehicles} 

Les robots manœuvrés par des IA et les morphs syntéhtique sont des choses commune à Eclipse Phase. Les robots sont utilisés pour une grande variété de tâches, de la surveillance, la maintenance et les services à la sécurité et à ploce - et peuvent donc souvent jouer un rôle dans les scènes d'actions et de combats. Bien que moins courants (au moins dans certains habitats), les véhicuels pilotés apr des IA sont également fréquemment utilisés et rencontrés. 

Notez que la différence entre un bots, un véhicule et une synthmorph est essentiellement sémantique. Les robots sont simplements des corps synthétique contrôlés par une IA. Les véhicules sont également des robots - contrôlé par IA - mais le terme "vhciule" dénote leur capacité à transporter des passagers. Les bots et les véhciules peuvent être utilisés comme morph synthétique - c'est à dire, habitées par un ego transhumain - du moment qu'ils sont équipés d'un cybercerveau (p. 300). Dans le cadre de ces règle, le mot "coque" est tilisé pour faire référence aux bots, aux véhicule et aux synthmorphs. 

Comme les synthmorphs, le sbot ste les véchiules sont traités comme tous les autres personnages: ils lancent leur Initiative, effectuent des actoins et utilisent des compétences. Quelques aspects de ces coques nécessitent cependant des considérations particulière qui sont couvertes ci-dessous. 

\subsubsection{Stats des coques} 

Tout comme les personnages en synthmorph, certaines stats des bots et de véhicules (Solidité, Seuil de Blessure, etc) et modificateur de stats (Initiative, Vitesse, etc) sont en fait déterminées par la coque physique. D'autre stats sont déterminés par l'IA contrlant le bot/véhciule (en lieu et place d'un ego). Les bots et les véhciules peuvent également avoir des traits qui s'appliquent à leur IA ou à leur coque physique. Pour des exemples de bto sou de véhicules, voir p. 342 du chapitre Équipement. 

\textbf{Maniabilité:} Le sbots et les véhicules ont une stat particulière appellée Maniabilité, qui est un modifictauer appliqué à tous les tests fait pour piloter le bot/véhicule. Ell représente la manœuvrabilité du bot/véhicule. 

\subsubsection{Compétences des coques} 

Les compétences et aptitudes utilisés par un bot ou un véhicule sont ceux possédés par son IA. Voir \emph{IA et Muses }, p. 264. 

\subsubsection{Déplacement des coques} 

Comme les personnages, les bots et les véhicules ont un score de Mouvement pour la marche et pour la course. Ils sont utilisés lorsque le bot/véhicule est impliqué dans une scène d'action ou de combat avec d'autres personnages. 

Les coques qui sont capables d'atteindre des vitesses plus élevées ont aussi une Vitesse Maximale - c'est la vitesse la plus élevé à laquelle la coque peut se déplacer de manière sûre, indiquée en kilomètre par heure. À la discrétion du maître de jeu, certaines coque peuvent dépasser leur limite et accélérer au-delà, masi avec une prise de risqaue significative - le mæître de jeu devrait appliquer des pénalités appropriées aux test de Piloter et aux autres tests. 

\subsubsection{Poursuites} 

Les coques qui se déplacent plus vite que leur score de course de Mouvement (jusqu'à leur Vitesse Max. ) sont gnéralement considérés comem se déplaçant trop vite pour intergair de manière normale avec les autres personnages lors de l'action ou du combat. C'est le moment ou la scène passe dans le mode "scène de poursuite" - un mouvement narratif composé de choix de maœuvre et de test avec différentes issues. Qu'une poursuite soit réellement en cours ou non, le maître de jeu devrait garder en tête que la Vitesse Maximale n'est pas le seul facteur à prendre en compte pour les situatiosn à haute-vitesse. Les facteur senvironneemntaux comme le terrain, les conditions météorologiques, la navigation, les piétons et le traffic peut fournir des obstacles que les coques doivent franchir. Une coque devant taverser un habitat pour atteindre une bombe avant qu'elle n'explose devra prendre plusieurs décisions et faire quelques tests qui détermineront si elle arrive à temps ou non. De manière similaire, une coque cherchant à semer des poursuivants devra effectuer quelques manœuvres imaginative et espérer trouver un raccourci ou deux pour semer ses opposants. 

\subsubsection{S'écraser} 

Les coques qui subissent des blessures en combat ou en poursuite pourraient être forcées de faire un Test de Piloter pour éviter de s'écraser ou s'écraser automatiquement. Les circonstances exacte d'un crash sont laissées à l'appréciation du maître de jeu en fonction de ce qui colle le mieux à l'histoire - la coque pourrait simplement s'arréter en glissant, percuter un arbre, tomber du ciel ou faire un tonneau et atterir sur un groupe de passants. Les coques qui percutent d'autres objets lorsqu'elles s'écrasent prennent générallement des dégats supplémenataire suite à la collision (voir Collisions) 

\subsubsection{Collisions} 

Si une coque s'écrase dans ou percute volontairement une personne ou un objet, quelqu'un sera blessé. Pour déterminer la quantité de VD infligée, lancez 1d10 et ajoutez-y la SOL de la coque divisée par 10 (arrondit au supérieur). Cela représente les dommages infligés à la vitesse de marche. Si la coque se déplaçait à la vitesse de coruse, multipliez la VD par 2. Si la coque se déplaçait à une vitesse de poursuite, multipliez la VD par la vitesse de la coque $\div$ 10 en mètres par tour. La coque et tout ce qu'elle percute subissent ces dégats, du moment que la collision a lieu vaec quelques choses équivalent en densité et en dureté. Les objets mous et déformable comem les biomoprhs feront moins de dégats à la coque (sauf si elles se trouvent dans une tenue rigide ou en armure de bataille), auquel cas la coque ne subira que la moitié des dommages suite à la collision. Les armures Cinétiques sont utilisées pour résister à la VD d'un crash. 

Si deux coques se percutent en face à face, calculez les dégats infligés par chacune d'elle et infligez les au deux. Si deux coques qui se déplacent dans la même sens, se percutent ne prenez en compte que la différence de vitesse. 

Les passagers dans les véhicules peuvent aussi être endommagés par les collisions si ils n'utilisent pas les mesures de sécurité correctes. Ils ne subissent que la moitié de la VD subie par leur véhciule (réduite de leur propre armure Cinétique). 

\begin{table} \begin{tabular}{|l|l|} \hline

\multicolumn{2}{|c|}{\textbf{Dégats de Collision}}	\\ \hline

VD de Collision de Base	&1d10 + (SOL $\div$ 10)	\\ \hline

Course	&VD $\times$ 2	\\ \hline

Vitesses de Poursuites	&VD $\times$ (vitesse $\div$ 10)	\\ \hline

\end{tabular} \label{tab:collision-damage} \end{table} 

\subsubsection{Attaque des passagers des véhicules} 

Pendant le combat, les passagers à l'intérieur d'un véhciule peuvent être ciblés séparément du véhicule. Les attaques effectués contre les passagers de cette manière n'infligent aucun dégât au véhicule (à l'exception des armes à aire d'effet). Les passagers ciblés bénéficient à la fois d'un couvert (généralemennt Majeur, -30) et de la structure du véhciule, ajoutant l'Armure du Véhciule à la leur. 

Les passagers à l'intérieur d'un véhciule ne sont généralement pas blessés par les attaques effectuées contre le véhciule. Les armes à aire d'effet sont une exception à cette règle, mais dans ce cas les apssagers bénéficient aussi de toute l'Armure du Véhicule. 

\subsubsection{Coque contrôllées à distance} 

Toute coque (ou biomorph) avect l'option marionnette (inclut dans tous els cybercerveaux) peut être contrôllée à distance, soit par un personnage soit par une IA distante. Cela nécessite un lien de communication entre le téléopérateur et la coque (le "drône"). Le téélopérateur contrôlle le drône grâce à une interface entoptique, et reçoit les retour sensorielles et les autres données grâce à l'insert de mesh du drône. 

Lorsque soumis à un contrôlle direct, l'IA (ou l'égo résident) de la coque est surchargé et mis en veille. Le drône n'agît que selon les instructiosn reçues. Chauqe instruction coûte une Action Rapide. Dance cette situation, le drône agît avec la même Initiative que le téléopérateur. Les compétences et stats du téléopérateur ont utilisés en lieu et place de ceux de l'IA de la coque. En raison de la nature des opérations distantes tous les tests sont cependant fait avec un modificateur de -10. Plusieurs drônes peuvent être contrôllés en même temps, mais leur donner des ordres nécessite une Action Rapide distincte à moins qu'ils ne reçoivent la même commande. Le contrôle direct par téélopération 'est pas faisable sur des distances extrêmes, en raison de l'impact du décalage temporel sur les communications. 

Le téléopérateur peut également mettre led drœne en mode autonome, permettant à l'IA de le coque de reprendre sona ctivité normale. Le drône continue de suivre les ordres du téléopérateurs dans la mesure de ses capacités. Dans ce mode, le drœne fonctionne normalement, utilisant sa propre initiative et les compétences et stats de l'IA. 

\subsubsection{Interception de  coque} 

L'"interception" est le terme familier pour une forme plus directe de contrôle distant, en utilisant les technologies de RV et d'XP. Pendant l'interception, le port marionnette du drône alimente directement l'insert de mesh du téléopérateur en données sensorielles. Le téléopérateur se surcharge au sensorium du drône, "devenant" le drône. Dans ce cas, le téléopérateur abandonne le contrôle de leur propre morph, qui devient inerte. Pendant qu'il intercepte, il subit un modificateur de -60 sur tous les Test de Perception ou sur toute tentative d'agir avec sa morph. 

Intercepter nécessite une Action Complexe pour se connecter, et une autre pour se déconnecter. Un téléopérateur intercepteur contrôle un drône comme si il était dans sa propre morph. Comme pour la téléopération en contrôle direct, les compétences et Initiative de l'intercepteur sont utilisées en lieu et place de l'IA du drône. L'intercepteur ne subit atcun modificateur de téléopération, mais un seul drône peut être intercepté à la fois. 

Si le drône est tué ou détruit, l'intercepteur est immédiatement éjecté de sa connexion, reprenant le contrôle de sa propre morph comme d'habitude. Être éjecté de cette manière est extrêmement perturbant, au moins parce que l'intercepteur expérimente le fait d'être tué/détruit. L'intercepteur subit donc 1d10 points de stress mental. 

\subsection{Tirs visés} \label{sec:called-shots} 

Parfois il n'est pas suffisant de simplement toucher sa cible - vous devez tirer à travers une fenêtre, désarmer un adversaire ou tirer dans ce trou dans l'armure de votre cible. Vous devez déclarer que vous faites un tir visé avant de commencer votre attaque, en choisissant l'une des possibilités notées ci-dessous. Les tirs visés subissent un modificateur de -10 et nécessitent une réussite Exceptionnelle (MdR 30+). Si vous dépassez cette marge, vous réussissez votre tir visé et les résultats notés ci-sessous s'appliquent. Si vous réussissez mais que vous ne dépassez aps la marge reuise, vous touchez votre cible de manière habituelle. 

\subsubsection{Contourner l'armure} 

Les tirs visés peuvent être utilisés pour cibler un trou ou une faiblesse dans l'armure de votre adversaire. Si vous dépasez la MdR, vous obtenez un tir surclassant l'armure, et son armure ne s'applique pas. Notez que dans certaines circonstances, un maître de jeu peut décider que l'armure d'una dversaire n'a tout silmplement pas de point faible ou de zone non protégées, interdisant de tels tir visés. 

\subsubsection{Désarmer} 

Vous pouvez effectuer un tir visé pour tenter de faire tomber une arme des mains de votre adversaire. Si vous battez la MdR, la victime subit la moitié des dégats de l'attaque (réduit par l'armure comme d'habitude) et doit ensuite réussir un Test de SOM $\times$ 3 avec un modificateur de -30 pour conserver sa prise sur l'arme. 

\subsubsection{Ciblage spécifique} 

Vous pouvezeffectuer un tir visé dans le but de toucher un endroit spécifique ou un composant de votre cible - par exemple pour désactiver les senseurs d'un bot, balayer la jambe de quelqu'un ou taper dans l'œil d'une personne. Si vous battez la MdR, vous touchez le point spécifiquement visé. Le maîþre de jeu détermine le résultat de manière appropriée à l'attaque et à la cible - le composant peut-être déteuit, l'adversaire peut tomber ou temporairement aveuglé, et ainsi de suite. 

\subsection{Charger} \label{sec:charging} 

Un adversaire qui court puis attaque un adversaire au corps à corps dans la même phase d'Action est considéré comme chargeant. Un attaquant charegant souffre toujours du modificateur de -10 pour la course, mais ils reçoivent un bonus de dégats grâce au moment cinétique engendré: augmentez les dommages infligés de +1d10. 

\subsubsection{Réceptionner une charge} 

Vous pouvez retarder votre action (voir p. 189) afin de réceptionner une charge, vous préparant à l'impact, interrompant l'action et frappant juste avant que l'attaquant ne le fasse. Dans cette situation, vous recevez un modificateur de +20 pour frapper l'adevrsaire qui vous charge. 

\subsection{Démolitions} \label{sec:demolitions} 

L'utilisation la plus commune de la compétence Démolitions est le placemen t, le désarmement et la fabricatoin d'appareils explosifs, telels que les charges de superthermites (p. 330) ou les grenades (p. 340). 

\subsubsection{Placer des explosifs} 

Un démolisseur compétent peut placer des charges d'une manière qui va aigmnter leurs effets. Il peut identifier les vulnérabilité structurelle et les poinst faibles et concentrer une explosion sur ces zones. Il peut déterminer comment faire exploser un coffre sans end étruire le contenu. Il peut concentrer la force de l'explosion dans une direction aprticulière, augmentant la force dirigée tout en réduisant les effets de souffle. 

Chacun de ces scénarios demande un Test de Démolitions réussi. Le résultat exact est déterminé par le maître de jeu en fonction des spécificités du scénario. Par exemple, en utiliant les exemples ci-dessus, cibler un point faible peut doubler les dommages infligés sur cette structure. Modeler la charge pour orinter sa force peut tripler les dommages dans une direction, tel que noté dans la description de la superthermite (p. 330). Une Réussite Exceptionnelle augmentera les dommages de l'explosif de +5, alors qu'un succès critique permettra à l'explosion d'ignorer l'armure. 

\subsubsection{Désarmer} 

Désarmer un apapreil explosif est géré par un Test en Opposition entre les compétences Démolitions du démineur et de l'artificier. 

\subsubsection{Farbiquer des explosifs} 

Un personnage entraîné dans la Démolitions peut fabriquer des explosifs à partir de matériau bruts. Ces matériau peuvent être assemblés de manière traditionnelle ou être fabriqués en utilisant un nanofabeur. Même les nanofabeurs ayant des configurations restreintes pour empêcher la création d'explosifs peuvent être utilisés, les explosifs pouvant êtres construit de bien des manières différentes à aprtir de produits chimiques et de matériau inoffensifs. L'intervalle pour fariquer des explosifs est de 1 heure pour 1d10 points de dégat que l'explosif infligera. Si le démolisseur obtiens un échec critique, il peut se faire sauter accidentellement à moins que la charge ne soit extrêmement faible ou beaucoup plus puissante que prévu (quelle que soit al situation la plus désastreuse). 

\subsection{Tomber} \label{sec:falling} 

Si un personnage tombe, utilisez la table des Dégats de Chute pour déterminer quelles blessures il va subir. L'armure cinétqiue absorbera les dégats en utilisant la moitié de sa valeur (arrondissez à l'inférieur). À la discrétion du maître de jeu, les dégats peuvent aussi être réduits si quelquechose ralenti la chute (branches, surface molles). 

\begin{table} \begin{tabular}{|l|l|} \hline

\multicolumn{2}{|c|}{\textbf{Dégâts de chute}}	\\ \hline

\textbf{Distance de chute}	&\textbf{Dégâts}	\\ \hline

1-2 mètres	&1d10	\\ \hline

3-5 mètres	&2d10	\\ \hline

6-8 mètres	&3d10	\\ \hline

Plus de 8 mètres	&+1 par mètre	\\ \hline

\end{tabular} \label{tab:falling-damage} \end{table} 

\subsection{Feu} \label{sec:fire} 

Les objets qui entrent en contact avec une chaleur extême ou avec des flammes peuvent s'enflammer à la discrétion du maître de jeu, gardez à l'esprit à la fois le caractère inflammable de l'objet et la force de la chaleur/des flammes. Les objets (ou les personnages) qui brûlent souffrent de 1d10 $\div$ 2 (arrondissez au supérieur) points de dégats à chaque Tour d'Action à moins que quelquechose d'autre ne soit précisé. Les armures énergétiques vous protègerosn contre ces dommages, même si elles peuvent prendre feu, réduisant les dégats subit de leur valeur. Enf onction des conditions environnementalers, les feux ont tendance à grandir à moins que quelquechose ne cherche à le réduire. Tous les 5 Tours d'Actions, augmentez la VD infligée (d'abord à 1d10, puis à 2d10, ensuite à 3d10 et enfin par incrément de +5). Des oncidtions défavorables (telels que la pluie) ou des efforts pour éteindre le foyer réduiront la VD de manière appropriée. 

Notez que le feu ne brûle pas dans le vide. En microgravité, le feu brûle dans une sphère et se développe plsu lentement, les gazs d'expansions éloignant l'oxygène (augmentez la VD tous les 10 Tours d'Actions). Si il y a un manque de circulation d'air, des feux en microgravité peuvent s'éteindre d'exu-même. 

\subsection{Modes de tir et cadence de tir} \label{sec:firing-modes-rate} 

Toute les armes à distance d'\emph{Eclipse Phase} possèdent un ou plusieurs mode de tir qui détermine leur cadence. Ces modes de tirs sont détaillés ci-dessous. 

\subsubsection{Coup par coup (CC)} 

Les amres en coup par coup ne peuvent faire feu qu'une fois par Action Complexe. Ce sont typiquement les armes els plus grosses ou les plus archaïques. 

\subsubsection{Semi-automatique (SA)} 

Les armes semi-automatiques sont capable de tri rapide et répétés. Elle peuvent effectuer deux tirs avec la même Action Complexe. Chaque tir est considéré comme une attaque séparés. 

\subsubsection{Tir en rafale (TR)} 

Les amres qui peuvent tirer en rafales peuvent libérer plusierus tirs rapide (une "rafale") avec une seule pression de la détente. Deux rafales peuvent être tirées avec la même Action Complexe. Chaque rafale est gérée comem une attaque distincte. Les rafales tirent jusqu'à 3 munitions. 

Une rafale peut être tirée sur une seule cible (tir concentré) ou contre deux cibles écartées de moins d'un mètre l'une de l'autre. En cas de tir concentré contre une seule cible, vous pouvez choisir un modificateur de +10 pour toucher ou augmenter la VD de +1d10. 

\subsubsection{Tir Automatique (TA)} 

Les armes automatiques peuvent libérer une grêle de tirs avec une seule pressions sur la gachette. Une seule attaque en tir automatique ne peut être fait avec chaque Action Complexe.
Cette attaque peut être effectuée sur une à trois cibles différentes, tant qu'elles ne sont pas éloignées de plus d'un mètre l'une de l'autre.
Dans le cas d'un tir concentré sur une seule cible, l'attaquant peu choisir sois un modificateur de +30 pour toucher, sois augmenter la VD de +3d10. Tirer en automatique utilise 10 munitions. 

\subsection{Défense Totale} \label{sec:full-defense} 

Si vous vous attendez à être sous le feu, vous pouvez dépenser une Action Complexe pour apsser en défense totale. Cela représente le fait que vous dépensez toute votre énergie à esquiver, plonger, à parer les attaques et de manière générale à vous sortir de là jusqu'à votre prochaine Phase d'Action. Pendant ce temps, vous recevez un modificateur de +30 pour vous défendre contre les attaques en cours. Les personnages qui sont en défense totale peuvent utilsier leur compétence Parkour au lieu de leur compétence Esquive pour esquiver les attaques, représentant les mouvements acrobatiques qu'ils font pour éviter d'être touchés. 

\subsection{Gravité} \label{sec:gravity} 

La plupaert des personnages dans Eclipse Phase ont une expérience importantes des manœuvres en faible ou micro gravité et peuvent y effectuer des actiosn normales sans pénalités. Même les personanges qui ont grandit sur un corps planétaires ou dans des habitats tournant sont relativement familier aves le gravités alternatives grâce aux entraînements dans les systèmes éducatifs en simulspace. La même chsoe est également vraie dans l'autre sens; les personnages qui ont grandit en chute libre ont vécus des simulations de la vie dans un puit gravitationnel. 

À la discrétion du maître de jeu, les personnages qui ont passé de lingue périodes à s'acclimater à un type de gravité peuvent avoir du mal à s'habituer à un changement de gravité, au moins jusqu'à ce qu'ils se soient habitués à la nouvelle gravité. Dans ce cas, le maître de jeu peut appliquer un modificateur de -10 aux compétences sociales et physiques. La pénalité physique est la résultante des difficultés à manœuvrer. La pénalité sociale s'appliquent parcequ'il est difficile d'avori l'air impressionant, intimidant ou séduisant lorsque vous n'avez pas trouvé comment arranger vos habits pour qu'ils ne flottent pas devant vos yeux. La pénalité physique peut être augmentée à -20 pour les situatiosn impliquant des compétences de combat et des compétences nécessitants la manipulation rpécise, la construction ou la réparation d'objets. Ces pénalités s'aplliquent jusqu'à ce que le personnage finisse par s'adpater, généralement en 3 jours. 

Toute biomorph ayant les biomods de bases (p. 300) est immunisée au mal de l'espace et des effets de l'exposition à long-terme à la micro-gravité. 

\subsubsection{Microgravité} 

La microgravité inclut les environnements zéro-G ainsi que ceux qui ont une gravité légèrement plus élevée mais négligeable. Ces conditions sont trouvés dans l'espace, sur les astéroïdes et les petites lune ainsi que sur (une aprtie) des vaisseaux et habitats qui ne tournent pas pour générer de la gravité. Les objets en microgravité n'ont pas de poids, mais leur taile et leur masse sont toujours des facteurs à considérer. Les choses se comportent différement en microgravité. Par exemple: 

\begin{itemize} \item Les objets qui ne sont pas attachés ont tendance à dériver dans la direction de leur dernier déplacement. Les objets flottants peuvent éventuellement se diriger vers la aprtie la plus dense de l'habitat ou du vaisseau. \item Les objets lancés ou poussés vont voyager en ligne droite jusqu'à ce qu'ils percutent quelque chose. \item La fumée ne s'élève pas en un flot. Elle forme plutôt des halos vaguement sphérique autour de leur source. \item Les liquides n'ont que peu de cohésion, se répandant en nuage de petites gouttes si ils sont libérés dans l'air. Les boissions sont préparés dans des ampoules ou des bouteilles scellés. La nourriture est mangée d'une manière qui empêches les sauces et les parties liquide de s'échapper. Le sang se propage partout. \end{itemize} 

Se déplacer et manœuvrer en microgravité est géré par la compétence Shute Libre (p. 179). La plupart des activités quotidienne en chute libre ne nécessite pas de test. Le maître de jeu peu cependant demander un test de Chute Libre pour toute les manœuvre complexes, voler à travers de grandes distances, changer brutalement de direction ou de vitesse ou pour le combat au contact. Un échec signifie que le personnage a mal caculé sa trajectoire et termine dans une poistion qui n'était pas celle voulue. Un Échec Catastrophique signifie que le personnage s'est complètement planté, se fracassant dans un mur ou s'éjectant dans l'orbite en tourbillonant. 

Pour faciliter les choses, la plupart des habitats en microgravité ont des meubles recouvert de sangles élastiques et sont maillés de poches pour empêcher les objets de flotter dans tous els sens, ainsi que des bandes périphériques mobiles équipées de poignées le long des principales voies de circulations. les chaussures magnétiques ou à velcro sont également utilisée pour se promener plutôt que d'escalader ou de voler. Les environnements en zéro-g sont souvent conçus pour utiliser l'espace au maximum et tirent parti de l'absence de plafonds et de planchers. Comme les objets n'ont pas de poids, les personnages peuevnt même déplacer facilement des objets massifs. 

\textbf{Allure:} Les personnages qui grimpent, tirent ou se propulsent eux-même se déplace à la moitié de leur allure (p. 191) en microgarvité. 

\textbf{Vitesse Terminale:} Il n'est pas difficile d'atteindre la vitesse d'échappement sur les petits astéroïdes et les corps similaires - c'est quelque chose qu'il faut garder en tête avec les objets lancés et les armes à projectile. Dans certains cas, les personnages qui se déplacent suffisament rapidement et qui sautent peuvent atteindre la vitesse d'échappement, bien que ces situations soient laissées à l'appréciation du maître de jeu. 

\subsubsection{Faible gravité} 

La faible gravité inclut tout ce qui va de 0,5g à la microgravité. On trouve ces conditions sur la Lune, sur Mars, sur Titan et sur les parties rotatives de la plupart des vaisseuax et habitats tournant. Le faible gravité n'est pas trés différente de la gravité standard, bien que les personanges peuvent sauter deux fois plus loin et que les projectiles et objets lancés ont une portée plus longue (p. 203). Augmenter l'allure de course des personnages en faible gravité de x1,5. 

\subsubsection{Gravité élevée} 

La gravité élevée est tout ce qui est significativement plus élevée que la gravité Terrestre standard (1,2g et plus). La gravité élevée à Eclipse Phase n'existe typiquement que sur les exoplanètes. La gravité élevée peut être particvulièrement dure pour les personnages, leur corps étant soumis à des contraintes plus élevés en raison du poids plus important, les muscles se fatiguent à cause de la poussée supplémentaire nécessaire pour se déplacer et le cœur doit battre plsu fort pour envoyer le sang aprtout dans l'organisme. Pour tous les 0,2g au delà de 1 pour lequel un personnage 'nest pas acclimaté, considéré que le prsonnage souffre des effets d'1 blessure. A la discrétion du maîþre de jeu, les allures de déplacement peuvent également être modifiées. 

\subsection{Grenades et chercheurs} \label{sec:combat-grenades-seekers} 

Les grenades, chercheurs et explosifs moderne similaires ne détonnent pas forcément au moment où ils sont lancés ou lorsqu'ils touchent leur cible. En fait, plusieurs options de déclenchement sont disponible, chacune configurée par l'utilisateur lorsqu'il déploit l'arme. Les attaques ratées ou celles qui n'explsoent ni pendant le trajet ni à l'impact, sont sujettes à la dispersion (p. 204). 

\textbf{Explosion aérienne:} l'explosion aérienne signifie que l'appareil explose en l'air dés qu'il a parcouru une distance programmée au lancement. Dans ce cas, les effets de l'exposion sont résolu immédiatement, lors de cette Phase d'Action de l'utilisateur. Notez que les munitions à explosion aérienne sont programmées avec une sûreté qui empéchera la détaonation si elles n'arrivent pas à parcourir une distance minimale de précuation depuis le lanceur, bien que ceci puisse être contourné. 

\textbf{Impact:} La grenade ou le missile explose aussitôt qu'il touche quelque chose, que ce soit la cible, le sol ou un objet interrompant la trajectoire. Résolvez les effets imméditament, lors de cette Phase d'Action de l'utilisateur. 

\textbf{Signal:} La munition est préparée pour détonner lorsqu'elle reçoit un signal de commande par un lien sans-fil. L'appareile reste simplement en attente jusqu'à ce qu'il reçoive le signal correct (qui doit inclure la clef de chiffrement qui lui a été assigné lorsque la grenade a été préparée), détonnant immédiatement lorsqu'il le reçoit. 

\textbf{Minuteur:} l'appareil possède un miniteur inerne permettant à l'utilisateur de régler précisément le moment de la détonnation. Cela peut-être n'improte quand entre 1 seconde et plusieurs jours, mois ou années plus tard, transformant dans les fait l'appareil en bombe, mais augmentant également la probabilité qu'il soit découvert et neutralisé. La période dee détoantion minimale - 1 seconde- fait que la munition explosera avec le Score d'Initiative (actuel) de l'utilsiateur lors de la prochaine Phase d'Action. Un délai de 2 seconde durera deux phases d'Actions, un délai de 3 seconde dure trois Phases d'Action et ainsi de suite. 

\subsubsection{RRenvoyer les grendades} 

Il est possible qu'un personnage soit capable d'atteindre une grande avant qu'elle n'explose et qu'il al relance à son expéditeur (ou qu'il l'envoie au lieu, dans une direction sûre). Le personnage doit être à portée de mouvement de la localisation de la grenade, et doit dépenser une Action Complexe pour tenter un test de REF + COO + COO pour attraper la grenade. Si il réussit, il peut relancer al grenade dans la direction de son choix avec la même action (considérez l'action comme une attaque de lancer standard). 

Si le eprsonnage rate le test il peut cependant se trouver à l'épicentre de l'explosion. 

\subsubsection{Sauter sur l'explosion} 

Étant donné les possibilités offertes par la réincarnation, un personnage peut décider de se sacrifier pour l'équipe et de se jetter sur une grenade, se sacrifiant dans le but de protéger les autres. Le personnage doit être à portée de mouvement de la localisation de la grenade, et doit dépenser une Action Complexe pour tenter un test de REF + COO + VOL pour sauter sur la grenade et la recouvrir avec sa morph. Cela implique que le personnage subisse 1d10 de dégats supplémentaire lorsque la grenade explose. Le côté positif est que les dégats de la grenade sont réduits par l'armure du eprsonnage +10 lorsque ses effets sont appliqués aux autres dans le rayon de l'explosion. 

Si le maîþre de jeu pense que c'est approprié, un Test de VOL $\times$ 3 peut être demandé pour qu'un personnage se sacrifient de cette manière. 

\subsection{Environnements hostiles} \label{sec:hostile-environments} Le système solaire a beau être un point de départ idéal pour le développement de la vie, si vous vous retrouvez coincé dans le puits gravitationnel de Jupiter pendant une tempête magnétique, que vous essayer de respirer sans filtre sur Mars ou que vous essayez de nager dans le vide sans combinaison spatiale, il ne semble plus aussi amical. Cette section décrit quelques uns des environnements hostiles auquel les personnages d'\emph{Eclipse Phase} pourraient bien avoir à faire face. 

\subsubsection{Contamination atmosphérique} 

Les habitats tombent parfois malades. les effets d'un habitat souffrant d'un déséquilibre écologique ou de pathogène hors de contrôle peuvent aller des atmosphères légèrement allergène aux infections environnementales dévastatrice. les personnages sans système de respiration ou de filtration dans les environnements contaminés devraient subir des pénalités aux compétences physiques et, probablement, sociale allant de -10 (allergie légère) à -30 (atmosphère sévèrement infecté). En fonction de la contamnination, d'autres effets peuvent également s'appliquer en fonction des souhaits du maître de jeu. 

\subsubsection{Températures extrêmes} 

Les environnements planétaires varient de l'etrêmement chaud (Vénus, face ensolleillée de Mercure) à l'extrêmement froid (Neptune, Titan, Uranus). Les deux sont capables de tuer une biomorph non protégée et non adaptée en quelques minute, voire quelques secondes. Les synthmorpsh et les véhicules s'en sortent mieux, particulièrement dans le froid, mais même eux ont des chances de rapidement succomber aux fournaises flamboyantes des planètes intérieures sans de gros boucliers thermique et des systèmes de refroidissements. 

\subsubsection{Pression extrême} 

De manière similaire, la presison atmosphérique de Jupiter, Saturne et vénus devient rapidement mortellement écrasante après les premiers niveaux. Seuls les synthmorphs et les véchicules adaptés pour ces pressions extrêmes peuvent espérer survivre à de telles profondeurs. 

\subsubsection{Zones de transition gravitationnelle} 

l'usage important de la gravité artificielle dans les habitats spatiaux signifie que les personnages rencontrerons souvent des endroits où la direction du bas change brutalement. Dans la plupart des habitats rotatifs, la conception standard incluent une zone axiale où les vaisseaux spatiaux peuvent s'arrimer en microgravité et une zone de transition balisée et minutieusement concçues (géénralement un ascensseur) dans lqeulle els personnes et les cargaisons allant et venant vers le spatioport axial peuvent s'orienter vers le "bas" local et se tenir au bon endroit lorsque la gravité prend effet. Les transitions gravitationnelle dans les habitats sont rpesque toujours graduelles mais peuvent être extrêmement danegreuses si un personnage les franchit en étant au mauvais endroit au mauvais moment. 

Un personnage qui se retrouve à dériver dans la zone de microgravité axiale d'un habitat spatial rotatif va dériver lentement vers l'extérieur jusqu'à ce qu'il commence à subir la gravité, moment à partir duqeul il tombera. Le temps que prends ce phénomnèe varie en fonction de la taille de l'habitat. Une bonne règle de base est que pour chaque kilomètre de diamètre de l'habitat, le personnage a 30 seconde avent de commencer à tomber. Si le personnage a reçu une bonne poussée hors de l'axe lorsqu'il a commencé à dériver, ce temps devrait être divisé par deux, quatre ou plus, à la discrétion du maîþre de jeu. 

\subsubsection{Champs magnétiques} 

Le magnétismen'est pas un problème direct pour la plupart des personnages; les transhumains n'ont pas besoin de s'inquiéter des radiations générées par une magnétosphère puissante. Pour les appareils électroniques non blindés et les transhumains non protégés arborant du titanium les effets des champs magnétiques puissants peuvent être dévastateurs. Notez que la plupart des problèmes réusltatnts de l'expositions des véhicules, du matériel et des bots à une forte activité magéntique coincide avec une radioactivité élevée. 

Les champs magnétiques affectent les synthmorphs, les robots, les véhicules, les implants cybernétiques et l'électronique après 1 minute d'exposition. Comme l'exposition aux radiations, ces effets peuvent varier de manière radicale. Au minimum, les communications et les capteurs souffrirontd 'interférence et de portée réduite: au maximum, les systèmes électoniques vont simplement subir des dégâts et tomber en panne. 

\subsubsection{Radiation} 

Les radiations ionisantes sont l'un des dangers les plus fréquemments rencontrés dans le système solaire et l'un de sproblèmes les plus difficiles à vaincre pour la transhumanité. Les radiations abîment les matériaux génétiques, rendent malades et tuent en ionisant les composés chimiques impliqués dans la division cellullaire à l'intérieur du corps humain. Les effets s'étendent la nausée et la fatigue à des défectiosn d'organes massives et la mort. Les maladies radiactives ne sont pas qu'un phénomène somatique. La véritable terreur des radiations pour les transhumains, particulièrement ceux aux niveaux de dosage ls plus élevés tels qu'à la surface de ganylède ou d'autres lunes Jovienne, sont les dommages aux réseaux de neurones. Cela peu amener à des uploads et des sauvegardes buguées. La nanomédecine peut rapidement inverser l'ionisation des composés cellullaire et les nouveaux matériaux permettant une protection plus fine et de meilleru qualité aident, mais la simple magnitude de la radiation émise par certains corps dans le système solaire surapsse même ces derniers. 

L'empoisonnemetn par radioations est uen affaire complexe, et des règles détaillées sont hors du spectre de ce livre. Les sources de radiations incluent entre autres la ceinture Van Allen de la Terre, la ceinture radiocative de Jupiter, la magnétosphère de Saturne, les rayons cosmiques, les éruptions solaires, les matériaux fissibles, les explosions de fusion ou d'antimatière non protégée et les explosions nucléaires. Les effets peuvent varier drastiquement en fonction de la force de la source, le temps d'exposition et le niveau de protection disponible. Les effets immédiats sur les biomorphs (mettant de quelques minutes à 6 heures pour apparaître) incluent les nausées, les vomissements, la fatigue (SOM réduite) ainsi que des dégats physique et un faible montant de stress mental. Cette phase est suivi d'une période qui ressemblant à une rémission, durant de 6 heures à 2 semaines. Après cette étape, la phase terminale démarre, ce qui peut inclure une perte de pillosité, la stérilité, une SOM et une SOL réduites, des dommages importants aux tissus intestinaux et gastriques, des infections, des hémorragies incontrôlées puis la mort. 

Les synthmorphs ne sont pas aussi vulnérable que les biomorphs, mais elles peuvent être endommagées et handicapées par de trés fortes doses de radiations. 

\subsubsection{Atmosphère toxique} 

Neptune, Titan, Uranus, et Vénus ont toutes des atmosphères toxiques. Des atmosphères similaires peuvent être trouvées sur certaines exoplanète, ou avoir été intentionnellement créée comme mesure de sécurité dans un habitat ou une structure. 

Un personnage qui ignore la toxicité de l'atmosphère et qui ne retiens pas immédiatement sa respiration (nécessitant un Test de REF $\times$ 3) subit 10 points de dégats par Tour d'Action. Un personnage qui aprvient à retenir sa repsiration peut tenir un peu plus longtemps; appliquez les règles d'asphixie (p. 194). 

\textbf{Atmosphères Corrosive:} En plus d'être toxique, l'atmosphère de Vénus est la seule natturelement corrosive dans le système. Les amtopshères corrosives sont immédiatement dangereuse: les personnages subissent 10 points de dégats par Tour d'Action, peu importe qu'ils retiennent leur respiration ou pas. Les atmosphères corrosives endommagent également les véhicule et l'équipement ne disposant pas de protection anticorrosive. De tels objets subissent 1 points de dégat par minute. Avec des concentrations plsu élevée, telles que dans les nuages denses d'acide sulfurique de la haute atmosphère Vénusienne, les objets subissent 5 points de dégats par minute. 

\subsubsection{Atmosphère irrespirable} 

De très rare corps planétaire dans le système solaire ont réellement des atmosphère toxiques. Dans la plupart des atmosphère irrespirable, le danger principal pour les transhumains ne possédant pas les modifications ou le système respiratoire nécessaire est le manque d'oxygène. Coinsidérez l'exposition à une atmosphère irrespirable de la même manière que l'asphyxie. 

\subsubsection{Sous l'eau} 

En général, toute compétence physique utilisée sous l'eau subit une pénalité de -20 en raison de la résistance du milieu. Les compétences reposant sur l'équipement non adapté à l'utilisation sous-marine peuevent être encore plus difficile voire impossible à utiliser. L'allure d'un personnage en angeant ou en marchant sous l'eau et le quart de leur allure normale sur la terre ferme. Si un personnage comemnce à se noyer sous-l'eau, utilisez les règles d'asphixie (p. 194). Notez que les personnages en train de s noyer ne récupère pas immédiatement si ils sont sortis de l'eau; ils continueront à s'asphyxier jusqu'à ce qu'un traitement médical soit utilisé pour vider l'eau de leur poumon. 

\subsubsection{Vide} 

Les biomorphs dépourvues de système d'étanchéité au vide (p. 305) peuvent passer uen minute dans le vide spatial sans effet secondaire, du moment qu'ils se recroqueveillent en boule, vident leur poumons et gardent leurs yeux fermés (c'est quelque chose que les enfants des ahabitats spatiaux aprennent trés tôt). Contrairement aux croyances populaires liés aux médias pré-Chute, un personnage exposé au vide total n'explose pas en décompression, de même que ses fluides internes n'entrent pas en ébulitions (autres que les liquides relativement exposs, comme la salive sur al langue). En fait, le danger principal pour les personnages dans le vide sans exocombi est l'asphyxie due au manque d'oxygène et les complications associées telles qu'un œdème des poumons. 

Au delà d'une minute dans l'espace, le personnage commence à subir de l'asphyxie (p. 194). Les dégats sont doublés si le personnage essaye de garder de l'air dans ses poumons ou qu'il n'est aps recroquevillé dans une position de survie dans le vide telle que décrite plus-haut. Additionnelement, les personnages piégés dans l'espace sans protection thermique adéquate subissent 10 points de dégats par minute en raison du froid extrême. 

\subsection{Armes improvisées} \label{sec:improvised-weapons} 

Quelque fois les personnages seront pris par surprise et ils devront utiliser ce qu'ils auront sous la main comme arme - à moins qu'ils ne pensent avoir l'air cool à tabasser quelqu'un avec un mètre de chaines. La table des Armes Improvisées propose des statistiques pour quelques objets susceptibles d'être utilisés comme arme. Les maitres de jeu peuvent utiliser ces guides pour gérer les objets non listés. 

\begin{table} \begin{tabularx}{\textwidth}{|l|l|l|l|X|} \hline

\multicolumn{5}{|c|}{\textbf{Armes improvisées}} \\ \hline

\textbf{Armes}	&\textbf{PA}	&\textbf{Valeur de dégats (VD)}	&\textbf{VD moyenne}	&\textbf{Compétence}	\\ \hline

Balle de baseball	&$-$	&(1d10 $\div$ 10) + (SOM $\div$ 10)	&2 + (SOM $\div$ 10)	&Armes de jet	\\ \hline

Bouteille	&$-$	&1 + (SOM $\div$ 10), Se casse après une utilisation	&1 + (SOM $\div$ 10)	&Massues ou armes de jets	\\ \hline

Bouteille (cassée)	&$-$	&1d10 $-$ 1 (min: 1)	&4	&Lames	\\ \hline

Chaînes	&$-$	&1d10 + (SOM $\div$ 10)	&5 + (SOM $\div$ 10)	&Armes de mélée exotique\\ \hline

Casque	&$-$	&1d10 + (SOM $\div$ 10)	&5 + (SOM $\div$ 10)	&Massues ou armes de jets	\\ \hline

Torche à plasma	&$-$6	&2d10	&11	&Arme à Distance Exotique \\ \hline

Pied de biche	&$-$ &1d10 + (SOM $\div$ 10)	&5 + (SOM $\div$ 10)	&Massues	\\ \hline

\end{tabularx} \label{tab:improvised-weapons} \end{table} 

\subsection{Projetter} \label{sec:knockdown-knockback} 

Si le but d'un attaquant est simplement de mettre à terre ou d'éloigner un adversaire au contact au lieu de le blesser, lancer l'attaque et la défense comme d'habitude. Si l'attaquant réussit, le défenseur est projetté en arrière d'1 mètre par tranche complète de 10 points de MdR. Pour jetter à terre un adversaire, l'attaquant doit obtenir une Réussite Exceptionnelle (MdR de 30+) Une attaque de projection doit être déclarée avant que les dés ne soient jettés. 

À moins que l'attaquant n'obtiennent une réussite critique, aucun dégâts n'est infligé avec cette attaque, le défenseur est juste projetté/mis à terre. Cependant, si l'attaquant obtient une réussite critique, appliqez les dégats comme d'habitude en plus de l'effet de projection. 

Notez qu'un personnage blessé par une attaque peut également être jetté à terre (voir \emph{Effets des Blessures }, p. 207). 

\subsection{Bonus de dégat en mélée et des armes de jets} \label{sec:melee-thrown-damage-bonus} 

Chaque attaque réussit au corps à corps et avec les armes de jet, que ce soit à mains nue ou avec des armes, reçoit un bonus égal à la SOM de l'attaquant $/div$ 10, arrondissez à l'inférieur. Voir \emph{Bonus de dommage }, p. 123. 

\subsection{Cibles multiples} \label{sec:multiple-targets} 

Lorsque vous infligez des dégâts, il n'y a aucune raison de ne pas partager avec les autres. 

\subsubsection{Combat au contact} 

Un personnage qui prend une Action Complexe pour démarrer une attaque de mélée peut choisir d'attaquer deux adversaire ou plus avec la même action. Chaque adversaire doit être à moins d'un mètre d'une autre cible de l'attaquant. Ces attaques doivent être déclarée avant que les dés ne soient lancés pour la première attaque. Chaque attaque souffre d'un modificateur cumulatif de -20 pour chaque cible supplémentaire. Un personnage qui déclare qu'il va attaquer trois personnes avec la même action subira un modificateur cumulatif de -60 sur chaque attaque. 

\subsubsection{Combat à distance} 

Un eprsonnage faisant feu deux fois avec uen arme semi-automatique lors d'une Actrion Complexe peut cibler différent adversaire à chaque tir. Dans ce cas, l'attaquant subit un modificateur de -20 contre la deuxième cible. 

Un personnage tirant en rafale peut cibler jusqu'à deux cibles à chaque rafale, tant que ces cibles sont éloignées d'un mètre maximum l'une de l'autre. Cela est considéré comme une seule atatque; voir \emph{Tir en Rafale}, p. 198. 

Un personnage tirant en rafale deux fois avec une Action Complexe peut cibler une personne différente ou une paire d'autres personne à chaque rafale. Dans ce cas, la deuxième rafale subit un modificateur de -20. Cce modificateur ne s'applique pas si la même personne/pair de personne ciblée avec la première rafale est de nouveau ciblée. 

Les attaques en mode automatique peuvent aussi être dirigée sur plus d'une cible, tant que chaque cible est à moins d'un mètre la précédente. Cela est considéré comme une seule atatque; voir \emph{Tir Automatique}, p. 198. 

\subsection{Objets et structures} \label{sec:objects-structures} 

Comme tout apuvre mur dans le vosiinnage d'une violente fusillade pourra vous le dire, les objets et les structures ne sont pas immunisés à la violence et à la dégradation. Pour refléter ceci, les objets inanimés et les structures ont des scores de Solidité, de Seuil de Blessure et d'Armure, tout comme les personnages. La Solidité mesure la quantité de dommage que peut subir la structure avant d'être détruite. L'Armure réduit les dommages infligés par les attaques, comme pour les personnages. Pour simplifier, un seul score d'Armure est donné et compte à la fois comme une Armure Cinétique et une Armure Énergétique; à la discrétion du maître de jeu, ces scores peuvent être modifiées en focntion de al situation. 

Les Blessures subient par les objets et les structures n'ont aps les mêmes effets que les blessures infligés aux auttres personnages. Chaque blessure est simplement considérées comme un trou, une démolition partielle ou une fonctionnalité réduite selon ce que le maître de jeu pense convenir le mieux. De manière alternative, un appareil endommagé peut fonctionner de manière moisn efficace, et donc infliger un modificateur négatif aux tests de compétences fait pendant l'utilisation de cet objet (un modificateur cumulatif de -10 par blessure). 

Dans le cas des grosses structures, il est recommander de traiter les parties individuelles de la structure comme des entités séparées dans le cadre des dégats subit. 

\subsubsection{Attaques à distance} 

Les attaques à distance n'infligent qu'un tiers de leurd égats (arrondisse à l'inférieur) aux grosse structures telles que les portes, les murs, etc Cela reflète le fait que la plupart des attaques à distance ne font que pénétrer la structure, ne faisant que des dégats mineurs. 

Les agoniseurs et les étourdisseurs n'ont aucun effets sur les objets et les structures. 

\subsubsection{Tirer à travers} 

Si un personnage essaye de tirer à travers un objet ou une structure sur une cible de l'autre cpté, l'attaque subira un modificateur de tir en aveugle d'au moins -30 sauf si l'attaquant peut voir sa cible d'une manière ou d'une autre. En plus de ce modificateur, la cible recçoit un bonus d'armure égal au score d'Armure $\times$ 2 de l'objet/la structure. 

\begin{table} \begin{tabularx}{\textwidth}{|X|r|r|r|} \hline

\multicolumn{4}{|c|}{\textbf{Exemple d'objets et de structures}} \\ \hline

\textbf{Objet/structure} &\textbf{Armure} &\textbf{Solidité} &\textbf{Seuil de blessure}	\\ \hline

Composites avancées (coque de vaisseau/d'habitat)	&50	&1000	&200	\\ \hline

Aérogel (murs, fenêtre, etc)	&-	&50	&10	\\ \hline

Porte de sas	&15	&100	&25	\\ \hline

Alliage, bétons, polymère renforcés (portes/murs renforcés)	&30	&100	&20	\\ \hline

Verre blindé	&10	&50	&20	\\ \hline

Comptoir	&7	&60	&12	\\ \hline

Bureau	&5	&50	&10	\\ \hline

Lien ecto	&-	&6	&1	\\ \hline

Mousse metallique (murs, portes, etc.)	&20	&70	&15	\\ \hline

Vitres métalliques	&30	&150	&30	\\ \hline

Polymère ou bois (murs, portes, meubles, etc)	&10	&40	&8	\\ \hline

Lien de farcast quantique	&3	&20	&4	\\ \hline

Aluminium transparent (murs, meubles)	&5	&60	&12	\\ \hline

Arbre	&2	&40	&10	\\ \hline

Fenêtre	&-	&5	&1	\\ \hline

\end{tabularx} \label{tab:sample-objects-structures} \end{table} 

\subsection{Portée} \label{sec:range} 

Chaque type d'arme a une portée limitée, au-delà de laquelle elle devient inefficace. La portée efficace d'une arem est ensuite divisée en quatre catégories: Courte, MOyenne, Longue et Extrême. Un modificateur est appliqué pour chaque catégorie, tel que noté dans la table des Modificateurs de Combats, p. 193. Pour des exemples de portée d'arme spécifique, reportez-vous à la table des Portées d'Armes. 

\begin{table} \begin{tabularx}{\textwidth}{|X|r|r|r|r|} \hline

\multicolumn{5}{|c|}{\textbf{Portée d'Armes}} \\ \hline

\textbf{Arme (type)} &\textbf{Courte} &\textbf{Moyenne (-10)} &\textbf{Longue (-20)} &\textbf{Extrême (-30)}\\ \hline

\multicolumn{5}{|l|}{\emph{Armes à feu}} \\ \hline

Pistolet Léger	&0-10	&11-25	&26-40	&41-60	\\ \hline

Pistolet	&0-10	&11-30	&31-50	&51-70	\\ \hline

Pistolet Lourd	&0-10	&11-35	&36-60	&61-80	\\ \hline

Mitraillette	&0-30	&31-80	&81-125	&126-230	\\ \hline

Fusil d'Assaut	&0-150	&151-250	&251-500	&501-900	\\ \hline

Fusil de Précision	&0-180	&181-400	&401-1100	&1100-2300	\\ \hline

Mitrailleuse	&0-100	&101-400	&401-1000	&1001-2000	\\ \hline

\multicolumn{5}{|l|}{\emph{Armes à rails}}\\ \hline

\multicolumn{5}{|l|}{identique aux Armes à Feu mais augmentez la portée effective de chaque catéogrie de +50\%} \\ \hline

\multicolumn{5}{|l|}{\emph{Armes à Rayon}} \\ \hline

Laser de Main Cybernétique	&0-30	&31-80	&81-125	&126-230 \\ \hline

Laser à Impulsion	&0-30	&31-100	&101-150	&151-250 \\ \hline

Agoniseur à Micro Ondes	&0-5	&6-15	&16-30	&31-50 \\ \hline

Bolter à Particule	&0-30	&31-100	&101-150	&151-300 \\ \hline

Fusil à Plasma	&0-20	&21-50	&51-100	&101-300 \\ \hline

Étourdisseur	&0-10	&11-25	&26-40	&41-60 \\ \hline

\multicolumn{5}{|l|}{\emph{Chercheurs}} \\ \hline

Micromissile Chercheur	&5-70	&71-180	&181-600	&601-2000 \\ \hline

Minimissile Chercheur	&5-150	&151-300	&301-1000	&1001-3000 \\ \hline

Missile Chercheur	&5-300	&301-1000	&1001-3000	&3001-10000 \\ \hline

\multicolumn{5}{|l|}{\emph{Armes à Spray}} \\ \hline

Buzzer	&0-5	&6-15	&16-30	&31-50\\ \hline

Geleur	&0-5	&6-15	&16-30	&31-50\\ \hline

Pistolet à Aiguilles	&0-10	&11-30	&31-50	&51-70\\ \hline

Déchiquetteur	&0-10	&11-40	&41-70	&71-100\\ \hline

Spray	&0-5	&6-15	&16-30	&31-50\\ \hline

Torche	&0-5	&6-15	&16-30	&31-50\\ \hline

Canon Vortex	&0-5	&6-15	&16-30	&31-50\\ \hline

\multicolumn{5}{|l|}{\emph{Armes de Jets}} \\ \hline

Lames &Jusqu'à SOM $\div$ 5 &Jusqu'à SOM $\div$ 2 &Jusqu'à SOM &Jusqu'à SOM $\times$ 2 \\ \hline

Minigrenades &Jusqu'à SOM $\div$ 2 &Jusqu'à SOM &Jusqu'à SOM $\times$ 2 &Jusqu'à SOM $\times$ 3 \\ \hline

Grenades &Jusqu'à SOM $\div$ 5 &Jusqu'à SOM $\div$ 2 &Jusqu'à SOM &Jusqu'à SOM $\times$ 3 \\ \hline

\end{tabularx} \end{table} 

\subsubsection{Portée, gravité et vide} 

Les portées listées dans la table des Portées d'Armes sont valable pour lees gravités de type Terrienne (1g). Alros que la portée effective des armes cinétiques, chercheuses, à spray et de jets peuvent potentiellement augmenter dans les environnements en faible gravité en raison du manque de forces gravitationnelle ou de frain aérofynamique, la précision reste le facteur déterminant si vous toucher la cible ou non. En gravité réduite, utilisés les mêmes portées effectives que celles listées, mais étendez la portée maximale en la divisant apr la gravité (par exemple, une portée maximale de 100 mètre, sera de 200 mètres en 0,5g). En microgravité et en zéro-g, la portée maximale est la ligne de vue effective. De manière similaire, dans les environnements en haute-gravité (au-delà de 1g), divisez le maximum de chaque catégorie de portée  par la gravité (i.e, une portée courte de 10 mètres sera réduite à 5 en 2 g). 

Les arems à rayons ne sont pas affectées par la gravité, mais sont bien plus efficace dans des conditions non-atmosphérique. La portée maximale des armes à rayon dans le vide est la ligne de vue effective. 

\subsection{Allonge} \label{sec:reach} 

Certaines armes augmentent l'allonge d'un personnage, lui donnant un avantage significatif face à un adversaire en combat de mélés. Cela s'applique à toutes les armes de plsu d'un mètre de long: haches, masses, épées, bâtons chocs, etc. Lorsqu'un personnage a l'avantage de l'allonge sur son adversaire, il reçoit un modificateur de +10 en attaque et en défense. 

\subsection{Dispersion} \label{sec:scatter} 

Lorsque vous utilisez une arme à explosion, vous pouvez toujours toucher votre cible grâc au souffle même si vous ne parvenez pas à la toucher directement. Les arems telles que les grenades doivent aller quelquepart lorsqu'elle ratent, et leur position exacte peut être importante pour l'issue de la bataille. Pour déterminer où aterri une attaques à explosion ratée, la règle de dispersion entre en jeu. 

pour déterminez la dispersion, lancer 1d10 et notez dans quelle direction "pointe" le dé (en vous servant de votre position comme référence). C'est la direction par rapport à la cible dans laquelle l'explosif va finalement aterrir. Le résultat du dé détermine également la distance de dispersion, en mètre. Si la MdE de l'attaque est de plus de 30, cette distance est doublée. Si la MdE dépasse 60, la distance est triplée. Ce point détermine l'épicentre de l'explosion; résolvez les effets des dommages contre tout ceux qui sonr pris dans la sphère d'effet comme d'habitude (voir \emph{Effets de Souffle}, p. 193). 

\subsection{Attaque de choc} \label{sec:shock-attacks} 

Les armes chocs utilisent des décharges électriques à haute tension pour étourdir physiquement et handicaper les cibles. Les armes à chocs sont particulièrement efficaces contre les biomorphs et les pods, même si ils sont lourdement protégés. Les synthmorphs, les bots et les véhciules sont immunisés aux effets des armes à chocs. Une biomorph piégées par une armes à chocs doit fair un Test de SOL + Armure Énergétique (utilisant leur score actuel de SOL, réduit par les dégats déjà subit). Si il échoue, il perd immédiatement le contrôle neuromusculaire, tombe, et est paralysé pendant 1 Tour d'Action par tranche complète de 10 points de MdS (minimum de 3 Tours d'Actions) Pendant cette période, il est étourdi et incapable d'accomplir la moindre action, il convulse probable, souffre de vertige, de nausées, etc. Après cette épriode, il peut agir mais il reste étourdit et en état de choc, subissant un modificateur de -30 à toutes ses actions. Ce modificateur réduit de 10 par minutes (et passe donc à -20 après 1 minutes, -10 après 2 minutes et disparaît au bout de 3). Beaucoup d'armes à chocs infligent également une VD, qui est gérée de manière normale. 

Une biomorph qui réussit son test de SOL est toujours choquée mais n'est pas paralysée. Elle souffre de la moitié de la VD indiquée et subit un modificateur de -30 jusqu'à la fin du prochain Tour d'Action. Ce modificateur réduit de 10 par Tour d'Action. Les modificateurs d'autres chocs ne sont pas cumulatif, mais ils élèveront le modificateur à sa valeru maximale. 

\subsection{Contrôle} \label{sec:subdual} 

Pour saisir un advesaire en combat au contact, vous devez déclarez votre intention de le contrôller avant de lancer les dés. Toute compétence de mélée appropriée peut être utilisée pour l'attaque; si une arem est utilisée, elle peut être utilisée comme faisant parti de la technique de prise. Si vous réussissez votre attaque avec une Réussite Exceptionnelle (MdR de 30+), vous avez réussit à contrôler votre adversaire (du moins, pour le moment). Les attaques de saisie n'infligent aps de dégats sauf si vous obtenez un succès critique (et même dans ce cas là, vous pouvez décider de ne pas faire de dégâts). 

Un adversaire contrôlé est temporairement maîtrisé ou immobilisé. Il peut communiquer, utiliser des compétences mentales et agir sur le mesh, mais ils ne peuvent faire aucune action physique autre que de tenter de s'échapper. (À la discrétion du maîþre de jeu, il peut toujours faire de pettites actions physique restreinte, telle qu'essayer d'attraper un couteau dans leur poche ou saisir un objet lancé à quelques centimètre sur le sol, mais ces actions devraient subir un modificateur d'au moins -30 et seront remarqué par le contrôleur). 

Pour se libérer, un personnage saisi doit prendre une Action Complexe et réussir soit un Test en Opposition de Combat à Mains Nues ou un Test en Opposition de SOM $\times$ 3, bien que le personnage contrôlé souffre d'un modificateur de -30 sur son test. 

\subsection{Tir suppressif} \label{sec:suppressive-fire} 

Un personnage tirant avec une arme en mode automatqiue (p. 198) peut choisir d'utiliser un tir suppressif sur une zone au lieu de viser quelqu'un spécifiquement, dans le but de forcer tout ceux qui sont dans la zone de rester à couvert. Cela prend une Action Complexe, utilise 20 munitions et dure jusqu'à la prochaine Phase d'Action du personnage. La zone balayée s'étend en cône, le diamètre le plus large du cône pouvant faire jusqu'à 20 mètres. Tout personnage qui n'est pas derrière un couvert ou qui ne se met pas immédiatement à couvert lors de son action risque de se faire toucher par le tir suppressif. S'il sort d'un couvert à l'intérieur de la zone de feu, le personange qui déclenche le tir suppressif gagne une attaque gratuite contre lui, attaque dont ils peuvent se défendre comme d'habitude. Aucun modificateurs ne sont à appliquer à ces tests, exceptés ceux liés à la portée, aux blessure et à la défense totale. Si il est touché, le personnage doit résister aux dégats comme si ils s'agissaient de ceux d'un tir unique. 

\subsection{Surprise} \label{sec:surprise} 

Les personnages désirant tendre une embuscade à d'autres doivent chercher à bénéficier de l'avantage de la surprise. Cela signifie en général de se faufiler discrètement, d'attendre dans les ombres ou de camper depuis une position éloignée difficile à percevoir. À chaque fois qu'un embusqué (ou un groupe d'embusqués) tentent de surprendre une cible (ou un groupe de cible), faites un test de Perception secret pour les cibles de l'embuscade. À moins qu'ils ne s'attendent à une surprise, ce test devrait typiquement subir le modificateur de -20 pour être distrait. Il s'agît d'un Test en Opposition contre la compétence Infiltration de l'embusqué/des embusqués. En fonction de la position de l'attaquant, d'autres modificateurs peuvent également s'appliquer (distance, visibilité, couvert, etc.). 

Si le Test de Perception échoue, le personnage est surpris par l'attaque et ne peux pas y régair ou se défendre contre. Dans ce cas, donnez simplement aux attaquants une Phase d'Action gratuite pour attaquer les personnages surpris. Une fois que les attaquants ont effectués leurs actions,lancez l'Intitiative comme d'habitude. 

Si le Test de Perception réussit, le personnage est alerté une fraction de seconde avant qu'il ne tombe dans l'embuscade, lui donnant une chance de réagir. Dans ce cas, lancez l'Initiative de manière normale, mais les personnages embusqués subissent un modificateur de -30 à leur Test d'Initiative. Les personnages embusqués peuvent toujours se défendre de manière normale. 

Dans les situations de groupe, les choses peuvent devenir plsu complexe lorsque certains personnages sont surpris et que d'autres ne le sont pas. Dans ce cas, déterminez l'Inititaive de manière normale, avec tous les personnages non-embusqués subissant un modificateur de -30. Tous les personnages qui sont surpris ne peuvent tout simplement pas agir lors de la première Phase d'Action, ils sont sans défense et doivent prendre un instant pour évaluer la situation et rattraper leur retard. Comme au-dessus, les personnages surpris ne peuvent pas se défendre lors de leur première Phase d'Action. 

\subsection{Réseaux tactiques} \label{sec:tactical-networks} 

Les réseaux tactiques sont des logiciels spécialisés utilsiés par les équipes qui bénéficient du partage de données tactiques. Ils sont fréquemment utilisés par les équipes sportives, les équipes de sécurités, les unités militaires, les joueurs RA, les resquilleurs, les surveillants, le smineurs, les crontrôlleurs du traffics, les récupérateurs et quiconque nécessite d'un aperçu tactique de la situation. Les équipes de Firewall profitent régulièrement des avantages de tels logiciels. 

En terme de jeu, les tacnets fournissent des compétences et des outils logiciels spécialisés aux muses ou aux IA, en fonction de ce qui convient le mieux aux besoin tactiques. Ces outils se connectent les uns aux autres et partagent et analysent les données entre tous les participants du réseau, créant un affichage entoptique configurable pour chaque utilisateur qui résume les donnés capitales, mets en évidence les interactions et les priorités, et alerte l'utilsie des choses qui nécessitent son attention. 

\subsubsection{Tacnets de combat} la liste suivante est un exemple des caractéristique standard d'un tacnet de combat. Les maîtres de jeu sont encouragés à modifier et à étendre ces options de manière appropriée à leur jeu: 

\begin{itemize} \item \textbf{Cartes:} Les tacnets assemblent toutes les cartes disponible et peuvent les présenter à l'utilisateur avec un point de vue aérien ou comme un ensemble tridimensionnele et interactif, les distances entre les caractéristiques intéressantes étant facilement acessible. L'IA ou la muse peut également tracer des cartes en se basant sur les netrées sensorielles, les systèmes de fils d'arianne (p. 332), et d'autres données. Des chemins et d'autres données de ces cartes peuvent être affichées comem image entoptique ou par d'autres données RA (par ex, un utrilisateur qui doit trouner à gauche pourrait voir une flêche rouge transparente ou sentir un picotement sur leur côté gauche). \item \textbf{Positionnement:} Le positionnement exact de l'utilisateur et de tous les autres participants sont mis à jour et cartographié grâce au positionnement apr mesh et par GPS. De manière similaire, le positionnement de personnes connues, de bots, de véchiukes et d'autres caractéristiques peuvent également être indiquées en fonction des entrées sensorielles. \item \textbf{Entrée Sensorielle:} Toute entrée sensorielle disponible pour un des personnage participant ou un appareil dans le réseau peut alimenter le système et être partagée. Cela inclut les données des sens cybernétique, les senseurs portables, les caméra des smartlink, les sorties XP, etc. Cela permet à un utilisateur d'accéder immédiatement au flux sensoriel d'un autre. \item \textbf{Gestion de Communications:} Le tacnet maintiens un lien chiffré entre tous les utilisateurs et reste conscient à la fois des utilisateurs qui se retirent ou des tentatives d'intrusion ou d'interférer avec le lien de communication. \item \textbf{Données de Smartlink/d'Armes:} Le tacnet supervise l'état des armes, des accessoires et d'autres équipement via l'interface smartlink ou un lien sans fil, rapportant les dégat, les épuisements de munitions et les autres problèmes à l'attention de l'utilisateur. \item \textbf{Tir Indirect:} Les membres d'un tacnet peuvent fournir ds données de ciblage aux autres dans le but de tirs indirects (p. 195). \item \textbf{Analyse:} Les muses et les IA participant au tacnet sont propulsées par des logiciles d compétences et des bases de données qui leur permettent d'interpréter les données entrantes et les flux sensoriels. C'est probablement l'aspect le plus utile des réseaux tactiques. Cela signifie que les muses/IAs peuvtn remarquer des faits ou des détails que les utilisateurs indivicuels n'auraient pas remarqués. Par exemple, le tacnet peut compter les tir effectués par un adversaire, noter lorsqu'ils vont se retrouver à cours, et même analyser les entrées sensorielles pour déterminer le type d'armement et de mhnitions utilisées. Les adversaires et leur matériel peuvent également être scannés et analysé pour noter les faiblesses potentielles, les blessures et les capacités. Si un contact sensoreil avec un adversaire ets perdu, sa dernière position connue esst mémorisé et les vecteurs de mouvements potentiels ainsi que les distances sont affichés. Le positionnement des adversaire peut également identifier leur lignes de vue et les champs de tirs, signalant les couverts potentiel ou les  zones de danger à l'utilisateur. Le tacnet peut également suggérer des manœuvres tactiques qui aiderons l'utilisateur, telles que le contournement d'un adversaire ou la possibilité de récupérer une postition plus élevée. \end{itemize} 

La plupart de ces caractéristiques sont immédiatement disponible à l'utilisateur via leur afficage RA; d'autres données peuvent être accédée avec une Action Rapide. De manière similaire, le maître de jeu décides quand la muse/l'IA fournit des alertes importantes aux utilisateurs. À la discrétion du maître de jeu, certaines de ces caractéristiques peuvent ajouter des modificateurs aux tests des personnages. 

\subsection{Attaque de toucher} \label{sec:touch-only-attack} Certains type d'attaque nécessite simplement que vous touchiez votre cible au lieu de les blesser, et sont donc plus facile à réaliser. Cela devrait s'appliquer lorsque vous essayer de coller un patch dermal à quelqu'un, que vous essayez de répandre un poison de contact sur sa peau ou que vous essayez d'obtenir un contact peau à peau pour utilsier un exploit psi. Dans ces situations, appliquez un modificateur de +20 aux attaques au contact. 

\subsection{Armes à Deux Maines} \label{sec:two-handed-weapons} 

Toute arme indiquée comme étant à deux mains nécessite l'usage de deux mains (ou autres membres préhensile) pour être maniée efficacement. Cela est avalbles pour certaines armes de mélée archaïque (épée longue, lance, etc) ainsi qu'à certaines armes à feu plus grosse et aux armes lourdes. Tout personaneg qui tente d'utiliser de telles armes d'une seule main subit un modificateur de -20. Ce modificateur ne s'applqiue pas aux armes montées. 

\subsection{Manier deux armes ou plus} \label{sec:weilding-two-or-more} 

Il est possible qu'un personnage manie deux armes en combat, ou même plus si c'est une octomorph ou une synthmorph ayant plusieurs membres. Dans ce cas, chaque arme supplémentaire maniée sur une mauvaise main subit le modificateur de mauvaise main de -20. Ce modificateur peut être annulé par le trait Ambidextre (p. 145). 

\subsubsection{Arme de mélée supplémentaire} 

L'utilisation de deux armes de mélée ou plus est traitée comme une seule attaque et non comme plusieurs. Chaque armes additionelle ajoute +1d10 de dégats à l'attaque (jusqu'à un amximum de +3d10). Les modificateurs de mauvaises mains sont ignorés. Si le personnage attaque des cibles multiples avec la même Action Complexe (voir \emph{Cibles Multiples}, p. 202), ces bonus ne s'appliquent pas. L'attaquant doit, bien entendu, être capable de manier l'arme additionelle. Un spliceur avec seulement deux mains ne peut pas manier un couteau et une épée à deux mains par exemple. De manière similaire, le maître de jeu peut ignorer le bonus de dégats pour les armes supplémentaires qui sont trop différentes pour être utilisées ensemble efficacement (comme un fouet et une queue de billard). Notez que ni les membres supplémentaires, ni les armes qui sont concues par paires commes les gants chocs) ne comptent comme une arme supplémentaire.  

Un personnage utilisant plus du'ne arme de mélée reçoit un bonus pour se défendre contre les attaques de mélée égal à +10 par armes supplémentaires (maximum +30). 

\subsubsection{Armes à distance supplémentaire} 

Un attaquant peu également manier un pistolet dans chaque main pour le combat à distance, ou de plsu grosses armes si ils ont plus de membres (un poulpe à huit membre peut, par exemple, manier quatres fusils d'assaut). Ces armes doivent toutes faire feu en une fois sur la même cible. Dans ce cas, chaque arme est gérée comme une attaque séparée, chaque armes dans une mauvaise main subissant un modificateur cumulatif de mauvaise main de -20 (pas de modificateur sur la première attaque, -20 pour la deuxième, -40 pour la troisième et -60 pour la quatrième), compensés par le trait Ambidextre (p. 145) comme d'habitude. 

\section{Santé Physique} \label{sec:physical-health} 

Dans un cadre aussi dangereux que celui d' \emph{Eclipse Phase}, les personnages vont inévitablement être blessés. Que votre orph soit biologique ou synthétique, vous pouvez être blessés par des armes, une bagarre, une chute, des accidents, des environnements extrêmes, des attaques de psi et ainsi de suite. Cette section traite de la façon de garder une trace de telles blessures et de déterminer quels effets elles auront sur votre peronnage. Deux méthodes sont utilisées pour mesurer la santé physique de votre personnage: points de dégats et blessures. 

\subsection{Points de dégats} \label{sec:damage-points} 

Toute blessure physique infligée à votre personnage est mesurée en points de dégats. Ces points sont cumulatifs, et sont notés sur votre fiche de personnage. Les points de dommage sont caractérisés par la fatigue, l'étoudissement, les bleus, les chocs, les entorses, les petites coupures et les autres blessures qui, tout en étant douloureuse, ne vont pas géner ou menacer significativement la vie de votre personnage à moins qu'elles ne s'accumulent jusqu'à un certain niveau. Toute source de blessure qui inflige une grosse quantité de points de dégats en une fois peut cependant avoir un effet plus important (voir \emph{Blessures}, p. 207). 

Les points de dommages peuvent être réduits par le repos, l'attention médicale, et/ou la réparation (voir \emph{Soins et Réparations }, p. 208). 

\subsection{Types de dégats} \label{sec:damage-types} 

Les dégats physiques sont de trois type: Énergétique, Cinétique et Psi. 

\subsubsection{Dégats énergétiques} 

Les dégats énergétiques incluent les lasers, les armes à plasma, le feu, l'électrocution, les explosions et d'autres sources de dégats énergétiques. 

\subsubsection{Dégats cinétiques} 

Les dégats cinétiques sont causés par les projetciles et les objets se déplaçant à haute vitesse et qui transfèrent leur énegrie à la cible au moment de l'impact. Les attaques cinétqiues incluent les frondes, les armes à fléchettes, les couteau et les coups. 

\subsubsection{Dégats Psi} 

Les dégats Psi sont causés par les exploits psi offensifs comme Attaque Psychic (p. 228). 

\subsection{Solidité et santé} \label{sec:durability-health} 

La santé physique de votre personnage est mesurée par leur stat de Solidité. Pour les personnages incarnés dans des biomorphs, ce nombre représente le niveau à partir duquel les dégats accumulées submergent votre personnage et qu'il sombre dans l'inconscience. Une fois que vous avez accumulés un nombre de points de dégats supérieur ou égal à votre stat Solidité, vous vous effondrez immédiatement sous l'épuisement et la violence physique. Vous demeurez inconscient et ne pouvez être réveillez tant que vos points de dégats n'ont pas été réduits sous votre Solidité, grâce à des soins médicaux ou par guérison naturelle. 

Si vous êtes morphé dans une coque synthétique, la Solidité représente votre intégrité structurelle. Vous devenez physiquement invalide lorsque vos points de dégats accumulez atteignent votre Solidité. Bien que les systèmes informatiques sseront probablement toujours fonctionnels, vous pouvez continuer d'accéder au mesh, mais votre morph est cassé et immobile jusqu'à ce qu'elle soit réparée. 

\subsubsection{Mort} 

Une accumulation extrême de points de dégats peut menacer la vie de votre personnage. Si les dégâts atteignent votre Solidité $\times$ 1.5 (pour les biomorphs) ou Solidité $\times$ 2 (pour les morphs synthétiques), votre corps décède. Ce seuil est appelé votre Seuil de Mort. Les morpsh synthétiques qui atteignent cet état sont détruites au-delà de toute réparation. 

\subsection{Valeur de dégats} 

Les armes (et les autres sources de blessures) dans \emph{Eclipse Phase} ont toutes une Valeur de Dégats (DV) - le montant de base de points de dégats infligés par l'arme. Cette valeur est souvent présentée comme une quantité variable, sous le forme d'un jet de dé; par exemple: 3d10. Dans ce cas, vous lancez trois dés à dix faces et vous additionnez els résultats (encomptant le 0 comme un 10). Parfois la VD sera présentée comme un jet de dé plus un modificateur; par exemple: 2d10 + 5. Dans ce cas vous lancez deux dés à dix faces, les additionnez, puis y ajoutez 5 pour obtenir le résultat. 

Pour simplifier, une quantité statique de dommage est également notée entre parenthèse après la valeur variable. Si vous préférez supprimer le lancer de dés, vous pouvez simplement apppliquez la valeur statique(générallement proche de la médiane) à la place. Par exemple, si les dégats sont notés 2d10 + 5 (15), vous pouvez simplement appliquer 15 points de dégâts au lieu de lancer le dé. 

Lorsque des dégats sont infligés à un personnage, déterminez la VD (lancer les dés) et soustrayez la valeur d'armure telle que noté à \emph{Étape 7: Déterminez les Dégâts} (p. 192). 

\subsection{Blessures} \label{sec:wounds} 

Les blessures représentent des belssures plus sérieuses: mauvaises coupures et hémoragie, fractures et cassures, membrs mutilés et autres dommages sérieux qui handicapent votre capacité à fonctionner et peuvent se terminer par la mort ou des dégats à long terme. 

À chaque fois que votre personnage encaisse des dégâts, comaprez le total infligé (arpès l'avoir réduit par l'armure) à votre Seuil de Blessure. Si la VD modifiée atteint ou surpasse votre Seuil de Blessure, vous subissez une blessure. Si les dommages infligés font le double de votre Seuil de Blessure, vous subissez 2 blessures; si c'est le triple, vous subissez 3 blessures; et ainsi de suite. 

Les blessurs sont cumulatives, et doivent être notée sur votre fiche de personnage. 

Notez que cs règles gèrent les dégats et les blessure comme étant un concept abstrait. Pour des raisons dramatiques et de réalisme, le maître de jeu peut souhaiter décrire les blessure d'une manière plus détaillée et macabres: une cheville cassée, un tendon sectionné, des hémorragies internes, une oreille perdue, et ainsi de suite. La nature de ces blessures descriptives peuvent aider le maître de jeu à aasigner d'autres effets. Par exemple, un personnage avec une main broyée pourrait ne pas être capable d'attraper une arme, quelqu'un ayant perdu beaucoup de sang peut laisser une piste que ses ennemis peuvent suivre, ou quelqu'un ayant un œil arraché pourrait subir des modificateurs de perception visuelle. Ces détails peuvent également impacter la manière dont un personnage sera traité ou soigné. 

\subsubsection{Effets des blessures} 

Chaque blessure applique un modificateur de -10 à toutes les actions du personnage. Un personnage ayant 3 blessurs, par exemple, subit un modificateur de -30 à toutes ses actions. Certains traits, morphs, implants, drogues et psis permettent à un personnage d'ignorer les modificateurs de blessures. Ces effets sont cumulatifs, bien que le total de modificateurs de blessure pouvant être annulés est -30. 

\textbf{Jetté au sol:} À chaque fois qu'un personnage prend une blessure, il doit immédiatement faire un Test de SOM $\times$ 3. Les modificateurs de Blessures s'appliquent. Si il échoue, il est jetté à terre et doit dépenser une Action Rapide pour se relever. Les bots et les véhicules doivent réussir un Test de Piloter pour éviter de s'écraser. 

\textbf{Incsonscience:} Â chaque fois qu'un personnage reçoit 2 blessure ou plus en une fois (de la même attaque), ils doivent également faire immédiatement un Test de SOM $\times$ 3; les modificateurs de blessures s'appliquent également. Si il échoue, il sombre dans l'inconscience (jusqu'à ce qu'ils soient réveillés ou soignés). Les bots et les véhicules qui subissent 2 blessures ou plus en une seule fois s'écrasent automatiquement (voir \emph{S'Écraser}, p. 196). 

\textbf{Saigner:} Toute les personnages en biomorph qui ont subit une belssure et qui suboissent des dégâts qui dépassent leur Solidité court le risque de saigner à mort. Ils subissent 1 point de dégats supplémentaires par Tour d'Action (20 par minutes) jusqu'à ce qu'ils reçoivent des soins médicaux ou qu'ils meurent. 

\subsection{Mort} Pour beaucoup de gens dans \emph{Eclipse Phase}, la mort n'est pas le bout du tunnel. Si la pile corticale du personnage peut-être récupérée, il peut être ressuscité et téléchargé dans une nouvelle morph (voir \emph{Réincarnation}, p. 271). Cela nécessite en général une assurance sauvegarde (p. 269) ou les bonnes grâces de quiconque finit par récupérer le corps/la pile. 

Si la pile corticale n'est pas récupérable, le personnage peut toujours être ré-instantié depuis une sauvegarde archivée (p. 268). Cela nécessite toujours soit une assurance sauvegarde soit quelqu'un qui voudra bien payer pour le faire revivre. 

Si la pile corticale du personnage n'est pas récupée et qu'il n'a pas de sauvegarde, ils sont complètement et définitivement mort. Parti. Kapout. (Sauf si ils ont un fork alpha se balladant quelque part; voir \emph{Forker et Fusionner}, p. 273.) 

\section{Soins et réparations} \label{sec:healing-repair} 

utilisez les règles suivantes pour soigner et réparer les personnages blessés et endommagés. 

\subsection{Soin des biomorphs} 

Grâce aux technologies médicales avancée, il y a beauocupd e manière pour les personnages en biomorphs (incluant les pods) de guérir de leur blessure. Les méchidaments nanoscopiques (p. 308) aide les personnages à se soigner rapidement, de même que les nanobandages (p. 333). Les cuves de soin (p. 326) soigenront même les blessures les plus graves en quelques jours, et peuvent même restaurer les personnages récemmenet décédés ou n'ayant plus que leur tête. 

bein enetendu, les personnages n'ayant pas accès à ces outils médicaux ne sont pas sans espoir. Les compétences médicales de professionnels entraînés peuvent réduire l'impact des blessures et les corps se guérissent d'eux-même avec le temps. 

\subsubsection{Attention médicale} 

Les personnages ayant une compétence Médecine appropriée (telle que Madecine: Premiers Soins ou Médecine: Chirurgie traumatique) peuvent feffecteur les premiers soins sur les prsonnages blessés ou endommagés. Un test de Médecine réussit, modifié de manière approprié en fonction de la situation, soignera 1d10 points de dégat et supprimera 1 blessure. Ce test doit être effectué dans les 24 heures suivant la blessure, et chaque blessure ne peut être traitée qu'une fois. Si le personnage est blessé de nouveau un peu plus tard, ces nouveaux dégâts peuvent également être traités. Les soins médicaux de ce type ne sont aps efficace contre les blessures qui ont été traitées avec des méchidaments, des nanobandages ou des cuves de soins. 

\subsubsection{Guérison naturelle} 

Les personnages coincés loins de toute technologie médicale - dans une station éloignée, dans les étendues sauvages de Mars ou n'importe quel équivalent - puevent être forcés de guérir naturellement si ils sont blessés. La guérison naturelle est un processuss lent qui est largement influencé par un grand nombre de facteurs. Afin qu'un personnage soigne ses blessures, tous les dégâts normaux doivent être soignés d'abord. Consultez la table de Guérison. 

\subsubsection{Chirurgie} 

À \emph{Eclipse Phase}, la plupart des blessures graves peuvent être soignées en passant du temps dans une cuve de guérison (p. 326) ou simplement en se reposant et en récupérant. Dans les circonstances où une cuve de guérison n'est pas disponible, le maître de jeu peut décider qu'une blessure particulière nécessite de la chirurgie pratiéquée apr un organisme conscient (qu'il s'agisse d'un personnage ou d'un bot de soin piloté par IA). Dans ces cas là, le personnage sera incapable de se soigner tant qu'il n'aura pas pu bénéficier de la chirurgie. La chirurgie est gérée par un Test de Meédecine utilisant un domaine approprié à la situation et avec un intervalle de 1 à 8 heures. Si le teste réussit, le personnage est soigné d'1d10 point de dégats et d'1 blessure et récupère ensuite depuis cette étape de manière normale. 

\begin{table} \begin{tabularx}{\textwidth}{|X|r|r|} \hline

\multicolumn{3}{|c|}{\textbf{Guérison}} \\ \hline

\textbf{État du personnage}	&\textbf{Vitesse de guérison des dégats}	&\textbf{Vitesse de guérison des blessures} \\ \hline

Personnage sans biomods de base	&1d10 (5) tous les  jours &1 toutes les semaines	\\ \hline

Personnage avec des biomods de base	&1d10 (5) toutes les 12 heures	&1 tous les 3 jours	\\ \hline

Personnage utilisant des nanobandages	&1d10 (5) toutes les 2 heures	&1 tous les jours	\\ \hline

Personnage avec des méchidaments	&1d10 (5) toutes les heures	&1 toutes les 12 heures	\\ \hline

Mauvaises conditions (malbouffe, pas assez de repos/activité importante, mal abrité/mauvaise conditions d'hygiène) &doublez l'intervalle &doublez l'intervalle \\ \hline

Conditions horiibles (malnutrition, pas de repos/activité stressante, pas ou peu d'abri et/ou d'hygiène) &triplez l'intervalle &pas de guérison des blessures\\ \hline

\end{tabularx} \label{tab:healing} \end{table} 

\subsection{Réparation des synthmoprhs et des objets} 

Contrairement aux biomorphs, les synthmorphs et les objets ne guérissent pas des dégâts par eux-même et doivent être réparés. Quelques synthmorphs et périphériques ont des systèmes d'autoréparation nanotechnologique, similaire au méchidaments des biomorphs (voir Réparateurs, p. 329). Les spray de Réparations (p. 333) peuvent également être utilsiés pour faire des réparations et est une option extrêmement utile pour les personnes non-techniciennes. En plus de ces options, les techniciens peuvent également réparer à l'ancienne, utilisant leurs compétences et outils (voir Réparations Physique, ci-dessous). En dernier recours, les synthmorphs et les objets peuvent être réparés dans un nanofabeur avec les schémas appropriés (en utilisant les mêmes règles que pour les cuves de soins, p. 326). 

\subsubsection{Réparation physique} 

Réparer manuellement une synthmorphs ou un objet nécessite un Test de Matériel avec un domaine approprié à l'objet (Matériel: Robotique pour les synthmorpsh et les bots, Matériel: Aéropspatial pour les avions, etc), avec un modificateur de -10 par blessure. Réparer est une Action de Tâche avec un intervalle de 2 heures par tranche de 10 points de dégats à réparer, augmenté de 8 heure par blessures. Des modificateurs appropriés peuvent être appliqués, basés sur les conditions et les outils disponibles. Par exmemples, les outils universels (p. 326) appliquent un modificateur de +20 aux tests de réparation, et les spray de réparation donnent un modificateur de +30. 

\subsubsection{Réparation d'armure} 

Les armures peuvent être réparées de la même manière que la Solidité mais les blessures n'affectent pas les modificateurs ou la durée du test. 

\section{Santé mentale} \label{sec:mental-health} 

À une époque dans laquelle les personnes peuvent abandonner leurs corps et les remplacer avec un nouveau, les traumatismes infligés à votre esprit et à votre ego - votre perception de \emph{soi} - sont souvent plus effrayants que les plus graves des dégâts physiques. Il y a de nombreuses situations qui peuevnt menacer votre intégrité et votre santé mentale: faire l'expérience de la mort, l'isolation prolongée, la perte des gens aimés, les situatiosn étrangères, la discontinuité de sa personne due à la perte de souvenirs ou au changement de morph, les attaques psi et ainsi de suite. Deux méthodes sont utilisées pour mesurer votre santé mentale: les \emph{points de stress} et les \emph{trauma}. 

\subsection{Points de stress} \label{sec:stress-points} 

Les points de stress représentent les fractures dans l'intégrtité de votre égo, les fissures dans l'image mentale de vous-même. Les dommages mentaux sont vécus comme des chcos cérébraux, de la désorientation, des déconnexions cognitives, des courts-circuits synaptique ou une perte des facultés intellectuelles. Ces stress points en eux-même n'handicapent pas significativement le focntionnement de votre personnage, mais si ils commencent à s'accumuler ils peuvent avoir de sérieuses répercussions. Additionellement, toute source qui inflige un montant élevé de points de stress en une seule fois à de bonnes cahnces d'avoir un impact plus grave (voir \emph{Trauma}). 

Les points de stress peuvent être réduit à long-terme par du repos, des soins psychiatrique et/ou de la psychochirurgie (voir p. 214). 

\subsection{Stress et lucidité} \label{sec:lucidity-stress} 

Votre stat Lucidité mesure la stabilité mentale de votre personnage. Si vous accumulez un total de points de stress supérieur ou égal à votre score de Lucidité, l'ego de votre personnage souffre immédiatement d'une dérpession nerveuse. Vous enrez en état de chocs et restez dans un état catatonique jusqu'à ce que vos points de stress soient réduits à un niveau inférieur à votre stat de Lucidité. Les points de stress accumlés surchargeront les egos hébergés dans des coques synthétiques ou des infomorphs de la mrme manière que pour les cerveaux bilogiques - le logiciel mental tombe effectivement en panne, incpabale de fonctionner jusqu'à ce qu'il soit débugué. 

\subsubsection{Seui de folie} 

Le montant extrême de points de stress subit peut endommanger la santé mentale de votre personnage de mnaière permanente. Si les points de stress accumulé atteignent votre Lucidité $\times$ 2, l'ego de votre personnage subit un effondrement permanent. Votre esprit est perdu, et aucune quantité d'aide psychologique ou de repos ne pourra le ramener. 

\subsection{Valeur de stress} \label{sec:stress-value} 

Toute source capable d'inliger un stress cognitif a une Valeur de Stress (VS). Elle indique le nombre de points de stress que l'attaque ou l'expérience inflige à un personnage. Comme la VD, la VS est souvent présenté comme une valeur variable, telle que 2d10, ou parfois avec un modificateur, telle que 2d10 + 10. Lancez simplement les dés et additionnez en les résultats pour déterminer les points de stress  infligés par cet évènement. Pour simplifier les choses, une VS statique est également proposée entre parenthèse juste après le montant variable; utilisez cette valeur lorsque vous voulez garder le rythme de la partie et ne voulez pas lancer de dés. 

\subsection{Trauma} \label{sec:trauma} 

Les tramuamtismes mentaux sont bien plus grave que les poitns de stress. Les traumas représentent des chocs mentaux graves, un effondrement de la personnalité/du soi, le délire, les changements de apradigmes et tous les autres disfonctionnement cognitifs graves. Lestraumas perturbent le fonctionnement de votre personnage et peuvent générer des dérangements temporaires ou des troubles permanents. 

Si votre personnage reçoit en une seule fois un nombre de points de stress supérieur ou égal à son Seuil de Traume, il subit un trauma. Si les points de stress infligés font le double ou le triple de votre Seuil de Trauma, vous subissez 2 ou 3 traumas respectivements, et ainsi de suite. Les traumas sont cumulatives, et doivent être notée sur votre fiche de personnage. 

\subsubsection{Effets des traumas} 

Chaque trauma applique un modificateur de -10 à toutes les actions du personnage. Par exemple, un personnage avec 2 traumas subit -20 à toutes ses actions. Ces odificateurs se cumulent également avec les modificateurs de blessures. 

\textbf{Désorientation:} À chaque fois qu'un personnage souffre d'un trauma, il doit immédiatement faire un Test de VOL $\times$ 3. Les modificateurs de Traumas s'appliquent. Si il échoue, il est temporairemen étourdi et désorienté, et doit dépenser une Action COmplexe pour reprendre ses esprits. 

\textbf{Troubles et Désordres:} À chaque fois qu'un personnage est frappé d'unt rauma, il souffre d'un trouble temporaire (voir \emph{Troubles Mentaux} Le premier trauma inflige un trouble \emph{mineur}. Si un deuxième trauma est appliqué, le premier troubme passe soit d'un mineur à un trouble \emph{modéré}, ou un deuxième trouble mineur est appliqué (à la discrétion du maître de jeu). De manière similaire, un troisième dérangement fera passer un trouble de modéré à \emph{majeur} ou en infligera sinon un mineur. Il est recommandé de choisir d'augmenter le niveau des troubles, particulièrement si ils sont la conséquence d'un même ensemble de circonstances. Dans le cas des traumas qui résultent de situations et sources distinctes, des troubles différents peuvent être appliqués. 

\textbf{Désordres:} Lrosque quatre trauma ou plus sont infligé à un personnage, un trouble majeur se transforme en désordre. Les désordre représentent des afflictions psychologiques à long-terme, qui nécessitent géénralement des semaiens ou des mois de psychothérapie et/ou de psychochirurgie pour être soignés (coir \emph{Désordres Mentaux}, p. 211). 

\subsection {Troubles mentaux} \label{sec:derangements} 

Les troubles metaux sont une condition mentale temporaire qui sont la conséquence des traumas. Les dérangements sont Mineur, Modéré ou Majeur. Le maîþre de jeu et le joueur devrait coopérer pour choisir quel dérangement appliqué, de manière appropriée au scénario et à la personalité du personnage. 

Les dérangements durent pendant 1d10 $\div$ 2 heures (arrondissez à l'inférieur) ou jusqu'à ce que le personnage reçoive une aide psychiatrique. A la discrétion du maître de jeu, un trouble pourrait durer plus longuement si le personnage n'as aps distancé la source de stress, ou si il reste empétré dans une situation stressante. 

Les effets des troubles mentaux sont fait pour être interprétés. Le personnage devrait intégrer le dérangement aux paroles et actions de leur personnage. Si le maître de jeu pense que le joueur n'intégre pas suffisament les effets, il peut les renforcer. Si le maître de jeu le pense approprié, il peut également utiliser des modificateurs ou des tests pour certaines actions. 

\subsubsection{Anxiété (mineur)} 

Vous subissez une attaque de panique, exhibant les conditions physiologique de peur et d'inquiétude: transpiration, poul accéléré, tremblement, insuffisance respiratoire, migraines, et ainsi de suite. 

\subsubsection{Fuite (mineur)} 

Vous êtes psychologiquement incapable de gérer la source de stress ou des circonstances en lmien avec cette source, et vous l'éviter dans la mesure du possible - allant jusqu'à couvrir vos oreilles, vous mettre en boule ou éteignez vos senseurs si il faut. 

\subsubsection{Vertige (mineur)} 

Vous êtes pris de vertige et désorienté par le stress. 

\subsubsection{Écholalie (mineur)} 

Vous répétez involontairement les mots et les phrases prononcés par d'autres. 

\subsubsection{Fixation (mineur)} 

Vous devenez obsédé par quelque chose que vous avez raté ou par les circonstances qui ont amené à ce stress. Vous êtes complètement obsédé par cela, répétant le comportement, essayant de le réparer, répétant les scénarios dans votre tête à voix haute, et ainsi de suite. 

\subsubsection{Affamé (mineur)} 

Vous êtes soudainement consumé par une un désir écrasat de manger quelque chose - peut-être même quelque chose d'inhabituel. 

\subsubsection{Indécision (mineur)} 

Vous êtes énervé par les causes de votre stress, trouvant difficile de faire des choix ou de choisir une suite d'action. 

\subsubsection{Logorrhoée (mineur)} 

Votre réponse au trauma est de vous lancez dans une conversation excessive. Vous ne pouvez pas vous taire. 

\subsubsection{Nausée (mineur)} 

Les tress vous rend malade, vous froçant à combattre le malaise. 

\subsubsection{Frissons (modéré)} 

Votre température corporelle augmente, vous faisant avoir froid et les frissons s'installent. Vous ne pouvez pas vous réchauffer. 

\subsubsection{Confusion (modéré)} 

Le trauma brouille votre concentration, vous faisant oublier ce que vous étiez en train de faire, vous faisant rater des tâches simples et hésiter face à des décisions simple. 

\subsubsection{Échopraxie (modéré)} 

Vous répétez et imitez involontairement les actions des autres autour de vous. 

\subsubsection{Lunatisme (modéré)} 

Vous perdez le contrôle de vos émotions. Vous basculez de l'extase aux larmes et entrez dans un état de colère sans avertissement. 

\subsubsection{Mutisme (modéré)} 

Le trauma vous mets dans l'incapacité de perler et dans une incapacité totale à communiquer efficacement. 

\subsubsection{Narcissisme (modéré)} 

Dans les secousses du choc mental, la seule chose à laquelle vous arrivez à penser c'est vous-même. Vous cessez de vous occupez des autres autour de vous. 

\subsubsection{Panique (modéré)} 

Vous êtes saturés par la peur et l'énaxiété et cherchez immédiatement à distancer la cause du stress. 

\subsubsection{Tramblements (modéré)} 

Vous tremblez violement, vous empéchant de rester immobile ou de tenir des choses. 

\subsubsection{Trou noir (majeur)} 

Vous fonctionnez en auto-pilote pendant un état de fugue temporaire. Plus tard, vous serez incapable de vous rappeller ce qui est arrivé pendant cette épriode. (les coques synthétique et les infomorphs peuvent faire appel à des enregistrements mémoriel depuis une zone de stockage). 

\subsubsection{Frénésie (majeur)} 

Vous êtes terrifiés par la source du stress et l'attaquez. 

\subsubsection{Hallucinations (majeur)} 

VOus voyez, entendez ou percevez d'une manière ou d'une autre des choses qui ne sont pas là. 

\subsubsection{Hystérie (majeur)} 

Vous perdez le contrôle, paniquant vis à vis de la source du stress. Cela amène typiquement à des crises émotionnelles de pleurs, de rire ou de peur irrationelles. 

\subsubsection{Irrationalité (majeur)} 

Vous êtes tellemen noyé par le stress que votre capacité pour le jugement logique s'effondre. Vous êtes énervé par des offenses imaginaires, vous attendez des autres des choses déraisonnables ou vous acceptez des faits sans preuve convaincantes. 

\subsubsection{Paralysie (majeur)} 

Vous êtes tellement sonné par le trauma que vous êtes effectivement gelé, incapable de prendre des décision ou d'agir. 

\subsubsection{Handicap psychosomatique (majeur)} 

Le trauma vous écrase, handicapant une partie de votre fonctionnement physique. Vous souffrez de maux inexpliqués telles que la perte de la vue ou de l'ouïe, des douleurs fantômes ou l'incapacité à utiliser un mmebre ou toute autres extrémité. 

\subsection{Désordres mentaux} \label{sec:disorders} 

Les désordres reflettent une folie plus permanente. Dans ce cas, "permanent" ne signifie pas nécessairement pour toujours, mais la maladie durea jusqu'à ce que le personnage reçoive une aide psychiatre longue et efficace. Les désordres sont infligés lorsqu'un personnage accumule 4 traumas. le maîþre de jeu et le joueur devrait choisir un désordre qui correspond à la situation et au personnage. 

Les désordres ne sont pas toujours "actifs" - ils restent dormant jusqu'à ce qu'ils soient déclenchés par certaines conditions. Bien qu'il soit toujorus possible d'agir en subissant un désordre, il représente un handicap important à la capacité d'une personne à maintenir des relations normales et à accomplir un travail. Les désordres ne devraient pas être valorisé comme étant de mignones bizarreries d'interprétation. Elle représente la meilleure tentative d'une psyché endommagé à gérer un monde qui s'est effondré d'une certaine manière. De plus, les personnes de beaucoup d'habitats, particulièrement ceux du système intérieur, considèrent toujours les désordres comme une marque de stigmatisation sociela et régaissent généralement négativement vis à vis des personnages handicapés. 

Les personnages ayant acquis en cours d'aventures des désordres peuevnt essayer de s'en débarasser de deux manières différentes, soit par des tentatives de les traiter en jeu (p. 214) ou en les rachetant comme ils le feraient avec un trait négatif (p. 153). 

\subsubsection{Addiction} 

L'addiction est un dsordre qui peut faire référence à toute sorte de comportement addictif concentré sur un comportement ou une substance particulière, au point où l'utilisateur est incapable de fonctionner sans l'addiction et où il est sévèrement handicapé en raison des effets de l'addiction. l'addiction est marquée par un désir de la part du sujet de chercher de l'aide ou à réduire l'utilisation de la substance/acte addictive, mais également par le fait que le sujet peut apsser beaucoup de temps à chercher à satisfaire leur addiction à l'exclusion de toutes activité. C'est un cran au-dessu du trait négatif Addiction listé à la p. 148 - c'est un comportement bien plus paralysant qui compense le fait de passer du time loin de l'addiction. Les addictiosn sont généralement en lien avec le trauma qui a casué le désordre (addictions à la RV ou aux drogues sont encouragées). 

\textbf{Effets Suggérés en  Jeu:} L'accroc ne fonctionne que dans deux états: sous l'influence de leur addiction ou en état de manque. De plus, ils peuvent passer une bonne aprtie du temps à ignorer leurs autres responsabilité en cherchant à satisfaire leur addiction. 

\subsubsection{Atavisme} 

L'atavisme est un désordre qui affecte principalement les élevés. Cela résulte en une régressoin vers un stade précode de non- ou de partiellement-élevé. Ils peuvent faire preuve d'un comportement plus proche de leurs instcints animaux, à moins qu'ils ne perdent une apr de leurs bénéfices d'être élevés tels que la capacité au raisonnement abstrait et au dialogue. 

\textbf{Effets Suggérés en Jeu:} Le jouer et le maître de jeu devrait décider du point auquel l'élevé a regressé et ajuster les pénalités de jeu de manière appropriée. Il est important de noter que d'autres élevés perçoivent les élevés atavistes d'une manière teintée d'horreur et éviterosn en général d'avoir à faire à eux. 

\subsubsection{Trouble du déficit de l'attention/hypercativité (TDAH)} 

Ce désordre se manifeste comme une incapacité marqué à se concentrer sur une seule tâche pur une période de temps étendues, et comme une incapacité à remarquer les détails dans la plupart des situations. Les malades ont tendance à démarrer de nombreuses tâches, en commençant une nouvelle juste après avoir commencé une tâche précédente. Les malades atteints du TDAH peuvent aussi avoir un comportement maniaque qui se manifeste comme uen confiance dans leur capacité à terminer un travail, même si ils vont rapidment y perdre tout intérêt. 

\textbf{Effets Suggérés en Jeu:} Pénalité à la Perception et aux compétences associées. Augmentez le modificateur de difficulté pour les actiosn de tâches, partciulièrement si l'action s'éternise. 

\subsubsection{Autophagie} 

C'ets un dérangement qui n'apparaît générallement que chez les poulpes élevés. C'est une foprme de désordre anxieux charactisé par l'auto cannibalisation des membres. Les sujets affligé par l'autophagie vont, lorsqu'ils sont soumis au stress, commencé à manger leurs mmebres, si c'est possible, pouvant leur causer de séirueses blessures. 

\textbf{Effets Suggérés en Jeu:} A chaque fois qu'un poulpe élevé souffrant de ce désordres est placé dans une situation stressante, il doit réussir un Test de VOL $\times$ 3 ou commencer à manger l'un de ses membres. 

\subsubsection{Bipolarité} 

La bipolarité est également connus sou le nom de maladie maniaco-dépressive. Elle est similaire à la dépression excepté que les périodes de dpéression sont interrompues par de brèves (quelques jours au plus) périodes maniaque pendant lesquelles le sujet se sentira inexplicablement "en forme" avec une énergie renforcée et généralement peu de considérations pour les conséquences. Les phases dépressives sont simialires par tous leurs aspect à la dépression. Les phases maniaques sont dangereuse puisque le sujet prendra des risques, dépensera sans faire attention, et se lancera de manière globale dans un comportement sans réellement réfléchir aux implication ou aux conséquence à long-terme. 

\textbf{Effets Suggérés en Jeu:} Similaire à la dépression, sauf lorsque le personnage est en phase maniaque, auquel cas il doit réussir un Test de VOL $\times$ 3 pour ne pas faire certaines actions qui pourraient être dangereuse. Ils essaieront également de convaincre les autres de les suivre. 

\subsubsection{Dysmorphie corporelle} 

Les sujets affectés par ce désordre pensent qu'ils sont tellement hideux qu'ils sont incapable d'interagir avec les autres ou d'agir normalement par peur du ridicule et de l'humiliation en raison de leur apparence. Ils tendent à être discret et répugnent à chercher de l'aide car ils sont terrifiés par ce que les autres penseront d'eux - à moins qu'ils ne se sentent trop embarassés pour le faire. Ironiquement, ce désordre est souvent perçu comme une obsession égoîstes, alors que c'est plutôt l'inverse; les personnes souffrant de dysmoprhie corporelle se perçoivent comme étant horrible ou défectueux. Un désordre similaire, désordre d'identité sexuelle, dans lequel un patient est en désaccord avec sa biologie sexuelle, précède souvent la dysmorphie corporelle. Le désordre de l'identité sexuelle est généralement dirigées contre les caractéristiques sexuelles dysmorphiques externe, qui sont en conlfit constant avec le genre psychiatrique interne du patient. 

\textbf{Effets Suggérés en Jeu:} En raison de la nature d'Eclipse Phase et de la possibilité de changer et de modifier un corps, c'est un désorbre relativement commun. Il est suggéré que les personnage avec ce désordre souffrent de pénalités de réincarnation prolongée ou augmentée puisqu'ils sont incapable de s'adapté complètement à la réalité de leur nouvelle morph. 

\subsubsection{Personnalité borderline} 

Ce désordre est amrqué par une ioncapacité globale a se supporter plus longtemps. Les états émotionnels sont variable est souvent extrêmes et impressionant. En bref, le sujet pense qu'il perd son sens de soir et cherche à être constamment rassuré par les autres, tout en étant incapable d'agir de manière appropriée. Ils peuvent également se lancer dans des copmportemenst impulsif dans une tentative de faire l'expérience de sentiments. Dans des cas extrêmes, ils peuvent avoir des pensées suicidaire ou tenter de mettre fin à leur jour. 

\textbf{Effets Suggérés en Jeu:} le personnage a besoin d'être avec d'autre personne et n'aime pas être laissé seul, il n'est cependant pas capable de se comproter avec eux de manière normale et prendra également des risques ou fera des décisions impulsives. 

\subsubsection{Dépression} 

La dépression clinique est caractérisée par un sentîment intense d'absence d'espoir et de but. Les malades font référence à leur sentiment comme si rien de ce qu'ils font n'a d'importance et que personne ne s'y intéresse de toute manière, et ils ne sont donc pas enclin a tenter quoi que ce soit. Le personnage est déprimé et trouve difficile d'être motivé pour faire quoi que ce soit. Même des actes simpels comme manger ou prendre un bain peuvent paraître être des tâches monumentales. 

\textbf{Effets Suggérés en Jeu:} Les dépressifs manquent souvent de volonté pour faire n'importe quelle action, souvent au point de nécessiter un Test de VOL $\times$ pour se lancer dans une activité continue. 

\subsubsection{Fugue} 

Le personnage entre dans un état de fugue dans lequel ils n'affichent que peu d'attention envers les stimulis extérieur. Il fonctionne toujours d'un point de vue physiologique mais se retiens de parler et reste à fixer le vague, incapable de se concentrer sur les évènement autour de lui. Contrairement à la catatonie, une personne en état de fugue peut marcher si un assistant le dirige, mais il a tendance à ne pas réagir d'une quelconque manière que ce soit. L'état de fugue est habituellemnt un état persistant, mais il peut s'agir d'un état occasionel déclenché par une sorte de stimuli externe similaire au trauma original qui a déclenché le désordre. 

\textbf{Effets Suggéré en Jeu:} Les personnages en état de fugue ne répondent pas à la plupart des stimuli les entourant. Ils ne se défendront pas d'eux-même si ils sont attaqués et tenteront généralement de se recroqueviller en position phœtale si ils sont agressés physiquement. 

\subsubsection{Anxiété généralisée} 

L'anxiété généralisée est un fort sentiment d'anxiété à propos de tout ce avec quoi le personnage entre en contact. Même de simple tâches constituent la base d'un échec potentiel à une échelle catastrophique et doivent être évitées ou minimisées. De plus, les issues négatives de n'importe quelle action sont toujours considérées comme étant la seule issue possible. 

\textbf{Effets Suggérés en Jeu:} Un personnage souffrant d'anxiété généralisé sera presque totalement inutile à moins de le convaincre du contraire, et ne sera alors effiace que pendant une courte période de temps. Un autre personnage peut tenter d'utiliser une compétence sociale adéquate pour amadouer le personnage souffrant d'anxiété de faire ce qui est nécessaire. Si le personnage souffrant du désordre finit par rater la tâche, toute tentative future d'amadouer le personnage souffrira d'une pénalité de -10 cumulative. 

\subsubsection{Hypochondrie} 

Les hypochondriaques délirent à propos de maladies qu'ils pensent avoir et dont ils ne souffrent pas. Il créeront les troubles et maladies dont ils pensent souffrir, généralement pour attirer l'attention des autres. Les hypocondriaque peuvent également s'infliger des belssurs ou ingérer des substances qui produiront des symptomes similaires aux désordres qu'ils sont persuads avoir. Ces tentatives pour simuler les symptomes peuvent blesser et blesseront réellement les hypocondriaques. 

\textbf{Effets Suggérés en Jeu:} Un sujet hypochondriaque se comportera souvent comme si ils sont affectés par un autre désordre ou une maladie physique. Ce trouble simulé peut être constant ou changer perpétuellement. Les sujets réagiront de manière hostile vis à vis de ceux qui prétendent qu'il simulent ou ne sont pas réellement malades. 

\subsubsection{Désordre du contrôle des pulsions} 

Les sujets ont une croyance qu'ils doivent accomplir une activité particulière lorsqu'elle leur vient en tête. Il peut s'agir de kleptomanie, de pyromanie, d'exhibitionnisme, etc. Ils développent une anxiété grandissante quand ils sont empéchés de se laisser aller à ce comportement pendant des périodes de temps prolongées (habituellement plusieurs fois par jour jusqu'à une fois par semaine, en fonction de la pulsion). C'est différent des TOC car les TOC sont habituellement un seul comportement défini qui foit être fait pour réduire l'anxiété. Le trouble de contrôle des pulsions regroupe une variété de comportement et peut virtuellement être tout type d'action fortement inappropriée. 

\textbf{Effets Suggérés en Jeu:} Comme pour les TOC, si le personnage ne peut se livrer au comportement qu'ils désirent, ils cont devenir extrêmement dérangés et souffiront de pénalités à toutes leurs actions jusqu'à ce qu'isl soient capable de se laisser aller à la compulsion qui alimente leur anxiété. 

\subsubsection{Insomnie} 

le sinsomniaques se trouvent régulièrement incapable de dormir, ou incapable de dormir sur de longues périodes de temps. Cela est souvent du à de l'anxiété à propos de leur vie ou comme une conséquence de la dépression et des schémas de pensée négative associés. Ce n'est pas le type de manque de sommeil qui est la conséquence du stress normal mais plutôt une incapacité presque totale à trouver le repos dans le sommeil lorsque désiré. Les insomniaques peuvent se trouver en train de piquer du nez lors de situations inopportunes, mais jamais pour longtemps et n'en ont jamais suffisament pour profiter d'un sommeil récupérateur. En conséquence, ils sont fréquemment létahrgique et inattentif car leur manque de sommeil leur vole leur acuité et éventuellement tout semblant d'alerte. De plus, les insomniaques sont fréquemment irritables car ils sont sur les nerfs et incapable de se reposer. 

\textbf{Effets Suggérés en Jeu: }En raison du manque de sommeil significatif, les insomniaques souffrent de grosse pénalités aux tâches liées à la perception ou à tout ce qui nécessite de la concentration ou un contrôle moteur précis sur le long terme. 

\subsubsection{Mégalomanie} 

Un mégalomane est persuadé d'être la personne la plus importante de l'univers. Rien n'est plus important que le mégalomane et tout ce qui se passe autour d'eux doit être fait en accord à leur volonté. Ne pas réussir à satisfaire les diktats d'un mégalomane peut souvent amener à des accès de colère ou a des agressions physiques de la part du sujet. 

\textbf{Effets Suggérés en Jeu:} Un personnage mégalomane demandera l'attention des autres et rencontrera beaucoup de difficulté dans quasiment toutes les situations sociales. De plus, il peut devenir violent si il pense qu'il a été négligé. 

\subsubsection{Personnalité multiples} 

Ce désordre est le dévelopement d'une personnalité séparée, distincte de la personnalité d'origine ou de contrôle. Les personnalités peuvent ou peuvent ne pas être consciente l'une de l'autre et "consciente" lors des actions de l'autre personnalité. Il y a habituellement une sorte de déclencheur qui provoque l'émergence de la personnalité secondaire. La plupart des sujets n'auront qu'une seule personnalité supplémentaire, mais il existe des cas de patients ayant de plus de deux personnalité. Il est important de noter que ces personnalités sont des personnaliutés distinctes et non pas de grossières caricatures de type Dr. Jekyll/Mr. Hyde. Chaque personnalité se perçoit comme une personne distincte avec ses propres envies, besoin et motivations. Additionnellement, elles sont habituellement inconsciente des expériences des autres, même si il y a une sorte de partage d'information basique (telle que les langues et les compétences principales). 

\textbf{Effets Suggérés en Jeu:} Lorsque le joueur est sous les effets d'une autre personnalité, il doit être considéré comme un PNJ. Dans de rares cas, le joueur et le maître de jeu peuvent s'accorder sur la seconde personnalité et permettre au joueru de l'interpréter. Cela ne constitue cependant pas un nouveau personnage qui peut êþre "activé" à volonté. 

\subsubsection{Trouble obsessionel compulsif (TOC)} 

les sujets souffrant de TOC sont marqués par des pensées ou des pulsions intrusive ou inappropriées qui peut déclencher un pic d'anxiété si une obsession ou une coimpulsion particulière ne peut être accomplie pour les soulager. Ces obsessions et compulsions peuvent être à peu près n'importe quel type de comportement qui doit être immédiatement assouci pour garder le pic d'anxiété à un seuil tolérable. Les joueurs et le maître de jeu sont encouragé a développer une manie adéquate. Des exemples de manie courantes incluent des tics répétitifs (toucher une partie de son corps avec chaque doigt de chaque main, taper du pied 20 fois), des comportements pathologiques tels que le jeu ou le fait de manger, ou un rituel mental qui doit être réalisé (réciter des passages d'un livre). 

\textbf{Effets Suggérés en Jeu:} Si le personnage ne succombe pas à ses pulsions il sera de plus en plus dérangés et subira des pénalités à toutes leurs actions en jeu jusqu'à ce qu'il soit capable d'assouvir la pulsion qui évacue son anxiété. 

\subsubsection{Stress Post traumatique (SPT)} 

Le SPT est une conséquence de l'exposition soit ç un incident isolé ou une érie d'incidents lors desquels la vie de la victime, ou la vie d'autres personnes, a été menacée. Ces incidents sont souvent marqués par une incapacité de la victime, réelle ou perçue, de faire quoi que ce soir pour modifier l'issue de la situation. Il développe donc une anxiété aigüe et une fixation sur ces incidents au point où il ne peut plus dormir, s'irrite et devient facilement énervé, ou est déprimé par le sentiment de perte de contrôle de sa propre vie. 

\textbf{Effet Suggérés en Jeu:} Des pénalités aux actions de tâches, traitez également tout épisode similaire au trauma initial comme une phobie. 

\subsubsection{Schizophrénie} 

Même si la schyzophrénie est générallement perçue comme un désordre génétique qui a se révèle à l'adolescence, il semble se développer dans un nombre croissant d'égo qui vivent de fréquent changement de morph. Il a été théorisé que ce serait lié à une sorte de répétitions d'erreur dans le processuss de téléchargement. Cela reste un désordre rare, en dépit du risque persistent de mourir et d'être ramené à la vie. La schyzophrénie est un désordre psychotique où le sujet perd sa capacité a discerner le réel et l'irréel. Elle peut impliquer le délire, des hallucinations (souvent en soutien du délire), et une fragementation ou désorganisation verbale. Le sujet n'a pas conscience de ce comportement et se perçoivent comme fonctionnement normalement, souvent jusqu'au pont de devenir paranoïaque vis à vis des autres, eprsuadé qu'ils font parti d'un vaste mensonge. 

\textbf{Effets Suggérés en Jeu:} La schizophrénie représente une rupture totale avec la réalité. Un personnage qui est schizophrène peut voir et entendre des chose et agir en fonction de ces délire et hallucinations toute en voyant les tentatives de ses amis pour arréter ou expliquer le délire comme faisant part d'une conspiration à grande échelle. Des diffcultés à communiquer de manière cohérente s'ajoutent à ça. les personnages devenus schizophréniue sont seulement partiellement fonctionnel et seulement pour de courtes périodes de temps jusqu'à ce que le désordre ait été traité. 

\subsection{Situations stressantes} \label{sec:stressful-situations} 

L'univers de \emph{Eclipse Phase } est fait d'expériences qui peuvent agresser la santé mentale d'un personnage. Certaines sont banale et humaine telle que la violence extrême, l'isolation étendue ou l'impuissance. D'autres sont moins commune, mais encore plus terrifiante: rencontrer des espèces étrangères, se faire infecter par le virus Exsurgent ou être incarné dans une morph non humaine. 

\subsection{Test de stress} \label{sec:willpower-stress-tests} 

A chaque fois qu'un personnage rencontre une situation qui pourrait impacter la psyché de leur ego, le maîþre de jeu peut demander un Test de (Volonté $\times$ 3). Ce test détermine si le personnage est capable de gérer la situation stressante ou si elle défigure leur paysage mental. Si le test est une réussite, le personnage est secoué mais reste non affecté. Si le test est un échec, ils subissent des dégats de stress (et peut-être des traumas) de manière approprié à la sutuation. Une liste de scnéarios générateur de stress et les VS suggérées sont listés sur al table des Expériences Stressantes, p. 215. Le maîþre de jeu devrait les utiliser comme des guides, les modifiant de manière approprié à la situation en cours. 

Notez que crtains incidents peuvent être suffisament horrifique qu'un modificateur est appliqué au Test de (Volonté $\times$ 3) du personnage. 

\begin{table} \begin{tabularx}{\textwidth}{|X|l|} \hline

\multicolumn{2}{|c|}{\textbf{Expériences Stressantes} } \\ \hline

\textbf{Situation}	&\textbf{VS} \\ \hline

Rater de manière spectaculaire en cherchant à atteindre un objectif motivationnel	&1d10 $\div$ 2 (arrondissez à l'inférieur)	\\ \hline

Impuissance	&1d10 $\div$ 2 (arrondissez à l'inférieur)	\\ \hline

Tahison d'un ami de confiance	&1d10 $\div$ 2 (arrondissez à l'inférieur)	\\ \hline

Isolation prolongée	&1d10 $\div$ 2 (arrondissez à l'inférieur)	\\ \hline

Violence extrême (spectateur)	&1d10 $\div$ 2 (arrondissez à l'inférieur)	\\ \hline

Violence extrême (participant)	&1d10	\\ \hline

Conscience que la mort est proche	&1d10	\\ \hline

Vivre la mort de quelqu'un via l'XP	&1d10	\\ \hline

Perdre l'être aimé	&1d10 $\div$ 2 (arrondissez à l'inférieur)	\\ \hline

Être témoin de la mort de l'être aimé	&1d10 + 2	\\ \hline

Être responsable de la mort de l'être aimé	&1d10 + 5	\\ \hline

Arriver sur une scène de meurtre particulièrment violente	&1d10	\\ \hline

Torture (témoin)	&1d10 + 2	\\ \hline

Torture (douleur modérée)	&2d10 + 3	\\ \hline

Torture (douleur critique)	&3d10 + 5	\\ \hline

Rencontres des aliens (non-conscient)	&1d10 $\div$ 2 (arrondissez à l'inférieur)	\\ \hline

Rencontres des aliens (conscient)	&1d10	\\ \hline

Rencontres des aliens hostiles	&1d10 + 3	\\ \hline

Rencontres une technologie trés en avance	&1d10 $\div$ 2 (arrondissez à l'inférieur)	\\ \hline

Rencontres une technologie modifiée par le virus Exsurgent	&1d10 $\div$ 2 (arrondissez à l'inférieur)	\\ \hline

Rencontrer des transhumains infectés par le virus Exsurgent	&1d10	\\ \hline

Rencontrer des formes de vie exsurgentes	&1d10 + 3	\\ \hline

Se faire infecter par le virus Exsurgent	&Variable; voir p. 366	\\ \hline

Être témoin d'exploit epsilon-psi	&1d10 + 2	\\ \hline

\end{tabularx} \label{tab:stressful-experiences} \end{table} 

\subsection{Endurcissement} \label{sec:hardening} 

Plsu vous êtes exposé à des choses horrible et terrifiante, moins elles deviennent effrayantes. Après une exposition répétée, vous devenez endurci à de telles choses, capable de les supporter sans effet. 

Chaque fois que vous réussissez un Test de Volonté pour éviter de subir du stress d'une source particulière, prenez en note. Si vous réussissez à résister à une telle situation 5 fois, vous devenez effectivement immunisé au stress généré par cette source. 

L'inconvénient de l'endurcissement et que vous devenez plus rude et détaché. Afin de vous protéger, vous avez appri à couper vos émotions - mais ce sont ces émotions qui vous rendent humain. Vous avez érigé des barrières mentales efficaces qui affecterons votre empathie et votre capacité à vous référer aux autres. 

À chaque fois que vous vous endurcissez à une source de stress, le maximum de votre stat moxie est réduit de 1. La psychothérapie peut être utilisé pour vaincre un tel endurcissement, de la même manière que l'on traite les désordres. 

\subsection{Soins mentaux et psychothérapie} \label{sec:mental-healing-psychotherapy} 

le stress est plus complexe a traiter que les dégâts physique. I n'y a pas de nano-traitement ou d'option de réparation rapide (autres que le suicide pour retourner à un backup non-stressé). Les options de récupérations sont réduite à la guérison naturelle avec le temps, à la psychothérapie et à la psychochirurgie. 

\subsubsection{Attention psychothérapique} 

Les personanges avec une compétence appropriée - Médecine: Psychiatrie, Académique: Psychologie ou Profession: Psychothérapeute - peuvent aider un personnage souffrant de stress mental ou d'un trauma avec la psychothérapie. Ce traitement et un procédé à long-terme, impliquant des méthodes telles que la psychoanalyse, le conseil, le jeu de rôle, la construction relationnelle, l'hypnothérapie, les modifications comportementales, les drogues, les traitements médicaux et même la psychochirurgie (p. 229). Des IA compétentes en psychothérapie sont également disponibles. 

la Psychothérapie est une action de tâche avec un intervalle de 1 heure par point de stress, 8 heures par trauma et 40 heures par désordre. Notez que cela ne prend en compte que le temps effectivement passé en psychothérapie avec un proffessionel compétent. Après chaque session de psychothérapie, faites un test pour voir si la session est une réussite. Une psychochirurgie réussi ajoute un modificateur de +30 à ce test; à la discrétion du maître de jeu, d'autres modificateurs peuvent s'appliquer. Chaque désordre du personnage ajoute également un modificateur de -10. Les traumas ne peuvent être soignés tant que tout le stress n'est pas éliminé. 

Lorsqu'un trauma est soigné, le dérangement associé est éliminé ou diminué. Les désordres sont traités séparément du trauma qui les ont causés, et ne peuvent être traités que lorsque tous les autres traumas ont été supprimés. 

Le maître de jeu et les joueurs sont encouragé à interpréter la souffrance d'un personnage et la rémission des traumas et des désordres. A chaque fois, il s'agît d'une expérience qui aura un impact fort sur la personnalité et la psyché d'un personnage. Le process de soin peut également changer un personnage, il pourrait donc devenir une personnae différente de ce qu'il était. Même si le personnage est soigné, les cicatrices resteront probablement pendant un certain temps. D'après certaines opinion, les désordres ne sont jamais complètement éradiqués, ils sont juste maintenus sous contrôle ... et ils peuvent donc reposer juste sous la surface, attendant qu'un trauma vienne les réveiller. 

\subsubsection{Guérison naturelle} 

Les personnages qui cherchent à éviter la psychothérapie peut travailler le problème par eux-même, sur leur temps libre. À chaque mois qui passe sans causer de nouveau stress, le personnage doit faire un test de Vol $\times$ 3. Si il réussit, il soigne 1d10 point de stress ou 1 truam (tout le stress ayant du être soigné d'abord). Les désordres sont encore plus dur à soigner, nécessitant 3 mois sans subir de stress ou de trauma, et ne pouvant être ensuite éliminé qu'en réussissant un Test de VOL. De fait, les désordres peuvent trainer pendant des années avant d'être soignés par psychothérapie. 



