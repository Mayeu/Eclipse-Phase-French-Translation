%%% EDITION: 2nd Printing

\chapter{Action and combat}
\label{chap:action-combat}

Roleplaying games are about creating drama and adventure, and that usually means action and combat. Action and combat scenes are the moments when the adrenaline really gets pumping and the characters’ lives and missions are on the line.

Combat and action scenarios can be confusing to run, especially if the gamemaster also needs to keep track of the actions of numerous NPCs. For these reasons, it’s important for the gamemaster to detail the action in a way that everyone can visualize, whether that means using a map and miniatures, software, a dry-erase board, or quick sketches on a piece of paper. Though many of the rules for handling action and combat are abstract - allowing room for interpretation and fudging results to fit the story - many tactical factors are also incorporated, so even small details can make a large difference. It also helps to have the capabilities of NPCs predetermined and to run them as a group when possible, to reduce the gamemaster’s burden in the middle of a hectic situation.


\section{Action turns}
\label{sec:combat-action-turns}

Action scenes in Eclipse Phase are handled in bitesize chunks called Action Turns, each approximately 3 seconds in length. We say ``approximately'' because the methodical, step-by-step system used to resolve actions does not necessarily always translate realistically to real life, where people often pause, take breaks to assess the situation, take a breather, and so on. A combat that begins and ends within 5 Action Turns (15 seconds) in Eclipse Phase could last half a minute to several minutes in real life. On the other hand, the characters may be in a situation where their breathing environment decompresses to vacuum in 15 seconds, so every second may in fact count. As a rule, gamemasters should stick with 3 seconds per turn, but they shouldn’t be afraid to fudge the timing either when a situation calls for it.

Action Turns are meant to be utilized for combat and other situations where timing and the order in which people act is important. If it is not necessary to keep track of who’s doing what so minutely, you can drop out of Action Turns and return to ``regular'' free form game time.

Each Action Turn is in turn broken down into distinct stages, described below


\subsection{Step 1: Roll initiative}
\label{sec:roll-initiative}

At the beginning of every Action Turn, each PLAYER involved in the scene rolls Initiative to determine the order in which each character acts. For more details, see \emph{Initiative}.


\subsection{Step 2: Begin first action phase}
\label{sec:begin-first-phase}

Once Initiative is rolled, the first Action Phase begins. Everyone gets to act in the first Action Phase (since everyone has a minimum Speed of 1), unless they happen to be unconscious/dead/disabled, starting with the character with the highest successful Initiative roll.


\subsection{Step 3: Declare and resolve actions}
\label{sec:declare-resolve}

The character going first now declares and resolves the actions they will take during this first Action Phase. Since some actions the character makes may depend on the outcome of others, there is no need to declare them all first - they may be announced and handled one at a time.

As described under Actions (p. 189), each character may perform a varying number of Quick Actions and/or a single Complex Action during their turn. Alternately, a character may begin or continue with a Task Action, or delay their action pending other developments (see Delayed Actions, p. 189).

A character who has delayed their action may interrupt another character at any point during this stage. That interrupting character must complete this stage in full, then the action returns to the interrupted character to finish the rest of their stage.


\subsection{Step 4: Rotate and repeat}
\label{sec:rotate-repeat}

Once the character has resolved their actions for that phase, the next character in the Initiative order gets to go, running through Step 3 for themselves.

If every character has completed their actions for that phase, return to Step 2 and go the second Action Phase. Every character with a Speed of 2 or more gets to go through Step 3 again, in the same Initiative order (modified by wound modifiers). Once the second Action Phase is completed, return to Step 2 for the 3rd Action Phase, where every character with a Speed of 3 or more gets to go for a third time. Finally, after everyone eligible to go in the 3rd Action Phase has gone, return to Step 2 for a fourth and last Action Phase, where every character with a Speed of 4 can act for one final time.

At the end of the fourth Action Phase, return to Step 1 and roll Initiative again for the next Action Turn.


\section{Initiative}
\label{sec:initiative}

Timing in an Action Turn can be critical - it may mean life or death for a character who needs to get behind cover before an opponent draws and fires their gun. The process of rolling Initiative determines if a character acts before or after another character.


\subsection{Initiative order}
\label{sec:initiative-order}

A character’s Initiative stat is equal to their Intuition + Reflexes aptitudes multiplied by 2. This score may be further modified by morph type, implants, drugs, psi, or wounds.

In the first step of each Action Turn, every character makes an Initiative Test, rolling d100 and adding their Initiative stat. Whoever rolls highest goes first, followed by the other characters in descending order, highest to lowest. In the event of a tie, characters go simultaneously.

\begin{quotation}
Adam, Bob, and Cami are rolling Initiative. Adam’s Initiative stat is 80, Bob’s is 110, and Cami’s is 60. Adam rolls a 38, Bob rolls a 24, and Cami rolls a 76. Adam’s total Initiative score is 118 (80 + 38), Bob’s is 134 (110 + 24), and Cami’s is 136 (60 + 76). Cami rolled highest, so she goes first, followed by Bob and then Adam. If Cami \& Bob had tied, they would both go at the same time.
\end{quotation}

\subsubsection{Initiative and damage} Characters who are suffering from wounds have their Initiative score temporarily reduced (see Wounds, p. 207). This modifier is applied immediately when the wound is taken, which means that it may modify an Initiative score in the middle of an Action Turn. If a character is wounded before they go in that Action Phase, their Initiative is reduced accordingly, which may mean they now go after someone they were previously ahead of in the Initiative order.

\begin{quotation}
Before Bob’s Action Phase comes up, Bob takes two wounds, knocking his Initiative down from 134 to 114. This means that Adam, with an Initiative of 118, now goes before him.
\end{quotation}

\subsubsection{Initiative, Moxie and Criticals} A character may spend a point of Moxie to go first in an Action Phase, regardless of their Initiative roll (see Moxie, p. 122). If more than one character chooses this option, then order is determined as normal first among those who spent Moxie, followed by those who didn’t.

Similarly, any character that rolls a critical on Initiative automatically goes first, even before someone who spent Moxie. If more that two characters rolled criticals, determine order between them as normal.

\subsection{Simplifying initiative}
\label{sec:simplifying-init}

For speedier resolution, simply have characters roll Initiative once for an entire scene. That Initiative result stays with them on each Action Turn until the combat or scenario is over. Likewise, ignore Initiative modifiers from wounds.

\subsection{Speed}
\label{sec:speed}

Speed determines how many times a character can act during an Action turn. Every character starts with a default Speed stat of 1, meaning they can act in the first Action Phase of the turn only. Certain morphs, implants, drugs, psi, and other factors may cumulatively increase their Speed to 2, 3, or even 4 (the maximum), allowing them to act in further Action Phases as well. For example, a character with Speed 2 can act in the first and second Action Phases, and a character with Speed 3 can act in the first through third Action Phases. A character with Speed 4 is able to act in every Action Phase. This represents the character’s enhanced reflexes and neurology, allowing them to think and act much faster than nonenhanced characters.

If a character’s Speed does not allow them to act during an Action Phase, they can initiate no actions during the pass - they must simply bide their time. The character may still defend themself, however, and any automatic actions remain ``on.'' Note that any movement the character initiated is considered to still be underway even during the Action Phases they do not participate in (see \emph{Movement}, p. 190).


\subsection{Delayed actions}
\label{sec:delayed-actions}

When it’s your turn to go during an Action Phase, you may decide that you’re not ready to act yet. You may be awaiting the outcome of another character’s actions, hoping to interrupt someone else’s action, or may simply be undecided about what to do yet. In this case, you may opt to delay your action.

When you delay your action, you’re putting yourself on standby. At some later point in that Action Phase, you can announce that you are now taking your action - even if you interrupt another character’s action. In this case, all other activity is put on hold until your action is resolved. Once your action has taken place, the Initiative order continues on where you interrupted.

You may delay your action into the next Action Phase, or even the next Action Turn, but if you do not take it by the time your next action comes around in the Initiative order, then you lose it. Additionally, if you do delay your action into another phase or turn, then once you take it you lose any action you might have in that Action Phase.


\section{Actions}
\label{sec:actions}

When it’s your turn to act during an Action Phase, you have many options for what you can do - far too many to list here. There is a limit to what you can accomplish in 3 seconds, however, so some limitations must be adhered to. The first step is to figure out what type of action you want to take. In Eclipse Phase, actions are categorized as Automatic, Quick, Complex, or Task, based on how much time and effort they entail.


\subsection{Automatic actions}
\label{sec:combat-automatic-actions}

Automatic Actions require no effort. These are abilities or activities that are ``always on'' (assuming you are conscious) or are otherwise reflexive (they happen automatically in response to certain conditions, with no effort from you). Breathing, for example, is an automatic action - your body does it without conscious effort or thinking on your part.

In most cases, Automatic Actions are not something that you initiate - they are always active, or at least on standby. Certain circumstances, however, will bring an Automatic Action to bear. Such Automatic Actions are invoked and handled immediately whenever they apply, without requiring effort from your character.

\subsubsection{Resistance} Resisting damage - whether from combat, a poison, or a psi attack - is one example of an Automatic Action that occurs in response to something else.

\subsubsection{Basic perception} Your senses are continuously active, accumulating data on the world around you. Basic perception is considered an Automatic Action, and so the gamemaster can call on you to make a Perception Test whenever you receive sensory input that your brain might want to take notice of (see Perception, p. 182). Likewise, you may ask the gamemaster at any time - even during other character’s actions - to make a basic Perception Test, just to find out what your character is noticing around them.

Because basic perception is an automatic, subconscious activity, however, you will suffer a -20 modifier for distraction - your attention is focused elsewhere. In order to avoid the distraction modifier, you must actively engage in detailed perception or use an oracle implant (p. 308).


\subsection{Quick actions}
\label{sec:combat-quick-actions}

Quick Actions are fast and simple, and they may often be multi-tasked. They require minimal thought and effort. You may undertake multiple Quick Actions on your turn during each Action Phase, limited only by the gamemaster’s judgment. If you are taking nothing but Quick Actions during an Action Phase, you should be allowed a minimum of 3 separate Quick Actions. If you are also engaging in a Complex or Task Action during that same Action Phase, you should be allowed a minimum of 1 Quick Action. Ultimately, the gamemaster decides what activity you can or can’t fit into a single Action Phase.

Some examples of Quick Actions include: talking, switching a safety, activating an implant, standing up, dropping prone, gesturing, drawing/readying a weapon, handling an object, or using a simple object.


\subsubsection{Aiming} Aiming is a special case in that it is a Quick Action but requires a degree of concentration that rules out other minor actions. If you wish to aim before making an attack in the same Action Phase, aiming is the only Quick Action you may make during that Action Phase (see Aimed Shots, p. 193).

\subsubsection{Detailed perception} Detailed perception involves taking a moment to actively use your senses in search of information and assess what you are perceiving (see Perception, p. 182). It requires slightly more effort and brainpower (or computer power) than basic perception, which is automatic. As a Quick Action, you may only engage in detailed perception on your turn during an Action Phase, but you do not suffer a modifier for distraction (unless you happen to be in a heavily distracting environment, such as a gunfight or agitated crowd).


\subsection{Complex actions}
\label{sec:combat-complex-actions}

Complex Actions require more concentration and effort than Quick Actions - they effectively monopolize your attention. You may only take one Complex Action on each your Action Phase turns. Additionally, you may not engage in a Complex Action and a Task Action during the same Action Phase.

Examples of Complex Actions include: attacking, shooting, acrobatics, full defense, disarming a bomb, using a complex device, or reloading a weapon.


\subsection{Task actions}
\label{sec:combat-task-actions}

A Task Action is any activity that requires longer than one Action Turn to complete. Each Task Action lists a timeframe for how long the task takes to accomplish. This timeframe may range anywhere from 2 Action Turns to 2 years. While engaged in a Task Action, you may not also undertake a Complex Action, though in some cases you may take a break from the task and return to it later. For more information, see Task Actions, p. 120.

Examples of Task Actions include: repairing a device, programming, conducting a scientific analysis, searching a room, climbing a wall, or cooking a meal.


\section{Movement}
\label{sec:combat-movement}

Movement in \emph{Eclipse Phase} is handled just like any other action, and may change from Action Phase to Action Phase. Walking and running both count as Quick Actions, as they do not require your full concentration. The same also applies to slithering, crawling, floating, hovering, or gliding. Running, however, may inflict a -10 modifier on other actions that are affected by your jostling movement. Even more, sprinting is an all-out run, and so requires a Complex Action (see Sprinting, p. 191).

At the gamemasters discretion, other movement may also call for a Complex Action. Hurdling a fence, pole vaulting, jumping from a height, swimming, or freerunning through a habitat in zero-gravity all require a bit of finesse and attention to detail, so would count as a Complex Action, and would apply the same modifier as running. Flying generally counts as a Quick Action, though intricate maneuvers would call for a Complex Action.


\subsection{Movement rates}
\label{sec:movement-rates}

Sometimes it’s important to know not just how you’re moving, but how far. For most of transhumanity, this movement rate is the same: 4 meters per Action Turn walking, 20 meters per turn running. To determine how far a character can move in a particular Action Phase, divide this movement rate by the total number of Action Phases in that turn. In a turn with 4 Action Phases, that breaks down to 1 meter walking per Action Phase, 5 meters running.

Movement such as swimming or crawling benchmarks at about 1 meter per Action Turn, or 0.25 meters per Action Phase. You can also sprint to increase your movement rate (see Sprinting). Vehicles, robots, creatures, and unusual morphs will have individual movement rates listed in the format of walking rate/running rate in meters per turn.

These movement rates assume standard Earth gravity of course. If you’re moving in a low-gravity, microgravity, or high-gravity environment, things change. See \emph{Gravity}, p. 198.


\subsubsection{Jumping} Characters making a running jump can cross SOM $\div$ 5 (round up) meters; use SOM $\div$ 20 (round up) meters for standing jumps. Vertical jumping height is 1 meter. Characters making a Freerunning Test can increase jumping distance by 1 meter (running jump) or 0.25 meters (standing/vertical jumps) per 10 points of MoS.

\subsubsection{Sprinting} You may use Freerunning to increase the distance you move during an Action Phase. You must spend a Complex Action to sprint and make a Freerunning Test. Every 10 points of MoS increases your running distance in that Action Phase by 1 meter, to a maximum bonus of +5 meters.


\section{Combat}
\label{sec:combat}

Sometimes words fail, and that’s when the knives and shredders come out. All combat in Eclipse Phase is conducted using the same basic mechanics, whether it’s conducted with claws, fists, weapons, guns, or psi: an Opposed Test between the attacker and defender(s).

\subsection{Resolving combat}

Use the following sequence of steps to determine the outcome of an attack.

\subsubsection{Step 1: Declare an attack} The attacker initiates by taking a Complex Action to attack on their turn during an Action Phase. The skill employed depends on the method used to attack. If the character lacks the appropriate Combat skill, they must default to the appropriate linked aptitude.

\subsubsection{Step 2: Declare defense} Once the attack is declared, the defender chooses how to respond. Defense is always considered an Automatic Action unless the defender is surprised (see Surprise, p. 204) or somehow incapacitated and incapable of defending themself.

\textbf{Melee:} A character defending against melee attacks uses Fray skill, representing dodging (if the character lacks this skill, they may default to Reflexes). Alternately, the character may use a melee combat skill to defend, representing blocks and parries rather than dodging.

\textbf{Ranged:} Against ranged attacks, a defending character may only use half their Fray skill (round down).

\textbf{Full Defense:} Characters who have taken a Complex Action to go on full defense (p. 198) receive a +30 modifier to their defensive roll.

\textbf{Psi:} A character defending against a psi attack rolls WIL x 2 (p. 222). A mental sort of full defense may also be rallied against psi attacks.

\subsubsection{Step 3: Apply modifiers} Any appropriate modifiers are now applied to the attacker and defender’s skills. See the Combat Modifiers table (p. 193) for common situational modifiers.

\subsubsection{Step 4: Make the opposed text} The attacker and defender both roll d100 and compare the results to their modified skill target numbers.

\subsubsection{Step 5: Determine outcome} If the attacker succeeds and the defender fails, the attack hits. If the attacker fails, the attack misses completely.

If both attacker and defender succeed in their tests, compare their dice rolls. If the attacker’s dice roll is higher, the attack hits despite a spirited defense; otherwise, the attack fails to connect.

\textbf{Excellent Success:} If the attacker rolled an Excellent Success (MoS of 30+), a solid hit is struck. Increase the Damage Value (DV) inflicted by +5. If the MoS is 60+, increase the DV by +10.

\textbf{Criticals:} If the attacker rolls a critical success, the attack is armor-defeating, meaning that the defender’s armor is bypassed completely  -  some kink or flaw was exploited, allowing the attack to get through completely.

If the defender rolls a critical success, they dodge with flair, reach cover that protects from follow-up attacks, maneuver to a superior position, or otherwise benefit.

\subsubsection{Step 6: Modify armor} If the target is hit, their armor will help to protect them against the attack (unless the attacker rolled a critical, see above). Determine which type of armor is appropriate to defending against that particular attack (see Armor, p. 194). The attack’s Armor Penetration (AP) value reduces the armor’s rating, however, representing the weapon’s ability to pierce through protective measures.

\subsubsection{Step 7: Determine damage} Every weapon and type of attack has a Damage Value (DV, see p. 207). This amount is reduced by the target’s AP-modified armor rating. If the damage is reduced to 0 or less, the armor is effective and the attack fails to injure the target. Otherwise, any remaining damage is applied to the defender. If the accumulated damage exceeds the defender’s Durability, they are incapacitated and may die (see Durability and Health, p. 207).

Note that some psi attacks inflict mental stress rather than physical damage (see Mental Health, p. 209). In this case, the Stress Value (SV) is handled the same as DV.

\subsubsection{Step 8: Determine wounds} The damage inflicted from a single attack is then compared to the victim’s Wound Threshold. If the armormodified DV equals or exceeds the Wound Threshold, the character suffers a wound. Multiple wounds may be applied with a single attack if the modified DV is two or more factors beyond the Wound Threshold. Wounds represent more serious injuries and apply modifiers and other effects to the character (see Wounds, p. 207).

\begin{quotation}
Stoya tried to get off the station quickly, but the Night Cartel’s assassin caught up, surprising her in a microgravity part of the habitat. The assassin’s INIT is 63, plus a dice roll of 23, for an Initiative of 86. Stoya’s INIT is 55, plus a roll of 27, for an Initiative of 82.

The assassin goes first, spending a Quick Action to draw a shredder. This flechette weapon is in burst-fire mode, so with a Complex Action the assassin can take two shots. His Spray Weapons skill is 65, he’s smartlinked (+10), and they’re at short range (+0), so he needs a 75 or less. Stoya is defending with her Fray skill (60) divided by 2, or 30.

The assassin rolls an 08 with the first shot. Amazingly, Stoya rolls a 28. They both succeeded, but Stoya rolled higher, so she dodges the first shot.

The assassin rolls a 20 for his second shot, another hit, and this time Stoya rolls an 83, a failure. The assassin also scored an Excellent Success with a MoS of 55, increasing the DV by +5.

The assassin’s base damage is 2d10 + 5, but he’s using burst fire against a single target for +1d10, and it’s also a cone effect weapon at short range, for an additional +1d10, for a total DV of 4d10 + 5. The assassin rolls 4d10 and gets 16, then adds the +5 for a total DV of 21.

Stoya’s wearing light body armor (AV 10/10), but the shredder’s Armor penetration is -10, so her armor is entirely negated. She takes a devastating 21 DV, exceeding her Wound Threshold of 10, not just once, but twice. This means Stoya suffers 2 wounds from the shot, suffering -20 to all actions. In addition, she must make two SOM x 3 Tests, one to avoid knockdown and the other to avoid unconsciousness. Her SOM is 30, meaning she needs a 70 (30 x 3 = 90, 90 - 20 wound modifiers = 70) on both rolls. She rolls a 40 and a 27, succeeding both.

Now it’s Stoya’s action. She takes a Quick Action to pull her own weapon: a stunner. Her Beam Weapons skill is 47, modified by wounds (-20) and a smartlink (+10), for 37. The assassin’s Fray is 48, divided by 2 for 24 against a ranged attack. Stoya rolls a 22 - a critical hit - and the assassin rolls a 68. The stunner only inflicts 1d10 $\div$ 2 DV, but since the attack is a critical hit, this is armor defeating. Stoya rolls an 8, for 4 points of DV, below the assassin’s Wound Threshold of 7.

Stunners, however, are shock weapons, so the assassin must make a DUR + Energy Armor Test. His DUR is 35 and he’s wearing an armor vest (AV 6/6), so his target number is 41. He rolls a 71 - a Margin of Failure of 30, meaning he is immediately incapacitated for 3 Action Turns.

Having disabled her opponent, Stoya takes the time to make a hasty getaway.
\end{quotation}


\subsection{Combat summary}

\begin{quotation}
\textbf{Ego stats}
\begin{itemize}
\item Attacker rolls attack skill +/- modifiers.
\item Melee: Defender rolls Fray or melee skill +/- modifiers.
\item Ranged: Defender rolls (Fray skill $\div$ 2, round down) +/- modifiers.
\item If attacker succeeds and rolls higher than the defender, the attack hits.
\item Critical hits are armor-defeating (armor does not apply).
\item Armor is reduced by the attack’s Armor Penetration value (AP).
\item The weapon’s damage is reduced by the target’s modified Armor rating (unless the attack is armor-defeating).
\item If the damage exceeds the target’s Wound Threshold, a wound is also scored. (If the damage exceeds the Wound Threshold by multiple factors, multiple wounds are inflicted.)
\end{itemize}
\end{quotation}

\begin{tabular}{|l|l|} %%% Note: No idea where to place this table. Open to suggestions
\hline
\multicolumn{2}{|c|}{\textbf{Combat modifiers}} \\
\hline
\textbf{General} & \textbf{Modifier} \\ 
\hline
Character using off-hand & -20 \\
\hline
Character wounded/traumatized & -10 per wound/trauma \\
\hline
Character has superior position & +20 \\
\hline
Touch-only attack & +20 \\
\hline
Called shot & -10 \\
\hline
Character wielding two-handed weapon with one hand & -20 \\
\hline
Small target (child-sized) & -10 \\
\hline
Very small target (mouse or insect) & -30 \\
\hline
Large target (car sized) & +10 \\
\hline
Very large target (side of a barn) & +30 \\
\hline
Visibility impaired (minor: glare, light smoke, dim light) & -10 \\
\hline
Visibility impaired (major: heavy smoke, dark) & -20 \\
\hline
Blind attack & -30 \\
\hline
\textbf{Melee Combat} & \textbf{Modifier} \\
\hline
Character has reach advantage & +10 \\
\hline
Character charging & -10 \\
\hline
Character receiving a charge & +20 \\
\hline
\textbf{Ranged combat (attacker)} & \textbf{Modifier} \\
\hline
Attacker using smartlink or laser sight & +10 \\
\hline
Attacker behind cover & -10 \\
\hline
Attacker running & -20 \\
\hline
Attacker in melee combat & -30 \\
\hline
Defender has minor cover & -10 \\
\hline
Defender has moderate cover & -20 \\
\hline
Defender has major cover & -30 \\
\hline
Defender prone and far (10+ meters) & -10 \\
\hline
Defender hidden & -60 \\
\hline
Aimed shot (quick) & +10 \\
\hline
Aimed shot (complex) & +30 \\
\hline
Sweeping fire with beam weapon & +10 on second shot \\
\hline
Multiple targets in same Action Phase & -20 per additional target \\
\hline
Indirect fire & -30 \\
\hline
Point-blank range (2 meters or less) & +10 \\
\hline
Short range & -- \\
\hline
Medium range & -10 \\
\hline
Long range & -20 \\
\hline
Extreme range & -30 \\ \hline
\end{tabular}


\section{Action and combat complications}
\label{sec:action-combat-comp}

Combat isn’t quite as simple as deciding if you hit or miss. Weapons, armor, ammunition, and numerous other factors may impact an attack’s outcome. Like- wise, various factors can impact an action scene, such as fire or microgravity effects.


\subsection{Aimed shots}
\label{sec:aimed-shots}

As noted under Aiming, p. 190, a character can sacrifice their other Quick Actions to concentrate on targeting a ranged attack and receive a +10 modifier on the attack. You can also sacrifice an entire Complex Action to fix your aim on a target. In this case, as long as the target remains in your sights until your next Action Phase, you receive a +30 modifier to hit.


\subsection{Ammunition and reloading}
\label{sec:ammunition-reloading}

Every weapon has a listed ammunition capacity that indicates how many shots the weapon can carry or holds. When this ammo runs out, a new supply must be loaded in. Players should keep track of the shots they fire.

Reloading almost always requires a Complex Action, whether you are slapping in a new clip of bullets or a fresh battery for a laser. At the gamemaster’s discretion, a reload that is immediately accessible (such as a new clip reverse-taped to the loaded clip, so that reloading just requires that you reverse the taped clips and slot the new one in) will only take a Quick Action. Archaic weapons such as magazine-fed rifles may require longer to fully load.


\subsection{Area effect weapons}
\label{sec:area-effect-weapons}

Some ranged attack weapons are designed to affect more than one target at a time. These weapons fall into three categories: blast, uniform blast, and cone.

\subsubsection{Blast effect}
Blast weapons include items like grenades, mines, and other explosives that expand outward from a central detonation point. Most blast attacks expand outward in a sphere, though certain shaped charges may direct an explosion in one direction only. The explosive force is stronger near the epicenter and weaker near the outer edges of the sphere. For every meter a target is from the center, reduce the damage of a blast weapon by -2.

\subsubsection{Uniform blast}
Uniform blast attacks distribute their power evenly throughout the area of effect. Examples include fuelair explosives and thermobaric weapons that disperse an explosive mixture in a vapor cloud and ignite it all at once. All targets within the noted blast radius suffer the same damage. Damage against targets outside of the main blast sphere is reduced by -2 per meter.

\subsubsection{Cone}
Weapons with a cone effect have an area effect that begins with the tip of the weapon and expands outward in a cone. At short range, this attack effects 1 target; at medium range it affects 2 targets within a meter of each other; and at long or extreme range it affects 3 targets within a meter of the next. Cone effect attacks do +1d10 damage at short range and -1d10 damage at long and extreme range.


\subsection{Armor}
\label{sec:armor}

Just as weapons technologies have advanced, so too has armor quality, allowing unprecedented levels of protection. As noted in Step 7: Determine Damage (see p. 192), the armor rating reduces the damage points of the attack. For a full listing of armor types and values, see p. 311.

\subsubsection{Energy vs. Kinetic}
Each type of armor has an Armor value (AV) with two ratings - Energy and Kinetic - representing the protection it applies against the respective type of attack. These are listed in the format of ``Energy armor/Kinetic armor.'' For example, an item with listed armor ``5/10'' provides 5 points of armor against energy-based attacks and 10 points of armor against kinetic attacks.

Energy damage includes that caused by beam weapons (laser, microwave, particle beam, plasma, etc.) as well as fire and high-energy explosives. Armor that protects against this damage is made of material that reflects or diffuses such energy, dissipates and transfers heat, or ablates.

Kinetic damage is the transfer of damaging energy when an object in motion (a fist, knife, club, or bullet, for example) impacts with another object (the target). Most melee and firearms attacks inflict kinetic damage, as would a rolling boulder, swinging pendulum, or explosion-driven fragments. Kinetic armors include impact-resistant plates, shear-thickening liquid and gels that harden upon impact, and ballistic and cutproof fiber weaves.

\subsubsection{Armor penetration}
Some weapons have an Armor Penetration (AP) rating. This represents the attack’s ability to pierce through protective layers. The AP rating reduces the value of armor used to defend against the attack (see Step 6: Modify Armor, p. 192).

\subsubsection{Layered armor}
If two or more types of armor are worn, the armor ratings are added together. However, wearing multiple armor units is cumbersome and annoying. Apply a -20 modifier to a character’s actions for each additional armor layer worn.

Note that items specifically noted as armor accessories -  helmets, shields, etc. - do not inflict the layered armor penalty, they simply add their armor bonus. Note also that the armor inherent to a synthetic morph or bot’s frame does not constitute a layer of armor (i.e., you may wear armor over the synthetic shell without penalty).


\subsection{Asphyxiation}
\label{sec:asphyxiation}

The average transhuman can hold their breath for two minutes before blacking out. Strenuous activity reduces the amount of time. For every 30 seconds after the first minute a biomorph is prevented from breathing, they must make a DUR Test. Apply a cumulative -10 modifier each time this test is rolled. If the character fails the test, they immediately fall unconscious and begin to suffer damage from asphyxiation, at the rate of 10 points per minute, until they die or are allowed to breathe again. This damage does not cause wounds.

Asphyxiating is a terrible process, often leading to panic. Characters who are being asphyxiated must make a WIL x 3 Test. If they fail, they suffer 1d10 $\div$ 2 (round up) mental stress and cannot perform any effective action to rescue themselves that Action turn. A character who succeeds may attempt to rescue themselves, and in fact they must make a WIL x 3 Test to perform any other action not directly related to rescuing themselves (attacks against another character, a creature, or an object holding the character underwater are exempt from this rule).


\subsection{Beam weapons}
\label{sec:beam-weapons}

Due to emitting a continuous beam of energy rather than single projectiles, beam weapons are easier to ``home in'' on a target. This means one of the following two rules may be used when making beam weapon attacks. These options are not available for ``pulse'' beam weapons (like the laser pulser), as such weapons emit energy in pulses rather than a continuous beam.

\subsubsection{Sweeping fire}
An attacker who is making two semi-auto (p. 198) attacks with a beam weapon with the same Complex Action and who misses with the first attack may treat that attack as a free Aim action (p. 190), receiving a +10 modifier for the second attack. In other words, though the first attack misses, the character takes the opportunity to sweep the beam closer to the target for the second attack. This only applies when both attacks are made against the same target.

\subsubsection{Concentrated fire}
A character firing a semi-auto beam weapon who hits with the first attack may choose to keep the beam on and concentrate their fire, cooking the target. In this case, the character foregos their second semi-auto attack with that Complex Action, but automatically bolsters the DV of the first attack by x 1.5 (round up). This decision must be made before the damage dice are rolled.


\subsection{Blind attacks}
\label{sec:blind-attacks}

Attacking a target that you cannot see is difficult at best and a matter of luck at worst. If you cannot see, you may make a Perception Test using some other available sense to detect your target. If this succeeds, you attack with a -30 modifier. If your Perception Test fails, your attack is primarily based on chance - your target number for the attack test is equal to your Moxie stat (no other modifiers apply).

\subsubsection{Indirect fire}
With the help of a spotter, you may target an enemy that you can’t see using indirect fire. In this case you must be meshed with a character, bot, or sensor system that has the target in its sights and which feeds you targeting data (the gamemaster may require a Perception Test from the spotter). Indirect attacks suffer a -30 modifier.

Seeker missiles (p. 340) can home in on a target that is ``painted'' with reflected energy from a laser sight (p. 342) or similar target designator system. An ``attack'' must first be made to paint the target with the laser sight using an appropriate skill. If this succeeds, it negates the -30 indirect fire modifier for the seeker launcher’s attack test. the target must be held in the spotter’s sights (requiring a Complex Action each Action Phase) until the seeker strikes.


\subsection{Bots, synthmorphs and vehicles}
\label{sec:bots-synthmorphs-vehicles}

AI-operated robots and synthetic morphs are a common sight in Eclipse Phase. Robots are used for a wide range of purposes, from surveillance, maintenance, and service jobs to security and policing -  and so may often play a role in action and combat scenes. Though less common (at least in habitats), AI-piloted vehicles are also frequently used and encountered.

Note that the difference between a robot, vehicle, and synthetic morph is in many ways semantic. Robots are simply synthetic bodies controlled by an AI. Vehicles are also robotic - AI controlled -  but the term ``vehicle'' is used to denote that they carry passengers. Both bots and vehicles can be used as synthetic morphs -  that is, inhabited by a transhuman ego -  assuming they are equipped with a cyberbrain (p. 300). For the purpose of these rules, the term ``shell'' is used to refer to bots, vehicles, and synthetic morphs alike.

Like synthmorphs, bots and vehicles are treated just like any other character: they roll Initiative, take actions, and use skills. A few specific aspects of these shells needs special consideration, however, covered below.

\subsubsection{Shell stats}
Just like synthmorph characters, certain bot and vehicle stats (Durability, Wound Threshold, etc.) and stat modifiers (Initiative, Speed, etc.) are determined by the actual physical shell. Other stats are determined by the bot/vehicle’s operating AI (in place of an ego). Bots and vehicles may also have traits that apply to their AI or physical shell. For sample bots and vehicles, see p. 342 of the Gear chapter.

\textbf{Handling:} Bots and vehicles have a special stat called Handling, which is a modifier applied to all tests made to pilot the bot/vehicle. This represents the bot/ vehicle’s maneuverability.

\subsubsection{Shell skills}
The skills and aptitudes used by a bot or vehicle are those possessed by its AI. See \emph{AIs and Muses}, p. 264.

\subsubsection{Shell movement}
Like characters, bots and vehicles have a walking and running Movement rate. This is used whenever the bot/vehicle is engaged in action or combat scenes with other characters.

Shells that are capable of greater speeds will also have a Maximum Velocity listed - this is the highest rate at which the shell may safely move, listed in kilometers per hour. At the gamemaster’s discretion, some shells may push their limits and accelerate further, but at significant risk -  the gamemaster should apply appropriate penalties to Pilot Tests and other tests.

\subsubsection{Chases}
Shells that are moving faster than their running Movement rate (up to their Max. Velocity) are generally considered to be moving too fast for standard action and combat interaction with other characters. This is when the action enters ``chase scene'' mode - a traveling narrative of maneuvering choices and tests with various outcomes. Whether or not a chase is actually occurring, the gamemaster should remember that Max Velocity is not the only factor in high-speed situations. Environmental factors like terrain, weather conditions, navigation, pedestrians, and traffic can provide obstacles for shells to overcome. A shell tearing across a habitat in order to reach a bomb before it detonates should have to make several decisions and tests that may affect whether it gets there in time or not. Likewise, a shell seeking to shake off some hot pursuit will have to pull off some fancy maneuvering and hopefully find a shortcut or two in order to outrace their opponents.

\subsubsection{Crashing}
Shells that suffer wounds during combat or chases may be force to make a Pilot Test to avoid crashing or may crash automatically. The exact circumstances of a crash are left up to the gamemaster, as best fits the story - the shell may simply skid to a stop, plow into a tree, fall from the sky, or flip over and land on a group of bystanders. Shells that strike other objects when they crash typically take further damage from the collision (see Collisions).

\subsubsection{Collisions}
If a shell crashes into or intentionally rams a person or object, someone is likely to get hurt. To determine how much DV is inflicted, roll 1d10 and add the shell’s DUR divided by 10 (round up). This is the damage applied at walking speeds. If the shell was moving at running speeds, multiply the DV by 2. If the shell was moving at chase speeds, multiply the DV by the shell’s velocity $\div$ 10 in meters per turn. Both the shell and whatever it strikes suffer this damage, assuming the collision is with something equal dense and hard. Soft and squishy objects like biomorphs will be less damaging to a shell (unless they happen to be in a hardsuit or battlesuit), in which case the shell will only suffer half damage from the collision. Kinetic armor defends against crash DV.

If two moving shells collide head-on, calculate the damage from both and inflict to both. If two shells moving in the same direction collide, only count the difference in velocity.

Passengers in a vehicle may also be damaged by collisions if they are not wearing proper safety restraints. They suffer one half the DV applied to their vehicle (less their own Kinetic armor).


\begin{tabular}{|l|l|}
\hline
\multicolumn{2}{|c|}{\textbf{Collision damage}} \\
\hline
Base Collision DV & 1d10 + (DUR $\div$ 10) \\
\hline
Running & DV x 2 \\
\hline
Chase Speeds & DV x (velocity $\div$ 10) \\ \hline
\end{tabular}

\subsubsection{Attacking passenger vehicles}
During combat, passengers within a vehicle may be targeted separately from the vehicle itself. Attacks made against passengers this way do not harm the vehicle itself (unless an area effect weapon is used). Targeted passengers benefit both from cover (usually Major, -30) and from the vehicle’s structure, adding the vehicle’s Armor Value to their own.

Passengers within a vehicle are generally not harmed by attacks made against the vehicle itself. Area effect weapons are an exception to this rule, but in this case the passengers also benefit from the vehicle Armor Value.

\subsubsection{Shell remote control}
Any shell (or biomorph) with a puppet sock (also included with all cyberbrains) may be remote controlled, either by a character or a remote AI. This requires a communications link between the teleoperator and the shell (the ``drone”). The teleoperator controls the drone via an entoptic interface, and receives sensory input and other data via the drone’s mesh inserts.

When under direct control, the shell’s AI (or resident ego) is subsumed and put on standby. The drone only acts as instructed. Each instruction counts as a Quick Action. In this situation, the drone acts with the same Initiative as the teleoperator. The teleoperator’s skills and stats are used in place of the shell AI’s. Due to the nature of remote operation, however, all tests are made with a -10 modifier. Multiple drones may be controlled at once, but commanding them requires separate Quick Actions unless they are receiving the same command. Direct control teleoperation is not very feasible at extreme distances, due to the light-speed lag with communications.

Alternately, the teleoperator can put the drone in autonomous mode, allowing the shell’s AI to resume normal operations. The drone still follows the teleoperator’s commands to the best of its abilities. In this mode, the drone functions normally, using its own Initiative and AI skills and stats.

\subsubsection{Shell jamming}
``Jamming'' is the colloquial term for a more direct form of remote-control, using VR and XP technology. When jamming, the drone’s puppet sock feeds the drone’s sensory data directly to the teleoperator’s mesh inserts. The teleoperator subsumes themself in the drone’s sensorium, essentially ``becoming'' the drone. In this case, the teleoperator surrenders control of their own morph, which slumps inertly. While jamming, they suffer -60 on all Perception Tests or attempts to take action with their morph.

Jamming takes a Complex Action to engage and disengage. A jamming teleoperator controls a drone as if it were their own morph. Like direct control teleoperation, the jammer’s own skills and Initiative are used in place of the drone’s AI. Jammers do not suffer any teleoperation modifiers, but only one drone may be jammed at a time.

If the drone is killed or destroyed, the jammer is immediately dumped from their connection, resuming control of their own morph as normal. Getting dumped in this manner is extremely jarring, not the least because the jammer experienced being killed/destroyed. As a result, the jammer suffers 1d10 mental stress.


\subsection{Called shots}
\label{sec:called-shots}

Sometimes it’s not enough to just hit your target -  you need to shoot out a window, knock the knife out of their hand, or hit that hole in their armor. You may declare that you are making a called shot before you initiate an attack, choosing one of the outcomes noted below. Called shots suffer a -10 modifier and require an Excellent Success (MoS 30+). If you beat that margin, you succeed with the called shot, and the results noted below apply. If you don’t beat the margin but still succeed in the attack, you simply strike your target as normal.

\subsubsection{Bypassing armor}
Called shots may be used to target a hole or weak point in your opponent’s armor. If you beat the MoS, you strike an armor-defeating hit, and their armor does not apply. Note that in certain circumstances, a gamemaster may rule that an opponent’s armor simply doesn’t have a weak spot or unprotected area, and so disallow such called shots.

\subsubsection{Disarming}
You may take a called shot to attempt to knock a weapon out of an opponent’s hand(s). If you beat the MoS, the victim suffers half damage from the attack (reduced by armor as normal) and must make a SOM x 3 Test with a -30 modifier to retain hold of the weapon.

\subsubsection{Specific targeting}
You may make a called shot with the intention of hitting a specific location or component on your target - for example: disabling the sensor unit on a bot, sweeping someone’s leg, or poking someone in the eye. If you beat the MoS, you hit the specific targeted spot. The gamemaster determines the result as appropriate to the attack and target - the component may be destroyed, the opponent may fall or be temporarily blinded, and so on.

\subsection{Charging}

\subsection{Demolitions}

\subsection{Falling}

\subsection{Fire}

\subsection{Firing modes and rate of fire}

\subsection{Full defense}

\subsection{Gravity}

\subsection{Grenades and seekers}

\subsection{Hostile environments}

\subsection{Improvised weapons}

\subsection{Knockdown/knockback}

\subsection{Melee and thrown damage bonus}

\subsection{Multiple targets}

\subsection{Objects and structures}

\subsection{Range}

\subsection{Reach}

\subsection{Scatter}

\subsection{Shock attacks}

\subsection{Subdual}

\subsection{Suppressive fire}

\subsection{Surprise}

\subsection{Tactical networks}

\subsection{Touch-only attack}

\subsection{Two-handed weapons}

\subsection{Weilding two or more weapons}

\section{Physical health}
\label{sec:physical-health}

\subsection{Damage points}

\subsection{Damage types}

\subsection{Durability and health}

\subsection{Damage value}

\subsection{Wounds}

\subsection{Death}

\section{Healing and repair}
\label{sec:healing-repair}

\subsection{Biomorph healing}

\subsection{Synthmorph and Object repair}

\section{Mental health}
\label{sec:mental-health}

\subsection{Stress points}

\subsection{Lucidity and stress}

\subsection{Stress value}

\subsection{Trauma}

\subsection{Derangements}

\subsection{Disorders}

\subsection{Stressful situations}

\subsection{Willpower stress tests}

\subsection{Hardening}

\subsection{Mental healing and psychotherapy}

<<<<<<< HEAD
%%% Local Variables: %%% mode: latex %%% TeX-master: ``ep'' %%% End: 
=======
%%% Local Variables: %%% mode: latex %%% TeX-master: ``ep'' %%% End: 
>>>>>>> 757d43a43af60acb5e92454c8503ed3bf1a3c703
