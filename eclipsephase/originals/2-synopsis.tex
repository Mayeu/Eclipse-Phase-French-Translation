\chapter{Enter the singularity}
\label{chap:enter-the-singularity}

We humans have a special way of pulling ourselves up and kicking ourselves down at the same time. We'd achieved more progress than ever before, at the cost of wrecking our planet and destabilizing our own governments. But things were starting to look up.

With exponentially accelerating technologies, we reached out into the solar system, terraforming worlds and seeding new life. We re-forged our bodies and minds, casting off sickness and death. We achieved immortality through the digitization of our minds, resleeving from one biological or synthetic body to the next at will. We uplifted animals and AIs to be our equals. We acquired the means to build anything we desired from the molecular level up, so that no one need want again.

Yet our race toward extinction was not slowed, and in fact received a machine-assist over the precipice. Billions died as our technologies rapidly bloomed into something beyond control ... further transforming humanity into something else, scattering us throughout the solar system, and reigniting vicious conflicts. Nuclear strikes, biowarfare plagues, nanoswarms, mass uploads ... a thousand horrors nearly wiped humanity from existence.

We still survive, divided into a patchwork of restrictive inner system hypercorp-backed oligarchies and libertarian outer system collectivist habitats, tribal networks, and new experimental societal models. We have spread to the outer reaches of the solar system and even gained footholds in the galaxy beyond. But we are no longer solely ``human'' ... we have evolved into something simultaneously more and different— something transhuman.


\section{Starting out}
\label{sec:starting-out}

Eclipse Phase is a post-apocalyptic roleplaying game of transhuman conspiracy and horror. Humans are enhanced and improved, but humanity is battered and bitterly divided. Technology allows the re-shaping of bodies and minds and liberates us from material needs, but also creates opportunities for oppression and puts the capability for mass destruction in the hands of everyone. Many threats lurk in the devastated habitats of the Fall, dangers both familiar and alien.


\subsection{What is a roleplaying game?}
\label{sec:what-roleplaying}

Have you ever read a book or seen a movie or a television show where a character does something really stupid, like heading into a basement at night when the character knows the serial killer is around? The whole time, you're thinking: ``I wouldn't walk down those creepy stairs to the dark basement, especially without a flashlight. I'd do X, Y, or Z instead!'' Since you're in the passenger's seat for the plot you're reading or watching, however, you simply have to sit back and let it unfold.

What if you could take hold of the driver's seat? What if you could take the plot in the direction you'd choose? That is the essence of a roleplaying game.

A roleplaying game (or RPG, for short) is part improvisational theater, part storytelling, and part game. A single person (the gamemaster) runs the game for a group of players that pretend to be characters in a fictitious world. The world could be a mystery game set in the 1920s that takes you adventuring around the globe, a fantasy realm inhabited by dragons and trolls and sword-wielding barbarians, or a science fiction setting with aliens and spaceship and world-crushing weaponry. The players pick a setting that they find cool and want to play in. The players then craft their own characters, providing a detailed history and personality to bring each to life. These characters have a set of statistics (numerical values) that represent skills, attributes, and other abilities. The gamemaster then explains the situation in which the characters find themselves. The players, through their characters, interact with the storyline and each others' characters, acting out the plot. As the players roleplay through some scenarios, the gamemaster will probably ask a given player to roll some dice and the resulting numbers will determine the success or failure of a character's attempted action. The gamemaster uses the rules of the game to interpret the dice rolls and the outcome of the character's actions.

As a group exercise, the players control the storyline (the adventure), which evolves much like any movie or book but within the flexible plot created by the gamemaster. This gamemaster plot provides a framework and ideas for potential courses of action and outcomes, but it is simply an outline of what might happen—it is not concrete until the players become involved. If you don't want to walk down those stairs, you don't. If you think you can talk yourself out of a situation in place of pulling a gun, then try and make it happen. The script of any roleplaying session is written by the players, and the story, based upon the character's actions and their responses to the events of the plot, will constantly change and evolve.

The best part is that there is no ``right'' or ``wrong'' way to play an RPG. Some games may involve more combat and dice rolling-related situations, where other games may involve more storytelling and improvised dialogue to resolve a situation. Each group of players decides for themselves the type and style of game they enjoy playing!


\subsection{What is transhumanism?} 
\label{sec:what-transhumanism}

Transhumanism is a term used synonymously to mean ``human enhancement.'' It is an international cultural and intellectual movement that endorses the use of science and technology to enhance the human condition, both mentally and physically. In support of this, transhumanism also embraces using emerging technologies to eliminate the undesirable elements of the human condition such as aging, disabilities, diseases, and involuntary death. Many transhumanists believe these technologies will be arriving in our near future at an exponentially accelerated pace and work to promote universal access to and democratic control of such technologies. In the long scheme of things, transhumanism can also be considered the transitional period between the current human condition and an entity so far advanced in capabilities (both physical and mental faculties) as to merit the label ``posthuman.''

As a theme, transhumanism embraces heady questions. What defines human? What does it mean to defeat death? If minds are software, where do you draw the line with programming them? If machines and animals can also be raised to sentience, what are our responsibilities to them? If you can copy yourself, where does ``you'' end and someone new begin? What are the potentials of these technologies in terms of both oppressive control and liberation? How will these technologies change our society, our cultures, and our lives?


\subsection{Post-apocalyptic, conspiracy and horror themes}
\label{sec:post-apoc-consp}

Several themes pervade Eclipse Phase, some of which the reader may not be intimately familiar with. The following helps define these themes so that as play ers read further into this rulebook, they gain a solid understanding of how Eclipse Phase builds on such themes to create its unique setting.

Post-apocalyptic is a term used to describe fiction set after a cataclysmic event has ended human civili zation as we know it (usually accompanied by loss of human life on an almost unthinkable scale). The exact mechanism of the disaster is usually unimportant nuclear war, plague, asteroid strike, and so on. The importance of the theme is the human condition. If the world we know is torn away from us and humans suffer horrors beyond imagining in this transforma tion to a post-apocalyptic setting, how does human ity cope? Do we survive and thrive and overcome? Or do we lose our own humanity in the process, o ultimately fall to extinction? Those are the questions that drive this genre.

To conspire means ``to join in a secret agreement to do an unlawful or wrongful act or to use such means to accomplish a lawful end.'' As such, a con spiracy theory attributes the ultimate cause of an event or a chain of events (whether political, societa or historical) to a secret group of individuals with immense power (including political, wealth and so on) who hide their activities from public view while manipulating events to achieve their goals, regard less of consequences. Many conspiracy theories contend that a host of the greatest events of history were initiated and ultimately controlled by such secret organizations. Of equal importance is the silent struggle between clandestine groups, waging a secret war behind the scenes to determine who influences the future.

Horror takes many forms, but in Eclipse Phase it is more psychological than gore. It is the uncertainty of survival, the suspense of finding malevolent things among the stars, the fear of the unknown, the dread of facing Things That Should Not Be, the revulsion when encountering alien things, and the sickening realization of the wrong and ghastly things that transhumans are capable of doing to themselves and each other. Horror also arises both from the comprehension that there are scary things beyond our understanding nhabiting our universe and that transhumanity may be its own worst enemy. Despite all of the technological tools and advances available to future transhumans, they still face terrors like losing control of their own dentities, their perceptions, and their mental faculties—not to mention their future as a species.

Eclipse Phase takes all of these themes and weaves them together in a transhuman setting. The postapocalyptic angle covers the understanding of all that transhumanity has lost, the fight against extinction, and how much of that is a struggle against our own nature. The conspiracy side delves into the nature of the secret organizations that play key roles n determining transhumanity's future and how the actions of determined individuals can change the ives of many. The horror perspective explores the results of humanity's self-inflicted transformations and how some of these changes effectively make us non-human. Tying it all together is an awareness of the massive indifference and the terrible alien-ness that pervades the universe and how transhumanity is insignificant against such a backdrop.

Offsetting these themes, however, Eclipse Phase also asserts that there is still hope, that there is still something worth fighting for, and that transhumanity can pave its own path toward the future.


\subsection{But how do you actually play?}
\label{sec:but-how-do}

To play a game of Eclipse Phase, you need the following:

\begin{itemize}
\item A group of players and a place to meet (real life or online!) 
\item One player to act as the gamemaster 
\item The contents of this book 
\item Something for everyone to take notes with (note pads, laptops,  whatever!) 
\item Two 10-sided dice per player (or a digital equivalent) 
\item Imagination 
\end{itemize}

\subsubsection{A group of players and a place to meet}
\label{sec:group-players}

While roleplaying games are flexible enough to allow any number of people, most gaming groups number around four to eight players. That number of people brings a good mix of personalities to the table and ensures great cooperative play.

Once a group of players have determined to play Eclipse Phase, they'll need to designate someone as the gamemaster (see below). Then they'll need to determine a time and place to meet.

Most roleplaying groups meet once a week at a regularly scheduled time and place: 7:00 PM, Thursday night, Rob's house, for example. However, each group determines where, how they'll play, and how often. One group may decide they can only get together once a month, while another group is so excited to dive into the story potential of Eclipse Phase that they want to meet twice a week (they decide to rotate between their houses, though, so as not to overload a particular player). If a group is lucky enough to have a favorite local gaming store that supports instore play, the group might meet there. Other gaming groups meet in libraries, common rooms at their school, bookstores that have generously-sized ``reading rooms,'' quiet restaurants, and so on. Whatever fits for your gaming group, make it work!

When getting together for a game, most RPGs use the phrase ``gaming session.'' The length of each gaming session is completely dependent upon the consensus of the playing group, as well as the limitations of the locale where they're playing. The particular story that unfolds in a given session can also impact a session's length. If playing in a game store, the group may only have a four-hour slot and the gamemaster and group may have determined—through several sessions of play—that this is a perfect time frame to enjoy the story they're participating in each week. Another group, however, may want an even shorter length of time. Yet another group may decide that while they'll usually do four-hour sessions, once a month they'll set aside an entire Saturday for a great all-day gaming session. Players will need to dive in and start playing and be flexible to decide what will provide the ultimate enjoyment for their gaming group.

While the camaraderie of a shared experience of playing face-to-face with a group of friends remains the strength of roleplaying games, groups need not confine themselves to a single mode of play. There are myriad options that can be used. Email, instant messages, message boards, video chats, phone/voip calls, text messages, wikis, (micro-)blogs: any and all of these can be utilized to play the game without having warm bodies in seats directly across the table from one another.

Finally, when playing groups meet for the first time, they should generate their characters (as opposed to generating characters by themselves). While a gaming group can decide to generate characters individually, often it is far easier once the players are together. This allows those more experienced in roleplaying games to help those new to RPGs. Even more important, it enables the entire group to tailor the characters so there is not too much overlap in capabilities and style. After all, with the wealth of character opportunities available, you don't want to show up at the table with an almost identical character to the player next to you.

\subsubsection{The gamemaster}
\label{sec:gamemaster}

Once a group has been organized, someone needs to step up and take the reins of the gamemaster. Some groups have a single gamemaster that runs all their gaming sessions month after month. Other groups rotate a gamemaster, with a single gamemaster running a given portion of the unfolding story for several sessions before handing the work off to another player. Once again, the participants should be flexible. Some groups may have the perfect person who loves the work involved and is more than willing to run session after session, while other groups may decide that they all want to take turns both as the gamemaster and as players.

The gamemaster controls the story. They keep track of what is supposed to happen when, describes events as they occur so that the players (as characters) can react to them, keep track of other characters in the game (referred to as non-player characters, or NPCs), and resolve attempts to take action using the game system. The game system comes into play when characters seek to use their skills or otherwise do something that requires a test to see whether or not they succeed. Specific rules are presented for situations that involve rolling dice to determine the outcome (see Game Mechanics, p. 112).

The gamemaster describes the world as the characters see it, functioning as their eyes, ears, and other senses. Gamemastering is not easy, but the thrill of creating an adventure that engages the other players' imaginations, testing their gaming skills and their characters' skills in the game world, makes it worthwhile. Posthuman Studios and Catalyst Game Labs will follow the publication of Eclipse Phase with supporting supplements and adventures to help this process along, but experienced gamemasters can always adapt the game universe to suit their own styles. In fact, since Eclipse Phase is published under a Creative Commons License (see p. 5), players are encouraged to tailor the universe to their style of play and also to share that with other players. You never know when a specific choice you've made in the running of a campaign is exactly what another gamemaster and his group is looking for.

\subsubsection{The contents of this book}
\label{sec:contents-this-book}

Whether you have purchased the print or electronic version, this book is specifically organized to present the information you need to know to start telling your stories in the Eclipse Phase universe. Below you'll find a summary of each chapter of the book.

\paragraph{A Time of Eclipse:} A comprehensive history and setting fully describes the Eclipse Phase universe and how humanity transitioned from here to there. See p. 30.

\paragraph{Game Mechanics:} The player's desired actions become reality within the universe through quick and easy-to-use game mechanics. See p. 112.

\paragraph{Character Creation and Advancement:} Creating a unique character can be one of the most enjoyable experiences of roleplaying. Even more rewarding is watching that character evolve and grow across numerous gaming sessions, far beyond anything your imagination first envisioned. See p. 128.

\paragraph{Skills:} Beyond a character's innate abilities, their skills are what set them apart. This is what your character knows and what they know how to do. See p. 170.

\paragraph{Action and Combat:} What is a dramatic story without action and violence? When words fail, weapons will blaze. See p. 186.

\paragraph{Mind Hacks:} The unusual possibilities offered by psi abilities and mental reprogramming. See p. 216.

\paragraph{The Mesh:} The all-pervasive nature of the mesh ensures that it is a key element to any story telling. See p. 234.

\paragraph{Accelerated Future:} The wonders of advanced technologies and how they work. See p. 266.

\paragraph{Gear:} Personal enhancements, weapons, robots, and everything else in between. See p. 294.

\paragraph{Game Information:} The quintessential set of insider secrets for gamemasters. See p. 350.


\subsubsection{Taking notes}
\label{sec:taking-notes}

Whether a gamemaster or player, you'll need a way to track information. Players will be generating characters and making changes to those characters from session to session. Meanwhile, the gamemaster will have a host of information to track: notes on how the story is unfolding due to player character interaction that you'll need to fold into next week's session; changes to NPCs; changes to player characters that the players are not yet aware off (such as a character has been mind hacked but doesn't yet know it); and so on.

Additionally, some groups enjoy a synopsis of each session that can be compiled and read at a later time in order to enjoy and share their exploits, just as you might fileshare clips from your favorite video game to show off your skill in taking the bad guy down (traditionally this has been called ``bluebooking''). This can be particularly useful if a player was unable to attend a given session, providing a quick re-cap that they can read before attending the next gaming session and thus avoiding a bog-down up-front as that player tries to catch up on current events in the game. The session scribe can be a shared responsibility or assigned, all based upon what a given playing group finds works best for them. Likewise, some gaming groups audiorecord their entire game session, both for later reference and for ``actual play'' podcasts.

The old standard of a pencil and paper still works wonders. A host of additional technologies, however, provide many new options for players. From a text file on a laptop to a shared wiki, the ability to track large amounts of information in a quick and useful fashion—while simultaneously making appropriate information available to each player from session to session—significantly decreases how much time everyone needs to spend tracking information. That time can now be redirected into the enjoyment of participating in a great story.


\subsubsection{Dice}
\label{sec:dice}

As described in the Game Mechanics section (p. 112), two ten-sided dice are required to play Eclipse Phase. While most players enjoy the feel of tossing dice onto a table, there are many other mechanisms for rolling two ten-sided dice to achieve a 00 to 99 result. Players who make heavy use of any online technologies for game play—such as using online chatting or video blogging—should find it easy to track down and implement a quick dice-rolling program.


\subsubsection{Imagination}
\label{sec:imagination}

All too often, it's easy for someone looking at an RPG to be intimidated. So many concepts to grasp, so many ideas that seem overwhelming. Just as described under What is a Roleplaying Game?, however, how often have you read a book or watched that movie and decided that you would have done it better? That's your imagination at work. Just dive in and you'll be amazed at how quickly you can immerse yourself in the Eclipse Phase universe. Soon you'll be spinning stories with the best of them.

Also, don't forget to tap your resources. Your gaming group is your best resource. What's going on, ideas for how to handle a situation, or how to take on a bad guy: these are just some of the things that can and should be discussed by the gaming group in between sessions, and each is an opportunity to strengthen your imagination.

Another resource is simply watching TV or reading a good book. Pay attention to how the story is put together, how the characters are built, and how the plot unfolds. Push your imagination and soon you'll be figuring out subplots and who the bad guy is long before it's revealed. Knowing how a story is put together enables you to put together your own stories during each gaming session.

Finally, eclipsephase.com is the offi cial site for Eclipse Phase. If you have questions about the game or want to see how another group of players handles a given situation, post on the forums. The online community can be just as helpful and enjoyable as a local gaming group.


\subsection{What do players do?}
\label{sec:what-do-players}

The players can take on a variety of roles in Eclipse Phase. Due to advances in digital mind emulation technology, uploading, and downloading into new morphs (physical bodies, biological or synthetic), it is possible to literally be a new person from session to session. With bodies taking on the role of gear, players can customize their forms for the task at hand.


\subsubsection{The default campaign}
\label{sec:default-campaign}

In the default story (also known as ``campaign setting''), every player character is a ``sentinel,'' an agent-on-call (or potential recruit) for a shadowy network known as ``Firewall.'' Firewall is dedicated to counteracting ``existential risks''—threats to the existence of transhumanity. These risks can and do include biowar plagues, nanotech swarm outbreaks, nuclear proliferation, terrorists with WMDs, netbreaking computer attacks, rogue AIs, alien encounters, and so on. Firewall isn't content to simply counteract these threats as they arise, of course, so characters may also be sent on information- gathering missions or to put in place pre-emptive or failsafe measures. Characters may be tasked to investigate seemingly innocuous people and places (who turn out not to be), make deals with shady criminal networks (who turn out not to be trustworthy), or travel through a Pandora's Gate wormhole to analyze the relics of some alien ruin (and see if the threat that killed them is still real). Sentinels are recruited from every faction of transhumanity; those who aren't ideologically loyal to the cause are hired as mercenaries. These campaigns tend to mix a bit of mystery and investigation with fierce bouts of action and combat, also stirring in a nice dose of awe and horror.


\subsubsection{Alternate campaigns}
\label{sec:alternate-campaigns}

When they're not saving the solar system, sentinels are free to pursue their own endeavors. The gamemaster and players can use this rulebook to generate any type of story they wish to tell. However, the following examples provide a brief look at the most obvious opportunities for adventure in Eclipse Phase.

After each campaign variant below, a list of ``archetypes'' for Eclipse Phase are provided in parenthesis. Archetypes are the names applied to the most common character types featured in those scenarios. For example, in a traditional detective story, the archetypes would be the Detective, the Damsel In Distress, the Hard-bitten Cop, and so on. In a cowboy movie, the archetypes would be the Gunfighter, the Bartender, the Marshal, the Indian Brave, and so on. Players will note that some archetypes fit into multiple story settings. The character creation system (p. 128) allows players to create any of the suggested archetypes. Just as roleplaying games are designed for players to build their own stories, however, these archetypes are just suggestions and players can mix and match how they will.

\paragraph{Salvage and Rescue/Retrieval Ops:} The Fall left two worlds and numerous habitats in ruins—but these devastated cities and stations contain untold riches for those who are brave and foolhardy enough. Potential hauls include: weapon systems; physical resources; lost databanks; left-behind uploads of friends, family, or important people; new technologies developed and lost in the brief singularity takeoff; valued heirlooms of immortal oligarchs; and much more. Outside of these once-inhabited realms, space itself is a big place and lots of people and things get lost out there. Some need to be saved and some are beyond saving. This option lets players explore the unknown or seek out specific targets on contract. (Archeologist/Scavenger/Pirate/Free Trader/ Smuggler/Black Marketeer)

\paragraph{Exploration:} There are plenty of opportunities to be had as an explorer, colonist, or long-range scout—perhaps even as one of the few lucky or suicidal individuals who explore through an untested Pandora's Gate. Even the Kuiper Belt, on the fringe of our solar system, is still sparsely explored; there may be riches and mysteries still to be found. Many dangers also lurk in odd corners of the system, from isolationist posthuman factions to secretive criminal cartels, as well as pirates, aliens, and others wishing to remain out of sight. (Explorer/Archeologist/ Scavenger/Singularity Seeker/Techie/Medic)

\paragraph{Trade:} While the majority of inner system trade is controlled by sleek hypercorporations, many of the smaller or more independent stations rely on small traders. In the post-scarcity outer system, trade takes on a different form, with information, favors, and creativity serving as currency among those who no longer want for anything due to the availability of cornucopia machines. (Free Trader/Smuggler/Black Marketeer/Pirate)

\paragraph{Crime:} The patchwork of city-state habitats and widely varying laws throughout the system create ample opportunity for those who would make a living from this situation. Black market commodities and activities include infomorph-slave trading, pleasure pod sex industries, data brokerage and theft, extracting/smuggling advanced technologies and scientists, political/economic espionage, assassination, drug and XP dealing, soul-trading, and much more. Whether as an independent or part of an organized criminal element, there are always opportunities for those with a thirst for adventure or profit and questionable morals. (Criminal/Smuggler/ Pirate/Fixer/Black Marketeer/Genehacker/Hacker/ Covert Ops)

\paragraph{Mercenaries:} The constant maneuvering of ideologically-driven factions, the squabbling over contested resources, and the rush to colonize new exoplanets beyond the Pandora Gates all spark new conflicts on a regular basis. Some of these simmer and seeth as low-intensity conflicts for years, occasionally flaring into raids and clashes. Others break out into all-out warfare. Women and men willing to bear arms for credits are always in demand for good wages. Players can engage in commando and military campaigns in habitats, between the stars, or in hostile planetary environments. (Merc/Security Specialist/Fixer/Bounty Hunter/Ex-Cop/Medic)

\paragraph{Socio-Political Intrigue:} The corporations and political factions that span the solar system do not always play nice with each other, but neither is it wise for them to openly confront each other except under extreme circumstances. Many battles are fought with diplomacy and political maneuvering, using words and ideas more potent than weapons. Even within factions, social cliques can compete ruthlessly, or heated class confl icts can come to a boil, tearing a society apart from within. In this campaign, the players can start as pawns of some entity who rise through the ranks as they become more enmeshed in the intrigues of their sponsor, play a group of ambassadors and spies stationed in the opposition's capital, or can play a group of activists and radicals fighting for social change. (Politico/ Socialite/Covert Ops/Hacker/Security Specialist/ Journalist/Memeticist)


\subsection{Where does it take place?}
\label{sec:where-does-it}

While Eclipse Phase is set in the not-too-distant future, the changes that have taken place due to the advancements of technology have transformed the Earth and its inhabitants almost beyond recognition. As players dive into the universe, they'll generally encounter one of the following settings.


\subsubsection{Humanity's habitats}
\label{sec:humanitys-habitats}

The Earth has been left an ecologically-devastated ruin, but humanity has taken to the stars. When Earth was abandoned, so too were the last of the great nation-states; transhumanity now lacks a single unifying governing body and is instead subject to the laws and regulations of whomever controls a given habitat.

The majority of transhumanity is confined to orbital habitats or satellite stations scattered throughout the Sol system. Some of these were constructed from scratch in the orbit or Lagrange points of planetary bodies, others have been hewn out of solid satellites and large asteroids. These stations have myriad purposes from trade to warfare, espionage to research.

Mars continues to be one of transhumanity's largest settlements, though it too, suffered heavily during the Fall. Numerous cities and settlements remain, however, though the planet is only partially terraformed. Venus, Luna, and Titan are also home to significant populations. Additionally, there are a small number of colonies that have been established on exoplanets (on the other side of the Pandora Gates) with environments that are not too hostile towards humanity.

Some transhumans prefer to live on large colony ships or linked swarms of smaller spacecraft, moving nomadically. Some of these rovers intentionally exile themselves to the far limits of the solar system, far from everyone else, while others actively trade from habitat to habitat, station to station, serving as mobile black markets.


\subsubsection{The great unknown}
\label{sec:great-unknown}

The areas of the galaxy that have felt the touch of humanity are few and far between. Lying betwixt these occasional outposts of questionable civilization are mysteries both dangerous and wonderful. Ever since the discovery of the Pandora Gates, there has been no shortage of adventurers brave or foolhardy enough to strike out on their own into the unknown regions of space in hopes of finding more alien artifacts, or even establishing contact with one of the other sentient races in the universe.


\subsubsection{The mesh}
\label{sec:mesh}

While not a ``setting'' in the traditional sense, as the sections describe above, the computer networks known as the ``mesh'' are all-pervasive. This ubiquitous computing environment is made possible thanks to advanced computer technologies and nanofabrication that allow unlimited data storage and near-instantaneous transmission capacities. With micro-scale, cheap-to-produce wireless transceivers so abundant, literally everything is wirelessly connected and online. Via implants or small personal computers, characters have access to archives of information that dwarf the entire 21st-century internet and sensor systems that pervade every public place. People's entire lives are recorded and lifelogged, shared with others on one of numerous social networks that link everyone together in a web of contacts, favors, and reputation systems.


\subsection{Ego vs. Morph}
\label{sec:ego-vs.-morph}

The distinction between ego (your mind and personality, including memories, knowledge, and skills) and morph (your physical body and its capabilities) is one of the defi ning characteristics of Eclipse Phase. A good understanding of the concept right up front will allow players a glimpse at all the story possibilities out of the gate.

Your body is disposable. If it gets old, sick, or too heavily damaged, you can digitize your consciousness and download it into a new one. The process isn't cheap or easy, but it does guarantee you effective immortality—as long as you remember to back yourself up and don't go insane. The term morph is used to describe any type of form your mind inhabits, whether a vat-grown clone sleeve, a synthetic robotic shell, a part-bio/part-synthetic ``pod,'' or even the purely electronic software state of an infomorph.

A character's morph may die, but the character's ego may live on, assuming appropriate backup measures have been taken. Morphs are expendable, but your character's ego represents the ongoing, continuous life path of your character's mind and personality. This continuity may be interrupted by an unexpected death (depending on how recently the backup was made), but it represents the totality of the character's mental state and experiences.

Some aspects of your character—particularly skills, along with some stats and traits—belong to your character's ego and so stay with them throughout the character's development. Some stats and traits, however, are determined by morph, as noted, and so will change if your character leaves one body and takes on another. Morphs may also affect other skills and stats, as detailed in the morph description.


\subsection{Where to go from here?}
\label{sec:where-go-from}

Now that you know what this game is about, we suggest that you next read the Time of Eclipse chapter (p. 30), to get a feel for the game's default setting (which you are, of course, free to change to suit your whims). Then read the Game Mechanics chapter (p. 112) to get a grasp of the rules. After that, you can move on to Character Creation and Advancement (p. 128) and create your first character!


\subsection{Terminology}
\label{sec:terminology}

Eclipse Phase uses a host of jargon to simply convey the numerous concepts covered within the pages of this book. While not all-inclusive, this list of terminology will allow players to quickly acclimate themselves for their journey into Eclipse Phase. If you read something and are confused, don't worry. These concepts are fully explained in later sections of this book.

Note that several of the words on this list are standard scientific terms, often used in astronomy. As Eclipse Phase attempts to remain as close to ``hard science'' as possible—while allowing players to interact with the great stories waiting to unfold—such terms are used liberally.

\begin{itemize}
\item Aerostat: A habitat designed to float like a balloon in a  planet's upper atmosphere.
\item AF: After the Fall (used for reference dating).
\item AGI: Artificial General Intelligence. An AI that has cogni tive  faculties comparable to that of a human or higher. Also known as  ``strong AI'' (differentiating from more specialized ``weak AI''). See  also ``seed AI.''
\item AI: Artificial Intelligence. Generally used to refer to weak  AIs; i.e., AIs that do not encompass (or in some cases, are  completely outside of) the full range of human cognitive  abilities. AIs differ from AGIs in that they are usually specialized  and/or intentionally crippled/limited.
\item Anarchist: Someone who believes government is unnecessary, that  power corrupts, and that people should control their own lives  through self-organized individual and collective action.
\item Arachnoid: A spider-like robotic synthmorph.
\item Argonauts: A faction of techno-progressive scientists that  promote responsible and ethical use of technology.
\item AR: Augmented Reality. Information from the mesh (universal data  network) that is overlaid on your real-world senses. AR data is  usually entoptic (visual), but can also be audio, tactile,  olfactory, kinesthetic (body awareness), emotional, or other types  of input.
\item Async: A person with psi abilities.
\item AU: Astronomical unit. The distance between the Earth and the  Sun, equal to 8.3 light minutes, or about 150 million kilometers.
\item Autonomists: The alliance of anarchists, Barsoomians,  Extropians, scum, and Titanians.
\item Barsoomian: A rural Martian, typically resentful of hypercorp  control.
\item Basilisk Hack: An image or other sensory input that affects the  brain's visual cortex and pattern recognition abilities in such a  way as to cause a glitch and possibly exploit it and rewrite neural  code.
\item Beehive: A microgravity habitat made from a tunneledout asteroid  or moon.
\item BF: Before the Fall (used for reference dating).
\item Bioconservative: An anti-technology movement that argues for  strict regulation of nanofabrication, AI, uploading, forking,  cognitive enhancements, and other disruptive technologies.
\item Biomorph: A biological body, whether a flat, splicer,  genetically engineered transhuman, or pod.
\item Body Bank: A service for leasing, selling, acquiring, or storing  a morph. Aka dollhouse, morgue.
\item Bots: Robots. AI-piloted synthetic shells.
\item Bracewell Probe: A type of autonomous monitoring deep- space  probe meant to make contact with alien civilizations.
\item Brinkers: Exiles who live on the fringes of the system, as well  as other isolated and well-hidden nooks and crannies. Also called  isolates, fringers, drifters.
\item Case: A cheap, common, mass-produced synthetic shell.
\item Chimeric: Transgenic, containing genetic traits from other  species.
\item Circumjovian: Orbiting Jupiter.
\item Circumlunar: Orbiting the Moon.
\item Circumsolar: Orbiting the Sun.
\item Cislunar: Between the Earth and the Moon.
\item Clade: A species or group of organisms with common  features. Used to refer to transhuman subspecies and morph types.
\item Cole Bubble: A habitat made from a hollowed-out asteroid or  moon, spun for gravity.
\item Cornucopia Machine: A general-purpose nanofabricator.
\item Cortical Stack: An implanted memory cell used for ego  backup. Located where the spine meets the skull; can be cut out.
\item Cyberbrain: An artificial brain, housing an ego. Used in both  synthmorphs and pods.
\item Darkcast: Illegal and black market farcasting and egocasting  services.
\item Domain Rules: The rules that govern the reality of a virtual  reality simulspace.
\item Drone: A robot controlled through teleoperation (rather than  directly via onboard AI).
\item Ecto: Personal mesh devices that are flexible, stretchable,  self-cleaning, translucent, and solar-powered. From ecto-link  (external link).
\item Ego: The part of you that switches from body to body. Also  known as ghost, soul, essence, spirit, persona.
\item Egocasting: Term for sending egos via farcasting.
\item Entoptics: Augmented-reality images that you ``see'' in your  head. (``Entoptic'' means ``within the eye.'')
\item ETI: Extraterrestial intelligence. The term Firewall uses to  refer to the god-like post-singularity alien intelligence theorized  to be responsible for the Exsurgent virus.
\item Exalts: Genetically-enhanced humans (between genefixed and  transhumans). Aka genefreaks, the ascended, the elevated.
\item Exoplanet: A planet in another solar system.
\item Exsurgent: Someone infected by the Exsurgent virus.
\item Exsurgent Virus: The multi-vector virus created by an unknown  ETI and seeded throughout the galaxy in Bracewell probes. The  Exsurgent virus is self-morphing and can infect both computer  systems and biological creatures.
\item Extrasolar: Outside the solar system.
\item Factors: The alien ambassadorial race that deals with  transhumanity. Also called Brokers.
\item The Fall: The apocalypse; the singularity and wars that nearly  brought about the downfall of transhumanity.
\item Farcasting: Intrasolar communication utilizing classical  communication technologies (radio, laser, etc.) and quantum  teleportation.
\item Farhauler: Long distance space shipper.
\item Firewall: The secret cross-faction conspiracy that works to  protect transhumanity from ``existential threats'' (risks to  transhumanity's continued existence).
\item Flatlander: Someone born or used to living on a planet or moon  with gravity.
\item Flats: Baseline humans (not genetically modified). Also called  norms.
\item Flexbot: A shape-changing synthmorph also capable of joining  together with other flexbots in a modular fashion to create larger  shapes.
\item Forking: Copying an ego. Not all forks are full copies. AKA  backups.
\item FTL: Faster-Than-Light.
\item Fury: A transhuman combat morph.
\item Gatecrashers: Explorers who take their chances using a Pandora  gate to go somewhere previously unexplored.
\item Genehacker: Someone who manipulates genetic code to create  genetic modifications or even new life.
\item Ghost: A transhuman combat morph optimized for stealth and  infiltration.
\item Ghost-riding: The act of carrying an infomorph in a special  implant module inside your head.
\item Greeks: Trojan asteroids or moons that share the same orbit as a  larger planet or moon, but are 60 degrees ahead in the orbit at the  L4 Lagrange point. The term Greeks normally refers to the asteroids  orbiting around Jupiter's L4 point. See also ``Trojans.''
\item Habtech: A habitat technician.
\item Heliopause: The point where pressure from the solar wind  balances with the interstellar medium (about 100 AU out).
\item Hibernoid: A transhuman modified for hibernation, for extensive  travel in space.
\item Iceteroid: An asteroid made from mostly ice rather than rock or  metals.
\item Iktomi: The name given to the mysterious alien race whose relics  have been found beyond the Pandora Gates.
\item Indentures: Indentured servants who have contracted their labor  to a hypercorp or other authority, usually in exchange for a morph.
\item Infolife: Artificial general intelligences and seed AIs.
\item Infomorph: A digitized ego; a virtual body. Also known as  datamorphs, uploads, backups.
\item Infugee: ``Infomorph refugee,'' or someone who left everything  behind on Earth during the Fall—even their own body.
\item Isolates: Those who live in isolated communities far outside the  system (in the Kuiper Belt and Oort Cloud); aka outsters, fringers.
\item Jamming: The act of ``becoming'' a teleoperated drone thanks to XP  technology. Also sometimes applied to accesing the real-time XP feed  from lifeloggers and others.
\item Kuiper Belt: A region of space extending from Neptune's orbit  out to about 55 AU, lightly populated with asteroids, comets, and  dwarf planets.
\item Lagrange Point: One of five areas in respect to a small  planetary body orbiting a larger one in which the gravitational  forces of those two bodies are neutralized. Lagrange points are  considered stable and ideal locations for habitats.
\item Lifelog: A recording of one's entire life experience, made  possible due to near unlimited computer memory.
\item Lost Generation: In an effort to repopulate post-Fall, a  generation of children were reared using forced-growth methods. The  results were disastrous: many died or went insane, and the rest were  stigmatized.
\item Main Belt: The main asteroid belt, a torus ring orbiting between  Mars and Jupiter.
\item Meme: A viral idea.
\item Mentons: Transhumans optimized for mental and cognitive ability.
\item Mercurials: The non-human sentient elements of the transhuman  ``family,'' including AGIs and uplifted animals.
\item Mesh: The omnipresent wireless mesh data network. Also used as  a verb (to mesh) and adjective (meshed or unmeshed).
\item Mesh ID: The unique signature attached to one's mesh activity.
\item Microgravity: Zero-g or near weightless environments.
\item Mist: The clouds of AR data that sometimes fog up your  perception/displays.
\item Morph: A physical body. Aka suit, jacket, sleeve, shell, form.
\item Muse: Personal AI helper programs.
\item Nanobot: A nano-scale machine.
\item Nano-ecology: Pro-tech ecological movement.
\item Nanoswarm: A mass of tiny nanobots unleashed into an  environment.
\item Neo-Avians: Uplifted ravens and gray parrots.
\item Neogenesis: The creation of new life forms via genetic  manipulation and biotechnology.
\item Neo-Hominids: Uplifted chimpanzees, gorillas, and orangutans.
\item Neotenics: Transhumans modified to retain a child-like form.
\item Novacrab: A pod created from genetically-engineered spider crab  stock.
\item Olympian: A transhuman biomorph modified for athleticism and  endurance.
\item O'Neill Cylinder: A soda-can shaped habitat, spun for gravity.
\item Oort Cloud: The spherical ``cloud'' of comets that surrounds the  solar system out to about one light-year from the sun.
\item PAN: Personal area network. The network created when you slave  all of your minor personal electronics to your ecto or mesh inserts.
\item Pandora Gates: The wormhole gateways left behind by the TITANs.
\item Pods: Mixed biological-synthetic morphs. Pod clones are  force-grown and feature computer brains. Also known as bio-bots,  skinjobs, replicants. From ``pod people.''
\item Posthuman: A human or transhuman individual or species that has  been genetically or cognitively modified so extensively as to no  longer be human (a step beyond transhuman). Aka parahuman.
\item Prometheans: A group of transhuman-friendly seed AIs that were  created by the Lifeboat Project (precursors to the argonauts) years  before the TITANs became self- aware and that (mostly) avoided  Exsurgent infection. The Prometheans secretly back Firewall and work  to defeat existential threats.
\item Proxies: Members of the Firewall internal structure.
\item Psi: Parapsychological powers acquired due to infection by the  Watts-MacLeod strain of Exsurgent virus.
\item Reaper: A warbot synthmorph.
\item Reclaimers: A transhuman faction that seeks to lift the  interdiction and reclaim Earth.
\item Redneck: A rural Martian. See Barsoomian. Aka Reds.
\item Reinstantiated: Refugees from Earth who escaped only as bodiless  infomorphs, but who have since been resleeved.
\item Resleeving: Changing bodies, or being downloaded into a new  one. Also called remorphing, reincarnation, shifting, rebirthing.
\item Rusters: Biomorphs optimized for life on Mars.
\item Scorching: Hostile programs that can damage or affect  cyberbrains.
\item Scum: The nomadic faction of space punks/gypsies that travel  from station to station in heavily-modified barges or swarms of  ships. Notorious for being a roving black market.
\item Seed AI: An AGI that is capable of recursive self-improvement,  allowing it to reach god-like levels of intelligence.
\item Sentinels: Agents of Firewall.
\item Shell: A synthetic physical morph. Aka synthmorph.
\item Simulmorph: The avatar you use in VR simulspace programs.
\item Simulspace: Full-immersion virtual reality environments.
\item Singularity: A point of rapid, exponential, and recursive  technological progress, beyond which the future becomes impossible  to predict. Often used to refer to the ascension of seed AI to  god-like levels of intelligence.
\item Singularity Seeker: People who pursue relics and evidence of the  TITANs or other possible avenues to super-intelligence, either to  learn more about it or to become part of a super-intelligence  themselves.
\item Skin: A biological physical morph. Aka meat, flesh.
\item Skinning: Changing your perceived environment via augmented  reality programming.
\item Sleight: A psi power.
\item Slitheroid: A snake-like robotic synthmorph.
\item Smart Animals: Partially-uplifted animal species (including  dogs, cats, rats, and pigs). Some other large smart animals (whales,  elephants) are nearly extinct.
\item Spime: Meshed, self-aware, location-aware devices.
\item Splicers: Humans that are genetically modified to eliminate  genetic diseases and some other traits. Also known as genefixed,  cleangenes, tweaks.
\item Swarmanoid: A synthetic morph composed from a swarm of tiny  insect-sized robots.
\item Sylphs: Transhuman biomorphs with exotic good looks.
\item Synthmorph: Synthetic morphs. Robotic shells possessed by  transhuman egos.
\item Synths: A specific type of synthmorph. Synths are standard  androids/gynoids; robots that are designed to look humanoid, though  they are usually noticeably not human.
\item Teleoperation: Remote control.
\item Titanian: Someone from Titan, a moon of Saturn.
\item TITANs: The human-created, recursively-improving, military seed  AIs that underwent a hard-takeoff singularity and prompted the  Fall. Original military designation was TITAN: Total Information  Tactical Awareness Network.
\item Torus: A donut-shaped habitat, spun for gravity.
\item Transgenic: Containing genetic traits from other species.
\item Transhuman: An extensively modified human.
\item Trojans: Asteroids or moons that share the same orbit as a  larger planet or moon, but follow about 60 degrees ahead or behind  at the L4 and L5 Lagrange points. The term Trojans normally refers  to the asteroids orbiting at Jupiter's Lagrange points, but Mars,  Saturn, Neptune, and other bodies also have Trojans. See also  ``Greeks.''
\item Uplifting: Genetically transforming an animal species to  sapience.
\item Vacworker: Space laborer.
\item Vapor: A failed mind emulation or crippled fork/infomorph (from  vaporware).
\item VPNs: Virtual private networks. Networks that operate within the  mesh, usually encrypted for privacy/security.
\item VR: Virtual Reality. Imposing an artificially-constructed  hyper-real reality over one's physical senses.
\item X-Caster: Someone who transmits/sells XP recordings of their  experiences.
\item Xenomorph: Alien life form.
\item Xer: As in ``X-er''—someone who is addicted or obsessed with  XP. Sometime used to refer to people making XP as well.
\item XP: Experience Playback. Experiencing someone else's sensory  input (in real-time or recorded). Also called experia, sim,  simsense, playback.
\item X-Risk: Existential risk. Something that threatens the very  existence of transhumanity.
\item Zeroes: People without wireless mesh access. Common with some  indentures. \end{itemize}

\begin{quotation}

\textbf{Welcome to Firewall}

[Incoming Message Received. Source: Unknown]

[Quantum Analysis: No Interception Detected]

[Decryption Complete]

Greetings,

Your references and background have been triple-checked and confirmed, and you are now vetted as a sentinel operative. Welcome to Firewall, friend.

For those new to our private network, Firewall is an organization dedicated to protecting transhumanity from threats—both internal and external—to our continued existence as a species. The Fall may have reminded us that our ability to survive and prosper is not guaranteed, but our kind has a remarkably short attention span. Despite our achievement of functional near-immortality, we continue to face numerous dangers that may contribute to our extinction. Some of these risks come from our own factionalism and divisiveness, combined with universally available technology that could cause widespread destruction and untold deaths in the wrong hands. Some stem from our short-sightedness, failing to see the dangers in which we place ourselves and our environments through careless actions. Some arise from our own creations turned against us, as the TITANs proved. Other risks may come from alien intelligences whose motivations we cannot yet fathom, and of whom we may not even be aware. Still others may threaten us by sheer chance and the mindless but deadly cause-and-effect of a universe in which we are but an insignificant speck.

Firewall exists to identify, analyze, and counter these risks. We are all volunteers. We are all placing our own lives at risk in order to ensure the survival of transhumanity.

Firewall has existed, under other names and guises, since before the Fall. Numerous agencies with a similar agenda banded together in the wake of those cataclysmic events to assess our situation and prepare for the worst. Now we operate under a single umbrella.

We are a private network for two reasons. First, our existence and operational abilities are protected by our secrecy. The less our opposition knows about us, the more effectively we can counter them. Similarly, certain authorities might be hostile to an organization such as ours operating in their claimed territory. Though some may be aware of our existence, we bypass numerous legal and jurisdictional hurdles that might otherwise hamper our actions and goals. Second, our mission sometimes brings to light information that is not only dangerous in the wrong hands, but might even trigger widespread panic if made public. In some cases, the very existence of such knowledge could be problematic. By retaining secrecy and operating on a need-to-know basis, we automatically counter certain risks.

Firewall is a decentralized, peer- to-peer network. We have minimal hierarchy and we answer to no one but ourselves. Our node structure enables us to share resources and talents without sacrificing the privacy and security of our operatives. You have been recruited because of your knowledge, assets or skills, and/or because you have come into contact with certain restricted data. You have proven your willingness to support our goals. Our lives and existence—and the future of transhumanity—may rest in your hands.

So here's to the future—may we all live to see it.

[End Message]

[This Message Has Self-Erased]

\end{quotation}

\begin{quotation}

\textbf{What you really need to know}

[Incoming Message Received. Source: Unknown]

[Quantum Analysis: No Interception Detected]

[Decryption Complete]

Sit down, and grab yourself a fucking drink.

Forget all of that AI-generated intro crap you just read. Here's the real deal.

You're probably dying to know what you've been dragged into. Maybe you've been told the party line already: that we're all that stands between transhumanity and extinction. Or maybe someone whispered to you that we're a rogue operation that meddles in heavy shit that we have no authority to get involved in, and that we sometimes get people killed as a result. You must be curious. Maybe you've got a vigilante streak, and you're looking to spill blood for a good cause. Would it matter to you if the cause was a deluded one? Maybe you're a conspiracy wingnut and you're dying to know what secrets Firewall is clutching to its collective chest. What if those secrets shattered the carefully constructed lies that we all tell to ourselves to keep our sanity intact?

Everything you've heard, good or bad, about Firewall very well may be true. We're not angels. We lost the sheen on our ideals when the TITANs forcibly uploaded their first human mind. Right now, you should be asking yourself what the fuck you just signed up for. I did.

Truth is, Firewall is lots of things. Most of it is good, but a lot of it so fucking horrible you'll be thinking about planting a bullet in your stack and resorting to an earlier backup, just so you can forget it all. If you have any romantic visions about being a hero, though, drop them now. You won't feel like a hero when you airlock some kid because he's carrying an infectious nanovirus. You won't feel brave when you run across some alien thing and crap your pants. And you won't even feel human anymore when you make a call that will cost dozens, hundreds, or even thousands of people their lives, even if you are saving millions more.

So why would anyone be crazy enough to be part of this thing? Because it needs to be done. Our survival depends on it. To some people, it's altruism, defending transhumanity. But really, it's about saving your own fucking neck too. Sure, you could abstain from taking responsibility and let some self- described authority take care of it. But if the anarchists have anything right, it's that people in power can't be trusted. As often as not, they're part of the problem. So Firewall does things the collective way. We're underground, but we're an open source operation. We share information and resources towards a common goal. We organize in networked ad-hoc cells, smart-mob style. We don't let anyone accrue too much power or control. Everyone involved in an op has an equal say. We police ourselves. We come from all sorts of backgrounds and factions, but we face a common enemy—and we fight to win. There is no alternative.

Maybe you've heard of the Fermi Paradox? That question asked why, with a galaxy so huge, there were so few signs of other life? Even though we've met the Factors and seen evidence of other aliens, our galactic neighborhood should be crawling with intelligence—but it's not.

I'll tell you why. The universe is not fucking fair. If transhumanity were wiped out, the galaxy wouldn't even notice. Just look at the Earth. That planet still exists, still supports life, even though we're far gone. Reality is an uncaring asshole. Forget all that utopian crap about living forever. We'll be lucky to survive another year. We've developed technologies that put weapons of mass destruction in the hands of everyone, but we're still an adolescent species that has trouble overcoming petty tribal bullshit. If you're really looking forward to exploring the universe as a postmortal, you're going to have to work hard at it. Survival isn't a right, it's a privilege.

When you sign up with Firewall, you put yourself on call. Anytime some shit goes down in your neck of the woods or that you might be particularly helpful in dealing with, you'll get a call. You'll be expected to drop whatever you're doing and put everything else on hold as if your life depended on it—it probably does. When you're in the field, on an op— ''going to the doctor,'' as we call it—your cell is empowered to act as it sees fit ... just keep in mind that you'll be answering to the rest of us later. You'll also have the Firewall network to back you up—though resources are often limited, so don't expect us to always save your ass. Other sentinels can be called on to pull strings, but every time we do so, it threatens to unveil an agent, create a trail that we need to clean up, and otherwise complicates matters. Self-reliance is key.

One last thing: don't ever, ever forget that we have enemies. I'm not just talking about the nutjob who wants to nuke a habitat to make a political statement or the neo-luddites who think biowar plagues will teach us all a lesson, I'm talking about the agencies that know Firewall exists and consider it a threat. If they tag you as a sentinel, your days are numbered. Maybe your backups too. So watch yer friggin' back.

So that's the real deal, as honest as I can give it. Welcome to our secret clubhouse, comrade. Remember: death is just another day on the job.

[End Message]

[This Message Has Self-Erased] \end{quotation}

%%% Local Variables:  %%% mode: latex %%% TeX-master: "ep" %%% End: 
