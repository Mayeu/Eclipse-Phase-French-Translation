\chapter{Game mechanics}
\label{chap:game-mechanics}

In every game, there comes a time when the gamemaster must decide if a character succeeds or fails in an action. This is when the players roll dice and the characters' stats and abilities come into play. This chapter defines the core mechanics and rules that govern the outcomes of events in Eclipse Phase.

\section{A note on terminology and Gender}
The Eclipse Phase setting raises a number of interesting questions about gender and personal identity. What does it mean when you are born female but you are occupying a male body? When it comes to language and editing, this also poses a number of interesting questions for what pronouns to use. The English language has a bit of a bias towards male-gendered pronouns that we hope to avoid in these rules. For purposes of this game, we’ve sidestepped some of these gender neutrality quandaries by adopting the “Singular They” rule. What this means is that rather than just going with male pronouns (“he”) or switching between gendered pronouns (“he” in one chapter, “she” in the next), we have adopted the use of “they” even when referring to a single person. To some folks, this is bad grammar, but there is actually some good evidence that this usage has strong historical roots (look it up), and it certainly gives our editors fewer headaches. When referring to specifi c characters, we use the gendered pronoun appropriate to the character’s personal gender identity, no matter the sex of the morph they are in.


\section{Core rules}
\label{sec:basics}

\subsection{The ultimate rule}
\label{sec:ultimate-rule}

One rule in Eclipse Phase outweighs all of the others: have fun. This means that you should never let the rules get in the way of the game. If you don't like a rule, change it. If you can't find a rule, make one up. If you disagree over a rule's interpretation, flip a coin. Try not to let rules interfere with the game's flow and mood. If you're in the middle of a really good scene or intense roleplaying and a rule suddenly comes into question, don't stop the game to look it up and argue about it. Just wing it, make a decision quickly, and move on. You can always look the rule up later so you'll remember it next time. If there are disagreements over a rule's interpretation, remember that the gamemaster gets the final say.

This rule also means that you shouldn't let the story be guided solely by rolls of the dice. The element of chance that dice rolls provide lends a sense of randomness, uncertainty, and surprise to the game. Sometimes this is exciting, like when a character makes an unexpectedly difficult roll and saves the day. At other times, it is brutal, such as when a lucky shot from an opponent takes one of the characters out for good in a fight. If the gamemaster wants a scenario to result in a pre-planned dramatic outcome and an unexpected die roll threatens that plan, they should feel free to ignore that roll and move the story in the direction they desire.

\subsection{Dice}
\label{sec:dice-1}

Eclipse Phase uses two ten-sided dice (each noted as a d10) for random rolls. In most cases, the rules will call for a percentile roll, noted as d100, meaning that you roll two ten-sided dice, choosing one to count first, and then read them as a result between 0 and 99 (with a roll of 00 counting as zero, not 100). The first die counts as the tens digit, and the second die counts as the ones digit. For example, you roll two ten-sided dice, one red and one black, calling out red first. The red one rolls a 1 and the black die rolls a 6, for a result of 16. Some sets of d10s, as shown above, are specifically marked for easier rolling and reading.

Occasionally the rules will call for individual die rolls, with each individual ten-sided die listed as a d10. If the rules call for several dice to be rolled, it will be noted as 2d10, 3d10, and so on. When multiple ten-sided dice are rolled in these instances, the results are added together. For example, a 3d10 roll of 4, 6, and 7 counts as 17. On d10 rolls, a result of 0 is treated as a 10, not a zero.

Most players of Eclipse Phase can get by with having two ten-sided dice, but it doesn't hurt to have more on hand. These dice can be purchased at your friendly local game store or borrowed from another gamer.

\subsection{Making tests}
\label{sec:making-tests}

In Eclipse Phase, your character is bound to find themself in adrenalin-pumping action scenes, high- stress social situations, lethal combats, spine-tingling investigations, and similar situations filled with drama, risk, and adventure. When your character is embroiled in these scenarios, you determine how well they do by making tests—rolling dice to determine if they succeed or fail, and to what degree.

You make tests in Eclipse Phase by rolling d100 and comparing the result to a target number. The target number is typically determined by one of your character's skills (discussed below) and ranges between 1 and 98. If you roll less than or equal to the target number, you succeed. If you roll higher than the target number, you fail.

A roll of 00 is always considered a success. A roll of 99 is always a failure.

\begin{quotation}
Jaqui's character needs to make a skill test. Her skill is 55. Jaqui takes two ten-sided dice and rolls a 53—she succeeds! If she had rolled a 55, she still would have been successful, but any roll higher than that would have been a failure.
\end{quotation}

\subsection{Target numbers}
\label{sec:target-numbers}

As noted above, the target number for a d100 roll in Eclipse Phase is usually the skill rating. Occasionally, however, a different figure will be used. In some cases, an aptitude score is used, which makes for much harder tests as aptitude scores are usually well below 50 (see Aptitudes, \ref{sec:aptitudes}). In other tests, the target number will be an aptitude rating x 2 or x 3 or two aptitudes added together. In these cases, the test description will note what rating(s) to use.

\subsection{When to make tests}
\label{sec:when-make-tests}

The gamemaster decides when a character must make a test. As a rule of thumb, tests are called for whenever there is a chance that a character might fail at an action or when success or failure may have an effect on the ongoing story. Tests are also called for whenever two or more characters act in opposition to one another (for example, if they are arm wrestling or haggling over a price). On the other hand, routine use of a skill by someone with at least a rating of 30 in that skill can be assumed to be successful with no test.

It is not necessary to make tests for everyday, run-of-the-mill activities, such as getting dressed or checking your email (especially in Eclipse Phase, where so many activities are automatically handled by the machines around you). Even an activity such as driving a car does not call for dice rolls as long as you have a small modicum of skill. A test might be necessary, however, if you happen to be driving while bleeding to death or are pursuing a gang of motorcycle-riding scavengers through the ruins of a devastated city.

Knowing when to call for tests and when to let the roleplaying flow without interruption is a skill every gamemaster must acquire. Sometimes it is better to simply make a call without rolling dice in order to maintain the pacing of the game. Likewise, in certain circumstances the gamemaster may decide to make tests for a character in secret, without the player noticing. If an enemy is trying to sneak past a character on guard, for example, the gamemaster will alert the player that something is amiss if they ask them to make a perception test. This means that the gamemaster should keep a copy of each character's record sheet on hand at all times.

\subsection{Difficulty and modifiers}
\label{sec:difficulty-modifiers}

The measure of a test's difficulty is reflected in its modifiers. Modifiers are adjustments made to the target number (not the roll), either raising or lowering it. A test of average difficulty will have no modifiers, whereas actions that are easier will have positive modifiers (raising the target number, making success more likely) and harder actions will have negative modifiers (lowering the target number, making success less likely). It is the gamemaster's job to determine if a particular test is harder or easier than normal and to what degree (as illustrated on the Test Difficulty table) and to then apply the appropriate modifier.

Other factors might also play a role in a test, applying additional modifiers aside from the test's general level of difficulty. These factors include the environment, equipment (or lack thereof), and the health of the character, among other things. The character might be using superior tools, working in poor conditions, or even wounded, and each of these factors must be taken into account, applying additional modifiers to the target number and adjusting the likelihood of success or failure.

For simplicity, modifiers are applied in multiples of 10 and come in three levels of severity: Minor (+/–10), Moderate (+/–20), and Major (+/–30). Any number of modifiers may be applied, as the gamemaster deems appropriate, but the cumulative modifiers may not exceed + or – 60.

\begin{quotation}
Jaqui is attempting to leap from one door to another across a large chamber in zero gravity. She's in a hurry. If she misses the door, she'll lose valuable time, so the gamemaster calls for a Freefall Skill Test. Jaqui's Freefall skill is 46. Unfortunately the chamber is filled with floating debris that could get in her way. The gamemaster determines this is a Moderate modifier, reducing the target number by 20. Jaqui must roll a 26 or less to succeed.
\end{quotation}

\subsection{Criticals: rolling doubles}
\label{sec:crit-roll-doubl}

Any time both dice come up with the same number -- 00, 11, 22, 33, 44, etc.—you have scored a critical success or critical failure, depending on whether your roll also beat your target number. 00 is always a critical success, whereas 99 is always a critical failure. Rolling doubles means that a little something extra happened with the outcome of the test, either positive or negative. Criticals have a very specific application in combat tests (see \ref{sec:combat}), but for all other purposes the gamemaster decides what exactly went wrong or right in a specific situation. Criticals can be used to amplify a success or failure: you finish with a flourish or fail so spectacularly that you remain the butt of jokes for weeks to come. They can also result in some sort of unexpected secondary effect: you repair the device and improve its performance; or you fail to shoot your enemy and hit an innocent bystander. Alternately, a critical can be used to give a boost (or a hindrance) to a follow-up action. For example, you not only spot a clue, but you immediately suspect it to be red herring; or you not only fail to strike the target, but your weapon breaks, leaving you defenseless. Gamemasters are encouraged to be inventive with their use of criticals and choose results that create comedy, drama, or tension.

\begin{quotation}
Audrey is attempting to intimidate a low-level triad mook into giving her information. Unfortu- nately she rolls a 99—a critical failure. Not only does she fail to scare the guy, but she accidentally lets slip an important piece of information that she didn't want the triad to know. If she rolled a 00 instead—a critical success—she would browbeat the man so thoroughly that he throws in some extra important information just so she'll leave him alone in the future.
\end{quotation}

\subsection{Defaulting: untrained skill use}
\label{sec:defa-untr-skill}

Certain tests may call for a character to use a skill they don't have—a process called defaulting. In this case, the character instead uses the rating of the aptitude (see p. 123) that is linked to the skill in question as the target number.

Not all skills may be defaulted; some of them are so complex or require such training than an unskilled character can't hope to succeed. Skills that may not be defaulted on are noted on the Skill List (p. 176) and in the skill description.

In rare cases, a gamemaster might allow a character to default to another skill that also relates to a test (see p. 173). When allowed, defaulting to another skill incurs a –30 modifier.

\begin{quotation}
Toljek is trying to casually sneak inside a hypercorp facility when he unexpectedly runs into a hypercorp employee. The woman he's encountered doesn't necessarily have grounds to be suspicious of Toljek's presence, but the gamemaster calls for Toljek to make a Protocol Test to pass himself off as someone that belongs there. Unfortunately, Toljek doesn't have that skill, so he must default to its linked aptitude, Savvy, instead. His Savvy score is only 18, so Toljek better hope he gets lucky.
\end{quotation}

\subsection{Simplifying modifiers}
\label{sec:simpl-modif}

Rather than looking up and accumulating a long list of modifiers for each action and doing the math, the gamemaster can instead choose to simply “eyeball” the situation and apply the modifier that best sums up the net effect. This method is quicker and allows for easier test resolution. One way to eyeball the situation is to simply apply the most severe modifier affecting the situation.

\begin{quotation}
Tyska is trying to escape from some thing that's chasing him through a derelict habitat. The gamemaster calls for a Freerunning Test, but there are a number of modifying conditions: it's dark, he's running with a flashlight, and there's debris everywhere. Tyska, however, has an entoptic map of the best route out of there to help him out. The gamemaster assesses the situation and decides the overall effect is that the test is challenging, and so a –20 modifier is applied.
\end{quotation}

\subsubsection{narrative modifiers}
\label{sec:narrative-modifiers}

If you wish to develop a more cinematic feel for your game, or if you simply wish to encourage your players to invest more detail and creativity into the storyline, you can award “narrative modifiers” to a character's test when that player describes what the character is doing in exceptionally colorful, inventive, or dramatic detail. The better the detail, the better the modifier.

\begin{quotation}
Cole doesn't just want his character to jump over the table, he wants to make an impact. Cole tells the gamemaster that his character kicks a chair out of the way, rolls over the dinner table on his shoulder, grabs a fork as he does it, makes sure to knock all of the fine china on the floor, then lands on his feet in a defensive martial arts posture, fork raised high. The gamemaster decides the extra description is worth +10 to his Freerunning Test.
\end{quotation}

\subsection{Teamwork}
\label{sec:teamwork}

If two or more characters join forces to tackle a test together, one of the characters must be chosen as the primary actor. This leading character will usually (but not always) be the one with the highest applicable skill. The primary acting character is the one who rolls the test, though they receive a +10 modifier for each additional character helping them out, up to a maximum +30 modifier. Note that helping characters do not necessarily need to know the skill being used if the gamemaster decides that they can follow the primary actor's lead.

\begin{quotation}
The robotic leg on Eva's synthetic morph has been badly damaged, so she needs to repair it. Max and Vic both sit down and help her out, giving her a +20 modifier (+10 for each helper) to her Hardware: Robotics Test.
\end{quotation}

\section{Types of tests}
\label{sec:types-tests}

There are two types of tests in Eclipse Phase: Success and Opposed.

\subsection{Success tests}
\label{sec:success-tests}

Success Tests are called for whenever a character is acting without direct opposition. They are the standard tests used to determine how well a character exercises a particular skill or ability.

Success Tests are handled exactly as described under Making Tests, p. 115. The player rolls d100 against a target number equal to the character's skill +/– modifiers. If they roll equal to or less than the target number, the test succeeds, and the action is completed as desired. If they roll higher than the target number, the test fails.

\subsubsection{Trying again}
\label{sec:trying-again}

If you fail at a test, you can take another shot. Each subsequent attempt at an action after a failure, however, incurs a cumulative –10 modifier. That means the second try suffers –10, the third –20, the fourth –30, and so on, up to the maximum –60.

\subsubsection{Taking the time}
\label{sec:taking-time}

Most skill tests are made for Automatic, Quick, or Complex Actions (see pp. 119–120) and so are resolved within one Action Turn (3 seconds, see p. 119). Tests made for Task actions (p. 120) take longer.

Players may choose to take extra time when their character undertakes an action, meaning that they choose to be especially careful when performing the action in order to enhance their chance of success. For every minute of extra time they take, they increase their target number by +10. Once they've modified their target number to over 99, they are practically assured of success, so the gamemaster can waive the dice roll and grant them an automatic success. Note that the maximum +60 modifier rule still applies, so if their skill is under 40 to start with, taking the time may still not guarantee a favorable outcome. You may take the time even when defaulting (see Defaulting, p. 116).

Taking extra time is a solid choice when time is not a factor to the character, as it eliminates the chance that a critical failure will be rolled and allows the player to skip needless dice rolling. For certain tests it may not be appropriate, however, if the gamemaster decides that no amount of extra time will increase the likelihood of success. In that case, the gamemaster simply rules that taking the time has no effect.

For Task action tests (p. 120), which already take time to complete, the duration of the task must be increased by 50 percent for each +10 modifier gained for taking extra time.

\begin{quotation}
Srit is searching through an abandoned space- ship, looking for a sign of what happened to the missing crew. The gamemaster tells her it will take twenty minutes to search the whole ship. She wants to be extra thorough, however, so she takes an extra thirty minutes. Fifty percent of the original timeframe is ten minutes, so taking an extra thirty minutes means that Srit receives a +30 modifier to her Investigation Test.
\end{quotation}

\subsubsection{Simple success tests}
\label{sec:simple-success-tests}

In some circumstances, the gamemaster may not be concerned that a character might fail a test, but instead simply wants to gauge how well the character performs. In this case, the gamemaster calls for a Simple Success Test, which is handled just like a standard Success Test (p. 117). Rather than determining success or failure, however, the test is assumed to succeed. The roll determines whether the character succeeds strongly (rolls equal to or less than the target number) or succeeds weakly (rolls above the target number).

\begin{quotation}
Dav is taking a short spacewalk between two parked ships. The gamemaster determines that this is a routine operation and calls for Dav to make a Simple Success Test using the Freefall skill. Dav's skill is only 35. He rolls a 76, but the gamemaster merely determines that Dav has some trouble orienting himself and has to take some extra time. If Dav had rolled a 77—a critical failure—his suit's maneuvering jets may have died and he may have accidentally propelled himself into deep space.
\end{quotation}

\subsubsection{Margin of success/failure}
\label{sec:marg-succ}

Sometimes it may be important that a character not only succeeds, but that they kick ass and take names while doing it. This is usually true of situations where the challenge is not only difficult but the action must be pulled off with finesse. Tests of this sort may call for a certain Margin of Success (MoS)—an amount by which the character must roll under the target number. For example, a character facing a target number of 55 and a MoS of 20 must roll equal to or less than a 35 to succeed at the level the situation calls for.

\begin{quotation}
An enemy has thrown an incendiary device near Stoya. She has only a moment to act and decides to try to kick it away from herself. Even better, she hopes to kick it into the open maintenance hatch a dozen meters away. The gamemaster determines that in order to kick it into the hatch, Stoya needs to succeed with an MoS of 30. Her Unarmed Combat skill is 66, so Stoya needs to roll 66 or less to kick the device away (though she may still be damaged when it explodes), and 36 or less to kick it into the hatch (in which case she will be completely safe when it detonates). She rolls a 44—missing the hatch, but scoring a critical success! Her aim is off, but the gamemaster decides that the device rebounds off some machinery and falls into the hatch anyway.
\end{quotation}

At other times, it may be important to know how badly a character fails, as determined by a Margin of Failure (MoF), which is the amount by which the character rolled over the target number. In some cases, a test may note that a character who fails with a certain MoF may suffer additional consequences for failing so dismally.

\begin{quotation}
Nico is trying to sketch out a picture of someone's face. He has eidetic memory, but his drawing needs to be good enough for someone else to identify the person. He rolls against his Art: Drawing skill of 34, scoring a 97—a MoF of 63. The illustration is so bad that the gamemaster determines that anyone using that picture to identify the person will need to score a MoS of at least 63 on a Perception Test to recognize the person.
\end{quotation}

\subsubsection{Excellent successes/severe failures}
\label{sec:excell-succ-fail}

Excellent Successes and Severe Failures are a method used to benchmark successes and failures with an MoS or MoF of 30+. Excellent Successes are used in situations where an especially good roll may boost the intended effect, such as inflicting more damage with a good hit in combat. Severe Failures denote a roll that is particularly bad and has a worse effect than a simple failure. Neither Excellent Successes or Severe Failures are as good or bad as criticals, however.

\begin{quotation}
Stoya has been caught in a deal gone bad. She moves to kick her opponent using her Unarmed Combat of 65. She rolls a 33 (for an MoS of 32), and her opponent rolls a 21 (also successful, but less than Stoya, so she wins). She has succeeded and beaten her opponent with an MoS of 30+, scoring an Excellent Success, meaning she will inflict extra damage with the kick.
\end{quotation}

\subsection{Opposed tests}
\label{sec:opposed-tests}

An Opposed Test is called for whenever a character's action may be directly opposed by another. Regardless of who initiates the action, both characters must make a test against each other, with the outcome favoring the winner.

To make an Opposed Test, each character rolls d100 against a target number equal to the relevant skill(s) along with any appropriate modifiers. If only one of the characters succeeds (rolls equal to or less than their target number), that character has won. If both succeed, the character who gets the highest dice roll wins. If both characters fail, or they both succeed and roll the same number, then a deadlock occurs—the characters remain pitted against each other, neither gaining ground, until one of them takes another action and either breaks away or makes another Opposed Test.

Note that critical successes trump high rolls in an Opposed Test—if both characters succeed and one rolls 54 while the other rolls 44, the critical roll of 44 wins.

Care must be taken when applying modifiers in an Opposed Test. Some modifiers will affect both participants equally, and should be applied to both tests. If a modifier arises from one character's advantage in relation to the other, however, that modifier should only be applied to benefit the favored character; it should not also be applied as a negative modifier to the disadvantaged character.

\begin{quotation}
Zhou has been hired by the Jovian Republic to infiltrate his old pirate band. Even though he's resleeved in a new skin, he's worried that one of his old buddies, Wen, might recognize his mannerisms, since they lived, whored, and raided together for years. After Zhou has spent some time in Wen's company, the gamemaster makes a secret Opposed Test, pitting Zhou's Impersonation skill of 57 against Wen's Kinesics of 34. The gamemaster decides to give Wen a bonus +20, since he is so familiar with his former buddy and has been on the lookout for him, eager to repay the old grudge that split them apart. Wen's target number is now 54.

 The gamemaster rolls for both. Zhou scores a 45 and Wen a 39. Both succeed, but Zhou rolled higher, so his deception is successful. The gamemaster decides that Wen finds something about Zhou to be familiar, but he can't put his finger on it.
\end{quotation}

\subsubsection{Opposed tests and margin of success/failure}
\label{sec:opposed-tests-margin}

In some cases, it may also be important to note a character's Margin of Success or Failure in an Opposed Test, as with a Success Test above. In this case, the MoS/MoF is still determined by the difference between the character's roll and their target number—it is not calculated by the difference between the character's roll and the opposing character's roll.

\subsubsection{Variable opposed test}
\label{sec:vari-oppos-test}

In some cases, the rules will call for a Variable Opposed Test, which allows for slightly more outcomes than a standard Opposed Test. If both characters succeed in a Variable Opposed Test, then an outcome is obtained which is different from just one character winning over the other. The exact outcomes are noted with each specific Variable Opposed Test.

\begin{quotation}
Jaqui needs to hack into a local network to retrieve some video footage. The network is ac- tively defended by an AI, so a Variable Opposed Test is called for, pitting Jaqui's Infosec skill of 48 against the AI's Infosec of 25. Jaqui rolls a 48—a success—but the AI rolls a 14—also a success. In this circumstance, Jaqui succeeds in hacking in, but the AI is aware of the infiltration and can take active countermeasures against her.
\end{quotation}

\section{Time and actions}
\label{sec:time-actions}

Though the gamemaster is responsible for managing the speed at which events unfold, there are times when it is important to know exactly who is acting when, especially if some people are acting before or after other people. In these circumstances, gameplay in Eclipse Phase is broken down into intervals called Action Turns.

\subsection{Action turns}
\label{sec:action-turns}

Each Action Turn is three seconds long, meaning there are twenty Action Turns per minute. The order in which characters act during a turn is determined by an Initiative Test (see Initiative, p. 121). Action Turns are further subdivided into Action Phases. Each character's Speed stat (p. 121) determines the amount of actions they can take in a turn, represented by how many Action Phases they may take.

\subsection{Types of actions}
\label{sec:types-actions}

The types of actions a character may take in an Action Turn are broken down to: Automatic, Quick, Complex, and Task actions.

\subsubsection{automatic actions}
\label{sec:automatic-actions}

Automatic actions are “always on” and require no effort from the character, assuming they are conscious.

Examples: basic perception, certain psi sleights

\subsubsection{Quick actions}
\label{sec:quick-actions}

Quick actions are simple, so they can be done fast and can be multi-tasked. The gamemaster determines how many Quick actions a character may take in a turn.

Examples: talking, switching a safety, activating an implant, standing up

\subsubsection{Complex actions}
\label{sec:complex-actions}

Complex actions require concentration or effort. The number of Complex actions a character may take per turn is determined by their Speed stat (see p. 121). Examples: attacking, shooting, acrobatics, disarming a bomb, detailed examination

\subsubsection{Task actions}
\label{sec:task-actions}

Task actions are any actions that require longer than one Action Turn to complete. Each Task action has a timeframe, usually listed in the task description or otherwise determined by the gamemaster. The time-frame determines how long the task takes to complete, though this may be reduced by 10 percent for every 10 full points of MoS the character scores on the test (see Margin of Success/Failure, p. 118). If a character fails on a Task action test roll, they work on the task for a minimum period equal to 10 percent of the timeframe for each 10 full points of MoF before realizing it's a failure. For Task actions with timeframes of one day or longer, it is assumed that the character only works eight hours per day. A character that works more hours per day may reduce the time accordingly. Characters working on Task actions may also interrupt their work to do something else and then pick up where they left off, unless the gamemaster rules that the action requires continuous and uninterrupted attention. Similar to taking the time (p. 117), a character may rush the job on a Task action, taking a penalty on the test in order to decrease the timeframe. The character must declare they are rushing the job before they roll the test. For every 10 percent they wish to reduce the timeframe, they incur a –10 modifier on the test (to a maximum reduction of 60 percent with a maximum modifier of –60).

\section{Defining your character}
\label{sec:defin-your-char}

In order to gauge and quantify what your character is merely good at and what they excel in—or what they are clueless about and suck at—Eclipse Phase uses a number of measurement factors: stats, skills, traits, and morphs. Each of these characteristics is recorded and tracked on your character's record sheet (p. 399).

\subsection{concept}
\label{sec:concept}

Your character concept defines who you are in the Eclipse Phase universe. You're not just a run-of-the-mill plebeian with a boring and mundane life, you're a participant in a post- apocalyptic transhuman future who gets caught up in intrigue, terrible danger, unspeakable horrors, and scrambling for survival. Much like a character in an adventure, drama, or horror story, you are a person to whom interesting things happen—or if not, you make them happen. This means your character needs a distinct personality and sense of identity. At the very least, you should be able to sum up your character concept in a single sentence, such as “cantankerous neotenic renegade archaeologist with anger management issues” or “thrill-seeking social animal who is dangerously obsessed with conspiracy theories and mysteries.” If it helps, you can always borrow ideas from characters you've seen in movies or books, modifying them to fit your tastes. Your character's concept will likely be influenced by two important factors: background and faction. Your background denotes the circumstances under which your character was raised, while your faction indicates the post-Fall grouping to which you most recently held ties and allegiances. Both of these play a role in character creation (p. 128).

\subsection{motivations}
\label{sec:motivations}

The clash of ideologies and memes is a core component of Eclipse Phase, and so every character has three motivations—personal memes that dominate the character's interests and pursuits. These memes may be as abstract as ideologies the character adheres to or supports—for example, social anarchism, Islamic jihad, or bioconservatism -— or they may be as concrete as specific outcomes the character desires, such as revealing a certain hypercorp's corruption, obtaining massive personal wealth, or winning victories for uplifted rights. A motivation may also be framed in opposition to something; for example, anti-capitalism or anti-pod-citizenship, or staying out of jail. In essence, these are ideas that motivate the character to do the things they do. Motivation is best noted as a term or short phrase on the character sheet, marked with a + (in favor of) or – (opposed to). Players are encouraged to develop their own distinct motivations for their characters, in cooperation with the gamemaster. Some examples are provided on p. 138. In game terms, motivation is used to help define the character's personality and influence their actions for roleplaying purposes. It also serves to regain Moxie points (p. 122) and earn Rez Points for character advancement (p. 152).

Motivational goals may be short-term or long-term, and may in fact change for a character over time. Short-term goals are more immediately obtainable objectives or short-lived interests, and these goals are likely to change once achieved. Even so, they should reflect intentions that will take more than one game session to reach, possibly covering weeks or months. These short-term goals may in fact tie directly into the gamemaster's current storyline. Examples include conducting a full analysis of an alien artifact, completing a research project, or living life as an uplifted dog for a while. Long-term goals reflect deeply rooted beliefs or tasks that require major efforts and time (possibly lifelong) to achieve. For example, finding the lost backup of a sibling missing since the Fall, overthrowing an autocratic regime, or making first contact with a new alien species. For purposes of awarding Moxie or Rez Points, long-term goals are best broken down into obtainable chunks. Someone whose goal is to track down the murderer who killed their parents when they were a child, for example, can be considered to achieve that goal every time they discover some evidence that brings them a little closer to solving the puzzle.

\subsection{Ego vs. morph}
\label{sec:ego-vs.-morph-1}

Eclipse Phase's setting dictates that a distinction must be made between a character's ego (their ingrained self, their personality, and inherent traits that perpetuate in continuity) and their morph (their ephemeral physical—and sometimes virtual—form). A character's morph may die while the character's ego lives on (assuming appropriate backup measures have been taken), transplanted into a new morph. Morphs are expendable, but your character's ego represents the ongoing, continuous life path of your character's mind, personality, memories, knowledge, and so forth. This continuity may be interrupted by an unexpected death (depending on how recent the backup was made), or by forking (see p. 273), but it represents the totality of the character's mental state and experiences.

Some aspects of your character—particularly skills, along with some stats and traits—belong to your character's ego, which means they stay with them throughout the character's development. Some stats and traits, however, are determined by morph, as noted, and so will change if your character leaves one body and takes on another. Morphs may also affect other skills and stats, as detailed in the morph description.

It is important that you keep ego- and morph-derived characteristics straight, especially when updating your character's record sheet.

\subsection{character stats}
\label{sec:character-stats}

Your character's stats measure several characteristics that are important to game play: Initiative, Speed, Durability, Wound Threshold, Lucidity, Trauma Threshold, and Moxie. Some of these stats are inherent to your character's ego, others are influenced or determined by morph.

\begin{quotation}
\textbf{Ego stats}
\begin{itemize}
\item Initiative
\item Lucidity
\item Trauma
\item Threshold
\item Insanity
\item Rating
\item Moxie
\end{itemize}
\end{quotation}

\begin{quotation}
\textbf{Morph stats}
\begin{itemize}
\item Speed
\item Durability
\item Wound
\item Threshold
\item Death
\item Rating
\item Damage
\item Bonus
\end{itemize}
\end{quotation}

\subsubsection{Initiative (init)}
\label{sec:initiative-init}

Your character's Initiative stat helps determine when they act in relation to other characters during the Action Turn (see Initiative, p. 188). Your Initiative stat is equal to your character's Intuition + Reflexes aptitudes (see Aptitudes, p. 123) multiplied by 2. Certain implants and other factors may modify this score.

\begin{quotation}
Lazaro's Intuition is 15 and his Reflexes score is 20. That means his Initiative is 70 (15 + 20 = 35, 35 x 2 = 70).
\end{quotation}

\subsubsection{Speed (spd)}
\label{sec:speed-spd}

The Speed stat determines how often your character gets to act in an Action Turn (see Initiative, p. 188). All characters start with a Speed stat of 1, meaning they act once per turn. Certain implants and other advantages may boost this up to a maximum of 4.

\subsubsection{Durability (dur)}
\label{sec:durability-dur}

Durability is your morph's physical health (or structural integrity in the case of synthetic shells, or system integrity in the case of infomorphs). It determines the amount of damage your morph can take before you are incapacitated or killed (see Physical Health, p. 206).

Durability is unlimited, though the range for baseline (unmodified) humans tends to fall between 20 and 60. Your Durability stat is determined by your morph.

\subsubsection{Wound threshold (wt)}
\label{sec:wound-threshold-wt}

A Wound Threshold is used to determine if you receive a wound each time you take physical damage (see Physical Health, p. 206). The higher the Wound Threshold, the more resistant to serious injury you are.

Wound Threshold is calculated by dividing Durability by 5 (rounding up).

\subsubsection{Death rating (dr)}
\label{sec:death-rating-dr}

Death Rating is the total amount of damage your morph can take before it is killed or destroyed beyond repair. Death Rating is equal to DUR x 1.5 for biomorphs and DUR x 2 for synthmorphs.

\begin{quotation}
Tyska is sleeved in a run-of-the-mill splicer morph with a Durability of 30. That gives him a Wound Threshold of 6 (30 / 5) and a Death Rating of 45 (30 x 1.5). If Tyska acquired an implant that boosted his Durability by +10 to 40, his Wound Threshold would be 8 (40 / 5) and his Death Rating would be 60 (40 x 1.5).
\end{quotation}

\subsubsection{Lucidity (luc)}
\label{sec:lucidity-luc}

Lucidity is similar to Durability, except that it measures mental health and state of mind rather than physical well-being. Your Lucidity determines how much stress (mental damage) you can take before you are incapacitated or driven insane (see Mental Health, p. 209).

Lucidity is unlimited, but generally ranges from 20 to 60 for baseline unmodified humans. Lucidity is determined by your Willpower aptitude x 2.

\subsubsection{Trauma threshold (tt)}
\label{sec:trauma-threshold-tt}

The Trauma Threshold determines if you suffer a trauma (mental wound) each time you take stress (see Mental Health, p. 209). A higher Trauma Threshold means that your mental state is more resilient against experiences that might inflict psychiatric disorders or other serious mental instabilities.

Trauma Threshold is calculated by dividing Lucidity by 5 (rounding up).

\subsubsection{Insanity rating (ir)}
\label{sec:insanity-rating-ir}

Your Insanity Rating is the total amount of stress your mind can take before you go permanently insane and are lost for good. Insanity Rating equals LUC x 2.

\begin{quotation}
Cole's Willpower is 16. That makes his Lucidity stat 32 (16 x 2), his Trauma Threshold 7 (32 / 5, rounded up), and his Insanity Rating 64 (32 x 2)
\end{quotation}

\subsubsection{Moxie}
\label{sec:moxie}

Moxie represents your character's inherent talent at facing down challenges and overcoming obstacles with spirited fervor. More than just luck, Moxie is your character's ability to run the edge and do what it takes, no matter the odds. Some people consider it the evolutionary trait that spurred humankind to pick up tools, expand our brains, and face the future head on, leaving other mammals in the dust. When the sky is falling, death is imminent, and no one can help you, Moxie is what saves the day.

The Moxie stat is rated between 1 and 10, as purchased during character creation (and perhaps raised later). In game play, Moxie is used to influence the odds in your favor. Every game session, your character begins with a number of Moxie points equal to their Moxie stat. Moxie points may be spent for any of the following effects:

\begin{itemize}
\item The character may ignore all modifiers that apply to a test. The Moxie point must be spent before dice are rolled.
\item The character may flip-flop a d100 roll result. For example, an 83 would become a 38.
\item The character may upgrade a success, making it a critical success, as if they rolled doubles. The character must succeed in the test before they spend the Moxie point.
\item The character may ignore a critical failure, treating it as a regular failure instead.
\item The character may go first in an Action Phase (p. 189).
\end{itemize}

Only 1 point of Moxie may be spent on a single roll. Moxie points will fluctuate during gameplay, as they are spent and sometimes regained.

Regaining Moxie: At the gamemaster's discretion, Moxie points may be refreshed up to the character's full Moxie stat any time the character rests for a significant period. Moxie points may also be regained if the character achieves a personal goal, as determined by their Motivation (see p. 121). The gamemaster determines how much Moxie is regained in proportion to the goal achieved.

\begin{quotation}
Audrey has a difficult Piloting: Aircraft roll to make. Her skill is 61, but she's facing a lot of modifiers (–30), and if she fails she's in big trouble. She could spend a point of Moxie before the test to ignore the modifiers, but she decides to take her chances against the target number of 31. Unfortunately, she rolls an 82. Luckily, she can spend a Moxie point to flip-flop that roll and make it a 28—a success!
\end{quotation}

\subsubsection{Damage bonus}
\label{sec:damage-bonus}

The Damage Bonus stat quantifies how much extra oomph your character is able to give their melee and thrown weapons attacks. Damage Bonus is determined by dividing your Somatics aptitude (see below) by 10 and rounding down.

\subsection{Character skills}
\label{sec:character-skills}

Skills represent your character's talents. Skills are broken down into aptitudes (ingrained abilities that everyone has) and learned skills (abilities and knowledge picked up over time). Skills determine the target number used for tests (see Making Tests, p. 115).

\subsubsection{Aptitudes}
\label{sec:aptitudes}

Aptitudes are the core skills that every character has by default. They are the foundation on which learned skills are built. Aptitudes are purchased during character creation and rate between 1 and 30, with 10 being average for a baseline unmodified human. They represent the ingrained characteristics and talents that your character has developed from birth and stick with you even when you change morphs—though some morphs may modify your aptitude ratings.

Each learned skill is linked to an aptitude. If a character doesn't have the skill necessary for a test, they may default to the aptitude instead (see Defaulting, p. 116).

There are 7 aptitudes in Eclipse Phase:

\begin{itemize}
\item \textbf{Cognition (COG)} is your aptitude for problem solving, logical analysis, and understanding. It also includes memory and recall.
\item \textbf{Coordination (COO)} is your skill at integrat ing the actions of different parts of your morph to produce smooth, successful movements. It includes manual dexterity, fine motor control, nimbleness, and balance.
\item \textbf{Intuition (INT)} is your skill at following your gut instincts and evaluating on the fly. It includes physical awareness, cleverness, and cunning.
\item \textbf{Reflexes (REF)} is your skill at acting quickly. This encompasses your reaction time, your gut-level response, and your ability to think fast.
\item \textbf{Savvy (SAV)} is your mental adaptability, social in tuition, and proficiency for interacting with others. It includes social awareness and manipulation.
\item \textbf{Somatics (SOM)} is your skill at pushing your morph to the best of its physical ability, including the fundamental utilization of the morph's strength, endurance, and sustained positioning and motion.
\item \textbf{Willpower (WIL)} is your skill for self-control, your ability to command your own destiny.
\end{itemize}

\subsubsection{Learned skills}
\label{sec:learned-skills}

Learned skills encompass a wide range of specialties and education, from combat training to negotiating to astrophysics (for a complete skill list, see p. 176). Learned skills range in rating from 1 to 99, with an average proficiency being 50. Each learned skill is linked to an aptitude, which represents the underlying competency in which the skill is based. When a learned skill is purchased (either during character generation or advancement), it is bought starting at the rating of the linked aptitude and then raised from there. If the linked aptitude is raised or modified, all skills built off it are modified appropriately as well.

Depending on your background and faction, you may receive some starting skills for free during character creation. Like aptitudes, learned skills stay with the character even when they change morphs, though certain morphs, implants, and other factors may sometimes modify your skill rating. If you lack a particular skill called for by a test, in most cases you can default to the linked aptitude for the test (see Defaulting, p. 116).

\subsubsection{Specializations}
\label{sec:specializations}

Specializations represent an area of concentration and focus in a particular learned skill. A character who learns a specialization is one who not only grasps the basic tenets of that skill, but they have trained hard to excel in one particular aspect of that skill's field. Specializations apply a +10 modifier when the character utilizes that skill in the area of specialization.

Specializations may be purchased during character creation or advancement for any existing skill the character possesses with a rating of 30 or more. Only one specialization may be purchased for each skill. Specific possible specializations are noted under individual the skill descriptions (see Skills, p. 170).

\begin{quotation}
Toljek has Palming skill of 63 with a specialization in Pickpocketing. Whenever he uses Palming to pick someone's pocket or otherwise steal from someone's person, his target number is 73, but for all other uses of Palming the standard 63 applies.
\end{quotation}

\subsection{Character traits}
\label{sec:character-traits}

Traits include a range of inherent qualities and features that help define your character. Some traits are positive, in that they give your character a bonus to certain stats, skills, or tests, or otherwise give them an edge in certain situations. Others are negative, in that they impair your abilities or occasionally create a glitch in your plans. Some traits apply to a character's ego, staying with them from body to body, while others only apply to a character's morph.

Traits are purchased during character generation. Positive traits cost customization points (CP), while negative traits give you extra CP to spend on other things (see Traits, p. 145). The maximum number of CP you may spend on traits is 50, while the maximum you may gain from negative traits is 50. In rare circumstances—and only with gamemaster approval—traits may be purchased, bought off, or inflicted during gameplay (see p. 153).

\subsection{Character morph}
\label{sec:character-morph}

In Eclipse Phase, your body is disposable. If your body gets old, sick, or too heavily damaged, you can digitize your consciousness and download it into a new body. The process isn't cheap or easy, but it offers effective immortality—as long as you remember to back yourself up and don't go insane. The term morph is used to describe any type of form your mind inhabits, whether it be a vat-grown clone sleeve, a synthetic robotic shell, a part-bio/part-flesh pod, or even the purely electronic software state of an infomorph.

You purchase your starting morph during character creation (see p. 128). This is likely the morph you were born with (assuming you were born), though it may simply be another morph you've moved onto.

Physical looks aside, your morph has a large impact on your characteristics. Your morph determines certain physical stats, such as Durability and Wound Threshold, and it may also influence Initiative and Speed. Morphs may also modify some of your aptitudes and learned skills. Some morphs come pre-loaded with specific traits and implants, representing how it was crafted, and you can always bling yourself out with more implants if you choose (see Implants, p. 126). All of these factors are noted in the individual morph descriptions (see p. 139).

If you plan on switching your current morph to another during gameplay, you may first want to back yourself up (see Backups and Uploads, p. 268). Backing up regularly is always a smart option in case you suffer an accidental or untimely death. Acquiring a new morph is not always easy, especially if you want it pre-loaded according to certain specifications. The full process is detailed under Resleeving, p. 271.

\subsubsection{Aptitude Maximums}
\label{sec:aptitude-maximums}

Every morph has an aptitude maximum, sometimes modified by traits. This maximum represents the highest value at which the character may use that aptitude while inhabiting that morph, reflecting an inherent limitation in some morphs. If a character's aptitude exceeds the aptitude maximum of their morph, they must use it at the maximum value for the duration of the time they remain in that morph. This may also affect the skills linked to that aptitude, which must be modified appropriately.

Some implants, gear, psi, and other factors may modify a character's natural aptitudes. These augmented values may exceed a morph's aptitude maximums, as they represent external factors boosting the morph's ability. No aptitude, however, augmented or not, may ever exceed a value of 40. Innate ability only takes a person so far—after that point, actual skill is what counts.

\begin{quotation}
Eva has a Cognition aptitude of 25. She is unfortunately forced to sleeve into a flat morph with an aptitude maximum of 20. For the duration of the period she inhabits that morph, her Cognition is reduced to 20, which also impacts all of her COG-linked skills, reducing them by 5.
\end{quotation}

\section{Things characters use}
\label{sec:things-char-use}

In the advanced technological setting of Eclipse Phase, characters don't get by on their wits and morphs alone; they take advantage of their credit and reputation to acquire gear and implants and use their social networks to gather information. Some characters also have the capability to use mental powers known as psi.

\subsection{Identity}
\label{sec:identity}

In an age of ubiquitous computing and omnipresent surveillance, privacy is a thing of the past—who you are and what you do is easily accessed online. Characters in Eclipse Phase, however, are often involved in secretive or less-than-legal activities, so the way to keep the bloggers, news, paparazzi, and law off your back is to make extensive use of fake IDs. While Firewall often provides covers for its sentinel agents, it doesn't hurt to keep a few extra personas in reserve, in case matters ever go out the airlock in a hurry. Thankfully, the patchwork allegiances of city-state habitats and faction stations means that identities aren't too difficult to fake, and the ability to switch morphs makes it even easier. On the other hand, anyone with a copy of your biometrics or geneprint is going to have an edge tracking you down or finding any forensic traces you leave behind (for more on ID, see p. 279).

\subsection{Social networks}
\label{sec:social-networks}

Social networks represent people the character knows and social groups with which they interact. These contacts, friends, and acquaintances are not just maintained in person, but also through heavy Mesh contact. Social software allows people to stay updated on what the people they know are doing, where they are, and what they are interested in, right up to the minute. Social networks also incorporate the online projects of individual members, whether it's a mesh-site loaded with a band member's songs, a personal archive of stored media, a decade of blog entries reviewing the best places to score cheap electronics, or a depository of research papers and studies someone has worked on or finds interesting.

In game play, social networks are quite useful to characters. Their friends list is an essential resource—a pool of people you can actively poll for ideas, troll for news, listen to for the latest rumors, buy or sell gear from, hit up for expert advice, and even ask for favors.

While a character's social networks are nebulous and constantly shifting, the use of them is not. A character takes advantage of their social networks via the Networking (Field) skill (p. 182). The exact use of this skill is covered under Reputation and Social Networks, p. 285.

\subsection{Cred}
\label{sec:cred}

The Fall devastated the global economies and currencies of the past. In the years of reconsolidation that followed, the hypercorps and governments inaugurated a new system-wide electronic monetary system. Called credit, this currency is backed by all of the large capitalist-oriented factions and is used to trade for goods and services as well for other financial transactions. Credit is mainly transferred electronically, though certified credit chips are also common (and favored for their anonymity). Hardcopy bills are even used in some habitats.

Depending on your background or faction, your character may be given an amount of credit at the start of the game. During game play, your character must earn credit the old-fashioned way: by earning or stealing it.

\subsection{Rep}
\label{sec:rep}

Capitalism is no longer the only economy in town. The development of nanofabricators allowed for the existence of post-scarcity economies, a fact eagerly taken advantage of by anarchist factions and others. When anyone can make anything, concepts like property and wealth become irrelevant. The advent of functional gift and communist economies, among other alternative economic models, means that in such systems you can acquire any goods or services you need via free exchange, reciprocity, or barter—presuming you are a contributing member of such a system and respected by your peers. Likewise, art, creativity, innovation, and various forms of cultural expression have a much higher worth than they do in capitalist economies.

In alternative economies, money is often meaningless, but reputation matters. Your reputation score represents your social capital—how esteemed you are to your peers. Rep can be increased by positively influencing, contributing to, or helping individuals or groups, and it can be decreased through antisocial behavior. In anarchist habitats, your likelihood of obtaining things that you need is entirely based on how you are viewed by others.

Reputation is easily measured by one of several online social networks. Your actions are rewarded or punished by those with whom you interact, who can ping your Rep score with positive or negative feedback. These networks are used by all of the factions, as reputation can affect your social activities in capitalist economies as well. The primary reputation networks include:

\begin{itemize}
\item \textbf{The @-list:} the Circle-A list for anarchists, Bar- soomians, Extropians, scum, and Titanians, noted as @-rep.
\item \textbf{CivicNet:} used by the Jovian Republic, Lunar- Lagrange Alliance, Morningstar Constellation, Planetary Consortium, and many hypercorps, referred to as c-rep.
\item \textbf{EcoWave:} used by nano-ecologists, preservation- ists, and reclaimers, referred to as e-rep.
\item \textbf{Fame:} the seen-and-be-seen network used by socialites, artists, glitterati, and media, referred to as f-rep.
\item \textbf{Guanxi:} used by the triads and numerous crimi- nal entities, referred to as g-rep.
\item \textbf{The Eye:} used by Firewall, noted as i-rep.
\item \textbf{RNA:} Research Network Affiliation, used by ar- gonauts, technologists, scientists, and researchers, referred to as r-rep.
\end{itemize}

Reputation is rated from 0-99. Depending on your background and faction, you may start with a Rep score in one or more networks. This can be bolstered through spending customization points during character creation. During game play, your Rep scores will depend entirely on your character's actions. For more information, see Reputation and Social Networks, p. 285.

Note that each Rep score is tied to a particular identity.

\subsection{Gear}
\label{sec:gear}

Gear is all of the equipment your character owns and keeps on their person, from weapons and armor to clothing and electronics. You buy gear for your character with customization points during character creation (see p. 136) and in the game with Credit or Rep. Certain restricted, illegal, or hard-to-find items may require special efforts to obtain (see Acquiring Gear, p. 298). If you have access to a nanofabricator, you may be able to simply build gear, presuming you have the proper blueprints (see Nanofabrication, p. 284). For a complete listing of equipment options, see the Gear chapter, p. 296.

Even among the remaining capitalist economies, prices can vary drastically. To represent this, all gear falls into a cost category. Each category defines a range of costs, so the gamemaster can adjust the prices of individual items as appropriate to the situation. Each category also lists an average price for that category, which is used during character generation and any time the gamemaster wants to keep costs simple. See the Gear Costs table on p. 137.

\subsection{Implants}
\label{sec:implants}

Implants include cybernetic, bionic, genetech, and nanoware enhancements to your character's morph (or mechanical enhancements in the case of a synthetic shell). These implants may give your character special abilities or modify their stats, skills, or traits. Some morphs come pre-equipped with implants, as noted in their descriptions (see p. 139). You may also special- order morphs with specific implants (see Morph Acquisition, p. 277). If you want to upgrade a morph you are already in, you can undergo surgery or other treatments to have an enhancement installed (see Healing Vats, p. 326. For a complete list of available implant/enhancement options, see pp. 300-311, Gear.

\subsection{Psi}
\label{sec:psi}

Psi is a rare and anomalous set of mental abilities that are acquired due to infection by a strange nanovirus released during the Fall. Psi abilities are not completely understood, but they give characters certain advantages—as well as some disadvantages. A character requires the Psi trait (p. 147) to use psi abilities, which are called sleights. Psi users are called asyncs. A full explanation of psi and details on the various sleights can be found in the Mind Hacks chapter, p. 216.

 \subsection{Game rules summary}
\label{sec:game-rules-summary}

Everything you need to know about the rules—summed up on a single page.

\subsubsection{Making tests (P. 115)}

\begin{itemize}
\item Roll d100 (two ten-sided dice, read as a percentile amount, from 00 to 99).
\item Target number is determined by the appropriate skill (or occasionally an aptitude).
\item Difficulty is represented by modifiers.
\item 00 is always a success.
\item 99 is always a failure.
\item Margin of Success of 30+ is an Excellent Success.
\item Margin of Failure of 30+ is a Severe Failure.
\item A roll of doubles (00, 11, 22, 33, etc.) equals a critical success or failure.
\end{itemize}

\subsubsection{Success test (P. 117)}

\begin{itemize}
\item To succeed, roll d100 and score equal to or less than the skill +/– modifiers.
\end{itemize}

\subsubsection{Opposed test (P. 119)}

\begin{itemize}
\item Each character rolls d100 against their skill +/– modifiers.
\item The character who succeeds with the highest roll wins. If both characters fail, or both succeed but tie, dead- lock occurs.
\end{itemize}

\subsubsection{Simple success test (P. 118)}

\begin{itemize}
\item Simple Success Tests automatically succeed.
\item Success or failure on the roll simply indicates if the character succeeded strongly or poorly.
\end{itemize}

\subsubsection{Defaulting (P. 116)}

\begin{itemize}
\item If a character does not have the appropriate skill for a test, they may default to the skill’s linked aptitude.
\end{itemize}

\subsubsection{Modifiers (P. 115)}

\begin{itemize}
\item Modifiers always affect the target number (skill), not the roll.
\item Modifiers (positive or negative) come in 3 levels of severity: \begin{itemize}
\item Minor (+/–10)
\item Moderate (+/–20)
\item Major (+/–30)
\end{itemize}
\item The maximum modifiers that can be applied are +/– 60.
\end{itemize}

\subsubsection{teamwork (P. 117)}

\begin{itemize}
\item One character is chosen as the primary actor; they make the test.
\item Each helper character adds a +10 modifier (max. +30).
\end{itemize}

\subsubsection{Taking the time (P. 118)}

\begin{itemize}
\item Character may take extra time to complete an action.
\item On Complex actions, each minute taken adds +10 to the test.
\item On Task actions, every 50 percent extension to the timeframe adds +10 to the test.
\end{itemize}

\subsubsection{Aptitudes (P. 123)}

\begin{itemize}
\item Aptitudes range from 1 to 30 (average 15).
\item Aptitudes are: Cognition, Coordination, Intuition, Reflexes, Savvy, Somatics, and Willpower.
\end{itemize}

\subsubsection{Learned skills (P. 123)}

\begin{itemize}
\item Skills range from 1-99 (average 50).
\item Each skill is linked to and based on an aptitude.
\item Morphs, gear, drugs, etc. may provide skill bonuses or penalties to individual skills.
\end{itemize}

\subsubsection{Specializations (P. 123)}

\begin{itemize}
\item Specializations add +10 when using a skill for that area of concentration.
\item Each skill may have only one specialization.
\end{itemize}

\subsubsection{Action turns (P. 120)}

\begin{itemize}
\item Action Turns are 3 seconds in length.
\item The order in which characters act is determined by Initiative.
\item Automatic actions are always "on."
\item Characters may take any number of Quick Actions in a Turn (minimum of 3), limited only by the gamemaster.
\item Characters may only take a number of Complex Actions equal to their Speed stat.
\end{itemize}

\subsubsection{Task actions (P. 120)}

\begin{itemize}
\item Task Actions are any action that requires longer than 1 Action Turn to complete.
\item Task Actions list a timeframe (anywhere from 2 Turns to 2 years).
\item Timeframe reduced by 10\% for each 10 points of MoS.
\item If character fails, they work on the task for a minimum period equal to 10\% of the timeframe for each 10 points of MoF before realizing it's a failure.
\end{itemize}

%%% Local Variables: %%% mode: latex %%% TeX-master: "ep" %%% End: 
