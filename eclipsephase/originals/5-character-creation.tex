\chapter{Character creation}
\label{chap:character-creation}

\section{Character generation}
\label{sec:character-creation}

There are two parts to every player character. The first is the sets of numbers and attributes that define what a character is good or bad at (or even what they can and can’t do) according to the game mechanics. These are more than just statistics, however—these characteristics help to define your character’s abilities and interests, and by extension their background, education, training, and experience. During the character creation process, you will have the ability to assign, adjust, and juggle these characteristics as you like. If you have a pre-conceived notion of what the character is about, you can optimize the stats to reflect that. Alternatively, you can tweak the stats until you get something you like, then base the character’s backstory off of what you develop.

The second part to every player character is their personality. What defines them as a person? What makes them tick? What pisses them off? What sparks their interest? What positive aspects of their personality make them appealing as a friend, comrade, or lover—or at least someone interesting to play? What character flaws and quirks do they have? These questions matter because they will also guide you as you assign stats, skills, and traits.

Character generation is a step-by-step process. Unlike some games, the process for creating an Eclipse Phase character is not random—you have complete control over every aspect of your character’s design. Some stages must be completed before you can move on to others. The complete process is broken down on the Step-By-Step Guide to Character Creation sidebar.

\subsection{Step-by-step guide to character creation}

\begin{enumerate}
\item Define Character Concept (p. 130)
\item Choose Background (p. 131)
\item Choose Faction (p. 132)
\item Spend Free Points (p. 134) 
\begin{itemize} 
\item 105 aptitude points 
\item 1 Moxie 
\item 5,000 credit 
\item 50 Rep 
\item Native tongue 
\end{itemize}
\item Spend Customization Points (p. 135) 
\begin{itemize} 
\item 1,000 CP to spend 
\begin{itemize} 
\item 15 CP = 1 Moxie 
\item 10 CP = 1 aptitude point 
\item 5 CP = 1 psi sleight 
\item 5 CP = 1 specialization 
\item 2 CP = 1 skill point (61-80) 
\item 1 CP = 1 skill point (up to 60) 
\item 1 CP = 1,000 credit 
\item 1 CP = 10 rep 
\end{itemize} 
\item Active skill minimum: 400 skill points 
\item Knowledge skill minimum: 300 skill points 
\item Choose Starting Morph (pp. 136 and 139) 
\item Choose Traits (pp. 136 and 145) 
\end{itemize}
\item Purchase Gear (p. 136)
\item Choose Motivation (p. 137)
\item Calculate Remaining Stats (p. 138)
\item Detail the Character (p. 138)
\end{enumerate}

\subsection{Character concept}
\label{character-concept}

Deciding what/who you want to play before you make the character is usually the best route. Pick a simple archetype that fits your character, and work from there. Do you want to play an explorer? Someone sneaky, like a spy or thief? Someone cerebral, like a scientist? A hardened criminal or ex-cop? Or do you prefer to be a rabble-rousing agitator? You can also start with a personality type and choose an associated profession. If you want a social butterfly who excels at manipulating people, you can play a media personality, blogger, or party-going socialite. Perhaps you’d prefer a bottomedout reject with substance abuse problems, in which case an ex-merc or former hypercapitalist who lost his fortune and family during the Fall might fit. How about an energetic, live-life-to-the-fullest, must-see-it-all character? Then a habitat freerunner or professional gatecrasher might be what you’re looking for.

Make sure to check in with the other players and try to create a character that’s complementary to the rest of the team—preferably one who provides some skill-set the group lacks. Why create a research archeologist if someone else is already set on playing one, especially when the team lacks a good combat specialist or async? On the other hand, if your team is going to be running an alien archeological expedition, then having more than one researcher (each with distinct areas of expertise) might not be bad.

Once you have the basic concept, try to fill it with a few more details, making it into a one-sentence summary. If you started with the concept of “xeno-so- ciologist,” expand it to “open-minded amateur linguist and expert xeno-sociologist who is fascinated by alien cultures, collects Factor kitsch, has a high-tolerance for ‘yuck factors,’ and whose best friends tend to be uplifts and AIs.” This will give you a few more details around which you can focus the character’s strengths and weaknesses.

\subsection{Choose background}
\label{choose-background}

The first step to creating your character is to choose a background. Was your character born on Earth before the Fall? Were they raised on a habitat commune? Or did they start existence as a disembodied AI?

You must choose one of the backgrounds for your character from the list below. Choose wisely, as each background may provide your character with certain skills, traits, limitations, or other characteristics to start with. Keep in mind that your background is where you came from, not who you are now. It is the past, whereas your faction represents whom your character is currently aligned with. Your future, of course, is yours to make.

The background options presented below cover a wide selection of transhumanity, but they cannot cover every possibility. If your gamemaster allows it, you may work with them to develop a background that is not included on this list, using these as guidelines to keep it balanced.

\subsubsection{Drifters}
\label{sec:drifters}

You were raised with a social grouping that remained on the move throughout the Sol system. This could have been free traders, pirates, asteroid farmers, scav- engers, or just migrant workers. You are used to roaming space travel between habitats and stations. 

Advantages: +10 Navigation skill, +20 Pilot: Spacecraft skill, +10 Networking: [Field] skill of your choice

Disadvantages: None

Common Morphs: All, especially Bouncers and Hibernoids

\subsubsection{Fall evacuee}
\label{sec:fall-evacuee}

You were born and raised on Earth and evacuated during the horrors of the Fall, leaving your old life (and possibly your friends, family, and loved ones) behind you. You were lucky enough to survive with your body intact and continue to make a life for yourself out in the system.

Advantages: +10 Pilot: Groundcraft skill, +10 Networking: [Field] skill of your choice, +1 Moxie

Disadvantages: Only 2,500 Starting Credit (can still buy credit with CP)

Common Morphs: Flats, Splicers

\subsubsection{Hyperelite}
\label{sec:hyperelite}

You are privileged to have been raised as part of the immortal upper class that rules many inner system habitats and hypercorps. You were pampered with wealth and influence that most people can only dream of.

Advantages: +10 Protocol skill, +10,000 Credit, +20 Networking: Hypercorps skill

Disadvantages: May not start with flat, splicer, or any pod, uplift, or synthetic morphs

Common Morphs: Exalts, Sylphs

\subsubsection{Infolife} 
\label{sec:infolife}

You entered existence as a digital consciousness— an artificial general intelligence (AGI). Your very existence is illegal in certain habitats (a legacy of those who place the Fall at the feet of rampant AIs). Unlike the seed AIs responsible for their Fall, your capacity for self-improvement is limited, though you do have full autonomy. 

Advantages: +30 Interfacing skill, Computer skills (Infosec, Interfacing, Programming, Research) bought with Customization Points are half price 

Disadvantages: Real World Naiveté trait, Social Stigma (AGI) trait, may not purchase Psi trait, Social skills bought with Customization Points are double price 

Common Morphs: Infomorphs, synthetic morphs

\subsubsection{Isolate}
\label{sec:isolite}

You were raised as part of a self-exiled grouping on the fringes of the system. Whether raised as part of a religious group, cult, social experiment, anti-tech cell, or a group that just wanted to be isolated, you spent most if not all of your upbringing isolated from other factions.

Advantages: +20 to two skills of your choice 

Disadvantages: -10 starting Rep 

Common Morphs: All

\subsubsection{Lost}
\label{sec:lost}

You are a legacy of one of the most infamous debacles since the Fall. As a member of the “Lost generation,” you went through an accelerated-growth childhood, somehow surviving where others of your kind died, went insane, or were persecuted (see The Lost, p. 233). Your background is a social stigma, but it does provide you with certain advantages ... and burdens. 

Advantages: +20 to two Knowledge skills of your choice, Psi trait 

Disadvantages: Mental Disorder (choose two) trait, Social Stigma (Lost) trait, must start with Futura morph

Common Morphs: Futuras

\subsubsection{Lunar Colonist}
\label{sec:lunar-colonist}

You experienced your childhood in one of the cramped dome cities or underground stations on Luna, Earth’s moon. You had a ringside seat to the Fall of Earth.

Advantages: +10 Pilot: Groundcraft skill, +10 to one Technical, Academic: [Field], or Profession: [Field] skill of your choice, +20 Networking: Hypercorps skill

Disadvantages: None

Common Morphs: Flats, Splicers

% %%% txt/134.txt

\subsubsection{Martian} % MARTIAN
\label{sec:martian}

You were raised in one of the stations on or above Mars, now the most populated planet in the system. Your home town may or may not have survived the Fall. 

Advantages: +10 Pilot: Groundcraft skill, +10 to one Technical, Academic: [Field], or Profession: [Field] skill of your choice, +20 Networking: Hypercorps skill 

Disadvantages: None

Common Morphs: Flats, Splicers, and Rusters

\subsubsection{Original space colonist} % ORIGINAL SPACE COLONIST
\label{sec:original-space-colonist}

You, or your parents, were part of the first “generations” of colonists/workers sent out from Earth to stake a claim in space, so you are familiar with the cramped confines of spaceflight and life aboard older stations and habitats. As a “zero-one G” (zero-gravity, first-gen), you were never part of the elite. People from your background typically have some sort of specialized tech training as vacworkers or habtechs.

Advantages: +10 Pilot: Spacecraft or Freefall skill, +10 to a Technical, Academic: [Field], or Profession: [Field] skill of your choice, +20 to a Networking: [Field] skill of your choice

Disadvantages: None

Common Morphs: All. Use of exotic morphs is common.

\subsubsection{Re-instantiated}
\label{sec:re-instantiated}

You were born and raised on Earth, but you did not survive the Fall. All that you know is that your body died there, but your backup was transmitted off-world, and you were one of the lucky few to be re-instantiated with a new morph. You may have spent years in dead storage, simulspace, or as an infomorph slave.

Advantages: +10 Pilot: Groundcraft skill, +10 to a Networking: [Field] skill of your choice, +2 Moxie

Disadvantages: Edited Memories trait, 0 Starting Credit (can still buy credit with CP)

Common Morphs: Cases, Infomorphs, Synths

\subsubsection{Scumborn}
\label{sec:scumborn}

You were raised in the nomadic and chaotic lifestyle common to Scum barges. Advantages: +10 Persuasion or Deception skill, +10 Scrounging skill, +20 Networking: Autonomists skill Disadvantages: None

Common Morphs: All, especially Bouncers

\subsubsection{Uplift}
\label{sec:uplift}

You are not even human. You were born as an uplifted animal: chimpanzee, gorilla, orangutan, parrot, raven, crow, or octopus.

Advantages: +10 Fray skill, +10 Perception skill, +20 to two Knowledge skills of your choice

Disadvantages: Must choose an uplift morph to start

Common Morphs: Neo-Avian, Neo-Hominid, Octomorph

\subsection{Choose Faction}
\label{sec:choose-faction}

After choosing your background, you now choose which primary faction your character belongs to. This faction most likely represents the grouping that controls your character’s current home habitat/station, and to which your character holds allegiance, but this need not be the case. You may be a dissident member of your faction, living among them but opposing some (or all) of their core memes and perhaps agitating for change. Whatever the case, your faction defines how your character represents themself in the struggle between ideologies post-Fall.

You must choose one of the factions listed below. Like your character’s background, it will provide your character with certain skills, traits, limitations, or other characteristics.

The factions presented here outline the most numerous and influential among transhumanity, but others may also exist. At your gamemaster’s discretion, you may develop another starting faction with them not included on this list.

\subsubsection{Anarchist}
\label{sec:anarchist}

You are opposed to hierarchy, favoring flat forms of social organization and directly democratic decisionmaking. You believe power is always corrupting and everyone should have a say in the decisions that affect their lives. According to the primitive and restrictive policies of the Inner system and Jovian Junta, this makes you an irresponsible hoodlum at best and a terrorist at worst. In your opinion, that’s comedy coming from governments that keep their populations in line with economic oppression and threats of violence.

Advantages: +10 to a skill of your choice, +30 Networking: Autonomists skill

Disadvatages: None

Common Morphs: All

\subsubsection{Argonaut}
\label{sec:argonaut}

You are part of a scientific techno-progressive movement that seeks to solve transhumanity’s injustices and inequalities with technology. You support universal access to technology and healthcare, open source models of production, morphological freedom, and democratization. You try to avoid factionalism and divisive politics, seeing transhumanity’s splintering as a hindrance to its perpetuation.

Advantages: +10 to two Technical, Academic: [Field], or Profession: [Field] skills; +20 Networking: Scientists

Disadvatages: None

Common Morphs: All

\subsubsection{Barsoomian}
\label{sec:barsoomian}

You call the Martian outback and wilds your home. You are a “redneck,” a lower-class Martian from the rural areas that often find themselves in conflict with the policies and goals of the hypercorp domes and Tharsis League. %%txt/135.txt

Advantages: +10 Freerunning, +10 to one skill of your choice, +20 Networking: Autonomists skill

Disadvatages: None

Common Morphs: Cases, Flats, Rusters, Splicers, Synths

\subsubsection{Brinker}
\label{sec:brinker}

You or your faction is reluctant to deal with the rest of transhumanity and the various goings-on in the rest of system. Your particular grouping may have sought out imposed isolation, to pursue their own interests, or they have been exiled for their unpopular beliefs. Or you simply be a loner who prefers the vast emptiness of sp to socializing with others. You might be a religious culti primitivist, a utopian, or something altogether unintere in transhumanity.

Advantages: +10 Pilot: Spacecraft skill, +10 to a skill of choice, +20 to a Networking: [Field] skill of your choice

Disadvatages: None

Common Morphs: All

\subsubsection{Criminal}
\label{sec:criminal}

You are involved with the crime-oriented underworld. may work with one of the Sol system’s major criminal tions—triads, the Night Cartel, the ID Crew, Nine Lives, Familae—or one of the smaller, local operators with a stake in a specific habitat. You might be a vetted mem for-life, a reluctant recruit, or just a freelancer looking the next gig.

Advantages: +10 Intimidation skill, +30 Networking: Criminal skill

Disadvatages: None

Common Morphs: All

\subsubsection{Extropian}
\label{sec:extropian}

You are an anarchistic supporter of the free market and private property. You oppose government and favor a system where security and legal matters are handled by private competitors. Whether you consider yourself an anarcho-capitalist or a mutualist (a difference only other Extropians can figure out), you occupy a middle-ground between the hypercorps and autonomists, dealing with both and yet trusted by neither.

Advantages: +10 Persuasion skill, +20 Networking: Autonomists skill, +10 Networking: Hypercorps skill

Disadvatages: None

Common Morphs: All

\subsubsection{Hypercorp}
\label{sec:hypercorp}

You hail from a habitat controlled by the hypercorps. You might be a hypercapitalist entrepeneur, a hedonistic socialite, or a lowly vacworker, but you accept that certain liberties must be sacrificed for security and freedom.

Advantages: +10 Protocol skill, +20 Networking: Hypercorps skill, +10 to any Networking: [Field] skill

Disadvatages: None

Common Morphs: Exalts, Olympians, Splicers, Sylphs %%txt/136.txt

\subsubsection{Jovian}
\label{sec:jovian}

Your faction is noted for its authoritarian regime, bio-conservative ideologies, and militaristic tendencies. Where you come from, technology is not to be trusted to everyone and humans need to be protected from themselves. To ensure its survival, humanity must be able to defend itself, and unfettered growth must be checked.

Advantages: +10 to two weapon skills of your choice, +10 Fray, +20 Networking: Hypercorps skill

Disadvatages: Must start with a Flat or Splicer morph, may not start with any nanoware or advanced nanotech

Common Morphs: Flats and Splicers

\subsubsection{Lunar}
\label{sec:lunar}

You hail from Luna, the original off-Earth colony world. Now overpopulated and in decline, Luna is one of the few places where people still cling to old-Earth ethnic and national identities. Your home is also within sight of Earth, a constant reminder that encourages many “Loonies” to be Reclaimers, deploring the hypercorp interdiction and arguing that you have a right to return to Earth, terraform it, and re-establish it as a living homeworld.

Advantages: +10 to one Language: [Field] of your choice, +20 Networking: Hypercorps skill, +10 Networking: Ecologists skill

Disadvatages: None

Common Morphs: Cases, Exalts, Flats, Splicers, Synths

\subsubsection{Mercurial}
\label{sec:mercurial}

Your faction has no interest in co-opting their true natures in order to become more “human.” You might be an AGI that does not necessarily intertwine its destiny with transhumanity, or an uplift that seeks to preserve and promote non-human life (or at least your own species). You might even be an infomorph or posthuman who has strayed so far from transhuman interests and values that you now consider yourself to be forging a unique new path of life.

Advantages: +10 to any two skills of your choice, +20 to a Networking: [Field] skill of your choice

Disadvatages: None

Common Morphs: Infomorphs, Synths, uplift morphs

\subsubsection{Scum}
\label{sec:background-scum}

This is the future we’ve all been waiting for, and you’re going to enjoy it to the max. A paradigm shift has occurred, and while everyone else is catching up, your faction embraces and revels in it. There is no more want, no more death, no more limits on what you can be. The scum have immersed themselves in a new way of life, changing themselves as they see fit, trying out new experiences, and pushing the boundaries wherever they can ... and fuck anyone who can’t deal with that.

Advantages: +10 Freefall skill, +10 to a skill of your choice, +20 Networking: Autonomists skill

Disadvatages: None

Common Morphs: All

\subsubsection{Socialite}
\label{sec:socialite}
You are a member of the inner system glitterati, the media-saturated social cliques that set trends, spread memes, and make or break lives with whispers, innuendo, and backroom deals. You are simultaneously an icon and a devout follower. Culture isn’t just your life, it’s your weapon of choice.

Advantages: +10 Persuasion skill, +10 Protocol skill, +20 Networking: Media skill

Disadvatages: May not start with flat, pod, uplift, or synthetic morphs

Common Morphs: Exalts, Olympians, Sylphs

\subsubsection{Titanian}
\label{sec:titanian}

You are a participant in the Titanian Commonwealth’s socialist cyberdemocracy. Unlike other autonomist projects, Titanian joint efforts have assembled some impressive infrastructural projects as approved by the Titanian Plurality and pursued by state-owned microcorps.

Advantages: +10 to two Technical or Academic skills of your choice, +20 Networking: Autonomists skill

Disadvatages: None

Common Morphs: All

\subsubsection{Ultimate}
\label{sec:ultimate}

Your faction sees the potential in transhumanity’s future and looks back upon the rest of transhumanity as weak and hedonistic. Transhumanity is set to take the next evolutionary step and it’s time for transhumans to be redesigned to the best of our capabilities.

Advantages: +10 to two skills of your choice, +20 to a Networking: [Field] skill of your choice

Disadvatages: May not start with Flat, Splicer, uplift, or pod morphs

Common Morphs: Exalts, Remades

\subsubsection{Venusian}
\label{sec:venusian}

You are a supporter of the Morningstar Confederation of Venusian aerostats, resentful of the growing influence of the Planetary Consortium and other entrenched and conservative inner system powers. You see your faction’s ascension as a chance to reform the old guard ways of inner system politics.

Advantages: +10 Pilot: Aircraft, +10 to one skill of your choice, +20 Networking: Hypercorps skill

Disadvatages: None

Common Morphs: Cases, Exalts, Mentons, Splicers, Sylphs, Synths

\subsection{Spend Free Points}
\label{sec:spend-free-points}

Each starting character receives an equal number of free points for things like rep and aptitudes. These free points are just the start for building your character, so don’t fret if you can’t get certain scores as high as you like. In the next stage of character creation, you will gain additional points with which you can customize your character (see Spend Customization Points, p. 135).

%%% txt/137.txt

\begin{quotation}
Example

Tai is making a character. She decides to create a salvage retrieval/scavenger type who started as a Lunar Colonist but is now a Brinker. Together, her background and faction give Tai +20 Networking: Autonomists skill, +20 Networking: Hypercorps skill, +10 Pilot: Spacecraft skill, and +10 Pilot: Groundcraft skill. She also has +10 to two other skills (one Academic, Professional, or Technical) that she’ll choose later. Tai starts with 105 points for aptitudes, which works out to 15 each. She wants her character to be impulsive and antisocial, so right away she lowers both SAV and WIL to 10. She also wants to be smart and fast on her feet, so takes the extra 10 points that gives her and raises both COG and REF to 20. So her aptitudes are:

\begin{center} 
\begin{tabular}{ccccccc}
COG & COO & INT & REF & SAV & SOM & WIL \\
 
20 & 15 & 15 & 20 & 10 & 15 & 10 \\

\end{tabular} 
\end{center}

She marks down her Moxie of 1 and gets her native language (Chinese) at 85, both for free. Noting her 5,000 Credits, Tai divides her Rep score points evenly among @-rep and c-rep, taking 25 in each.

Tai now has 1,000 points to customize. She wants to be lucky, so she starts right off spending 60 (4 x 15) CP to raise her Moxie from 1 to 5. She also decides that she wants her character to be better at spotting things, so she raises her INT from 15 to 20, at a cost of 50 CP (5 x 10). So far, she’s spent 110 CP. She must buy at least 400 points of Active skills, so she tackles that next. She knows that skills are based on aptitudes and they get more expensive over 60, so she decides the most she’ll spend on any single skill is 40 (since her highest aptitude is 20). She picks out her skills, assigns the points, and adds them to the starting aptitudes. This is what she starts with, noting the points she spent on each and the total value (counting aptitude) in parentheses. Beam Weapons (COO) 30 (45), Climbing (SOM) 30 (45), Demolitions (COG) 40 (60), Fray (REF) 30 (50), Freefall (REF) 40 (60), Freerunning (SOM) 30 (45), Hardware: Aerospace (COG) 40 (60), Infiltration (COO) 30 (45), Interfacing (COG) 20 (40), Navigation (INT) 40 (60), Perception (INT) 40 (60), Persuasion (SAV) 20 (30), Research (COG) 20 (40), and Scrounging (INT) 40 (60). This costs her 450 CP, so she’s spent a total of 560 CP so far. Now she spends her 300 points of Knowledge skills: Academics: Astrophysics (COG) 40 (60), Academics: Engineering (COG) 40 (60), Academics: Fall History (COG) 40 (60), Art: Sculpture (INT) 40 (60), Interest: Brinker Stations (COG) 40 (60), Interest: Conspiracies (COG) 30 (50), Language: English (INT) 40 (60), Profes- sion: Appraisal (COG) 40 (60), Profession: Scavenger Trade (COG) 40 (60).

This costs her another 350 CP, bringing her total spent CP to 910.

Adding in her background and faction skills, she also has Networking: Autonomists (SAV) 30, Networking: Hypercorps (SAV) 30, Pilot: Spacecraft (REF) 30 (50), Pilot: Groundcraft (REF) 30 (50). She takes the freebie +10 and adds it to Fray (raising it to 60) and applies the other +10 to Academics: Economics (COG) 30.

With 90 CP left, Tai moves on to Rep. Tai wants to have a lot of good connections, so she raises both of her Rep scores by 30 points each, at a cost of 6 CP. She also decides she needs some credibility with criminal types, so she buys g-rep at 40, for 4 more CP. Now she has 80 CP left.

Tai’s character needs a body, and she decides a bouncer is most suited for the nomadic, spacefaring lifestyle of her brinker. That costs another 40 CP, leaving her with 50 CP left to spend.

Looking back at her skills, she decides to raise her Pilot: Spacecraft from 50 to 70. It costs her 10 CP to raise the skill to 60, and another 20 CP to raise it from 60 to 70, for a total cost of 30 CP. She also wants to raise her Scrounging from 60 to 70, for a 20 CP cost. That nicely uses up the last of her CP.

Scanning the traits, though, Tai also decides that Situational Awareness would be a good choice for her scavenger. At a cost of 10 CP, she would need to take another negative trait to compensate. She chooses Neural Damage (synaesthesia)—a condition she inherited from a rampaging nanovirus during the Fall.

Tai’s points are now all evened out and spent.
\end{quotation}

\subsubsection{Starting Aptitudes}
\label{sec:starting-aptitudes}

Your character receives 105 free points to distribute among their 7 aptitudes: Cognition, Coordination, Intuition, Reflexes, Savvy, Somatics, and Willpower (see Aptitudes, p. 123). (That breaks down to an average of 15 per aptitude, so it may be easiest to give each 15 and then adjust accordingly, raising some and lowering others.) Each aptitude must be given at least 5 points (unless you take the Feeble trait, see p. 149), and no aptitude may be raised higher than 30 (unless you take the Exceptional Aptitude trait, p. 146). Note that certain morphs (flats and splicers, for example) may also put a cap on how high your aptitudes may be (see Aptitude Maximums, p. 124).

For simplicity, it is recommended that aptitude scores be handled as multiples of 5, but this is not a necessity.

\subsubsection{Native Tongue}
\label{sec:native-tongue}

Every character receives their natural Language skill at a rating of 70 + INT for free. This skill may be raised with CP (see below).

\subsubsection{Starting Moxie}
\label{sec:starting-moxie}

Every character starts off with a Moxie stat of 1 (see Moxie, p. 122).

\subsubsection{Credit}
\label{sec:starting-credit}

All characters are given 5,000 credits with which to purchase gear during character creation, unless you have the Fall Evacuee or Re-instantiated background (in which case you start with 2,500 or 0 credits, re- spectively). See Purchasing Gear, p. 136, for more details.

\subsubsection{Rep}
\label{sec:starting-rep}
Your character isn’t a complete newbie. You get 50 rep points to divide between the reputation networks of your choice (see Reputation and Social Networks, p. 285).

\subsection{Spend Customization Points} 
\label{sec:spend-customization-points}
Now that you have the basics of your character fleshed out, you can spend additional Customization Points (CP) to fine-tune your character. Each character is given 1,000 CP, which may be used to increase aptitudes, buy skills, acquire more Moxie, buy more credit, elevate your Rep, or purchase positive traits. You may also take on negative traits in order to get even more CP with which to customize your character. This customization process should be used to tweak your character and specialize them in the ways you desire.

If a gamemaster seeks a different level of gameplay, they can adjust this CP amount. If the gamemaster wants a scenario where the starting characters are younger or less experienced, they can lower the CP to 800 or even 700. On the other hand, if you want to create characters who start off as grizzled veterans, you can raise the CP to 1,100 or even 1,200.

Not all customizations are equal - aptitudes, for example, are considerably more valuable than individual skills. To reflect this, CP must be spent at a specific ratio according to the specific boost desired.

\begin{quotation}
Customization Points 
\begin{itemize} 
\item 15 CP = 1 Moxie point 
\item 10 CP = 1 aptitude point 
\item 5 CP = 1 psi sleight 
\item 5 CP = 1 specialization 
\item 2 CP = 1 skill point (61-80) 
\item 1 CP = 1 skill point (up to 60) 
\item 1 CP = 1,000 credit 
\item 1 CP = 10 Rep
\end{itemize} \textit{Trait and morph costs vary as noted.}
\end{quotation}

\subsubsection{Customizing Aptitudes}
\label{sec:customizing-aptitudes}

Raising your aptitude score is quite expensive at 10 CP per aptitude point. As noted above, no aptitude may be increased above 30. Keep in mind that your morph may also provide certain aptitude bonuses.

\subsubsection{Increasing Moxie}
\label{sec:increasing-moxie}

Moxie may be raised at the cost of 15 CP per Moxie point. The maximum to which Moxie may be raised is 10. %%% txt/138.txt

\subsubsection{Learned Skills}
\label{sec:buying-learned-skills}

Each character must purchase a minimum of 400 skill points of Active skills and 300 skill points of Knowledge skills (see Skills, p. 170). Skills are bought at the cost of 1 CP per point. Keep in mind that learned skills start at the rating of the linked aptitude. For example, if you want to raise a skill to 30 and the skill’s linked aptitude is 10, you’ll need to spend 20 CP. Skill bonuses from background or faction should also be applied to the rating before you start raising the skill. For simplicity, it is recommended that skills be purchased as multiples of 5, but this is not a necessity. Raising a skill over 60 is expensive. Each point over 60 costs double. Raising a skill with a linked attribute of 20 up to 70 would cost 60 CP: 40 points to get from 20 to 60, and 20 more points to get from 60 to 70. No learned skill may be raised over 80 during character creation (unless you have the Expert trait, p. 146). Though Knowledge skills are grouped into 5 skills, each is a field skill (p. 172), meaning that it can be taken multiple times with different fields. A complete list of skills can be found on p. 176.

\subsubsection{Specializations}
\label{sec:buying-specializations}

Specializations (p. 173) may also be purchased at the cost of 5 CP per specialization. You may purchase specializations for both Active and Knowledge skills. Only 1 specialization may be purchased per skill, and they may only be bought for skills with a rating of 30+.

\subsubsection{Buying More Credit}
\label{sec:buying-credit}

If you want more cred to spend on gear, every CP will get you 1,000 credits. See Purchase Gear, p. 136, for details on buying stuff. The maximum CP you can spend on additional credits is 100.

\subsubsection{Increasing Rep}
\label{sec:increasing-rep}

If you want your character to start play with lots of social capital, you can increase your Rep score(s) at the cost of 1 CP per 10 additional points. No individual Rep score may be raised above 80, and the maximum amount of CP that may be spent on Rep is 35 points.

\subsubsection{Starting Morph}
\label{sec:starting-morph}

Perhaps the most important use of CP is to buy the morph with which your character begins play. This may be the original bodily form in which your character started life, or it may simply be the sleeve they are currently inhabiting. Available morphs are listed starting on p. 139. Note that any aptitude or skill bonuses provided by the morph are applied after all CP are spent. In other words, these bonuses do not affect the costs of buying aptitude and skill points during character generation. No aptitude may be modified over 40.

\subsubsection{Purchasing Traits}
\label{sec:purchasing-traits}

Traits represent specific qualities your character has that may help or hinder them. Positive traits supply bonuses in certain situations, and each has a listed CP cost. You may not spend more than 50 CP on positive traits. Negative traits inflict disadvantages on your character, but they also give you extra CP that you can spend on customizing your character. You may not purchase more than 50 CP worth of negative traits, and no more than 25 CP may be negative morph traits. Positive traits are listed on p. 145, negative traits on p. 148. Note that traits you receive from your background or faction do not cost or provide you with bonus CP. Traits listed as morph traits apply to the morph, and not the ego. If the character switches to a new morph, these traits are lost (and new morph traits may be gained). Morph traits may be bought like other traits during character generation.

\subsubsection{Psi Sleights}
\label{sec:purchasing-psi-sleights}

Characters who purchase the Psi trait (p. 147) may spend CP to purchase sleights (see Sleights, p. 223). These represent specific psi abilities the character has learned. The cost to buy a sleight is 5 CP. No more than 5 psi-chi and 5 psi-gamma sleights may be bought during character creation. Note that any skill or aptitude bonuses from sleights are treated as modifications; they are applied after all CP are spent and do not affect the cost of buying skills or aptitudes during character creation.

\subsection{Purchase Gear}
\label{sec:purchase-gear}

No matter what faction you are from, you use Credit to buy gear during character creation. A complete list of gear and costs can be found in the Gear chapter, p. 294. The average costs for each cost category should be used when calculating gear prices.     Every character starts off with one piece of gear for free: a standard muse (p. 332). This is the digital AI companion that the character has had since they were a child. Additionally, each character starts with 1 month of backup insurance (p. 330) at no cost.

\begin{center} 
\begin{tabular}{|c|c|c|}
\hline
\multicolumn{3}{|c|}{Gear Costs} \\
\hline
Category & Range (Credits) & Average (Credits)\\
\hline
Trivial & 1-99 & 50\\
\hline
Low & 100-499 & 250\\
\hline
Moderate & 500-1,499 & 1,000\\
\hline
High & 1,500-9,999 & 5,000\\
\hline
Expensive & 10,000+ & 20,000\\
\hline 
\end{tabular}
\end{center}

There is no limitation other than what the gamemaster allows on what gear characters can and cannot buy during character creation. Both the players and gamemaster should keep the character’s background and faction in mind. Since some gear is likely very restricted in some habitats if not outright illegal, there needs to be a plausible explanation for who and how a character from such a place might have such gear. If there isn’t, then the gamemaster can choose not to allow it. The starting locale for a game should also be considered. A character from the restrictive Jovian Republic might have a hard time explaining how they have an illegal cornucopia machine back in the Republic, but if the game takes place on board a scum barge where everything is available and anything goes, then such an explanation becomes much easier.

The one exception to buying gear with Credit is the purchase of additional morphs. Characters may buy extra morphs during character creation, but they must be bought with CP. The player must choose one morph in which the character is sleeved. Extra morphs also require body bank service fees (p. 331).

Note that any skill or aptitude bonuses from gear are treated as modifications; they are applied after all CP are spent and do not affect the cost of buying skills or aptitudes during character creation. %%% txt/140.txt

\subsection{Choose Motivations}
\label{sec:choose-motivations}

The next step is to choose 3 personal motivations for your character (see Motivations, p. 121). These are memes, in the form of ideologies or goals, which your character is pursuing. These may be as specific “undermine the local triad boss” or as broad as “promote hypercapitalism,” and they may be short term or long term. Some sample motivations are provided on the Example Motivations table (p. 138). You should work with your gamemaster when choosing your motivations, as they can be used to propel the storyline forward and specific scenarios can be constructed around your character’s goals. Some of your character’s motivations may change later (see Changing Motivation, p. 152). Motivations will help your character regain Moxie (p. 122) and earn extra Rez Points during gameplay (p. 384).

Motivations should be listed on your character sheet as a single term or short phrase, along with a + or - symbol to denote whether they support or oppose it. For example, “+Fame” would indicate that your character seeks to become a famous media personality, whereas “-Reclaim Earth” means that your character opposes the goal of reclaiming Earth.

\subsubsection{Example Motivations}
\label{sec:example-motivations}

\begin{tabular}{lll}
Alien Contact & Anarchism & Artistic Expression \\
Bioconservatism & Education & Exploration \\
Fame & Fascism & Hedonism \\
Hypercapitalism & Immortality & Libertarianism \\
Martian Liberation & Morphological Freedom & Nano-ecology \\
Open Source & Personal Career & Personal Development \\
Philanthropy & Preservationism & Reclaiming Earth \\
Religion & Research & (AI/Infomorph/Pod/Uplift) Rights \\
(AI/Infomorph/Pod/Uplift) Slavery & Socialism & Techno-Progressivism \\
Vengeance & Venusian Sovereignty & Wealth \\
\end{tabular}

\subsection{Final Touches}
\label{sec:final-touches}

Now that you have everything settled, there are a few final steps.

\subsubsection{Remaining Stats}
\label{sec:remaining-stats}

A few stats now need to be calculated and added to your character sheet:

\begin{itemize}
\item Lucidity (p. 122) equals your character’s WIL x 2.
\item Trauma Threshold (p. 122) equals your LUC divided by 5 (round up).
\item Insanity Rating (p. 122) equals LUC x 2.
\item Initiative (p. 121) equals your character’s (REF + INT) x 2.
\item Damage Bonus (p. 123) for melee equals SOM $\div$ 10 (round down).
\item Death Rating (p. 122) equals DUR x 1.5 (biomorphs, round up) or DUR x 2 (synthmorphs)
\item Speed (p. 121) equals 1 (3 for infomorphs), modified as appropriate by implants.
\end{itemize} %%% txt/141.txt

\subsubsection{Detailing The Character}
\label{sec:detailing-the-character}

The final step in character creation is filling in the details and figuring out what your character is like and what they are all about. Your character’s Background is a good place to start as it says where they came, but it could be expanded. What did they think of their childhood? Do they still have ties from there? How did they move from such origins to the Faction they are part of? Are they fully supportive of their Faction’s goals, or are they in opposition? How does the character view other Factions?

Next, take a look at the skills and other defining points—these also tell a story. How did they acquire those skills? Why? How did they develop their Rep score (or lack of one)? How did they get connected with the groupings represented by their Networking skills? What do the character’s traits say about them? How did they get their current morph? Is it their original? If not, what happened to their first body? Also taking into account the major factor of Motivations, all of these questions will help you build a de- fining picture of your character. Not everything about your character needs to be filled out, of course—it’s ok to leave a few blanks that you can fill in later. Assembling the points you have deduced so far will help you to present your character as a whole, unique individual, however, rather than just a blank template. As a final step, take a few minutes to pick out some specific identifying features and personality quirks that will help you define the character to others. This could be a way of talking, a strongly-projected attitude, a catchphrase they use frequently, a unique look or style of dress, a repetitive behavior, an annoying mannerism, or anything else similar that is easy to latch onto. Such idiosyncrasies give something that other players can latch onto, spurring roleplaying opportunities.

\section{Starting Morphs}

Each morph has an associated CP cost. It also supplies the character’s \textbf{Durability} and \textbf{Wound Threshold} stats, and may modify Initiative, Speed, and certain aptitudes and learned skills. A credit cost is also listed, but this refers to the cost of buying such a morph in gameplay.

Flexible Aptitude Bonuses: Some morphs have aptitude bonuses that may be applied to an aptitude of the player’s choice. This reflects that not all morphs are created equal. When assigning these universal aptitude bonuses, each boost must be applied to a separate aptitude; you may not elevate an aptitude that is already raised by that morph. Once an individual morph’s aptitude bonuses have been assigned, they are permanent for that morph (i.e., if another character resleeves into that morph, the bonuses remain the same).

\subsection{Biomorphs}
\label{sec:starting-biomorphs}

Biomorphs are fully biological sleeves (usually equipped with implants), birthed naturally or in an exowomb, and grown to adulthood either naturally or at a slightly accelerated rate.

\subsubsection{Flats}
\label{sec:starting-flats}

Flats are baseline unmodified humans, born with all of the natural defects, hereditary diseases, and other genetic mutations that evolution so lovingly applies. Flats are increasingly rare—most died off with the rest of humanity during the Fall. Most new children are splicers—screened and genefixed at the least—except in habitats where flats are treated as second-class citizens and indentured labor.

\begin{description*}
\item[Implants] None
\item[Aptitude Maximum] 20
\item[Durability] 30
\item[Wound Threshold] 6
\item[Disadvantages] None (Genetic Defects trait common)
\item[CP Cost] 0
\item[Credit Cost] High
\end{description*}

\subsubsection{Splicers}
\label{sec:starting-splicers}

Splicers are genefixed humans. Their genome has been cleansed of hereditary diseases and optimized for looks and health, but has not otherwise been substantially upgraded. Splicers make up the majority of transhumanity.

\begin{description*}
\item[Implants] Basic Biomods, Basic Mesh Inserts, Cortical Stack
\item[Aptitude Maximum] 25
\item[Durability] 30
\item[Wound Threshold] 6
\item[Advantages] +5 to one aptitude of the player’s choice
\item[CP Cost] 10
\item[Credit Cost] High
\end{description*}

\subsubsection{Exalts}
\label{sec:starting-exalts}

Exalt morphs are genetically-enhanced humans, designed to emphasize specific traits. Their genetic code has been tweaked to make them healthier, smarter, and more attractive. Their metabolism is modified to predispose them towards staying fit and athletic for the duration of an extended lifespan.

\begin{description*}
\item[Implants] Basic Biomods, Basic Mesh Inserts, Cortical Stack
\item[Aptitude Maximum] 30 
\item[Durability] 35 
\item[Wound Threshold] 7 
\item[Advantages] +5 COG, +5 to three other aptitudes of the player’s choice 
\item[CP Cost] 30 
\item[Credit Cost] Expensive
\end{description*}

\subsubsection{Mentons}
\label{sec:starting-mentons}

Mentons are genetically modified to increase cognitive abilities, particularly learning ability, creativity, attentiveness, and memory. Rumors exist of superenhanced mentons with more extreme intelligence mods, but brain-hacking is notoriously difficult, and many attempts to redesign mental faculties result in impaired functioning, instability, or insanity.

\begin{description*}
\item[Implants] Basic Biomods, Basic Mesh Inserts, Cortical Stack, Eidetic Memory, Hyper-Linguist, Math Boost
\item[Aptitude Maximum] 30 
\item[Durability] 35 
\item[Wound Threshold] 7 
\item[Advantages] +10 COG, +5 INT, +5 WIL, +5 to one aptitude of the player’s choice
\item[CP Cost] 40 
\item[Credit Cost] Expensive
\end{description*}

\subsubsection{Olympians}
\label{sec:starting-olympians}

Olympians are human upgrades with improved athletic capabilities like endurance, eye-hand coordination, and cardio-vascular functions. Olympians are common among athletes, dancers, freerunners, and soldiers.

\begin{description*}
\item[Implants] Basic Biomods, Basic Mesh Inserts,Cortical Stack 
\item[Aptitude Maximum] 30 
\item[Durability] 40 
\item[Wound Threshold] 8 
\item[Advantages] +5 COO, +5 REF, +10 SOM, +5 to one other aptitude of the player’s choice
\item[CP Cost] 40 
\item[Credit Cost] Expensive 
\end{description*}

\subsubsection{Sylphs}
\label{sec:starting-sylphs}

Sylph morphs are tailor-made for media icons, elite socialites, XP stars, models, and narcissists. Sylph gene sequences are specifi cally designed for distinctive good looks. Ethereal and elfin features are common, with slim and lithe bodies. Their metabolism has also been sanitized to eliminate unpleasant bodily odors and their pheromones adjusted for universal appeal.

\begin{description*}
\item[Implants] Basic Biomods, Basic Mesh Inserts, Clean Metabolism, Cortical Stack, Enhanced Pheromones
\item[Aptitude Maximum] 30 
\item[Durability] 35 
\item[Wound Threshold] 7 
\item[Advantages] Striking Looks (Level 1) trait, +5 COO, +10 SAV, +5 to one other aptitude of the player’s choice
\item[CP Cost] 40 
\item[Credit Cost] Expensive 
\end{description*}

\subsubsection{Bouncers}
\label{sec:starting-bouncers}

Bouncers are humans genetically adapted for zero-G and microgravity environments. Their legs are more limber, and their feet can grasp as well as their hands.

\begin{description*}
\item[Implants] Basic Biomods, Basic Mesh Inserts, Cortical Stack, Grip Pads, Oxygen Reserve, Prehensile Feet
\item[Aptitude Maximum] 30 
\item[Durability] 35 
\item[Wound Threshold] 7 
\item[Advantages] Limber (Level 1) trait, +5 COO, +5 SOM, +5 to one aptitude of the player’s choice
\item[CP Cost] 40 
\item[Credit Cost] Expensive 
\end{description*}

\subsubsection{Furies}
\label{sec:starting-furies}

Furies are combat morphs. These transgenic human upgrades feature genetics tailored for endurance, strength, and reflexes, as well as behavioral modifica- tions for aggressiveness and cunning. To offset tendencies for unruliness and macho behavior patterns, furies feature gene sequences promoting pack mentalities and cooperation, and they tend to be biologically female.

\begin{description*}
\item[Implants] Basic Biomods, Basic Mesh Inserts, Bioweave Armor (Light), Cortical Stack, Enhanced Vision, Neurachem (Level 1), Toxin Filters
\item[Aptitude Maximum] 30 
\item[Speed Modifier] +1 (neurachem) 
\item[Durability] 50 
\item[Wound Threshold] 10 
\item[Advantages] +5 COO, +5 REF, +10 SOM, +5 WIL, +5 to one aptitude of the player’s choice
\item[CP Cost] 75 
\item[Credit Cost] Expensive (minimum 40,000) 
\end{description*}

\subsubsection{Futuras}
\label{sec:starting-futuras}

An exalt variant, futura morphs were specially crafted for the “Lost generation.” Tailor-made for accelerated growth and adjusted for confidence, self-reliance, and adaptability, futuras were intended to help transhumanity regain its foothold. These programs proved disastrous and the line was discontinued, but some models remain, viewed by some with distaste and others as collectibles or exotic oddities.

\begin{description*}
\item[Implants] Basic Biomods, Basic Mesh Inserts, Cortical Stack, Eidetic Memory, Emotional Dampers
\item[Aptitude Maximum] 30 
\item[Durability] 35 
\item[Wound Threshold] 7 
\item[Advantages] +5 COG, +5 SAV, +10 WIL, +5 to one other aptitude of the player’s choice
\item[CP Cost] 40 
\item[Credit Cost] Expensive (exceptionally rare; 50,000+) 
\end{description*}

\subsubsection{Ghosts}
\label{sec:starting-ghosts}

Ghosts are partially designed for combat applications, but their primary focus is stealth and infiltration. Their genetic profile encourages speed, agility, and reflexes, and their minds are modified for patience and problem-solving.

\begin{description*}
\item[Implants] Basic Biomods, Basic Mesh Inserts, Chameleon Skin, Cortical Stack, Adrenal Boost, Enhanced Vision, Grip Pads
\item[Aptitude Maximum] 30 
\item[Durability] 45 
\item[Wound Threshold] 9 
\item[Advantages] +10 COO, +5 REF, +5 SOM, +5 WIL, +5 to one aptitude of the player’s choice
\item[CP Cost] 70 
\item[Credit Cost] Expensive (minimum 40,000) 
\end{description*}

\subsubsection{Hibernoids}
\label{sec:starting-hibernoids}

Hibernoids are transgenic-modified humans with heavily-altered sleep patterns and metabolic processes. Hibernoids have a decreased need for sleep, requiring only 1-2 hours a day on average. They also have the ability to trigger a form of voluntary hibernation, effectively stopping their metabolism and need for oxygen. Hibernoids make excellent long-duration space travelers and habtechs, but these morphs are also favored by personal aides and hypercapitalists with non-stop lifestyles.

\begin{description*}
\item[Implants] Basic Biomods, Basic Mesh Inserts, Circadian Regulation, Cortical Stack, Hibernation
\item[Aptitude Maximum] 25 
\item[Durability] 35 
\item[Wound Threshold] 7 
\item[Advantages] +5 INT, +5 to one aptitude of the player’s choice
\item[CP Cost] 25 
\item[Credit Cost] Expensive 
\end{description*}

\subsubsection{Neotenics}
\label{sec:starting-neonetics}

Neotenics are transhumans modified to retain a childlike form. They are smaller, more agile, inquisitive, and less resource-depleting, making them ideal for habitat living and spacecraft. Some people find neotenic sleeves distasteful, especially when employed in certain media and sex work capacities.

\begin{description*}
\item[Implants] Basic Biomods, Basic Mesh Inserts, Cortical Stack 
\item[Aptitude Maximum] 20 (SOM), 30 (all else) 
\item[Durability] 30 
\item[Wound Threshold] 6 
\item[Advantages] +5 COO, +5 INT, +5 REF, +5 to one aptitude of the player’s choice; neotenics count as a small target (-10 modifier to hit in combat) 
\item[Disadvantages] Social Stigma (Neotenic) trait 
\item[CP Cost] 25 
\item[Credit Cost] Expensive 
\end{description*}

\subsubsection{Remade}
\label{sec:starting-remade}

The remade are completely redesigned humans: humans 2.0. Their cardiovascular systems are stronger, the digestive tract has been sanitized and restructured to eliminate flaws, and they have otherwise been optimized for good health, smarts, and longevity with numerous transgenic mods. The remade are popular with the ultimates faction. The remade look close to human, but are different in very noticeable and sometimes eerie ways: taller, lack of hair, slightly larger craniums, wider eyes, smaller noses, smaller teeth, and elongated digits.

\begin{description*}
\item[Implants] Basic Biomods, Basic Mesh Inserts, Circadian Regulation, Clean Metabolism, Cortical Stack, Eidetic Memory, Enhanced Respiration, Temperature Tolerance, Toxin Filters
\item[Aptitude Maximum] 40 
\item[Durability] 40 
\item[Wound Threshold] 8 
\item[Advantages] +10 COG, +5 SAV, +10 SOM, +5 to two other aptitudes of the player’s choice
\item[Disadvantages] Uncanny Valley trait 
\item[CP Cost] 60 
\item[Credit Cost] Expensive (minimum 40,000+) 
\end{description*}

\subsubsection{Rusters}
\label{sec:starting-rusters}

Adapted for survival with minimum gear in the not-yet-terraformed Martian environment, these transgenic morphs feature insulated skin for more effective thermoregulation and respiratory system improvements to require less oxygen and filter carbon dioxyde, among other mods.

\begin{description*}
\item[Implants] Basic Biomods, Basic Mesh Inserts, Cortical Stack, Enhanced Respiration, Temperature Tolerance
\item[Aptitude Maximum] 25 
\item[Durability] 35 
\item[Wound Threshold] 7 
\item[Advantages] +5 SOM, +5 to one aptitude of the player’s choice 
\item[CP Cost] 25 
\item[Credit Cost] Expensive 
\end{description*}

\subsubsection{Neo-Avians}
\label{sec:starting-neo-avians}

Neo-avians include ravens, crows, and gray parrots uplifted to human-level intelligence. Their physical sizes are much larger than their non-uplifted cousins (to the size of a human child), with larger heads for their increased brain size. Numerous transgenic modifications have been made to their wings, allowing them to retain limited flight capabilities at 1 g, but giving them a more bat-like physiology so they can bend and fold better, and adding primitive digits for basic tool manipulation. Their toes are also more articulated and now accompanied with an opposable thumb. Neo-avians have adapted well to microgravity environments, and are favored for their small size and reduced resource use.

\begin{description*}
\item[Implants] Basic Biomods, Basic Mesh Inserts, Cortical Stack
\item[Aptitude Maximum] 25 (20 SOM) 
\item[Durability] 20 
\item[Wound Threshold] 4 
\item[Advantages] Beak/Claw Attack (1d10 DV, use Unarmed Combat skill), Flight, +5 INT, +10 REF, +5 to one other aptitude of the player’s choice
\item[CP Cost] 25 
\item[Credit Cost] Expensive 
\end{description*}

\subsubsection{Neo-Hominids}
\label{sec:starting-neo-hominids}

Neo-hominids are uplifted chimpanzees, gorillas, and orangutans. All feature enhanced intelligence and bipedal frames.

\begin{description*}
\item[Implants] Basic Biomods, Basic Mesh Inserts, Cortical Stack 
\item[Aptitude Maximum] 25 
\item[Durability] 30 
\item[Wound Threshold] 6 
\item[Advantages] +5 COO, +5 INT, +5 SOM, +5 to one other aptitude of the player’s choice, +10 Climbing skill
\item[CP Cost] 25 
\item[Credit Cost] Expensive 
\end{description*}

\subsubsection{Octomorphs}
\label{sec:starting-octomorphs}

These uplifted octopi sleeves have proven quite useful in zero-gravity environments. They retain eight arms, their chameleon ability to change skin color, ink sacs, and a sharp beak. They also have increased cranial capacity and longevity, can breathe both air and water, and lack a skeletal structure so they can squeeze through tight spaces. Octomorphs typically crawl along in zero-gravity using their arm suckers and expelling air for propulsion and can even walk on two of their arms in low gravity. Their eyes have been enhanced with color vision, provide a 360-degree field of vision, and they rotationally adjust to keep the slit-shaped pupil aligned with “up.” A transgenic vocal system allows them to speak.

\begin{description*}
\item[Implants] Basic Biomods, Basic Mesh Inserts, Cortical Stack, Chameleon Skin
\item[Aptitude Maximum] 30 
\item[Durability] 30 
\item[Wound Threshold] 6 
\item[Advantages] 8 Arms, Beak Attack (1d10 DV, use Unarmed Combat skill), Ink Attack (blinding, use Exotic Ranged: Ink Attack skill), Limber (Level 2) trait, 360-degree Vision, +30 Swimming skill, +10 Climbing skill, +5 COO, +5 INT, +5 to one other aptitude of the player’s choice
\item[CP Cost] 50 
\item[Credit Cost] Expensive (minimum 30,000+) 
\end{description*}

\subsection{Pods}
\label{sec:starting-pods}

Pods (from “pod people”) are vat-grown, biological bodies with extremely undeveloped brains that are augmented with an implanted computer and cybernet- ics system. Though typically run by an AI, pods are socially unfavored in some stations, utilized in slave labor in others, and even illegal in some areas. Because pods underwent accelerated growth in their creation, and were mostly grown as separate parts and then assembled, their biological design includes some shortcuts and limitations, offset with implants and regular maintenance. They lack reproductive capabilities. In many habitats, their legal status is a hotly-contested issue. Unless otherwise noted, pods are also considered biomorphs for all rules purposes.

\subsubsection{Pleasure Pods}
\label{sec:starting-pleasure-pods}

Pleasure pods are exactly what they seem—faux humans designed purely for intimate entertainment purposes. Pleasure pods have extra nerve clusters in their erogenous zones, fine motor control over certain muscle groups, enhanced pheromones, sanitized metabolisms, and the genetics for purring. Naturally, they are crafted for good looks and charisma and enhanced in other areas as well. Pleasure pods are capable of switching their sex at will to male, female, hermaphrodite, or neuter.

\begin{description*}
\item[Implants] Basic Biomods, Basic Mesh Inserts, Clean Metabolism, Cortical Stack, Cyberbrain, Enhanced Pheromones, Mnemonic Augmentation, Puppet Sock, Sex Switch
\item[Aptitude Maximum] 30 
\item[Durability] 30 
\item[Wound Threshold] 6 
\item[Advantages] +5 INT, +5 SAV, +5 to one aptitude of the player’s choice 
\item[Disadvantages] Social Stigma (Pleasure Pod) trait 
\item[CP Cost] 20 
\item[Credit Cost] High 
\end{description*}

\subsubsection{Worker Pods}
\label{sec:starting-worker-pods}

Part exalt human, part machine, these basic pods are virtually indistinguishable from humans. Worker pods are often used in menial labor jobs where interaction with humans is necessary.

\begin{description*}
\item[Implants] Basic Biomods, Basic Mesh Inserts, Cortical Stack, Cyberbrain, Mnemonic Augmentation, Puppet Sock
\item[Aptitude Maximum] 30 
\item[Durability] 35 
\item[Wound Threshold] 7 
\item[Advantages] +10 SOM, +5 to one aptitude of the player’s choice 
\item[Disadvantages] Social Stigma (Pod) trait 
\item[CP Cost] 20 
\item[Credit Cost] High 
\end{description*}

\subsubsection{Novacrab}
\label{sec:starting-novacrab}

Novacrabs are a pod design bio-engineered from coconut crab and spider crab stock and grown to a larger (human) size. Novacrabs are ideal for hazardous work environments as well as vacworker, police, or bodyguard duties, giving their ten 2-meter long legs, massive claws, and chitinous armor. They climb and handle microgravity well and can withstand a wide range of atmospheric pressure (and sudden pressure changes) from vacuum to deep sea. Novacrabs feature compound eyes (with human-equivalent image resolution), gills, dexterous manipulatory digits on their fifth set of limbs, and transgenic vocal cords.

\begin{description*}
\item[Implants] Basic Biomods, Basic Mesh Inserts, Carapace Armor, Cortical Stack, Cyberbrain, Enhanced Respiration, Gills, Mnemonic Augmentation, Oxygen Reserve, Puppet Sock, Temperature Tolerance, Vacuum Sealing
\item[Aptitude Maximum] 30 
\item[Durability] 40 
\item[Wound Threshold] 8 
\item[Advantages] 10 legs, Carapace Armor (11/11), Claw Attack (DV 2d10), +10 SOM, +5 to two other aptitudes of the player’s choice
\item[CP Cost] 60 
\item[Credit Cost] Expensive (minimum 30,000+) 
\end{description*}

\subsection{Synthetic Morphs}
\label{sec:starting-syntheticmorphs}

Synthetic morphs are completely artificial/robotic. They are usually operated by AIs or via remote control, but the lack of available biomorphs after the Fall meant that many infugees resorted to resleeving in robotic shells, which were also cheaper, quicker to manufacture, and more widely available. Nevertheless, synthmorphs are viewed with disdain in many habitats, an option that only the poor and desperate accept to be sleeved in. Synthetic morphs are not without with their advantages, however, and so are commonly used for menial labor, heavy labor, habitat construction, and security services.

All synthmorphs have the following advantages:

\begin{itemize}
\item Lack of Biological Functions. Synthmorphs need not be bothered with trivialities like breathing, eating, defecating, aging, sleeping, or any similar minor but crucial aspects of biological life.
\item Pain Filter. Synthmorphs can filter out their pain receptors, so that they are unhampered by wounds or physical damage. This allows them to ignore the -10 modifier from 1 wound (see Wound Effects, p. 207), but they suffer -30 on any tactile-based Perception Tests and will not even notice they have been damaged unless they succeed in a (modified) Perception Test.
\item Immunity to Shock Weapons. Synthmorphs have no nervous system to disrupt, and their optical electronics are carefully shielded from interfer- ence. Shock attacks may temporarily disrupt their wireless radio communications, however, for the duration of the attack.
\item Environmental Durability. Synthmorphs are built to withstand a wide range of environments, from dusty Mars to the oceans of Europa to the vacuum of space. They are unaffected by any but the most extreme temperatures and atmospheric pressures. Treat as Temperature Tolerance (p. 305) and Vacuum Sealing (p. 305).
\item Toughness. Synthetic shells are made to last—a fact reflected in their higher Durability and built-in Armor ratings. Their composition also makes their physical strikes more damaging: apply a +2 DV modifier on unarmed attacks for human-sized shells and larger.
\end{itemize}

\subsubsection{Case}
\label{sec:starting-case}

Cases are extremely cheap, mass-produced robotic shells intended to provide an affordable remorphing option for the millions of infugees created by the Fall. Though many varieties of case bot models exist, they are uniformly regarded as shoddy and inferior. Most case morphs are vaguely anthromorphic, with a thin framework body, standing just shorter than an average human, and suffer from frequent malfunctions.

\begin{description*}
\item[Enhancements] Access Jacks, Basic Mesh Inserts, Cortical Stack, Cyberbrain, Mnemonic Augmentation
\item[Mobility System] (Movement Rate) Walker (4/16) 
\item[Aptitude Maximum] 20 
\item[Durability] 20 
\item[Wound Threshold] 4 
\item[Advantages] Armor (4/4) 
\item[Disadvantages] -5 to one chosen aptitude, Lemon trait, Social Stigma (Clanking Masses) trait
\item[CP Cost] 5 
\item[Credit Cost] Moderate 
\end{description*}

\subsubsection{Synth}
\label{sec:starting-synths}

Synths are anthromorphic robotic shells (androids and gynoids). They are typically used for menial labor jobs where pods are not as good of an option. Cheaper than many other morphs, they are commonly used for people who need a morph quickly and cheaply or simply on a transient basis. Though they look humanoid, synths are easily recognizable as non-biological unless they have the synthetic mask option (p. 311).

\begin{description*}
\item[Enhancements] Access Jacks, Basic Mesh Inserts, Cortical Stack, Cyberbrain, Mnemonic Augmentation
\item[Mobility System] Walker (4/20) 
\item[Aptitude Maximum] 30 
\item[Durability] 40 
\item[Wound Threshold] 8 
\item[Advantages] +5 SOM, +5 to one other aptitude of the player’s choice, Armor 6/6
\item[Disadvantages] Social Stigma (Clanking Masses) trait, Uncanny Valley trait
\item[CP Cost] 30 
\item[Credit Cost] High 
\end{description*}

\subsubsection{Arachnoids}
\label{sec:starting-arachnoids}

Arachnoid robotic shells are 1-meter in length, segmented into two parts, with a smaller head, like a spider or termite. They feature four pairs of 1.5-meter- long retractable arms/legs, capable of rotating around the axis of the body, with built-in hydraulics for propelling the bot with small leaps. The manipulator claws on each arm/leg can be switched out with extendable mini-wheels for high-speed skating movement. A smaller pair of manipulator arms near the head allows for closer handling and tool use. In zero-G environments, arachnoids can retract their arms/legs and maneuver with vectored air thrusters.

\begin{description*}
\item[Enhancements] Access Jacks, Basic Mesh Inserts, Cortical Stack, Cyberbrain, Enhanced Vision, Extra Limbs (6 Arms/Legs), Lidar, Mnemonic Augmentation, Pneumatic Limbs, Radar
\item[Mobility System] Walker (4/24), Thrust Vector (8/40) 
\item[Aptitude Maximum] 30 
\item[Durability] 40 
\item[Wound Threshold] 8 
\item[Advantages] +5 COO, +10 SOM, Armor 8/8 
\item[CP Cost] 45 
\item[Credit Cost] Expensive (minimum 40,000+) 
\end{description*}

\subsubsection{Dragonfly}
\label{sec:starting-dragonfly}

The dragonfly robotic morph takes the shape of a meter-long flexible shell with multiple wings and manipulator arms. Capable of near-silent turbofan-aided flight in Earth gravity, dragonfly bots fare even better in microgravity.

\begin{description*}
\item[Enhancements] Access Jacks, Basic Mesh Inserts, Cortical Stack, Cyberbrain, Mnemonic Augmentation
\item[Mobility System] Winged (8/32) 
\item[Aptitude Maximum] 30 (20 SOM) 
\item[Durability] 25 
\item[Wound Threshold] 5 
\item[Advantages] Flight, +5 REF, Armor (2/2)
\item[CP Cost] 20 
\item[Credit Cost] High
\end{description*}

\subsubsection{Flexbots}
\label{sec:starting-flexbots}

Designed for multi-purpose functions, flexbots can transform their shells to suit a range of situations and tasks. Their core frame consists of a half-dozen interlocking and shape-adjustable modules capable of auto-transforming into a variety of shapes: multilegged walker, tentacle, hovercraft, and many others. Each module features its own sensor units and “bush robot” fractal-branching digits (each capable of breaking into smaller digits, down to the micrometer scale, allowing for ultra-fine manipulation). The flexbot control computer is also distributed between modules. Individual flexbots are only the size of a large dog, but multiple flexbots can join together for larger mass operations, even taking on heavy-duty tasks such as demolition, excavation, manufacturing, robotics assembly, and so on.

\begin{description*}
\item[Enhancements] Access Jacks, Basic Mesh Inserts, Cortical Stack, Cyberbrain, Fractal Digits, Mnemonic Augmentation, Modular Design, Shape Adjusting
\item[Mobility System] Walker (4/16), Hover (8/40) 
\item[Aptitude Maximum] 30 
\item[Durability] 25 
\item[Wound Threshold] 5 
\item[Advantages] Armor 4/4 
\item[CP Cost] 20 
\item[Credit Cost] Expensive (minimum 30,000+)
\end{description*}

\subsubsection{Reaper}
\label{sec:starting-reaper}

The reaper is a common combat bot, used in place of biomorph soldiers and typically operated via teleoperation or by autonomous AI. The reaper’s core form is an armored disc, so that it can turn and present a thin profile to an enemy. It uses vector thrust nozzles to maneuver in microgravity, and also takes advantage of an ionic drive for fast movement over distance. Four legs/manipulating arms and four weapon pods are folded inside its frame. The reaper’s shell is made of smart materials, allowing these limbs and weapon mounts to extrude in any direction desired and even to change shape and length. In gravity environments, the reaper walks or hops on two or four of these limbs. Reapers are infamous due to numerous war XPs, and bringing one into most habitats will undoubtedly raise eyebrows, if not get you arrested.

\begin{description*}
\item[Enhancements] 360-Degree Vision, Access Jacks, Anti-Glare, Basic Mesh Inserts, Cortical Stack, Cyberbrain, Cyber Claws, Extra Limbs (4), Heavy Combat Armor, Magnetic System, Pneumatic Limbs, Puppet Sock, Radar, Reflex Booster, Shape Adjusting, Structural Enhancement, T-Ray Emitter, Weapon Mount (Articulated, 4)
\item[Mobility System] Walker (4/20), Hopper (4/20), Ionic (12/40), Vectored Thrust (4/20)
\item[Aptitude Maximum] 40 
\item[Speed Modifier] +1 (Reflex Booster) 
\item[Durability] 50 (60 with Structural Enhancement) 
\item[Wound Threshold] 10 (12 w/Structural Enhancement) 
\item[Advantages] 4 Limbs, +5 COO, +10 REF (+20 with Reflex Booster), +10 SOM, Armor 16/16
\item[CP Cost] 100 
\item[Credit Cost] Expensive (minimum 50,000+)
\end{description*}

\subsubsection{Slitheroids}
\label{sec:starting-slitheroids}

Slitheroid bots are synthetic shells taking the form of a 2-meter-long segmented metallic snake, with two retractable arms for tool use. Snake bots can coil, twist, and roll their bodies into a ball or hoop, moving either by slithering, burrowing, rolling, or pulling themselves along by their arms. The sensor suite and control computer are housed in the head.

\begin{description*}
\item[Enhancements] Access Jacks, Basic Mesh Inserts, Cortical Stack, Cyberbrain, Enhanced Vision, Mnemonic Augmentation
\item[Mobility System] Snake (4/16; 8/32 rolling) 
\item[Aptitude Maximum] 30 
\item[Durability] 45 
\item[Wound Threshold] 9 
\item[Advantages] +5 COO, +5 SOM, +5 to one other aptitude of the player’s choice, Armor 8/8
\item[CP Cost] 40 
\item[Credit Cost] Expensive
\end{description*}

\subsubsection{Swarmanoid}
\label{sec:starting-swarmanoid}

The swarmanoid is not a single shell per se, but rather a swarm of hundreds of insect-sized robotic microdrones. Each individual “bug” is capable of crawling, rolling, hopping several meters, or using nanocopter fan blades for airlift. The controlling computer and sensor systems are distributed throughout the swarm. Though the swarm can “meld” together into a roughly child-sized shape, the swarm is incapable of tackling physical tasks like grabbing, lifting, or holding as a unit. Individual bugs are quite capable of interfacing with electronics.

\begin{description*}
\item[Enhancements] Access Jacks, Basic Mesh Inserts, Cortical Stack, Cyberbrain, Mnemonic Augmentation, Swarm Composition
\item[Mobility System] Walker (2/8), Hopper (4/20), Rotor (4/32) 
\item[Aptitude Maximum] 30 
\item[Durability] 30 
\item[Wound Threshold] 6 
\item[Advantages] See Swarm Composition (p. 311) 
\item[Disadvantages] See Swarm Composition (p. 311) 
\item[CP Cost] 25 
\item[Credit Cost] Expensive
\end{description*}

\subsection{Infomorphs}
\label{sec:starting-infomorphs}

Infomorphs are digital-only forms—they lack a physical body. Infomorphs are sometimes carried by other characters instead of (or in addition to) a muse in a ghostrider module (p. 307). Full rules for infomorphs can be found on p. 264.

\begin{description*}
\item[Enhancements] Mnemonic Augmentation 
\item[Aptitude Maximum] 40 
\item[Speed Modifier] +2 
\item[Disadvantages] No physical form
\item[CP Cost] 0 
\item[Credit Cost] 0
\end{description*}

\section{Traits} Unless otherwise noted, listed traits are ego traits.

\section{Positive Traits}
\label{sec:positive-traits}
Positive traits provide bonuses to the character in certain situations.

\subsection{Adaptability}
\label{sec:traits-adaptability}

\textbf{Cost:} 10 (Level 1) or 20 (Level 2) CP

Resleeving is a breeze for this character. They adjust to new morphs much more quickly than most other people. Apply a +10 modifier per level for Integration Tests and Alienation Tests (p. 272).

\subsection{Allies}
\label{sec:traits-allies}

\textbf{Cost:} 30 CP

The character is part of or has a relationship with some influential group that they can occasionally call on for support. For example, this could be their old gatecrashing crew, former research lab co-workers, a criminal cartel they are part of, or an elite social clique. The gamemaster and player should work out what the character’s relationship is with this group, and why the character can call on them for aid. Gamemaster’s should take care that these allies are not abused, such as calling on them more than once per game session. The character’s ties to this group are also a two-way street—they will be expected to perform duties for the group on occasion as well (a potential plot seed for scenarios).

\subsection{Ambidextrous}
\label{sec:traits-ambidextrous}

\textbf{Cost:} 10 CP

The character can use and manipulate objects equally well with both hands (they do not suffer the off-hand modifier, as noted on p. 193). If the character has other prehensile limbs (feet, tail, tentacles, etc), this trait may be applied to a limb other than the hand. This trait may be taken multiple times for multiple limbs.

\subsection{Animal Empathy}
\label{sec:traits-animal-empathy}

\textbf{Cost:} 5 CP

The character has an instinctive feel for handling and working with non-sapient animals of all kinds. Apply a +10 modifier to Animal Handling skill tests or whenever the character makes a test to influence or interact with an animal.

\subsection{Brave}
\label{sec:traits-brave}

\textbf{Cost:} 10 CP

This character does not scare easily, and will face threats, intimidation, and certain bodily harm without flinching. As a side effect, the character is not always the best at gauging risks, especially when it comes to factoring in danger to others. The character receives a +10 modifier on all tests to resist fear or intimidation.

\subsection{Common Sense}
\label{sec:traits-common-sense}

\textbf{Cost:} 10 CP

The character has an innate sense of judgment that cuts through other distractions and factors that might cloud a decision. Once per game session, the player may ask the gamemaster what choice they should make or what course of action they should take, and the gamemaster should give them solid advice based on what the character knows. Alternately, if the character is about to make a disastrous decision, the gamemaster can use the character’s free hint and warn the player they are making a mistake.

\subsection{Danger Sense}
\label{sec:traits-danger-sense}

\textbf{Cost:} 10 CP

The character has an intuitive sixth sense that warns them of imminent threats. They receive a +10 modifier on Surprise Tests (p. 204).

\subsection{Direction Sense}
\label{sec:traits-direction-sense}

\textbf{Cost:} 5 CP

Somehow the character always knows which way is up, north, etc., even when blinded. The character receives a +10 modifier for figuring out complex di- rections, reading maps, and remembering or retracing a path they have taken.

%%% txt/148.txt

\subsubsection{Eidetic Memory (Ego Or Morph Trait)}
\label{sec:traits-eidetic-memory}
\textbf{Cost:} 10 CP

Much like a computer, the character has perfect memory recall. They can remember anything they have sensed, often even from a single glance. This works the same as the eidetic memory implant (p. 301).

\subsection{Exceptional Aptitude}
\label{sec:traits-exceptional-aptitude}

\textbf{Cost:} 20 CP

The character may raise one of their maximum aptitude up to 10 points over the normal aptitude cap (30 for flats, 35 for splicers, 40 for all others). Note that this trait just raises the maximum, it does not give the character more 10 aptitude points. This trait may be taken only once.

\subsection{Expert}
\label{sec:traits-expert}

\textbf{Cost:} 10 CP

The character is a legend in the use of one particular skill. The character may raise one learned skill over 80, to a maximum of 90, during character creation. This trait does not actually increase the skill, it just raises the maximum. This trait may only be taken once.

\subsection{Fast Learner}
\label{sec:traits-fast-learnier}

\textbf{Cost:} 10 CP

The character improves skills and learns new ones in half the time it normally takes (see Improving Skills, p. 152).

\subsection{First Impression}
\label{sec:traits-first-impression}

\textbf{Cost:} 10 CP

The character has a way of charming or otherwise making a good impression the first time they interact with someone. This innate social lubricant allows them to more readily deal with new contacts and slip right into new social environments. Apply a +10 modifier on social skill tests when the character is interacting with another character for the first time only.

\subsection{Hyper Linguist}
\label{sec:traits-hyper-linguist}

\textbf{Cost:} 10 CP

The character has an intuitive understanding of linguistic structures that facilitates learning new languages easily. The character requires one-third the normal amount of time and experience to learn any language (see Improving Skills, p. 152). The character can also learn any human language in one day simply by constant immersive exposure to it. Additionally, the character receives a +10 modifier when attempting to interpret languages they don’t know.

\subsection{Improved Immune System (Morph Trait)}
\label{sec:traits-improved-immune-system}

\textbf{Cost:} 10 (Level 1) or 20 (Level 2) CP

The morph’s immune system is robust and more resistant to diseases, drugs, and toxins—even more than basic bio-mods. At Level 1, apply a +10 modifier whenever making a test to resist infection or the effects of a toxin or drug. At Level 2, increase this modifier to +20. This trait is only available to biomorphs.

\subsubsection{Innocuous (Morph Trait)}
\label{sec:traits-innocuous}

\textbf{Cost:} 10 CP

In an age when exotic appearances and good looks are commonplace, the morph’s look is surprisingly bland and undistinguished, in that cookie cutter sort of way. The character’s physical looks are so mundane that others have a hard time picking them out of a crowd, describing their appearance, or otherwise remembering physical details. Apply a -10 modifier to all tests made to spot, describe, or remember the character. This modifier does not apply to psi or mesh searches.

\subsection{Limber (Morph Trait)}
\label{sec:traits-limber}

\textbf{Cost:} 10 (Level 1) or 20 (Level 2) CP

The morph is especially flexible and supple, capable of graceful contortions and interesting positions. At Level 1, the character can smoke with their toes, do the splits, and squeeze into small, cramped spaces. At Level 2, they are double-jointed escape artists. Each level provides a +10 modifier to escaping from bonds, fitting into narrow confines, and other acts relying on contortion or flexibility. This trait is only available to biomorphs.

\subsection{Math Wiz}
\label{sec:traits-mathwiz}
\textbf{Cost:} 10 CP

The character can perform any feat of calculation, including the most complex and advanced mathematics, instantly and with great precision, with the same ease an unmodified human can add 2 + 3. The character can calculate odds with great precision, find correlations in numerical data, and perform similar tasks with great ease. Apply a +30 modifier on tests involving math calculations.

\subsection{Natural Immunity (Morph Trait)}
\label{sec:traits-natural-immunity}

\textbf{Cost:} 10 CP

The morph has a natural immunity to a specific drug, disease, or toxin. When afflicted with that specific chemical, poison, or pathogen, the character re- mains unaffected. At the gamemaster’s discretion, this immunity may not apply to certain agents. It may not be applied to nanodrugs or nanotoxins. This trait is only available to biomorphs.

\subsection{Pain Tolerance (Ego Or Morph Trait)}
\label{sec:traits-pain-tolerance}

\textbf{Cost:} 10 (Level 1) or 20 (Level 2) CP

The character has a high threshold for pain tolerance and is better at ignoring the effects of pain on their abilities and concentration. Level 1 allows them to ignore the -10 modifier from 1 wound. Level 2 allows them to ignore the -10 modifiers from 2 wounds. This trait is only available for biomorphs.

\subsection{Patron}
\label{sec:traits-patron}

\textbf{Cost:} 30 CP

The character has an influential person in their life who can be relied on for occasional support. This could be a wealthy hyperelite family member, a high-ranking triad boss, or an anarchist networker with an unbeatable reputation. When called upon, this patron can pull strings on the character’s behalf, supply resources, introduce them to people they need to know, and bail them out of trouble. The player and gamemaster should work together to define exactly who this NPC is and what their relationship with the player character is. Specifically, the question of why this patron is supporting the character should be answered (familial obligation? childhood buddies? the character saved their life once?). Gamemasters should be careful that this trait does not get abused. The patron should be an occasional help (probably no more than once per game session at most) but is not always at the character’s beck-and-call. If the character asks for too much, too often, they should find the patron’s support drying up. Additionally, the character may need to take action to maintain the relationship, such as undertaking a mission on the patron’s behalf. In fact, the character may only have their patronage because they are on-call or of use to the NPC in some way.

\subsection{Psi}
\label{sec:traits-psi}
\textbf{Cost:} 20 CP (Level 1), 25 CP (Level 2)

The character has been infected with the MacLeod strain of the Exsurgent virus, which altered their brain structure and opened the potential for their mind to enhance their cognitive abilities and read and manipulate the biological minds of others (see Psi, p. 220). The character may purchase and learn psi sleights (p. 223). At Level 1, the character may only use psi-chi sleights. At Level 2, the character may use both psi-chi and psi-gamma sleights.

Though this trait is not very expensive, gamemasters should not allow it to be abused. There are a number of negative side effects to Watts-MacLeod infection, noted under Psi Drawbacks, p. 220.

\subsection{Psi Chameleon (Ego Or Morph Trait)}
\label{sec:traits-psi-chameleon}

\textbf{Cost:} 10 CP

The character’s mental state is naturally resistant to psi sensing. Apply a -10 modifier to any attempts to locate or detect the character via psi sleights.

\subsection{Psi Defense (Ego Or Morph Trait)}
\label{sec:traits-psi-defense}

\textbf{Cost:} 10 (Level 1) or 20 (Level 2) CP

The character’s mind is inherently resistant to mental attacks. At Level 1, apply a +10 modifier to all defense tests made against psi attacks. At Level 2, apply a +20 modifier.

\subsection{Rapid Healer (Morph Trait)}
\label{sec:traits-rapid-healer}

\textbf{Cost:} 10 CP

The morph recovers from damage more quickly. Reduce the timeframes for healing by half, as noted on the Healing table, p. 208. This trait is only available to biomorphs.

\subsection{Right At Home}
\label{sec:traits-right-at-home}

\textbf{Cost:} 10 CP

The character chooses one type of morph (splicer, neo-hominid, case, etc.). The character always feels right at home in morphs of this type. When resleeving into this type of morph, the character automatically adjusts to the new body, no Integration or Alienation Test needed, suffering no penalties and no mental stress.

\subsection{Second Skin}
\label{sec:traits-secondskin}

\textbf{Cost:} 15 CP

If your character background or faction enforces a restriction on your starting morph (for example, uplifts must start with an uplift morph), this trait allows you to ignore that restriction and purchase a starting morph of your choice.

%%% txt/150.txt \subsubsection{Situational Awareness} \textbf{Cost:} 10 CP

The character is very good at maintaining continuous partial awareness of the goings-on in their immediate environment. In game terms, they do not suffer the Distracted modifier on Perception Tests to notice things even when their attention is focused elsewhere, or when making Quick Perception Tests during combat.

\subsection{Striking Looks (Morph Trait)}
\label{sec:traits-striking-looks}

\textbf{Cost:} 10 (Level 1) or 20 (Level 2) CP

In an age where biosculpting is easy, good looks are both cheap and commonplace. This morph, however, possesses a physical look that can only be described as striking and unusual, but also somehow alluring and fascinating—even the gorgeous and chiseled glitterati take notice. On social skill tests where the character’s beauty may affect the outcome, they receive a +10 (for Level 1) or +20 (for Level 2) modifier. This modifier is ineffective against xenomorphs or those with the infolife or uplift backgrounds. This trait is only available to biomorphs.

This modifier may be purchased for uplift morphs, but at half the cost, and it is only effective against characters with that specific uplift background (i.e., neo-avians, neo-hominids, etc.).

The one drawback to this trait is that the character is more easily noticed and remembered.

\subsection{Tough (Morph Trait)} 
\label{sec:traits-tough}

\textbf{Cost:} 10 (Level 1), 20 (Level 2), or 30 (Level 3) CP

This morph is resilient than others of its type and can take more physical abuse. Increase their Durability by +5 per level (+5 at Level 1, +10 at Level 2, and +15 at Level 3). This also increases Wound Threshold by +1, +2, and +3 respectively.

\subsection{Zoosemiotics}
\label{sec:traits-zoosemiotics}

\textbf{Cost:} 5

A character with this trait and the Psi trait does not suffer a modifier when using psi sleights on nonsentient or partly-sentient animal species.

\section{Negative Traits}
\label{sec:negative-traits}

Negative traits generally hinder the character and apply negative modifiers in certain circumstances.

\subsection{Addiction (Ego Or Morph Trait)}
\label{sec:traits-addiction}

\textbf{Bonus:} 5 CP (Minor), 10 CP (Moderate), or 20 CP

\textbf{Major:} Addiction comes in two forms: mental (affecting the ego) and physical (affecting the biomorph). The character or morph is addicted to a drug (p. 317), stimulus (XP), or activity (mesh use) to a degree that impacts the character’s physical or mental health. Players and gamemasters should work together to agree on addictions that are appropriate for their game. Addiction comes in three levels of severity: minor, moderate, or major:

\textbf{Minor:} A minor addiction is largely kept under control—it does not ruin the character’s life, though it may create some difficulties. The character may not even recognize or admit they have a problem. The character must indulge the addiction at least once a week, though they can go for longer without too much difficulty. If they fail to get their weekly dose, they suffer a -10 modifier on all actions until they get their fix.

\textbf{Moderate:} A moderate addiction is in full swing. The character obviously has a problem, and must satisfy the addiction at least once a day. If they fail to do so, they may suffer mood swings, compulsive behavior, physical sickness, or other side effects until they indulge their craving. Apply a -20 modifier to all of the character’s actions until they get their fix. Additionally, a character with this level of addiction suffers a -5 DUR penalty.

\textbf{Major:} A character with a major addiction is on the rapid road to ruin. They face cravings every 6 hours, and suffer a -10 DUR penalty as their health is affected. If they fail to get their regular dosage, they suffer a -30 modifier on all actions until they do. If their life hasn’t already been ruined by their obsession, it soon will be.

\subsection{Aged (Morph Trait)} \textbf{Bonus:} 10 CP

The morph is physically aged, and has not been rejuvenated. Old morphs are increasingly uncommon, though some people adopt them hoping to gain an air of seniority and respectability. Reduce the character’s aptitude maximums by 5, and apply a -10 modifier on all physical actions.

This trait may only be applied to flat and splicer morphs.

\subsection{Bad Luck}
\label{sec:traits-bad-luck}

\textbf{Bonus:} 30 CP

Due to some inexplicable cosmic coincidence, things seem to go wrong around the character. The gamemaster is given a pool of Moxie points equal to the character’s Moxie stat, which also refreshes at the same rate as the character’s Moxie. Only the gamemaster may utilize this Moxie, however, and the purpose is to use it against the character. In other words, the gamemaster can use this bad Moxie to cause the character to automatically fail, flip-flop a roll, and so on. To reflect the black cloud that follows the character, the gamemaster can even use this bad Moxie against the character’s friends and allies, when they are doing something with or related to the character, though this should be used sparingly. Gamemasters who might be reluctant to sabotage the character should remember that the player asked for it by purchasing this trait.

%%% txt/151.txt \subsubsection{Blacklisted} \textbf{Bonus:} 5 or 20 CP

The character has managed to get themselves blacklisted in certain circles, whether they actually did something to deserve it or not. In game terms, the character is barred from having a Rep score higher than 0 in one particular reputation network. People within that network will refuse to help the character out of fear of reprisals and ruining their own reputation. The bonus for this trait is 20 CP if chosen for the rep network pertaining to the character’s own starting faction, and 5 CP if chosen for any other.

\subsection{Black Mark}
\label{sec:traits-black-mark}

\textbf{Bonus:} 10 (Level 1), 20 (Level 2), or 30 (Level 3) CP

At some point in the character’s past, they managed to do something that earned a black mark on their reputation. For some reason, no matter what they do, this black mark cannot be shaken off and continues to haunt their interactions. In game terms, the character picks one faction. Every time they interact with this faction (such as a Networking Test) or with an NPC from this faction (Social Skill Tests) who knows who the character is, they suffer a -10 modifier per level.

\subsection{Combat Paralysis}
\label{sec:traits-combat-paralysis}

\textbf{Bonus:} 20 CP

The character has an unfortunate habit of freezing in combat or stressful situations, like a deer caught in headlights. Anytime violence breaks out around the character, or they are surprised, the character must make a Willpower Test in order to act or respond in any way. If they fail the test, they lose their action and simply stand there, remaining incapable of reacting to the situation.

\subsection{Edited Memories}
\label{sec:traits-edited-memories}

\textbf{Bonus:} 10 CP

At some point in the character’s past, the character had certain memories strategically removed or otherwise lost to them. This may have been done to intentionally forget an unpleasant or shameful experience or to make a break with the past. The memory may also have been lost by an unexpected death (with no recent backup), or it may have been erased against the character’s will. Whatever the case, the memory should bear some importance, and there should exist either evidence of what happened or NPCs who know the full story. This is a tool the gamemaster can use to haunt the character at some future point with ghosts from their past.

\subsection{Enemy}
\label{sec:traits-enemy}

\textbf{Bonus:} 10 CP

At some point in their past, the character made an enemy for life who continues to haunt them. The gamemaster and player should work out the details on this enmity, and the gamemaster should use the enemy as an occasional threat, surprise, and hindrance.

\subsection{Feeble}
\label{sec:traits-feeble}

\textbf{Bonus:} 20 CP

The character is particularly weak with one aptitude. That aptitude must be purchased at a rating lower than 5, and may never be upgraded during character advancement. The aptitude maximum is 10, no matter what morph the character is wearing.

\subsection{Frail (Morph Trait)}
\label{sec:traits-frail}

\textbf{Bonus:} 10 (Level 1) or 20 (Level 2) CP

This morph is not as resilient as others of its type. Its Durability is reduced by 5 per level. This also reduces Wound Threshold by 1 or 2, respectively.

\subsection{Genetic Defect (Morph Trait)}
\label{sec:traits-genetic-defect}

\textbf{Bonus:} 10 CP or 20 CP

The morph is not genefixed, and in fact suffers from a genetic disorder or other impairing mutation. The player and gamemaster should agree on a defect appropriate to their game. Some possibilities include: heart disease, diabetes, cystic fibrosis, sickle-cell disease, hypertension, hemophilia, or color blindness. A genetic disorder that creates minor complications and/or occasional health problems would be worth 10 CP, a defect that significantly impairs the character’s regular functioning or that inflicts chronic health problems is worth 20 CP. The gamemaster must determine the exact effects of the disorder on gameplay, as appropriate.

This trait is only available for flats.

\subsection{Identity Crisis}
\label{sec:traits-identity-crisis}

\textbf{Bonus:} 10 CP

The character’s ego has trouble adapting itself to the changed look of a new morph—they are stuck with the mental image of their original body, and simply do not grow accustomed to their new face(s). As a result, the character has difficulty identifying themselves in the mirror, photos, surveillance feeds, etc. They frequently forget the look and shape of their current morph, acting inappropriately, describing themselves by their original body, forgetting to duck when walking through doorways, etc. This is primarily a roleplaying trait, but the gamemaster may apply appropriate modifiers (usually -10) to tests affected by this inability to adapt.

\subsection{Illiterate}
\label{sec:traits-illiterate}

\textbf{Bonus:} 10 CP

The character knows how to speak, but has difficulty reading or writing. Due to the entoptic-saturated and icon-driven nature of transhuman society, they are able to get by quite comfortably with this handicap. Reduce the character’s Language skills by half (round down) whenever reading or writing.

\subsection{Immortality Blues}
\label{sec:traits-immortality-blues}

\textbf{Bonus:} 10 CP

The character has lived so long - over 100 years - they’re bored with life and now have difficulty motivating themselves. They were old when longevity treatments first became available, survived the Fall, and continue to soldier onward—though they find it increasingly harder to care, take interest in things around them, or fear final death. The character only receives half the Moxie and Rez Points award for completing motivational goals.

This trait may not be purchased by characters with the infolife or uplift backgrounds.

\subsection{Implant Rejection (Morph Trait)}
\label{sec:traits-implant-rejection}

\textbf{Bonus:} 5 (Level 1) or 15 (Level 2) CP

This morph does not accept implants well. At Level 1, any implants acquired are more expensive as they required specialized anti-rejection treatments. Increase the Cost category of the implant by one. At Level 2, the morph cannot accept implants of any kind.

\subsection{Incompetent}
\label{sec:traits-incompetent}

\textbf{Bonus:} 10 CP

The character is completely incapable of performing a particular chosen active skill, no matter any training they may receive. They may not buy this skill during character creation or later advancement, and the modifier for defaulting to the linked aptitude of this particular skill is -10. This may not be used for exotic weapon skills, and should be used for a skill that could be of use to the character.

\subsection{Lemon (Morph Trait)}
\label{sec:traits-lemon}

\textbf{Bonus:} 10 CP

This trait is only available for synthetic morphs. This particular morph has some unfixable flaws. Once per game session (preferably at a time that will maximize drama or hilarity), the gamemaster can call for the character to make a MOX x 10 Test (using their current Moxie score). If the character fails, the morph immediately suffers 1 wound resulting from some mechanical failure, electrical glitch, or other breakdown. This wound may be repaired as normal.

\subsection{Low Pain Tolerance (Ego Or Morph Trait)}
\label{sec:traits-low-pain-tolerance}

\textbf{Bonus:} 20 CP

Pain is the character’s enemy. The character has a very low threshold for pain tolerance and is more severely impaired when suffering. Increase the modifier for each wound take by an additional -10 (so the character suffers -20 with one wound, -40 with another, and -60 with a third). Additionally, the character suffers a -30 modifier on any test involving pain resistance. This morph version of this trait is only available for biomorphs.

\subsection{Mental Disorder}
\label{sec:traits-mental-disorder}

\textbf{Bonus:} 10 CP

You have a psychological disorder from a previous traumatic experience in your life. Choose one of the disorders listed on p. 211.

\subsection{Mild Allergy (Morph Trait)}
\label{sec:traits-mild-allergy}

\textbf{Bonus:} 5 CP

The morph is allergic to a specific chosen allergen (dust, dander, plant pollen, certain chemicals) and suffers mild discomfort when exposed to it (eye irritation, sneezing, difficult breathing). Apply a -10 modifier to all tests while the character remains exposed. This trait is only available for biomorphs.

\subsection{Modified Behavior}
\label{sec:traits-modified-behaviour}

\textbf{Bonus:} 5 (Level 1), 10 (Level 2), or 20 (Level 3) CP

The character has been conditioned via time-accelerated behavioral control psychosurgery. This is common among ex-felons, who have been conditioned to respond to a specific idea or activity with vehement horror and disgust, but may have occurred for some other reason or even been self-inflicted. At Level 1, the chosen behavior is either limited or boosted, at Level 2 it is either blocked or encouraged, and at Level 3 it is expunged or enforced (see p. 231 for details). This trait should only be allowed for behaviors that are either limited or, if encouraged, impact the character in a negative way.

\subsection{Morphing Disorder}
\label{sec:traits-morphing-disorder}

\textbf{Cost:} 10 (Level 1), 20 (Level 2), or 30 (Level 3) CP

Adapting to new morphs is particularly challenging for this character. The character suffers a -10 modifier per level on Integration Tests and Alien- ation Tests (p. 272).

\subsection{Neural Damage}
\label{sec:traits-neural-damage}

\textbf{Bonus:} 10 CP

The character has suffered some type of neurological damage that simply cannot be cured. The affliction is now part of the character’s ego and remains with them even when remorphing. This damage may have been inherited, it may have resulted from a poorly designed morph or implant, or it may have been inflicted by one of the TITAN nanovirii that targeted neural systems during the Fall (p. 384). The gamemaster and player should agree on a specific disorder appropriate to their game. Some possibilities are:

\begin{itemize}
\item Partial aphasia (difficulty communicating or using words)
\item Color blindness
\item Amusica (inability to make or understand music)
\item Synaesthesia
\item Logorrhoea (excessive use of words)
\item Loss of face recognition
\item Loss of depth perception (double range modifiers)
\item Repetitive behavior
\item Mood swings
\item The inability to shift attention quickly
\end{itemize}

The gamemaster may decide to inflict modifiers resulting from this affliction as appropriate.

%%% txt/153.txt \subsubsection{No Cortical Stack (Morph Trait)} \textbf{Bonus:} 10 CP

The morph lacks the cortical stack that is common to morphs of its type. This means the character cannot be resleeved from the cortical stack if the character dies, they can only be resleeved from a standard backup. This trait is not available for flats.

\subsection{Oblivious}
\label{sec:traits-oblivious}

\textbf{Bonus:} 10 CP

The character is particularly oblivious to events around them or anything other than what their attention is focused on. They suffer a -10 modifier to Surprise Tests and their modifier for being Distracted is -30 rather than the usual -20 (see Basic Perception, p. 190).

\subsection{On The Run}
\label{sec:traits-on-the-run}

\textbf{Bonus:} 10 CP

The character is wanted by the authorities of a particular habitat/station or faction, who continue to actively search for the character. They either commit- ted a crime or somehow displeased someone in power. The character deals with that faction in question at their own risk, and may occasionally be forced to deal with bounty hunters.

\subsection{Psi Vulnerability (Ego Or Morph Trait)}
\label{sec:traits-psi-vulnerability}

\textbf{Bonus:} 10 CP

Something about the character’s mind makes them particularly vulnerable to psi attack. They suffer a -10 modifier when resisting such attacks. The morph version of this trait may only be taken by biomorphs.

\subsection{Real World Naiveté}
\label{sec:traits-real-world-naivite}

\textbf{Bonus:} 10 CP

Due to their background, the character has very limited personal experience with the real (physical) world—or they have spent so much time in simulspace that their functioning in real life is impaired. They lack an understanding of many physical properties, social cues, and other factors that people with standard human upbringings take for granted. This lack of common sense may lead the character to misunderstand how a device works or to misinterpret someone’s body language.

Once per game session, the gamemaster may intentionally mislead the character when giving them a description about some thing or some social in- teraction. This falsehood represents the character’s misunderstanding of the situation, and should be roleplayed appropriately, even if the player realizes the character’s mistake.

This trait should only be available to characters with the infolife or reinstantiated backgrounds, though the gamemaster may allow it for characters who have extensive virtual reality/XP use in their personal histories.

\subsection{Severe Allergy (Morph Trait)}
\label{sec:traits-severe-allergy}

\textbf{Bonus:} 10 (uncommon) or 20 (common) CP

The morph’s biochemistry suffers a severe allergic reaction (anaphylaxis) when it comes into contact (touched, inhaled, or ingested) with a specific allergen. The allergen may be common (dust, dander, plant pollen, certain foods, latex) or uncommon (certain drugs, insect stings). The player and gamemaster should agree on an allergen that fits the game. If exposed to the allergen, the character breaks into hives, has difficulty to breathing (-30 modifier to all actions), and must make a DUR Test or go into anaphylactic shock (dying of respiratory failure in 2d10 minutes unless medical care is applied). This trait is only available to biomorphs.

\subsection{Slow Learner}
\label{sec:traits-slow-learner}

\textbf{Bonus:} 10 CP

New skills are not easy for this character to pick up. The character takes twice as long as normal to improve skills or learn new ones (p. 152).

\subsection{Social Stigma (Ego Or Morph Trait)}
\label{sec:traits-social-stigma}

\textbf{Bonus:} 10 CP

An unfortunate aspect of the character’s background means that they suffer from a stigma in certain social situations. They may be sleeved in a morph viewed with repugnance, be a survivor of the infamous Lost generation, or may be an AGI in a post-Fall society plagued by fear of artificial intelli- gence. In social situations where the character’s nature is known to someone who view that nature with distaste, fear, or repugnance, they suffer a -10 to -30 modifier (gamemaster’s discretion) to social skill tests.

\subsection{Timid}
\label{sec:traits-timid}

\textbf{Bonus:} 10 CP

This character frightens easily. They suffer a -10 modifier when resisting fear or intimidation.

\subsection{Unattractive (Morph Trait)}
\label{sec:traits-unattractive}

\textbf{Bonus:} 10 CP (Level 1), 20 CP (Level 2), 30 CP (Level 3)

In a time when good looks are easily purchased, this morph is conspicuously ugly. As unattractiveness is increasingly associated with being poor, backward, or genetically defective, responses to a lack of good looks range from distaste to horror. The character suffers a -10 modifier on social tests for Level 1, -20 for Level 2, and -30 for Level 3.

Only biomorphs may take this trait. This modifier does not apply to interactions with xenomorphs or those with the infolife or uplift backgrounds. This modifier may be purchased for uplift morphs, but at half the bonus, and it is only effective against characters with that specific uplift background (i.e., neo-avians, neo-hominids, etc.).

\subsection{Uncanny Valley (Morph Trait)}
\label{sec:traits-uncanny-valley}

\textbf{Bonus:} 10 CP

There is a point where synthetic human looks become uncannily realistic and human-seeming, but they remain just different enough that their looks seem creepy or even repulsive—a phenomenon called the “uncanny valley.” Morphs whose looks fall into this range suffer a -10 modifier on social skill tests when dealing with humans. This modifier does not apply to interactions with xenomorphs or those with the infolife or uplift backgrounds.

\subsection{Unfit (Morph Trait)}
\label{sec:traits-unfit}

\textbf{Bonus:} 10 CP (Level 1), 20 CP (Level 2)

The morph is either not optimized for health and/or just in bad shape. Reduce the aptitude maximums for Coordination, Reflexes, and Somatics by 5 (Level 1 ) or 10 (Level 2).

\subsection{VR Vertigo}
\label{sec:traits-vr-vertigo}

\textbf{Bonus:} 10 CP

The character experiences intense vertigo and nausea when interfacing with any type of virtual reality, XP, or simulspace. Augmented reality has no effect, but VR inflicts a -30 modifier to all of the character’s actions. Prolonged use of VR (gamemaster’s discretion) may actually incapacitate the character should they fail a WIL x 2 Test.

\subsection{Weak Immune System (Morph Trait)}
\label{sec:traits-weak-immune-system}

\textbf{Bonus:} 10 (Level 1) or 20 (Level 2) CP

The morph’s immune system is susceptible to diseases, drugs, and toxins. At Level 1, apply a -10 modifier whenever making a test to resist infection or the effects of a toxin or drug. At Level 2, increase this modifier to -20. This trait is only available to biomorphs.

\subsection{Zero-G Nausea (Morph Trait)}
\label{sec:traits-zero-g-nausea}

\textbf{Bonus:} 10 CP

This morph suffers from space sickness and does not fair well in zero-gravity. The character suffers a -10 modifier in any microgravity climate. Additionally, whenever the character is first getting acclimated or anytime they must endure excessive movement in microgravity, they must make a WIL Test or spend 1 hour incapacitated by nausea per 10 points of MoF.

\section{Character Advancement}:
\label{sec:character-advancement}

As characters accomplish goals and gather experience during gameplay, they accumulate Rez Points (see Awarding Rez Points, p. 384). Rez Points may be used to improve the character’s skills, aptitudes, and other characteristics per the following rules. The costs for spending Rez Points for advancement are the same as the costs for spending Customization Points.

\begin{quotation}
Spending Rez Points

\begin{itemize}
\item 15 RP = 1 Moxie point
\item 10 RP = 1 aptitude point
\item 5 RP = 1 psi sleight
\item 5 RP = 1 specialization
\item 2 RP = 1 skill point (61-99)
\item 1 RP = 1 skill point (up to 60)
\item 1 RP = 10 Rep
\item 1 RP = 1,000 Credits
\end{itemize}
\end{quotation}

\subsection{Changing Motivation}
\label{sec:changing-motivation}

It is only natural that over time a character’s driving goals and interests will change. The character may reach a turning point where they feel certain personal agendas have been fulfi lled and it is time to move on, or they have failed and need to be discarded. New urgencies or philosophies may have entered the character’s life, or the character may have become disenchanted with particular memes and ideas they previously took to heart.

Changing a character’s motivation does not cost Rez Points, but it is something that should only happen in accordance with roleplaying and with life-altering events. Players should not be allowed to simply switch their motivations at whim, there should be a driving reason or explanation for doing so. For this reason, changing a motivation should only happen when the player and gamemaster discuss the matter and both agree that the swap is appropriate to the character’s development and circumstances. If these conditions are met, the character simply drops a previously held motivation and takes on a new one. Only one motivation should be switched out at a time.

\subsection{Switching Morphs}
\label{sec:switching-morphs}

Resleeving—switching from one morph to another— is handled as an in-character interaction, not with Rez Points. See Resleeving, p. 271.

\subsection{Improving Aptitudes}
\label{sec:improving-aptitudes}

Aptitudes may be raised with Rez Points at the cost of 10 RP per aptitude point. This represents the character’s improvement in their core characteristics, gained from exercise, learning, and experience. Aptitudes may not be raised above 30 (bonuses from morphs, implants, traits, or other sources do not count towards this total). Raising the value of an aptitude also raises the value of all linked skills by an equivalent amount. If this raises any linked skills over 60, an additional 1 RP must be spent per linked skill over 60.

\subsection{Improving Skills}
\label{sec:improving-skills}

Characters may also spend Rez Points to increase existing skills or learn new ones. To improve an existing skill, the character must have successfully used that skill in the recent past or must actively practice it in order to get better, perhaps with the aid of an instructor. In the case of Knowledge skills, this means actively studying. As a rough timeframe, this should require around 1 week of learning per skill point. A number of educational resources are freely available via the mesh, though some areas of interest may be restricted or hard to fi nd. This can be handled via roleplaying or designated as something the character is doing during downtime between sessions. If the gamemaster decides that a character has not put enough effort into improving a skill, they may call for more practice/study. The cost to increase a skill is 1 RP per skill point, and no skill may be increased over 99. No skill may be raised by more than 5 points per month. When a character’s skill reaches the level of expertise (skill of 60+), however, they tend to reach a plateau where improvement progresses more slowly and even consistent practice and study have diminished returns. In this case, the Rez Point cost per skill point doubles (i.e., 2 RP = +1 skill point). When a skill reaches 80, improvement slows down even further—a skill of 80+ may not be increased by more than 1 point per month.

\subsection{Learning New Skills}
\label{sec:learning-new-skills}

Similarly, to learn a new skill, the character must actively study/practice and/or seek instruction. No test to learn is required, unless the period of study was hampered or in some way defi cient, in which case the gamemaster may call for a COG x 3 Test to pick up the new skill. Otherwise, once a character has spent approximately a week learning a new skill, they may purchase their fi rst skill point at the usual cost (1 RP). The skill is bought up from the aptitude rating, per normal. Once a new skill is acquired, it is raised according to the standard rules above.

\subsection{Specializations}
\label{sec:new-specializations}
Specializations may be purchased for existing skills, as long as that skill is at least rating 30. Specializations require a total of 1 month of training. The cost to learn a specialization is 5 RP. Only 1 specialization may be purchased per skill.

\subsection{Improving Moxie}
\label{sec:improving-moxie}

Moxie may be raised at the cost of 15 RP per Moxie point. The maximum to which Moxie may be raised is 10.

\subsection{Gaining/Losing Traits}
\label{sec:gaining-losing-traits}

At the gamemaster’s discretion, both positive and negative traits may be acquired or lost during gameplay, though such changes should be rare and only made in accordance with the storyline and unfolding events in the game. Both positive and negative traits may be picked up by a character during gameplay as a consequence of something that did or something that happened to them. In the case of a positive trait, the character must immediately spend Rez Points equal to the trait’s CP cost for the privilege (whether they wanted the new trait or not). If the character has no unspent RP available, they must pay out immediately from any future RP they earn until the debt is paid off. In the case of a negative trait, however, the character is simply saddled with the new fl aw—they do not acquire any extra RP for gaining the negative trait. Getting rid of traits is somewhat more diffi cult. Positive traits may be lost due to unfortunate effects on the character, as the gamemaster sees fi t. Such lost positive traits are simply gone—the character does not receive any Rez Point reimbursement. Negative traits are occasionally eliminated in the same way, but more typically they can only be worked off through the hard work and diligence of a character that seeks to overcome their handicap. Such endeavors should require weeks if not months of effort on the character’s part, with appropriate roleplaying and possibly some diffi cult tests. In fact, overcoming such traits could be the source of an adventure. Once a gamemaster feels that the character has made a strong-enough effort, the character may pay a number of Rez Points equal to the trait’s original CP bonus to negate it. Note, however, that some negative traits may simply not be discarded, no matter what the character does.

\subsection{Improving Rep}
\label{sec:improving-rep}

Reputation is something that can be increased with appropriate roleplaying and actions during gameplay (see Reputation Gain and Loss, p. 384). Characters that prefer to handle their Rep-boosting activities “off-screen,” however, can simply spend Rez Points to boost their score(s). Each RP spent boosts the character’s Rep by +10 in a single network. Only one such boost may be made to a single rep network per month.

\subsection{Making Credit}
\label{sec:making-credit}

Rez Points may be spent on Credit at a ratio of 1 RP for 1,000 Credits. This represents income the character has earned “off-screen” or during downtime, such as from odd jobs, selling off possessions, and so on.

\subsection{Improving Psi}
\label{sec:improving-psi}

Characters who have the Psi trait (p. 147) may purchase new sleights (see Sleights, p. 223) at the cost of 5 RP per sleight. Sleights must be learned through study, training, and practice, requiring approximately 1 month per sleight. No more than one sleight may be learned per month.

%%% EXAMPLE CHARACTERS SKIPPED

%%% Local Variables: %%% mode: latex %%% TeX-master: "ep" %%% End: 
