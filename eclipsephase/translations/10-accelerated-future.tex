



\chapter{Futur Accéléré} \label{cha:accelerated-future} 



















Le cadre futuriste de \textit{Eclipse Phase} introduit un grand nombre d'éléments technologiques qui ont un impact élevé sur la société transhumaine. Cela inclut entre autres la sauveagrde et l'upload, la réincarnation, l'égocast, le frok, la nano-fabrication, les systèmes de réputations, les habitats spataiux et le voyage spatial. 

\section{Suavegarde et Upload} 

L'esprit Transhumain n'est plus prisonnier du matériel biologique duquel il est originaire. Grâce à différents mécanismes, les cerveaux bilogiques peuvent être émulés numériquement, permettant aux personnes de faire une \textit{sauvegarde} de leur esprit, incluant toute leur personnalité, leurs souvenirs et compétence - un procédé appelé \textit{upload.} 

L'utilisation principale des sauvegarde et de garantir que l'ego de la personne pourra être récupéré en cas de mort, ce qui dans ce cas leur pemettra de se \textit{réincarner} (p. 271). Pour cette raison, presque tout le monde dans le système solaire est équipé d'une pile corticale (p. 300). Les sauvegarde peuvent également être archivé dans des stockages sécurisés (p. 269) ou utilisés pour créer des infomorphs (p. 264). Une personne peut aussi s'egocast à travers le systèmle solaire comem moyen de voyager (p. 276). 

\subsection{Sauvegardes par Piles Corticale} 

Les implants de piles corticales déploient un réseau de nanobots dans le cerveau qui rpend un paerçu de l'état neural de l'esprit, sauvegardant les données à l'intérieur de la pile corticale. La pile corticale transhumaine moyenne sauvegarde 86,400 fois l'ego par jour. Seul la sauvegarde la plus récente est préservée dans la pile; les plus anciens sont écrasés. Les pods et les synthmorphs peuvent aussi être équipés avec des piles corticales (bien que les bots pilotés par des IA n'ont géénralement pas cette caractériqtique), bien que ces versions maintienent une copie à jour de l'ego fonctionnant das le cybercerveau de la morph. 

En cas de mort, accidentelle ou non, une pile corticale peut-être récupérée sur un cadavre et être utilisée pour restaurer le personnage, soit à l'état d'infomorph, soit en le réincarnant dans une nouvelle morph. Les piles corticales sont blindées au diament, et peuvent donc être souvent récupérées, même si le corps est gravement attaqué ou endommagé. Si le cadavre ne peut pas être récupéré ou si la pile corticale est détruite, la sauvegarde est perdue. 

Les risques-tout, les bordés sur-équippés et d'aures ayant une profession dangereuse optent souvent pour un farcaster d'urgence en accessoire (p. 306) qui transmet périodiquement (habituellement toutes les 48 heures, mais cela peut varier en fonction du contrat) une sauveagrde de la pile corticale vers une installation de stockage distante. Cette option reste relativement chère et n'est abordabel que par les plus riches. 

\subsubsection{Récupérer Une pile Corticale} 

La plupart des piles corticales sont minutieusement excisées du cadavre par chirugie. Dans certaines circonstances un personnage peut avoir besoin d'extraire une pile corticale sur le terrain, que cela soit parceque transporter le cadavre n'est pas pratique ou parceque le mort est un  ennemis et que le personnage ne veut pas qu'il sache qui l'a tué ou qu'il veuillent l'interroger avec de la psychochirurgie en simulspace. 

La procédure pour extraire la pile corticale est appelée "poper," car un extracteur compétent peut habituellement faire sortir la coque lisse de l'implant en faisant une incision au bon endroit et en y appliquant une pression. Il faut faire attention à ce que la petite pile dégoulinante de sang ne glisse pas au loin lorsqu'elle est popée. 

On peut poper une pile avec un couteau aiguisé et de l'huile de coude, bien que cela soit sanglant. Poper une pile est une Action de Tâche qui nécessite un Test de médecine: [n'importe quel domaine utile] avec un Intervalle d'1 minute et un modificateur de +20. Les morphs avec une pile à un endorit non-standard ou avec un renforcement anatomique (plaques de carapaces, etc.) autour de la pile peuvent infliger des pénalités à ce test, à la discrétion du maître de jeu. Bien entendu, si vous n'avez pas le temps pour une extraction précise, vous pouvez toujouts enlever toute la tête et l'emmener avec vous. 

Une fois que la pile corticale est récupérée, elel peut-être chargée dans un connecteur d'ego (p. 328) et utilisée pour ramener l'ego, en tant qu'infomorph ou par la réincarnation. 

\textbf{Sujets Vivants:} Les piles corticales peuvent être excisée d'une personne vivante, mais le procédé est généralement fatal (ou au moins apralysant) car il implique de sectionner la colonne vertébrale. Si la cible n'est pas inconsciente ou incapacitée d'une manière ou d'une autre, elel doit d'abord être immobilisée en coimbat au corps à corps (voir \textit{Contrôle, } p. 204). Extraire la pile est considéré comme une Action de Tâche comme ci-dessus, mais le procédé inflige 3d10 + 10 points de dégats à la cible. Si le test échoue, il inflige tout de même 1d10 + 10 dégat à la cible. Si la personne supprimant la pile veut laisser la cible vivante ou la blesser le moins possible, elle subit un modificateur de -20 à son test, mais réduit les dommage d'1d10 par tranche de 10 points de MdR. Subir l'opération d'extraction de votre pile est traumatisant; quiconque subissant cette opération souffre d'1d10 points de stress mental. 

\subsubsection{Détruire Une Pile Corticale} 

Les piles corticales ont une Armure de 20 et une Solidité de 20 pour quiconque essaye de les détruire. 

\subsection{Upload} 

Uploader une sauveagrde dans un stockage sécuréis est générallement effectué par une analyse cérébrale à la clinique de l'installation de stockage en utilisant une unité grosse comme un grille pain et appelée un \textit{connecteur d'ego} (p. 328). Lorsqu'il est activé, la matrice de capteur du connecteur d'ego s'ouvre comme un tournesol, révélant une boîte équipée d'un appui-tête qui s'ajuste automatiquement à la tête des morphs, même de celles les plus étrange. L'appui tête déploie des millions de nanobots spécialisés 

dans le cerveau et le système nerveux central. Les pétales sont remplis de capteurs qui prennent des images du cevreau en utilisant une combinaison d'IRM, de sonogramme et de la diffusion d'information positionelle faites par les nanobots dans le cerveau de la morph. Le connecteur d'ego construit ensuite une copie du cerveau de la personne, qui est ensuite stockée dans les entrepœts de données filaire, hautement sécurisé et déconnectés du mesh du service. 

Dans le cas des pods, le connecteur d'ego analyse les parts du cerveau biologique et accède également au cybercerveau pour copier les parties de l'ego qui y résident. Pour les synthmorphs, qui n'ont pas de cerveau bilogique, le procédé est bien plus simple puisqu'il nécessite de simplement accéder au cybercerveau et de le copier. 

Dans une clinique standard et avec une morph non abîmée, l'upload ne prend que 10 minutes, 5 avec un pod. Dans d'autres situations, l'opération peut durer plus longtemps si le maître de jeu le désire. L'upload depuis une synthmorph ou une pile corticale extraite est instantané. Le connecteur d'ego opère essnetiellement de lui-même. Bien qu'une supervision par un spécialiste médical est une bonne idée, aucun test n'est nécessaire. 

Si un personnage uploadé ne prévoit pas de retourner dans sa morph, elle est générallement cryogénisée jusqu'à ce que quelqu'un s'y réincarne. Si une nouvelle incarnation n'est pas disponible et que le personnage s'uploadant ne veut pas laisser une copie potentielle de lui-même derrière lui, il peut complètement effacer l'esprit de la morph par les nanobots lors de l'opération. 

\subsubsection{Continuité Upload-Réincarnation} 

Dans des circonstances idéales, une personne qui se réincarne intentionnellement (p. 271) peut s'arranger pour que les opérations d'upload et de réincarnation se déroulent sans perte de continuité notable. Bien que l'expérience de basculer d'une morph à une autre est toujours un peu perturbant, la transition en elle-même peut-être faite lors d'une opération propre, sans faille de conscience ou de souvenir, ce qui aide à réduire le stress mental associé. 

Dans ce cas, pendant le processuss d'upload, le connecteur d'ego est également connecté à un autre connecteur et à la nouvelle incarnation. Cette connexion peut même être sans-fil ou par un farcaster (aevc une distance maximale de 10,000 kilomètres). 

Alros que l'esprit est uploadé, le connecteur d'ego construit un cerveau virtuel en copiant le cerveau de la morph bit par bit, utilisant les données récupérées de l'analyse du cerveau. En même temps, ces données sont lentement copiées dans la nouvelle incarnation laors que les nanobots recâblent la strructure cognitive de la morph (une opération bien plus lente). Pendant le trasnfert, les nanobots dans le cerveau coupent les connexions neurale et les reroutes vers leur double dans le cerveau virtuel, pusi dans le nouveau cerveau. En pratique, l'efo du personnage tourne partiellement sur le c erveau de viande et partiellement sur la copie virtuelle de celui-ci. Au moment où les nanobots coupent la dernière connexion neurale dans le vieux cerveau, l'ego fonctionne complètement sur le cerveau virtuele t sur celui de la nouvelle incarnation. une fois que la réincarnation est complète, le cerveau virtuel est éteint. 

En terme de perception, le personnage, qui est éveillé pendant le procédé, perçoit un glissement trés graduel d'une morph vers l'autre. COmme le procédé peut prendre des heures (voire plus si l'opération se fait via un farcasteur), le sujet se divertit générallement avec des média RA, RV ou même en XP pour passer le temps. 

\subsubsection{Upload Après la Mort} 

Il est possible d'uploader l'esprit d'une personne qui est récemment décédée tant que les nanobots ont le temps de scanner le cerveau avant que la déterioration cellullaire n'intevienne de manière trop devastatrice, générallement au bout de 2 heures. Il est possible de maintenir un cadavre pour une durée plus longue en le plaçant dans une cuve de soin (p. 326) pour une nanostase. L'upload post-mort peut subir des pertes d'intégrité; voir \textit{Complication de Sauvegarde}\textit{, } p. 270. Les cybercerveaux peuvent aussi être récupérés depuis une synthmorph détruite et réactivée, du moment qu'elle n'a pas été trop lourdement endommagée (à la discrétion du maître de jeu). 

\subsubsection{Upload Destructif} 

Bien que cela soit rare, des personnes se lancent dans un procédé appelé upload destrcutif dans lequel le cerveau bilogique est littéralement découpé en tranche et analysé morceau par morceau. Considérer comme exécrable et dispendieux par la plupart des transhumains, le décorticage de cerveau est pratiqué par certaines factions bioconservative qui la considère comem la seule méthode "pure" d'upload ou comme la seule réelle façon de transférer l'âme." De telles personnes refusent typiquement de se réincarner, vivant le reste de leur vie en tant qu'infomorph, généralement dans des simulspaces déidées et considérés comme une sorte d'au-delà virtuel. 

\subsection{Assurance Sauvegarde} 

Presque tout le monde, à l'exception des néo-primitivistes et des trés jeune enfants, a une pile corticale. En cas de mort, une pile corticale seule ne permete cependant pas d'assurer la résurection, sauf si vous avez une assurance sauvegarde (p. 330) pour couvrir les frais de votre réincarantion. Vivre sans assurance sauvegarde pour n'importe quelle durée représente un risque énorme. Certaines juridictions (comme le Commonwealth Titanien) ont pour pratique de ramener tout le monde, même si c'est juste à l'état d'une infomorph, ou d'au moins enregistrer la dernière sauvegarde en mémoire morte, au cas où quelqu'un décide de payer pour le resusciter plus tard. D'autres autorités détruirons simplement la pile ou, pire, la revendront sur le marché noir à un syndicat trafficant les âmes tel que Nine Lives. Les assurances sauvegarde incluent typiquement un abonnement à une instalaltiond d'upload, nécessitant généralement une visite tous les 6 mois, pour s'assurer que cette sauvegarde est conservée dans un stockage sécurisé en cas de perte de la pile corticale. Les personnes avec des métiers à risque (superviseur de bots de construction, personnel hypercorporatiste sur des exoplanète, fille combattant des anguilles génate et vicieuses pour un public riche et blasé, etc.) peuvent se sauvegarder une fois par semaine, voire même quotidiennement. En cas de mort vérifiée et si la pile corticale ne peut pas être récupérée, la sauvegarde la plus récente est utilisée pour réincarner la personne. Au niveau le plus basique, l'assurance sauvegarde ramènera le personnage en infomorph, point à partir duquel ils puevent accéder à leur crédit et acheter une nouvelle morph. Des versions plus chère permettront de vous réincarner automatiquement 

dans une morph pré-acheté de votre choix. Les personnes excessivement riches auront souvent des clônes personnalisés (souvent de leur corps originel) attendant dans la glace. 

Les assurances sauvegardes impliquent souvent une clause de personne manquante, qui établit qu'une personne sera ramené si elle n'a pas donné signe de vie pendant une période X (une fonction calendaire gérée automatiquement par la muse) et qu'elle ne peut être localisée. 

A noter que certains syndicats du crime proposent également des assurances sauvegarde à des prix réduit. la probabilité qu'une copie de votre sauvegarde soit utilisées dans un but illicite est cependant relativement élevée. Cependant, certaines personnes ne se préoccupent pas de savoir ce qui peut arriver à une copie d'eux-même. 

\subsubsection{Limite des Assurances Sauvegarde} 

Les assurances sauvegardes ne sont pas toujours parfaites. Bien que les assureurs sont obligés de faire un effort raisonnable pour récupérer votre pile corticale, pour beaucoup d'hypercorp c'est simplement une analyse coût/bénéfice qui ne fonctionnera que rarement en faveur du personnage. Si vous êtes mort dans une zone dangereuse telle que la Zone sur Mars, dasn une zone éloignée telle que la Ceinture de Kuiper, ou vous êtes simplement difficile à pister (pousser hors d'un sas quelque part), la probabilité que votre pile corticale soit récupérée est très faible - vous serez plutôt réinstantié à partir d'une sauvegarde. 

La juridiction peut également avoir un rôle important. l'assurance offerte par de nombreux assureurs du système intérieur sont automatiquement annulées si vous voyagez sur un habitat anrchiste, que vous resquilliez, violiez la loi ou vous lanciez dans certaines activités pouvant menacer votre vie comme le suicide sprotif ou la récupération dans les ruines infectés par les TITAN. Ils refuseront au minimum de récupérer votre pile dans ces circonstances. De manière similaire, si vous négociez une assurance sauvegarde avec un collectif médical d'un habitat autonomiste et allez ensuite mourir sur une stationhypercorporatiste, l'hypercorporation refusera trés probablement de reconnaître l'autorité d'une bande d'anarchiste et ils ne leur renverrons pas votre pile. 

Même une sauvegarde archivée et une clause de personne disparue ne sont pas des garanties. Un ennemi déterminé pourrait vous capturer, récupérer les codes d'accès de l'assurance sauvegarde auprès de votre muse, vous cryogéniser ou vous tuer discrètement puis régulièrement "valider" votre existence en utilisant les codes d'accès pour que l'assureur ne réalise jamais que vous êtes mort ou disparu. Bien que cela nécessite un peu d'effort, c'est souvent moins difficile que de devoir gérer un adversaire immortel qui continue de revenir peu importe le nombre de fois que vous le tuiez. 

D'autres dangers existent aussi. Un habitat complet peut-être détruit, vous emmenant avec lui, ainsi que vos sauvegarde et les enregistrements de votre assureur. Un ennemi plein de ressource pourrait pénétrer la sécurité de votre assureur et supprimer vos sauvegardes, ou simplement corrompre les bonnes personnes pour s'assurer que'elles soient "accidentellement" corrompues. Étant donné ces possibilité, les paranoïaques s'assurent souvent d'avoir plusieurs politiques de sauvegardes redondantes, du moment qu'ils peuvent se l'offrir. 

\subsection{Complication de Sauvegarde} 

Dans la pluaprt des cas, sauveagrder/s'uploader est sans risque tant que personne ne bidouille l'équipement. Si le personnage a subit des dégêts cérébraux ou neurologique, la sauveagrde est ransferré par farcaster ou l'upload est fait depuis un personnage mort, la sauvegarde peut être endommagée en raison d'infomration neurale manquante. Pour chacun de ces tests, faites un Test de LUC pour le personnage. En cas d'échec, il subit un point de stress mental par tranche complète de 10 points de MdE. Notez que ce stress (et les trauma éventuels) s'appliquent à la sauveagrde, pas au personnage d'origine. Cependant, si la sauvegarde est utilisé pour réinstantier le personnage, le stress est appliqué. 

\section{Réincarnation} 

\textit{La réincarnation} (ou le remorph) est l'opération consistant à donner un corps à un ego. Changer de corps fait partie du cours normal de la vie pour des centaines de millions de transhumains, et c'est encore plus fréquent pour les personnes dans certaines professiones. Les personnages impliqués dasn un travail spécialisé peuvent avoir à se réincarner à une fréquence mensuelle. Ceux qui voyagent beaucoup peuvent le faire encore plus souvent. De plus, étant donné le nombre d'infugiés 

morts pendant al Chute et qui on mainteannt acquis une nouvelle morph, la vaste majorité de la transhumanité s'est réincrané au moins une fois. En tant que tel, la plupart des transhumains sont accoutumés à la réincarnation. 

S'ajuster à un nouveau corps prend du temps et un peu d'effort (voir \textit{Intégration,} p. 272). La réincarnation est également difficile psychologiquement, comme reflété par la continuité (p. 272) et l'aliénation (p. 272). 

Une fois qu'un ego habite pleinement une nouvelle morph, la pile corticale de celle-ci à besoin de dix minutes pour amasser une sauvegardde complète de l'ego. 

\subsection{Réincarnation dans des Biomorphs et des Pods} 

Se réincarner prend à peu près une heure dans une clinique proprement équipée. par essence, l'opération est un upload inversé. La nouvelle incarantion est branchée à un connecteur d'ego qui inflitre le cerveau avec des nanobots qui restructurent physiquement la structure neurale et les connexion du cerveau en fonction de la carte fournit par la sauvegarde. S'incarner prend six fois plus de temps que l'upload car l'essaim de nanobots fonctionnant comme une imprimante à encre dans le cerveau modèle a besoin de dupliquer la structure physique complète du réseau neural de l'ego. Pour la réincarantion un connecteur d'ego "mouillé"  est utilisé, l'incarnation et le connecteur d'ego sont submergé dans une cuve emplie de nanogel. 

l'opération de réincarnation pour les pods ne prends qu'une demi-heure, les cerveaux des pods étant à moitié biologique et à moitié des cybercerveaux. 

\begin{quotation} \textbf{Le Maître de jeu et la Réincarnation} \\ le maître de jeu a un contrôle assez large de ce que les personnages peuvent obtenir lorsqu'ils se réincarnent. Les personnages peuvent recevoir de nouvelels morphs de la aprt de Firewall ou de n'importe quel employeur/patron pour qui ils travaillent actuellement. Dans ce cas, le maîþre de jeu peut simplement assigner n'importe quelle morph qu'ils pensent adaptées à la situation - avec un contrôle complet des améliorations, des traits, etc. Alors que les morphs doivent être dimensionnée pour al misson en cours, cela présente une opportunité pour le maîþre de jeu de donner de nouveaux jouets aux personnages ainsi que de nouveaux défis à relever. Les maîtres de jeu sont encouragés à inventer, s'amuser et à donner aux joueurs quelque chose qvec lequel ils peuvent travailler sans pour autant leur donner tout ce qu'ils veulent.\\ Dans d'autres situations, la disponibilité des morphs voulues peut-être limlitée par l'endroit de la réincarnation. Un petit avant-poste dans les étendues sauvage de Mars n'aura probablement pas une grande sélection de morph - en fait, quelques rusteurs et des ynthmorphs pourraient être l'ensemble du stock disponible. De manière similaires, les plus grands habitats ont une forte demande pour de bonnes morphs, il pourrait donc y avoir une liste d'attente pour les sylphs haut de gamme ou les refaites par exemple. Dans la même veine, les morphs disponibles seront soumises aux restrictions légales locales, obteni cette morph reapeur pourrait donc bien être hors de question. Les personnages pourront toujorus se tourner vers les fournisseurs de morph du marché noir, mais à leur risques et périls.\\ Cela signifie que le maîþre de jeu ne devrait jamais avoir peur de dire non à un personnage qui cherche à obtenir une morph qui n'est pas raisonnable ou qui peut potentiellement perturber le jeu. Bien qu'il soit de bon ton de donner à vos joueurs ce qu'ils veulent une fois de temps en temps, il est également interessant pour l'interprétation de leur géner avec des morphs un peu différentes de celle qu'ils espéraient ou qui fournit des défis interessants, telles que des addictions physique. Pour plus de fun, laissez le personnage ignorant des traits négatifs ou des implants secrets de la morph jusqu'à ce qu'ils se révèlent d'eux-même. Comem toujours, le but est de s'amuser, mais la variété aide souvent. \end{quotation} 

\subsection{Réincarnation dans des Synthmorphs} 

La réincarnation dans le cybercerveau d'une synthmorph est bien plus facile et rapide, il ne s'agît que de cpier la suavegarde dans le cybercerveau (et c'ets instantané) puis de faire fonctionner la sauvegarde dans son état de cerveau virtuel (1 Tour d'Action). Les inconvénients des synthmorphs sont qu'il est plus difficile de s'y acclimater (voir \textit{Intégration, } p. 272), elles sont vulnérables au piratage (p. 261) et elles sont perçues comme étant bas de gamme dans certaines cultures. 

\subsubsection{Évacuer un Cybercerveau} 

Les personnages habitant un cybercerveau peuvent choisir de le vider volontairement en se copiant en infomorph dans un autre appareil. Cela prend 1 Tour d'Action complet. Voir \textit{Réincarnation en Infomorph,} p. 273. 

\subsection{Coût de Réincarnation} 

Les coûts impliqués dans le processuss de réincarnation sont générallement inclus dans les coûts de l'assurance sauvegarde et/ou de la nouvelle incarnation. Les coûts des morphs individduels sont indiqués dans les descriptions démarrant à la p. 139. Voir \textit{Courtage de Morph} (p. 276) pour les règles concenrant l'obtention des morphs. 

\subsection{Intégration} 

S'habituer à un nouveau corps prends toujours un peu de temps. Le personnage doit s'habituer aux changement de taille, de poids, de sexe et de possibilités, qui nécessitent souvent de désapprendre des façon de faire certaines choses qui marchaient bien dans la forme précédente. La réincarnation dans une morph synthétique ou dans une morph élevée est également relativement perturbant au début, étant donné les écarts morphologiques drastiques, les changements dans la structure des membres (et parfois le nombre de membres), et ainsi de suite. Heureusement, les esprits transhumains sont des objets adaptatifs, et ce processus est aidé par l'application de "patchs" mentaux pendant l'opération de réincarnation et qui fournissent une aide préciseuse au personnage pour l'utilisation de son corps. 

Un ego dans une nouvelle morph fait un Test d'Intégration lorsqu'il prend le contrôle de son nouveau corps, lançant SOM x 3 (les bonus de morph ne s'appliquent pas) et en appliquant les modificateurs de la table des Modificateurs d'Intégration et d'Aliénation. Le résultat du test est expliqué sur la table des Test d'intégration, p. 272. 

\\ 



\begin{table} \caption{Test d'Intégration} \begin{tabular}{|l|l|} 



\hline

RÉSULTAT DU TEST &EFFET\\ \hline

Échec Critique &Le personange est incapable de s'acllimater à la nouvelel morph - \\ &quelque chose ne passe pas. Les personnages subissent un modificateur\\ &de -30 à toutesles actions physique jusqu'à ce qu'il se réincarne. \\ \hline

Échec Catastrophique (MdE 30+) &Le personnage a de grosse difficulté à s'acclimater à la nouelle\\ &morph. Il subit un modificateur de -10 à toutes ses actions pendant 2 jours\\ &et pour 1 jour de plus par tranche de 10 points de MdE. \\ \hline

Échec &Le personnage a quelques difficulté à s'acclimater à la nouvelle morph. \\ &Il subit un modificateur de -10 à toutes ses actiosn phyisque pour 2\\ &jours et pour 1 jour de plus par tranche de 10 points de MdE. \\ \hline

Réussite &Période d'acclimatation standard. Le personnage subit un modificateur\\ &de -10 à toutes ses actions physique pour 1 jour. \\ \hline

Réussite Exceptionnelle (MdR 30+) &Aucun effet secondaire. Le personnage s'acclimate à la nouvelle morph en \\ &quelques minutes. \\ \hline

Succès Critique &En pleine forme! Cette morph va exceptionnellemnt bien au\\ &personnage. Aucun effet secondaire; gagnez 1 point de moxie pour utilisation\\ &lors de cette session uniquement. \\ \hline

\end{tabular} \end{table} 

\begin{table} \caption{Modificateurs d'intégration et d'&liénation} \begin{tabular}{|l|l|} 



\hline

RÉSULTAT DU TEST &EFFET\\ \hline

Familier; le personnage a utiliser cette morph de manière intense dans le passé &+30 \\ \hline

Clône d'une ancienne morph &+20 \\ \hline

Type de morph originel du personnage (celui dans lequel il a grandi) &+20 \\ \hline

Trait Adaptabilité (Niveau 2) &+20 \\ \hline

Trait Adaptabilité (Niveau 1) &+10 \\ \hline

Le personnage a déjà utilisé ce type de morph &+10 \\ \hline

Première réincarnation &–10 \\ \hline

Le personnage est une IAG s'incarnant dans un corps physique &–10 \\ \hline

Le personnage est un élevé s'incarnant dans un corps non élevé (ou pas de son type) &–10 \\ \hline

Morph synthétique &–10 \\ \hline

Changement de sexe (par rapport à la dernière morph) &–10 \\ \hline

La morph est lourdement modifiée &–10 \\ \hline

Trait Trouble Morphique (Niveau 1) &–10 \\ \hline

Trait Trouble Morphique (Niveau 2) &–20 \\ \hline

Infomorph (ne s'applqiue pas aux IAG) (Test d'Aliénation uniquement) &–20 \\ \hline

Fork (Test d'Aliénation uniquement) &–20 \\ \hline

Trait Trouble Morphique (Niveau 3) &–30 \\ \hline

morph exotique (octomorph, néo-avienne, novacrabe, swarmanoïde, etc.) &–30 \\ \hline

\end{tabular} \end{table} 

\subsection{Aliénation} 

Après une perte de continuité, le deuxième facteur principal qui impacte les personnages se réincarnant et l'aliénation. Une fois que l'ego a son nouveau corps sous contrôle, il est temps de regarder dans le miroir. Le test d'aliénaation reflettent l'expérience de la redécouverte de soi, de son nouveau vissage, de sa nouvelle peau et de son nouveau cerveau. Par exemple, se transférer dans une morph radicallement différentes (telles qu'un swarmanoïde par exemple) peut être difficile à saisir. Les élevés ont souvent des difficultés pour s'habituer aux différentes poussées hormonales d'une biomorph humaine et réciproquement. Alors que l'ego des personnage est dans le mrme état que dans leur précédente incarnation, les cerveaux et la biochimie de nombreuses morphs peuvent altérer des aptitudes telles que la VOL ou la COG. Tout cela pouvant générer de la frustration ou de la désorientation. Chaque personnage fait un Test d'Aliénation pour refléter le stress mental nécessaire à la compréhension de son nouveau corps, lançant INT x 3 et en appliquant les modificateurs de la table de Modificateurs d'Intégration et d'Aliénation. Consultez la table des Tests d'Aliénation pour en déterminer les effets. \\ 

\begin{table} \caption{Test d'aliénation} \begin{tabular}{|l|l|} 



\hline

RÉSULTAT DU TEST &EFFET\\ \hline

Échec Critique &Dysmorphie Extrême. Le personnage n'aime pas du tout sa nouvelle incarnation \\ &et subit 2 points de stress par tranche de 10 points de MdE. \\ \hline

Échec &Le personnage n'est pas à l'aise avec sa nouvelle morph et subit 1 point de stress \\ &par tranche de 10 points de MdE. \\ \hline

Réussite &Le personnage s'adapte bien à sa nouvelle apparence. Aucun effet secondaire. \\ \hline

Succès Critique &Meilleure. Morph. Au. Monde. La nouvelle morph correspond exactement à la perception de soi \\ &du personnage, et l'améliore même quelque peu. Le personnage soigne en fait \\ &1d10 $\div$ 2 (arrondissez au supérieur) points de stress. \\ \hline

\end{tabular} \end{table} 

\subsection{Test de Continuité} 

Le plus grand choc qui frappe la plupart des personnages se réincarnant est probablement la perte de la continuite de soi. C'est particulièrement vrai pour les personnages qui sont morts. Si leur pile corticae a été récupérée, ils se rappeleront de leur propre mort. Si ils ont été restaurés depuis une sauvegarde archivée, ils ne se rappelerons pas de leur mort, mais ils auront perdu unt période complète de leur vie - tout depuis leur dernière sauvegarde. En fait, si leur corps n'est pas récupéré, ils peuvent même ne pas être sûr d'être mort - il peut exister une copie d'eux survivant quelque part. Le point culminant de cette perte de continuité est une sorte de crise existentielle - le personnage n'est plus la personne qu'il était auparavant. Cela amène à des questions pour déterminer si ils sont ce qu'il spensent être, ou juste une mauvaise imitation et pas du tout une personne réelle? pour déterminer comment cette perte de continuité affectera un personnage, faites un test de Continuité en lançant VOL x 3. Chaque personnage subit du stress suite à la perte de continuité, tel que noté sur la table du Stress de Continuité. Réduisez ce stress d'1 point par tranche de 10 points de MdRsur le Test de Continuité, ou augmentez le d'1 point pour chaque tranche de 10 points de MdE. 

\subsection{Réincarnation en Infomorph} 

Au lieu de se réincarner dans un corps physique, une sauvegarde peut-être instantiée en temps qu'infomorph, une forme purement numérique. Les infomorphs se distinguent des sauvegarde par le fait que els sauvegarde sont des fichiers inerte. Les infomorphs sont des sauvegarde imprimées sur un cerveau virtuel générique et exécuté comme un programme. Ceté tat de cerveau virtuel doit être exécuté sur un appareil spécifique et obéit aux règles des infomorphs notées p. 264. Les infomorphs peuvent se copier vers d'autres appareils, s'effaçant générallement de l'ancien appareil en partant. Les infomorphs qui se copient sans s'effacer sont considérées comme des forks. Les personnages décidant de s'instancier en tant qu'infomorphs doivent faire les Tests de Continuité et d'Aliénation, comme pour toute réincarnation. Les infomorphs peuvent être réincarnées dans des morphs physique, en suivant les règles normale de réincarnation. 



\begin{quotation} Je me réveille avec un goût de goyave et d'umami frais sur la langue. La nuit dernière n'était qu'une méprise. On a bu du vin de quinoa, et j'ai été présenté à des personnes que je n'avait jamais rencontré avant, bien que j'avais des années de connaissances intimes de la plupart d'entre eux. La moitié des habitatnts du Module Illyria sont nus, recroqueveillé autour de moi, dans ma chambre à coucher. La nuit dernière nous avons jouer de la musique sur synthétiseur, des morceaux de bois et un luth. Nous avons bus du thé de champigon infusé dans de l'eau d'une comète rebelle. En regardant autour de moi alors que les étoiles du matin commencent à éclaire l'hrizon orbital de Cérès, il devient évident que nous avons eu une orgie. Hier soir, c'était ma fête de réincarnation. Cette version de moi -le moi 3.0 - est prête pour la vie. —Zheng du Thierry, Carnaval du Capricorne \end{quotation} 



\section{Forker et Fusionner} 

Avec toutes ces sauvegardes d'esprits transhumains sur fichier et l'abondance d'espace meshé sur lequel les faire fonctionenr en tant que cerveaux virtuels, on pourrait se demander ce qui a stopper la transhumanité fpost-Chute de se multiplier en lançant des copies additionnelel d'eux-même. La réponse rapide est: rien, à part la stigmatisation sociale massive et les problèmes psychologiques épineux. Prendre une sauvegarde d'un esprit transhumain, le copier et le ré-instantiet en infomorph est appelé \textit{faire un fork.} C'est l'une des applications les plus puissantes et la plus controversée de la science cognitive transhumaine. 

Il y a quatre classifications des forks: alpha, béta, delta et gamma. Bien qu'étant générallement copiée comme infomorph, il n'y a rien qui empêche un fork d'être incarné dans une morph physique, à part les coutumes et les lois. 

\begin{table} \caption{Stress de continuité} \begin{tabular}{|l|l|} \hline

SITUATION &VALEUR DE STRESS\\ \hline

\textbf{Sauvegarde de la pile corticale} &\\ \hline

Le personnage se souvient d'une mort paisible ou non remarquable. &1d10 $\div$ 2 (arrondissez à l'inférieur) \\ \hline

Le personnage se souvient d'une mort rapide ou violente. &1d10 \\ \hline

\textbf{Sauvegarde d'archive} &\\ \hline

Petit trou mémoriel (moins d'1 jour) &1d10 $\div$ 2 (arrondissez à l'inférieur) \\ \hline

Trou mémoriel de plus d'1 jour &1d10 \\ \hline

Ignorer si/comment vous êtes mort. &+2 \\ \hline

Upload vers réincarnation en continuité (p. 269) &0 \\ \hline

Upload vers réincarnation sans continuité (p. &1d10 $\div$ 2 (arrondissez à l'inférieur) \\ \hline

Le personnage est un fork &2 \\ \hline

\end{tabular} \label{table:continuity-stress} \end{table} 

\subsection{Forks Alpha} 

Un \textit{fork alpha} est une copie exacte de l'ego d'origine et ré-instantié comme infomorph séparée. Un fork alpha peut-être créé en copiant et en exécutant une infomorph (depuis une sauvegarde, une finomorph, un cybercerveau ou une pile corticale extraite et branchée dans un connecteur d'ego). Les forks alpha peuvent être généré depuis un cerveau de biomorph en utilisant un connecteur d'ego et les même traitements que l'upload (p. 268). Les forks alpaha sont des copies exacte de l'ego du personnage, avec les mêmes compétences, souvenirs, stats, traits, personnalité, etc. Les nouveaux forks alpha doivent faire un Test d'Aliénation (p. 272), et probablement un Test de Continuité (p. 272) si il est copié depuis une sauvegarde. 

Créer des forks alpha est illégal dans la plupart des juridcictions, incluant la plupart du système intérieur et la République Jovienne. Dans d'autres endroits, c'est considéré avec dégoût, bien qu'il y ait certains habitats/cultures dans lesquels cette opération est encouragée. 

\subsection{Forks Béta} 

Les forks béta sont des copies partielles de l'ego. Elles sont intentionnellemnt entravée afin de ne pas être considérée un égal du personnage, d'un point de vue légal notamment. Les forks béta ont la plupart des compétences de l'égo original, bien que parfois réduites. leurs souvenirs sont également drastiquement réduit, générallement taillé pour les tâches qu'ils sont sensé accomplir. Les forks béta sont créés en prenant un frk alpha et en le faisant passer passer par le process connus sous le nom \textit{d'élagage neural} (p. 274). Ils sont légaux voire commun par endroit, exceptés dans les places fortes bioconservatives telels que la République Jovienne, bien que els forks delta sont souvent préférés. Les forks béta ont rarement quoi que ce soit se rapprochant de droits civils ou de citoyenneté et ont sont usuellement considéré comme la propriété de l'ego d'origine. Ils sont fréquemment utilisés comme assistant numérique ou pour représenter l'ego d'origine lors des communications longue distance avec d'autres. \\ Les stats d'un fork béta sont déterminées ainsi: 

\begin{itemize} \item Réduisez toutes les aptitudes de 5 (avec un minimum de 1). Cela affecte également toutes les compétences. De manière identique, cela réduit la LUC de 10 et l'INIT de 20. \item Les compétences active sont limitées à 60 maximum. \item Le Moxie est réduit à 1 \item Le trait Psi est supprimé. À la discrétion du maître de jeu, d'autres traits pourrainet être supprimés également. \end{itemize} 

Des changements additionels peuvent s'appliquer en fonction du test d'élagaage neural. Les forks béta nécessitent 1 minute pour être créés. 

\subsection{Forks Delta} 

Les delta forks sont des copies extrêmement limitée d'un ego. Elles sont plus proches des modèles d'IA sur lesquel les traits de personnalité en surface de l'ego sont imprimés. Également créés par l'élagage neural, les foks delta sont pleinement fonctionnel (autant compétent qu'un fork béta ou une IA) mais ont des compétences extrêmements limitées et des souvenirs trés lourdement édités, généralement au point d'être des amnésique fonctionnels. \\ Les stats d'un fork delta sont déterminées ainsi: 

\begin{itemize} \item Réduisez toutes les aptitudes de 10 (avec un minimum de 1). Cela affecte également toutes les compétences. De manière identique, cela réduit la LUC de 20 et l'INIT de 40. \item Les compétences active sont limitées à 40 maximum. Le fork ne peut pas avoir plus de 5 compétences Actives. 

\item Les compétences de Connaissances sont limitées à 80. Le fork ne peut pas avoir plus de 5 compétences de Connaissances. \item Le Moxie est réduit à 0. \item Le trait Psi est supprimé. À la discrétion du maître de jeu, d'autres traits pourrainet être supprimés également. \end{itemize} 

Des changements additionels peuvent s'appliquer en fonction du test d'élagaage neural. Les forks béta nécessitent 1 Tour d'Action pour être créés. 

\subsection{Forks Gamma} 

Plus communament appelés \textit{vapeurs,} les forks gamma sont essentiellement des copies incomplètes, corrompues ou fortement endommagées d'un ego. Les vapeurs ne sont pas intentionnellement créés et sont généralement le résultat d'upload ratés, de sauvegarde brouillées, de farcast incomplets ou interceptés ou d'infomorphs/forks qui ont été endommagés d'une manière ou d'une autre ou qui sont devenus fous. Il est extrêmement rare que quiconque crée vvolontairement une vapeur pour autre chose que la recherche, bien qu'ils puissent se développer dans certains endroits intéressants. Par exemple, de mauvais logiciels de compétences peuvent occasionnellement inclure suffisament des traits de personnalités et des souvenirs de la personne a qui la compétence a été prise pour qu'elle puisse se comporter de manière vaporeuse lorsqu'il est utilisé. 

Comme les vapeurs sont des anomalies plutôt que des créations volontaires, les caractéristiques du'n fork gamma sont laissé à l'appréciation du maître de jeu. Ils devraient avoir une partie ou toutes les caractréistiques suivantes: compétences réduites, aptitudes réduites, souvenirs incomplets ou incohérents, traits mentaux négatifs et du stress mental ou des traumas persistant inclaunt des dérangements et/ou des troubles. 

\subsection{Élagage Neural} 

L'élagage neural est l'art de prendre une sauvegarde/infomorph et d'en réduire la taille pour q'uelle fontcionne comme un fork béta ou delta. 

Les forks béta sont créés en utilisant un état d'esprit virtuel qui est intentionnellement inhibé et en filtrant un égo au travers de celui-ci. Comme pour les arbustes topiaires, les portions du réseau de neurones du personnages qui dépassent la taille du fork voulu sont retirés. En plus des changements notés sous \textit{Forks Béta} (p. 273), les personnages peuvent vouloir effacer/réduire certaines compétences et enlever certains souvenirs. 

Les forks delta sont créer en excisant les traits de personnalité de surface de l'ego et les appliquant sur une IA générique. Dans ce cas, les souvenirs de l'ego sont généralement complètement exclus - il est facile de commencer avec un frok delta vierge et de l'alimneter ensuite des souvenirs/connaissances spécifiques dont ils ont besoin. Comme pour les forks béta, les personnage faisant des forks delta peuvent volontairement choisir d'effacer/diminuser certaines compétences et de conserver certains souvenirs. Si un fork alpha n'est pas disponible pour l'élagage, un fork delta peut être extrait d'un cerveau de biomorph avec un connecteur d'ego et en 1 minute. Beaucoup de personnes incarnés dans des biomorphs conservent des forks delta à portée de main, pour pouvoir les élaguer à la volée au cas où. 

La compréhension transhumaine des neurosciences s'étendent à l'analyse et à la copie d'un esprit, mais le fonctionnement intrinsèque des souvenirs et toujours imparfaitement compris. Faire des éditions précises sur des portions d'un réseau de neurones (pour modifier des souvenirs, des compétences et autre) est toujours un art mystique. La difficulté dans l'élagage neural est que donner un coup de sécateur sur une branche mémorielle n'est pas vraiment une science exacte. Les souvenirs spécifiques ne peuvent être excisés ou conservés - au mieux, les souvenirs sont gérés en buisson, générallement groupés par période de 6 mois minimum. Dans un but de simplicité, la plupart des forks béta sont créés en supprimant tous les souvenirs vieux de plus d'un an. 

En créant un fork béta ou delta, le personnage doit faire un Test de Psychochirurgie (d'autres personnes peuvent faire ce test à la place du personnage, représentant le fait qu'il leur a donné les accès pour élaguer le fork de manière appropriée). Si le personnage réussit son jet, le fork est créé comme désiré. Si le test échoue, le maîþre de jeu choisit l'une des pénalités suivantes pour chaque tranche de 10 points de MdR. Certaines de ces pénalités peuvent être combinés pour avoir un effet cumulatif: 

\begin{itemize} \item 1 compétence supplémentaire réduite de –20 \item Le fork développe un trait mental négatif d'une valeur de 10 PP \item Le fork subit 1d10 $\div$ 2 (arrondissez au supérieur) points de stress\item Des souvenirs supplémentaires sont perdus (à la discrétion du maîþre de jeu; fork béta uniquement) \item 1 trait positif perdu \end{itemize} 

\subsubsection{Élagage Neural par Psychochirurgie à Long-Terme} 

Au lieu de générer des forks à la volée, certains personnages préfèrent avoir des forks minutieusement élagués sous la main, stockés dans des fichiers inertes et qui peuvent être appelés, copiés et lancés en fonction des besoins. Ces forls sont créés avec de la psychochirurgie à long-terme, ce qui signifie qu'ils ne souffrent que de quelques inconvénients et que leurs souvenirs sont plus finement ajustés. 

L'élagage neural à long-terme nécessite un Test de Psychochirurgie comem au-dessus, mais avec un modificateur de +30. Les forks deltas sont élagués de cette manière en 1 semaine et les forks bétas en 1 mois. Des modifications supplémentaires peuvent être effectuées sur le fork en utilisant les règles normale de psychochirurgie (p. 229). 

Il n'est pas rare que certaines personnes préfèrent utiliser des forks d'eux-même ou de personnes chéries plutôt qu'une muse. De manière similaires, certaines hyperélite prospères sont connus pour conserver des copies de sauvegarde plus jeune à portée de main, parfois pendant des décennies, et de les réinstantier lorsque leur ego primaire a suffisament de compétence et d'expérience pour surclasser leur eux plus jeunes. Bien qu'ils soient techniquement des forks alpha, leur décalage vis à vis de l'ego original est comparable par certains aspects à celui d'un fork béta. Il se dit que ce serait la méthode utilisée par pax Familiae lors de l'instantiation de son arémes de clône d'elle. 

\subsection{Gérer les Forks} 

Les maîtres de jeu sont encouragés à autoriser les joueurs à interpréter les forks de leur personnage. Il est important de noter cependant que, mrme avec des forks alpha, une fois que le fork et l'ego originele ont divergés, ils se développent par la suite comme des personnes distinctes. Les évènements qui forment la personnalité principale, le caractère et la connaissance d'un ego n'arriveront pas au fork - et même si c'est le cas, ils n'affecteront pas le fork de la mrme manière - et vice-versa. La frotnière exacte entre un ego et un fork est un débat philosophique et juridique central pour de nombreux transhumains. 

Cela signifie qu'un maître de jeu ne devrait pas être effrayé d'enlever un fork des mains d'un joueur et de le transformer en PNJ si ils commencent à diverger grandement. De même, si un fork commence à apprendre des informations auquelles le personnage principal n'a pas (encore) accès, il est probablement mieux d'utiliser le fork comme un PNJ afin d'éviter le métajeu. Il est complètement possible qu'un fork décide qu'il n'obéira plus à l'ego d'origine et décide de s'occuper de gérer sesd propres problèmes. Cela n'arrive généralement qu'avec les forks alpha,q ui sont essentiellement des copies compltes, et plus le temps passe, plus l'idée de retourner dans l'ego originel perd en attrait. Les forks béta et delta sont relativement conscient de leur nature de copie incomplètes, et retournent génrallement vers leur ego d'origine pour être réintégrés. Cependant, il y a des cas rares dans lesquels mrme ceux-ci décident d'aller vivre leur propre vie. 

\subsection{Fusionner} 

La fusion est le procédé de ré-intégration d'un fork avec son ego d'origine. Cette opération est effectuée sur des egos/forks conscient, transferrant les deux en un seul égo fusionné. Le procédé n'est pas difficile à mener lorsque les deux forks n'ont été séparés que pendant une courte période de temps. Plus les forks passent du temps sépras, plus le procédé de fusion devient complexe, un test de Psychochirurgie devient nécessaire (effectué soit par l'ego, soir par un autre personnage supervisant les opérations). La table de Fusion liste les modificateurs de ce test ainsi que les résultats de réussite ou d'échec. Pour les synthmorphs, la fusion prend un Tour d'Action complet. pourt les biomorphs, un connecteur d'ego (p. 328) ou des augmentations mnémoniques (p. 307) sont nécessaire pour fusionner, et le procédé prend 10 minutes. Le résultat de l'opération est un ego unifié, que le Test de Fusion réussisse ou pas. La psychothérapie (p. 209) et la psychochirurgie (p. 229) peuvent réparer les mauvaises fusions au fil du temps. \\ 

\begin{table} \caption{Fusionner} \begin{tabular}{|l|l|l|l|} \hline

TEMPS SÉPARÉS &MODIFICATEUR &RÉUSSITE &ÉCHEC\\ \hline

Moins d'1 heure &+30 &Ego sans déaut avec les souvenirs &Souvenris intacts, (1d10 $\div$ 2, arrondi à l'inférieur) – 1 VS \\ &&des deux parties. &\\ \hline

1–4 heures &+20 &Lien solide, souvenirs intact &Souvenris intacts, (1d10 $\div$ 2, arrondi à l'inférieur) – 1 VS \\ \hline

4–12 heures &+10 &Souvenris intacts, 1 VS &Perte de mémoire mineure, (1d10 $\div$ 2, arrondissez au supérieur) VS \\ \hline

12 heures - 1 jour &+0 &Souvenris intacts, 2 VS &Perte de mémoire modérée, (1d10 $\div$ 2, arrondissez au supérieur) + 2 VS \\ \hline

1 jour - 3 jours &–10 &Souvenris intacts, 3 VS &Perte de mémoire majeure, 1d10 + 2 VS \\ \hline

3 jours - 1 semaine &–20 &Souvenris intacts, 4 VS &Perte de mémoire majeure, 1d10 + 4 VS \\ \hline

1 semaine et + &–30 &Perte de mémoire mineure, 5 VS &Perte de mémoire critique, 1d10 + 6 VS \\ \hline

\end{tabular} \label{table:merging} \end{table} 

\begin{quotation} \textbf{LE SOI} \\ Forker et fusionner peut avoir changer la manière dont la transhumanité se définit et ce que signifie le fait d'avoiur une personnalité bien intégrée. \\ Alors que faire des fork est un jeu d'enfant d'un point de vue technologique, les effets psychologiques et sociaux du clonage d'un esprit fait que la plupart des gens hésitent à utiliser des forks. Certaines juridictions bannissent l'utilisation des forks, à l'exception des utilisations médicales, alors que d'autres ont des restricions élevées. Dans beaucoup de juridiction hypercorporatiste, par exemple, les forks alphas sont illagaux et laisser un fork béta actif pendant plus de 4 heures sans fusionner viole les descendantes modernes des lois antitrust du 20° siècle. De amnière similaire, la Junte Jovienne et d'autres bioconservateurs interdisent purement et simplement les forks. \\ S'occuper des forks non désirés est un autre problème épineux. Dans certains endroits, cela se résume simplement à leur suppression, un esprit stocké n'ayant aucun statut légal. Dans d'autres lieux, un fork qui ne souhaite pas réintégrer son ego d'origine pourrait se voir accorder certains droits, bien qu'ils ne soient généralement accordés qu'aux forks alpha. \\ De manère plsu significative, faire fonctionner un fork de soi pour des périodes d'une heure ou moins est une tâche realtiavement simple pour de nombreux transhumains. Beaucoup de personnes font usage de forks pour accomplir les tâces quotidienne, et presque tout le monde a fait l'expérience du fork à un certain niveau. \\ Les transhumains regardent la création de fork comme les humains du 21° siècle considéraient la consommation d'alcool ou de drogue. Un peu ne pose pas de propblème, mais en abuser sera stigmatisé. C'est princiaplement parceque la plupart des transhumains comprennent les conséquences psychologiques de l'abus de fork. \end{quotation} 

\section{Egocast} 

En dépit d'être une civilisation spatiale avec des avant-postes dans tout le système solaire et au-delà, la transhumanité n'utilise que trés peu les vaisseaux spatiaux pour le voyage interplanétaire. Des navettes utilisant divers système de propulsions assurent des liaisons régulières entre les habitats, la surface plantéaires et les lunes. mais pour tous les voyages de plus de 1.5 million de kilomètres - la distance qu'un moteur à fusion peut couvrir en une journée, les gens préfèrent s'egocaster. L'egocast est la technologie transhumaine de transport personnelle la plus avancée, bien que seul l'ego du personnage voyage effectivement. L'egocast combine les technologie d'upload et de farcast quantique pour transférer une suavegarde (ou parfois un égo conscient, voir p. 269) à travers des distances interplanétaires. 



Bien que l'égocast se apsse à la vitesse de la lumière, les temps d'egocast varient en fonction de la distance. S'egocaster dans un système en grappe ou planétaire est généralllement l'histoire de quelque minutes. S'egocaster depuis le Soleil jusque dans la Ceinture de Kuiper prend cependant entre 40 et 70 heures, et s'égocaster à tarvers tout le système solaire peut prendre encore plus de temps. 

une fois qu'un ego arrive à destination, il peut être archivé, damérré en infomorph ou réincarner normalement. 

\subsection{Sécurité des Egocasteurs} 

S'envoyer à travers l'espace interplanétaire est une technologie époruvée et fonctionne généralement sans accroc. En raison de l'utilisation de farcast quantiques, l'egocast ne peut être perturbé par des interférence radio brouillant le signal et causant une eprte de donnée. Normalement l'ensemble du procédé est véhiculé par le prestataire de service de sauvegarde du personnage, et les brèches de sécurités sont rares. \\ Il existe cependant plsuieurs risues intrinsèque à l'egocast. Le plsu évident est que la conscience du personnage est transférée en tant que fichier de sauvegarde vers la destination. Si l'egocasteur à l'autre extrémité n'est pas certifié ou si le réseau de destination est contrôlé par le réceptionniste, le personnage se mets potentiellement à la merci de leur hôte. beaucoup d'hypercorp considèrent que bidouiller un ego transmlis est une brèche sérieuse de l'étiquette, alors que les autonomistes le considère incroyablement répressif. Cependant, les groupes d'extrémistes politques et les organisations criminelles qui contrôllent les egocasteurs n'acceptent que peu de restrictions. \\ Un risque plus subtil est la possibilité pour les hackeurs d'exploiter des trous de sécurité dans l'egocasteur et de l'espace virtuel lié pour voler un fork du personnage. Cela reste extrêmement difficile à faire. Cela n'arrive presque jamais lross des uploads normaux, car les services d'upload sont conscient des problèmes de sécurité au point de la paranoïa. Mais même ainsi, les forks volés par de telles tentatives finissent plus souvent qu'à leur tour sous forme de vapeur, l'intrus étant géénralement arrété avant qu'une copie complète ne pusise être obtenue. 

\subsection{Darkcast} 

les personnages désirant s'egocaster sans attirer l'attention des officiels tels que les services des Douanes et de l'Immigration peuvent passer par des services appelés darkcast - des farcast émetteurs-récepteurs illagaux et contrôlé par les syndicats criminels et d'autres groupes clandestins. Pour trouver un service de ce type, un personnage doit utiliser sa coméptence Réseau et probablement sa réputation (p. 285). 

\section{Courtage de Morph} 

Les morphs sont un bien de consommation dans la société transhumaine. La technologie et les matériaux nécessaire à la croissance de nouvelles morphs sont bons marchés et abondant, même si cela prend du temps. Les biomorphs clônés prenent au moins un an et demi pour se développer, même avec une croissance accélérée. Les pods, qui sont généralement assemblés à partir de morceau qui ont poussés en cuve, nécessitent 6 mois. Les synthymorphs comme les boîte et les synths peuvent être produite dans la journée, mais des modèles plus complexe peuvent prendre plus d'une semaine. Théoriquement, l'offre va un jour surpasser la demande au point où la chair sera gratuite. 

Les personnages ont plusieurs options pour acquérir des morphs lorsqu'ils voyagent par egocast, subissent de gros dégâts ou ont simplement envie d'un nouveau corps. Lors de l'egocast, la méthode la plus fréquente pour les voyageurs de la classe moyenne est de stocker leur morph actuelle dans une maison de poupée sécuriése et de louer une morph à destination. Moins fréquemment, les personnages peuvent se reoposer sur des installations de réincarnation publique ou, si ils ont les moyens, ils peuvent carrément acheter une nouvelle morph. Les personnage qui s'attendent à rester à destination indéfinemment ou qui décident de se réincarner mais qui ne voyagent pas peuvent opter pour échanger leur vieux corps, l'abandonnant de manière permanente. 

\subsection{Disponibilité des Morphs} 

Comme noté au paragraphe \textit{Le Maître de Jeu et la Réincarnation} (p. 271), trouver le modèle de morph voulu n'est pas toujours simple. Alors que de nombreux type de morph (boîte, synths, spliceurs) sont générallement disponible, les personnages peuvent aussi de nouvelles 

morphs en utilisant leur compétence Réseau (voir \textit{Réputation }\textit{ et réseaux Sociaux,} p. 285). Certains types de morphs sont plus difficile à trouver que d'autres; le maîþre de jeu devrait appliquer un modificateur approprié pour toute les morphs qui semblent rare ou inhabituelle (par exemple, les swarmanoïde ou les reapeurs). De même, de nouvelles morphs peuvent être bêtement indisponible à un endorit donné. Les rusteurs sont rarement disponibles hors de Mars, par exemple, alors que sur Europe la plupart des morphs sont des variantes locales exotiques et aquatiques. 

Le maîþre de jeu détermine quelels factions sont capable de fournir de nouvelels morphs dans une zone donnée. Les factions ne fourniront pas de morphs qui leur sont indisponible aux personnages débutants. Si la faction n'est pas dominante dans une zone, une pénalité de -1à à -30 devrait être appliquée. En dépit de la présence dans une zone donnée, certaines factions peuvent être incapable de fournir des morphs. 

Si un personnage cherche une morph customisée avec des implants ou des amélioratiosn spécifiques, la recherche sera encore plus difficile. Le maître de jeu devrait également y appliquer un modificateur de -10 à -30, en fonction de l'étendu et de la légalité des modifications recherchées. 

\subsection{Acquisition de Morph} 

Une fois que la morph est localisée, le personnage doit faire appel à des faveurs (p. 285) ou payer des crédits pour l'obtenir. Le coût des morphs est noté sur la table de Coût des Morphs. Dans le système intérieur, le prix des morphs est souvent augmenté par la demande du marché et les morpsh les plus désirées peuvent valoir une petite fortune. A l'extérieur, les prix en rep sont plus raisonnable mais toujours élevés en raison de la pression de la population sur les abris dépendant d'un soutien du système extérieur. Pour les voyageurs et ceux qui changent fréquemment de corps, il y a de nombreuses façon de réduire ces coûts. 

\subsubsection{Revente Et Entremetteurs} 

Trouver une morph pour les voyageurs et ceux qui n'ont pas de corps est une compétence spécialisée nécessitant des résedaux sociaux solide et un sens de la négociation. En géénrale, c'est un marché de revente, les revendeurs (ou "entremetteurs," comme ils sont appelés dans la nouvelle économie) agissent comme des agents de la personne cherchant un corps. La table du Coût des Morphs prend en compte la marge de 10\% de l'intermédiaire. Les personnages souhaitant éviter cet agent peuvent réduire le coût de 10\% mais subissent une pénalité de -30 à leur Test de Réseau.pour localiser une morph disponible. 

\subsubsection{Morphs Personnalisées} 

Si un personnage cherche à obtenir une morph personnalisée (avec du bioware, du cyberwware ou des implants nano supllémenatire, ou des amélioratiosn raobotiques), le coût de ces améliorations est ajouté au coût de la morph (si le maître de jeu le décide, une réduction peut-être appliquée sur le lot). De même, les morphs peuvent être fournit avec des trait de morphs positifs ou négatif (p. 145). Ces traits augmentent ou diminuent le coût de la morph au taux de +500 crédits par PP pour les traits positifs ou de -200 crédits par PP de traits négatifs. Les traits négatifs reflettent typiquement les abus que la morph a subit entre les mains de l'occupant précédent. 

\subsubsection{Échange} 

Pour ceux qui désirent abandonner définitivement leur ancienne morph, les échanges de morph sont des options courantes. La demande de corps élevée signifie qu'un acheteur est presque toujours disponible, sauf si le maître de jeu trouve des circonstances particulières. Les morphs peuvent être échangée à la valeur indiquée dans la table des Coûts de Morph ajustée par tous les traist positif ou négatifs, avec une réduction de 10\% pour les examens physique et les frais du dénicheur. Cette somme est soit payée au vendeur en credits soit rendues comme faveur en utilisant la rep. 

\subsubsection{Provisionnement d'un Mentor} 

Les personnages en mission pour un mentor riche ou intégral peuvent avoir des morphs fournit par lui. Normalement de telles provisions sont faites pour la durée du travail, même si parfois la morph peut considéer en soi une forme de paiement pour le service rendu. les maîþres de jeu sont encouragés à être créatif lors de tels arranegemnt, bien que les joueurs devraient être conscient que de tels échanges peuvent rapidement se révéler Faustiens. 

\subsubsection{Morpsh du Marché Noir} 

Les vendeurs de crops du marché noir promettent de fournir l'acheteur avec les morphs et les mises à jour de son choix indépendamment des lois contre les armes ou les implants, en plus de contourner les procédures d'enregsitrelment des arrivées standard grâce au darkcast. Les morphs illégales sont généralement fournies avec une hausse de prix (au moins +25\%), alors que les morphs d'occasion ayant un passif louches (et ds traits bizarre) sont générallement moins chère (-25\%). 

\begin{table} \caption{Coût des morphs} \begin{tabular}{|l|l|} \hline

TYPE DE MORPH &COÛT \\ \hline

\multicolumn{2}{|c|}{Biomorphs} \\ \hline

Plates, Spliceurs &Élevé \\ \hline

Octomorphs &Chère (30 000+) \\ \hline

Furie, Ghosts, Refaite &Chère (40 000+) \\ \hline

Futuras &Chère (50 000+) \\ \hline

Toutes les autres &Chère \\ \hline

\multicolumn{2}{|c|}{Pods} \\ \hline

Pods de Travail, de Plaisir &Élevé \\ \hline

Novacrabes &Chère (30 000+) \\ \hline

\multicolumn{2}{|c|}{Synthmorphs} \\ \hline

Boîtiers &Modéré \\ \hline

Synths, Libellulles &Élevé \\ \hline

Slitheroïdes, Swarmanoïdes &Chère \\ \hline

Transformers &Chère (30 000+) \\ \hline

Arachnoïdes &Chère (40 000+) \\ \hline

Reapeurs &Chère (50 000+) \\ \hline

Traits de morphs positifs &+500 par PP \\ \hline

Traits de morphs négatifs &–200 par PP \\ \hline

\end{tabular} \label{table:morph-costs} \end{table} 



\subsubsection{Contraction} 

Les personnages trop fauchés pour s'offrir une nouvelle morph peuvent conclure un marché pour du service contracté - un "accord" qui est rarement à l'avantage du 

nouveau contracté. Les contrat typique nécessitent des années de travail contractés - terraformation de Mars, exploitation de comète, minage d'astéroïde, construction d'habitats, colonisation d'exoplanètes, etc. - en échange d'une synthmorph ou d'un spliceur bon marché à la fin du contrat. Les maîþre de jeu doivent user de leur discrétion lorsqu'ils proposent de telles clauses, bien que dans de nombreux cas les clauses proposées terminerons temporairement ou définitivement la carrière d'agent indépendant du personnage. Les hypercorporations utilisant des travailleurs contractées sont connues pour changer les clauses à volonté, allongeant la période de service ou frappant le contracté avec un lot de charges cachée ou outrageuses qui n'étaient pas claire à la signature. Les personnages peuvent, bien entendu, entrer pleinement dans le système pour avoir leur morph et partir en courant à la première opportunité, mais les hypercorporations protègent jalousement leurs investissements. Les contractés sont surveillés et pisté de près, et les hypercorporations ne se privent pas d'envoyer des chasseurd d'ego pour récupérer un fuyard. 

\subsubsection{Réincarnation Publique} 

À certains endroits, sur Titan par exemple, il existe une infrastructure de réincarnation publique bien développée afin de fournir un corps à tout ceux qui en ont besoin. Les morphs fournies sont généralement des boîtiers génériques, des synths ou des spliceurs basiques sans traits Positif ou implant optionnels. Quiconque ayant la citoyenneté d'une zone ayant des services publiques de réincarnation peut postuler pour obtenir un corps. Les temps d'attentes varient d'un mois à deux ans,  la Réputation influençant ce délai d'attente à la discrétion du maître de jeu. 

\subsection{Location de Morph} 

Pour les visites temporaire et lorsqu'un infomorph ne sera pas suffisante, des morphs peuvent être louée au lieu d'être achetée. Le coût de location d'une morph est de 1\% de son coût par jour, augmenté d'un prix Bas pour la réincarnation. Ce coût inclut l'assurance locative (voir plus bas). Si l'assurance locative est omise (ce n'est pas toujorus possible, à moins d'avoir une bonne Rep), le coût de location peut être réduit de moitié. 

Les personnages qui louent une moprh peuvent également utiliser leur morph précédente comme caution. Dans ce cas, déduisez le coût de la morph actulle du personnage de la morph louée avant de calculer le coût de 1\% par jour, avec un coût de location minimum de 10 crédits par jour. 

\subsubsection{Location Pénale} 

Les personnages visitant le système intérieur ou la République Jovienne peuvent avoir la possibilité de louer des morphs appartenant à des prisonniers. Dans la plupart des juridictions, les criminels sont condamnées à des peines en simulsapces de réhabilitation, peine qui stipule que la morph du pprisonnier devient propriété de l'étât pendant leur peine d'incarcération. Les morphs acquisent de cette façon ont souvent des historiques compliqués mais tendent à avoir des modifications utiles pour els agents de Firewall. Inversement, un perrsonnage qui se trouve emprisoné peut-être sujet à voir son corps loué à d'autres pendant son incarcération. 

Les effets de la location pénale sont laissées à la discrétion du maître de jeu. Un personnage pourrait avoir à tirer quelques ficelles avec sa Réputation pour louer de telle morph, particulièrement si elle a des modifications d'accès restreinte ou illégale. Les traits négatifs, les erreurs d'identité et les rencontres impromptues avec des amis et des associés de l'ancien propriataire de la morph font parti des inconvénients possible pour ce type d'arrangement. Du bon côté des choses, les locations pénales peuvent réduire les coûts pour la location et l'assurance de la morph, encore une fois ceci est laissé à l'appréciation du maître de jeu. 

\subsubsection{Assurance Locative} 

Les morphs louées doivent être couverte par une police d'assurance, qui empêche généralement l'utilisateur de violer la loi ou d'amener la morph dans les zones hors la loi ou trop dangereuses. les personnages peuevnt acheter une assurance de risque qui couvrira le fait d'emmener la morph dans certaines situations dangereuse, mais cela doublera (au moins) le prix de la location. 



Si un personnage subit des dégâþs organique ou décède alors qu'il est assuré, l'assurance couvrira 80\% du coût de la morph, ce qui veut dire que le personnage devra payer les 20\% restant. Si iles ne peuvent pas payer, leurs possession ou leur moprh peuvent être saisi en gage. 

Si un personnage viole les clauses de la police d'assurance en se mettant en danger volontairement et au-delà du niveau de menace pour lequel l'assurance à été acheté, sans le signaler à l'assureure et sans le payer en conséquence, la olice d'assurance peut-être déclarée nulle. Si la morph louée décède sous une police d'assurance annulée et que le personnage ne peut pas payer pour la remplacer, ses possessions et sa morph stockée peuvent être saisies. 

La saisies prend différentes formes en focntion de l'économie locale et du système légal. Dans l'espace hypercorporatiste, il s'agît d'une saisie simple des liquidités, incluant un upload forcé si la morph du personnage est saisie. Partout ailleurs, le personnage se retrouvera à devoir énormément de faveurs ou à subir de grave perte de réputation, mais ils ne subiront pas d'upload forcé ou d'autres mesures de saisie de leur morph. 

\section{Identité} 

Vu la nature des technologies de réincarnation, l'identité est un concept fluide à \textit{Eclipse Phase.} Les transhumains sont habitués à l'idée d'identifier une personne par leur apparence ou mrme par leur données biométriques, mais ce n'est plus une méthode certifiée. Votre apparence peut changer drastiquement d'un jour à l'autre. Vous pouvez croiser un olympien que vous reconnaissez, mais ils e peut que cela fasse un peu de temps depuis la dernière fois, et vous n'êtes donc plus vraiment sûr que c'est la même personne dans cette morph. Si vous êtes incarné dans une morph de série populaire, il peut y avoir des centaines d'autres morphs clônés qui ont exactement la mrme apparence que vous - cela peut-être utile si vous voulez vous fondre dans la masse. De manière similaire, les services de sécurté ne reposent plus sur les technologies de biométrie. La criminalistique permet d'identifier la présence d'une morph en particulier sur une scène de crime, mais elle ne peut prouver qui était dans cette morph. 

Bien entendu, l'identité est liée à l'ego et certaines autorités ont instituée des vérifications et des mesures de sécurité basées dessus. Dans le ssytème intérieur, chaque ego reçoit un numéro d'ID, qui est utilisé pour valider son identité, sa citoyenneté, son statut légal, ses comptes, ses permis, etc. Cet ID d'ego est vérifiable par la forme des ondes cérébrales d'une eprsonne, qui restent inchangée, même après une réincarnation. Lorsqu'un ego s'upload, le servie d'upload doit intégrer cet ID d'ego dans la sauvegarde/l'infomorph de la personne. De même, lorsqu'une personne se réincarne, le service s'occupant de la procédure est contraint par al loi de vérifier l'ID d'ego de la personne avant de la télécharger. l'ID d'ego est ensuite codé en dur dans la morph sous la forme d'un nanotatouage à l'extrémité de l'index de la perosnne. Ce nanotatouage peut être facielemnt scanné lors des contrôles de sécurité pour vérifier l'identité. 

Bien que ce système soit efficace, il est loin d'être parfait. Par exemple, conserver une trace de l'identité est loin d'être standardisé et varie drastiquement d'un habitat à l'autre. La plupart ne partagent pas ces enregsitrements avec les autres afin de protéger la vie privée de leur citoyens, à moins qu'ils ne fassent partis de la même alliance politique. Par exemple, les stations de l'alliance Lunaire-Lagrange ne aprtagent pas les données d'identité de leur citoyens avec le Consortium Planétaire, bien qu'ils les partagent entre eux. En plus de cela, beaucoup d'enregistrement d'identité ont été perdus pednant la Chute, une situation qui a trés probablement été exploitée par ceux qui ont préférés effacer leur passé ou adopter une nouvelle identité. Tout cela fait que les enregistrement d'identité sont, au mieux, un patchwork. Les officiels doivent également faire avec la sécurité d'autres habitats pour les vérifications d'identité. Si une personne s'egocast de Qing Long dans les Troyens Martiens vers Nectar sur Mars, et que les officiels de Nectar n'ont aucun enregistrement de cette personne, ils sont obligés de faire confaicne au travails des officiels de Qing Jong lorsqu'ils ont vérifiés l'identité et le passif du sujet. Pour empirer les choses, beaucoup d'habitats autonomistes fonctionnent sans vérification d'identité. Bien que certaines mesures d'identification, à la fois pour limiter de tricher avec le système de réputation et pour être capable d'identifier les corps en cas de décés, ces utilisation sont significativement plus laxe et peu d'enregistrements sont conservés. Et donc, lorsque les autonomistes et équivalent s'egocast vers des habitats nécessitant une identité, ils en reçoivent une temporaire pour la durée de leur séjour (et parfois pour les visites futures). 

\subsection{Vérification d'identité} 

il y a trois façon de vérifier l'identié de quelqu'un: analyse de nanotatouage, analyse des ondes cérébrales et vérification d'un hachage cryptographique de l'esprit numérique. 

\subsubsection{Analyse de Nanotatouage} 

Des nanbobots spécialement codés sotn utilisés pour créer un petit nanotatouage sur l'index d'une personne. Ces nanobots contiennent de l'information encodées qui inclut le nom et l'identité d'une personne, ses motifs d'ondes cérébrales, ca citoyenneté/son statut légal, son numéro de comtpe, ses informations d'assurance et ses permis. En focntion des lois locales, il peut inclure d'autres informations telles que le casier judiciaire, l'historique des voyages, les implants restreints, l'historique d'embauche, et ainsi de suite. Ce nanotatouage peut-être lu par quiconque ayant un scanner d'identité spécifique et qui peut lire l'encodage des nanobots. Les nanotatouages d'ID incluent des informations sur la société qui a effectué la réincarnation, pour que les données soient accessibles et qu'elles pusisent être vérifiées avec les enregistrements en ligne. Les données sur le nanotatouage sont également signé cryptographiquement avec la clef publique de la société, ce qui signifie que quiconque vérifie les données et la signature en ligne peut déterminer si les données on été altérées. 

\subsubsection{Analyse d'Onde Cérébrale} 

Les analyse d'ondes cérébrales sont l'une des rares empreintes biométriques qui suivent un ego peut importe la morph qu'il occupe. Elles sont inutilisable pour la plupart des objectsif de sécurité car elles nécessitent une analyse avec une combinaisp, électroencépholgramme et des appareils de neuroimagerie, appelé un scnanner d'empreinte cérébrale, ce qui prend en moyenne 5 minutes. Cet apapreil mesure le motif de base des ondes cérébrales ainsi que la signature d'onde cérébrale du sujet en réponse à certaines pensées 

ou à la perception de certains motifs. Ces analyses sont cependant quasiment impossible à tromper, à l'execption du piratage du scanner d'empreinte cérébrale lui-même, et sont donc considéré comme extrêmement fiable. Ils sont donc occasionnellement utilisés dans des installations de haute-sécurité. 

Cela vaut le coup de signaler que l'infection par certaines variantes du virus Exsurgent, notamment la souche Watts-MacLeod (p. 367), peuvent parfois altérer la forme d'onde cérébrale bien que ce ne soit pas systématique. 

\subsubsection{Code Numérique} 

Des codes d'identité numériques sont souvent incorporés dans les sauvegardes et les infomorphs. Cela permet de savoir plus facilement à qui appartient une sauvegarde, mais cela sert aussi de signature électronique pour vérifier une identité lorsqu'une sauveagrde va être réincarné. Le code numérique contient généralement la même information que le nanotatouage d'ID, et est signé par un hash cryptographique qui rend la falsficiation complexe et qui peut être vérifié en ligne. Les IA et les IAG possèden t également de tels code intégrés à leur structure. 

\subsection{Contourner les Vérification d'Identité} 

Les sentinelles de Firewall et les agents clandestins ont souvent besoin de cahcer ou de modifier leurs identités. Bien que les systèmes d'identification soient difficiles à battre, ils ne sont pas insurmontables. 

\subsubsection{Faux ID} 

La méthode la plus simple de contourner les vérificatiosn de sécurité est d'obtenir un faux ID. Étant donné la nature parcellaire des enregistrements d'identité et le manque d'autorité centrale, ce n'est pas spécialement difficile. De nombreux syndicats du crime et même certains groupe autonomistes maintiennent un marché prospère de fabrication d'identité, souvent avec des historiques complets et une couverture médicale pour tous les implants qui peuvent être restreints ou illégaux. 

Ces identités sont générallement enregistrés auprès d'habitats qui sont soit connus pour être des bases criminelles, soit avoir des sympathies autonomistes ou qui sont isolés et distant. Bien que l'identité soit réellement vérifiable et enregistrée auprès de ces stations, les origines potentiellement douteuses de telles identités sont connus de la plupart des autorités du système intérieur et un personnage pourrait donc être exposé à des vérifications ou à une surveillance plus poussée. Des faux ID enregistrés auprès d'autorités plus respectés peuvent être acquis, mais cela nécessite souvent un investissement plus élevé ou ds connexions aux opérations clandestines des hypecroporations. 

bien entendu, les options de réincarnation et de darkcast du marché noir fournissent également de faux ID. 

\subsubsection{Modifier les Nanotatouages d'ID} 

Des traitemenst de nanpmachines spécifiques peuvent être conçus pour effacer, réécrire ou rempalcer les nanotatouages d'ID. Effacer un nanotat est facile, mais ne pas en avoir un est unc rime et déclenches la suspicion immédiate dans beaucoup d'habitat. Réécrire un nanotatouage est également relativement simple, bien que cela imlique que le nanotatouage échouera à sa validation en-ligne sauf si le chiffrement a également été cassée (p. 253). Remplacer un nanotatouage avec un faux est également possible, et fait parti des opérations de récupération de fausse identité. 

\subsubsection{Falsification d'ID Numérique} 

Les codes d'ID numériques peuvent également êre falsifié, bien que, comme pour les nanotatouages d'ID, cela signifiera que l'identité échoueras aux vérifications en-ligne sauf si le chiffrement est également cassé (p. 253). 

\section{La Vie Dans l'Espace} 

La transhumanité n'est pas juste une espèce qui s'étend jusqu'aux confins de l'espae, elle habite essentiellement dans l'espace. Alors qu'une part substantielle de la transhumanité habite les coprs planétaires tels que Mars, la Lune, Vénus, et les lunes des géantes gazeuses, le reste vit dans une variété d'habitat spatiaux, allant des cylindres O'Neill à l'ancienne du système intérieur aux bulles de Cole du système extérieur. 

\subsection{Habitats spatiaux} 

Les habitats spatiaux existent dans de nombreuses tailels et configurations, allant des avant-postes survivalistes conçus pour dix personnes ou moins aux mondes miniatures dans les zones abondantes en ressources et qui hébergent jusqu'à dix-millions de personnes. Dans les régions spatiales fortement peuplées, telles que l'orbite Martienne, les habitats peuvent être intégré dans l'infrastructure locale, dépendant d'une aprtie des expéditions de ressources d'autres instalaltions orbitales. 

Plus courament, et en particulier dans le système extérieur, les habitats sont des entités indépendantes. Cela signifie généralement que, en plus de la principale stations spatial, un habitat est assisté par un tas de structures de soutien, incluant des usines en zéro-g, des rafineries de gaz et de composés organiques volatils, des fonderies, des sattelites de défense et des bases d'exploitation minières. 

Les habitats - et en particulier les plus gros d'entre eux - ont également des visiteurs. Les habitats pricnipaux sont des carrefours spatiaux. En plus des arrêts des arrêts planifiés des gros-porteurs, il peut y avoir des passagers impromptus - tel que les barges racailles, les propsecteur ou les nuées de robots autonomes et désœuvrées. 

beaucoup d'habitat ont une sorte de réseaux de tranports. C'est bien plus commun dans les gros habitats cylindrique avec une gravité centrifuge. Des solutions de transports publiques commune, incluent les trains monorails, les trams et les dirigeable bus. Des options de transport personnels incluent les vélos, scooters, motos et les ultra léger motorisés, les plus gros véhicules étant plutôt rares et générallement réservés à l'utilisation des officiels. 

La pluaprt des habitats ayant de grands espaes internes utilisent également la réalité augmentée pour créer des hallucinations consensuels de ciels et de nuages, sur lesquels la plupart des résident connectent leur canaux RA. On pourrait penser que, dans l'espace, parler de la pluie et du beau temps aurait disparu du répertoire de discussion de la tranhumanité, mais l'habitude persiste - sauf que la météo discutée est généralement virtuelle (si ce n'est une "météo" réel telles que les éruptions et vents solaires). 

\subsubsection{Colonnie en Grappes} 

Les grappes sont la forme d'habitats à microgravité la plus commune. Les grappes sont composées d'un réseau de modules sphériques ou rectangulaires fabriqués dans des matériaux léger et connectés par des canaux. généralement, les modules d'affaires et résidentiels sont regroupés autour des canaux principaux et les modules d'infrastructures tels que les fermes, l'énergie et le recyclage des ordures sont connectés sur des canaux secondaires. Des zones de gravités artificielle et limitée peuvent exister, couramment dans les parcs ou autres endroits publics et dans les modules spécialisés comme les installations de réincarnations (les morphs vivent souvent mieux le fait d'être stocké en gravité). Les principales voies de circulation du cluster peuvent avoir des "voies rapides" où des convoyeurs de boucles d'attaches en mouvement constant accélèrent les personnes qui s'y accrochent. 



Les grappes sont plsu fréquemment trouvés dans les zones riches en composés organiques volatiles comme els Troyens et les systèmes d'anneaux des géantes gazeuses (et Saturne en particulier). Les grappes sont rare dans le système Jovien car protéger une grappe de modules individuels contre la magnétoshpère intense de Jupiter est hideusement inefficace en comparaison de la protection d'une seule station plus grande. 

Les colonies en grappes peuvent avoir entre 50 et 250 000 habitants. 

\subsubsection{Bulles de Cole} 

Les bulles de Cole (ou "mondebulle") sont eseentiellement situés dans la ceintrue d'astéroïde principale, où les gros astéroîde de nickel-fer utilisés pour les construire sont abondant. Les mondebulles sont moins courants dans les Troyens et les Grecs, là où dominent les astéroïdes à croûte gelée. Une bulle de Cole est similaire par beaucoup d'aspect à un cylindre O'Neill, mais ils n'ont pas de fenêtre longitudinale. La lumière du soleil entre pas des jeux de miroirs axiaux. La bulle est également construite de manière trés différente, en utilisant un grand miroir solaire pour chauffer une poche d'eau à l'intérieur de l'astéroïde métallique pour que le métal se dilate. Les astéroîde tournant forcent le matériau malléable à prendre la forme d'un cylindre, dont les extrrmités sont refermés et l'eau est ensuite drainée. L'intérieur peut ensuite être pressurisé, aménagé et occupé. Les bulles de Cole peut également être mise en rotation pour fournir une gravité, en focntion des désirs des habitants, bien que la gravité réduise lorsque vous vous approchez des pôle et qu'elle soit nulle au niveau de l'axe de rotation. 

Les bulles de Cole font parti des plus grosse structures créées par la transhumanité dasn l'espace. L'habitat de Cole le plus grand, Extropia, a une population de 10 millions d'habitants. 

\subsubsection{Cylindre Hamilton} 

Les cylindres hamilton sont une nouvelel technologie. Il n'y a que trois cylindre Hamilton entièrement opérationnels dans le système, mais la conception laisse présager de bonens choses et sera probablement largement adopté dans les années à venir. Les cylindres hamilton se développent en utilisant un algorithme génétique complexe qui orchestre des machines de construction nanoscopiques. Ces nanomachines construise l'habitat lentement et au fil du temps, un procédé qui resemble plus à du jardinage qu'à de la construction. 

De manière similaire aux cylindre O'Neill et aux bulles de Cole, un cylindre hamilton est un habitat cylindrique tournant sur son axe le plus long pour fournir de la gravité. Deux des cylindres hamilton connus orbitent autour de Saturne entre deux anneaux, non loin de la division Cassini. De cette endroit, ils peuvent se repaître des silicates et des composés organiques volatiles en utilisant des vaisseaux moissoneurs. 

Aucune des cylindre hamilton actuellement opérationnels n'a fini sa croissance, mais on estime qu'ils pourraient héberger chacun jusqu'à 3 millions de personne. 

\subsubsection{Cylindres O'Neill} 

Trouvé principalement dans les orbites de la Terre, de la Lune, de Vénus et de Mars, les cylindres O'Neill sont les premiers modèles d'habitats spatial de grande taille et ne sont plus construit, ayant été remplacés par des modèles plus efficace, mais ils abritent toujours des dizaines de millions de transhumains. Les cylindres O'Neill ont étté construits à parti de métaux extraits de la Lune ou de Mercure, de composés organiques volatiles Lunaire (y compris de la glace polaire) et de silicates d'astéroïdes. 

Un habitat O'Neill typique fait trente cinq kilomètre de long, huit de diamètre et tourne autour de son axe long à une vitesse suffisante pour fournir une gravité Terrestre sur les murs internes du cylindre. Des cylindres plus petits existent, bien qu'ils fournissent généralement une gravité réduite (typiquement basée sur les standards Martiens). Les cylindres sont parfois connectés les uns aux autres pour des habitats particulièrement longs. Un spatioport est localisé à l'une des extrémité de l'axe de rotation du cylindre (où il n'y a pas de gravité). Ceux qui arrivent par l'espace utilisent un ascenseeur ou des micro propulsuers pour descendre sur le sol de l'habitat. 



L'intérieur d'un cylindre O'Neill est composé de six bandes alternant entre fenêtre et sol et s'tendant d'une extrémité du cylindre à l'autre. L'une des extrémité du cylindre pointe directement sur le soleil. L'extrémité opposée est le point d'amarrage de trois gigantesque réflecteurs orientés pour refléter la lumière solaire dans les fenêtres. Des matériaux intelligents revêtent les fenêtre et les réflecteurs pour prévenir les fluctuations d'activité solaire et pour ne pas rediriger trop de chaleur. l'ai à l'intérieur du cylindre et de sa superstructure métallique fournissent une protection contre les radiations. 

Le sol dans la plupart des cylindres O'Neill est composé d'un tiers d'agriculture (une combinaison de cuves nourricière et des gigantesque champs photosynthétique), d'un tiers d'espace ouvert et public et d'un tiers d'espace mixte résidentiel/affaires. Les habitats O'neill ont un cycle jour/nuit régulé par la position des mirroirs extérieurs. Les sections d'affaires et de résidences du cylindre est généralement alterné avec les espaces ouverts sur deux bandes de terrain; les terres de cultures occupent habituellement la troisième. Des passerelles traversent les fenêtre tous les kilomètres connectant les bandes de terrain. Le climat interne, le style architectural des structures et le type de faune et de flore présent peuvent varier en fonction des goûts des concepteurs de l'habitat. 

En fonction de leur taille, les cylindres O'Neill puevent héberger de 25 000 à 2 millions de personnes. 

\subsubsection{Boîte de conserve} 

Les stations de recherche antique et les avant-postes de prospecteurs survivalistes correspondent souvent à cette description. Les boîtes de conserves sont à peine plus évoluées que la Station Spatiale Internationale du début du 21° siècle. Elles sont généralement composées d'un modules ou plus, connectés à des panneaux solaires et à d'autres services par un treillis ouvert. les modèles de luxe disposent de canaux ou de voies de circulation entre les modules, alors que les configuration dépouillées nécessitent une exocombinaison ou une morph résitante au vide pour aller d'une pièce à l'autre. Les possibilités d'agriculture sont fortement limitées et il peut ne pas y avoir de lien farcast, mais les fabeurs sont disponible, ainsi que des points d'amarrage pour les navettes et les équipages de prospection. 

Les boîtes de conserve hébergent rarement plsu de 50 personnes. 

\subsubsection{Tores} 

Indifféremment appelés tore, donuts et roues, ces habitats spatiaux ciruclaires étaient une alternative bon marchés aux cylindres O'neill et ont été utilisés poru des installations plus petite. Comme els cylindres O'Neill, les tores ne sont plus construit actuellement, mais ils sont encore nombreux à être en service dans le système intérieur, plus particulièremen en orbite Terrestre et Lunaire. 

Un habitat toroïde ressemble à un donut d'1 kilomètre de diamètre, tournant auatour de grands rayons. Il y a un spatioport en zéro-g au niveau du moyeu de la roue. Les visiteurs prennent un ascensseur le long de l'un des rayons jusqu'au niveau du donut, là où la rotation crée une gravité Terrestre. 

Le plan des habitats toroïdes varient grandement, beaucoup d'entre eux ayant été conçus dasn un but scientifique ou militaire puis reconverti plus tard en habitat par des entrepreneurs ou des squatteurs. Beaucoup ont une succession de ponts dans le donut. La pluaprt d'entre eux ont été conçus comme des habitations pour le long-terme et auto-suffisante et ont donc des fenêtre en verre intelligent permettant de faire pousser des plantes sur une grande partie de la surface interne du tore. Les habitats toroïde équippés pour l'élevage sont géénralement orientés face au soleil perpendiculairement à leur axe de rotation, mais utilise ensuite une lente variation de cet axe pour créer un cycle jour/nuit. 

Les tores ont générallement été construits pour supporter de petits équipages de 500 personnes ou moins, bien que de plus gros existent et soient capables d'héberger 50 000 persones. Quelques rares habitats à double-tore existent, comme deux roues tournant des des snes opposés et connectés à leur axe. 

\subsection{Douanes Et Immigration} 

La façon dont les personnage obtiennent l'accès à un habitat et le type de contrôle qu'ils subiront dépendent de la manière dont ils arrivent. Certains habitats sont proche d'autres abris, alors que d'autres sont physiquement isolés par les vastes distances interplanétaire et le vide spatial. 

Les habitats dans les systèmes planétaire relativement densent, reçoivent la plupart de leurs visites grâce au voyage spatial conventionnel. L'infrastructure des douanes et de l'immigration est orienté vers l'acceuil des visiteurs au spatioport, et par la gestion ds arrivées d'une manière trés proche des aéroports du vingtième siècle. Les habitats isolés, d'un autre côté, tendent à recevoir la plupart de leurs visiteurs par egoscast. 

\subsubsection{Arrivés Physiques} 

Les arrivant par spatioport subissent, au minimum, une vérification d'identité de l'ego, des analyse pour détecter les apathogènes, les nanobots hostiles, les explosifs ou les raidations et une inspection de leurs effets personnels. Certains habitats vont plus loin, et incluent un contrôle secondaire rigoureux, en utilisant des essaim de nanites pour analyser tout le matériel électronique à la recherche de virus et/ou une interrogation agressive d'un fork du sujet. Même les enclaves autonomistes effectuent une analyse automatique à la recherche de tout ce qui pourrait mettre en danger l'habitat ou d'un signe d'efforts de sabotage hypercorporatiste. 

Les marchandises restreintes varient en fonction des lois locales. De nombreux habitats, et en particulier ceux qui sont contrôllés par les autonomistes ou les factions criminelles, autorisent l'armement personnel tant que rien de ce que vous utilisez ne puisse eprcer la coquee de l'habitat ou tuer des douzaines de personnes. D'autres, et notablement la République Jovienne et les stations hypercorporatistes, interdisent les armes léthales de tout type, à l'exception des personnes qui ont acquis des permis et des autorisations sépcifiques (et souvent disponibles en achetants les bonnes personnes ou en obtenant des faverus par la rep). Les armes non léthales sont géénralement utilisées. D'autres objets restreint peuvent inclure les nanofabbeurs, les essaims de nanites, les virus et logiciels de piratage, les drogues et narcorithme, certains type d'enregsitrements XP, les outils d'opérations clandestines, et ainsi de suite. Certians type de morphs peuvent aussi être restreinte, telles que les reapeurs, les furies et les élevés. 

Certains habitats insistent pour que les visiteurs - ou au moins ceux dont ils n'aiment pas la tête - se soumettent à des formes de surveillance ou de supervision spécifique pendant la durée de leur séjour. Cela peut inclure des nuée de nanites marquantes, d'héberger une IA de police dans votre insert de mesh ou même d'être physiquement attaché à un drône de sécurité. D'autres stations vont nécessiter que leurs visiteurs leur laisse un fork en guise d'assurance - au cas où ils commettent un crime, le fork peut-être interrogé. 



Enfin, et bien que rare, certains habitats vont jusqu'à charger tous les visiteurs d'une "taxe aérienne" - une taxe sur l'utilisation des ressources publiquement disponibles pendant qu'ils sont présent. C'est généralement fréquent dans les habitats isolés avec des ressources rationnées et est considéré comme particulièrement odieux par la plupart des autonomistes. 

Certains syndicats font beaucoup d'affaires dans la contrebande de certaines marchandises ou même de personens vers les habitats. Cela est géénrallement accompli grâce à du personnel de sécurit corrompu, mais c'est aussi parfois géré par des autorisations falsifiée qui permettrons au sujet d'ignorer les contrôles de sécurité. De tels services sont généralement chers. 

Pour ceux qui espèrent pouvoir entrer discrètement, il y a toujours l'option de faire une ballade dans l'espace et d'essayer de rentrer par un sas non-surveillé. De tels tentatives sont relativement dangereuses et inutile, la plupart des habitats ont des capteurs et des systèmes de sécurités dédiés à la surveillance de la surface externe et des points d'entrée en particulier. Mais cela reste une possibilité pour une équipe pleine de ressource avec un hackeur compétent, les bots de périmétrique lourdement armés représentant un danger particulier. 

\subsubsection{Arrivés Électroniques} 

Les arrivants par egocast sont parfois interrogés par les autorités de l'habitat dans un simulspace avant de pouvoir se réincarner. En fonction de l'attitude de l'habitat vis à vis des droits civiques, ce procédé peut-être relativement raisonnable ou particulièrement invasif. Un contrôle d'entrrée minium consiste en une vérification d'identité, un entretien rapide avec une IA des douanes et d'une revue des spécificités de la morph dans laquelle l'ego arrivant prévoit de s'incarner. Les habitats avec des mesures dimmigration draconienne peuvent utiliser des techniques d'interrogations violentes par psychochirurgie sur les infomorphs suspectes. Les sauvegarde d'egocast n'ont que peu de recours pour éviter ce traitement - les autorités des stations peuvent simplement les archiver en stockage mort si elles le décide - il est docn sage de se renseigner sur les procédures douanièrs avant de s'expédier quelque part. 

De nombreuses personnes, en particulier les autonomistes et les bordés, n'appréciant pas ce type de réception, de nombreux services ont commencé à fournir des réincarnation pré-douanes pour les personnage voyageant vers des habitats ayant des méthodes de contrôles suspectes. Pour un prix souvent exorbitant, les voyageurs s'egocatsent dans une sous-statione xtraterritoriale proche de la destination voulue, se réincarnent là-bas puis voyagent vers la destination par navette. 

Différents services de darkcast, généralement proposés par les syndicats du crime en place, proposent parfois une méthode alternative pour s'egocast vers l'intérieur et parfois même de se réincarner. Les services de darkcast sont particulièrement cher cependant, et le personnage et à la merci des opérateurs du syndicat. Dans de rare cas, certaines factions politique ou même des hypercorporation peuvent faire fonctionner leur propres système de darkcast, qu'un personnage ayant de bonnes compétences en réseau pourrait utiliser. 

\subsection{Voyage Spatial} 

Dans certaines circonstances, les personnages préfèreront voyager physiquement à travers l'espace plutôt que par egocast. À \textit{Eclipse Phase} les vaisseaux spatiaux sont considérés comme un environnement plutôt que comme des véhicules/équipements à utiliser. Les vaisseaux restent essentiellement pilotées par l'IA de bord. Même si les personanges peuvent également prendre le relai avec leur compétence Piloter: Vaisseaux, la situation le nécessite rarement. 

\subsubsection{Voyage Local} 

Dans les systèmes planétaires densément habités tels que Mars ou Saturne, la plupart des voyages entre les cités, les stations de surface et les habitats orbitaux dans les 200 000 kilomètres se font par des fusées propulsée à lhydrogène (ou parfois au méthane). Cette forme de voyage est extraordianirement peu onéreuse, rapide et évite els défauts de personnalités qui plombent l'egocast. Les VTOT (véhicule de transfert orbital et terrestre, p. 348) sont couramment utilisés. les vaiusseaux quittant un corps planétaires ont besoin d'être capable de générer suffisament de poussée pour s'échapper du puit gravitationnel (voir \textit{Échapper Aux Puits Gravitationnels}, p. 346). 

\subsubsection{Voyage Distant} 

Pour les distances entre 200 000 et 1,5 millions de kilomètres, des vaisseaux bien plus gros (et plus chers) propulsés par des motuers à fusion - et à plasma - fotn des lignes régulières. Les moteurs nucléaire à ionisateur électrique étaient auparavant utilisé sur certaines de ces routes, mais le rendement moindre de ces système à fission et le besoin de réaction radiocative à base de métaux lourd font qu'ils ne sont presque plus utilisés. Des courers bien plus rapide et propulsés à l'antimatière sont également couramment utilisés. Ces vaisseaux n'ont pas la poussée nécessaire pour échapper aux puits gravitationnels des grosses planètes et des lunes, ils stationnent donc en orbite, et utilisent des vaisseaux plus petits (des VTOT typiquement) avec des poussées plus importante pour transporter les personnes depuis et vers la surface de la planète. Pour les distances supérieures à 1,5 millions de kilomètres, quasiment tout le monde utilise l'egocast. 

\subsubsection{Base du Voyage Spatial} 

Les vaisseaux spatiaux utilisent différent type de propulsion à réaction (voir \textit{Propulsion des Vaisseaux}, p. 347), ce qui signifient qu'ils brûlent un carburant (une masse réactive) et qu'il dirigent la sortie de chaleur dasn une direction, ce qui pousse le vaisseau spatial dans la direction opposée. Voyager sur de longues distance implique généralement une phase de combustion pendant plusieur heures pour avaoir une accélération élevé, pendant laquelle près de la moitié de la masse réactive est brûlée pour augmenter la vitesse du vaisseau. Il passe ensuite la majorité du voyage à cette vitesse, jusqu'au moment où il approche de sa destination, moment où il se retourne et brûle une quantité agle de masse réactive dans la direction opposée pour réduire la vitesse. Comme certains vaisseaux brûlent la moitié de leur masse réactive pour aller à la meilleure vitesse possible, cela ne laisse pas beaucoup de place pour des manœuvres supplémenataire ou pour les urgences. Beaucoup de vaisseaux ne brûlent donc qu'entre un quart et un tiers de leur carburant lors de l'accélération intiale afion de conserver une réserve au cas où. Quelques astuces peuvent être utilisés pour économiser du carburant et augmenter la vitesse, telles que l'utilisation de l'effet de fronde des puits gravitationnels des plus grosse planète ou l'aérofreinage dans l'atmosphère haute des planètes. les temps de voyage entre les lieux changent constament, les différents corps planétaires se déplaçant sur leurs orbites autour du système solaire. À l'intérieur d'un groupement ou d'un système planétaire, le voyage est une histoire d'heures. À l'intérieur du système intérieur, le voyage peut prendre plusieurs jours, voire semaines. Le voyage vers, depuis ou à l'intérieur du système extérieur peut prendre bien plus de temps, et est généralement l'affaire de quelques mois. 

La pluaprt des vaisseaux fonctionnent en gravité zéro, à l'exception des quelques vaisseaux les plus gros et qui sont capable de faire tourner les modules d'habitations pour entretnir une gravité faible. Les périodes d'accélération fournissent également une gravité temporaire vers le bas, en direction de la combustion. 

L'espace est un luxe à bord des vaisseau, les pièces sont docn souvent étroites. Les dortoirs et quartiers pesonnels sont rarement plus gros que de grand placards, avec juste suffisament de place pour un sac de couchage et quelques effets personnels. En fonction de la taille du vaisseau, il pourrait y avoir une zone commune de divertissement. L'équipage a tendance à être les seuls actifs au début et à la fin du voyage, lorsqu'ils doivent s'occuper de l'accélération/décélération et pour les manœuvres dans le traffic spatial. Ils passent le reste du voyage à s'occuper des réparations ou en tuant le temps d'une manière ou d'une autre, souvent en se connectant à des simulations XP ou RV ou en jouant à des jeux RA. Bien que les vaisseaux spatiaux aient généralement leur propre réseau meshé, ils sont généralement trop éloignés des réseaux meshés d'autres habitat pour intergair sans un décalage significatif, ils doivent docn embarquer leurs propres archives d'options de divertissement. Beaucoup de vaisseaux long-courrier sont pilotés apr des morphs hibernoïdes, qui se renferment pour de longues siestes. 

\subsubsection{Combat Spatiaux} 

Le combat spatial à tendance à se prendre place sur de longues distances, en utilisant des armes à rayons massives, des armes à rail et des missiles. Il a aussi tendance à être vicieux, brutal et rapide. Des dégats significatifs à un vaisseau peuvent causer une décompression atmosphérique, tuant toutes les biomorphs qui ne sont pas équipés et attachés. 

Pour l'essentiel, il est recommandé de traiter le combat spatial comme un outil scénraistique, un élément de l'histoire qui aide à créer du drame et de la tension plutôt que comme un évènement auquel les personnages participent activement. Cela ne veux pas dire que les personnages n'ont aucun rôle à jouer das le combat, ni que leurs actions n'auront aucun effet sur l'issue du combat. Ils peuvent être impliqué dans le contrôle des dégâts, dans la négociation avec les forces hostiles, faire aprti des équipes d'abordages, utiliser les armements avec la compétence Artillerie, démarrer une mutinerie, tenter de pirater les réseaux des vaisseaux approchant, s'échapper par le sas, se cacher pendant que les pirates pillent le vaisseau ou toute autre problématique similaire. Il est cependant recommandé au maître de jeu d'éviter les situations de combat dans l'esapce profond qui pourraient rapidement mener à la disparition de toute l'équipe à cause de quelques mauvais jets de dés. 

\section{Nanofabrication} 

Afin de créer un objet dans un nanofabeur (qu'il s'agisse d'une machine d'abondance, d'un fabeur ou d'un faiseur; voir p. 327), trois choses sont nécessaires: des matériau bruts, des plans et du temps. 

\subsection{Matériaux Bruts} 

Les matériau bruts sont générallement facile à acquérir, la plupart des nanofabeurs sont équipés d'unité de désassemblage qui sont capable de réduire à peu près tout en molécules consistantes. Des cartouches peuvent aussi être achetée (à un prix Trivial). beaucoup d'habitat rediregent leur système de gestion des déchets et de recyclage directement dans les désassembleurs. 

\subsection{Plans} 

La plupart des nanofabeur sont pré-chargés avec des plans d'usage général: nourriture, vêtements de base, outils basique, etc. Les plans pour d'autres biens peuvent  être acquis de différentes manière: 

\begin{itemize} \item Ils peuvent être achetés en ligne (légalement ou via le marché noir). \item Ils peuvent être trouvés gratuitement en ligne (voir plus bas). \item Ils peuvent être récupéré par dela Rep, en suivant les règles habituelles des réseaux sociaux (p. 285). \item Ils peuvent être volés (généralement en piratant un site mesh ou un naofabeur contenant de tels plans). \item Ils peuvent être auto-programmés (voir plus bas). \end{itemize} 

\noindent Une fois que le plan a été récupéré, ils peuvent être simplement chargés dans le nanofabeur. 

\subsubsection{Plans Open Source} 

Les plans de nombreux bien peuvent être trouvé librement en ligne, disséminés par un mouvement open source dynamique. La disponibilité de tels plans dépend générallement du mesh local. Dans les habitats autonomistes, un simple Test de Recherche est suffisant pour trouver les plans open source dont vous avez besoin (en appliquant des modificateurs pour les objets inhabituels). Dans les habitats plus restrictifs, les plans open source peuvent être plsu difficile à trouver, puisque qu'ils seront cachés et tenues à l'écart des yeux des autorités. Dans ce cas, le personnage devra utiliser sa Rep, payer un groupe de hackeur locaux ou quelque chose de similaire pour pouvoir accéder aux plans. 

Notez que les nanofabeurs restreint peuvent ne pas accepter les plans open source (voir \textit{Restrictions des Plans}). 

\subsubsection{Restrictions des Plans} 

Certains nanofabeur sont équipés de restrictions pré-programmés pour ne pas accepter de plans pour des objets restreints (tels que les armes) ou des objets sans licence (tels que les plans issus du marché noir ou open source). Ces restrictions peuvent être contournés en piratant le nanofabeurs et en le re-programmant, en suivant les règles de piratage normale (p. 254). 

\subsubsection{Programmation de Plans} 

Un personnage motivé peut simplement décider de programmer ses propres plan, bien que c'est une activité consommatrice de temps. Pour y parvenir, le personnage doit faire un Test de Programmer (Nanofabrication) avec un intervalle d'une semaine par niveau de prix de l'objet. Par exemple, un objet de prix Trivial nécessite 1 semaine, un objet à prix Bas nécessite 2 semaines, un prix modéré en nécessite 3, et ainsi de suite. Une compténce Académique: Nanotechnology ou une autre appropriée à la conception de l'objet peut-être utilisée comme compétence complémentaire (voir p. 173) pour ce test. Un fork ou une muse peut également être assigné à une telle tâche de programmation. 



\subsection{Temps} 

Une fois que les matériaux bruts et les schémas sont prêts, la plupart des nanofabeurs n'ont besoin que de temps. L'intervalle exact pour créer un objet varie, mais est grosso modo équivalent à 1 heure par catégorie de prix de l'objet (1 heure pour Trivial, 2 pour Bas, 3 pour Modéré, etc.). Le maîþre de jeu doit se sentir libre de modifier cette période de manière appropriée à l'objet créé. 

\subsection{Le Test de Programmation} 

La nanofabrication est généralement gérée par un Test de Programmer: Nanofabrication. Généralement, il s'agît d'un Test de Réussite Simple (p. 118), un jet raté indiquant simplement que l'objet à quelques imperfections mineures, ou prend plus de temps à faire. 

Dans certains cas, le maître de jeu peut demander un Test de réussite standard, indiquant que l'échec est plus qu'une possibilité. Cela ne devrait être fait que pour les objets exotique, extrêmement complexes ou pour ceux dont les plans sont incomplets ou suspects. Ce test peut aussi être fait si les matériaux de bruts sont limités. 

Le personnage manipualnt le nanofabeur peut faire ce test ou il peut être laissé à la charge de l'IA intégrée au nanofabeur. De tels IA ont une compétence de Programmer (Nanofabrication) de 30 (voir \textit{ IA et Muses}\textit{,} p. 331). 

\section{Réputation et Réseaux Sociaux} 

\begin{quote} "Il était une fois, il y avait une planète tellement primitive que ses habitants utilisaient encore de l'argent. Cette planète s'appelait 'Mars'." 

— Professeur Magnus Ming, Université Autonome de Titan \end{quote} 

Le conflit entre l'économie de marché et d'autres formes d'économie est l'une des plus grande guerre culturelle de la transhumanité, et elle n'est toujours pas terminée. L'expansion transhumaine à travers le système solaire a créé une myriade d'opportunité pour expérimenter de nouveau systèmes économiques. beaucoup ont échoués, mais l'économie réputationnelle du système extérieur s'est montré à la fois utilisable et robuste d'une manière qu'aucun adversaire de l'aconomie de marché n'avait atteinte. 

L'économie réputationnelle, parfois appelée économie du don ou économie ouverte, est l'une dans laquelle l'abondance créée par la nanofabrication et la longévité accordée par l'upload et la sauvegarde ont permis de supprimer les considérations d'approvisionnement contre rareté de 'léquation économique - détruisant l'économie classique dans le procédé. 

Les sociétés régulés du système intérieur et la Junte Jovienne ont utilisé un contrôle sociétale et une régulation attentive des technologie d'abondance dans leur population, les maintenant dans une économie transitionnelle qui est largement une extension des économies classique. Personne ne pourrait s'en tirer avec un tel système dans le système extérieur. Dans les Troyens et les Grecs, l'essentiel de la ceinture, la partie libre de Jupiter, et partout à partir de SSaturne, l'économie réputationnelle domine. 

Comment cela est arrivé? D'une aprt, la monnaie est une nuisance lorsque vous êtes un membre autonome d'un collectif autonome dont le voisin le plus proche (à au moins 100 000 kilomètres de là) est également un collectif autonome. Vous êtes tous complètement auto-suffisant d'un point de vue des ressources matérielles. Vous avez une flotte de robot qui moissone l'eau, les composés organiques volatiles, les masses réactives, les métaux et les silicates. Vous avez un nanofabeur pour fabriquer tous les petits objets, une usine communautaire pour les plus gros et un atelier automatisé ou vous pouvez fabriquer tout le reste - avec l'aide et les conseils d'une IA ayant l'expérience et les compétences combinées d'une équipe de conception de haut-vol, si jamais vous en avez besoin. Vous faites pousser votre propre nourriture. 



L'argent est pour les personnes qui ne savent pas prendre soin d'eux-même. La transhumanité n'est qu'à quelques décennies d'être une civilisation Kardashev mature de Type I, ayant largement maîtrisé les ressources matérielle de son propre système solaire. Un personnage du système extérieur trouvera probablement le concept de monnaie comme génant. 

Cependant, l'abondance matérielle n'a pas éliminé la valeur de certains biens et services. Le repas d'un transhumain est peut-être gratuit, mais des idéées innovantes, une nouvelel conception, les soins médicaux, le sexe et le sale boulot ne pousse pas dans les fabeur. Que se passe-t-il si vous avez besoin d'une thérapie génique pour vous faire pousser des cellule senseibles aux infrarouges sur votre visage? Et si vous voulez assassiner votre fork béta renégat après qu'elle ait balancé une grenade hallucinogène lors du vernissage de votre gallerie et qu'elle ait kidnappé votre petit copain? Et si vous avez réllement besoin d'une fessée? Vous faites appel à votre réseau social. Si votre réseau est suffisant densément peuplé et que votre réputation est suffisament bonne, vous trouverez quelqu'un pour vous aider. 

Dans le système intérieur, l'économie de réputation ne remplace pas la monnaie pour l'échange de biens et de services, mais elle règne sur un réseau de faveurs et d'influence. Appeler vos contacts, obtenir de l'information et s'assurer que vous êtes au meilleur endroit pour voir et être vu, tout cela implique de faire appel à votre réseau social. 

\subsection{Réseaux Sociaux} 

Les réseaux sociaux représentent les perosnnes que vous connaissez, ceux qu'ils connaissent et ainsi de suite. Il commence avec vos amis et votre famille, se propage à vos collègues, voisins et co-travailleirs et s'étend jusqu'à vos connaissance, de la serveuses néo-hominidée de votre café préféré à la sylph avec qui vous avez flirté en boîte. Dans l'univers perpétuellement en ligne et entioèrement meshé d'\textit{Eclipse Phase,} cela va encore plus loin, englobalnt toutes les personnes avec qui vous vous êtes liés via les réseaux sociaux sur le mesh, quiconque suit votre blog/lifelog/mises à jour, et tout ceux qui interagissent avec vous sur différents forums du mesh. Ajoutez-y maintenant les amis d'amis et tout ceux qui ont l'impressionante capacité à atteindre les eprsonnes squ'ils connaissent, ceux qu'ils connaissent de vue et ceux que vous ne connaissez pas mais qui vous sont connectés d'une manière ou d'une autre. 

Bien entendu, les réseaux sociaux ne sont pas homogènes. Parmi les rang évoluant constament de la transhumanité, il y a une tendance à se regrouper autour de différentes caractéristiques communes, qu'il s'agisse d'un historique culturel, d'intérêts personnels, de liens professionels, de connexion locale, d'affiliations politiques, d'obessions sous-culturelle ou simplement d'intérêts communs tels que faire partie de la même clique. Le réseau sociale d'un hackeur info-anarchiste n'aura que peu de point commun avec celui d'une célébrité hypercorporatiste ou de celui d'un bordé isolé. Néanmoins, les réseaux sociaux se recoupent relativement fréquement, souvent de manière inattendue et intéressante. La plupart des personnes peuvent être considérés comme membres de différnts type de réseaux sociaux. Cette superposition est ce qui connecte ensemble des groupes ddisparates de transhumains. 

\subsubsection{Réseau} 

Être simplement connecté ne signifie bien entendu pas que vous avez plusierus milliers de personnes disponibles à la demande. Si vous espérez rassembler els dernières rumeurs, obtenir des conseils d'un expert, trouver quelqu'un qui peut résoudre votre problème, acquérir de la technologie du marché gris ou propager un mème, vous avez besoin de savoir à la fois \textit{à qui} parler dans ce réseau social et \textit{comment} obtenir ce dont vous avez besoin, particulièrement si vous espérez gardez le tout discret sans attirer l'attention. 

C'est là que votre compétence Réseau: [Domaine] entre en jeu (p. 182). Réseau représente votre capacité à manœuvrer dans cette toile de connexions personnelle et impersonnelles pour trouver ce dont vous avez besoin. Cela peut se faire par le bouche à oreille, en postant les bonnes requètes aux bons endroits sur le mesh, en supervisant les bons profils personnels et forums, et utilisant la puissance de la foule par le crowdsourcing, ou par n'importe quelel autre tactique créative similaire. 

Chaque domaine que vous possédez avec la compétence Réseau représente un sous-réseau particulier, un intérêt commun qui lie les personnes ensemble. La pluaprt de ces domaines sont basés sur des factions (Autonomistes, Hypercorporatistes, etc.) et se connecte sur un réseau réputationnel dédéi (voir table des Réseaux Réputationnels, p. 287). À la discrétion du maîþre de jeu, d'autre groupement de personnes peuvent être connectés grâce à ces compétences et des systèmes de rep. Par exemple, les artistes et les journalistes de tout bords peuvent être regroupés sous la compétence Réseau:Média et le f-rep, peu importe qu'ils soient autonomistes ou hypercorporatistes. De manière similaire, étant un groupe diversifié, les bordés ne tombent pas universellement dans l'une des catégories, et sont plutôt répartis entre elles. Si le maîþre de jeu et les joueurs se mettent d'accord, d'autre domaine de Réseau peuvent être ajoutés, représentants d'autres sphères d'intérêts, tels que Jeux RA, Sports, Fiction Gore, etc. 

La façon exacte d'exploiter vos réseaux sociaux sont notés ci-dessous. Dans certains cas les éléments clef sont qui vous connaissez et à quel point vous êtes bon pour les atteindre, dans d'autres ce sont comment \textit{vous} êtes connus. Vous pourriez être connectés à des milliers de personnes, mais si vous n'avez aucun poids, vos efforts pour utiliser ces connexions seront limités. C'est ici que la Réputation entre en jeu. 

\subsection{Réputation} 

La réputation est une estimation de votre valeur sociale. Dans léconomie de don du système extérieur, la réputation sociale a remplacé efficacement la monnaie. La répitation est cependant bien plus stable que le crédit. 

Dans \textit{Eclipse Phase,} les scores de réputation sont facilités par les réseaux sociaux en-ligne. Presque tout le monde est membre d'au moins l'un de ces réseau réputationnels. C'est une tâche triviale d'obtenir le score actuel et l'historique de la Rep de quelquu'n avec qui vous faites affaire - votre muse le fait souvent automatiquement, affichant un badge entoptique de score de Rep a tous ceux avec qui vous interagissez, mis à jour en temps réel, vous verrez donc si ils sombrent d'un coup ou si ils deviennent populaire. Les 7 réseaux les plus communs sont notés sur la table des Réseaux 

Réputationnels. Les maîtres de jeu et les personnages peuvent décider d'en ajouter d'autres appropriés à leur partie. Vous achetez un scoe de Rep dans un ou plus de ces réseaux lors de la création de personnage. Les scores de Rep sont compris entre 0 et 99, comme les coméptences. Ce score représente votre capacité à acquérir des biens, des services, des informatiosn et des faveurs, comme noté ci-dessous. Ces scores peuvent fluctuer pendant la sessiond e jeu, en fonction des actiosn de votre personnage. 

\subsection{Utilisation des Réseaux et de la Rep} 

En terme de jeu, vous bénéficiez de vos connexions et de votre crédit personnel à chaque fois que vous avez besoin d'une \textit{faveur.} Une faveur est en gros tout ce que vous pouvez essayer d'obtenir par votre réseau social, qu'il s'agisse d'information, d'asssitance, de marchandises et ainsi de suite. Différents type de servcies sont décrit à \textit{Faveurs,} p. 289. 

\subsubsection{Le Test de Réseau} 

Pour obtenir une faveur, il faut commencer par par demander. Cela nécessite un Test de Réseau pour déterminer si vous pouver trouver la/les personnes/informations dont vous avez besoins. Cela représentent le fait de parler aux personnes que vous connaissez, de faire passer le mot aux personnes qu'ils connaissent, d'envoyer des demandes sur les réseaux sociaux, de creuser différents profils et salles de discussion, etc. pour trouver quelqu'un qui pourrait vous aider. Les Test de Réseau sont soumis à des modificateurs en fonction du niveau de la faveur (voir plus bas), le niveau de discrétion voulu par le personnage (voir plus bas) et tous les autres facteurs noté sur la table des Modificateurs de Réseau ou déterminés apr le maître de jeu. Les Test de Réseau sont des Actiosn de Tâches - cela nécessite du temps de demander des faveurs ou de recouper des informations. L'intervalle dépend du niveau de la faveur, comme noté sur la table des faveurs, p. 289. 



\begin{table} \caption{Réseaux Réputationnels} \begin{tabularx}{\textwidth}{|X|l|l|X|} \hline



Nom du réseau &Nom de la rep &Domaine de Réseau &Factions et autres \\ \hline

La Liste Cercle-A &@-Rep &Autonomistes &Anarchistes, Barsoomiens, Extropiens, Titaniens et racaille \\ \hline

CivicNet &c-Rep &Hypercorporations &Hypercorporatistes, Joviens, Lunaires, Martiens, Vénusiens \\ \hline

EcoWave &$e-Rep &Ecologistes &Nano-écologistes, préervationnistes et réclamationnistes \\ \hline

Notoriété &f-Rep &Média &célébrités (ainsi que les artistes, l'élite et les médias) \\ \hline

Guanxi &g-Rep &Criminels &Criminels \\ \hline

The Eye &i-Rep &Firewall &Firewall \\ \hline

Research Network Associates &r-Rep &Scientifiques &argonautes (ainsi que les technologistes, les chercheurs et les scientifiques) \\ \hline

\end{tabularx} \end{table} 

\subsubsection{Niveau de faveur et Modificateurs} 

Les scores de Rep sont divisés en cinq niveaux, reflétant votre attitude au sein de cette communauté. Chaque tranche de 20 points de Rep équivaut à un niveau. Voir la table des Niveaux de Réputation pour la répartition. 

De manière similaire, les faveurs sont également réparties en cinq niveaux, de Triviale à Rare (voir \textit{Faveurs, } p. 289, pour des exemples spécifiques). Le niveau standard de faveur auquel vous pouvez vous attendre sur un réseau social est basé sur votere niveau de Rep dans ce réseau. Si vous cherchez à obtenir une faveur au-delà de votre niveau, cla reste possible, mais vous subirez un modificateur négatif sur votre Test de Réseau. Cela reflète le fait que quelqu'un ayant un niveau assez bas peut trouver difficile d'atteindre certaines personnes pour les aider. De manière similaire, si vous cherchez à obtenir une faveur d'un niveau inférieur au votre, vous recevrez un modificateur positif à votre Test de Réseau, reflétant le fait que le prestige facilite l'acquisition des petites choses dont vous pourriez avoir besoin. Pour cahque niveau de faveur au-dessus ou au-dessous de votre score de Rep, appliquez un modificateur de + ou -10 respectivement. 

\begin{table} \caption{Modificateurs de réseau} \begin{tabular}{|l|l|} \hline

SITUATION &MODIFICATEUR\\ \hline

Le niveau de Faveur dépasse le niveau de Rep &–10 par niveau \\ \hline

Le niveau de Rep dépasse le niveau de Faveur &+10 par niveau \\ \hline

Rester discret &- Variable (voir p. 288) \\ \hline

Brûler de la Rep &+quantité de Rep brûlée\\ \hline

Payer plus cher &+10 par niveau \\ \hline

\end{tabular} \end{table} 

\begin{table} \caption{Niveaux de Réputations} \begin{tabular}{|l|l|} \hline

SCORE DE REPUTATION &NIVEAU DE REPUTATION \\ \hline

0–19 &Niveau 1 \\ \hline

20–39 &Niveau 2 \\ \hline

40–59 &Niveau 3 \\ \hline

60–79 &Niveau 4 \\ \hline

80–99 &Niveau 5 \\ \hline

\end{tabular} \end{table} 

\begin{quotation} Jacqui est sur une barge racaille et elle a besoin de mettre rapidement la main sur des armes. Elle a une arme particulière en tête, mais elle est chère - son prix est Élevé. Elle décide que sa meilleure approche est de aprler à la racaille à bord du vaisseau et de trouver quelqu'un qui pourra lui fournir ou lui vendre cette arme, en utilisant son @-rep et sa compétence Réseau: Autonomiste de 50. Acuqérir un objet de prix Élevé compte comme une faveur Élevée de niveau 4 (voir Acquérir/Fourguer des Marchandises, p. 289). La @-rep de jacqui est de 53, ce qui n'est que de Niveau 3. Puisque la faveur est d'un level plus élevé que son niveau de rep, elle souffre d'un modificateur de -10 à son Test de Réseau. Jacqui doit docn obtenir 40 ou moins (50 - 10) pour trouver un fournisseur d'arme. \end{quotation} 





\subsubsection{Payer/Troquer pour des Faveurs} 

Les faveurs ne sont pas forcément gratuite. En fonction de ce que vous recherchez, vous pourriezz devoir faire des échanges pour l'obtenir. 

Dans les économies capitalistes et transitionnelles du système intérieur et de la Junte Jovienne, vous pouvez devoir acheter les biens et services que vous recherchez en utiliasnt des crédits. On peut même payer pour avoir des informations en arrosant la bonne personne. Une fois qu'il est dépensé, ces crédits disparaissent jusqu'à ce que vous en regagniez. 

Dans l'économie réputationnelle anarchiste du système extérieur, vous pouvez avoir ce que vous voulez gratuitement. Dans ce cas, vous récupérez les marchandises et les servcies en fonction de la force de votre réputation. 

\begin{quotation} Jacqui obtiens un 39 - elle réussit! Après avoir publié quelques demande sur le réseau social de la racaille (ellene s'inquiète pas vraiment de la légalité ou de dissimuler ce qu'elle fait - c'ets un vaisseau racaille après tout), elle est orientée vers un veandeur d'arme ayant une bonne Rep. Bien qu'un amrchant d'amre racaille vend normalement du matériel contre des crédits, Jaqui est elle-mrme une racaille et elle est donc capable d'utiliser son attitude communautaire pour obtenir gratuitement l'arme. Cela utilise cependant une faveur Élevé. \end{quotation} 



\subsubsection{Les Limites de la Réputation} 

Même dans l'économie du don, la réputation à ses limites. Il y a des limites à la fréquence à laquelle vous pouvez demander de l'aide avant que l'on commence à vous considérer comme insistant ou comem un parasite. En terme de jeu, cela est exprimé avec un \textit{délai de rafraichissement} - la quantité de temps que vous devez attendre avant de chercher de nouveau une faveur de cette importace sans paraître exigeant. Les délais de rafraîchissement sont notés sur la table de Faveur (p. 289). 

Si vous devez obtenir une autre faveur avant que le délai de rafraichissement soit expiré, vous avez deux choix. Vous pouvez dépenser une faveur de niveau plus élevé, en gardant en tête que els faveurs de plus haut niveau se raffraichissent plus lentement. Vous pouvez aussi brûler de al réputation (voir plus bas). 

\begin{quotation} Maintenant que Jaqui a récupéré son arme, elle a besoind 'une autre faveur - elle doit trouver quelqu'un qui ne veut pas être trouvé. La personne qu'elle recherche est une racaille, elle décide donc de refaire appel à la racaille pour l'aider. Le maître de jeu décide quec 'est une autre faveur de Niveau 4 (voir Acquérir des informations, p. 291). Encore une fois, avec sa compétence Réseau: Autonomiste de 50 et sa rep de Niveau 3, elle doit obtenir 40 ou moins. Elle obtient un 21 et trouve quelqu'un qui a l'information dont elle a besoin. Elle a maintenant un choix à faire. Pour obtenir cette information, elle doit soit payer la personne en crédit (un prix Élevé) ou dépenser une autre faveur de Niveau 4. Elle est un à sec, elle décide donc d'utiliser de nouveau sa réputation. Les faveurs de Niveau 4 ne se rafraichissent qu'une fois par mois et elle a utilisé sa dernière il y a quelques heures. Son seul choix est de dépenser une faveur plus élevée, elle l'étend donc au Niveau 5 pour avoir les infos dont elle a besoin. \end{quotation} 



\subsubsection{Brûler de la Réputation} 

Dans certains cas, obtenir ce que vous voulez peut-être plus important que de ne pas marcher sur les tentacules de quelqu'un. Dans les situations urgente, vous pouvez \textit{brûler} une partie de votre score de Rep pour réussir à terminer le boulot, ce qui signifie que vous échangez une perte de Rep pour obtenir une faveur. Cela reflète le fait que vous dépassez les limites pour forcer la main des personnes qui veulent vous aider. Bien que vous pourriez bien obtenir ce que vous voudriez, votre réputation en-ligne encaisse une baisse, les gens vous étiquetant comme étant toujours dans le besoin. 

Il y a deux raisons pour brûler un score de Rep. La première est d'obtenir un bonus à votre Test de Réseau. Cela indique que vous tirer des ficelles et que vous insistez auprès des gens pour obtenir la faveur que vous recherchez. Cela est particulièrement utile lorsque vous essayez d'obtenir des faverus d'un niveau suéprieur à votre Rep, mais usez en trop et vous n'aurez bientôt plus de statut social. Chaque point de Rep que vous brûlez vous donne un modificateur positif équivalent sur le test de Réseau, jusqu'à un maximum de +30. 

La deuxième option est de brûler de la Rep pour obtenir une faveur qui n'est pas encore rafraichie. Cela reflète le fait de demander trop de chsoes pendant une période trop courte. La quantité de Rep que vous devez brûler dans ce cas dépend du niveau de faveur que vous cherchez, tel que noté sur la table des faveurs (p. 289). 

\begin{quotation} Jaqui a son arme et la localisation de sa cible, mais elle a encore besoin de quelque chose d'autre: un hackeur. Elle a besoind e quelqu'un qui peut ouvrir quelques portes et défaire la sécurité système pour qu'elle puisse atteindre sa cible dans sa planque. Comme elle est sur une barge racaille, Jaqui penses que, encore une fois, sa meilleure option est de travailler sur ses contacts racailles. Le maître de jeu détermine qu'il s'agît d'une autre faveur de Niveau 4. Elle réussit encore une fois son Test de Réseau avec un 13 pour une difficulté de 40 - sa chance ne semble pas l'abandonner. Elle trouve un hackeur, masi elle doit maintenant faire un échange pour obtenir ses services. Elle décide encore une fois de ne pas dépenser de crédit et d'utiliser son @-rep. Elle a déjà utilisé ses faveurs de Niveau 4 et 5 sur son @-rep, elle n'a donc pas d'autre choix que de brûler de la réputation. Une faveur de Niveau 4 coûte 10 points de Rep. Jaqui les dépense, descendant son @-rep de 53 à 43 - elle a demandé beaucoup de faveurs en trés peu de temps, et ses amis et connaissances lui font comprendre leur désagrément en réduisant son niveau social. \end{quotation} 

\subsubsection{Rester discret} 

Le problème avec l'utilisation des réseaux sociaux pour obtenir des faveurs est que vous finissez par laisser beaucoup de personnes savori ce que vous magouillez. Lorsque vous êtes impliqués dans des opérations clandestines, cela est exactement ce dont vous ne voulez \textit{pas}. La seule façon de diminuer cela est de ne transmettre votre requète qu'à des amis de confiance et de leur demander de faire profil bas, mais cela réduit la quantité de personne qui peuvent vous aider. 

En terme de jeu, vous pouvez essayer de rester discret par rapport à ce que vous faites, mais cela augmente la difficulté pour trouver ce que vous cherchez. Pour chaque modificateur négatif que vous appliquez à votre Test de Réseau, le même modificateur négatif s'applique aux Tests de réseau de tout ceux qui veulent savoir ce que vous fabriquez. 

\begin{quotation} Enr evisitant l'un des exemples précédents, nous revenons au momemtn ou Jaqui essayait de valider la localisation de la planque de sa cible. Comme sa cible est une racaille, qu'elle est sur un vaisseau racaille et que Jaqui utilise sa compétence Réseau: Autonomiste pour la trouver, il y a une bonne chance que si Jaqui commence à poser trop de questions, sa cible pourrait finir par l'apprendre. Jaqui ne veut pas que sa cible sache qu'elle est sur ses traces, elle décide donc de demander plus discrètement. Elle applique un modificateur de -20 à son Test de Réseau, passant le seuil de 40 à 20. Commenoté plus haut, elle obtient un 21 et échoue donc. Elle dépense un point de Moxie pour inverser le jet, le transformant en 12 -une réussite. Comme Jaqui a prit cette pénalité de -20, représentatn le fait qu'elle garde un profil bas, sa cible subit un modificateur de -20 lorsqu'elle doit faire son Test de Réseau pour voir si elle entend parler de quelqu'un posant des questiosn à son sujet. \end{quotation} 

\subsection{Faveurs} 

Les jouerus créatives peuvent sans aucun doute trouver de nombreuses utilisations pour leur réseaux sociaux, mais quelques uns des plus communs sont détaillés ici. Les maîtres de jeu devraient utilsier leur discrétion quand au niveau d'interprétation des interaction et des Test de Réseau sont inclut dans l'utilisation d'un réseau social. Pour les biens normaux, les demandes d'informations basiques ou les faveurs les plus petites, ni lancement de dé ni intrerprétation pourraient n'être nécessaire. Pour les requètes improtante, les interactions avec les contact et l'assistance en mission, les jets de dés et/ou l'interprétation des interactions avec les contacts du réseau social devraient généralement être fait. Las maîtres de jeu peuvent souhaiter suivre les contacts des PNJ dans chaque réseau social des personnage et les transformer en personnages récurrents. 

\begin{table} \caption{Faveurs} \begin{tabular}{|l|l|l|l|} \hline

NIVEAU DE FAVEUR &INTERVALLE &COÛT EN REP BRÛLÉE &TEMPS DE RAFRAÎCHISSEMENT\\ \hline

1 (Trivial) &1 minute &0 &1 heure \\ \hline

2 (Bas) &30 minutes &1 &1 jour \\ \hline

3 (Modérée) &1 heure &5 &1 semaine\\ \hline

4 (Élevée) &1 jour &10 &1 mois \\ \hline

5 (Rare) &3 jours &20 &3 mois \\ \hline

\end{tabular} \end{table} 



\subsubsection{Acquérir/Fourguer des Biens} 

Les réseaux sociaux sont un bon moyen de trouver des objets que vous ne pouvez légalement acheter ou fabriquer chez vous. En fonction de la personne auprès de qui vous aller obtenir les biens, cela vous coûtera des crédits ou nécessiter un socre de Rep approprié. Cette faveur peut aussi être utilisée pour vendre ou donner de tels objets, en récupérant quelques crédits ou même un peu de Rep dans l'opération. 

\begin{table} \caption{Acquérir/Fourguer des Biens} \begin{tabular}{|I|I|} \hline

NIVEAU &SERVICE \\ \hline

1 &Acquérir/fourguer un objet ayant un prix Trivial. \\ \hline

2 &Acquérir/fourguer un objet ayant un prix Bas. \\ \hline

3 &Acquérir/fourguer un objet ayant un prix Modéré. \\ \hline

4 &Acquérir/fourguer un objet ayant un prix Élevé. \\ \hline

5 &Acquérir/fourguer un objet ayant un prix Cher. \\ \hline

\end{tabular} \end{table} 



\subsubsection{Obtenir des Services} 

Lorsque vous manquez des compétences ou des connaissances nécessaire, ou que vous avez juste besoin d'une paire de bras supplémentaire, vous pouvez faire appel à votre réseau socail pour trouver quelqu'un pour vous aider. Si vous recherchez quelqu'un ayant une compétence particulière, le résultat de votre Test de Réseau réussi est le niveau de compétence de la personne que vous trouvez. Plus votre compétence Réseau est élevée, plus vous êtes capable de trouver des professionel hautement-compétent. 

\begin{quotation} Cole doit trouver un astobilogiste qui peut l'aider à identifier une menace étrangère. Il lance sa compétence Réseau: Scientifique de 50 et obteint 43 - une réussite. Il trouve quelqu'un avec la compétence Académique: Astrobioloie de 43 (son résultat) qui est prêt à l'aider. Lorsque l'astrobiologiste examine la créature, le maîþre de jeu fait un jet pour le PNJ en utilisantr cette compétence de 43. \end{quotation} 



\subsubsection{Obtenir de l'Information} 

Lorsque vous devez trouver de l'information en ligne ou que vous n'avez ni le temps ni la capacité à le faire, vous pouvez vous tourner vers les personnes dans votre réseau social et faire appel à leur base de connaissance accumulée. 

\subsection{Identité et Réputation} 

Il est important de noter que la réputation est intimement liée à l'identité. Si vous êtes sous couverture en utilisant une fausse identité, vous ne pouvez pas réellement faire appel à votres score de Rep sans faire sauter votre couverture. En conséquence, beaucoup de personnes utilisant de fausses identités finissent par construire des jeux de Rep spécifique à leurs alter ego. Comme de nombreuses interactions sociale se passent en ligne, il est possible que quelqu'un utilise secrètement son identité réelle tout en se faisant passer pour quelqu'un d'autre, tant qu'ils sont attentifs à ce qu'ils font. Si par hasard quelqu'un espionne son activité par le mesh, il y a une chance que le mensonge soit mis à jour. 



\begin{table} \caption{Obtenir des Services} \begin{tabularx}{\textwidth}{|l|X|} \hline

NIVEAU &SERVICE \\ \hline

1 &\textbf{Faveur Triviale}: Demander à quelqu'un de rendre un service pendant 15 minutes. Bouger une chaise. Intimider quelqu'un. Trouver un véhciule. Rehcercher quelqu'un en ligne. Emprunter 50 crédits. D'autres services de prix Trivial. \\ \hline

2 &\textbf{Faveur Mineure}: Demander à quelqu'un de rendre un service pendant une heure. Déménager. Tabasser quelqu'un. Louer un véhciule. Obtenir un alibi. Location de cuve de soin. Aide pour un piratage mineur. Aide légale ou policière basique. Emprunter 250 crédits. D'autres services de prix Bas. \\ \hline

3 &\textbf{Faveur Modérée}: Demander à quelqu'un de rendre un service pendant une journée. Déménager dans un habitat du même groupe. Tabasser violemment quelqu'un. Surveillance. Egocast court-courrier. Petit voyage en navette (moins de 50 000 km). Psychochirurgie mineure. Upload. Réservation dans le meilleur restaurant du monde. Faveur policière ou représentation légale importante. Emprunter 1 000 crédits. D'autres services de prix Modéré. \\ \hline

4 &\textbf{Faveur Importante}: Demander à quelqu'un de rendre un service pendant un mois. Déplacer un corps. Meurtre. Navette sans pilote. Sabotage industriel. Contart d'expédition de gros-volume sur un cargo. Egocast moyen-courrier. Voyage en navette sur une distance moyenne (50 000 - 150 000 km). Psychochirurgie modérée. Réincarnation. Carte "sortie de prison". Emprunter 5 000 crédits. D'autres services de prix Élevé. \\ \hline

5 &\textbf{Partenariat}: Demander à quelqu'un de rendre un service pendant un an. Déplacerf un corps démembré. Meurtre de masse. Détournement massif de fond. Acte de terrorisme. Relocaliser un astéroïde de taille moyenne. Egocast long-courrier. Voyage en navette sur une longue distance (150 000 km et plus). Emprunter 20 000 crédits. D'autres services de prix Cher. \\ \hline

\end{tabularx} \end{table} 





\begin{table} \caption{Obtenir des Services} \begin{tabularx}{\textwidth}{|l|X|} \hline

NIVEAU &SERVICE \\ \hline

1 &\textbf{Information courante}: Où manger. Dans quelles affaires traine telle hyeprcorporation. Qui est le chef. \\ \hline

2 &Information publique: Obtenir des connexions sur le marché gris. Où est le "mauvais voisinnage". Des informations issues de bases de données publique obscure. Qui est le syndicat du crime local. Information hypercorporatistes publique. \\ \hline

3 &\textbf{Information privée}: Obtenir des connexions sur le marché noir. Où se touve une installation hypercoporatiste non-officielle. Qui est un flic. QUi est membre d'un syndicat du crime. Où traîne quelqu'un. Information hypercorporatistes interne. Qui couche avec qui. \\ \hline

4 &\textbf{Information secrète}: Obtenir des connexions exotiques sur le marché noir. Où se touve une installation hypercoporatiste secrète. Où se cache quelqu'un. Projets hypercorporatistes secret. Qui trompe qui. \\ \hline

5 &\textbf{Information Top Secrète}: Où se situe un laboratoire top-secret financé par une caisse noire. Projets hypercorporatistes illégaux. Données scandaleuses. Matériel de chantage. \\ \hline

\end{tabularx} \end{table} 







\section{Sécurité} 

Les snetinelles de Firewall ont pour habitude d'être dans des endroits où ils ne sont pas censés être et d'emmener avec eux des choses que d'autres auraient préféré ne pas avoir. La sécurité post-Chute à un caractère bien déifférent de celle du 21° siècle. En raison de la sur-abondance, les mesures de sécurité physique telles que les verrous, les portes et les murs sont bien moins important que dans le passé. Le s personnes ne se préoccupent plus autant du vol que dans le passé car la pluaprt des objets peuvent être remplacés par un nanofabeur. Les objets qui ont tendance à engendrer ce type de sécurité sont cex qui sont irremplaçables ou rares, telles que les artefacts de la Terre. 

La sécurité physique post-Chute se concentre fortement sur la surveillance - indentifier les intrus et les pister pour qu'ils puissent être arrété par des défenseurs transhumains ou robotiques. La surveillance est bien plus efficace que dans les sociétés pré-Chute car les IA avec des facultés presque humaines de reconnaissance de motifs et des infomorphs contractés peuvent être embauchées pour superviser les données de surveillance. 

L'emphase sur la surveillance a été renforcée par la facilité avec laquelle la plupart des barrières matérielles on pu être abattues par des armes personnelles à haut-epusisance et des appareils comme les outils d'opération clandestine (p. 315). Cependant, les barrières physiques activement conçue pour résister aux intrus en s'auto-réparant ou en attaquant les outils utilsiés pour les endommager sont utilisées aux endroits clefs des installations sécurisées. De telles barrières sont généralement extrêmement chère et ne sont donc que rarement utilisées. 

Les défenseurs transhumains, animaux et informatiques sont la clef de voutes de la plupart des systèmes de sécurité. La disponibilité d'une large réserve de travailleur infomorph pour garder les installations signifie qu'il y a toujours quelqu'un de service, soit en tant qu'élément du système de surveillance, soit dans une coque robotique. 

\subsection{Contrôle d'Accès} 

La première étape dans tous les systèmes de sécurité est de pouvoir maintenir les indésriables à l'extérieur. À un niveau basique, cela implique des murs, des verrous, des clotûres, l'aménagement de l'espace, de l'éclairage de sécurité et des avertissements entoptiques. 

Des barrières de différents type présentent un obstacle qui doit être percé ou explosé afin de pouvoir être franchie. les barrières  sont considérées comem tous les autres objets inanimés dans le cadre des attaques et des dégâts; voir \textit{Objets et Structures,}\textit{} p. 202. 

\subsubsection{Bug Zappers} 

Les bugs zappers créent de petites EMP qui sotn sans danger pour la plupart des équipements électronique et les implants mais qui sème la destruction dans les nuées de nanites, les microbugs et les poussières. Les bug zappers sont géénralement appliquées sur des surfaces et, en tant que tels, ne font que détruire les essaims flottant/volant ou les poussières si ils se posent. Dans les zones où l'électronique est lourdement blindées, ils peuvent être installés pour détruire toute les cibles dans une pièce complète. Un zapper détruit instantanément tout les nanobots et poussière volant ou rampant librement dans une pièce lorsqu'il s'éteint, mais la chair transhumaine fournit sudffisament de protection pour l'empêcher de détruire les méchidaments ou les autres nanobots implantés. Les infiltrateurs essayant d'accéder à des zones protégées par des zappers se reposent généralement  sur leur contournement ou sur l'instalaltion d'appareil macroscopique. 

\subsubsection{Verrous Électroniques} 

Les verrous électroniques sont courament utilisés comme un moyen de conserver un peu d'intimité. Ils sont cependant facile à casser et ne sont donc que rarement utilisés dans les zones trés sécurisées. Les verrous électroniques ont différents avantages sur les bon vieux verrous mécanique. Différntes utilisateurs peuvent avoir différentes méthodes d'identificfation, ils peuvent enregistrer tout type d'évènements (entrés, sortie, échec d'identification) et ils peuvent être connectés (généralement pas un lien filaire mais parfois par une connexion chiffrée sans-fil) à des systèmes de sécurité pour le contrôle distant et le déclenchement d'alarme. 

Les verrous électronique utilisent différents système d'identification, ou une combinaison de ces systèmes: 

\textbf{Biométrique:} Le verouu analyse une ou plusieurs empreinte biométrique de l'utilisateur. Les mesures biométriques classiques incluent les empreintes ADN, de thermographie faciale, digitales, de démarche, des veines de la main, de l'iris, de la frappe clavier, d'odeur, palmaire et vocale. 

\textbf{Clavier:} Il s'agît d'un clavier alphanumérique sur lequel els utilisateurs doivent saisir un code spécifique. Différents utilisateurs peuvent avoir différents codes. 

\textbf{Jeton:} Les utilisateurs autorisés peuvent porter un type de jeton physique qui interagit avec le verrou pour ouvrir la porte, tel qu'une carte magéntique, une clef électronique, etc. 

\textbf{Code Sans fil:} Les utilisateurs peuvent émettre un code cryptographique par un signal raio à courte portée. 

Bien que différentes technologies existent pour défaire chacun de ces systèmes, il y a trois méthodes qui marchent contre presque tous les verrous électronique. La première est l'utilisation d'outil d'opération clandestines (p. 315), qui infiltre un verrou avec des nanomachines qui essaime et attaque le mécanisme électronique. L'inconvéinent de l'utilisation des OOC et que l'usage est immédaitement enregistré par le verrou et que l'alarme est déclenchée. Certains verrous léectroniques sont même équipés d'essaim de nanites guardien (p. 329) pour contrer les OOC, mais les nanomachines des OOC s'arrangent généralement pour ouvrir le verrou avec que les gaurdiens ne les mange. 

La deuxième méthode est de pirater le verrou électronique. La plupart d'entre eux sont asservis à un système de sécurtié, ce qui implique qu'il faut d'abord s'introduire dans le système de sécurité puis ouvrir le verrou de l'intérieur. Cela peut être difficile, en particulier si le système de sécurité est isolé du réseau sans-fil ou s'il est filaire. L'avantage est que, si 'est fait proprement, toutes les preuves de l'ouverture du verrou peuvent être effacées. La troisième méthode est d'ouvrir et de manipuler le verrou physiquement. Cela nécessite d'ouvrir le boîtier du verrou puis de déclencher le mécanisme du verrou pour ouvrir la porte. Ces deux actions sont gérées par des Action de Tâches de Matériel: Électronique avec un intervalle d'1 minute chacune. De plus, la pluaprt des verrous électroniques ont des circuits anti-piratage qui déclencheront une alarme si l'attaquant n'obtiens pas une Réussite Exceptionnelle en ouvrant le boîtier. 





\subsubsection{Autoverrou} 

Le 21° siècle à vu une transition des verrous mécanique vers les verrous électroniques et d'autres méchinsmes de verrouillage électronique. Ces appareils ont bien fonctionns apendant 50 ans, jusqu'à ce que les capacités d'infiltratiosn électronique les redent largement inutiles. Le développement récent des autoverrous a bien plus en commun avec leurs cousins éloignés mécanique que leurf frêres électornique. Ce sont des objets d'artisanat unique et cher. 

Un autoverrou typique est largement intégré avec le portail et la barrière qu'il protège. Les autoverrous incluent géénralement une IA ou une infomorph contracté, des matériaux d'auto-réparation (traiter les comme des barrière auto-réparante) et une nuée de nanomachines guardienne (p. 329). Un autoverrou surveille sont environnement et possède des logiciels de reconnaissance qui sait à quoi ressemblent ses utilisateurs et ses clefs (compétence de Perception à 40). Corcheter un autoverrou est donc incroyablement difficile, car il fermera ses orifices et n'acceptera aucune clef qui n'as pas l'air correcte ou venant d'un utilisateur inconnu. Les nanomachines étrangères essayant de pénétrer l'orifice seront ciblées et détruites par les nanomachines guardiennes. Enfin, les outils extérieurs utilisés pour attaquer le portail ou le verrou seront également attaqué par des appendices fractals formés sur la surface du portail ou par le verrou lui-même. Ces appendices ont une portée d'1 mètre, attaquent avec une compétence de 40 et infligent 1d10 + 2 de VD. 

Les autobots sont généralement immunisés au piratage car, pour des raisons de sécruité, ils ne sont pas connecté au mesh. Si ils sont attaqués, les autoverrous sont cependant programmés pour envoyer un signal d'alarme par le mesh. 

Il y a plusieurs manière de battre un autoverrou. L'une est d'obtenir une copie ou une image de la clef puis d'en forger une copie (en utilisant la nanofabrication). Une autre est d'attaquer l'autoverrou ou le poratil qu'il protège avec suffisament de force pour que l'autoverrou soit incapable de se réparer (généralement en utilisant des armes à distances, tout ce qui se trouve à un mètre de l'autoverrou pouvant être contre attaqué). Un troisième est de prendre une image des cavités iunterne de l'orifice de l'autoverrou sans que l'appareil d'imageire ne soit détruit puis de forger une clef. Toutes ces méthodes sont extrêmement difficile et consomatrice de temps. 

Certains autoverrou ont la capacité à détruire ce qu'il protège. Par exemple, les autoverrous sont une forme de rpotection commune pour les interfaces physique des réseaux filaires. Si l'autoverrou est compromis, il peut, en dernier recours, détruire l'interface qu'il protégeait. 

\subsubsection{Système de Déni de Portail} 

Installé dans les couloir ou les vestibules, il s'agît grosso modo d'un piège laser. Lorsqu'une personen non autorisée entre dans la zone du système de déni de portail, le système utilise des lasers pour créer une grille de plasma qui sont utilisés pour infliger une grosse décharge électrique à la cible. Ce système possède à la fois des réglages létaux et non-létaux. 

\textbf{Non-létal:} 1d10 VD + étourdit (p. 204) 

\textbf{Létal:} 2d10 + 5 VD 

\subsubsection{Barrière Auto-Réparatrice} 

Les murs et les portes qui sont capables de se réparer rapidement sont parfois trouvé dans des installations de haute sécurité. Ces barrières sont faites de amtériau qui s'étendent automiquement pour 'réparer' de petits trous et sont équipés de nanosystèmes qui réparent lentement des dégats plus importants. Les meilleures de ces barrières ne font pas que ralentir l'assaillant le plus déterminé, mais combinées à des systèmes de surveillance, elles deviennent des nuisances pour les envahisseurs et peuvent ralentir les tentatives de s'échapepr des lieux. 

Les barrières auto-réparatriuces soignent une seule zone de dégâts ayant subit moins de 5 points de dégats presqu'instantanément, rebouchant le trou en 1 Tour d'Action. Elles reboucheront aussi les trous créé par des outils d'opérations clandestine (p. 315) dans le même intervalle de temps. De plus, ces barrières se reparent elle-même au rythme de 1d10 points de dégats toutes les 2 heures; une fois que tous les dégats ont été réparés, les blessures seront réparées au rythme de 1 par jour. Les dégâts ayant infligés 3 blessures ou plus ne peuvent être auto-réparées. 

\subsubsection{Murs Glissants} 

A la surface des planètes, des murs et des clotûres élevées sont relativement fréquente en tant que première ligne de défense contre les trus. Les murs glissants sont des murs traités avec un produit chimique glissant (p. 323), créant une surface virtuellement sans-frictiopn et qui est exceptionnellement difficile à escalader. 

\subsubsection{Inhibiteurs Sans Fil} 

Les inhibiteurs sans fil sont de simples peintures ou matériau de constructions qui bloquent les ondes radios. Ils sont utilisés pour cérer une zone fermée à l'intérieur de laquelle un réseau san-fil peut fonctionner librement sans s'inquiéter du fait que le signal sorte du lieu et puisse être intercepté. Les inhibiteurs snas fil permettent d'utiliser les liens sans-fil trés pratique à l'intérieur d'une zone sécurissé plutôt que de passer par les connexion filaires contraignantes. Si un intrus parvient à accéder à la zone interne ils peuvent toujours intercepter, écouter et pirater les appareils sans fils comme d'habitude. 

\subsection{Détection Et Surveillance} 

Si le mesures de sécurités ne parviennent pas à maintenir un inytrus à l'extérieur, l'étape suivante et de le détecter et de suivre son activité. 

\subsubsection{Nanomarquage} 

De nombreux système de sécurité post-Chute ne se concentrent pas sur le fait de maintenir les personnes hors des zones privatives, mais préfère les pister après qu'ils soient rentrés. Les transhumains chérissent le peu d'intimité qu'il leur reste. L'effraction est souvent considéré comme bien pire que le vol dans de nombreux endroits. 

Une salle protégée par un essaim de naniotes marqueurs (p. 329) possède générallement deux ruches ou plus, une au niveau du sol, l'autre au niveau du plafonf (en cas de gravité; sur des faces opposées de la pièce en microgravité) qui génèrent et recyclent des nanomachines. Les marqueurs émergent d'une ruche, flottent à travers la pièces puis retournenr à l'autre pour  se recharger et être réutilisés. Une boucle de retour connectent généralement les 

ruches pour qu'elles puissent paratger ressource et énergie. 

Quiconque passe à travers la pièce sera trés probablement aspergé de nanomachines marquante. Une fois qu'ils sont éloignés du reste de la ruche, ils se cachent et envoie périodiquement des impulsions radio ayant pour but de donner leur position aux porusuivant ou enquéteurs. Certains peuvent tomber en grappe pour former un fil d'arianne permettant de pister l'intrus. 

\subsubsection{Capteurs} 

Toutes les variations de capteurs décrites dans le chapitre \textit{Équipement} (p. 294) peuvent être déployés dans une installation pour superviser et enregistrer le passage du personnel, autorisé ou non. Ces capteurs sont typiquement asservis au réseau de sécurité de l'instalaltion et supervisé attentivement par les IA de sécurité, car ils sont vulénrable au piratage et à l'interception. Quelques autres type de capteurs méritent d'être mentionnés ici: 

\textbf{Renifleurs Chimique:} Le renifleur décrit p. 311 peut également être configuré pour détecter le dioxyde de carbone exhalé par les transhumains. Cet apapreil est utile pour détecter les biomorphs s'introduisant dans des zones abandonnées/hors-limite. 

\textbf{Capteurs Électriques:} Les capteurs électriques peuvent être installés dans les portails pour détecter le champ électromagnétique d'une biomorph ainsi que les champs électriques des synthmorphs. 

\textbf{Capeturs Cardio:} Ces capteurs sensibles détectent les vibrations causées par les battement d'un cœur transhumain. Ils peuvent même être utilisés pour détecter les battements cardiaques des passagers d'un véhicule. 

\textbf{Capteurs Sismiques:} Embarqués dans les sols, ces capeturs détectent la pression et la vibration d'un poid en mouvement. 

\subsubsection{Scanners d'Armes} 

les scanner d'arme existent sous différentes formes, incluant ceux qui cherchent des éléments rares utilisé dans les armes extrêmement destructives telles qu les têtes nucléaire, ceux qui essayent de localiser les armes personnelles et ceux qui recherche des marqueurs de détection. 

Les détecteurs d'éléments rares sont quasiment infaillibles et omniprésents dans les spatiports et les douanes des habitats. La seule façon de les contourner est de trouver une route alternative vers la zone protégée. 

Les scanners d'armes personnel peuvent superviser une zone spécifique, telles qu'une petite pièce ou un vestibule. Ils peuvent utiliser utiliser beaucoup de système de détection pour trouver et identifier les armes et autres objest dangereux, incluant les renifleurs chimiques et les système d'imageire radar/micrométrique/infrarouge/à rayon-x/ultrasonique. Ils peuvent détecter les objets et substances suivantes: 

\begin{itemize} \item Les métaux utilsiés dans les armes cinétiques, les armes chercheurs et les armes à fléchettes \item Les appareils ayant des ruches de nanobots métalliques (par, ex: les outils d'opération clandestines, les chargeurs) \item Les éléments mégnatiques des fusils à plasma et des armes à rails \item Le carburant des munitions d'armes à feu et d'armes à tête chercheuse (–30 pour les dissimuler) \item Les produist chimiques utilisé dans les torches des armes à spray (–30 pour les dissimuler) \item Tous les explosifs et les grenaces grâce à leur émissions de aprticules chimique (–30 à dissimuler) \item Les poisons et agent biologiques dans les arems à fléchettes \item Toute arme ou appareil plus grand que la main (en utilisant des ondes sonore et de la détection de forme). \end{itemize} 

Les personnages essayant de faire passer des armes et du matériel à travers des scanners d'amres doivent faire un Test de Manipulation (si ils essayent de dissimuler l'objet) ou un Test d'infiltration (si ils essayent de manœuvrer sans se faire remarquer). Cela est opposé à un Test de Perception fait par le personnage ou l'IA manipulant le systèlme de capteur. 

\subsubsection{Analyse Sans-fil} 

Certaines ones de haute-sécurité peuvent superviser intentionnellement les réseaux sans-fil qui ont pour origine leur zone comem moyen de détecter els intrus grâce à leurs émmissions de communications. Ces signaux peuvent même être utilisés pour pister la localisation de l'intrus via la triangulation et d'autres moyens (voir \textit{Pistage Physique,} p. 251). Pour contourner les système de détection sans-fil, les agents clandestins peuvent utiliser des liens lasers (p. 313) ou des interface épidermiques (p. 309) pour communiquer. 

\subsection{Contremesures Actives} 

Lorsque tout le reste à échoué, des contremesures actives peuvent être déployées contre les intrus. Même si des gardes transhumains sont quelquefois utilisé, les sentinelles robotiques sont bien plus communes, généralement sous la forme de synthmorphs comme les synths, les slitheroïdes, les arachnoïdes ou les reapers pilotées par des IA avec des anges guardiens (p. 346) pour fournir le support aérien. De temps en temps, des emplacement d'artillerie manœuvrés par des IA - des tourelles blindés qui sortent des murs et des plafonds - sont également utilisés. Dans certaines circonstances, ces coques sont manipulées à distance ou même intercepté par la sécurtié transhumaine. 

Les contremsueres additionnelles mises en places dépendront des installations en question. Certains sites se livreront à de l'interception active, pour bloquer les communications des intrus. D'autres déploierons des nuées de nanomachines et même des armes chimiques. 