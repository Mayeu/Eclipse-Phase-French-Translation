



\chapter{Futur Accéléré} \label{cha:accelerated-future} 



















Le cadre futuriste de \textit{Eclipse Phase} introduit un grand nombre d'éléments technologiques qui ont un impact élevé sur la société transhumaine. Cela inclut entre autres la sauveagrde et l'upload, la réincarnation, l'égocast, le frok, la nano-fabrication, les systèmes de réputations, les habitats spataiux et le voyage spatial. 

\section{Suavegarde et Upload} 

L'esprit Transhumain n'est plus prisonnier du matériel biologique duquel il est originaire. Grâce à différents mécanismes, les cerveaux bilogiques peuvent être émulés numériquement, permettant aux personnes de faire une \textit{sauvegarde} de leur esprit, incluant toute leur personnalité, leurs souvenirs et compétence - un procédé appelé \textit{upload.} 

L'utilisation principale des sauvegarde et de garantir que l'ego de la personne pourra être récupéré en cas de mort, ce qui dans ce cas leur pemettra de se \textit{réincarner} (p. 271). Pour cette raison, presque tout le monde dans le système solaire est équipé d'une pile corticale (p. 300). Les sauvegarde peuvent également être archivé dans des stockages sécurisés (p. 269) ou utilisés pour créer des infomorphs (p. 264). Une personne peut aussi s'egocast à travers le systèmle solaire comem moyen de voyager (p. 276). 

\subsection{Sauvegardes par Piles Corticale} 

Les implants de piles corticales déploient un réseau de nanobots dans le cerveau qui rpend un paerçu de l'état neural de l'esprit, sauvegardant les données à l'intérieur de la pile corticale. La pile corticale transhumaine moyenne sauvegarde 86,400 fois l'ego par jour. Seul la sauvegarde la plus récente est préservée dans la pile; les plus anciens sont écrasés. Les pods et les synthmorphs peuvent aussi être équipés avec des piles corticales (bien que les bots pilotés par des IA n'ont géénralement pas cette caractériqtique), bien que ces versions maintienent une copie à jour de l'ego fonctionnant das le cybercerveau de la morph. 

En cas de mort, accidentelle ou non, une pile corticale peut-être récupérée sur un cadavre et être utilisée pour restaurer le personnage, soit à l'état d'infomorph, soit en le réincarnant dans une nouvelle morph. Les piles corticales sont blindées au diament, et peuvent donc être souvent récupérées, même si le corps est gravement attaqué ou endommagé. Si le cadavre ne peut pas être récupéré ou si la pile corticale est détruite, la sauvegarde est perdue. 

Les risques-tout, les bordés sur-équippés et d'aures ayant une profession dangereuse optent souvent pour un farcaster d'urgence en accessoire (p. 306) qui transmet périodiquement (habituellement toutes les 48 heures, mais cela peut varier en fonction du contrat) une sauveagrde de la pile corticale vers une installation de stockage distante. Cette option reste relativement chère et n'est abordabel que par les plus riches. 

\subsubsection{Récupérer Une pile Corticale} 

La plupart des piles corticales sont minutieusement excisées du cadavre par chirugie. Dans certaines circonstances un personnage peut avoir besoin d'extraire une pile corticale sur le terrain, que cela soit parceque transporter le cadavre n'est pas pratique ou parceque le mort est un  ennemis et que le personnage ne veut pas qu'il sache qui l'a tué ou qu'il veuillent l'interroger avec de la psychochirurgie en simulspace. 

La procédure pour extraire la pile corticale est appelée "poper," car un extracteur compétent peut habituellement faire sortir la coque lisse de l'implant en faisant une incision au bon endroit et en y appliquant une pression. Il faut faire attention à ce que la petite pile dégoulinante de sang ne glisse pas au loin lorsqu'elle est popée. 

On peut poper une pile avec un couteau aiguisé et de l'huile de coude, bien que cela soit sanglant. Poper une pile est une Action de Tâche qui nécessite un Test de médecine: [n'importe quel domaine utile] avec un Intervalle d'1 minute et un modificateur de +20. Les morphs avec une pile à un endorit non-standard ou avec un renforcement anatomique (plaques de carapaces, etc.) autour de la pile peuvent infliger des pénalités à ce test, à la discrétion du maître de jeu. Bien entendu, si vous n'avez pas le temps pour une extraction précise, vous pouvez toujouts enlever toute la tête et l'emmener avec vous. 

Une fois que la pile corticale est récupérée, elel peut-être chargée dans un connecteur d'ego (p. 328) et utilisée pour ramener l'ego, en tant qu'infomorph ou par la réincarnation. 

\textbf{Sujets Vivants:} Les piles corticales peuvent être excisée d'une personne vivante, mais le procédé est généralement fatal (ou au moins apralysant) car il implique de sectionner la colonne vertébrale. Si la cible n'est pas inconsciente ou incapacitée d'une manière ou d'une autre, elel doit d'abord être immobilisée en coimbat au corps à corps (voir \textit{Contrôle, } p. 204). Extraire la pile est considéré comme une Action de Tâche comme ci-dessus, mais le procédé inflige 3d10 + 10 points de dégats à la cible. Si le test échoue, il inflige tout de même 1d10 + 10 dégat à la cible. Si la personne supprimant la pile veut laisser la cible vivante ou la blesser le moins possible, elle subit un modificateur de -20 à son test, mais réduit les dommage d'1d10 par tranche de 10 points de MdR. Subir l'opération d'extraction de votre pile est traumatisant; quiconque subissant cette opération souffre d'1d10 points de stress mental. 

\subsubsection{Détruire Une Pile Corticale} 

Les piles corticales ont une Armure de 20 et une Solidité de 20 pour quiconque essaye de les détruire. 

\subsection{Upload} 

Uploader une sauveagrde dans un stockage sécuréis est générallement effectué par une analyse cérébrale à la clinique de l'installation de stockage en utilisant une unité grosse comme un grille pain et appelée un \textit{connecteur d'ego} (p. 328). Lorsqu'il est activé, la matrice de capteur du connecteur d'ego s'ouvre comme un tournesol, révélant une boîte équipée d'un appui-tête qui s'ajuste automatiquement à la tête des morphs, même de celles les plus étrange. L'appui tête déploie des millions de nanobots spécialisés 

dans le cerveau et le système nerveux central. Les pétales sont remplis de capteurs qui prennent des images du cevreau en utilisant une combinaison d'IRM, de sonogramme et de la diffusion d'information positionelle faites par les nanobots dans le cerveau de la morph. Le connecteur d'ego construit ensuite une copie du cerveau de la personne, qui est ensuite stockée dans les entrepœts de données filaire, hautement sécurisé et déconnectés du mesh du service. 

Dans le cas des pods, le connecteur d'ego analyse les parts du cerveau biologique et accède également au cybercerveau pour copier les parties de l'ego qui y résident. Pour les synthmorphs, qui n'ont pas de cerveau bilogique, le procédé est bien plus simple puisqu'il nécessite de simplement accéder au cybercerveau et de le copier. 

Dans une clinique standard et avec une morph non abîmée, l'upload ne prend que 10 minutes, 5 avec un pod. Dans d'autres situations, l'opération peut durer plus longtemps si le maître de jeu le désire. L'upload depuis une synthmorph ou une pile corticale extraite est instantané. Le connecteur d'ego opère essnetiellement de lui-même. Bien qu'une supervision par un spécialiste médical est une bonne idée, aucun test n'est nécessaire. 

Si un personnage uploadé ne prévoit pas de retourner dans sa morph, elle est générallement cryogénisée jusqu'à ce que quelqu'un s'y réincarne. Si une nouvelle incarnation n'est pas disponible et que le personnage s'uploadant ne veut pas laisser une copie potentielle de lui-même derrière lui, il peut complètement effacer l'esprit de la morph par les nanobots lors de l'opération. 

\subsubsection{Continuité Upload-Réincarnation} 

Dans des circonstances idéales, une personne qui se réincarne intentionnellement (p. 271) peut s'arranger pour que les opérations d'upload et de réincarnation se déroulent sans perte de continuité notable. Bien que l'expérience de basculer d'une morph à une autre est toujours un peu perturbant, la transition en elle-même peut-être faite lors d'une opération propre, sans faille de conscience ou de souvenir, ce qui aide à réduire le stress mental associé. 

Dans ce cas, pendant le processuss d'upload, le connecteur d'ego est également connecté à un autre connecteur et à la nouvelle incarnation. Cette connexion peut même être sans-fil ou par un farcaster (aevc une distance maximale de 10,000 kilomètres). 

Alros que l'esprit est uploadé, le connecteur d'ego construit un cerveau virtuel en copiant le cerveau de la morph bit par bit, utilisant les données récupérées de l'analyse du cerveau. En même temps, ces données sont lentement copiées dans la nouvelle incarnation laors que les nanobots recâblent la strructure cognitive de la morph (une opération bien plus lente). Pendant le trasnfert, les nanobots dans le cerveau coupent les connexions neurale et les reroutes vers leur double dans le cerveau virtuel, pusi dans le nouveau cerveau. En pratique, l'efo du personnage tourne partiellement sur le c erveau de viande et partiellement sur la copie virtuelle de celui-ci. Au moment où les nanobots coupent la dernière connexion neurale dans le vieux cerveau, l'ego fonctionne complètement sur le cerveau virtuele t sur celui de la nouvelle incarnation. une fois que la réincarnation est complète, le cerveau virtuel est éteint. 

En terme de perception, le personnage, qui est éveillé pendant le procédé, perçoit un glissement trés graduel d'une morph vers l'autre. COmme le procédé peut prendre des heures (voire plus si l'opération se fait via un farcasteur), le sujet se divertit générallement avec des média RA, RV ou même en XP pour passer le temps. 

\subsubsection{Upload Après la Mort} 

Il est possible d'uploader l'esprit d'une personne qui est récemment décédée tant que les nanobots ont le temps de scanner le cerveau avant que la déterioration cellullaire n'intevienne de manière trop devastatrice, générallement au bout de 2 heures. Il est possible de maintenir un cadavre pour une durée plus longue en le plaçant dans une cuve de soin (p. 326) pour une nanostase. L'upload post-mort peut subir des pertes d'intégrité; voir \textit{Complication de Sauvegarde}\textit{, } p. 270. Les cybercerveaux peuvent aussi être récupérés depuis une synthmorph détruite et réactivée, du moment qu'elle n'a pas été trop lourdement endommagée (à la discrétion du maître de jeu). 

\subsubsection{Upload Destructif} 

Bien que cela soit rare, des personnes se lancent dans un procédé appelé upload destrcutif dans lequel le cerveau bilogique est littéralement découpé en tranche et analysé morceau par morceau. Considérer comme exécrable et dispendieux par la plupart des transhumains, le décorticage de cerveau est pratiqué par certaines factions bioconservative qui la considère comem la seule méthode "pure" d'upload ou comme la seule réelle façon de transférer l'âme." De telles personnes refusent typiquement de se réincarner, vivant le reste de leur vie en tant qu'infomorph, généralement dans des simulspaces déidées et considérés comme une sorte d'au-delà virtuel. 

\subsection{Assurance Sauvegarde} 

Presque tout le monde, à l'exception des néo-primitivistes et des trés jeune enfants, a une pile corticale. En cas de mort, une pile corticale seule ne permete cependant pas d'assurer la résurection, sauf si vous avez une assurance sauvegarde (p. 330) pour couvrir les frais de votre réincarantion. Vivre sans assurance sauvegarde pour n'importe quelle durée représente un risque énorme. Certaines juridictions (comme le Commonwealth Titanien) ont pour pratique de ramener tout le monde, même si c'est juste à l'état d'une infomorph, ou d'au moins enregistrer la dernière sauvegarde en mémoire morte, au cas où quelqu'un décide de payer pour le resusciter plus tard. D'autres autorités détruirons simplement la pile ou, pire, la revendront sur le marché noir à un syndicat trafficant les âmes tel que Nine Lives. Les assurances sauvegarde incluent typiquement un abonnement à une instalaltiond d'upload, nécessitant généralement une visite tous les 6 mois, pour s'assurer que cette sauvegarde est conservée dans un stockage sécurisé en cas de perte de la pile corticale. Les personnes avec des métiers à risque (superviseur de bots de construction, personnel hypercorporatiste sur des exoplanète, fille combattant des anguilles génate et vicieuses pour un public riche et blasé, etc.) peuvent se sauvegarder une fois par semaine, voire même quotidiennement. En cas de mort vérifiée et si la pile corticale ne peut pas être récupérée, la sauvegarde la plus récente est utilisée pour réincarner la personne. Au niveau le plus basique, l'assurance sauvegarde ramènera le personnage en infomorph, point à partir duquel ils puevent accéder à leur crédit et acheter une nouvelle morph. Des versions plus chère permettront de vous réincarner automatiquement 

dans une morph pré-acheté de votre choix. Les personnes excessivement riches auront souvent des clônes personnalisés (souvent de leur corps originel) attendant dans la glace. 

Les assurances sauvegardes impliquent souvent une clause de personne manquante, qui établit qu'une personne sera ramené si elle n'a pas donné signe de vie pendant une période X (une fonction calendaire gérée automatiquement par la muse) et qu'elle ne peut être localisée. 

A noter que certains syndicats du crime proposent également des assurances sauvegarde à des prix réduit. la probabilité qu'une copie de votre sauvegarde soit utilisées dans un but illicite est cependant relativement élevée. Cependant, certaines personnes ne se préoccupent pas de savoir ce qui peut arriver à une copie d'eux-même. 

\subsubsection{Limite des Assurances Sauvegarde} 

Les assurances sauvegardes ne sont pas toujours parfaites. Bien que les assureurs sont obligés de faire un effort raisonnable pour récupérer votre pile corticale, pour beaucoup d'hypercorp c'est simplement une analyse coût/bénéfice qui ne fonctionnera que rarement en faveur du personnage. Si vous êtes mort dans une zone dangereuse telle que la Zone sur Mars, dasn une zone éloignée telle que la Ceinture de Kuiper, ou vous êtes simplement difficile à pister (pousser hors d'un sas quelque part), la probabilité que votre pile corticale soit récupérée est très faible - vous serez plutôt réinstantié à partir d'une sauvegarde. 

La juridiction peut également avoir un rôle important. l'assurance offerte par de nombreux assureurs du système intérieur sont automatiquement annulées si vous voyagez sur un habitat anrchiste, que vous resquilliez, violiez la loi ou vous lanciez dans certaines activités pouvant menacer votre vie comme le suicide sprotif ou la récupération dans les ruines infectés par les TITAN. Ils refuseront au minimum de récupérer votre pile dans ces circonstances. De manière similaire, si vous négociez une assurance sauvegarde avec un collectif médical d'un habitat autonomiste et allez ensuite mourir sur une stationhypercorporatiste, l'hypercorporation refusera trés probablement de reconnaître l'autorité d'une bande d'anarchiste et ils ne leur renverrons pas votre pile. 

Même une sauvegarde archivée et une clause de personne disparue ne sont pas des garanties. Un ennemi déterminé pourrait vous capturer, récupérer les codes d'accès de l'assurance sauvegarde auprès de votre muse, vous cryogéniser ou vous tuer discrètement puis régulièrement "valider" votre existence en utilisant les codes d'accès pour que l'assureur ne réalise jamais que vous êtes mort ou disparu. Bien que cela nécessite un peu d'effort, c'est souvent moins difficile que de devoir gérer un adversaire immortel qui continue de revenir peu importe le nombre de fois que vous le tuiez. 

D'autres dangers existent aussi. Un habitat complet peut-être détruit, vous emmenant avec lui, ainsi que vos sauvegarde et les enregistrements de votre assureur. Un ennemi plein de ressource pourrait pénétrer la sécurité de votre assureur et supprimer vos sauvegardes, ou simplement corrompre les bonnes personnes pour s'assurer que'elles soient "accidentellement" corrompues. Étant donné ces possibilité, les paranoïaques s'assurent souvent d'avoir plusieurs politiques de sauvegardes redondantes, du moment qu'ils peuvent se l'offrir. 

\subsection{Complication de Sauvegarde} 

Dans la pluaprt des cas, sauveagrder/s'uploader est sans risque tant que personne ne bidouille l'équipement. Si le personnage a subit des dégêts cérébraux ou neurologique, la sauveagrde est ransferré par farcaster ou l'upload est fait depuis un personnage mort, la sauvegarde peut être endommagée en raison d'infomration neurale manquante. Pour chacun de ces tests, faites un Test de LUC pour le personnage. En cas d'échec, il subit un point de stress mental par tranche complète de 10 points de MdE. Notez que ce stress (et les trauma éventuels) s'appliquent à la sauveagrde, pas au personnage d'origine. Cependant, si la sauvegarde est utilisé pour réinstantier le personnage, le stress est appliqué. 

\section{Réincarnation} 

\textit{La réincarnation} (ou le remorph) est l'opération consistant à donner un corps à un ego. Changer de corps fait partie du cours normal de la vie pour des centaines de millions de transhumains, et c'est encore plus fréquent pour les personnes dans certaines professiones. Les personnages impliqués dasn un travail spécialisé peuvent avoir à se réincarner à une fréquence mensuelle. Ceux qui voyagent beaucoup peuvent le faire encore plus souvent. De plus, étant donné le nombre d'infugiés 

morts pendant al Chute et qui on mainteannt acquis une nouvelle morph, la vaste majorité de la transhumanité s'est réincrané au moins une fois. En tant que tel, la plupart des transhumains sont accoutumés à la réincarnation. 

S'ajuster à un nouveau corps prend du temps et un peu d'effort (voir \textit{Intégration,} p. 272). La réincarnation est également difficile psychologiquement, comme reflété par la continuité (p. 272) et l'aliénation (p. 272). 

Une fois qu'un ego habite pleinement une nouvelle morph, la pile corticale de celle-ci à besoin de dix minutes pour amasser une sauvegardde complète de l'ego. 

\subsection{Réincarnation dans des Biomorphs et des Pods} 

Se réincarner prend à peu près une heure dans une clinique proprement équipée. par essence, l'opération est un upload inversé. La nouvelle incarantion est branchée à un connecteur d'ego qui inflitre le cerveau avec des nanobots qui restructurent physiquement la structure neurale et les connexion du cerveau en fonction de la carte fournit par la sauvegarde. S'incarner prend six fois plus de temps que l'upload car l'essaim de nanobots fonctionnant comme une imprimante à encre dans le cerveau modèle a besoin de dupliquer la structure physique complète du réseau neural de l'ego. Pour la réincarantion un connecteur d'ego "mouillé"  est utilisé, l'incarnation et le connecteur d'ego sont submergé dans une cuve emplie de nanogel. 

l'opération de réincarnation pour les pods ne prends qu'une demi-heure, les cerveaux des pods étant à moitié biologique et à moitié des cybercerveaux. 

\begin{quotation} \textbf{Le Maître de jeu et la Réincarnation} \\ le maître de jeu a un contrôle assez large de ce que les personnages peuvent obtenir lorsqu'ils se réincarnent. Les personnages peuvent recevoir de nouvelels morphs de la aprt de Firewall ou de n'importe quel employeur/patron pour qui ils travaillent actuellement. Dans ce cas, le maîþre de jeu peut simplement assigner n'importe quelle morph qu'ils pensent adaptées à la situation - avec un contrôle complet des améliorations, des traits, etc. Alors que les morphs doivent être dimensionnée pour al misson en cours, cela présente une opportunité pour le maîþre de jeu de donner de nouveaux jouets aux personnages ainsi que de nouveaux défis à relever. Les maîtres de jeu sont encouragés à inventer, s'amuser et à donner aux joueurs quelque chose qvec lequel ils peuvent travailler sans pour autant leur donner tout ce qu'ils veulent.\\ Dans d'autres situations, la disponibilité des morphs voulues peut-être limlitée par l'endroit de la réincarnation. Un petit avant-poste dans les étendues sauvage de Mars n'aura probablement pas une grande sélection de morph - en fait, quelques rusteurs et des ynthmorphs pourraient être l'ensemble du stock disponible. De manière similaires, les plus grands habitats ont une forte demande pour de bonnes morphs, il pourrait donc y avoir une liste d'attente pour les sylphs haut de gamme ou les refaites par exemple. Dans la même veine, les morphs disponibles seront soumises aux restrictions légales locales, obteni cette morph reapeur pourrait donc bien être hors de question. Les personnages pourront toujorus se tourner vers les fournisseurs de morph du marché noir, mais à leur risques et périls.\\ Cela signifie que le maîþre de jeu ne devrait jamais avoir peur de dire non à un personnage qui cherche à obtenir une morph qui n'est pas raisonnable ou qui peut potentiellement perturber le jeu. Bien qu'il soit de bon ton de donner à vos joueurs ce qu'ils veulent une fois de temps en temps, il est également interessant pour l'interprétation de leur géner avec des morphs un peu différentes de celle qu'ils espéraient ou qui fournit des défis interessants, telles que des addictions physique. Pour plus de fun, laissez le personnage ignorant des traits négatifs ou des implants secrets de la morph jusqu'à ce qu'ils se révèlent d'eux-même. Comem toujours, le but est de s'amuser, mais la variété aide souvent. \end{quotation} 

\subsection{Réincarnation dans des Synthmorphs} 

La réincarnation dans le cybercerveau d'une synthmorph est bien plus facile et rapide, il ne s'agît que de cpier la suavegarde dans le cybercerveau (et c'ets instantané) puis de faire fonctionner la sauvegarde dans son état de cerveau virtuel (1 Tour d'Action). Les inconvénients des synthmorphs sont qu'il est plus difficile de s'y acclimater (voir \textit{Intégration, } p. 272), elles sont vulnérables au piratage (p. 261) et elles sont perçues comme étant bas de gamme dans certaines cultures. 

\subsubsection{Évacuer un Cybercerveau} 

Les personnages habitant un cybercerveau peuvent choisir de le vider volontairement en se copiant en infomorph dans un autre appareil. Cela prend 1 Tour d'Action complet. Voir \textit{Réincarnation en Infomorph,} p. 273. 

\subsection{Coût de Réincarnation} 

Les coûts impliqués dans le processuss de réincarnation sont générallement inclus dans les coûts de l'assurance sauvegarde et/ou de la nouvelle incarnation. Les coûts des morphs individduels sont indiqués dans les descriptions démarrant à la p. 139. Voir \textit{Courtage de Morph} (p. 276) pour les règles concenrant l'obtention des morphs. 

\subsection{Intégration} 

S'habituer à un nouveau corps prends toujours un peu de temps. Le personnage doit s'habituer aux changement de taille, de poids, de sexe et de possibilités, qui nécessitent souvent de désapprendre des façon de faire certaines choses qui marchaient bien dans la forme précédente. La réincarnation dans une morph synthétique ou dans une morph élevée est également relativement perturbant au début, étant donné les écarts morphologiques drastiques, les changements dans la structure des membres (et parfois le nombre de membres), et ainsi de suite. Heureusement, les esprits transhumains sont des objets adaptatifs, et ce processus est aidé par l'application de "patchs" mentaux pendant l'opération de réincarnation et qui fournissent une aide préciseuse au personnage pour l'utilisation de son corps. 

Un ego dans une nouvelle morph fait un Test d'Intégration lorsqu'il prend le contrôle de son nouveau corps, lançant SOM x 3 (les bonus de morph ne s'appliquent pas) et en appliquant les modificateurs de la table des Modificateurs d'Intégration et d'Aliénation. Le résultat du test est expliqué sur la table des Test d'intégration, p. 272. 

\\ 



\begin{table} \caption{Test d'Intégration} \begin{tabular}{|l|l|} 



\hline

RÉSULTAT DU TEST &EFFET\\ \hline

Échec Critique &Le personange est incapable de s'acllimater à la nouvelel morph - \\ &quelque chose ne passe pas. Les personnages subissent un modificateur\\ &de -30 à toutesles actions physique jusqu'à ce qu'il se réincarne. \\ \hline

Échec Catastrophique (MdE 30+) &Le personnage a de grosse difficulté à s'acclimater à la nouelle\\ &morph. Il subit un modificateur de -10 à toutes ses actions pendant 2 jours\\ &et pour 1 jour de plus par tranche de 10 points de MdE. \\ \hline

Échec &Le personnage a quelques difficulté à s'acclimater à la nouvelle morph. \\ &Il subit un modificateur de -10 à toutes ses actiosn phyisque pour 2\\ &jours et pour 1 jour de plus par tranche de 10 points de MdE. \\ \hline

Réussite &Période d'acclimatation standard. Le personnage subit un modificateur\\ &de -10 à toutes ses actions physique pour 1 jour. \\ \hline

Réussite Exceptionnelle (MdR 30+) &Aucun effet secondaire. Le personnage s'acclimate à la nouvelle morph en \\ &quelques minutes. \\ \hline

Succès Critique &En pleine forme! Cette morph va exceptionnellemnt bien au\\ &personnage. Aucun effet secondaire; gagnez 1 point de moxie pour utilisation\\ &lors de cette session uniquement. \\ \hline

\end{tabular} \end{table} 

\begin{table} \caption{Modificateurs d'intégration et d'&liénation} \begin{tabular}{|l|l|} 



\hline

RÉSULTAT DU TEST &EFFET\\ \hline

Familier; le personnage a utiliser cette morph de manière intense dans le passé &+30 \\ \hline

Clône d'une ancienne morph &+20 \\ \hline

Type de morph originel du personnage (celui dans lequel il a grandi) &+20 \\ \hline

Trait Adaptabilité (Niveau 2) &+20 \\ \hline

Trait Adaptabilité (Niveau 1) &+10 \\ \hline

Le personnage a déjà utilisé ce type de morph &+10 \\ \hline

Première réincarnation &–10 \\ \hline

Le personnage est une IAG s'incarnant dans un corps physique &–10 \\ \hline

Le personnage est un élevé s'incarnant dans un corps non élevé (ou pas de son type) &–10 \\ \hline

Morph synthétique &–10 \\ \hline

Changement de sexe (par rapport à la dernière morph) &–10 \\ \hline

La morph est lourdement modifiée &–10 \\ \hline

Trait Trouble Morphique (Niveau 1) &–10 \\ \hline

Trait Trouble Morphique (Niveau 2) &–20 \\ \hline

Infomorph (ne s'applqiue pas aux IAG) (Test d'Aliénation uniquement) &–20 \\ \hline

Fork (Test d'Aliénation uniquement) &–20 \\ \hline

Trait Trouble Morphique (Niveau 3) &–30 \\ \hline

morph exotique (octomorph, néo-avienne, novacrabe, swarmanoïde, etc.) &–30 \\ \hline

\end{tabular} \end{table} 

\subsection{Aliénation} 

Après une perte de continuité, le deuxième facteur principal qui impacte les personnages se réincarnant et l'aliénation. Une fois que l'ego a son nouveau corps sous contrôle, il est temps de regarder dans le miroir. Le test d'aliénaation reflettent l'expérience de la redécouverte de soi, de son nouveau vissage, de sa nouvelle peau et de son nouveau cerveau. Par exemple, se transférer dans une morph radicallement différentes (telles qu'un swarmanoïde par exemple) peut être difficile à saisir. Les élevés ont souvent des difficultés pour s'habituer aux différentes poussées hormonales d'une biomorph humaine et réciproquement. Alors que l'ego des personnage est dans le mrme état que dans leur précédente incarnation, les cerveaux et la biochimie de nombreuses morphs peuvent altérer des aptitudes telles que la VOL ou la COG. Tout cela pouvant générer de la frustration ou de la désorientation. Chaque personnage fait un Test d'Aliénation pour refléter le stress mental nécessaire à la compréhension de son nouveau corps, lançant INT x 3 et en appliquant les modificateurs de la table de Modificateurs d'Intégration et d'Aliénation. Consultez la table des Tests d'Aliénation pour en déterminer les effets. \\ 

\begin{table} \caption{Test d'aliénation} \begin{tabular}{|l|l|} 



\hline

RÉSULTAT DU TEST &EFFET\\ \hline

Échec Critique &Dysmorphie Extrême. Le personnage n'aime pas du tout sa nouvelle incarnation \\ &et subit 2 points de stress par tranche de 10 points de MdE. \\ \hline

Échec &Le personnage n'est pas à l'aise avec sa nouvelle morph et subit 1 point de stress \\ &par tranche de 10 points de MdE. \\ \hline

Réussite &Le personnage s'adapte bien à sa nouvelle apparence. Aucun effet secondaire. \\ \hline

Succès Critique &Meilleure. Morph. Au. Monde. La nouvelle morph correspond exactement à la perception de soi \\ &du personnage, et l'améliore même quelque peu. Le personnage soigne en fait \\ &1d10 $\div$ 2 (arrondissez au supérieur) points de stress. \\ \hline

\end{tabular} \end{table} 

\subsection{Test de Continuité} 

Le plus grand choc qui frappe la plupart des personnages se réincarnant est probablement la perte de la continuite de soi. C'est particulièrement vrai pour les personnages qui sont morts. Si leur pile corticae a été récupérée, ils se rappeleront de leur propre mort. Si ils ont été restaurés depuis une sauvegarde archivée, ils ne se rappelerons pas de leur mort, mais ils auront perdu unt période complète de leur vie - tout depuis leur dernière sauvegarde. En fait, si leur corps n'est pas récupéré, ils peuvent même ne pas être sûr d'être mort - il peut exister une copie d'eux survivant quelque part. Le point culminant de cette perte de continuité est une sorte de crise existentielle - le personnage n'est plus la personne qu'il était auparavant. Cela amène à des questions pour déterminer si ils sont ce qu'il spensent être, ou juste une mauvaise imitation et pas du tout une personne réelle? pour déterminer comment cette perte de continuité affectera un personnage, faites un test de Continuité en lançant VOL x 3. Chaque personnage subit du stress suite à la perte de continuité, tel que noté sur la table du Stress de Continuité. Réduisez ce stress d'1 point par tranche de 10 points de MdRsur le Test de Continuité, ou augmentez le d'1 point pour chaque tranche de 10 points de MdE. 

\subsection{Réincarnation en Infomorph} 

Au lieu de se réincarner dans un corps physique, une sauvegarde peut-être instantiée en temps qu'infomorph, une forme purement numérique. Les infomorphs se distinguent des sauvegarde par le fait que els sauvegarde sont des fichiers inerte. Les infomorphs sont des sauvegarde imprimées sur un cerveau virtuel générique et exécuté comme un programme. Ceté tat de cerveau virtuel doit être exécuté sur un appareil spécifique et obéit aux règles des infomorphs notées p. 264. Les infomorphs peuvent se copier vers d'autres appareils, s'effaçant générallement de l'ancien appareil en partant. Les infomorphs qui se copient sans s'effacer sont considérées comme des forks. Les personnages décidant de s'instancier en tant qu'infomorphs doivent faire les Tests de Continuité et d'Aliénation, comme pour toute réincarnation. Les infomorphs peuvent être réincarnées dans des morphs physique, en suivant les règles normale de réincarnation. 



\begin{quotation} Je me réveille avec un goût de goyave et d'umami frais sur la langue. La nuit dernière n'était qu'une méprise. On a bu du vin de quinoa, et j'ai été présenté à des personnes que je n'avait jamais rencontré avant, bien que j'avais des années de connaissances intimes de la plupart d'entre eux. La moitié des habitatnts du Module Illyria sont nus, recroqueveillé autour de moi, dans ma chambre à coucher. La nuit dernière nous avons jouer de la musique sur synthétiseur, des morceaux de bois et un luth. Nous avons bus du thé de champigon infusé dans de l'eau d'une comète rebelle. En regardant autour de moi alors que les étoiles du matin commencent à éclaire l'hrizon orbital de Cérès, il devient évident que nous avons eu une orgie. Hier soir, c'était ma fête de réincarnation. Cette version de moi -le moi 3.0 - est prête pour la vie. —Zheng du Thierry, Carnaval du Capricorne \end{quotation} 



\section{Forker et Fusionner} 

Avec toutes ces sauvegardes d'esprits transhumains sur fichier et l'abondance d'espace meshé sur lequel les faire fonctionenr en tant que cerveaux virtuels, on pourrait se demander ce qui a stopper la transhumanité fpost-Chute de se multiplier en lançant des copies additionnelel d'eux-même. La réponse rapide est: rien, à part la stigmatisation sociale massive et les problèmes psychologiques épineux. Prendre une sauvegarde d'un esprit transhumain, le copier et le ré-instantiet en infomorph est appelé \textit{faire un fork.} C'est l'une des applications les plus puissantes et la plus controversée de la science cognitive transhumaine. 

Il y a quatre classifications des forks: alpha, béta, delta et gamma. Bien qu'étant générallement copiée comme infomorph, il n'y a rien qui empêche un fork d'être incarné dans une morph physique, à part les coutumes et les lois. 

\begin{table} \caption{Stress de continuité} \begin{tabular}{|l|l|} \hline

SITUATION &VALEUR DE STRESS\\ \hline

\textbf{Sauvegarde de la pile corticale} &\\ \hline

Le personnage se souvient d'une mort paisible ou non remarquable. &1d10 $\div$ 2 (arrondissez à l'inférieur) \\ \hline

Le personnage se souvient d'une mort rapide ou violente. &1d10 \\ \hline

\textbf{Sauvegarde d'archive} &\\ \hline

Petit trou mémoriel (moins d'1 jour) &1d10 $\div$ 2 (arrondissez à l'inférieur) \\ \hline

Trou mémoriel de plus d'1 jour &1d10 \\ \hline

Ignorer si/comment vous êtes mort. &+2 \\ \hline

Upload vers réincarnation en continuité (p. 269) &0 \\ \hline

Upload vers réincarnation sans continuité (p. &1d10 $\div$ 2 (arrondissez à l'inférieur) \\ \hline

Le personnage est un fork &2 \\ \hline

\end{tabular} \label{table:continuity-stress} \end{table} 

\subsection{Forks Alpha} 

Un \textit{fork alpha} est une copie exacte de l'ego d'origine et ré-instantié comme infomorph séparée. Un fork alpha peut-être créé en copiant et en exécutant une infomorph (depuis une sauvegarde, une finomorph, un cybercerveau ou une pile corticale extraite et branchée dans un connecteur d'ego). Les forks alpha peuvent être généré depuis un cerveau de biomorph en utilisant un connecteur d'ego et les même traitements que l'upload (p. 268). Les forks alpaha sont des copies exacte de l'ego du personnage, avec les mêmes compétences, souvenirs, stats, traits, personnalité, etc. Les nouveaux forks alpha doivent faire un Test d'Aliénation (p. 272), et probablement un Test de Continuité (p. 272) si il est copié depuis une sauvegarde. 

Créer des forks alpha est illégal dans la plupart des juridcictions, incluant la plupart du système intérieur et la République Jovienne. Dans d'autres endroits, c'est considéré avec dégoût, bien qu'il y ait certains habitats/cultures dans lesquels cette opération est encouragée. 

\subsection{Forks Béta} 

Les forks béta sont des copies partielles de l'ego. Elles sont intentionnellemnt entravée afin de ne pas être considérée un égal du personnage, d'un point de vue légal notamment. Les forks béta ont la plupart des compétences de l'égo original, bien que parfois réduites. leurs souvenirs sont également drastiquement réduit, générallement taillé pour les tâches qu'ils sont sensé accomplir. Les forks béta sont créés en prenant un frk alpha et en le faisant passer passer par le process connus sous le nom \textit{d'élagage neural} (p. 274). Ils sont légaux voire commun par endroit, exceptés dans les places fortes bioconservatives telels que la République Jovienne, bien que els forks delta sont souvent préférés. Les forks béta ont rarement quoi que ce soit se rapprochant de droits civils ou de citoyenneté et ont sont usuellement considéré comme la propriété de l'ego d'origine. Ils sont fréquemment utilisés comme assistant numérique ou pour représenter l'ego d'origine lors des communications longue distance avec d'autres. \\ Les stats d'un fork béta sont déterminées ainsi: 

\begin{itemize} \item Réduisez toutes les aptitudes de 5 (avec un minimum de 1). Cela affecte également toutes les compétences. De manière identique, cela réduit la LUC de 10 et l'INIT de 20. \item Les compétences active sont limitées à 60 maximum. \item Le Moxie est réduit à 1 \item Le trait Psi est supprimé. À la discrétion du maître de jeu, d'autres traits pourrainet être supprimés également. \end{itemize} 

Des changements additionels peuvent s'appliquer en fonction du test d'élagaage neural. Les forks béta nécessitent 1 minute pour être créés. 

\subsection{Forks Delta} 

Les delta forks sont des copies extrêmement limitée d'un ego. Elles sont plus proches des modèles d'IA sur lesquel les traits de personnalité en surface de l'ego sont imprimés. Également créés par l'élagage neural, les foks delta sont pleinement fonctionnel (autant compétent qu'un fork béta ou une IA) mais ont des compétences extrêmements limitées et des souvenirs trés lourdement édités, généralement au point d'être des amnésique fonctionnels. \\ Les stats d'un fork delta sont déterminées ainsi: 

\begin{itemize} \item Réduisez toutes les aptitudes de 10 (avec un minimum de 1). Cela affecte également toutes les compétences. De manière identique, cela réduit la LUC de 20 et l'INIT de 40. \item Les compétences active sont limitées à 40 maximum. Le fork ne peut pas avoir plus de 5 compétences Actives. 

\item Les compétences de Connaissances sont limitées à 80. Le fork ne peut pas avoir plus de 5 compétences de Connaissances. \item Le Moxie est réduit à 0. \item Le trait Psi est supprimé. À la discrétion du maître de jeu, d'autres traits pourrainet être supprimés également. \end{itemize} 

Des changements additionels peuvent s'appliquer en fonction du test d'élagaage neural. Les forks béta nécessitent 1 Tour d'Action pour être créés. 

\subsection{Forks Gamma} 

Plus communament appelés \textit{vapeurs,} les forks gamma sont essentiellement des copies incomplètes, corrompues ou fortement endommagées d'un ego. Les vapeurs ne sont pas intentionnellement créés et sont généralement le résultat d'upload ratés, de sauvegarde brouillées, de farcast incomplets ou interceptés ou d'infomorphs/forks qui ont été endommagés d'une manière ou d'une autre ou qui sont devenus fous. Il est extrêmement rare que quiconque crée vvolontairement une vapeur pour autre chose que la recherche, bien qu'ils puissent se développer dans certains endroits intéressants. Par exemple, de mauvais logiciels de compétences peuvent occasionnellement inclure suffisament des traits de personnalités et des souvenirs de la personne a qui la compétence a été prise pour qu'elle puisse se comporter de manière vaporeuse lorsqu'il est utilisé. 

Comme les vapeurs sont des anomalies plutôt que des créations volontaires, les caractéristiques du'n fork gamma sont laissé à l'appréciation du maître de jeu. Ils devraient avoir une partie ou toutes les caractréistiques suivantes: compétences réduites, aptitudes réduites, souvenirs incomplets ou incohérents, traits mentaux négatifs et du stress mental ou des traumas persistant inclaunt des dérangements et/ou des troubles. 

\subsection{Élagage Neural} 

L'élagage neural est l'art de prendre une sauvegarde/infomorph et d'en réduire la taille pour q'uelle fontcionne comme un fork béta ou delta. 

Les forks béta sont créés en utilisant un état d'esprit virtuel qui est intentionnellement inhibé et en filtrant un égo au travers de celui-ci. Comme pour les arbustes topiaires, les portions du réseau de neurones du personnages qui dépassent la taille du fork voulu sont retirés. En plus des changements notés sous \textit{Forks Béta} (p. 273), les personnages peuvent vouloir effacer/réduire certaines compétences et enlever certains souvenirs. 

Les forks delta sont créer en excisant les traits de personnalité de surface de l'ego et les appliquant sur une IA générique. Dans ce cas, les souvenirs de l'ego sont généralement complètement exclus - il est facile de commencer avec un frok delta vierge et de l'alimneter ensuite des souvenirs/connaissances spécifiques dont ils ont besoin. Comme pour les forks béta, les personnage faisant des forks delta peuvent volontairement choisir d'effacer/diminuser certaines compétences et de conserver certains souvenirs. Si un fork alpha n'est pas disponible pour l'élagage, un fork delta peut être extrait d'un cerveau de biomorph avec un connecteur d'ego et en 1 minute. Beaucoup de personnes incarnés dans des biomorphs conservent des forks delta à portée de main, pour pouvoir les élaguer à la volée au cas où. 

La compréhension transhumaine des neurosciences s'étendent à l'analyse et à la copie d'un esprit, mais le fonctionnement intrinsèque des souvenirs et toujours imparfaitement compris. Faire des éditions précises sur des portions d'un réseau de neurones (pour modifier des souvenirs, des compétences et autre) est toujours un art mystique. La difficulté dans l'élagage neural est que donner un coup de sécateur sur une branche mémorielle n'est pas vraiment une science exacte. Les souvenirs spécifiques ne peuvent être excisés ou conservés - au mieux, les souvenirs sont gérés en buisson, générallement groupés par période de 6 mois minimum. Dans un but de simplicité, la plupart des forks béta sont créés en supprimant tous les souvenirs vieux de plus d'un an. 

En créant un fork béta ou delta, le personnage doit faire un Test de Psychochirurgie (d'autres personnes peuvent faire ce test à la place du personnage, représentant le fait qu'il leur a donné les accès pour élaguer le fork de manière appropriée). Si le personnage réussit son jet, le fork est créé comme désiré. Si le test échoue, le maîþre de jeu choisit l'une des pénalités suivantes pour chaque tranche de 10 points de MdR. Certaines de ces pénalités peuvent être combinés pour avoir un effet cumulatif: 

\begin{itemize} \item 1 compétence supplémentaire réduite de –20 \item Le fork développe un trait mental négatif d'une valeur de 10 PP \item Le fork subit 1d10 $\div$ 2 (arrondissez au supérieur) points de stress\item Des souvenirs supplémentaires sont perdus (à la discrétion du maîþre de jeu; fork béta uniquement) \item 1 trait positif perdu \end{itemize} 

\subsubsection{Élagage Neural par Psychochirurgie à Long-Terme} 

Au lieu de générer des forks à la volée, certains personnages préfèrent avoir des forks minutieusement élagués sous la main, stockés dans des fichiers inertes et qui peuvent être appelés, copiés et lancés en fonction des besoins. Ces forls sont créés avec de la psychochirurgie à long-terme, ce qui signifie qu'ils ne souffrent que de quelques inconvénients et que leurs souvenirs sont plus finement ajustés. 

L'élagage neural à long-terme nécessite un Test de Psychochirurgie comem au-dessus, mais avec un modificateur de +30. Les forks deltas sont élagués de cette manière en 1 semaine et les forks bétas en 1 mois. Des modifications supplémentaires peuvent être effectuées sur le fork en utilisant les règles normale de psychochirurgie (p. 229). 

Il n'est pas rare que certaines personnes préfèrent utiliser des forks d'eux-même ou de personnes chéries plutôt qu'une muse. De manière similaires, certaines hyperélite prospères sont connus pour conserver des copies de sauvegarde plus jeune à portée de main, parfois pendant des décennies, et de les réinstantier lorsque leur ego primaire a suffisament de compétence et d'expérience pour surclasser leur eux plus jeunes. Bien qu'ils soient techniquement des forks alpha, leur décalage vis à vis de l'ego original est comparable par certains aspects à celui d'un fork béta. Il se dit que ce serait la méthode utilisée par pax Familiae lors de l'instantiation de son arémes de clône d'elle. 

\subsection{Gérer les Forks} 

Les maîtres de jeu sont encouragés à autoriser les joueurs à interpréter les forks de leur personnage. Il est important de noter cependant que, mrme avec des forks alpha, une fois que le fork et l'ego originele ont divergés, ils se développent par la suite comme des personnes distinctes. Les évènements qui forment la personnalité principale, le caractère et la connaissance d'un ego n'arriveront pas au fork - et même si c'est le cas, ils n'affecteront pas le fork de la mrme manière - et vice-versa. La frotnière exacte entre un ego et un fork est un débat philosophique et juridique central pour de nombreux transhumains. 

Cela signifie qu'un maître de jeu ne devrait pas être effrayé d'enlever un fork des mains d'un joueur et de le transformer en PNJ si ils commencent à diverger grandement. De même, si un fork commence à apprendre des informations auquelles le personnage principal n'a pas (encore) accès, il est probablement mieux d'utiliser le fork comme un PNJ afin d'éviter le métajeu. Il est complètement possible qu'un fork décide qu'il n'obéira plus à l'ego d'origine et décide de s'occuper de gérer sesd propres problèmes. Cela n'arrive généralement qu'avec les forks alpha,q ui sont essentiellement des copies compltes, et plus le temps passe, plus l'idée de retourner dans l'ego originel perd en attrait. Les forks béta et delta sont relativement conscient de leur nature de copie incomplètes, et retournent génrallement vers leur ego d'origine pour être réintégrés. Cependant, il y a des cas rares dans lesquels mrme ceux-ci décident d'aller vivre leur propre vie. 

\subsection{Fusionner} 

La fusion est le procédé de ré-intégration d'un fork avec son ego d'origine. Cette opération est effectuée sur des egos/forks conscient, transferrant les deux en un seul égo fusionné. Le procédé n'est pas difficile à mener lorsque les deux forks n'ont été séparés que pendant une courte période de temps. Plus les forks passent du temps sépras, plus le procédé de fusion devient complexe, un test de Psychochirurgie devient nécessaire (effectué soit par l'ego, soir par un autre personnage supervisant les opérations). La table de Fusion liste les modificateurs de ce test ainsi que les résultats de réussite ou d'échec. Pour les synthmorphs, la fusion prend un Tour d'Action complet. pourt les biomorphs, un connecteur d'ego (p. 328) ou des augmentations mnémoniques (p. 307) sont nécessaire pour fusionner, et le procédé prend 10 minutes. Le résultat de l'opération est un ego unifié, que le Test de Fusion réussisse ou pas. La psychothérapie (p. 209) et la psychochirurgie (p. 229) peuvent réparer les mauvaises fusions au fil du temps. \\ 

\begin{table} \caption{Fusionner} \begin{tabular}{|l|l|l|l|} \hline

TEMPS SÉPARÉS &MODIFICATEUR &RÉUSSITE &ÉCHEC\\ \hline

Moins d'1 heure &+30 &Ego sans déaut avec les souvenirs &Souvenris intacts, (1d10 $\div$ 2, arrondi à l'inférieur) – 1 VS \\ &&des deux parties. &\\ \hline

1–4 heures &+20 &Lien solide, souvenirs intact &Souvenris intacts, (1d10 $\div$ 2, arrondi à l'inférieur) – 1 VS \\ \hline

4–12 heures &+10 &Souvenris intacts, 1 VS &Perte de mémoire mineure, (1d10 $\div$ 2, arrondissez au supérieur) VS \\ \hline

12 heures - 1 jour &+0 &Souvenris intacts, 2 VS &Perte de mémoire modérée, (1d10 $\div$ 2, arrondissez au supérieur) + 2 VS \\ \hline

1 jour - 3 jours &–10 &Souvenris intacts, 3 VS &Perte de mémoire majeure, 1d10 + 2 VS \\ \hline

3 jours - 1 semaine &–20 &Souvenris intacts, 4 VS &Perte de mémoire majeure, 1d10 + 4 VS \\ \hline

1 semaine et + &–30 &Perte de mémoire mineure, 5 VS &Perte de mémoire critique, 1d10 + 6 VS \\ \hline

\end{tabular} \label{table:merging} \end{table} 

\begin{quotation} \textbf{LE SOI} \\ Forker et fusionner peut avoir changer la manière dont la transhumanité se définit et ce que signifie le fait d'avoiur une personnalité bien intégrée. \\ Alors que faire des fork est un jeu d'enfant d'un point de vue technologique, les effets psychologiques et sociaux du clonage d'un esprit fait que la plupart des gens hésitent à utiliser des forks. Certaines juridictions bannissent l'utilisation des forks, à l'exception des utilisations médicales, alors que d'autres ont des restricions élevées. Dans beaucoup de juridiction hypercorporatiste, par exemple, les forks alphas sont illagaux et laisser un fork béta actif pendant plus de 4 heures sans fusionner viole les descendantes modernes des lois antitrust du 20° siècle. De amnière similaire, la Junte Jovienne et d'autres bioconservateurs interdisent purement et simplement les forks. \\ S'occuper des forks non désirés est un autre problème épineux. Dans certains endroits, cela se résume simplement à leur suppression, un esprit stocké n'ayant aucun statut légal. Dans d'autres lieux, un fork qui ne souhaite pas réintégrer son ego d'origine pourrait se voir accorder certains droits, bien qu'ils ne soient généralement accordés qu'aux forks alpha. \\ De manère plsu significative, faire fonctionner un fork de soi pour des périodes d'une heure ou moins est une tâche realtiavement simple pour de nombreux transhumains. Beaucoup de personnes font usage de forks pour accomplir les tâces quotidienne, et presque tout le monde a fait l'expérience du fork à un certain niveau. \\ Les transhumains regardent la création de fork comme les humains du 21° siècle considéraient la consommation d'alcool ou de drogue. Un peu ne pose pas de propblème, mais en abuser sera stigmatisé. C'est princiaplement parceque la plupart des transhumains comprennent les conséquences psychologiques de l'abus de fork. \end{quotation} 

\section{Egocast} 

En dépit d'être une civilisation spatiale avec des avant-postes dans tout le système solaire et au-delà, la transhumanité n'utilise que trés peu les vaisseaux spatiaux pour le voyage interplanétaire. Des navettes utilisant divers système de propulsions assurent des liaisons régulières entre les habitats, la surface plantéaires et les lunes. mais pour tous les voyages de plus de 1.5 million de kilomètres - la distance qu'un moteur à fusion peut couvrir en une journée, les gens préfèrent s'egocaster. L'egocast est la technologie transhumaine de transport personnelle la plus avancée, bien que seul l'ego du personnage voyage effectivement. L'egocast combine les technologie d'upload et de farcast quantique pour transférer une suavegarde (ou parfois un égo conscient, voir p. 269) à travers des distances interplanétaires. 



Bien que l'égocast se apsse à la vitesse de la lumière, les temps d'egocast varient en fonction de la distance. S'egocaster dans un système en grappe ou planétaire est généralllement l'histoire de quelque minutes. S'egocaster depuis le Soleil jusque dans la Ceinture de Kuiper prend cependant entre 40 et 70 heures, et s'égocaster à tarvers tout le système solaire peut prendre encore plus de temps. 

une fois qu'un ego arrive à destination, il peut être archivé, damérré en infomorph ou réincarner normalement. 

\subsection{Sécurité des Egocasteurs} 

S'envoyer à travers l'espace interplanétaire est une technologie époruvée et fonctionne généralement sans accroc. En raison de l'utilisation de farcast quantiques, l'egocast ne peut être perturbé par des interférence radio brouillant le signal et causant une eprte de donnée. Normalement l'ensemble du procédé est véhiculé par le prestataire de service de sauvegarde du personnage, et les brèches de sécurités sont rares. \\ Il existe cependant plsuieurs risues intrinsèque à l'egocast. Le plsu évident est que la conscience du personnage est transférée en tant que fichier de sauvegarde vers la destination. Si l'egocasteur à l'autre extrémité n'est pas certifié ou si le réseau de destination est contrôlé par le réceptionniste, le personnage se mets potentiellement à la merci de leur hôte. beaucoup d'hypercorp considèrent que bidouiller un ego transmlis est une brèche sérieuse de l'étiquette, alors que les autonomistes le considère incroyablement répressif. Cependant, les groupes d'extrémistes politques et les organisations criminelles qui contrôllent les egocasteurs n'acceptent que peu de restrictions. \\ Un risque plus subtil est la possibilité pour les hackeurs d'exploiter des trous de sécurité dans l'egocasteur et de l'espace virtuel lié pour voler un fork du personnage. Cela reste extrêmement difficile à faire. Cela n'arrive presque jamais lross des uploads normaux, car les services d'upload sont conscient des problèmes de sécurité au point de la paranoïa. Mais même ainsi, les forks volés par de telles tentatives finissent plus souvent qu'à leur tour sous forme de vapeur, l'intrus étant géénralement arrété avant qu'une copie complète ne pusise être obtenue. 

\subsection{Darkcast} 

les personnages désirant s'egocaster sans attirer l'attention des officiels tels que les services des Douanes et de l'Immigration peuvent passer par des services appelés darkcast - des farcast émetteurs-récepteurs illagaux et contrôlé par les syndicats criminels et d'autres groupes clandestins. Pour trouver un service de ce type, un personnage doit utiliser sa coméptence Réseau et probablement sa réputation (p. 285). 

\section{Courtage de Morph} 

Les morphs sont un bien de consommation dans la société transhumaine. La technologie et les matériaux nécessaire à la croissance de nouvelles morphs sont bons marchés et abondant, même si cela prend du temps. Les biomorphs clônés prenent au moins un an et demi pour se développer, même avec une croissance accélérée. Les pods, qui sont généralement assemblés à partir de morceau qui ont poussés en cuve, nécessitent 6 mois. Les synthymorphs comme les boîte et les synths peuvent être produite dans la journée, mais des modèles plus complexe peuvent prendre plus d'une semaine. Théoriquement, l'offre va un jour surpasser la demande au point où la chair sera gratuite. 

Les personnages ont plusieurs options pour acquérir des morphs lorsqu'ils voyagent par egocast, subissent de gros dégâts ou ont simplement envie d'un nouveau corps. Lors de l'egocast, la méthode la plus fréquente pour les voyageurs de la classe moyenne est de stocker leur morph actuelle dans une maison de poupée sécuriése et de louer une morph à destination. Moins fréquemment, les personnages peuvent se reoposer sur des installations de réincarnation publique ou, si ils ont les moyens, ils peuvent carrément acheter une nouvelle morph. Les personnage qui s'attendent à rester à destination indéfinemment ou qui décident de se réincarner mais qui ne voyagent pas peuvent opter pour échanger leur vieux corps, l'abandonnant de manière permanente. 

\subsection{Disponibilité des Morphs} 

Comme noté au paragraphe \textit{Le Maître de Jeu et la Réincarnation} (p. 271), trouver le modèle de morph voulu n'est pas toujours simple. Alors que de nombreux type de morph (boîte, synths, spliceurs) sont générallement disponible, les personnages peuvent aussi de nouvelles 

morphs en utilisant leur compétence Réseau (voir \textit{Réputation }\textit{ et réseaux Sociaux,} p. 285). Certains types de morphs sont plus difficile à trouver que d'autres; le maîþre de jeu devrait appliquer un modificateur approprié pour toute les morphs qui semblent rare ou inhabituelle (par exemple, les swarmanoïde ou les reapeurs). De même, de nouvelles morphs peuvent être bêtement indisponible à un endorit donné. Les rusteurs sont rarement disponibles hors de Mars, par exemple, alors que sur Europe la plupart des morphs sont des variantes locales exotiques et aquatiques. 

Le maîþre de jeu détermine quelels factions sont capable de fournir de nouvelels morphs dans une zone donnée. Les factions ne fourniront pas de morphs qui leur sont indisponible aux personnages débutants. Si la faction n'est pas dominante dans une zone, une pénalité de -1à à -30 devrait être appliquée. En dépit de la présence dans une zone donnée, certaines factions peuvent être incapable de fournir des morphs. 

Si un personnage cherche une morph customisée avec des implants ou des amélioratiosn spécifiques, la recherche sera encore plus difficile. Le maître de jeu devrait également y appliquer un modificateur de -10 à -30, en fonction de l'étendu et de la légalité des modifications recherchées. 

\subsection{Acquisition de Morph} 

Une fois que la morph est localisée, le personnage doit faire appel à des faveurs (p. 285) ou payer des crédits pour l'obtenir. Le coût des morphs est noté sur la table de Coût des Morphs. Dans le système intérieur, le prix des morphs est souvent augmenté par la demande du marché et les morpsh les plus désirées peuvent valoir une petite fortune. A l'extérieur, les prix en rep sont plus raisonnable mais toujours élevés en raison de la pression de la population sur les abris dépendant d'un soutien du système extérieur. Pour les voyageurs et ceux qui changent fréquemment de corps, il y a de nombreuses façon de réduire ces coûts. 

\subsubsection{Revente Et Entremetteurs} 

Trouver une morph pour les voyageurs et ceux qui n'ont pas de corps est une compétence spécialisée nécessitant des résedaux sociaux solide et un sens de la négociation. En géénrale, c'est un marché de revente, les revendeurs (ou "entremetteurs," comme ils sont appelés dans la nouvelle économie) agissent comme des agents de la personne cherchant un corps. La table du Coût des Morphs prend en compte la marge de 10\% de l'intermédiaire. Les personnages souhaitant éviter cet agent peuvent réduire le coût de 10\% mais subissent une pénalité de -30 à leur Test de Réseau.pour localiser une morph disponible. 

\subsubsection{Morphs Personnalisées} 

Si un personnage cherche à obtenir une morph personnalisée (avec du bioware, du cyberwware ou des implants nano supllémenatire, ou des amélioratiosn raobotiques), le coût de ces améliorations est ajouté au coût de la morph (si le maître de jeu le décide, une réduction peut-être appliquée sur le lot). De même, les morphs peuvent être fournit avec des trait de morphs positifs ou négatif (p. 145). Ces traits augmentent ou diminuent le coût de la morph au taux de +500 crédits par PP pour les traits positifs ou de -200 crédits par PP de traits négatifs. Les traits négatifs reflettent typiquement les abus que la morph a subit entre les mains de l'occupant précédent. 

\subsubsection{Échange} 

Pour ceux qui désirent abandonner définitivement leur ancienne morph, les échanges de morph sont des options courantes. La demande de corps élevée signifie qu'un acheteur est presque toujours disponible, sauf si le maître de jeu trouve des circonstances particulières. Les morphs peuvent être échangée à la valeur indiquée dans la table des Coûts de Morph ajustée par tous les traist positif ou négatifs, avec une réduction de 10\% pour les examens physique et les frais du dénicheur. Cette somme est soit payée au vendeur en credits soit rendues comme faveur en utilisant la rep. 

\subsubsection{Provisionnement d'un Mentor} 

Les personnages en mission pour un mentor riche ou intégral peuvent avoir des morphs fournit par lui. Normalement de telles provisions sont faites pour la durée du travail, même si parfois la morph peut considéer en soi une forme de paiement pour le service rendu. les maîþres de jeu sont encouragés à être créatif lors de tels arranegemnt, bien que les joueurs devraient être conscient que de tels échanges peuvent rapidement se révéler Faustiens. 

\subsubsection{Morpsh du Marché Noir} 

Les vendeurs de crops du marché noir promettent de fournir l'acheteur avec les morphs et les mises à jour de son choix indépendamment des lois contre les armes ou les implants, en plus de contourner les procédures d'enregsitrelment des arrivées standard grâce au darkcast. Les morphs illégales sont généralement fournies avec une hausse de prix (au moins +25\%), alors que les morphs d'occasion ayant un passif louches (et ds traits bizarre) sont générallement moins chère (-25\%). 

\begin{table} \caption{Coût des morphs} \begin{tabular}{|l|l|} \hline

TYPE DE MORPH &COÛT \\ \hline

\multicolumn{2}{|c|}{Biomorphs} \\ \hline

Plates, Spliceurs &Élevé \\ \hline

Octomorphs &Chère (30 000+) \\ \hline

Furie, Ghosts, Refaite &Chère (40 000+) \\ \hline

Futuras &Chère (50 000+) \\ \hline

Toutes les autres &Chère \\ \hline

\multicolumn{2}{|c|}{Pods} \\ \hline

Pods de Travail, de Plaisir &Élevé \\ \hline

Novacrabes &Chère (30 000+) \\ \hline

\multicolumn{2}{|c|}{Synthmorphs} \\ \hline

Boîtiers &Modéré \\ \hline

Synths, Libellulles &Élevé \\ \hline

Slitheroïdes, Swarmanoïdes &Chère \\ \hline

Transformers &Chère (30 000+) \\ \hline

Arachnoïdes &Chère (40 000+) \\ \hline

Reapeurs &Chère (50 000+) \\ \hline

Traits de morphs positifs &+500 par PP \\ \hline

Traits de morphs négatifs &–200 par PP \\ \hline

\end{tabular} \label{table:morph-costs} \end{table} 



\subsubsection{Contraction} 

Les personnages trop fauchés pour s'offrir une nouvelle morph peuvent conclure un marché pour du service contracté - un "accord" qui est rarement à l'avantage du 

nouveau contracté. Les contrat typique nécessitent des années de travail contractés - terraformation de Mars, exploitation de comète, minage d'astéroïde, construction d'habitats, colonisation d'exoplanètes, etc. - en échange d'une synthmorph ou d'un spliceur bon marché à la fin du contrat. Les maîþre de jeu doivent user de leur discrétion lorsqu'ils proposent de telles clauses, bien que dans de nombreux cas les clauses proposées terminerons temporairement ou définitivement la carrière d'agent indépendant du personnage. Les hypercorporations utilisant des travailleurs contractées sont connues pour changer les clauses à volonté, allongeant la période de service ou frappant le contracté avec un lot de charges cachée ou outrageuses qui n'étaient pas claire à la signature. Les personnages peuvent, bien entendu, entrer pleinement dans le système pour avoir leur morph et partir en courant à la première opportunité, mais les hypercorporations protègent jalousement leurs investissements. Les contractés sont surveillés et pisté de près, et les hypercorporations ne se privent pas d'envoyer des chasseurd d'ego pour récupérer un fuyard. 

\subsubsection{Réincarnation Publique} 

À certains endroits, sur Titan par exemple, il existe une infrastructure de réincarnation publique bien développée afin de fournir un corps à tout ceux qui en ont besoin. Les morphs fournies sont généralement des boîtiers génériques, des synths ou des spliceurs basiques sans traits Positif ou implant optionnels. Quiconque ayant la citoyenneté d'une zone ayant des services publiques de réincarnation peut postuler pour obtenir un corps. Les temps d'attentes varient d'un mois à deux ans,  la Réputation influençant ce délai d'attente à la discrétion du maître de jeu. 

\subsection{Location de Morph} 

Pour les visites temporaire et lorsqu'un infomorph ne sera pas suffisante, des morphs peuvent être louée au lieu d'être achetée. Le coût de location d'une morph est de 1\% de son coût par jour, augmenté d'un prix Bas pour la réincarnation. Ce coût inclut l'assurance locative (voir plus bas). Si l'assurance locative est omise (ce n'est pas toujorus possible, à moins d'avoir une bonne Rep), le coût de location peut être réduit de moitié. 

Les personnages qui louent une moprh peuvent également utiliser leur morph précédente comme caution. Dans ce cas, déduisez le coût de la morph actulle du personnage de la morph louée avant de calculer le coût de 1\% par jour, avec un coût de location minimum de 10 crédits par jour. 

\subsubsection{Location Pénale} 

Les personnages visitant le système intérieur ou la République Jovienne peuvent avoir la possibilité de louer des morphs appartenant à des prisonniers. Dans la plupart des juridictions, les criminels sont condamnées à des peines en simulsapces de réhabilitation, peine qui stipule que la morph du pprisonnier devient propriété de l'étât pendant leur peine d'incarcération. Les morphs acquisent de cette façon ont souvent des historiques compliqués mais tendent à avoir des modifications utiles pour els agents de Firewall. Inversement, un perrsonnage qui se trouve emprisoné peut-être sujet à voir son corps loué à d'autres pendant son incarcération. 

Les effets de la location pénale sont laissées à la discrétion du maître de jeu. Un personnage pourrait avoir à tirer quelques ficelles avec sa Réputation pour louer de telle morph, particulièrement si elle a des modifications d'accès restreinte ou illégale. Les traits négatifs, les erreurs d'identité et les rencontres impromptues avec des amis et des associés de l'ancien propriataire de la morph font parti des inconvénients possible pour ce type d'arrangement. Du bon côté des choses, les locations pénales peuvent réduire les coûts pour la location et l'assurance de la morph, encore une fois ceci est laissé à l'appréciation du maître de jeu. 

\subsubsection{Assurance Locative} 

Les morphs louées doivent être couverte par une police d'assurance, qui empêche généralement l'utilisateur de violer la loi ou d'amener la morph dans les zones hors la loi ou trop dangereuses. les personnages peuevnt acheter une assurance de risque qui couvrira le fait d'emmener la morph dans certaines situations dangereuse, mais cela doublera (au moins) le prix de la location. 



Si un personnage subit des dégâþs organique ou décède alors qu'il est assuré, l'assurance couvrira 80\% du coût de la morph, ce qui veut dire que le personnage devra payer les 20\% restant. Si iles ne peuvent pas payer, leurs possession ou leur moprh peuvent être saisi en gage. 

Si un personnage viole les clauses de la police d'assurance en se mettant en danger volontairement et au-delà du niveau de menace pour lequel l'assurance à été acheté, sans le signaler à l'assureure et sans le payer en conséquence, la olice d'assurance peut-être déclarée nulle. Si la morph louée décède sous une police d'assurance annulée et que le personnage ne peut pas payer pour la remplacer, ses possessions et sa morph stockée peuvent être saisies. 

La saisies prend différentes formes en focntion de l'économie locale et du système légal. Dans l'espace hypercorporatiste, il s'agît d'une saisie simple des liquidités, incluant un upload forcé si la morph du personnage est saisie. Partout ailleurs, le personnage se retrouvera à devoir énormément de faveurs ou à subir de grave perte de réputation, mais ils ne subiront pas d'upload forcé ou d'autres mesures de saisie de leur morph. 

\section{Identité} 

Vu la nature des technologies de réincarnation, l'identité est un concept fluide à \textit{Eclipse Phase.} Les transhumains sont habitués à l'idée d'identifier une personne par leur apparence ou mrme par leur données biométriques, mais ce n'est plus une méthode certifiée. Votre apparence peut changer drastiquement d'un jour à l'autre. Vous pouvez croiser un olympien que vous reconnaissez, mais ils e peut que cela fasse un peu de temps depuis la dernière fois, et vous n'êtes donc plus vraiment sûr que c'est la même personne dans cette morph. Si vous êtes incarné dans une morph de série populaire, il peut y avoir des centaines d'autres morphs clônés qui ont exactement la mrme apparence que vous - cela peut-être utile si vous voulez vous fondre dans la masse. De manière similaire, les services de sécurté ne reposent plus sur les technologies de biométrie. La criminalistique permet d'identifier la présence d'une morph en particulier sur une scène de crime, mais elle ne peut prouver qui était dans cette morph. 

Bien entendu, l'identité est liée à l'ego et certaines autorités ont instituée des vérifications et des mesures de sécurité basées dessus. Dans le ssytème intérieur, chaque ego reçoit un numéro d'ID, qui est utilisé pour valider son identité, sa citoyenneté, son statut légal, ses comptes, ses permis, etc. Cet ID d'ego est vérifiable par la forme des ondes cérébrales d'une eprsonne, qui restent inchangée, même après une réincarnation. Lorsqu'un ego s'upload, le servie d'upload doit intégrer cet ID d'ego dans la sauvegarde/l'infomorph de la personne. De même, lorsqu'une personne se réincarne, le service s'occupant de la procédure est contraint par al loi de vérifier l'ID d'ego de la personne avant de la télécharger. l'ID d'ego est ensuite codé en dur dans la morph sous la forme d'un nanotatouage à l'extrémité de l'index de la perosnne. Ce nanotatouage peut être facielemnt scanné lors des contrôles de sécurité pour vérifier l'identité. 

Bien que ce système soit efficace, il est loin d'être parfait. Par exemple, conserver une trace de l'identité est loin d'être standardisé et varie drastiquement d'un habitat à l'autre. La plupart ne partagent pas ces enregsitrements avec les autres afin de protéger la vie privée de leur citoyens, à moins qu'ils ne fassent partis de la même alliance politique. Par exemple, les stations de l'alliance Lunaire-Lagrange ne aprtagent pas les données d'identité de leur citoyens avec le Consortium Planétaire, bien qu'ils les partagent entre eux. En plus de cela, beaucoup d'enregistrement d'identité ont été perdus pednant la Chute, une situation qui a trés probablement été exploitée par ceux qui ont préférés effacer leur passé ou adopter une nouvelle identité. Tout cela fait que les enregistrement d'identité sont, au mieux, un patchwork. Les officiels doivent également faire avec la sécurité d'autres habitats pour les vérifications d'identité. Si une personne s'egocast de Qing Long dans les Troyens Martiens vers Nectar sur Mars, et que les officiels de Nectar n'ont aucun enregistrement de cette personne, ils sont obligés de faire confaicne au travails des officiels de Qing Jong lorsqu'ils ont vérifiés l'identité et le passif du sujet. Pour empirer les choses, beaucoup d'habitats autonomistes fonctionnent sans vérification d'identité. Bien que certaines mesures d'identification, à la fois pour limiter de tricher avec le système de réputation et pour être capable d'identifier les corps en cas de décés, ces utilisation sont significativement plus laxe et peu d'enregistrements sont conservés. Et donc, lorsque les autonomistes et équivalent s'egocast vers des habitats nécessitant une identité, ils en reçoivent une temporaire pour la durée de leur séjour (et parfois pour les visites futures). 

\subsection{Vérification d'identité} 

il y a trois façon de vérifier l'identié de quelqu'un: analyse de nanotatouage, analyse des ondes cérébrales et vérification d'un hachage cryptographique de l'esprit numérique. 

\subsubsection{Analyse de Nanotatouage} 

Des nanbobots spécialement codés sotn utilisés pour créer un petit nanotatouage sur l'index d'une personne. Ces nanobots contiennent de l'information encodées qui inclut le nom et l'identité d'une personne, ses motifs d'ondes cérébrales, ca citoyenneté/son statut légal, son numéro de comtpe, ses informations d'assurance et ses permis. En focntion des lois locales, il peut inclure d'autres informations telles que le casier judiciaire, l'historique des voyages, les implants restreints, l'historique d'embauche, et ainsi de suite. Ce nanotatouage peut-être lu par quiconque ayant un scanner d'identité spécifique et qui peut lire l'encodage des nanobots. Les nanotatouages d'ID incluent des informations sur la société qui a effectué la réincarnation, pour que les données soient accessibles et qu'elles pusisent être vérifiées avec les enregistrements en ligne. Les données sur le nanotatouage sont également signé cryptographiquement avec la clef publique de la société, ce qui signifie que quiconque vérifie les données et la signature en ligne peut déterminer si les données on été altérées. 

\subsubsection{Analyse d'Onde Cérébrale} 

Les analyse d'ondes cérébrales sont l'une des rares empreintes biométriques qui suivent un ego peut importe la morph qu'il occupe. Elles sont inutilisable pour la plupart des objectsif de sécurité car elles nécessitent une analyse avec une combinaisp, électroencépholgramme et des appareils de neuroimagerie, appelé un scnanner d'empreinte cérébrale, ce qui prend en moyenne 5 minutes. Cet apapreil mesure le motif de base des ondes cérébrales ainsi que la signature d'onde cérébrale du sujet en réponse à certaines pensées 

ou à la perception de certains motifs. Ces analyses sont cependant quasiment impossible à tromper, à l'execption du piratage du scanner d'empreinte cérébrale lui-même, et sont donc considéré comme extrêmement fiable. Ils sont donc occasionnellement utilisés dans des installations de haute-sécurité. 

Cela vaut le coup de signaler que l'infection par certaines variantes du virus Exsurgent, notamment la souche Watts-MacLeod (p. 367), peuvent parfois altérer la forme d'onde cérébrale bien que ce ne soit pas systématique. 

\subsubsection{Code Numérique} 

Des codes d'identité numériques sont souvent incorporés dans les sauvegardes et les infomorphs. Cela permet de savoir plus facilement à qui appartient une sauvegarde, mais cela sert aussi de signature électronique pour vérifier une identité lorsqu'une sauveagrde va être réincarné. Le code numérique contient généralement la même information que le nanotatouage d'ID, et est signé par un hash cryptographique qui rend la falsficiation complexe et qui peut être vérifié en ligne. Les IA et les IAG possèden t également de tels code intégrés à leur structure. 

\subsection{Contourner les Vérification d'Identité} 

Les sentinelles de Firewall et les agents clandestins ont souvent besoin de cahcer ou de modifier leurs identités. Bien que les systèmes d'identification soient difficiles à battre, ils ne sont pas insurmontables. 

\subsubsection{Faux ID} 

La méthode la plus simple de contourner les vérificatiosn de sécurité est d'obtenir un faux ID. Étant donné la nature parcellaire des enregistrements d'identité et le manque d'autorité centrale, ce n'est pas spécialement difficile. De nombreux syndicats du crime et même certains groupe autonomistes maintiennent un marché prospère de fabrication d'identité, souvent avec des historiques complets et une couverture médicale pour tous les implants qui peuvent être restreints ou illégaux. 

Ces identités sont générallement enregistrés auprès d'habitats qui sont soit connus pour être des bases criminelles, soit avoir des sympathies autonomistes ou qui sont isolés et distant. Bien que l'identité soit réellement vérifiable et enregistrée auprès de ces stations, les origines potentiellement douteuses de telles identités sont connus de la plupart des autorités du système intérieur et un personnage pourrait donc être exposé à des vérifications ou à une surveillance plus poussée. Des faux ID enregistrés auprès d'autorités plus respectés peuvent être acquis, mais cela nécessite souvent un investissement plus élevé ou ds connexions aux opérations clandestines des hypecroporations. 

bien entendu, les options de réincarnation et de darkcast du marché noir fournissent également de faux ID. 

\subsubsection{Modifier les Nanotatouages d'ID} 

Des traitemenst de nanpmachines spécifiques peuvent être conçus pour effacer, réécrire ou rempalcer les nanotatouages d'ID. Effacer un nanotat est facile, mais ne pas en avoir un est unc rime et déclenches la suspicion immédiate dans beaucoup d'habitat. Réécrire un nanotatouage est également relativement simple, bien que cela imlique que le nanotatouage échouera à sa validation en-ligne sauf si le chiffrement a également été cassée (p. 253). Remplacer un nanotatouage avec un faux est également possible, et fait parti des opérations de récupération de fausse identité. 

\subsubsection{Falsification d'ID Numérique} 

Les codes d'ID numériques peuvent également êre falsifié, bien que, comme pour les nanotatouages d'ID, cela signifiera que l'identité échoueras aux vérifications en-ligne sauf si le chiffrement est également cassé (p. 253). 

\section{La Vie Dans l'Espace} 

La transhumanité n'est pas juste une espèce qui s'étend jusqu'aux confins de l'espae, elle habite essentiellement dans l'espace. Alors qu'une part substantielle de la transhumanité habite les coprs planétaires tels que Mars, la Lune, Vénus, et les lunes des géantes gazeuses, le reste vit dans une variété d'habitat spatiaux, allant des cylindres O'Neill à l'ancienne du système intérieur aux bulles de Cole du système extérieur. 

\subsection{Habitats spatiaux} 

Les habitats spatiaux existent dans de nombreuses tailels et configurations, allant des avant-postes survivalistes conçus pour dix personnes ou moins aux mondes miniatures dans les zones abondantes en ressources et qui hébergent jusqu'à dix-millions de personnes. Dans les régions spatiales fortement peuplées, telles que l'orbite Martienne, les habitats peuvent être intégré dans l'infrastructure locale, dépendant d'une aprtie des expéditions de ressources d'autres instalaltions orbitales. 

Plus courament, et en particulier dans le système extérieur, les habitats sont des entités indépendantes. Cela signifie généralement que, en plus de la principale stations spatial, un habitat est assisté par un tas de structures de soutien, incluant des usines en zéro-g, des rafineries de gaz et de composés organiques volatils, des fonderies, des sattelites de défense et des bases d'exploitation minières. 

Les habitats - et en particulier les plus gros d'entre eux - ont également des visiteurs. Les habitats pricnipaux sont des carrefours spatiaux. En plus des arrêts des arrêts planifiés des gros-porteurs, il peut y avoir des passagers impromptus - tel que les barges racailles, les propsecteur ou les nuées de robots autonomes et désœuvrées. 

beaucoup d'habitat ont une sorte de réseaux de tranports. C'est bien plus commun dans les gros habitats cylindrique avec une gravité centrifuge. Des solutions de transports publiques commune, incluent les trains monorails, les trams et les dirigeable bus. Des options de transport personnels incluent les vélos, scooters, motos et les ultra léger motorisés, les plus gros véhicules étant plutôt rares et générallement réservés à l'utilisation des officiels. 

La pluaprt des habitats ayant de grands espaes internes utilisent également la réalité augmentée pour créer des hallucinations consensuels de ciels et de nuages, sur lesquels la plupart des résident connectent leur canaux RA. On pourrait penser que, dans l'espace, parler de la pluie et du beau temps aurait disparu du répertoire de discussion de la tranhumanité, mais l'habitude persiste - sauf que la météo discutée est généralement virtuelle (si ce n'est une "météo" réel telles que les éruptions et vents solaires). 

\subsubsection{Colonnie en Grappes} 

Les grappes sont la forme d'habitats à microgravité la plus commune. Les grappes sont composées d'un réseau de modules sphériques ou rectangulaires fabriqués dans des matériaux léger et connectés par des canaux. généralement, les modules d'affaires et résidentiels sont regroupés autour des canaux principaux et les modules d'infrastructures tels que les fermes, l'énergie et le recyclage des ordures sont connectés sur des canaux secondaires. Des zones de gravités artificielle et limitée peuvent exister, couramment dans les parcs ou autres endroits publics et dans les modules spécialisés comme les installations de réincarnations (les morphs vivent souvent mieux le fait d'être stocké en gravité). Les principales voies de circulation du cluster peuvent avoir des "voies rapides" où des convoyeurs de boucles d'attaches en mouvement constant accélèrent les personnes qui s'y accrochent. 



Les grappes sont plsu fréquemment trouvés dans les zones riches en composés organiques volatiles comme els Troyens et les systèmes d'anneaux des géantes gazeuses (et Saturne en particulier). Les grappes sont rare dans le système Jovien car protéger une grappe de modules individuels contre la magnétoshpère intense de Jupiter est hideusement inefficace en comparaison de la protection d'une seule station plus grande. 

Les colonies en grappes peuvent avoir entre 50 et 250 000 habitants. 

\subsubsection{Bulles de Cole} 

Les bulles de Cole (ou "mondebulle") sont eseentiellement situés dans la ceintrue d'astéroïde principale, où les gros astéroîde de nickel-fer utilisés pour les construire sont abondant. Les mondebulles sont moins courants dans les Troyens et les Grecs, là où dominent les astéroïdes à croûte gelée. Une bulle de Cole est similaire par beaucoup d'aspect à un cylindre O'Neill, mais ils n'ont pas de fenêtre longitudinale. La lumière du soleil entre pas des jeux de miroirs axiaux. La bulle est également construite de manière trés différente, en utilisant un grand miroir solaire pour chauffer une poche d'eau à l'intérieur de l'astéroïde métallique pour que le métal se dilate. Les astéroîde tournant forcent le matériau malléable à prendre la forme d'un cylindre, dont les extrrmités sont refermés et l'eau est ensuite drainée. L'intérieur peut ensuite être pressurisé, aménagé et occupé. Les bulles de Cole peut également être mise en rotation pour fournir une gravité, en focntion des désirs des habitants, bien que la gravité réduise lorsque vous vous approchez des pôle et qu'elle soit nulle au niveau de l'axe de rotation. 

Les bulles de Cole font parti des plus grosse structures créées par la transhumanité dasn l'espace. L'habitat de Cole le plus grand, Extropia, a une population de 10 millions d'habitants. 

\subsubsection{Cylindre Hamilton} 

Les cylindres hamilton sont une nouvelel technologie. Il n'y a que trois cylindre Hamilton entièrement opérationnels dans le système, mais la conception laisse présager de bonens choses et sera probablement largement adopté dans les années à venir. Les cylindres hamilton se développent en utilisant un algorithme génétique complexe qui orchestre des machines de construction nanoscopiques. Ces nanomachines construise l'habitat lentement et au fil du temps, un procédé qui resemble plus à du jardinage qu'à de la construction. 

De manière similaire aux cylindre O'Neill et aux bulles de Cole, un cylindre hamilton est un habitat cylindrique tournant sur son axe le plus long pour fournir de la gravité. Deux des cylindres hamilton connus orbitent autour de Saturne entre deux anneaux, non loin de la division Cassini. De cette endroit, ils peuvent se repaître des silicates et des composés organiques volatiles en utilisant des vaisseaux moissoneurs. 

Aucune des cylindre hamilton actuellement opérationnels n'a fini sa croissance, mais on estime qu'ils pourraient héberger chacun jusqu'à 3 millions de personne. 

\subsubsection{Cylindres O'Neill} 

Trouvé principalement dans les orbites de la Terre, de la Lune, de Vénus et de Mars, les cylindres O'Neill sont les premiers modèles d'habitats spatial de grande taille et ne sont plus construit, ayant été remplacés par des modèles plus efficace, mais ils abritent toujours des dizaines de millions de transhumains. Les cylindres O'Neill ont étté construits à parti de métaux extraits de la Lune ou de Mercure, de composés organiques volatiles Lunaire (y compris de la glace polaire) et de silicates d'astéroïdes. 

Un habitat O'Neill typique fait trente cinq kilomètre de long, huit de diamètre et tourne autour de son axe long à une vitesse suffisante pour fournir une gravité Terrestre sur les murs internes du cylindre. Des cylindres plus petits existent, bien qu'ils fournissent généralement une gravité réduite (typiquement basée sur les standards Martiens). Les cylindres sont parfois connectés les uns aux autres pour des habitats particulièrement longs. Un spatioport est localisé à l'une des extrémité de l'axe de rotation du cylindre (où il n'y a pas de gravité). Ceux qui arrivent par l'espace utilisent un ascenseeur ou des micro propulsuers pour descendre sur le sol de l'habitat. 



L'intérieur d'un cylindre O'Neill est composé de six bandes alternant entre fenêtre et sol et s'tendant d'une extrémité du cylindre à l'autre. L'une des extrémité du cylindre pointe directement sur le soleil. L'extrémité opposée est le point d'amarrage de trois gigantesque réflecteurs orientés pour refléter la lumière solaire dans les fenêtres. Des matériaux intelligents revêtent les fenêtre et les réflecteurs pour prévenir les fluctuations d'activité solaire et pour ne pas rediriger trop de chaleur. l'ai à l'intérieur du cylindre et de sa superstructure métallique fournissent une protection contre les radiations. 

Le sol dans la plupart des cylindres O'Neill est composé d'un tiers d'agriculture (une combinaison de cuves nourricière et des gigantesque champs photosynthétique), d'un tiers d'espace ouvert et public et d'un tiers d'espace mixte résidentiel/affaires. Les habitats O'neill ont un cycle jour/nuit régulé par la position des mirroirs extérieurs. Les sections d'affaires et de résidences du cylindre est généralement alterné avec les espaces ouverts sur deux bandes de terrain; les terres de cultures occupent habituellement la troisième. Des passerelles traversent les fenêtre tous les kilomètres connectant les bandes de terrain. Le climat interne, le style architectural des structures et le type de faune et de flore présent peuvent varier en fonction des goûts des concepteurs de l'habitat. 

En fonction de leur taille, les cylindres O'Neill puevent héberger de 25 000 à 2 millions de personnes. 

\subsubsection{Boîte de conserve} 

Les stations de recherche antique et les avant-postes de prospecteurs survivalistes correspondent souvent à cette description. Les boîtes de conserves sont à peine plus évoluées que la Station Spatiale Internationale du début du 21° siècle. Elles sont généralement composées d'un modules ou plus, connectés à des panneaux solaires et à d'autres services par un treillis ouvert. les modèles de luxe disposent de canaux ou de voies de circulation entre les modules, alors que les configuration dépouillées nécessitent une exocombinaison ou une morph résitante au vide pour aller d'une pièce à l'autre. Les possibilités d'agriculture sont fortement limitées et il peut ne pas y avoir de lien farcast, mais les fabeurs sont disponible, ainsi que des points d'amarrage pour les navettes et les équipages de prospection. 

Les boîtes de conserve hébergent rarement plsu de 50 personnes. 

\subsubsection{Tores} 

Indifféremment appelés tore, donuts et roues, ces habitats spatiaux ciruclaires étaient une alternative bon marchés aux cylindres O'neill et ont été utilisés poru des installations plus petite. Comme els cylindres O'Neill, les tores ne sont plus construit actuellement, mais ils sont encore nombreux à être en service dans le système intérieur, plus particulièremen en orbite Terrestre et Lunaire. 

Un habitat toroïde ressemble à un donut d'1 kilomètre de diamètre, tournant auatour de grands rayons. Il y a un spatioport en zéro-g au niveau du moyeu de la roue. Les visiteurs prennent un ascensseur le long de l'un des rayons jusqu'au niveau du donut, là où la rotation crée une gravité Terrestre. 

Le plan des habitats toroïdes varient grandement, beaucoup d'entre eux ayant été conçus dasn un but scientifique ou militaire puis reconverti plus tard en habitat par des entrepreneurs ou des squatteurs. Beaucoup ont une succession de ponts dans le donut. La pluaprt d'entre eux ont été conçus comme des habitations pour le long-terme et auto-suffisante et ont donc des fenêtre en verre intelligent permettant de faire pousser des plantes sur une grande partie de la surface interne du tore. Les habitats toroïde équippés pour l'élevage sont géénralement orientés face au soleil perpendiculairement à leur axe de rotation, mais utilise ensuite une lente variation de cet axe pour créer un cycle jour/nuit. 

Les tores ont générallement été construits pour supporter de petits équipages de 500 personnes ou moins, bien que de plus gros existent et soient capables d'héberger 50 000 persones. Quelques rares habitats à double-tore existent, comme deux roues tournant des des snes opposés et connectés à leur axe. 

\subsection{Douanes Et Immigration} 

La façon dont les personnage obtiennent l'accès à un habitat et le type de contrôle qu'ils subiront dépendent de la manière dont ils arrivent. Certains habitats sont proche d'autres abris, alors que d'autres sont physiquement isolés par les vastes distances interplanétaire et le vide spatial. 

Les habitats dans les systèmes planétaire relativement densent, reçoivent la plupart de leurs visites grâce au voyage spatial conventionnel. L'infrastructure des douanes et de l'immigration est orienté vers l'acceuil des visiteurs au spatioport, et par la gestion ds arrivées d'une manière trés proche des aéroports du vingtième siècle. Les habitats isolés, d'un autre côté, tendent à recevoir la plupart de leurs visiteurs par egoscast. 

\subsubsection{Arrivés Physiques} 

Les arrivant par spatioport subissent, au minimum, une vérification d'identité de l'ego, des analyse pour détecter les apathogènes, les nanobots hostiles, les explosifs ou les raidations et une inspection de leurs effets personnels. Certains habitats vont plus loin, et incluent un contrôle secondaire rigoureux, en utilisant des essaim de nanites pour analyser tout le matériel électronique à la recherche de virus et/ou une interrogation agressive d'un fork du sujet. Même les enclaves autonomistes effectuent une analyse automatique à la recherche de tout ce qui pourrait mettre en danger l'habitat ou d'un signe d'efforts de sabotage hypercorporatiste. 

Les marchandises restreintes varient en fonction des lois locales. De nombreux habitats, et en particulier ceux qui sont contrôllés par les autonomistes ou les factions criminelles, autorisent l'armement personnel tant que rien de ce que vous utilisez ne puisse eprcer la coquee de l'habitat ou tuer des douzaines de personnes. D'autres, et notablement la République Jovienne et les stations hypercorporatistes, interdisent les armes léthales de tout type, à l'exception des personnes qui ont acquis des permis et des autorisations sépcifiques (et souvent disponibles en achetants les bonnes personnes ou en obtenant des faverus par la rep). Les armes non léthales sont géénralement utilisées. D'autres objets restreint peuvent inclure les nanofabbeurs, les essaims de nanites, les virus et logiciels de piratage, les drogues et narcorithme, certains type d'enregsitrements XP, les outils d'opérations clandestines, et ainsi de suite. Certians type de morphs peuvent aussi être restreinte, telles que les reapeurs, les furies et les élevés. 

Certains habitats insistent pour que les visiteurs - ou au moins ceux dont ils n'aiment pas la tête - se soumettent à des formes de surveillance ou de supervision spécifique pendant la durée de leur séjour. Cela peut inclure des nuée de nanites marquantes, d'héberger une IA de police dans votre insert de mesh ou même d'être physiquement attaché à un drône de sécurité. D'autres stations vont nécessiter que leurs visiteurs leur laisse un fork en guise d'assurance - au cas où ils commettent un crime, le fork peut-être interrogé. 



Enfin, et bien que rare, certains habitats vont jusqu'à charger tous les visiteurs d'une "taxe aérienne" - une taxe sur l'utilisation des ressources publiquement disponibles pendant qu'ils sont présent. C'est généralement fréquent dans les habitats isolés avec des ressources rationnées et est considéré comme particulièrement odieux par la plupart des autonomistes. 

Certains syndicats font beaucoup d'affaires dans la contrebande de certaines marchandises ou même de personens vers les habitats. Cela est géénrallement accompli grâce à du personnel de sécurit corrompu, mais c'est aussi parfois géré par des autorisations falsifiée qui permettrons au sujet d'ignorer les contrôles de sécurité. De tels services sont généralement chers. 

Pour ceux qui espèrent pouvoir entrer discrètement, il y a toujours l'option de faire une ballade dans l'espace et d'essayer de rentrer par un sas non-surveillé. De tels tentatives sont relativement dangereuses et inutile, la plupart des habitats ont des capteurs et des systèmes de sécurités dédiés à la surveillance de la surface externe et des points d'entrée en particulier. Mais cela reste une possibilité pour une équipe pleine de ressource avec un hackeur compétent, les bots de périmétrique lourdement armés représentant un danger particulier. 

\subsubsection{Electronic Arrivals} 

Arrivals by egocast are sometimes interviewed by habitat authorities in a simulspace before resleeving. Depending upon the habitat's attitude toward civil rights, this process can be relatively reasonable or quite invasive. A minimal entry inspection includes an ID check, a brief interview with a customs AI, and a review of the specs of the morph into which the arriving ego plans to resleeve. Habitats with draconian immigration measures may use harsh psychosurgery interrogation techniques on suspect infomorphs. Egocast backups have little recourse to avoid this treatment—station authorities can simply file them away in cold storage if they choose—so it is wise to investigate custom procedures before you send yourself over. 

Because many people, particularly autonomists and brinkers, don't appreciate this kind of reception, various uploading services have stepped in to provide pre-customs resleeving for characters traveling to habitats with suspect screening methods. For often-exorbitant fees, the traveler egocasts into an extraterritorial substation close to their intended destination, resleeves there, and then travels to their destination by rocket. 

Various darkcast services, normally run by established crime syndicates, sometimes offer an alternative method of egocasting in and possibly even resleeving. Darkcast services are quite expensive, however, and the character is at the mercy of the syndicate operators In rare cases, some political factions or even hypercorps might operate their own darkcast systems, which a character with good networking skills might be able to take advantage of. 

\subsection{Space Travel} 

In some circumstances, characters will prefer to travel physically through space rather than egocasting. In \textit{Eclipse Phase,} spacecraft are primarily dealt with as a setting environment rather than a vehicle/gear to use. Spacecraft largely pilot themselves via the onboard AI. Though characters can also take over with their Pilot: Spacecraft skill, the situation rarely calls for it. 

\subsubsection{Local Travel} 

In densely inhabited planetary systems such as Mars and Saturn, most travel between cities, surface stations, and orbital habitats within 200,000 kilometers is by small hydrogen-fueled (or sometimes methane-fueled) rockets. This form of travel is incredibly cheap, very fast, and avoids the occasional personality glitches that crop up during egocasting. LOTVs (lander and orbital transfer vehicles, p. 348) are commonly used. Spacecraft leaving a planetary body need to be able to generate enough thrust to escape the gravity well (see \textit{Escaping Gravity Wells,} p. 346). 

\subsubsection{Distance Travel} 

For distances of 200,000 to 1.5 million kilometers, somewhat larger (and more expensive) fusion- and plasma-drive craft make regular runs. Nuclear electric ion drives were once used on some of these routes, but the poor efficiency of these fission systems and the need for radioactive heavy metal reaction mass means that they are almost never used anymore. Faster antimatter-drive couriers are also commonly used. These ships lack the thrust to escape from the gravity wells of large planets or moons, so they station themselves in orbit and use smaller ships (typically LOTVs) with higher thrust to transport people to and from the planetary surface. For distances beyond 1.5 million kilometers, almost everyone uses egocasting 

\subsubsection{Space Travel Basics} 

Spacecraft use various types of reaction drives (see \textit{Spacecraft Propulsion,} p. 347), meaning that they burn fuel (reaction mass) and direct the heated output in one direction, which pushes the spacecraft in the opposite direction. Travel over any major distance typically involves a period of high-acceleration burn for several hours at the beginning of the flight, where up to half of the reaction mass is spent to drive up the craft's velocity The ship then coasts for the majority of the flight at that speed, until it approaches its destination, where it flips over and burns an equal amount of reaction mass in the opposite direction to decrease velocity. Though some craft burn half their reaction mass to get up to the best speed possible, this doesn't leave much room for additional maneuvering or emergencies Many craft therefore only burn up to a quarter or a third of their fuel in initial accelerations, so they have some to spare in case they need it. A few tricks can be used to save fuel and build speed, such as slingshotting around the gravity wells of larger planets or aerobraking in a planet's upper atmosphere. Travel times between locations are constantly changing as various bodies move in their orbits around the solar system. Within a cluster or planetary system, travel takes a matter of hours. Within the 286 inner system, travel can take days or weeks. Travel to, from, or within the outer system can take much longer, and is usually a matter of several months. 

Most ships operate at zero-g, except for a few larger craft that are able to spin habitat modules for low gravity. Periods of high-acceleration also produce temporary gravity in a downward direction, towards the burn. 

Space is a valuable commodity on board spacecraft, so room is often tight. Sleeping and personal quarters are rarely bigger than large closets, just enough room for a sleeping bag and personal effects. Depending on the size of the craft, there may be a communal recreation area. The crew tend to only be busy at the beginning and end of a trip, when they must deal with acceleration/deceleration and maneuvering around other space traffic. The rest of the trip they spend dealing with repairs or otherwise killing time, often by accessing XP or VR simulations or playing AR games. While spacecraft have their own local mesh network, they are usually too far to interact with the mesh networks of other habitats without significant communications lag, so they must make do with their own archive of entertainment options. Many long-haul ships are crewed by hibernoid morphs, who hunker down for a long nap. 

\subsubsection{Spaceship Combat} 

Combat in space tends to take place over long distances using massive beam weapons, railguns, and missiles. It also tends to be nasty, brutish, and short. Significant damage to a vessel can cause atmospheric decompression, killing any biomorph crew who aren't suited up and strapped down. 

For the most part, it is recommended that space combat be treated as a plot device, part of the background story that helps create drama and tension, rather than an event that characters actively participate in. This is not to say the characters cannot play a role in the combat, or that their actions will have no effect on the outcome. They may become involved in damage control, negotiate with hostile forces, repel boarders, target weapons with Gunnery skill, stage a mutiny, attempt to hack the networks of approaching vessels, escape out the airlock, hide out while the pirates sack the ship, or similar affairs. It is recommended however, that gamemasters steer clear of space combat situations that could easily lead to the whole team dying due to a few bad dice rolls. 

\section{Nanofabrication} 

In order to create an object in a nanofabricator (whether a cornucopia machine, fabber, or maker; see p. 327), three things are needed: raw materials, blueprints and time. 

\subsection{Raw Materials} 

Raw materials are generally easy to acquire, as most nanofabricators are equipped with disassembler units that will break down just about anything into its constituent molecules. Feedstock may also be purchased (at a cost of Trivial). Many habitats route their recycling and waste products directly into disassemblers. 

\subsection{Blueprints} 

Most nanofabricators are pre-loaded with blueprints for general purpose items: food, simple clothing, basic tools, etc. Blueprints for other goods may be acquired in several ways: 

\begin{itemize} \item They may be purchased online (legally or on the black market). \item They may be found for free online (see below). \item They may be acquired with Rep, following the usual rules for social networking (p. 285). \item They may stolen (usually by hacking a mesh site or a nanofabricator containing such plans). \item They may be self-programmed (see below). \end{itemize} 

\noindent Once the blueprints are acquired, they are simply loaded into the nanofabricator. 

\subsubsection{Open Source Blueprints} 

Blueprints for many goods may be found for free online, disseminated by an active open source software movement. The availability of such plans typically depends on the local mesh. In autonomist habitats, a simple Research Test is likely to turn up the open source blueprints you need (applying modifiers for unusual items). In more restricted habitats, open source blueprints may be harder to find, as they will be securely hidden from the prying eyes of the authorities In this case, the character will need to use their Rep to gain access, bribe a local hacker group, or do something similar. 

Note that restricted nanofabricators may not accept open source blueprints (see \textit{Blueprint Restrictions}). 

\subsubsection{Blueprint Restrictions} 

Some nanofabricators are equipped with pre-programmed restrictions not to accept blueprints for restricted items (such as weapons) or non-licensed items (such as black market or open source blueprints). These restrictions may be circumvented by hacking the nanofabricator and re-programming it, following normal hacking rules (p. 254). 

\subsubsection{Programming Blueprints} 

A dedicated character may simply decide to program their own blueprints, though this is a time-consuming endeavor. To do so, the character must make a Programming (Nanofabrication) Test with a timeframe of one week per cost level of the item. For example, a Trivial cost item takes 1 week, a Low cost item takes 2 weeks, a Moderate item 3 weeks, and so on. Academics Nanotechnology skill or a skill appropriate to the object's design may be used as a complementary skill (p. 173) for this test. A fork or muse may also be assigned to such a programming task. 



\subsection{Time} 

Once the raw materials and blueprints are in, most nanofabrication is simply a matter of time. The exact timeframe to create an object varies, but roughly approximates 1 hour per cost category of the item (1 hour for Trivial, 2 for Low, 3 for Moderate, etc.). The gamemaster may feel free to modify this period as appropriate for the object. 

\subsection{The Programming Test} 

Nanofabrication is typically handled as a Programming Nanofabrication Test. In most cases, this can be treated as a Simple Success Test (p. 118), with a failed roll simply indicating that the item has some minor imperfections, or perhaps took longer to make. 

In some cases, the gamemaster may call for an actual Success Test, meaning that failure is more of a possibility. This should only be done for items that are exotic, extremely complicated, or for which the blueprints are incomplete or otherwise suspect. This test can also be made if the raw materials are limited. 

The character operating the nanofabricator can make this test or it can be left up to the nanofabricator's built-in AI. Most such Such AIs have a Programming (Nanofabrication) skill of 30 (see \textit{AIs and } \textit{Muses,} p. 331). 

\section{Reputation And Social Networks} 

\begin{quote} ``Once upon a time, there was a planet so incredibly primitive that its inhabitants still used money. That planet is called ‘Mars.''' 

—Professor Magnus Ming, Titan Autonomous University \end{quote} 

The conflict between market capitalism and other forms of economics is one of transhumanity's last great culture wars, and it's still being fought. Transhumanity's expansion into the solar system created myriad opportunities to experiment with new economic systems Many failed, but the reputation economies of the outer system have proven both utilitarian and robust in a way that no previous challenger to market capitalism has managed. 

The reputation economy, sometimes called the gift economy or open economy, is one in which the material plenty created by nanofabrication and the longevity granted by uploading and backups have removed considerations of supply versus scarcity from the economic equation—destroying classical economics in the process. 

The regimented societies of the inner system and the Jovian Junta have used societal controls and careful regulation of the technologies of abundance on their populations, thus keeping to a transitional economy system that is largely an outgrowth of classical economics. No one could get away with doing this in the outer system. In the Trojans and Greeks, much of the belt, free Jupiter, and anywhere outward from Saturn, the reputation economy rules. 

How did this happen? For one thing, money is a nuisance when you're an autonomous member of an autonomous collective whose nearest three neighbors (each 100,000 kilometers away) are also autonomous collectives. All of you are almost completely self-sufficient in terms of material resources. You have a fleet of robots that harvest water, volatiles, reactor mass, metals, and silicates. You have a nanofabricator to make all of your small items, a community factory for large ones, and a machine shop where you can build anything else—with help and advice from an AI with the combined knowledge and experience of a top flight engineering team, if you even need it. You grow your own food. 



Money is for people who don't know how to take care of themselves. Transhumanity is only a few decades away from being a mature Type I Kardashev civilization, having largely mastered the material resources of its own solar system. A character from the outer system most likely finds the whole concept of money an embarrassment. 

However, material abundance hasn't eliminated the value of certain goods and services. A transhuman's lunch might be free, but innovative ideas, new designs, health care, sex, and dirty work don't grow in fabricators. What if you need gene therapy on your morph to grow infrared sensing cells on your face? How about someone to assassinate your renegade beta fork after she set off a hallucinogen grenade at your gallery opening and kidnapped your boyfriend? What if you really need a spanking? You call on your social network. If your network is sufficiently deep and numerous, and your reputation is good enough, someone will help you out. 

In the inner system, the reputation economy doesn't replace money for the exchange of goods and services, but it does hold sway over the network of favors and influence. Calling on contacts, getting information, and making sure you're in the best place to see and be seen all involve calling on your social network. 

\subsection{Réseaux Sociaux} 

Social networks represent the people you know, and the people they know, and so on. It starts with your friends and family, spreads out to your co-workers, neighbors, and colleagues, and expands all the way out to your acquaintances, from the neo-hominid waitron at your favorite cafe to the sylph you flirt with at the club. In the always-online, fully-meshed universe of \textit{Eclipse Phase,} this goes even further, encompassing all of the people you've linked to via social mesh networks, everyone who watches your blog/lifelog/updates, and everyone you interact with on various mesh forums. Now add in the friend-of-a-friend factor, and everyone has an impressive ability to reach out to people they know, people they sort of know, and people you don't know but who are somehow linked to you in one degree or another. 

Of course, social networks are not homogeneous. Among the ever-diversifying ranks of transhumanity, there is a tendency to coalesce around various shared characteristics, whether those be cultural background, personal interests, professional ties, local connections, political affiliations, subcultural obsessions, or simply common interest from being part of the same subspecies clade. The social network of an info-anarchist hacker is likely to bear little resemblance to that of a hypercorp socialite or an isolate brinker. Nevertheless social networks quite frequently overlap, often in unexpected and interesting ways. Most people can be considered members of several different types of social networks. This overlap is what links disparate groupings of transhumans together. 

\subsubsection{Networking} 

Just being connected, of course, doesn't mean you have several thousand idle transhumans at your beck and call. If you hope to gather the latest gossip, get advice from an expert, find someone who can fix your problems acquire a piece of gray market tech, or spread a meme, you need to know both \textit{who} to talk to in that social network and \textit{how} to go about getting what you need, especially if you hope to keep things quiet and not raise any flags. 

This is where your Networking: [Field] skills come in (p. 182). Networking represents your ability to maneuver through this web of personal and impersonal connections to find who and what you need. This could be handled by word-of-mouth, posting the right queries in the right places on the mesh, monitoring the right personal profiles and forums, harnessing the power of the mob with crowdsourcing, or any number of similar creative tactics. 

Each field you have in Networking represents a particular network grouping, a common interest that ties people together. Most of these fields are based on factions (Autonomists, Hypercorp, etc.) and tie into a special reputation network (see the Reputation Networks table, p. 287). At the gamemaster's discretion other groupings of people could be connected through these skills and rep systems. For example, artists and journalists of all stripes can fall under the Networking: Media skill and f-rep, no matter if they are autonomist or hypercorp. Likewise, being a diverse group, brinkers do not universally fall into any of the categories, and are instead spread out between them. If the gamemaster and players agree, other Networking fields and rep networks may be added, representing other spheres of interest, such as AR Games, Sports, Slash Fiction, etc. 

The exact uses for which you may exploit your social networks are noted below. While in some cases the defining element is who you know and how good you are at reaching out to them, in others the defining element is how known \textit{you} are. You might be connected to thousands of people, but if you don't have clout, your efforts to make use of these connections is limited. This is where Reputation comes into play. 

\subsection{Reputation} 

Reputation is a measurement of your social currency. In the gift economies of the outer system, social reputation has effectively replaced money. Unlike credit, however, reputation is far more stable. 

Within \textit{Eclipse Phase,} reputation scores are facilitated by online social networks. Almost everyone is a member of one or more of these reputation networks. It is a trivial matter to ping the current Rep score and history of someone you are dealing with—your muse often does this automatically, marking an entoptic Rep score badge on anyone with whom you interact, updated in real time, so you will see if they suddenly take a hit or become popular. The 7 most common networks are noted on the Reputation Networks 

table. Gamemasters and characters may decide to add others appropriate to their game. You purchase a Rep score in one or more of these networks during character creation. Rep scores are rated between 0 and 99, just like skills. These ratings determine your ability to acquire goods, services, and information and favors, as noted below. These scores may be raised or lowered during game play according to your character's actions. 

\subsection{Using Networks And Rep} 

In game terms, you take advantage of your connections and personal cred every time you need a \textit{favor.} A favor is broadly defined as anything you try to get via your social networks, whether that be information, aid, goods, and so on. Different types of favors are described under \textit{Favors,} p. 289. 

\subsubsection{The Networking Test} 

To pursue a favor, you start by looking around. This calls for a Networking Test to determine if you can find the person, people, or information you need. This represents talking to people you know, spreading the word to people they know, posting queries to the social network at large, digging through various profiles, chat rooms, etc. to find someone who might help you out, and so on. Networking Tests are subject to modifiers for the level of the favor (see below), the amount the character is trying to keep quiet about the request (see below), and any other factors noted on the Networking Modifiers table or determined by the gamemaster. Networking Tests are Task Actions—it takes time to call in favors or track down information. The timeframe depends on the level of favor, as noted on the Favors table, p. 289. 



\begin{table} \caption{Reputation Networks} \begin{tabularx}{\textwidth}{|X|l|l|X|} \hline



Network name &Rep name &Networking field &Factions and others \\ \hline

The Circle-A List &@-Rep &Autonomists &anarchists, Barsoomians, Extropians, Titanian, and scum \\ \hline

CivicNet &c-Rep &Hypercorps &hypercorps, Jovians, Lunars, Martians, Venusians \\ \hline

EcoWave &e-Rep &Ecologists &nano-ecologists, preservationists, and reclaimers \\ \hline

Notoriété &f-Rep &Media &socialites (also artists, glitterati, and media) \\ \hline

Guanxi &g-Rep &Criminals &criminals \\ \hline

The Eye &i-Rep &Firewall &Firewall \\ \hline

Research Network Associates &r-Rep &Scientists &argonauts (also technologists, researchers, and scientists) \\ \hline

\end{tabularx} \end{table} 

\subsubsection{Favor Levels And Modifiers} 

Rep scores are broken down into five levels, reflecting your standing within that community. Every 20 points of Rep equals one level. See the Reputation Levels table for a breakdown. 

Likewise, favors are also broken down into five levels, rated from Trivial to Scarce (see \textit{Favors,} p. 289, for specific examples). The standard level of favor you can expect to get from a social network is based on your level of Rep in that network. If you want to pursue a favor above your level, you can do so, but you will suffer a negative modifier on your Networking Test. This reflects that someone with low standing has a hard time getting people to go out of their way for them. Similarly, if you pursue a favor below your level, you receive a positive modifier to your Networking Test, reflecting that your prestige makes it easier to acquire minor things that you need. For each level the favor falls under or above your Rep score level, apply a + or –10 modifier, as appropriate. 

\begin{table} \caption{Networking modifiers} \begin{tabular}{|l|l|} \hline

SITUATION &MODIFICATEUR\\ \hline

Favor level exceeds Rep level &–10 per level \\ \hline

Rep level exceeds favor level &+10 per level \\ \hline

Keeping quiet &–Variable (see p. 288) \\ \hline

Burning Rep &+Rep amount burned \\ \hline

Paying extra &+10 per level \\ \hline

\end{tabular} \end{table} 

\begin{table} \caption{Reputation Levels} \begin{tabular}{|l|l|} \hline

REPUTATION SCORE &REPUTATION LEVEL \\ \hline

0–19 &Levek 1 \\ \hline

20–39 &Levek 2 \\ \hline

40–59 &Levek 3 \\ \hline

60–79 &Levek 4 \\ \hline

80–99 &Levek 5 \\ \hline

\end{tabular} \end{table} 

\begin{quotation} Jaqui’s on a scum barge and she needs to get a hold of a weapon fast. She has a specific weapon in mind, but it’s pricey—its cost is High. She decides her best approach is to try talking to the scum on the ship to try and find someone who can lend or sell her such a weapon, using her @-rep and her Networking: Autonomist skill of 50. Acquiring a High cost item counts as a Level 4 High favor (see Acquire/Unloads Goods, p. 289). Jaqui’s @-rep is 53, which is only Level 3. Since the favor is one level higher than her rep level, she suffers a –10 modifier on her Networking Test. Jaqui must roll a 40 or less (50 – 10) to find a weapon supplier. \end{quotation} 





\subsubsection{Paying/Exchanging For Favors} 

Favors don't necessarily come for free. Depending on what you're after, you may also need to exchange for it. 

In the capitalist and transitional economies of the inner system and Jovian Junta, you may need to buy the goods or services you are after with credit. Even information might be paid for by bribing the right person. Once spent, that credit is gone until you earn more. 

In the anarchistic reputation economies of the outer system, you can get what you need for free. In this case, you are acquiring goods and services based on the strength of your reputation. 

\begin{quotation} Jaqui rolls a 39—she makes it! After posting some public notices on the scum social network (she’s not worried about legalities or hiding what she’s doing—this is a scum ship after all), she gets directed to a weapons dealer with a good rep. While a scum arms merchant normally sells their wares for credit, Jaqui is scum herself, so she’s able to use her scum community standing and get the weapon for free. This uses up a High favor, however. \end{quotation} 



\subsubsection{The Limits Of Reputation} 

Even in the gift economies, reputation only gets you so far. There are limits to how often you can ask for help before you start coming across as pushy or a leech. In game terms, this is expressed as a \textit{refresh rate}—the amount of time you must wait to pass before you can seek out a favor of that level again without seeming demanding. Refresh rates are noted on the Favors table (p. 289). 

If you need to seek another favor before the refresh rate has expired, you have two choices. You can expend a higher level favor instead, keeping in mind that higher level favors refresh more slowly. Alternatively you can burn reputation (see below). 

\begin{quotation} Now that Jaqui’s got her weapon, she needs another favor—she needs to find someone who doesn’t want to be found. The person she’s after is scum, so once again she turns to the scum for help. The gamemaster decides that this is another Level 4 favor (see Acquire Information, p. 291). Once again, with her Networking: Autonomist of 50 and Level 3 rep, she must roll a 40 or less. She gets a 21, and finds someone who has the information she needs. Jaqui now has a choice. To get this information, she either needs to pay the person in credits (a High cost) or she she needs to expend another Level 4 favor. She’s low on money, so she decides to use her rep again. Level 4 favors only refresh once a month, though, and Jaqui used her last one just a few hours ago. Her only choice is to expend a higher favor, so she expends a Level 5 to get the intel she needs. \end{quotation} 



\subsubsection{Burning Reputation} 

In some cases, getting what you need may be more important than not stepping on people's tentacles. In situations of dire need, you can \textit{burn }some of your Rep score to get the job done, meaning that you exchange a loss of Rep for a shot at a favor. This reflects that you are pushing the bounds of how far people are willing to go for you. While you still might get what you need, your online reputation rating takes a hit as people flag you for being needy. 

There are two reasons to burn Rep score. The first is to get a bonus on your Networking Test. This indicates that you are pulling strings and calling in markers to get the favor you're after. This is particularly useful when you are trying to obtain a favor that's of a level higher than your Rep, but abuse it too often and you will soon have no social standing at all. Every point of Rep you burn gives you an equivalent positive modifier on the Networking Test, up to a maximum of +30. 

The second option is to burn Rep to seek a favor before it has refreshed. This reflects that you are asking for too much in a short period. The amount of Rep you must burn in this case depends on the level of favor you are seeking, as noted on the Favors table (p. 289). 

\begin{quotation} Jaqui’s got her weapon and her target’s whereabouts, but she needs one more thing: a hacker. She needs someone who can open some doors and defeat some security systems so she can get to the target she’s after in his hideout. Since she’s on a scum barge, Jaqui feels that, once again, her best option is to work her scum contacts. The gamemaster determines that this will be another Level 4 favor. Rolling against a target number of 40 again, she gets a 13—her luck is holding. She finds a hacker, but now she needs to make an exchange for their services. Once again she decides not to spend credit and use her @-rep instead. Jaqui’s already used up both her Level 4 and Level 5 @-rep favors, though, so she has no choice but to burn reputation. A Level 4 favor costs 10 Rep to burn. Jaqui spends it, sending her @-rep from 53 to 43—she’s been pulling in a lot of big favors in a short amount of time, and her friends and acquaintances are expressing their annoyance by lowering her social standing. \end{quotation} 

\subsubsection{Keeping Quiet} 

The problem with using social networks for favors is that you end up letting lots of other people know what you're up to. When you're involved in a clandestine operation, that could be exactly what you \textit{don't} want. The only way to diminish this is to take your requests to trusted friends and ask them to keep quiet, but this diminishes the pool of people at your disposal. 

In game terms, you can try to keep word of what you're doing quiet, but this makes it harder to get what you need. For every negative modifier you apply to your Networking Test, the same negative modifier applies to anyone making a Networking Test to find out what you're up to. 

\begin{quotation} Revisiting one of our previous examples, we go back to the point where Jaqui was trying to ascertain someone’s hideout location. Because the person she’s after is scum, they’re on a scum ship, and Jaqui is using her Networking: Autonomist skill to find them, there’s a good chance that if she starts asking around to everyone, word might trickle back to the person she’s after. She doesn’t want them to know she’s on their tail, though, so she decides to make her inquiries more discreet. She applies a –20 modifier to her Networking Test, which lowers her target number from 40 to 20. As noted before, she rolls a 21, which is a failure. She spends a Moxie point to flip the roll, though, making it a 12—a success. Because Jaqui took that –20 hit, representing the fact that she was keeping her research quiet, her target will suffer a –20 modifier when he makes his Networking Test to see if he gets word that someone is asking around about his hideout. \end{quotation} 

\subsection{Favors} 

Creative players can undoubtedly come up with many uses for their social networks, but a few of the more common are detailed here. Gamemasters should use their discretion as to how much roleplaying interaction and Networking Tests are included in using a social network. For normal goods, straightforward information queries, or small favors, neither dice rolling nor roleplaying may be required For major requests, interactions with contacts, and mission assistance, dice rolls and/or roleplaying interaction with contacts from the social network should usually occur. Gamemasters may wish to keep track of the NPC contacts in each character's social networks and make them recurring characters. 

\begin{table} \caption{Favors} \begin{tabular}{|l|l|l|l|} \hline

FAVOR LEVEL &TIMEFRAME &BURNING REP COST &REFRESH RATE \\ \hline

1 (Trivial) &1 minute &0 &1 hour \\ \hline

2 (Low) &30 minutes &1 &1 day \\ \hline

3 (Moderate) &1 hour &5 &1 week \\ \hline

4 (High) &1 day &10 &1 month \\ \hline

5 (Scarce) &3 days &20 &3 months \\ \hline

\end{tabular} \end{table} 



\subsubsection{Acquire/Unload Goods} 

Social networks are a good way to find items that you can't buy legally or make at home. Depending on who you're getting the goods from, this will cost you credit or require an appropriate Rep score. This favor can also be used to sell or give away such items, making some money or perhaps even some Rep in the process. 

\begin{table} \caption{Acquire/unload goods} \begin{tabular}{|l|l|} \hline

LEVEL &SERVICE \\ \hline

1 &Acquire/unload item with an expense of Trivial. \\ \hline

2 &Acquire/unload item with an expense of Low. \\ \hline

3 &Acquire/unload item with an expense of Moderate. \\ \hline

4 &Acquire/unload item with an expense of High. \\ \hline

5 &Acquire/unload item with an expense of Expensive \\ \hline

\end{tabular} \end{table} 



\subsubsection{Acquire Services} 

When you lack the skills or education you need, or you just need another set of arms, you can call out to your social network to find someone to help you out. If you are looking for someone with a particular skill, the result of your successful Networking Test roll is the skill rating of the person you find. The higher your Networking skill, the better able you are to find highly-skilled professionals. 

\begin{quotation} Cole needs to find an astrobiologist who can help him identify an alien critter. He rolls his Networking: Scientist skill of 50 and gets a 43—a success. He tracks down someone with Academics: Astrobiology skill of 43 (his roll) who can help him out. When the astrobiologist looks the critter over, the gamemaster makes a roll for the NPC using that skill of 43. \end{quotation} 



\subsubsection{Acquire Information} 

When you can't find the information online or you don't have the time or capability to look, you can turn to people in your social network and tap their accumulated knowledge base. 

\subsection{Reputation And Identity} 

It is important to note that reputation is closely tied to identity. If you are undercover and using a fake ID, you can't really call on your Rep score without giving yourself away. As a result, many people using false identities end up building up a separate set of Rep scores for their alter ego. Note that since many social network interactions take place online, it is possible for someone to secretly make use of their real identity while masquerading as someone else, as long as they're careful about it. If anyone happens to be spying on their activity via the mesh, they stand a chance of being found out. 



\begin{table} \caption{Acquire Services} \begin{tabularx}{\textwidth}{|l|X|} \hline

LEVEL &SERVICE \\ \hline

1 &\textbf{Trivial favor}: Get someone to perform services for 15 minutes. Move a chair. Browbeat someone. Catch a ride. Research someone online. Borrow 50 credits. Other Trivial cost services. \\ \hline

2 &\textbf{Minor favor}: Get someone to perform services for an hour. Move to a new cubicle. Rough someone up. Loan a vehicle. Provide an alibi. Healing vat rental. Minor hacking assistance. Basic legal or police assistance. Borrow 250 credits. Other Low cost services. \\ \hline

3 &\textbf{Moderate favor}: Get someone to perform services for a day. Move to a habitat in the same cluster. Serious beating. Lookout. Short-distance egocast. Short shuttle trip (under 50,000 km). Minor psychosurgery. Uploading. Reservations at the best restaurant ever. Major legal representation or police favors. Borrow 1,000 credits. Other Moderate cost services. \\ \hline

4 &\textbf{Major favor}: Get someone to perform services for a month. Move a body. Homicide. Getaway shuttle pilot. Industrial sabotage. Large-volume shipping contract on bulk freighter. Medium-distance egocast. Mid-range shuttle trip (50,000–150,000 km). Moderate psychosurgery. Resleeving. Get out of jail free. Borrow 5,000 credits. Other High cost services. \\ \hline

5 &\textbf{Partnership}: Get someone to perform services for a year. Move dismembered body. Mass murder. Major embezzlement. Acts of terrorism. Relocate a mid-size asteroid. Long-distance egocast. Long-range shuttle trip (150,000 km or more). Borrow 20,000 credits. Other Expensive cost services. \\ \hline

\end{tabularx} \end{table} 





\begin{table} \caption{Acquire Services} \begin{tabularx}{\textwidth}{|l|X|} \hline

LEVEL &SERVICE \\ \hline

1 &\textbf{Common Information}: Where to eat. What biz a certain hypercorp is in. Who’s in charge. \\ \hline

2 &Public Information: Make gray market connections. Where the “bad neighborhood” is. Obscure public database info. Who’s the local crime syndicate. Public hypercorp news. \\ \hline

3 &\textbf{Private Information}: Make black market connections. Where an unlisted hypercorp facility is. Who’s a cop. Who’s a crime syndicate member. Where someone hangs out. Internal hypercorp news. Who’s sleeping with whom. \\ \hline

4 &\textbf{Secret Information}: Make exotic black market connections. Where a secret corp facility is. Where someone’s hiding out. Secret hypercorp projects. Who’s cheating on whom. \\ \hline

5 &\textbf{Top Secret Intel}: Where a top secret black-budget lab is. Illegal hypercorp projects. Scandalous data. Blackmail material. \\ \hline

\end{tabularx} \end{table} 







\section{Security} 

Firewall sentinels make a regular habit of being in places where they are not supposed to be and bringing things with them that others would prefer they not have. Security has a different character post-Fall than in the 21st century. Due to hyper-abundance, physical security measures such as locks, doors, and walls are less important than in the past to common citizens. People don't worry about theft as much as in the past because most items can be replaced by a nanofabricator. The items that do tend to engender this type of security are irreplaceable or rare items such as artifacts of Earth. 

Post-Fall physical security focuses heavily on surveillance—identifying intruders and tracking them so that they can be interdicted by transhuman or robotic defenders. Surveillance is more effective than in pre-Fall societies because AIs with near-human faculties of pattern recognition and indentured infomorphs can be employed to monitor surveillance data. 

The emphasis on surveillance results from the ease with which most material barriers can be breached by high-powered hand weaponry and devices like the covert operations tool (p. 315). However, physical barriers designed to actively resist intruders by healing themselves or attacking tools used to damage them are used at key points in secure installations. Such barriers are typically very expensive and so are used sparingly. 

Transhuman, animal, and infolife defenders are cornerstones of most security systems. The availability of a huge pool of infomorph labor to guard facilities means that someone is always on duty, whether as part of the surveillance system or in a robotic shell. 

\subsection{Access Control} 

The first step in any security system is simply to enact measure to keep unwanted people out. At a basic level this involves walls, locks, fencing, defensive landscaping security lighting, and entoptic warnings. 

Barriers of different sorts present an obstacle that must be cut through or blown apart in order to defeat. Barriers are treated just like other inanimate objects for purposes of attack sand damage; see \textit{Ob-} \textit{jects and Structures,} p. 202. 

\subsubsection{Bug Zappers} 

Bug zappers create minute EMP pulses that are harmless to most electronic equipment and implants but wreak havoc on nanobot swarms, microbugs, and specks. Bug zappers are generally applied to surfaces, and as such they only destroy floating/flying swarms or specks if they land. In areas with heavily shielded electronics, they may be installed to destroy targets in an entire room. A zapper instantly destroys all free-crawling or flying nanobots and specks in a room when it goes off, but transhuman flesh is sufficient to prevent it destroying medichines or other implanted nanobots. Infiltrators trying to gather data in areas protected by zappers generally resort to going around them or trying to plant macroscale devices. 

\subsubsection{Electronic Locks} 

Electronic locks (e-locks) are commonly used as a means of maintaining privacy. They are easy to defeat, however, and so are rarely used in very secure areas. E-locks have several advantages over old-fashioned mechanical locks. Different users can have different authentication methods, they can log all events (entry, exit, failed authentications), and they can be connected (usually hardwired but sometimes encrypted wireless) to security systems for remote control and alarm triggering. 

E-locks use one of several authentication systems, or sometimes a combination of systems: 

\textbf{Biometric:} The lock scans one or more of the user's biometric prints. Common biometrics include DNA, facial thermographic, fingerprint, gait, hand veins, iris, keystroke, odor, palm, retinal, and voice prints. 

\textbf{Keypad:} This is an alphanumeric keypad upon which users enter a specific code. Different users can have different codes. 

\textbf{Token:} Authorized users must carry some sort of physical token that interacts with the lock to open the door, such as a keycard, electronic key, etc. 

\textbf{Wireless Code:} Users must emit a cryptographic code via near-proximity wireless signal. 

Though various technologies exist to defeat each of these systems, there are three methods that work against almost all e-locks. The first is use of a covert operations tool (p. 315), which infiltrates a lock with nanobots that swarm in and engage the electronic mechanism. The drawback to using a COT is that its use is immediately logged by the e-lock and an alarm is triggered. Some e-locks are equipped with guardian nanoswarms (p. 329) to defeat COTs, but the COT nanobots usually manage to open the lock before the guardians eat them. 

The second method is to hack the e-lock. Most e-locks are slaved to a security system, so this often means intruding into the security system and then opening the lock from within. This can be difficult, however, especially if the security system is wirelessly isolated or hardwired. The advantage is that, if done right, all evidence of the lock being opened can be erased. The third method is to physically open and manipulate the lock. This requires first opening the lock's case and then triggering the lock mechanism to open the door. Both of these are handled as separate Hardware: Electronics Task Actions with a timeframe of 1 minute each. In addition, most e-locks have anti-tamper circuits that will set off an alarm if the attacker does not achieve an Excellent Success when opening the case. 





\subsubsection{Lockbots} 

The 21st century saw a move from mechanical locks to e-locks and other largely electronic locking mechanisms. These devices worked well for about 50 years, until electronic infiltration capabilities rendered them largely useless. The more recent development of lockbots has more in common with their early mechanical forebears. They are unique, expensive, artisan items. 

A typical lockbot is heavily integrated with the portal and barrier it protects. Lockbots usually include an AI or indentured infomorph, self-healing materials (treat as a self-healing barrier), and a swarm of guardian nanobots (p. 329). A lockbot monitors its surroundings and has visual recognition software that knows what its users and its keys look like (Perception skill 40). Picking a lockbot is thus incredibly difficult, because it will shut its orifice and not accept a key that doesn't look right or that comes from an unrecognized user. Unfamiliar nanobots trying to enter the orifice are targeted and destroyed by the guardian nanobots. Finally, external tools used to harm the portal or the lock will be attacked by fractal appendages extruded from the portal surface or the lock itself. These appendages have a range of 1 meter, attack with skill of 40, and inflict 1d10 +2 DV. 

Lockbots are generally immune to being hacked because, for security, they aren't connected to the mesh. If attacked, however, lockbots are programmed to send out an alarm signal via the mesh. 

There are several ways to defeat a lockbot. One is to get a copy or image of the key and then forge a copy (using nanofabrication). Another is to attack the lockbot or the portal it guards with so much force that the lockbot is unable to repair it (usually using ranged weapons, as anything within a meter of a lockbot may be counterattacked). A third is to somehow image the cavity beyond the lockbot's orifice without the imaging device being destroyed and to then forge the key. All of these are difficult and time-consuming processes. 

Some lockbots have the ability to destroy what they're protecting. For example, lockbots are a common protection for the physical interfaces to hardwired networks. If the lockbot is compromised, it may, as a last resort, destroy the interface it was protecting. 

\subsubsection{Portal Denial System} 

Installed in corridors or doorways, this is essentially a laser trap device. When an unauthorized person enters the portal denial system's area, it uses lasers to create a grid of plasma channels that are used to deliver a powerful electric current to the target. This system has both lethal and nonlethal settings. 

\textbf{Nonlethal:} 1d10 DV + shock (p. 204) 

\textbf{Lethal:} 2d10 +5 DV 

\subsubsection{Self-Healing Barriers} 

Walls and doors that are able to rapidly repair themselves are sometimes found in high security installations These barriers are made of materials that automatically expand to ``heal'' small holes and that are equipped with nanosystems that slowly repair larger amounts of damage. The best of these barriers do no more than slow down the most determined assailants, but in combination with surveillance systems they are a nuisance to invaders and can slow down attempts to flee the scene. 

Self-healing barriers heal any single source of damage that is less than 5 points of damage almost immediately, sealing the hole in 1 Action Turn. They will also seal the holes inflicted by a covert ops tool (p. 315) in the same time period. Additionally, these barriers repair larger themselves at the rate of 1d10 damage per 2 hours; once all damage is fixed any wounds are repaired at the rate of 1 per day. Damage of 3 wounds or more may not be repaired by self-healing. 

\subsubsection{Slippery Walls} 

On planetary surfaces, high walls and fences are still common as a first line of defense against interlopers Slippery walls are surface treated with the slip chemical (p. 323), creating a virtually frictionless surface that is exceptionally difficult to climb. 

\subsubsection{Wireless Inhibitors} 

Wireless inhibitors are simple paint jobs or construction materials that block radio signals. They are used to create a contained area in which a wireless network may operate freely without worry that the signals will escape out of the area, where they can be intercepted. Wireless inhibitors allow the convenience of using wireless links within a secure area rather than the clumsier hardwired connections. If an intruder manages to gain access inside the area, however, they can intercept, sniff, and hack wireless devices as normal. 

\subsection{Detection And Surveillance} 

Should security measures fail to keep an intruder out, the second step is to detect an interloper and track their activity. 

\subsubsection{Nanotagging} 

A lot of post-Fall security centers not around keeping people out of private spaces, but tracking them after they come and go. What little privacy transhumans have, they cherish. Trespassing is a worse offense than theft in many places. 

A room protected by a taggant nanoswarm (p. 329) usually has two or more hives, one each at floor and ceiling level (if in gravity; on the opposite side of the room if in microgravity) that generate and recycle nanobots. The taggants emerge from one hive, float through the room, and then return to the other for recharging and reuse. A feed line usually connects the 

hives so that they can share materials and power. 

Anyone passing through the room is likely to be dosed with taggant nanobots. Once they lose proximity to the rest of the hive, they hide and periodically broadcast pulsed transmissions meant to give their position to pursuers or investigators. Some may drop off in clusters to form a breadcrumb trail to the interloper. 

\subsubsection{Sensors} 

Any of the various sensors described in the \textit{Gear} chapter (p. 294) may be deployed within a facility to monitor and record the passage of personnel, both authorized and not. These sensors are typically slaved to the facility's security network and closely monitored by security AIs, meaning they are vulnerable to hacking and possibly jamming. A few other sensor types deserve mention here: 

\textbf{Chemical Sniffers:} The chem sniffer described on p. 311 can also be set to detect the carbon dioxide exhaled in transhuman breaths. This is useful for detecting intruding biomorphs in areas that are abandoned/off-limits. 

\textbf{Electrical Sensors:} Electrical sensors can be set in portals to detect a biomorph's electromagnetic field in addition to the electrical fields of synthmorphs. 

\textbf{Heartbeat Sensors:} These sensitive sensors detect the vibration caused by transhuman heart beats. They can even be used to detect the heartbeats of passengers inside a large vehicle. 

\textbf{Seismic Sensors:} Embedded in flooring, these sensors pick up the pressure and vibration of weight and movement. 

\subsubsection{Weapon Scanners} 

Weapon scanners come in several varieties, including those that look for the rare elements used in extremely destructive weapons such as nukes, those that attempt to locate personal weaponry, and those that look for detection taggants. 

Rare element scanners are nearly flawless and are ubiquitous in habitat customs and spaceports. The only way to circumvent them is to find an alternate route into the protected area. 

Personal weapon scanners can monitor a specific area, such as a small room or doorway. They use a number of sensing systems to detect and identify weapons and other dangerous objects, including chemical sniffers and radar/terahertz/infrared/x-ray/ultrasound imaging. They can detect the following items and substances: 

\begin{itemize} \item Metal used in kinetic weapons, seekers, and flechette weapons \item Devices with onboard hives of metallic nano- bots (e.g., covert operations tools, spindles) \item Magnetic elements in plasma guns and railguns \item Propellant from firearms ammunition and seekers (–30 to conceal) \item Chemical fuels used in torch spray weapons (–30 to conceal) \item All explosives and grenades by their chemical particulate emissions (–30 to conceal) \item Poisons and bioagents in flechette weapons \item Any weapon or device larger than palm size (using sound waves and shape recognition) \end{itemize} 

Characters trying to sneak weapons and gear past personal weapon scanners must make a Palming Test (if concealing) or an Infiltration Test (if somehow maneuvering around without notice). This is opposed by a Perception Test from the character or AI manning the sensor system. 

\subsubsection{Analyse Sans-fil} 

Some high-security areas will intentionally monitor for wireless radio signals originating within their area as a way of detecting intruders by their communications emissions. These signals can even be used to track the intruder's location via triangulation and other means (see \textit{Physical Tracking,} p. 251). To bypass wireless detection systems covert operatives can use line-of-sight laser links (p. 313) for communication or touch-based skinlinks (p. 309). 

\subsection{Contremesures Actives} 

When all else fails, active countermeasures may be deployed against intruders. While live transhuman guards are sometimes used, robotic sentries are more common, typically AI-driven synthmorphs such as synths, slitheroids, arachnoids, or reapers, with guardian angels (p. 346) providing air support. Occasionally AI-operated gun emplacements—armored turrets that pop out of walls and ceilings—are also applied. In some circumstances, these shells are teleoperated or even jammed by transhuman security. 

Additional countermeasures brought to bear will depend on the facility in question. Some sites will engage in active jamming, to deny the intruders any communication. Others will deploy hostile nanoswarms and even chemical weapons. 