



\chapter{Gear} \label{cha:gear} 

The accelerated technological levels of \emph{Eclipse Phase} enable a number of devices for personal enhancement, survival, and other uses. 



\section{Equipment rules} \label{sec:equipment-rules} 

The following rules apply to all technological items in \emph{Eclipse Phase}. 



\subsection{Acquiring gear} \label{sec:acquiring-gear} 

During character creation, players purchase gear for their characters using the credits they have during the character creation process. Once play begins, however, characters must obtain any equipment they need the usual way: by buying, borrowing, making, or stealing it. 

In the inner system, hypercorp, and Jovian Republic settlements --- and other places where capitalism still reigns --- gear acquisition is simply a matter of finding a seller and buying it. Each item has a listed cost, from Trivial to Expensive, as noted on the Gear Costs table. Due to local availability of resources, supply and demand, and legalities, these listed costs are meant to be approximations. When no other factors apply, the listed Average Cost for that category can be used. Otherwise the gamemaster should modify the item’s worth as they see fit, according to local economic factors, while still keeping it within that cost category range. The Cost Modifiers table lists out some suggested changes to an item’s cost, but these are simply recommendations, and can be ignored or followed as the gamemaster deems fit. The exact local conditions are largely up to the gamemaster to determine, as best fits their game. 

In some circumstances, characters may attempt to haggle over gear prices. This is best handled as roleplaying, but the gamemaster may also call for an Opposed Persuasion Test (or possibly an Intimidation Test). The character who wins may increase or reduce the price by 10\% per 10 points of MoS. 

In the outer system, anarchist, Titanian, scum, and other habitats that use the reputation economy, characters must rely on their rep scores to acquire the goods and services they need. The mechanics for this are covered under \emph{Reputation and Social Networks}, p. 285. 

Characters are of course free to get their hands on equipment by any other means they devise --- con schemes, borrowing from friends, and outright theft, with all of the appropriate tests and consequences. In some cases, acquiring gear may be an adventure unto itself. 

\begin{table} \begin{tabular}{|l|l|l|} \hline

\hline{3}{|c|}{\textbf{Gear costs}}	\\ \hline

\textbf{Category}	&\textbf{Range (in credits)}	&\textbf{Average (in credits)} \\ \hline

Trivial	&1-99	&50 \\ \hline

Low	&100-499	&250 \\ \hline

Moderate	&500-1499	&1000 \\ \hline

High	&1500-9999	&5000 \\ \hline

Expensive	&10000+	&20000 \\ \hline

\label{tab:gear-costs} \label{tab:gear-costs} \end{table} 

\begin{table} \begin{tabular}{|l|l|} \hline

\hline{2}{|c|}{\textbf{Gear cost modifiers}}	\\ \hline

\textbf{Economic factor}	&\textbf{Suggested cost modifier} \\ \hline

Item Stolen	&-50\% \\ \hline

Item Used	&-25\% \\ \hline

Item Restricted	&+25\% \\ \hline

Item Illegal	&+50\% \\ \hline

Item Scarce	&+25\% \\ \hline

Item Extremely Rare	&+50\% \\ \hline

Item Common	&-25\% \\ \hline

\label{tab:gear-cost-modifiers} \label{tab:gear-cost-modifiers} \end{table} 

\subsubsection{Fabricating gear} 

Thanks to nanofabrication technology, characters may also create their own equipment using cornucopia machines and similar nanofab devices (p. 327). The character must have the appropriate blueprints to do so, whether they come with the fabber, are bought legitimately or on the black market, acquired with rep, or found online. Characters may also code their own blueprint desires, using the Programming: Nanofabrication skill. 



\subsection{Gear modifiers} \label{sec:gear-modifiers} 

In the technological future, gear is a necessity. In many cases, use of equipment provides no bonuses, it simply allows a character to perform a task they would otherwise be unable to do. For example, it is impossible to pick a mechanical lock without lockpick or some sort of tool. 

In other cases, however, gear provides a bonus to the task at hand. Climbing a wall may be possible without tools, but if you happen to have gecko gloves or other climbing gear, it’s going to be a lot easier. The specific modifier applied is usually noted in the gear item’s description, typically ranging from +10 to +30. 

\subsubsection{Gear quality} 

In both of the situations above, it is possible to have items that are of either exceptional or inferior quality, with corresponding positive or negative modifiers. The gear may be well-crafted, state-of-the-art, cutting-edge experimental, or simply top-of-the-line, applying an additional +10 to +30. Or it may be outdated, shoddy, or in disrepair, inflicting a -10 to -30 modifier (in some cases canceling out the basic gear bonus). 

\subsubsection{Gear sizes} 

On occasion, you’ll need to know how small or large a certain piece of equipment is. Though this is largely something the gamemaster can wing on the fly using common sense, we’ve listed sizes for many gear items that are unusual or so futuristic that the average player may not have a feel for what dimensions the tech likely is. These size categories are listed on the Gear Sizes table (p. 297). These sizes should be considered approximations, as depending on the manufacturer and process, some items may be smaller or larger than similar items. It is also important to keep in mind that as technology advances, the size and components of various equipment items shrink, so when in doubt, go with smaller. 



\begin{table} \begin{tabularx}{\hline}{|l|X|} \hline

\hline{2}{|c|}{\textbf{Gear sizes}}	\\ \hline

\textbf{Size category}	&\textbf{General dimensions and notes} \\ Nano	&So small that the item cannot be seen without the aid of a microscope or nanoscopic vision (p. 311), and may not be manipulated without fractal digits (p. 311) or similar tools. \\ \hline

Micro	&Anything ranging from the size of a barely visible small dot to an average insect. \\ \hline

Mini	&Mini items may be concealed within someone’s palm or small pockets. \\ \hline

Small	&Small items may be held in one hand and concealed in normal pockets.\\ \hline

Medium	&Medium size items are cumbersome to hold with one hand, ranging from the size of a 2-liter bottle to the size of a medium dog. They do not fit in pockets, but they may be concealed by larger coverings. \\ \hline

Large	&Roughly human-sized. \\ \hline

Huge	&Vehicles and other more massive objects. \\ \hline

\label{tab:gear-sizes} \label{tab:gear-sizes} \end{table} 



\subsubsection{Mass and encumbrance} 

A character who is carrying too much gear should be slowed down, suffering negative modifiers both to their movement rates and their skill tests. Rather than micromanaging the weights of individual pieces of equipment, however, this matter is largely left to the gamemaster’s discretion, using common sense. If a character loads up beyond reason, apply modifiers as seem appropriate. The gamemaster should, however, keep in mind that many of the manufacturing materials used in \emph{Eclipse Phase} allow for items that are much lighter than current standards without any loss of durability or function (see \emph{Future Materials}, p. 298). Likewise, characters in low or microgravity environments can carry much larger loads. 

\subsubsection{Concealing gear} 

Characters may attempt to conceal items on their person, hoping at least to hide them from casual notice if not an intensive search. To determine how effectively the character conceals the equipment, make a Palming Test and note the MoS (the gamemaster may wish to roll this secretly). Whenever another character has a chance to notice the concealed item, they must succeed in a Perception Test and achieve a higher MoS than was scored on the Palming Test. The gamemaster should apply modifiers to both tests as appropriate. For example, concealing a large item like a sword would be difficult (-30), whereas wearing concealing clothing like a longcoat or multi-pocketed jumpsuit would help (+20). Likewise, a character who is not actively looking is less likely to notice the hidden gear (-30), whereas someone who conducts a physical search (+30) or who has enhanced vision to pierce protective layers will fare better. 



\subsection{Design and fashion} \label{sec:design-fashion} 

Many objects in \emph{Eclipse Phase} closely resemble their early 21st century equivalents --- a bottle of soda is still a transparent container holding a brightly colored liquid, clothing is obviously something you wear, and a knife still consists of a blade and a handle. The materials, processes, and mindsets that go into making them, however, are quite different. To start, very few items look have a uniform, mass-produced look, even if they were. The procedures of minifacturing and nanofabrication allow every individual item to be manufactured with a unique (or at least different) look. In areas with anarchist/reputation economies, in fact, where personal possessions have very little intrinsic value, expression and creativity are favored and so many items are artistically personalized (and actual hand-crafted items are rare and prized). Likewise, almost all equipment is designed with ergonomics and ease-of-use prioritized, so gear with soft curves, pleasing colors, and form-fitting shapes are common. Many items of personal technology, such as flashlights or small tools, are made in the form of ovoids that fit comfortably in the user’s hand or in similar forms that can be easily worn or attached to clothing. To someone from the 20th century, many common devices look like oddly colored rocks or decorative pieces of plastic or ceramic (in fact, many such items are referred to as ``blobjects'' by older transhumans). 

The materials used to create everyday items are also advanced, ranging from aerogel and graphene to smart materials (p. 298) and exotic metamaterials with unusual physical properties. In practice, this means that most items are light, durable (with both tensile strength and/or flexibility, as needed), waterproof, dirt-repellent, and self-cleaning. Most gear is also designed with zero-G or microgravity functionality in mind, and can easily be clipped, tethered, or stuck to a surface with grip pads. 

Almost all gear available in \emph{Eclipse Phase} is also available in forms that are wearable/usable by uplifted animals and non-humanoid morphs, such as novacrabs, slitheroids, and so on. Even if such customized gear is not immediately available, it is usually not difficult to nanofabricate. Smart materials (p. 298) also make interoperability between different morphs easy. 

\subsubsection{Interface} 

It is not uncommon for everyday devices to have no visible controls as they are designed to be operated via radio broadcasts from the user’s ecto or mesh inserts. Any items crafted for use in emergency, combat, survival, or exploration situations, however, will feature basic physical controls, just in case. Physical interfaces are typically controlled by touch pads that are nothing more than colored spots on the device’s surface, though some may also project a holographic interface display. Most equipment of this sort can can also be voice-activated and controlled. 

Almost all devices are loaded with a complete set of help files and tutorials. Most electronics are also mesh-capable and equipped with specialized AIs (see \emph{Meshed Gear}, next page). 

\subsubsection{Smart materials} 

Many common items of technology are made from so-called smart materials. These devices contain --- or sometimes consist entirely of --- many small nanomachines that can both move and reshape themselves to alter the object’s shape, color, and texture. For example, smart clothing can transform from a suit of specialized cold weather clothing suitable for the Martian poles in winter to a fashionable suit in the latest style due to hundreds of thousands of tiny nanomachines in the clothing that shift and move to reshape the garment. Similarly, a tool made of smart materials can switch from a powered screwdriver to a wrench or a hammer, as the nanomachines move around and completely reshape the tool. Smart materials all contain specialized advanced nanomachine generators (p. 328) that keep them in perfect repair as long as they are regularly recharged. 



\subsection{Future materials} \label{sec:future-materials} 

Many materials are available and commonly used in \emph{Eclipse Phase} that are rare, theorized, or unheardof today. The following entries note some of the more interesting. 

\subsubsection{Aerogel} 

Low-density, solid-state ``Frozen smoke'' is made by carefully foaming various materials, typically glasses or ceramics, to an ultra-low density state. Aerogel is semi-transparent and light-weight, feels like styrofoam, but acts as an incredible insulator against heat and cold. It is commonly used in habitats. 

\subsubsection{Diamond} 

Artificial diamond is lightweight and super-strong, has an extremely high melting point, and has nearperfect thermal conductivity. This makes it an ideal substance for hardening coated surfaces (armor) and creating super-tough diamond machinery. 

\subsubsection{Fullerenes/Fullerites} 

Fullerenes are molecular carbon structures (known as buckyballs, carbon nanotubes, and graphene) that are extremely strong (vastly stronger by weight than steel), heat-resistant, and can be either insulative or superconductive. This makes them useful in equipment as diverse as armor, electronics, sensor systems, or the cables of space elevators. 

\subsubsection{Metallic foam} 

Metal foam is created by adding foaming agents to liquid metals, resulting in extremely lightweight metallic structures --- light enough to float on water. Ideal for habitat construction and floating cities. 

\subsubsection{Metallic glass} 

Metallic glass are metals highly alloyed to possess a disordered (rather than crystalline) atomic structure with unique combinations of stiffness and strength, making it a good wear surface and alternative to ceramics in armor. It is also useful for its unusual (for a metal) electrical resistance properties. 

\subsubsection{Metamaterials} 

Metamaterials have unusual physical properties (usually electromagnetic) due to their structure, such as having a negative refractive index. Metamaterials are used to create invisibility cloaks (p. 316), superlenses, phased array optics, and impressive 2-D holograms. 

\subsubsection{Refactory metals} 

These metallic alloys have extremely high melting points, making them ideal for extremely hot engine systems, atmospheric entry vehicles, and hypersonic craft. 

\subsubsection{Transparent alumina} 

In transparent form, this ceramic is often known as sapphire. Transparent alumina is harder than steel and zero-g casting techniques allow for intriguing transparent construction designs, so long as its poor tensile strength is respected. 



\section{Meshed gear} \label{sec:meshed-gear} 

Almost all technology in \emph{Eclipse Phase} is designed to be operated via radio signals from the user’s basic implant, although models usable by characters without basic implants are also available. In addition all devices contain a nearly microscopic computer and radio link (known as a ``voice'') that allows the user to easily locate the object and that reports on the condition of the object or device, how to properly use and care for it, as well as telling the user when it needs to be repaired and how. Most are discrete and highly useful, but cheaply made goods sometimes have overly annoying voices. 

This means that almost all devices can be accessed via the mesh or directly if within radio range. This makes them vulnerable to hacking and intrusion attempts (p. 254) as well as radio jamming (p. 262). Many devices are, however, publicly accessible (see \emph{Spimes}, p. 238). Meshed gear may also be tracked through the mesh (p. 251). For privacy and security, these devices are often slaved to other systems (see \emph{Slaving Devices}, p. 248); devices worn/carried by characters are usually made part of the personal area network and slaved to the character’s mesh inserts/ ecto. For more info on meshed devices, see the \emph{Mesh chapter}, p. 234. 

Many devices come equipped with AIs, who are equipped with skillsofts that enable them to operate the device on their own, as according to voiced instructions or commands issued through the net. AIs are described on p. 264 and p. 331. 



\subsection{Radio and sensor ranges} \label{sec:radio-sensor-ranges} 

In \emph{Eclipse Phase}, almost all devices are equipped with small radios so that they may be meshed. Likewise, many pieces of gear are equipped with sensors such as cameras, microphones, or other detectors. The Radio and Sensor Ranges table notes what range these devices operate at. 

\begin{table} \begin{tabularx}{\hline}{|l|l|l|X|} \hline

\hline{4}{|c|}{\textbf{Radio and sensor ranges}}	\\ \hline

\textbf{Size category}	&\textbf{Urban range}	&\textbf{Urban range}	&\textbf{Examples} \\ \hline

Nano &20 meters &100 meters	&Smart Dust, Nanobot/Microbot Swarms \\ Micro	&50 meters	&500 meters	&Microbugs \\ Mini	&1 kilometer	&20 kilometers	&Mesh Inserts \\ Small	&5 kilometers	&50 kilometers	&Ectos, Miniature Radio Farcasters, Portable Sensors \\ Medium	&25 kilometers	&250 kilometers	&Radio Boosters, Vehicle Sensors \\ Large	&500 kilometers	&5000 kilometers	&\\ \hline

\label{tab:radio-sensor-ranges} \label{tab:radio-sensor-ranges} \end{table} 



\subsection{Power} \label{sec:power} 

All of the powered devices in \emph{Eclipse Phase} require electricity to function. With rare exceptions, most of them rely on either solar cells or powerful batteries. These batteries are high-density, room-temperature superconductors with 25 times the capacity of the best batteries in common use in the early 21st century. Such batteries may also be constructed so that they are flexible, printed on devices, or woven into fabric. They are good for 100-500 hours of use, and will alert the user when they start running low. More powerful radio-isotope nuclear batteries are also available, heavily shielded so they do not emit radiation and good for 3 years or more of use. 

In short, power should rarely be an issue in \emph{Eclipse Phase} games, unless it happens to fit the plot. Power failure could also result from a critical failure roll. 



\section{Personal augmentation} \label{sec:personal-augmentation} 

Almost all citizens of the solar system, whether human, AI, or uplifted animal, use various forms of biological, cybernetic, or nanotechnological augmentation. The following is a list of the most common types. 

Unless otherwise noted, any bonuses from personal augmentations are both compatible and cumulative with bonuses from other enhancements. 



\subsection{Standard augmentations} \label{sec:std-augmentations} 

Most morphs produced in the solar system include the following augmentations. 

\subsubsection{Basic biomods} 

Almost universal in biomorphs, many habitats will not allow individuals to visit/immigrate if their biomorph does not possess these biomods in order to preserve public health. Basic biomods consists of a series of genetic tweaks, tailored virii, and bacteria that speed healing, greatly increase disease resistance, and impede aging. A morph with basic biomods heals twice as fast as an early 21st century human, gradually regrows lost body parts, is immune to all normal diseases (from cancer to the flu), and is largely immune to aging. In addition, the morph requires no more than 3-4 hours of sleep per night, is immune to ill-effects from longterm exposure to low or zero gravity, and does not naturally suffer from biological problems like depression, shock reactions after being injured, or allergies. \textbf{[Moderate, but included for free in most biomorphs]} 

\subsubsection{Basic mesh inserts} 

Mesh inserts are ubiquitous among modern morphs. This network of cybernetic brain implants is essential equipment for anyone who wants to stay connected and make full use of the wireless mesh. The interconnected components of this system include: 

\end{itemize} \item \emph{Cranial computer:} This computer serves as the hub for the character’s personal area network and is home to their muse (p. 264). It has all of the functions of a smartphone and PDA, acting as a media player, meshbrowser, alarm clock/calendar, positioning and map system, address book, advanced calculator, file storage system, search engine, social networking client, messaging program, and note pad. It manages the user’s augmented reality input and can run any software the character desires (see \emph{Software}, p. 331). It also processes XP data, allowing the user to experience other people’s recorded memories, and also allowing the user to share their own XP sensory input with others in real-time. Facial/image recognition and encryption software (p. 331) are included by default. \item \textbf{Radio tranciever:} This transceiver connects the user to the mesh and other characters/devices within range. It has an effective range of 20 kilometers in deep space or other locations far from radio interference and 1 kilometer in crowded habitats. \item \textbf{Medical sensors:} This array of implants monitors the user’s medical status, including heart rate, respiration, blood pressure, temperature, neural activity, and much more. A sophisticated medical diagnostic system interprets the data and warns the user of any concerns or dangers. \end{itemize} 

Using any of these functions is as easy as thinking. \textbf{[Moderate, but included for free in most morphs]} 

\subsubsection{Cortical stack} A cortical stack is a tiny cyberware data storage unit protected within a synthdiamond case the size of a grape, implanted at the base of the skull where the brain stem and spinal cord connect. It contains a digital backup of that character’s ego. Part nanoware, the implant maintains a network of nanobots that monitor synaptic connections and brain architecture, noting any changes and updating the ego backup in real time, right up to the moment of death. If the character dies, the cortical stack can be recovered and they may be restored from the backup (see Resleeving, p. 271). Cortical stacks do not have external or wireless access (for security), they must be surgically removed (see Retrieving a Cortical Stack, p. 268). Cortical stacks are extremely durable, requiring special effort to damage or destroy. They are commonly recovered from bodies that have otherwise been pulped or mangled. Cortical stacks are intentionally isolated from mesh inserts and other implants, as a security measure to prevent hacking or external tampering. \textbf{[Moderate, but included for free with most morphs]} 

\subsubsection{Cyberbrain} 

Cybernetic brains are where the ego (or controlling AI) resides in synthmorphs and pods. Modeled on biological brains, cyberbrains have a holistic architecture and serve as the command node and central processing point for sensory input and decision-making. Only one ego or AI may ``inhabit'' a cyberbrain at a time; to accommodate extras, mesh inserts (p. 300) or a ghostrider module (p. 307) must be used. Since cyberbrains store memories digitally, they have the equivalent of mnemonic augmentation (p. 307). They also have a built-in puppet sock (p. 307) may be remote-controlled, though this option may be removed by those who value their security. Cyberbrains are vulnerable to brainhacking (p. 261) and other forms of electronic infiltration/attack. Cyberbrains come equipped with two or more pairs of external access jacks (p. 306), usually located at the base of the skull, which allow for direct wired connections. \textbf{[Moderate, but included for free in all synthetic morphs and pods]} 

\subsection{Bioware} \label{sec:bioware} 

Bioware augmentations can be acquired either as a genemod when the morph is designed and grown or as a later modification to an existing morph, either by using nanomachines to modify the morph’s tissue or by externally growing the organ and implanting it. Bioware may be used to enhance biomorphs (including pods and uplifts), but not synthmorphs. Bioware may be used to enhance biomorphs (including pods and uplifts), but not synthmorphs (see Synthmorphs and Bioware, p. 306). 

\subsubsection{Enhanced senses} 

The following are a list of the most common enhanced senses. Each is also available as a cybernetic implant, but bioware is much more common. 

\textbf{Direction Sense:} The character has an innate sense of direction and distance using advanced inertial navigation. The character can arbitrarily define any point as ``north'' and keep track of which direction that is, as well as knowing approximately how far they have come. Characters with this augmentation can always retrace any route they have taken, only experiencing difficulty with three-dimensional routes lacking navigational markers (such as deep space or undersea; apply a -30 modifier). Since positioning inside habitats by anyone with basic mesh inserts is an automatic affair, only characters venturing to remote locations require this augmentation. \textbf{[Low]} 

\textbf{Echolocation:} The character possesses sonar similar to that of a bat or dolphin. The character bounces brief ultrasonic pulses off their surroundings and uses them to form an image of these surroundings through the pattern of reflections of these pulses received by the character’s ears. For more details, see Using Enhanced Senses, p. 302. This augmentation works in both air and water and has a range of 20 meters in air and 100 meters in water. \textbf{[Low]} 

\textbf{Enhanced Hearing:} The morph’s ears are enhanced to hear both higher and lower frequency sounds --- the range of sounds they can hear is twice that of normal human ears (see Using Enhanced Senses, p. 302). In addition, their hearing is considerably more sensitive, allowing them to hear sounds as if they were five times closer than they are. A character with this augmentation can easily overhear even a softly spoken conversation at another table in a small restaurant. This augmentation provides a +20 modifier to all Perception Tests involving hearing. \textbf{[Low]} 

\textbf{Enhanced Smell:} The morph’s sense of smell is equal to that of a bloodhound. The user can identify both chemicals and individuals by smell, and can track people and chemically reactive objects by smell as long as the trail was made within the last several hours and has not been obscured. The character can also gain a general sense of the emotions and health of any character within 5 meters (+20 to Perception or Kinesics Tests to do so). \textbf{[Low]} 

\textbf{Enhanced Vision:} The morph’s eyes have tetrachromatic vision capable of exceptional color differentiation. These eyes can also see the electromagnetic spectrum from terahertz wave frequencies to gamma rays, enabling them to see a total of several dozen colors, instead of the seven ordinary human eyes can perceive. In addition, these eyes have a variable focus equivalent to 5 power magnifiers or binoculars. This augmentation provides a +20 modifier to all Perception Tests involving vision. For further applications, see Using Enhances Senses, p. 302. \textbf{[Low]} 

\subsubsection{Mental augmentations} 

Mental augmentations are extremely common. 

\textbf{Eidetic Memory:} The character can remember everything that ever happened to them, in detail, with no long term memory loss. For example, they can recite a page they read in a book a month ago, recall a string of 200 random characters they viewed a year ago, or even tell you what they had for breakfast on a particular date a decade ago. However, they can only remember things they paid attention to. The character will not remember the contents of a note on someone’s desk if they merely glanced at it; they must specifically have read it. No effort is required to use this augmentation, the character merely needs to attempt to remember a specific fact. \textbf{[Low]} 

\textbf{Hyper Linguist:} The morph’s brain maintains the linguistic flexibility of a small child, allowing the character to learn languages with great ease. This functions as the Hyper Linguist trait, p. 146. \textbf{[Low]} 

\textbf{Math Boost:} This implants functions as the Math Wiz trait, p. 146. \textbf{[Low]} 

\textbf{Multiple Personalities:} The character’s brain is intentionally partitioned to accommodate an extra personality. This multiplicity is not viewed as a disorder, but as a cognitive tool to help people deal with their hypercomplex environments. This extra personality can be an NPC run by the gamemaster, a separate character (in ego form only) made by the player, or the downloaded fork of another character. For all intents and purposes, the extra personality is treated as a separate ego (i.e., it may fork separately), except that both personalities are backed up in the same cortical stack and if downloaded they must be placed in separate morphs or in another morph with this implant. 

Only one ego may be in control of the morph at a time. The other resides in the background, still active, but not on a surface level. Each ego is completely aware of what the other is doing, thinking, etc. If for some reason the subsumed personality wants to come to the fore, but the other personality won’t relinquish control, make an Opposed WIL $\times$ 3 Test. Each ego has its own Lucidity and Trauma Threshold, and they track stress and trauma separately. Any psi attacks or social/ mental influences only affect the personality at the fore. Having an extra ego in your head, working in the background, is helpful for multitasking. The character receives an extra Complex Action each turn that may only be used for mental or mesh actions. \textbf{[High]} 

\subsubsection{Physical augmentations} 

Most physical bioware augmentations are derived from the capabilities of animals. 

\textbf{Adrenal Boost:} This adrenal gland enhancement supercharges the character’s adrenal response to situations that invoke stress, pain, or strong emotions (fear, anger, lust, hate). When activated, the concentrated burst of norepinephrine accelerates heart rate and blood flow and burns carbohydrates. In game terms, this allows the character to ignore the -10 modifier from 1 wound and temporarily increases REF by +10 (also boosting REF-linked skills and Initiative). These modifiers apply until the character has calmed down (if the character also has endocrine control, p. 304, then adrenal boosts can be activated and deactivated at will, and the negated wounds are cumulative). \textbf{[High]} 

\textbf{Bioweave Armor (Light):} Bioweave armor involves lacing the morph’s skin with artificial spider silk biological fibers. This provides an Armor rating of 2/3 without changing the appearance, texture, or sensitivity of the morph’s skin. This armor is cumulative with worn armor. \textbf{[Low]} 

\textbf{Bioweave Armor (Heavy):} Heavy bioweave armor involves lacing the morph’s skin with a denser and thicker network of the same fibers. The morph’s skin becomes thicker and somewhat less flexible except at the joints. The morph’s skin also has an unusually smooth look, and a distinctively smooth and tough-feeling texture. This provides an Armor rating of 3/4 without decreasing the morph’s mobility. The character’s sense of touch, however, is significantly reduced (-20 modifier) except on their hands, feet, and face. This armor is cumulative with worn armor. \textbf{[Moderate]} 

\textbf{Carapace Armor:} Carapace armor combines bioweave armor with hard but flexible plates of a chitin-ceramic hybrid material modeled on the microscopic structure and texture of arthropod exoskeletons. This armor is obvious and has a somewhat crocodilian or insectoid appearance (character’s choice). The morph is completely hairless as well. This provides an Armor rating of 11/11. This armor is not cumulative with worn armor. \textbf{[Moderate]} 

\textbf{Chameleon Skin:} The morph’s skin is augmented with complex chromatophores so that it changes color like the skin of a chameleon or an octopus. The morph can match the appearance of almost any color and most patterns. This provides a +20 modifier to Infiltration Tests to avoid being seen or noticed, as long as the character is stationary or not moving faster than a slow walk. The character must be nude or wearing smart clothing (p. 325) of the same color/pattern. If incompletely camouflaged, or if moving faster, reduce the modifier to +10. In addition to blending in, the character can also consciously change the color and pattern of their skin to deliberately stand out (+20 on Perception Tests to notice) or simply to produce attractive or interesting colors or patterns. \textbf{[Low]} 

\textbf{Circadian Regulation:} The morph only requires 2 hours of sleep to maintain health and function at peak mental capacity. The character dreams constantly while asleep and can both fall asleep and wake up almost instantly. In addition, the character can easily and with no ill-effects shift to a 2-day cycle, where they are awake for 44 hours and sleep for 4. \textbf{[Moderate]} 

\textbf{Claws:} The morph has retractable claws like those of a cat. These claws do not interfere with the character’s manual dexterity and are razor sharp. However, they are relatively small and only do 1d10 + 1 + (SOM $\div$ 10) damage, with an AP of -1. As a result, they are legal in almost all habitats and are considered tools as much as weapons. \textbf{[Low]} 

\textbf{Clean Metabolism:} The morph’s symbiotic bacteria, gut flora, and glands have been genetically engineered to keep the morph ``clean.'' The morph also produces smart antibiotics that prevent the growth of any bacteria or yeasts in it or on its skin. As a result, the morph is completely immune to infections, dental cavities, and bad breath, its sweat has no scent, and the morph’s efficient digestion produces somewhat less solid waste and less odorous chemicals. \textbf{[Moderate]} 

\textbf{Drug Glands:} The morph has specially-tailored glands designed to produce specific hormones or chemicals and release them in the body. The character has control over these glands and can release the chemicals at will. Each type of drug gland is considered a separate enhancement. For potential drugs and chemicals, see p. 317. \textbf{[One Cost Category Higher Than Drug Cost]} 

\textbf{Eelware:} Derived from electric eel genetics, a character can have eelware implanted so that it connects to a network of bioconductors in the hands and feet (or other limbs), allowing the character to generate stunning shocks with a touch. Eelware inflicts shock damage (p. 204) exactly like a pair of shock gloves. Eelware can also be used to power implants and specially designed handheld devices by touch. \textbf{[Low]} 

\textbf{Emotional Dampers:} This low-cost alternative to endocrine control (p. 304) allows the user to voluntarily damp their morph’s emotional responses and various non-verbal cues like pupil dilation, eye movement, or vocal tone. Using this augmentation allows the user to lie and conceal their emotions in such as way as oo fool the keenest observer; apply a +30 modifier to Deception and Impersonation Tests. This modification does not affect methods of detecting lies and emotions that involve reading the character’s neural state, including psi-gamma sleights. However, this augmentation damps out all emotional responses and so causes the character to be less persuasive in real- time personal interactions, imposing a -10 modifier to other Social skill tests like Persuasion. Characters can turn this augmentation on or off at will. \textbf{[Low]} 

\textbf{Endocrine Control:} This augmentation modifies the morph’s endocrine system, giving the character fine control over their hormone output. This allows the character to completely control their appetite and emotions and to regulate pain. They receive a +30 modifier against the effects of hunger, fear, and any forms of emotional manipulation, such as the Drive Emotion sleight. This augmentation also allows character to lie with perfect conviction and to completely fool all methods of lie detection that do not rely on the target’s neural output; apply a +20 modifier to Deception Tests. It also allows the character to remain awake for 48 hours without penalty, but after this time the character begins experiencing normal fatigue. Finally, the ability to regulate pain reception allows the character to ignore the -10 modifier from 1 wound. \textbf{[High]} 

\textbf{Enhanced Pheromones: }The morph’s biochemistry has been altered so that it produces enhanced pheromonal signals that subconsciously affect the behavior of other humans in the vicinity. These pheromones make the character more attractive and trustworthy to the target; apply a +10 modifier to appropriate Social skill tests, such as Persuasion. This augmentation only affects characters who can smell the pheromones, and it does not affect uplifts or xenomorphs. [Low] Enhanced Respiration: By boosting both lung efficiency and the blood’s oxygen-carrying capacity, the character can live comfortably in both high and low pressure environments, from 0.2 atmospheres to 5 atmospheres, with no dizziness or need for gradual decompression. In addition, the character can hold their breath for up to 30 minutes when performing minimal activity or for up to 10 minutes while performing highly strenuous activity. \textbf{[Low]} 

\textbf{Gills:} The morph’s lung tissue has been adapted to function as gills, allowing the morph to breathe both air and water, as long as the water is not toxic or too stagnant. Characters with this augmentation breathe in water and then expel the water through slits just underneath their lowest pair of ribs that seal when the character is not underwater. \textbf{[Low]} 

\textbf{Grip Pads:} The morph possesses specialized pads on its palms, lower arms, shins, and the bottoms of its feet. Designed to emulate the pads on gecko feet, characters can support themselves on a wall or ceiling by placing any two of these pads against any surface not made from a material specially designed to resist this augmentation. Characters can climb any surface and move easily across ceilings that can support their weight. Apply a +30 modifier to Climbing Tests. The pads must be free to touch the surface the character is climbing (no gloves). The nature of these pads is obvious to anyone looking at them, but they do not impair the character’s sense of touch or manual dexterity. If combined with the vacuum sealing augmentation, the character can even stick to surfaces in the vacuum of space. \textbf{[Low]} 

\textbf{Hibernation:} The character can voluntarily reduce the morph’s metabolism to the point that the morph requires only 5\% of the normal amount of food, water, and air. The character appears to sink into a deep sleep, but can maintain a dim awareness of both touch and sound and so can be easily awakened. Entering or leaving this state requires 3 minutes where the character is relatively helpless. With sufficient air, characters can safely hibernate for up to 40 days without food or water. \textbf{[Low]} 

\textbf{Muscle Augmentation:} The morph’s muscle mass has been enhanced and toned and myofibers strengthened. Apply a +5 modifier to SOM. \textbf{[High]} 

\textbf{Neurachem:} This bioware modification enhances the character’s chemical synapses and juices their neurotransmitters, drastically speeding up neural connections. Neurachem can be mentally activated or triggered by charged emotions. Level 1 neurachem increases the character’s Speed stat by +1, with no side effect. Level 2 raises the Speed stat by +2, but each time it is used the character suffers a nervous system fatigue hangover for 1 hour after the boost wears off (apply a -20 modifier to all actions). \textbf{[High (Level 1), Expensive (Level 2)]} 

\textbf{Poison Gland:} Similar to the drug gland, this morph has special glands that produce poisons, like the venom glands of a snake. The morph has poison glands in its fingers and mouth, so that it can deliver either poison by scratching someone with a fingernail, biting them hard enough to draw blood, or even by sharing a beverage with someone or spitting into their drink. The morph is immune to the poisons it produces. These glands may not produce nanotoxins. \textbf{[Low]} 

\textbf{Prehensile Feet:} The morph’s feet and leg joints are altered so that its toes are longer and more dexterous and the big toe is transformed into an opposable thumb. Physically, the morph’s feet resemble a longer narrower hand or a human foot with finger (and thumb)-like toes. The character can walk normally but must wear specially designed shoes. However, this morph runs somewhat slower than a morph with unmodified feet (-1 meter per Action Turn). In addition, the morph’s hips are slightly modified to allow greater mobility. In a properly constructed chair, or when floating in zero-G, the character can use both their hands and their feet to manipulate the same object. Most morphs used by characters who live in zero-G possess this augmentation. \textbf{[Low]} 

\textbf{Prehensile Tail:} A long (1.5 meters) prehensile tail is added to the morph’s backside, extending out from the tailbone. This tail is prehensile and may be used to grab, hold, and even manipulate objects. The character can control the tail’s movements with concentration, but it otherwise tends to move on its own. The tail also improves the character’s balance; apply a +10 to any Physical skill tests where balance is a factor. \textbf{[Low]} 

\textbf{Sex Switch:} A complex suite of alterations allows the character to switch their physical sex to male, female, hermaphrodite, or neuter. This change is mentally triggered but takes approximately 1 week to complete. \textbf{[Moderate]} 

\textbf{Skin Pocket:} The morph has a pocket within its skin layer, capable of holding and providing concealment (+30) for small items. \textbf{[Trivial]} 

\textbf{Temperature Tolerance:} The morph’s temperature regulation and circulation are both substantially enhanced allowing the character to survive in temperatures as low as -30 degrees Celsius and as high as 60 degrees Celsius without discomfort or ill effects. \textbf{[Low]} 

\textbf{Toxin Filters:} The morph gains an improved liver and kidneys and biological filters in its lungs. Characters with this augmentation are immune to all chemical and biological toxins, including everything from recreational chemicals to nerve agents to spoiled food. In addition, the character can safely and comfortably breathe smoke and drink salt water. Unlike medichines, toxin immunity prevents the character from experiencing even brief harm or discomfort from a toxin (medichines merely rapidly repair damage caused by the toxin and then remove it from the morph). This augmentation provides no resistance to concentrated acid, nanotechnological attacks, or similar destructive agents. Some characters with this augmentation learn to enjoy the taste of various chemical toxins like cyanide or arsenic. \textbf{[Moderate]} 

\textbf{Vacuum Sealing:} To possess this augmentation, the character must also possess some form of bioware armor or carapace armor. The morph has been specially designed to survive the effects of vacuum. The character’s skin resists vacuum as well as protecting the wearer from temperatures from -75 to 100$^{\circ}$ C. In addition, the character’s mouth, nose, and other orifices can seal sufficiently well to resist vacuum, and the morph possesses a special membrane that extends over their eyes, allowing the character to see in vacuum without risking any eye damage. This augmentation is usually combined with either the enhanced respiration or oxygen storage augmentation, or both together. \textbf{[High]} 

\subsubsection{Synthmorphs and bioware} 

Though bioware is preferred and more common, many types of bioware can be mimicked with cybernetics. This is especially useful for synthmorphs/ robots, which cannot be enhanced with bioware. The following bioware items may be replicated as cybernetics for synthmorphs and robots: 

\end{itemize} \item Chameleon Skin \item Drug Glands \item Eelware \item Emotional Dampers \item Enhanced Senses (All) \item Grip Pads \item Mental Augmentations (All) \item Muscle Augmentation \item Neurachem \item Poison Glands \item Prehensile Feet \item Prehensile Tail \end{itemize} 

\subsection{Cyberware} \label{sec:cyberware} 

Very little cyberware is physically implanted. Instead, the morph is placed in a healing vat (p. 326) and the vat’s nanobots construct the cyberware inside the biomorph’s body. Cyberware is rarely used for anything that can be accomplished using bioware. 

Synthmorphs and bots may also also use cyberware. 

\subsubsection{Enhanced senses} 

In addition to being able to duplicate the affects of all bioware enhanced senses, there are a few enhanced senses that can only be produced using cyberware. 

\textbf{Anti-Glare:} This visual mod eliminates penalties for glare. \textbf{[Low]} 

\textbf{Electrical Sense:} The character can sense electric fields. Within 5 meters, the character can instantly tell if an electrical device is on or off and can see the precise location of electrical wiring behind a wall or inside a device. This sense gives the character a +10 modifier on any test involving analyzing, repairing, or modifying electrical equipment. \textbf{[Low]} 

\textbf{Radiation sense:} The character can sense the presence and approximate source of all forms of dangerous radiation, including neutrons, charged particles, and cosmic rays. \textbf{[Low]} 

\textbf{T-Ray Emitter:} Mounted under the skin of the user’s forehead, this implant generates low-powered beams of terahertz radiation (T-rays) that allow the character to see using reflected T-rays. As discussed in Using Enhanced Senses, p. 302, this implant combined with the enhanced vision enhancement (or a terahertz sensor) allows the user to effectively see through cloth, plastic, wood, masonry, composites, and ceramics as well as being able to determine the composition of various materials. This implant allows the user to see using reflected T-rays for 20 meters in a normal atmosphere and for 100 meters in vacuum. \textbf{[Low]} 

\subsubsection{Mental augmentations} 

These cybernetic augmentations enhance the brain and mental functions. 

\textbf{Access Jacks:} Usually located in the base of the skull or neck, this implant is an external socket with a direct neural interface. It allows the character to establish a direct wired connection using a fiberoptic cable to external devices or other characters, which can be useful in places where wireless links are unreliable or complete privacy is required. Two characters linked via access jack can ``speak'' mind-to-mind and transfer information between their mesh inserts and other implants. All synthmorphs have these by default. \textbf{[Low]} 

\textbf{Dead Switch:} This cortical stack (p. 300) accessory is designed to keep the stack from falling into the wrong hands. If the morph is killed, the dead switch wipes and melts the cortical stack completely, so that the ego cannot be recovered. This option is generally only used by covert operatives with recent backups. \textbf{[Low]} 

\textbf{Emergency Farcaster:} Only characters with cortical stacks can possess this augmentation. The morph has an implanted quantum farcaster (p. 314) linked to a highly secure storage facility. The high cost of this implant also covers the cost of this storage. Using standard radio and quantum encryption, the farcaster broadcasts full backups of the character’s ego (pulled from the cortical stack) once every 48 hours. At the gamemaster’s discretion, the backup interval may be scheduled more or less frequently, keeping in mind that ego broadcasts are generally limited for security purposes and because they hog bandwidth. These broadcasts only work when the character is in radio contact with the storage facility and is typically only used inside a habitat to broadcast backups back to a nearby space ship. If the radio broadcasts are blocked or jammed, this device cannot make backups. 

In the event of a farcaster failure, this augmentation also includes a single-use emergency neutrino broadcaster (p. 314) as well. This broadcaster contains approximately 10 nanograms of antimatter stored in an orange-sized triply-redundant magnetic containment vessel. If the character is dying or urgently wishes to depart the morph, this tiny amount of antimatter is brought into contact with a similarly tiny amount of matter in a controlled fashion that generates a single brief and carefully coded neutrino pulse of the ego’s most recent backup. However, the heat generated by this process literally cooks the entire morph, killing it and destroying all implants and electronics in or on it. 

This entire process takes less than 0.1 second and the broadcast can be received as long as the neutrino receiver is within 100 astronomical units of the character. Within the solar system, this implant effectively guarantees the character’s backup. It is less useful on exoplanets where the character is out of neutrino range of their backup facility. The amount of antimatter carried by this implant is sufficiently small enough that it does not produce an explosion and will not damage any surrounding objects. Most habitats carefully scan all visitors to determine if they have this implant and if the amounts of antimatter involved are sufficiently low as not to pose a danger to the habitat and its inhabitants, and some ban this implant entirely. \textbf{[Expensive]} 

\textbf{Ghostrider Module:} This implant allows the character to carry another infomorph inside their head. This infomorph could be another muse, an AI, a backed-up ego, or a fork. The module is linked to the character’s mesh inserts, so the ghost-rider can access the mesh. The character may limit the ghostrider’s access, or may allow them direct access to their sensory information, thoughts, communications, and other implants. \textbf{[Low]} 

\textbf{Mnemonic Augmentation:} A character with this augmentation and a cortical stack can access digital recordings of all of the sensory data they have experienced in XP format (and they may share these recordings with others). Mnemonic augmentation differs from the eidetic memory bioware because it allows characters to digitally share all of their sensory data with others. It also allows them to closely examine sensory data they did not initially look at. For example, If the character glanced at a note but did not read it, they can later use image enhancement software to enhance this image and in most cases actually read what the note said. Mnemonic augmentation allows the character to clearly hear all background noises, like a conversation at a nearby table that the character only initially heard a few words of. Using mnemonic augmentation to retrieve a specific piece of information is quite easy, but usually requires between 2 and 20 minutes of concentration. \textbf{[Low]} 

\textbf{Multi-Tasking:} Only characters with cortical stacks can possess this augmentation. The character has an advanced computer installed in their brain that uses the data in the cortical stack to create several simultaneous short-term forks to handle various mental tasks. By design, this computer automatically reintegrates all of these forks into the character’s core personality after a maximum of 4 hours, earlier if desired. This augmentation allows the character to both plan a speech and engage in intensive mesh-browsing while simultaneously fighting a gun battle or running from pursuit, since each of the forks operates independently. However, these forks can only perform purely mental or on-line interactions. This augmentation can produce a maximum of two forks at a time, giving the character an extra two Complex Actions on every Action Phase for mental or on-line actions. This implant cannot be used simultaneously with any other augmentation that allows for extra actions, or with the mental speed augmentation (p. 308). \textbf{[High]} 

\textbf{Puppet Sock:} This implanted computer allows the biomorph’s body (the ``puppet'') to be controlled by another character (the ``puppeteer''). While active, the puppet has no control over their body and is simply along for the ride (at the gamemaster’s discretion, puppets who are tormented by repeated or extensive loss of control may suffer mental stress). The puppeteer may directly ``jam'' the puppet or remote control it in the same way that robots and pods are teleoperated (p. 196). The puppeteer must either be ghost-riding the puppet (see the Ghostrider Module, p. 307) or have a direct communications link (via mesh, radio, laser, etc.). \textbf{[Moderate]} 

\subsubsection{Physical augmentations} 

This implants enhance the morph’s physical body. 

\textbf{Cyberclaws:} The bones on the back of the morph’s hand are bonded to smart material claws. These claws can extend through concealed ports in the morph’s skin and extend 6 inches past the morph’s knuckles. These razor-sharp weapons inflict 1d10 + 3 + (SOM $\div$ 10) damage and have an AP of -2. If combined with eelware (p. 304), they can also inflict electric shocks. Likewise, cyberclaws can also deliver poison or nanotoxins secreted from a poison gland (p. 305) or implanted nanotoxins. \textbf{[Low]} 

\textbf{Cyberlimb:} In an age when arms and legs can easily be regrown, many people consider cybernetic prostheses to be vulgar and distasteful. The Scum and others, however, treat them as iconic symbols of self-expression. Standard replacement cyberlimbs function the same as their biological equivalents, though that particular limb receives a +3/+3 Armor bonus when targeted specifically (this bonus does not apply to synthmorphs). Cyberlimbs may be masked to look real (see Synthetic Mask, p. 311), and may also feature small compartments for hiding/storing small objects. \textbf{[Moderate]} 

\textbf{Cyberlimb Plus:} More extravagant cyberlimb models are also available, though they require more severe body alteration to accommodate. These limbs apply a +5 SOM bonus per limb (maximum +10). They may be replacement limbs or ``extra'' limbs anchored in the body’s skeletal frame. These cyberlimbs may not be masked. \textbf{[High]} 

\textbf{Hand Laser:} The morph has a weapon-grade laser implanted in its forearm, with a flexible waveguide leading to a lens located between the first two knuckles on the morph’s dominant hand. The laser fires from this waveguide, inflicting 2d10 damage with 0 AP. The laser is powered by a small nuclear battery located in the morph’s torso, good for 50 shots before it must be recharged like other beam weapon batteries (p. 338). \textbf{[Moderate]} 

\textbf{Hardened Skeleton:} The morph’s skeleton has been laced with strengthening materials. Apply a +5 DUR and +5 SOM bonus. \textbf{[High]} 

\textbf{Oxygen Reserve:} The morph has a miniature oxygen tank and rebreather installed in its torso. This implant provides the equivalent of the life support system in a light vacsuit (p. 333), allowing the character to breathe comfortably for up to 3 hours. It feeds oxygen directly to the morph’s blood stream, avoiding problems with pressure changes. Implanted sensors automatically cause the character to use the stored oxygen if they detect poisonous or insufficient atmosphere. Without vacuum sealing, the character can only survive in vacuum for 5 minutes, but remains conscious and active for the entire time, giving them far more time to find shelter or a vacsuit than characters without this implant. For every hour the character is in a breathable atmosphere, this implant recovers one hour of oxygen storage. The implant can be fully recharged within 15 minutes if the character is in a high-pressure mostly oxygen atmosphere. \textbf{[Low]} 

\textbf{Reflex Boosters:} The morph’s spinal column and nervous system is rewired with superconducting materials, boosting transmission speed. This raises the character’s REF by +10 and improves Speed by +1. \textbf{[Expensive]} 

\subsection{Using enhanced senses} \label{sec:using-enhanced-senses} 

Personal augmentations and technological aids have drastically increased the sensory capabilities of most transhumans. The following notes provide some details on what capabilities these sensory functions provide. The capabilities are typically the same whether it’s a biological sense or a technological sensor, though tech sensors can ``turn off'' certain wavelengths and sense only specific frequencies, whereas biological senses perceive the full spectrum with no ability to filter parts out. 

\subsubsection{Sensory databases} 

Both technological sensors and enhanced biological senses come equipped with databases of scanned ``signatures'' that make it easier to identify whatever the user is sensing (in the case of bioware, these databases are stored and accessed via the character’s mesh inserts). For example, infrared sensors feature databases listing the heat signatures of different animals and items, making it easier to identify such things. In relevant situations, apply a +20 modifier for identifying targets sensed this way. 

\subsubsection{Active vs. passive} 

An active scanner must actually emit its particular frequency and then measure the reflections; this means a similar sensor can detect it and home in on the emitting source. For example, a character with enhanced vision can literally see the terahertz radiation emitted by someone using an active terahertz sensor, much like someone with normal vision can see the light emitted by a flashlight. 

A passive scanner simply scans frequencies that occur naturally --- there is nothing to give the sensor away. 

\subsubsection{Electromagnetic spectrum} 

For \textit{Eclipse Phase} rules purposes, the EM spectrum is broken down by wavelength and frequency into these categories: radio, microwave, terahertz, infrared, visible light, ultraviolet, X-rays, and gamma rays. 

\textbf{Radar (Radio/Microwave): }Radar sensors work by actively emitting radio waves and microwaves and measuring them as they bounce off the target. Radar works best when detecting metallic objects, and is less effective (-20 modifier) against biomorphs and small items. Resolution is not high, however, so it can see shapes but not colors or fine details. It can be used to detect both speed and movement, can ``see'' through walls (up to a cumulative Armor + Durability of 100), and can detect cybernetic implants or concealed items. At close ranges (1-2 meters), it can detect pulse rate and respiration by measuring the motion of the chest cavity. 

\textbf{Terahertz:} Terahertz sensors emit t-rays, measure the reflections, and compare them to a database of terahertz signatures that different items/materials have. The resolution is higher than radar, but with slightly less detail than normal vision. Similar to radar, terahertz sensors can see through walls and other materials, but to a lesser extent (up to a cumulative Armor + Durability of 50). T-rays occur naturally, but terahertz sensors normally require an emitter as they are absorbed by atmosphere (as well as water and metal). In space, however, an emitter would not be required. Likewise, passive terahertz scans within atmosphere have an effective range of 25 meters. T-rays do not penetrate skin, so are ineffective for locating implants. 

\textbf{Infrared:} Near-infrared wavelengths are used for night vision, providing resolution and detail equivalent to regular vision under low-light conditions. Mid-long infrared is excellent for detecting heat sources (unobstructed by fog or smoke) and temperature differences (as small as 0.1 degree C), and such thermal imaging will sense the dissipating heat traces left by warm sources on colder ones, allowing the user to see where someone was sitting, trace fading heat footprints, or see what buttons were pressed if they are quick enough. Infrared also detects the blood flow in a biomorph’s face, which can be useful in judging emotional states (+20 modifier to Kinesics Tests), and can spot subsurface implants. Some normally white surfaces are reflective (mirrored) in infrared, potentially allowing an infrared viewer to see around corners or behind themselves. On the other hand, some glass is opaque to infrared light. Infrared is also useful for determining chemical composition (enabling Chemistry Tests by sight alone). Infrared sensory input is passive. 

\textbf{Lidar (Visible Light):} Similar to radar, but with much higher resolution, lidar actively bounces light from the infrared through ultraviolet spectrum off a target and measures the backscatter, fluorescence, and other properties. Lidar is very useful for detecting atmospheric chemical properties and weather. Like radar, it can be used to measure a target’s range and speed, or develop a three-dimensional image. One clever use of lidar is to precisely ``map'' the position of everything in a room (taking several turns of scanning) and then check that positioning later to see if anything has been moved. 

\textbf{Ultraviolet:} Some objects are fluorescent in ultraviolet light, including some animals, flowers, insects, urine, and minerals (which show up much better in ultraviolet than regular light). Some plants and animals have patterns that can only be seen in ultraviolet. Likewise, chemical dyes that only show up under ultraviolet, or that make certain substances (like blood) fluoresce under ultraviolet light, have various security purposes. Some glass is opaque at ultraviolet wavelengths. 

\textbf{X-Ray/Gamma-Ray:} Backscatter imaging systems using X- and gamma-ray frequencies produce high-resolution three-dimensional images and are very useful for detecting concealed weapons and implants. Such imagers are very good at penetrating walls and metal (up to a cumulative Armor + Durability of 200, at least at levels safe to transhumans). These sensors can, of course, also detect the presence of harmful radiation. 

\subsubsection{Soundwaves} 

The transmission of vibrations through a medium, sound is broken down into infrasound (frequencies below standard human hearing), normal acoustic range, and ultrasound (frequencies above standard human hearing). Soundwaves do not propagate in vacuum. 

\textbf{Ultrasound:} Ultrasound sonar operates much like radar, bouncing sound waves off a target and measuring the returning echoes. Ultrasound imaging is similarly low-resolution, showing shapes and movement but no colors and few details unless measured closely (1-2 meters). Ultrasound is good for identifying how dense a material is, however, can detect denser materials hidden beneath less dense ones. Many medical devices utilize ultrasound, and ultrasound sensors can also detect gas leaks, frictional motor noises, and similar mechanical emissions. Ultrasound sensors are typically unaffected by noise clutter from standard acoustic frequencies. 

\textbf{Infrasound:} Infrasound travels much further than regular sound frequencies (hundreds of kilometers). Mechanical machinery, seismic disturbances, tornados, explosions, waterfalls, and certain weather phenomena create infrasound waves. Large animals such as elephants and whales use infrasound to communicate via the ground over large distances, though infrasound data transfer is too slow for complex communications. 

\subsubsection{Combined sensor systems} 

When used in combination, these sensor technologies can be potent. For example, the use lidar, thermal imaging, and radar can provide a threedimensional map of a building and everyone and everything inside. 



\subsection{Nanoware} \label{sec:nanoware} 

All augmentation nanoware is advanced nanotechnology (p. 328), consisting of a grape-sized nanobot generator that produces specialized nanomachines. Nanoware is available for synthmorphs and bots in addition to biomorphs. 

\textbf{Implanted Nanotoxins:} The morph has an implanted nanobot hive that produces nanotoxins (p. 324). This implant is designed so that the character can deploy these nanobots instantly via a scratch with claws, spraying with saliva, or simply making continuous bare-skin contact. Characters can choose whether or not to deploy these nanobots. Each nanotoxin generator only produces a single variety of nanobots, with the most common types being ones designed to kill or incapacitate almost any living target or ones designed to destroy delicate machinery. Characters are immune to their own nanotoxins. Nanotoxins are highly restricted and many habitats will not allow anyone with this implant on board. \textbf{[Moderate]} 

\textbf{Medichines:} This is the most common form of nanoware. These nanobots monitor the user’s body at a cellular level and fix any problems that arise. Medichines eliminate most diseases, drugs, and toxins (but not nanodrugs or nanotoxins) before they can do more than minor harm to the host (see Drug Effects, p. 318). If desired, the user can temporarily override this protection to permit intoxication or other effects, but unless the character activates a second specially labeled override, medichines prevent the toxins from accumulating to lethal or permanently harmful levels. In this case, they can also be activated at a later point to reduce a drug or toxin’s remaining duration by half. 

Medichines allow the character to ignore the effects of 1 wound. They also speed normal healing as noted under Biomorph Healing, p. 208. If the user suffers 5 or more wounds at once, or more than 6 wounds in an hour, the damage has exceeded the medichines’ ability to repair. In this case, the medichines place the character into a medical stasis, where their mind and body are perfectly preserved, but where the character cannot act in any way. Under these circumstances the medichines also send out a priority call for emergency services via the character’s mesh inserts. Medichines for synthmorphs and bots consist of nanobots that monitor and repair the shell’s integrity and internal system functions. Note that the synthmorph version of medichines allows the synthmorph to self-repair in the same manner by which a biomorph with medichines would naturally heal (p. 208). \textbf{[Low]} 

\textbf{Mental Speed:} With this nanoware system, nanobots alter the character’s neural architecture and augment the functioning of their neurons. The character can deliberately speed up their mind to think and also receive and process sensory information far faster than ordinary humans. Time seems to subjectively slow down for the character, allowing them to carefully plan their next action, even if they only have a split second to do so. With this system active, the character can discern things occurring too fast for a normal human to perceive, such as the individual frames of an old analog film or understanding sounds that were accelerated to many times their normal speed. The character can also read 10 times faster than normal and can track the paths of bullets and similar fast-moving objects with a successful Perception Test. 

When using this augmentation, the character gains two extra Complex Actions during each Action Phase that may only be used for mental actions. The character also receives a +30 Initiative bonus. The character thinks at normal speed whenever this nanoware is inactive. This nanoware is incompatible with any other augmentation that provides any form of extra actions, such as multi-tasking. This augmentation can be used as often as desired, but actively using it renders ordinary conversation and social interactions difficult and requires concentration to maintain. \textbf{[High]} 

\textbf{Nanophages:} These nanobots patrol the body, alert for signs of intrusive nanodrugs or -toxins and destroying them before they have more than a minor effect. Nanophages provide automatic immunity against nanodrugs and nanotoxins unless they are specifically commanded to stand down by the user, via their mesh implants. \textbf{[Moderate]} 

\textbf{Oracles:} These neural macrosensing nanobots pay attention to the sensory input on which the character is not focusing, alerting them about important things they might otherwise overlook. Oracles also act as a sort of memory buffer and search aid, extending short term memory, helping the character recall memories and details, and crosschecking them with other memories. Oracles negate Perception modifiers for distraction, apply a +10 modifier to Investigation Tests, and add a +30 bonus to memory-related tests. \textbf{[Moderate]} 

\textbf{Respirocytes:} These nanobots act as highly-efficient artificial red blood cells, increasing the ability to transfer oxygen and carbon dioxide. This increases the morph’s ability to hold their breath to 4 hours and increases DUR by +5. \textbf{[Moderate]} 

\textbf{Skillware:} The morph’s brain is laced with a network of artificial neurons that may be formatted with downloaded information. This allows the user to download skillsofts (p. 332) into their brains, gaining the use of those programmed skills until the skillsoft is erased or replaced. Skillware systems are only capable of handling 100 total skill points worth of skillsofts at a time. \textbf{[High]} 

\textbf{Skinflex:} This disguise implant allows the user to restructure their facial features and musculature and alter skin tone and hair color. The entire process takes a mere 20 minutes. Skinflex adds +30 to Disguise Tests. \textbf{[Moderate]} 

\textbf{Skinlink:} Skinlink nanobots live on the morph’s external skin or shell, automatically swarming over and creating a physical connection with any electronics the user touches. They also take advantage of the electrical field in a biomorph’s skin for communication. They allow the user to communicate and mesh with any devices merely by touching them. This is considered a wired link, and so is not subject to wireless interception or interference. Two skinlinked characters can also communicate and mesh simply by touching. \textbf{[Moderate]} 

\textbf{Wrist-Mounted Tools:} The morph has a 6 centimeter- wide metal band containing nanobot generators implanted around each wrist. These nanobots link together to duplicate the function of a utilitool (p. 326), creating narrow, highly flexible arms that each ends in a specialized tool. These nanobots can also produce tiny fiber optics to allow the character to see through small openings, as well as being able to create small weapons equal to bioware claws. The fact that these tool are mentally controlled gives the character a +20 modifier to skills involving repairing or modifying devices with mechanical parts, opening locks or disarming alarm systems, or performing first aid. \textbf{[Moderate]} 



\subsection{Cosmetic mods} \label{sec:cosmetic-mods} 

In an age of universal beauty, artistic cosmetic modification of your body is commonly pursued by many transhumans. Body mods once considered dangerous or edgy are now safe and commonplace, especially among factions like the anarchists, scum, or brinkers. 

\textbf{Bodysculpting:} If your morph’s enhanced physique isn’t enough, you can take it further with custom bodysculpting such as as elongated ears or fingers, nose alteration, hair addition/removal, feathers, exotic eyes, snakeskin, endowed genitalia, and more unusual physical alterations. \textbf{[Low]} 

\textbf{Nanotats:} Tattoos created with nanobots can move around the body, change shape/color/brightness, texture, alternate text and images, and/or even create minor holographic effects on the skin’s surface, all controllable via mesh inserts. \textbf{[Low]} 

\textbf{Piercings:} Name any part of the body and someone’s figured out a way to pierce it, probably multiple times. Hoops, barbells, plugs, and chains are extremely common, often made of shapechanging smart materials. \textbf{[Trivial]} 

\textbf{Scarification:} Given modern medical abilities, scars of any sort are purely an affectation. \textbf{[Trivial]} 

\textbf{Scent Alteration:} Minor changes to a body’s biochemistry can alter a character’s natural smell or constantly perfume them. \textbf{[Low]} 

\textbf{Skindyes:} Dye jobs are available in all conceivable colors and patterns. \textbf{[Trivial]} 

\textbf{Subdermal Implants:} Adding small implants under the skin can create bumps, ridges, piercing anchors, and similar textures and alterations. \textbf{[Trivial]} 



\subsection{Robotic enhancements} \label{sec:robotic-enhancements} 

The following modifications are only available to synthmorphs/robots. 

\subsubsection{Armor} 

These armor modifications replace the synthmorph’s built-in Armor rating. 

\textbf{Heavy Combat Armor:} The synthmorph’s frame is loaded with armor that offers protection from heavy weapons for serious combat operations. This modification is bulky and noticeable; the bot frame is encased in a heavy-duty carapace. It increases the bot’s built-in Armor to 16/16. The shell’s mobility systems and power output are also enhanced to deal with the extra load. \textbf{[High]} 

\textbf{Industrial Armor:} The shell is equipped with protection against collisions, extreme weather, industrial accidents, and similar wear-and-tear. Increase the bot’s built-in Armor rating to 10/10. [Moderate] Light Combat Armor: The synthmorph’s frame is protected by armor designed for policing and security duties. This increases the bot’s built-in Armor to 14/12. \textbf{[Moderate]} 

\subsubsection{Mobility systems} 

Shells are designed with a wide-range of propulsion systems, and are sometimes built for a specific environment/ gravity. Some synthmorphs may have multiple mobility systems. Many such systems are retractable, meaning they can be folded away into the shell’s frame. 

\textbf{Hopper:} Hoppers have two or more legs designed to propel the morph forward or up, much like a frog or grasshopper. \textbf{[Moderate]} 

\textbf{Hovercraft:} The shell uses an impeller to blast a cushion of high-pressure air off the surface below, repelling the frame off the ground (modern hovercraft do not use rubber skirts). Most hovercraft travel a meter or so above the ground, but can temporarily levitate themselves higher for short periods. \textbf{[Low]} 

\textbf{Ionic:} The shell uses principles of magnetohydrodynamics to levitate and fly, by ionizing surrounding air into plasma to create lift and momentum. The shell is also spun for stability. This system does not work in vacuum, but an underwater version uses the same mechanics for propulsion in liquid environments. \textbf{[High]} 

\textbf{Microlight:} Popular in low-grav and microgravity environments, microlights encompass several types of ultralight or lighter-than-air systems, such as powered paragliders, autogyros, balloons, aerostats, and blimps. These systems do not work in vacuum. [Low] Roller: Only for circular shells, this system allows the synthmorph to roll like a ball. The shell rolls around an interior axle, propelled by a motor-driven pendulum. \textbf{[Moderate]} 

\textbf{Rotorcraft:} Rotating blades create lift, allowing the shell to move and hover like a helicopter. Most models use tilt-rotors or tilt-wings so that the rotorblades may be moved forward (for faster propellerlike propulsion) and for better maneuverability in zero-G. This system does not work in vacuum. \textbf{[Low]} 

\textbf{Snake:} Commonly used by slitheroids, these shells use lateral undulation, flexing their body from left to right and waving their frame forward. Such shells may also use sidewinding or a concertina motion (straightening forward, then retracting the rear) to move. They also featured gyroscope stabilization so that they may circle into a hoop and roll like a wheel. [Moderate] Submarine: Designed for undersea mobility, submarine shells use propellers or pumpjets to push through water. \textbf{[Moderate]} 

\textbf{Tracked:} Tracked shells use smart rotating treads to work their way across surfaces that would bog down other ground vehicles. They can prop themselves up in order to overcome taller obstacles or to lay themselves down to bridge across a ditch or crevice. \textbf{[Low]} 

\textbf{Thrust Vector:} These shells use either turbofans or turbojets to create atmospheric lift with a set of wings. The engines may be maneuvered to point and generate thrust in different directions for vertical takeoffs/landings and better maneuverability in zero-G. \textbf{[Moderate]} 

\textbf{Walker:} Walkers use two or more limbs to walk or crawl across a surface. Many use grip pads (p. 305) or magnetic systems (p. 310) to stick to surfaces. \textbf{[Low]} 

\textbf{Wheeled:} Most wheeled shells feature smart spokes that allow the wheels to conform their shape to obstacles and even climb stairs. Some low-grav shells feature puncture-resistant and self-repairing compressed-gas tires. \textbf{[Low]} 

\textbf{Winged:} Primarily used by smaller shells, this system of four independently-controlled wings allows the shell to hover or move rapidly in any direction. \textbf{[Low]} 

\subsubsection{Physical modifications} 

These mods are applied to the shell’s physical frame. 

\textbf{Extra Limbs:} The shell is equipped with one or more extra limbs. A character using these limbs suffers an off-hand modifier (p. 193). These limbs may be arms (with hand/grippers/etc.), legs, tentaclelike, or otherwise articulated and/or prehensile. Some shells have rotational frames that allow them to move limbs around their body. \textbf{[Low]} 

\textbf{Fractal Digits:} The synthmorph has ``bush robot'' digits that are capable of splitting into smaller digits, and those smaller digits into micro digits, and so on down to the micrometer scale, allowing for ultra-fine manipulation. Apply a +20 COO modifier where such fine manipulation is a factor (such as detailed repair work). The bot must have functioning nanoscopic vision (p. 311) to get this bonus. \textbf{[Moderate]} 

\textbf{Hidden Compartment:} The shell has a concealed aperture for a shielded interior compartment, ideal for storing valuables or smuggling contraband. Apply a $-$30 modifier to detect this compartment either manually or with sensor scans. \textbf{[Low]} 

\textbf{Magnetic System:} A magnetic system allows the shell to cling to most ferrous materials. This enables the character to walk in zero-G situations by magnetically adhering surfaces, hang upside down, and hold onto devices without letting them drop or drift away. The shell receives a +30 modifier whenever maintaining a magnetic hold on something. \textbf{[Low]} 

\textbf{Modular Design:} This shell is designed to lock together with similar modular morphs in different architectural patterns to create larger gestalt forms. When united with other modules, the group is treated as a single unit/morph, with shared capabilities. If damaged and then separated, damage and wounds are distributed evenly between modules; uneven amounts are allocated randomly. The exact capabilities of different shapes depends on the composition, and is largely left in the gamemaster’s hands. \textbf{[High]} 

\textbf{Pneumatic Limbs:} The limbs are equipped with pneumatic cylinder systems that can generate up to 1,500 pounds of thrust. This allows the shell to push off and make impressive jumps (a synth of human size/weight can leap over 2 meters up). Apply a +20 to Freerunning Tests. A pneumatic limb used to strike an opponent in unarmed combat inflicts an extra 1d10 damage. \textbf{[Low]} 

\textbf{Retracting/Telescoping Limbs:} The shell’s limbs can either be retracted completely inside it’s frame and/or extended for extra length (usually up to 1 or 2 meters extra). Telescoping limbs may give the shell a reach advantage in melee combat (p. 204). \textbf{[Low]} 

\textbf{Shape Adjusting:} This shell is made from smart materials that allow it to alter its shape, altering its height, width, circumference, and external features, while retaining the same mass. This modification is typically employed to reshape the morph into special configurations adapted to specific tasks (for example, lengthening to crawl through a tunnel, widening its base for stability, expanding to reach out and attach to multiple access point simultaneously, and so on). This mod also allows the morph to change its features for disguise purposes; apply a +30 modifier to Disguise Tests. \textbf{[High]} 

\textbf{Structural Enhancement:} This modification bolsters the shell’s structural integrity, boosting its ability to take damage. Increase Durability by 10 and Wound Threshold by 2. \textbf{[Moderate]} 

\textbf{Swarm Composition:} The shell is not a single unit but a swarm of hundreds of insect-sized robotic microdrones. Each individual ``bug'' is capable of crawling, rolling, hopping several meters, or using nanocopter fan blades for airlift. The cyberbrain, sensor systems, and implants are distributed throughout the swarm. Though the swarm can ``meld'' together into a roughly child-sized shape, the swarm is incapable of tackling physical tasks like grabbing, lifting, or holding as a unit. Individual bugs, however, are quite capable of interfacing with electronics. Swarms cannot carry most gear or wear armor, and may not make strength-based SOM-linked skill tests. For combat purposes, use the same rules as given for nanoswarms, p. 328. Damage and wounds are reflected as damaged/ massacred bugs. The swarm may be ``healed'' by manufacturing more bugs. \textbf{[High]} 

\textbf{Synthetic Mask:} The synthmorph is equipped with a realistic outer casing of faux-skin and carefully sculpted to pass as a biomorph (perhaps even a particular person). The morph can cry, spit, have sex, and will even bleed if cut. Only a detailed physical examination or a radar, terahertz, or x-ray scan will detect the synthmorph’s true nature, and even then such exams/scans suffer a $-$30 modifier. [Moderate] Weapon Mount: The shell carries a built-in (or builton) weapon. This weapon mount may be either internal (concealed, only weapons small in relation to the shell may fit, $-$30 to Perception Tests to detect) or external (visible). It may be fixed (one direction only), swiveling (limited field of fire), or on an articulated mount (all directions). \textbf{[Low; Moderate for concealed/articulated]} 

\subsubsection{Sensors} 

\textbf{360$^{\circ}$ Vision:} The shell’s visual sensors are situated for a 360-degree field of vision. \textbf{[Low]} 

\textbf{Chemical Sniffer:} This sensor detects molecules in the air and analyzes their chemical composition. It enables Chemistry Tests to determine the presence of gases, including toxins and other fumes. It can also detect the presence of explosives and firearms. \textbf{[Moderate]} 

\textbf{Lidar:} This sensor emits laser light and measures the reflections to judge range, speed, and image the target. See Using Enhanced Senses, p. 302. \textbf{[Low]} 

\textbf{Nanoscopic Vision:} The shell’s visual sensors can focus like a microscope, using advanced superlens techniques to beat the optical diffraction limit and image objects as small as a nanometer. This allows the character to view and analyze objects as small as blood cells and even individual nanobots. The synthmorph must stay relatively steady to view objects at this scale. \textbf{[Moderate]} 

\textbf{Radar:} This sensor system bounces radio or microwaves off targets and measures the reflected waves to judge size, composition, and motion. See Using Enhanced Senses, p. 302. \textbf{[Low]} 

\subsection{Armor} \label{sec:armor} 

Modern personal armor systems have advanced from the high modulus polyethylene thermoplastics and aramid fabrics of the early 21st century. Armor in \emph{Eclipse Phase} is derived from biotech, in the form of organoweave fibers and crystalline-grown plates, and nanotech, in the form of shock-absorbing fullerene (p. 298) materials. Occasionally other materials are used, such as metallic glass plates or shear-resistant fluids that harden against impacts. Such armor protects against (armor-piercing) bullets and kinetic impacts as well as bladed weapons and piercing sharp objects. They also insulate against both the explosive heating of energy weapons and electrical shocks. While such armor protects against bullets, the layers of material catch the bullet and redistribute its kinetic energy across the body, which can still result in severe blunt force trauma. 

Rules for armor in combat can be found on p. 194. Armored exoskeletons are listed on p. 343. 

\textbf{Armor Clothing:} The extra-resilient organoweave fibers and fullerene materials that offer basic protection against kinetic and energy weapons can be woven in with normal smart materials to create a wide range of discreet armor clothing that provides a subtle level of security. Such protective garments are indistinguishable from regular clothing and come in all styles and designs. Armor clothing provides an Armor Value of 3/4. \textbf{[Trivial]} 

\textbf{Armor Vest:} Armor vests provide more thorough protection to a body’s vital areas, covering the abdomen and torso completely, protecting the neck with a rigid collar, and even providing wrap-under protection for the groin. Though armor vests are not bulky, they are obvious as armor. Armor vests may be worn with armor clothing without penalty. Armor vests provide an Armor Value of 6/6. \textbf{[Low]} 

\textbf{Body Armor (Light):} These high performance armor outfits protect the wearer from head to toe. An integrated armor vest is supplemented with increased protection on the limbs and joints, while still managing to be flexible and non-restrictive. Body armor is typically worn by security and police forces, and supplemented with a helmet. It provides an Armor Value of 10/10. \textbf{[Low]} 

\textbf{Body Armor (Heavy):} Similar to light body armor, but with extra protective layers, often ergonomically manufactured to conform to a specific character’s body, and an environmental seal with climate control to protect the wearer from hostile environments. It provides an Armor Value of 13/13. \textbf{[Moderate]} 

\textbf{Crash Suit:} Designed for both industrial worksite safety and protection from accidental zero-G collisions, crash suits are also favored by sports enthusiasts and explorers. The basic jumpsuit offers comfortable protection equal to that of armor clothing. When activated with an electronic signal, however, elastic polymers within the suit stiffen and form rigid impact protection for vital areas. Crash suits provide an Armor Value of 3/4 when inactive and 4/6 when activated. \textbf{[Low]} 

\textbf{Helmet:} This armor accessory is usually worn with body armor or a battle suit. Light helmets are open, whereas full helmets latch on and provide an environmental seal with a 12 hour supply of air. Light helmets provide an Armor Value bonus of +2/+2, whereas full helmets add +3/+3. Helmets are often equipped with an ecto (p. 325), a radio booster (p. 313), and sensors equal to specs (see p. 325). \textbf{[Trivial]} 

\textbf{Riot Shield:} Used for mob suppression, riots shields are light-weight, tough, and may be set to electrify on command, stunning anyone who comes into contact with the outer surface (treat as shock glove effects, p. 334). Riot shields provide an Armor Value bonus of +3/+2. \textbf{[Low]} 

\textbf{Second Skin:} This lightweight bodysuit, woven from spider silks and fullerenes, is typically worn as an underlayer, though some athletes use it as a uniform. It provides minimal protection, but may be worn with other armor without penalty. It provides an Armor Value of 1/3. \textbf{[Low]} 

\textbf{Smart Skin:} Smart skin is an advanced nanofluid that covers the wearer’s skin. It resembles liquid mercury but retains the texture and flexibility of normal skin until activated, at which point the material becomes rigid enough to protect the wearer and distribute the kinetic energy (though still flexible enough at the joints not to impede movement). A specialized hive, worn by the character, replenishes the nanobots and stores them when not in use. Deploying the nanobots across the body takes a full Action Turn. Smart skin has an Armor Value of 3/2, and may be worn with other armor without penalty. \textbf{[Low]} 

\textbf{Spray Armor:} This fast armor application comes in a spray can and disperses a smart chemical polymer that sticks to bare flesh (but does not adhere to hair and eyes). The polymer solidifies into a form fitting body armor fabric when exposed to body temperature with the look and feel of a latex suit. Spray armor does not work on synthetic morphs or on clothing or other armor. The color and feel of the armor can be adjusted with electric currents and additional polymers, making it popular among some socialite and nightlife scenes. The spray-on armor does not wash off, but degrades 1 point of armor (both energy and kinetic) every 12 hours. It may be removed with a special nanotech solvent. Spray armor has an Armor Value of 2/2. \textbf{[Low]} 

\begin{table} \begin{tabular}{|l|l|l|l|} \hline

\hline{2}{|c|}{\textbf{Armor values}}	\\ \hline

\textbf{Armor}	&\textbf{Energy} &\textbf{Kinetic}	&\textbf{Page} \\ \hline

Armor Clothing	&3	&4	&311	\\ \hline

Armor Vest	&6	&6	&312	\\ \hline

Battle Suit Powered Exoskeleton	&18	&18	&344	\\ \hline

Bioweave Armor (Light)	&2	&3	&302	\\ \hline

Bioweave Armor (Heavy)	&3	&4	&302	\\ \hline

Body Armor (Light)	&10	&10	&312	\\ \hline

Body Armor (Heavy)	&13	&13	&312	\\ \hline

Carapace Armor	&11	&11	&303	\\ \hline

Crash Suit (Inactive)	&3	&4	&312	\\ \hline

Crash Suit (Active)	&4	&6	&312	\\ \hline

Exowalker	&2	&4	&344	\\ \hline

Hard Suit	&15	&15	&334	\\ \hline

Helmet (Light)	&+2	&+2	&312	\\ \hline

Helmet (Full)	&+3	&+3	&312	\\ \hline

Hyperdense Exoskeleton	&6	&12	&344	\\ \hline

Riot Shield	&+3	&+2	&312	\\ \hline

Second Skin	&1	&3	&312	\\ \hline

Smart Skin	&3	&2	&312	\\ \hline

Smart Vac Clothing	&2	&4	&325	\\ \hline

Spray Armor	&2	&2	&312	\\ \hline

Synthmorph Industrial Armor	&10	&10	&310	\\ \hline

Synthmorph Combat Armor (Light)	&14	&12	&310	\\ \hline

Synthmorph Combat Armor (Heavy)	&16	&16	&310	\\ \hline

Transporter Exoskeleton	&2	&4	&344	\\ \hline

Trike Exoskeleton	&2	&4	&344	\\ \hline

Vacsuit (Light)	&5	&5	&333	\\ \hline

Vacsuit (Standard)	&7	&7	&333	\\ \hline

\label{tab:armor-values} \label{tab:armor-values} \end{table} 



\subsection{Armor mods} \label{sec:armor-mods} 

Armor modifications add extra materials or coatings that either enhance the armor’s resistance to certain dangers or provide other effects. Armor mods may be easily added or removed with the appropriate nanobot applicators. 

\textbf{Ablative Patches:} These thin and light slap-on patches of stick to armor and are designed to absorb heat and energy from beams and explosions, safely vaporizing and blowing hot gas away. Ablative patches increases the Armor Value by +4/+2, but each hit reduces both the energy and kinetic value of the ablative armor by 1. \textbf{[Trivial]} 

\textbf{Chameleon Coating:} This provides the armor with the same effect as the chameleon cloak (p. 315). \textbf{[Trivial]} 

\textbf{Fireproofing:} Fireproofing includes the addition of heat-resistant ceramic or fire-resistant layers, both capable of withstanding extremely high temperatures. Fireproofing increases the Armor Value by +2/+0, and provides an additional 10 points of armor against heat or fire specifically. \textbf{[Trivial]} 

\textbf{Immunogenic System:} The immunogenic mod adds an active nanobot swarm, maintained by a specialized hive, that coats the outer layer of armor and also the non-armored parts of the wearer’s morph. It acts as an outer immune system designed to neutralize toxic agents and nanotoxins with which it comes into contact. This provides immunity to drugs, toxins, and nanotoxins applied dermally, such as with a slap patch or splash grenade. It has no effect on inhaled, oral, or injected drugs (including coated weapons). \textbf{[Low]} 

\textbf{Lotus Coating:} The armor has been impregnated with a superhydrophic coating (contact angle of around 170$^{\circ}$) that repels all water-like liquids. If the armor is splashed by liquid toxins or chemicals, the effect is reduced since the liquids starts to roll off the armor. Apply a +30 modifier when defending against liquid-based attacks. \textbf{[Trivial]} 

Offensive Armor: When activated, the outer layer of this armor is rigged to shock anyone or anything that contacts it with electricity. Treat its DV and effect as a shock baton (p. 334). \textbf{[Low]} 

\textbf{Reactive Coating:} A thick layer of advanced nanotech is applied to the armor, protecting it with a colony of nanobots designed to sense incoming attacks. When an attack strikes the coating, it detonates to disrupt the attack. Bursts and full autofire are treated as a single attack. A reactive coating increases the Armor Value by +5/+5, but each detonation automatically inflicts 1 point of damage on the wearer. Reactive armor also works against melee attacks, but the attacker also suffers 1d10 $\div$ 2 (round up) points of damage per attack (armor protects) from the microexplosion. Reactive coating only works against 5 attacks, after which the specialized nanobot hive replenishes the coating at the rate of 1 use per hour. \textbf{[Moderate]} 

\textbf{Refractive Glazing:} A combination of reflectors, refractive metamaterials, and an energy transfer system with heat radiators provides extra protection against energy weapons. Increase the Armor Value by +3/+0. \textbf{[Low]} 

\textbf{Self-Healing:} The armor is equipped with a nanohive that acts like repair spray (p. 333). \textbf{[Moderate]} 

\textbf{Shock Proof:} Shock proof armor is electronically insulated to discharge and reduce the effect of shock weapons. Apply an additional +10 modifier when resisting the DV and effects of shock weapons (p. 204). \textbf{[Low]} 

\textbf{Thermal Dampening:} Thermal dampening obfuscates heat signatures by converting body heat into electric energy. It makes the target more difficult to spot with thermal sensors; apply a $-$30 modifier for Perception Tests. \textbf{[Moderate]} 

\subsection{Communications} \label{sec:communications} 

The oldest and most widespread communications technology still in regular use is radio. Every habitat and world inhabited by transhumanity is awash in radio traffic, with humans, machines, and uplifts all constantly communicating with one another. The smallest radios are no larger than a spec of dust and have a range of no more than 20 meters, while the largest are the size of a truck and have a range of many thousands of miles. Radios large and small are ubiquitous and almost all devices contain at least short-range radios so they may interact with the mesh. Most morphs are equipped with basic mesh inserts (p. 300) that include an implanted radio. For radio ranges, see p. 296. 

\textbf{Fiberoptic Cable:} Fiberoptic cables are used to establish wired connections between two devices. Given the ubiquity of radios and the tangled mess wires cause, they are typically only used for privacy (unlike radio communication, fiberoptic signals may not be intercepted) or in areas with heavy radio interference. \textbf{[Trivial]} 

\textbf{Laser/Microwave Link:} These portable devices are used to establish a tight-beam, line-of-sight communications channel with another laser or microwave link. The range of these transceivers varies widely with environmental factors, but approximates 50 kilometers in atmosphere and 500 kilometers in space (though horizon limits must be kept in mind, being 5 kilometers at ground level on Earth and less on smaller bodies). Lasers are subject to interference from fog, dirt, smoke, and similar visual chaff, while microwaves may be hindered by metallic obstructions. These links may only be intercepted by getting directly in between the beams. Some teams carry a micro version of this system, worn on their person, allowing line of sight intra-team communications that cannot be intercepted like radio. \textbf{[Moderate]} 

\textbf{Radio Booster:} This device boosts the range and sensitivity of short-range radios, like those from implants, ectos, or microbugs. The booster must be with the shorter-ranged device’s range (or directly linked via fiberoptic cable). It will repeat any transmissions received from that device, but at its extended range of 25 kilometers in urban areas (250 kilometers remote areas). Broadcasts from a radio booster are easy to receive by anyone looking for broadcasts (see Wireless Scanning, p. 251), though transmissions may be stealthed (p. 252). Boosters are commonly used by characters traveling far from habitats or other civilized regions. \textbf{[Low]} 



\subsection{Neutrino communicators} \label{sec:neutrino-communicators} 

Neutrinos are particles that can pass through any solid matter with ease and are impossible to block. As a result, they make an ideal medium for communications. Unfortunately, they are also easy to intercept. Even a tight beam of neutrinos sent between two locations can be intercepted simply by placing another receiver behind the location the broadcaster is sending to. Neutrino communicators require a large power plant to power the high energy particle interactions required to generate the neutrino broadcast. Neutrino receivers are also relatively large, with the smallest occupying 100 cubic meters. In most cases, neutrino communicators are designed to broadcast neutrinos in all directions, though tight-beam transmissions are also possible. Quite often neutrino communications take advantage of quantum farcasting for security. 

\textbf{Neutrino Transceiver:} This transceiver is capable of generating and receiving neutrino signals at a range of at least 100 astronomical units. It is large, with a size of 8 cubic meters (in a cube 2 meters on a side), but they can be loaded onto large vehicles. To function, it must be connected to a large power plant, such as one found in habitats or large spacecraft. The cost and size of this device includes the computer necessary for quantum farcasting. \textbf{[Expensive]} 

\subsection{Quantum farcasters} \label{sec:quantum-farcasters} 

Quantum farcasters are special computers designed to protect a communications channel (such as fiberoptic, radio, laser/microwave, or neutrino) with unbreakable encryption. To function, two or more quantum farcaster computers must first be entangled together (on a quantum level) in the same physical location. The farcasters may then be separated, at which point they may continue to exchange encrypted data via quantum teleportation. This data exchange requires a standard communications link (fiberoptic, radio, laser/microwave, or neutrino), and so is limited by the speed of light, but it is a high bandwidth form of communications. The quantum encryption used by these entangled farcasters is unbreakable, and any attempted interception is immediately detected and neutralized. A quantum farcaster may not be used to securely communicate with any farcasters other than the ones it is entangled with. 

Because it is exceptionally safe and secure, quantum farcasting via neutrino communications is the primary means of both long-distance communication between habitats and egocasting (p. 276). The neutrino signal cannot be blocked and it can only be decrypted if a character has access to the computer that is sending or receiving the signal. 

\textbf{Miniature Radio Farcaster:} Miniature farcasters communicate with each other using standard radio transceivers. As noted above, they may only securely communicate with the other farcasters with which they are entangled. Most miniature farcasters are worn as jewelry or fitted into clothing or other equipment. \textbf{[Low]} 



\subsection{Quantum entanglement communication} \label{sec:quantum-entanglement-communication} 

The rarest form of communications is quantum entangled (QE) communication. QE communication is instantaneous and works over any distance, but is also very limited. QE communication requires pairs of entangled particles known as qubits. To use QE, large number of pairs of qubits are created and then separated from each other. Millions of these separated pairs of particles are stored in special containers known as qubit reservoirs. If two QE communicators each have a qubit reservoir containing qubits that are each entangled with qubits in the other communicator’s qubit reservoir, then characters can use the two QE communicators to commutate with one another instantaneously. Characters can use QE to instantly communicate between any two locations, even if one character is in the solar system and the other has passed through a Pandora gate and is standing on a planet 500 light years away. 

Each bit of data transmitted between these two QE comms uses up one qubit. Once all of the qubits are used up, the two QE comms can no longer communicate with each other until they each get a new batch of entangled qubits. Qubits are expensive to produce, contain, and transport, making this an exceedingly expensive form of communication. As a result, extremely high bandwidth communications like full sensory AR and egocasting cannot be performed using QE communication. 

\textbf{Portable QE Comm:} This is a handheld FTL communications device. The actual communications unit can be made as small as desired, but must be large enough to connect to or hold a qubit reservoir. Because qubit reservoirs are relatively large and must be replaced, they are rarely implanted. Some miniature farcasters are designed so that users can also attach qubit reservoirs to enable them to be used for both light speed and FTL communication. \textbf{[Low]} 

\textbf{Low-Capacity Qubit Reservoir:} Low-capacity qubit reservoirs can be used for 10 hours of high-resolution video conferencing or meshbrowsing and 100 hours of voice or text only communications. \textbf{[High]} 

High-Capacity Qubit Reservoir: High-capacity qubit reservoirs can be used for 100 hours of high-resolution video conferencing or meshbrowsing and 1,000 hours of voice or text only communications. \textbf{[Expensive]} 



\subsection{Covert and espionage technologies} \label{sec:covert-espionage-tech} 

These technologies allow characters to acquire protected information and to gain access to places that others try to keep them out of. Many of these devices are mesh-capable and equipped with radios, see p. 296 for radio ranges. 

\textbf{Chameleon Cloak:} This loose, poncho-like cloak contains a network of sensors that perceive wavelengths from microwave to ultra-violet. A similar network of miniature emitters precisely replicate the information its sensors receive, making the wearer seem transparent to those wavelengths. A chameleon cloak allows a character to effectively become invisible as long as they are stationary or not moving faster than a slow walk. When worn by someone moving faster, the cloak still provides a +30 modifier to Infiltration Tests to avoid being seen or noticed. 

Chameleon cloaks are not effective against radar, x-ray, or gamma-ray sensors. They do hide the character from thermal infrared, however, by absorbing the character’s body heat into its heat sink. The cloak can only absorb a character’s body heat for one hour before it must emit this heat. Heat emission also requires one hour, during which time the character is easily visible in the thermal infrared spectrum. \textbf{[Low]} 

\textbf{Covert Operations Tool (COT):} This handheld device is the ultimate in infiltration technology. It contains both smart matter micromanipulators, cutting tools, and an advanced nanotechnology generator capable of producing nanobots that can bore or cut through almost any material and disable or open almost any electronic lock. 

Cutting out a lock or boring a 1-millimeter hole in a wall with a COT requires ((Durability + Armor) $\div$ 10) seconds. Cutting out a 1-meter diameter hole in a wall requires ((Durability + Armor) $\div$ 10) minutes. These same nanobots can later be used to repair this damage so that it is invisible to any but the most careful and detailed examination. 

A COT can easily open any old-fashioned mechanical lock simply by analyzing it and shaping an appropriate key, though this takes a full Action Turn. It can also open electronic locks by infiltrating them with nanobots that influence the lock’s electronics, no matter what authentication system the lock uses. Opening electronic locks takes a full Action Turn, but success is practically guaranteed. Opening an electronic lock in this manner will, however, trigger an alarm and/or be logged as an event. For more details, see \textbf{Electronic Locks}, p. 291. \textbf{[High]} 

\textbf{Cuffband:} This smart plastic loop restricts around a prisoner’s limbs when activated. If the prisoner struggles, it will tighten more. Cuffbands will inform the user if they are cut or loosened and are electronically- controlled, so the user can release the prisoner remotely. Some cuffband variants including a shock system (treat as a shock baton, p. 334) to zap and restrain unruly prisoners. \textbf{[Low]} 

\textbf{Dazzler:} The dazzler is a tiny laser system set on a rotating ball. When activated, it consistently spins and emits laser pulses in all directions. These laser pulses are not dangerous, but they detect the lenses of camera systems (including specs, viewers, and bot/ synthmorph sensors) and repeatedly zap them with laser pulses of varying strength to overload and dazzle them. For as long as a dazzler is active, any camera system (visual, infrared, and ultraviolet) within line of sight and within 200 meters is blinded. \textbf{[Moderate]} 

\textbf{Disabler:} This handy device emits an overloading surge that completely incapacitates and disables a synthetic morph or pod (anything with a cyberbrain) when it is plugged into an access jack and activated. The affected cyberbrain will be unable to function until the signal is deactivated, effectively shutting down the ego (or AI). In order to plug a disabler into an unwilling target, the target must first be grappled or a called shot must be successfully made in melee combat. This device does not work on larger synthetic morphs (like vehicles) or on cyberbrainless robots. \textbf{[High]} 

\textbf{Fiber Eye:} This is a flexible and electronically-controllable length of fiberoptic cable and viewer, which can be worked through cracks, under doors, and around corners to peep unobtrusively. \textbf{[Low]} 

\textbf{Invisibility Cloak:} This cloak is made of metamaterials with a negative refractive index, so that light actually bends around it, making it and anything it covers invisible. This invisibility works from the microwave to ultraviolet spectrums, but not against radar or x-rays. The drawback is that anything concealed within the cloak can’t see out. This is easily overcome by using external sensor feeds (if available) and entoptics to navigate. Alternately, a small piece of anti-cloak, which cancels the cloak’s invisibility properties when touched together, can be used to create a small window to peep out of, though this increases the chance of being spotted. Noticing such a window requires a Perception Test with a $-$30 modifier. \textbf{[High]} 

\textbf{Microbug:} This device is a tiny camera and microphone 1 millimeter across. It has the visual capabilities of a set of specs (p. 325). It can hear everything within 20 meters and see everything within the same range that is in its line of sight. A microbug can record up to 100 hours of information. Microbugs can be set to broadcast continuously, at set intervals, or only when they receive a special signal. If desired, they can also be set to only record if there is movement or voices in the room they are in. Microbugs have adhesive backs and can stick to almost any surface. Microbugs can also establish their location via mesh positioning or GPS, and so double as tracking devices. To avoid being detected by their radio transmissions, some microbugs are attached to miniature quantum farcasters (p. 314). These microbugs are much larger (1 centimeter) and easy to see, but their transmissions cannot be detected or blocked. \textbf{[Trivial, Low for quantum farcaster bugs]} 

\textbf{Prisoner Mask: }This hood tightens around the head of a prisoner, blocks all vision frequencies, and engages in low-level jamming in order to prevent any wireless communication via mesh inserts. \textbf{[Medium]} 

\textbf{Psi Jammer:} This device jams frequencies used by brainwaves within a 20-meter radius. This has no effect on brain functions, but it does prevent any ranged used of psi sleights within this area of effect. \textbf{[Moderate]} 

\textbf{Quantum Computer:} These advanced devices make use of quantum computation, allowing them to handle extremely large numbers with ease. This makes them especially useful for codebreaking, as noted on p. 254. \textbf{[Expensive]} 

\textbf{Smart Dust:} This device is a walnut-sized specialized nanobot generator that creates tiny sensor nanobots, each one of which is a tiny sphere the diameter of a human hair. A packet of smart dust nanobots is sufficient to perform detailed surveillance on a large room like an auditorium has a volume of 1 cubic centimeter and contains 3 million nanobots. Each nanobot contains tiny cameras, microphones, a tiny computer, a radio, and chemical sensors, as well as short legs that allow them to walk and climb at a rate of 5 cm per second. 

When a character dumps a packet of smart dust in a room, it will cover every surface in the room within 20 minutes, including all furniture and the insides of every drawer and other space that is not airtight. At this point, the smart dust has recorded all data about the room that can be obtained by exceedingly detailed observation, including the DNA of everyone who has visited the room in the last week or two. The smart dust can then either broadcast a brief, highly compressed signal, or it can send all of its information to a few hundred nanobots that then walk to a pre-arranged destination for pickup and downloading by their user. The user need only find a single nanobot with a nanodetector to acquire the information obtained by the smart dust. If ordered to do so, the remaining nanobots can either power down and await further orders or self-destruct in a fashion that turns them into a tiny amount of dust made mostly of metal and silicon. \textbf{[Moderate]} 

\textbf{Traction Pads:} This set of specialized fingerless gloves, shoes, and kneepads is designed to emulate the pads on geckos’ feet. Characters can support themselves on a wall or ceiling by placing any two of these pads against any surface not made from a material specially designed to resist such devices. Characters can climb any surface and move easily across walls and ceilings that can support their weight (+30 to Climbing Tests). In addition to climbing, these devices are also very popular in zero-g environments. Wearing this item does not impair the user’s agility or manual dexterity. \textbf{[Low]} 

\textbf{White Noise Machine:} This small and wearable device generates masking sounds that protect a conversation from being audibly recorded or overheard by anyone not in the immediate vicinity. \textbf{[Trivial]} 

\textbf{X-Ray Emitter:} This device is designed to be used with either the enhanced vision augmentation (p. 301) or specs (p. 325). It emits a focused beam of low-powered x-rays that allows the user of either device to both see and see through most objects using backscatter x-ray radiation (p. 303). This allows the character to literally see through walls and into containers, including ones made of metal. \textbf{[Low]} 

\subsection{Bugs and surveilance} \label{sec:bugs-surveilance} 

Though surveillance technologies are pervasive and easy to come by in Eclipse Phase, secretly obtaining information on someone who wants to retain privacy can be quite difficult. Microbugs, smart dust, and similar recording devices that are all but invisible may be exceptionally easy to put into place, but once they begin actively transmitting, they are easy to to detect (see Wireless Scanning, p. 251). An eavesdropper may attempt to stealth the signal (see Stealthed Signals, p. 252), but this is not guaranteed to work. Once a signal is detected, locating the broadcasting device is usually just a matter of time (see \textit{Tracking}, p. 251). 

Some recording devices attempt to avoid this problem by using miniature quantum farcasters (p. 314), but those are far larger and more difficult to hide. Often the most effective way to acquire discrete information is to plant a surveillance device, set to record but not transmit, and then retrieve it later. While doing this is often difficult and risky, the recording device never reveals its presence by broadcasting and so is more difficult to detect. 

\section{Drugs, chemicals and toxins} \label{sec:drugs-chemicals-toxins} 

In \emph{Eclipse Phase}, the transhuman desire to enhance the body and mind --- especially with chemicals --- merges right into humanity’s popular pastime of recreational substance abuse. Drugs of all kinds, whether they be chemical, nano-based, or electronic, are not only popular but widespread. While advances in biotechnology have eliminated many of the side effects that once plagued drug users, transhuman bodies remain complicated environments, and so side effects (especially with long-term use) are still a factor. Additionally, addiction is always a consideration for anyone who gets comfortable with popping the same pills too often, though there are also drugs for addiction of course. 

Drug descriptions include benefits, side effects, noticeable signs that a person is using the drug, addictiveness, and effects from long-term use). Descriptions also include the drug’s Duration and its Addiction Modifier (see \emph{Addiction and Substance Abuse}). 

\subsection{Substance rules} \label{sec:substance-rules} 

These rules explain how to handle drugs and toxins. 

\subsubsection{Classification of substances} 

Substances fall into four categories: 

\textbf{Chemicals:} These are pharmacologically-active small chemical compounds (toxins, pharmaceuticals, chemical drugs) that have been produced by chemical synthesis, nanotech fabrication, or enzymatic biosynthesis in (transgenic) organisms. They include naturally- occurring drugs from known species of (exo-)flora and fauna, endotoxins produced by biological organisms, enhancements of endogenic substances (designer drugs), and de novo developments designed for a specific medical or recreational application. Chemical drugs affect only biological morphs and pods. 

\textbf{Biologicals:} These include peptides, hormones, and biologically-based substances like biotoxins, bacteria, and viral organisms --- drugs devised or based on naturally-occurring endogenic biological substances. This category also includes infectious biological organisms that can produce drug-like effects, like virii and bacteria. Biologicals affect biomorphs and pods but not synthetic morphs or infomorphs. 

\textbf{Nanodrugs:} These are temporary nanobot colonies programmed to create a certain effect. While nanobots are generally able to target or infect all morph types except infomorphs, exactly which morphs are affected usually depends on the pre-programmed effect (i.e., whether it targets a biological or mechanical mechanism). 

\textbf{Electronic:} Electronic drugs include software and technology that affect the brain directly, such as manipulative XP programs or retro-tech like transcranial magnetic stimulation or cranial electrotherapy. It also includes narcoalgorithms --- programs that reproduce drug-like effects for AIs, infomorphs, and egos residing in cyberbrains. 

\subsubsection{Application methods} 

There are number of vectors by which a substance may be applied to a morph. 

\textbf{Dermal (D):} This drug or toxin is absorbed via the skin (or exterior hull with some nanotoxins) as either a gas, liquid, or solid (e.g., paste). Slap patches and slap bands are commonly used, loaded with the chemical DMSO, which transfers the drug through the skin. 

\textbf{Inhalation (INH):} This is a gas that is breathed into the lungs or snorted nasally. Used for inhalers, aerosols, powders, and gas grenades/seekers. 

\textbf{Injected (INJ):} This liquid is applied via either an intramuscular or intravenous injection. Used for needles and piercing weapons. 

\textbf{Oral (O):} This is a liquid or solid that is absorbed through the stomach or oral cavity (eating or drinking). Used with pills and liquids. 

\subsubsection{Drug effects} 

If a character is exposed to a drug via its method of application --- for example, they pop a pill, slap on a dermal patch, are soaked with a splash grenade, breathe in gas, or get stabbed with a coated weapon --- then they are subject to the drug’s effects. The onset time determines how long these effects take to kick in, and the duration determines how long they last. While there is no resistance test to ignore a drug or toxin’s effects once exposed, in some cases (especially toxins) a test might be called for to determine the \emph{severity} of the effects. 

Unless otherwise noted or specifically overridden, medichines (p. 308) will protect a character from drug/toxin effects (but not nanodrugs/nanotoxins). Enhancements like toxin filters (p. 305) may also impede a drug’s effect or provide complete resistance. If an antidote is taken in advance or before the effects kick in, the drug will not work. 

\subsubsection{Addiction and substance abuse} 

Some drugs are addictive, either physically (affecting the morph) or mentally (affecting the ego) --- and sometimes both. Every time a character uses the drug (or after an appropriate amount of use, as determined by the gamemaster), they must make a WIL $\times$ 3 Test to avoid addiction. Each drug has an Addiction Modifier that will modify this test. 

Failure indicates that the character has become addicted --- they immediately acquire the Addiction negative trait (p. 148). Addiction is measured in three levels: Minor, Moderate, and Major. The severity determines how often an addicted character needs the drug and what the negative effects of not using the drug are. 

An addicted character must continue to make WIL $\times$ 3 Tests as they use the drug, as determined by the gamemaster. Failure indicates the character’s addiction severity increases. 

The negative effects from not using a drug end whenever the character does the drug again. Durability and Lucidity penalties are not damage, but temporary decreases to the character’s maximum values; the character immediately regains the lost Durability or Lucidity when they do the drug again. 

Addiction is of indefinite duration. To clean up, the character must stay off the drug for 1 week for each level of addiction. Resisting this craving is difficult, and should at least require another WIL $\times$ 3 Test, modified by the drug’s Addiction modifier. Players and gamemasters are encouraged to roleplay an attempt to kick a habit. Each week the character is off the drug, the addiction drops by one level. When it reaches 0, the character is clean ... though there is always danger of a relapse. 

Physical addictions do not carry over to a new morph if the character resleeves, but mental addictions do. If the character uploads and resleeves, the mental addictions persist, and the morph the character leaves behind remains physically addicted. This means that poor or unlucky characters may occasionally find themselves resleeved into a morph that has a physical addiction. In this case, the character is subject to the physical addictiveness of the drug but not the mental addiction, although if they break down and indulge in the drug, they may themself become physically addicted. 

Characters who resleeve as infomorphs can remain mentally addicted to a substance despite no longer having a body. The market is always happy to provide, though; a wide variety of narcoalgorithms mirroring the effects of most of the drugs described below are available for infomorphs and AIs. For the infomorphported narcoalgorithm version of any physicallyonly addictive drug described below, consider the Addictiveness to be effectively physical. The character remains addicted as long as they are an infomorph, but they do not remain addicted if they sleeve into a physical morph. 

\subsection{Drugs} \label{sec:drugs} 

The drugs described here are usually (but not always beneficial), and are typically taken intentionally. Drugs and chemicals used offensively are described under Chemicals and Toxins, both on p. 323. Note that the drugs here are just a representative sampling. There are thousands if not millions of drugs in circulation in \emph{Eclipse Phase} --- gamemasters are encouraged to introduce their own, using these as guidelines. 

\subsubsection{Cognitive drugs} 

Nootropics and similar drugs are intended to boost the user’s mental faculties. 

\textbf{Drive:} This nootropic speeds up left-right brain hemisphere communication, stimulates idea production, and improves concentration, with no usual side effects. Users receive a +5 bonus to COG while the drug lasts. \textbf{[Low]} 

\textbf{Klar:} Klar boosts alertness and enhances clarity and perception. Users report a feeling of being ``elevated'' to a higher level. They receive +5 INT while the drug lasts. \textbf{[Low]} 

\emph{Neem:} Neem is a mnemonic drug that works by ``tagging'' experiences and mental input with a set of unique sensations that contribute to the formation of state-based memories. Neem gummy chews come in a variety of fruit flavors shaped like extinct old Earth animals. Neem gives characters a +20 bonus on COG Tests to recall information they learned while on Neem (see \emph{Memorizing and Remembering}, p. 176). The drawback to Neem is that memories they accumulate while under the drug’s influence have no emotional association. For example, a character who witnessed something horrible happening to a friend or who had a fight with a romantic partner while on Neem would feel no emotional connection whatsoever to what happened. \textbf{[Moderate]} 

\subsubsection{Combat drugs} 

Combat drugs are an easy way of evening the odds in a fight. 

\textbf{BringIt:} In some respects more a social than a combat drug, BringIt stimulates massive bursts of aggression pheromones designed to make the user the center of attention in a fight. In combat, opponents within 3 meters of the character not already in unarmed or melee combat with another character must pass a WIL $\times$ 3 Test or attack the character using BringIt. The nature of airborne pheromones is imprecise, however, so if the character using BringIt is within 1 meter of another character hostile to the character affected, the affected character may opt to attack the proximate character instead of the BringIt user. Characters using this drug suffer a $-$20 modifier on social skill tests. \textbf{[Low]} 

\textbf{Grin:} Grin is an effective opiate and pain suppressant. Users may ignore the $-$10 modifiers from 2 wounds (not cumulative with similar effects), and in fact may not even be aware they are injured. Grin users suffer from tunnel vision, however, and so suffer a $-$10 modifier on Perception Tests. \textbf{[Low]} 

\textbf{Kick:} Kick is a strong stimulant that increases the user’s response time and puts them on edge. The character gains +10 REF and +1 Speed for the duration of the drug. Characters under the influence of Kick are twitchy, however, reacting in a jumpy, cat-like fashion to sudden or unexpected stimuli. At the gamemaster’s discretion, they must make a WIL $\times$ 2 Test or react without thinking towards unexpected noises or other surprises. Long-term users suffer $-$5 COO. \textbf{[Moderate]} 

\textbf{MRDR:} MRDR is a straightforward and brutal combat drug. It increases pain tolerance, speed, and strength. The character receives +10 SOM, +1 Speed, +10 Durability, and may ignore the $-$10 modifier of one wound. Any damage incurred while under the effects of the drug is taken from the bonus Durability first. MRDR users are easily identifiable by the broken blood vessels in their eyes, tense posture, and visible tension in the muscles of the face, arms, and legs. Long-term users suffer $-$5 SOM. \textbf{[Low]} 

\textbf{Phlo:} Phlo increases alertness and coordination, making the user more graceful and nimble in a fray. The character gains +5 COO and +10 on Perception Tests for the duration of the drug. Everything feels possible to a character on Phlo, and so they are vulnerable to being goaded into actions that might be foolish or dangerous (apply a $-$10 modifier to appropriate Social Skill Tests). \textbf{[Moderate]} 

\subsubsection{Health drugs} 

Pharma-foods that boost the consumer’s health and physical state are common. 

\textbf{Bananas Furiosas:} This drug reverses some of the effects of de-ionizing radiation on the cells of the body. Although a pill form is available, it most commonly comes in large bunches of bright orange-red bananas. Bananas reduce the severity of a radiation dosage (gamemaster determines effect). \textbf{[Low]} 

Comfurt: This tasty yogurt treat blocks stress hormones, stabilizes mood, and relieves anxiety, allowing them to ignore the effect of 1 trauma and temporarily boosting Lucidity by +5. Any stress suffered while the drug is in effect is taken from the bonus Lucidity first. Comfurt also provides a +10 bonus when resisting attempts to manipulate the user’s emotions. Excessive use of Comfurt can lead to chronic itchiness caused by histamine release. [Low] 

\subsubsection{Recreational drugs} 

These drugs compete with petals (p. 321) and black market XP for wasting people’s time and lives away. 

\textbf{Buzz:} This gene-modified variant of BZ is an odorless, invisible, extremely powerful hallucinogen. Users or affected characters will undergo extremely realistic hallucinations for the duration, and may even ``share'' hallucinations with other affected characters. Characters will suffer a $-$30 modifier to any tests to remember what occurred while under the influence. \textbf{[Moderate]} 

\textbf{Mono No Aware:} Taken from the Japanese term for sadness at the ephemerality of worldly things, this drug, typically ingested as a tea, is a depressant that induces a meditative state. Mono No Aware gives the character a +10 bonus on Art and Sense Tests. With frequent use, Mono No Aware reacts with pigments in the skin to create a pallor with a slight bluish tinge, even in darker-skinned morphs. \textbf{[Low]} 

\textbf{Orbital Hash:} Good ol’ reefer --- but grown in space using powerful lighting and post-singularity hydroponics. Because space is at a premium in habitats and scum barges, blocks of hashish are the preferred mode of transport and delivery. However, for the wealthy and on planets, buds in leaf form are not uncommon. Hash allows the character to ignore the effects of 1 trauma, but inflicts a $-$10 penalty on all memory-related tests and Knowledge Skill Tests. Hash users exhibit bloodshot eyes, lethargic behaviors, and the munchies. \textbf{[Low]}\subsubsection{Social drugs}These social lubricants affect the user’s interactions with others. 

\textbf{Alpha:} Alpha is a more subtle version of BringIt, popular with hypercorp execs, street thugs, and anyone else who wants to come across as a domineering asshole. The pharm designer who invented it had a retro sensibility (and maybe a sick sense of humor); Alpha is typically synthesized as a sparkling white powder designed to be snorted. Alpha stimulates production of threat pheromones, but less bluntly than BringIt. Alpha imparts confidence, a feeling of power, and alertness. Users can function without sleep for 4 days, after which point they need to catch up with at least 4 hours of sleep (remember morphs with basic biomods require less sleep). Dosed characters receive a +20 modifier on Intimidation Tests and +10 on Persuasion and Networking Tests where attitude is a factor (gamemaster discretion). These bonuses only apply to characters within 2 meters of the Alpha user. 

On the downside, alpha users are impatient, unfocused assholes. At the gamemaster’s discretion, Social skill modifiers may be reversed to penalties with certain types of people. Additionally, Alpha users suffer $-$10 on all COG skill tests related to memory and coherent or logical thinking. Long-term users may suffer the COG penalty even when not on the drug; on it, they may be worse. \textbf{[High]} 

\textbf{Hither:} Want to ooze sexy like a pleasure morph on a hot tin roof? For those desiring that slinky je-nesais- quoi, Hither is the tool. Hither is a clear, slippery gel, sometimes with a faint, musky, floral scent. Hither is applied to parts of the body with large concentrations of sweat glands, where the skin quickly absorbs it. Hither is a mild euphoriant, imparting a feeling of confidence and you-know-you-want-it-ness to the user. It also stimulates abundant production of lust pheromones. The character gains a +10 bonus on Persuasion Tests against targets who are possible to seduce. At the gamemaster’s discretion, this extends to Deception, Impersonate, and Networking Tests. \textbf{[Low]} 

\textbf{Juice:} This potent anti-depressant makes it almost impossible to have bad feelings or negative thoughts. The character is unnaturally happy --- often irritatingly or strangely so. The character receives a +30 bonus against fear or attempts to manipulate their emotions in a negative direction, but is also likely to act inappropriately, like giggling over the massive amount of spilled blood or cheerfully changing the subject to inane topics when someone else is freaking out. \textbf{[Low]} 



\begin{table} \begin{tabularx}{\hline}{|X|l|l|l|l|X|X|} \hline

\hline{7}{|c|}{\textbf{Drugs}}	\\ \hline

&\textbf{Type}	&\textbf{Application}	&\textbf{Onset}	&\textbf{Duration}	&\textbf{Addiction\newline mod}	&\textbf{Addiction\hline type} \\ \hline

\hline{7}{|l|}{\emph{Cognitive Drugs}}	\\ \hline

Drive	&Chem	&O	&20 min &8 hours	&$-$	&Mental	\\ \hline

Klar	&Chem	&O	&20 min	&8 hours	&$-$	&Mental	\\ \hline

Neem	&Chem	&O	&20 min	&12 hours	&$-$	&Mental	\\ \hline

\hline{7}{|l|}{\emph{Combat drugs}}	\\ \hline

BringIt	&Bio	&Inh, Inj, O	&1 min	&15 min	&+10	&Physical	\\ \hline

Grin	&Chem	&Inh, Inj, O	&3 turns	&3 hours	&$-$10	&Physical	\\ \hline

Kick	&Chem	&Inh, Inj, O	&3 turns	&2 hours	&$-$10	&Physical	\\ \hline

MRDR	&Chem	&O	&20 min	&1 hour	&$-$10	&Physical	\\ \hline

Phlo	&Chem	&O	&20 min	&1 hour	&$-$10	&Physical	\\ \hline

\hline{7}{|l|}{\emph{Health drugs}}	\\ \hline

Bananas Furiosas	&Chem	&O	&20 min	&1 day	&$-$	&$-$	\\ \hline

Comfurt	&Bio	&O	&20 min	&12 hours	&$-$10	&Mental	\\ \hline

\hline{7}{|l|}{\emph{Recreational drugs}}	\\ \hline

Buzz	&Chem	&Inh, O	&1 hour	&36 hours	&$-$	&Mental	\\ \hline

Mono No Aware	&Chem	&O	&20 min	&8 hours	&$-$10	&Mental	\\ \hline

Orbital Hash	&Chem	&Inh	&3 min	&3 hours	&$-$	&Mental	\\ \hline

\hline{7}{|l|}{\emph{Social drugs}}	\\ \hline

Alpha	&Bio	&Inh	&1 minute	&2 hours	&$-$10	&Mental	\\ \hline

Hither	&Bio	&D	&1 minute	&6 hours	&$-$10	&Physical	\\ \hline

Juice	&Chem	&O, Inh	&20 min	&8 hours	&$-$	&Mental	\\ \hline

\end{tabularx} \end{table} 

\clearpage



\label{tab:drugs} 



\subsection{Nanodrugs} \label{sec:nanodrugs} 

Nanodrugs are temporary nanobot infestations that apply a specific effect. 

\textbf{Frequency:} Frequency (or Freeq) is a nanodrug designed as a tool for scientific visualization. It releases a small swarm of nanobots into the character’s bloodstream that settle in the epidermis, where they act as sensors of electromagnetic radiation. This sensory input is then injected into the character’s visual and tactile sensoria, hitting the user with a sequence of novel stimuli, typically a light show or weird tactile sensations. Aside from its recreational uses, Frequency is good at picking up on localized field radiation with a standard Perception Test. A character can take advantage of this to spot sensors and hidden electronics. Similar to now-obsolete 20th-century hallucinogens like LSD and psilocybin, however, a Frequency trip can be disorienting and upsetting (the gamemaster should apply any modifiers, mental stress, or even trauma as they feel appropriate). Characters typically experience a period about 1/3 of the way through their trip in which sensory input is extremely intense; during this period, which usually lasts about 2 hours, they are unable to read. \textbf{[Moderate]} 

\textbf{Gravy:} Gravy assists characters in acclimating to high gravity environments. It comes in a variety of flavors and is often added as a sauce to food. For Gravy to be 100\% effective, the character must begin using it in advance. Reduce penalties for high-gravity acclimation by 20. \textbf{[Low]} 

\textbf{Schizo:} Schizo is a nanodrug that mirrors the effects of paranoid schizophrenia. It is popular in some hyperelite social circles as a truly daring and intriguing experience. A dose of schizo looks like a disposable antique razor blade. Making an incision in the skin releases a swarm of nanobots that travel to the central nervous system and induce the effects of the drug. While in effect, the character is severely paranoid and hears voices. How this plays out is at the discretion of the gamemaster, but should include irrational fears, unusual compulsions based on the instructions of the voice or voices, and a strong possibility that the character will behave in a violent or destructive fashion. The character may make WIL $\times$ 3 Tests to avoid violent acts against objects or strangers. Friends and trusted acquaintances are probably less likely to be targets of violence (+30 modifier to avoid hurting people the character cares about or destroying important possessions). Note that the character’s muse is unaffected by Schizo and can make efforts to babysit the character. Characters who take Schizo suffer 1d10 mental stress. \textbf{[Low]} 

\subsubsection{Petals} 

Petals is a term for a type of narrative hallucinogen, a nanodrug that hijacks the senses and takes the user on a game-like, highly immersive trip. Known by a myriad of intriguing names --- Forgotten Hand, Darkly Selving, Inquisitive Green, to name a few --- Petals are post-Fall society’s heroin --- the drug of choice for the desperate and fucked. Petals almost always appear as nanopharmaceutical flowers, potted or with a nutrient pack attached to the stem. Plucking and swallowing the petals from the flower triggers the effects immediately. Flowers have 5-10 petals. Multiple users may share the experience if they take the Petals within 1 minute of the first one being plucked; after this all petals remaining on the flower fade to translucent white and become inert. 

Petal experiences are like entire scenarios in and of themselves. Some take place entirely in the user’s mesh inserts (the user must cede control of their implants voluntarily; if they do not, the drug has no effect other than producing very low-intensity LSD-like visual hallucinations), taking control of the character’s entoptic displays, linking to secretive mesh servers and other trippers, and invading the character’s sensorium with AR ``hallucinations.'' Others put the character into a near-comatose state during which they go on a head trip. Normally there is some kind of well-developed theme or plot to a Petal experience, although in some cases they just experience a stream of images. 

Though most societies seek to suppress Petals, new ones appear constantly, fueled by a persistent subculture of crafters and users. Petalcrafters view their work as an art form (or at least as really good entertainment), and the better Petals are lovingly crafted, hauntingly beautiful experiences --- even if they’re also terrifying. The subculture of Petal use ranges from casual users who occasionally do an easy, short-duration flower to hardcore addicts who spend much of their time not on Petals trying to hunt down the most intense and esoteric varieties. From this subculture comes a lot of information on what various Petals look like and their effects. Because Petals combine custom nanobots with tailored chemical payloads and sometimes connections to mesh servers, duplicating them using fabricators is impossible, leading to an active market of crafters, dealers, and traders. 

Petals sometimes contain easter eggs and rewards, called ``sweets'' by petal users. Getting the sweets usually requires fulfilling certain conditions within the trip, such as correctly answering questions or fulfilling goals. Typical sweets include skillsofts, new clothing or product designs, and custom infomorph sleeves. 

On the negative side, some Petal trips go bad, flooring 1d10 mental stress or more on the user. Perhaps worse, some Petals are loaded with malware that takes over the user’s mesh inserts and worse --- some sentinels even whisper of Petals carrying strains of the Exsurgent virus. \textbf{[Trivial to High]} 

\subsubsection{Sample petals} 

A few examples of Petal experiences: 

\textbf{Forgotten hand} 

One of the character’s hands detaches and makes a run for it. The character is conscious and able to interact normally with the real world, but they cannot perceive the ``escaped'' hand and firmly believe that it’s getting away. The hand will lead the character a merry chase, but at some point, a new hand appears on the character’s wrist. It may be glittery and opalescent, demonic and clawed, or bestial. Eventually, after an hour or two, the character will catch up to their hand, but to get rid of their new hand and re-attach the old, they must answer cryptic questions posed by a gnome-like being. 

\textbf{Darkly selving} 

This petal is believed to achieve many of its effects by connecting to the mesh, where an AI observes and controls some of the event flow, and only works for multiple trippers. Like Forgotten Hand, it works by overlaying AR perceptions on the real world, but because of the effects, it’s highly inadvisable to take in places where any non-trippers will be present. Darkly Selving creates an epsilon fork of each character tripping and sleeves the fork in an infomorph that looks like a demonic version of themself, using visual input from the character’s co-trippers. AR overlays cause the characters to perceive themselves as angelic beings, while the realseeming demonic infomorphs appear as AR overlays on their real world perceptions. What happens next varies, but generally both the characters and their forks are subjected to a series of strong chemical and narcoalgorithmic stimuli, ranging from Hitherlike effects to massive doses of MRDR (or sometimes both). The effects directed against the forks are generally much more intense. The objective --- hinted at via environmental clues --- is to merge with one’s fork, which can be accomplished in a variety of ways, ranging from hunting them down and eating their heart to solving a puzzle or reaching a goal before their forks can. 

\textbf{Delphinium six} 

The last and rarest in a series of petals, Delphinium Six is the Grail of petal users, a supposedly transcendental experience that might not even exist. Delphinium One is scarce, Two and Three are quite rare, Four is an amazing find, and Five and Six are only rumors. Hints of what Six might hold are based largely on extrapolation from the little that is known about the lower-numbered petals. The following facts are generally accepted. It is a group experience, but not all members of the tripping group are rewarded equally. It is intensely surreal, yet in a purposeful way, as are all of the Delphinium series. It concludes the loosely-built narrative of a drugged-out version of a fairy tale princess and her quest for enlightenment begun in Delphinium One, replete with strange omens and mythological creatures. Rumors of what the ending might hold are more fanciful, and range from the trippers being resleeved in god-like infomorphs to them being trapped forever in an ego prison. Delphinium Six is completely virtual, leaving the characters comatose for the duration, and probably lasts a long time, perhaps 40 hours. 

\subsubsection{Other nanodrugs} 

Nanodrugs have the capability of making fundamental changes to a body’s biochemistry and mental state. The potential effects are too numerous to list, but gamemasters should consider allowing nanodrugs that temporarily apply certain traits, such as Brave, Direction Sense, Math Wiz, Pain Tolerance, Psi Chameleon, Psi Defense, Situational Awareness, Tough, Feeble, Frail, Low Pain Tolerance, Mental Disorder, Mild Allergy, Neural Damage, Psi Vulnerability, Severe Allergy, Timid, VR Conditioning, VR Vertigo, Weak Immune System, or Zero-G Nausea. Similarly, the nanodrug could force the character into a particular mental emotional state, such as a bad mood, edginess, contentment, or overconfidence. Gamemasters are encouraged to experiment with different possibilities and effects. 

\hspace{1cm} 

\begin{tabular}{|l|l|l|l|l|l|l|} \hline

\hline{6}{|c|}{\textbf{Nanodrugs}}	\\ \hline

\textbf{Nanodrugs}	&\textbf{Type}	&\textbf{Application}	&\textbf{Duration}	&\textbf{Addiction mod}	&\textbf{Addiction type} \\ \hline

Frequency	&Nano	&Inj, O	&8 hours &$-$10	&Mental \\ \hline

Gravy	&Nano	&Inj, O	&special &--- &--- \\ \hline

Petals	&Nano	&O	&2 hours--1 day	&+10 to $-$20	&Mental \\ \hline

Schizo	&Nano	&Inj	&1 day	&--- &Mental \\ \hline

\label{tab:nanodrugs} \label{tab:nanodrugs} 



\subsection{Narcoalgorithms} \label{sec:narcoalgorithms} 

Narcoalgorithms are software programs that simulate the effects of drugs on biological bodies. Almost all bio, chemical, and nano drugs can be replicated as narcoalgorithms, with corresponding effect (gamemaster discretion). Narcoalgorithms may be run by infomorphs, egos encased in cyberbrains (pods and synthmorphs), simulmorphs, and even AIs. 

\textbf{DDR:} Originally crafted by prankster hackers and distributed as a virus, DDR (for ``Dance Dance Robot'') triggers impulses in the target’s motor control circuits. Primary targeting robot AIs, the effect is that targets ``dance'' in jerky, automated movements. Pleasure receptors are also activated so that dancing --- and movement of any kind --- feels good. Different software variants invoke different motions and styles. The target suffers a $-$20 modifier on other actions while dancing, but the dancing may be overridden with a WIL $\times$ 3 Test. \textbf{[Low]} 

\textbf{Linkstate:} This software actually connects the user to a peer-to-peer network, where it randomly connects to other linkstate users and samples a bit of their XP feed and randomly accessed memories --- typically just enough to provide context, but not enough to acquire private personal details. These inputs are spliced together, their emotional inputs amplified, and then the entire package is spiked with some hormonal circuit triggers and artificial synaesthesia. The effect is a mind-blowing mixed sampling of people’s lives, mashed together in a sensory soup, that hits the mind with a euphoric rush. Linkstate users are catatonic while under the effects (typical sessions run 3-4 hours), but afterwards they often report that they have flashbacks of events in other people’s lives. \textbf{[Low]} 



\subsection{Chemicals} \label{sec:chemicals} 

\textbf{Atropine:} Though poisonous in large doses, atropine is an effective antidote against nerve agents like BTX2 and Nervex. Easily synthesized in a maker, atropine will avert the effect whether taken soon before or after dosage by a nerve agent. \textbf{[Trivial]} 

\textbf{DMSO:} This chemical acts as a carrier, allowing other chemicals to be absorbed through the skin. It allows any chemical agent to be applied dermally. \textbf{[Trivial]} 

\textbf{Liquid Thermite:} Similar to scrapper’s gel, liquid thermite comes in a gel form that is easily applied under all environmental conditions (by the nature of its chemical reaction, thermite is oxygenated and will burn underwater or in space). It is ignited with an electric charge, burning at temperatures exceeding 2,500 degrees Celsius and melting through whatever it is touching. Liquid thermite floors 3d10 + 5 DV per Action turn to whatever it is touching. Armor will also be burnt through, offering no protection once the full Armor rating has been reached. \textbf{[Moderate]} 

\textbf{NotWater:} NotWater is an effective liquid fire retardant that does not get objects wet, no matter how absorbent they are --- it simply beads up and slides right off. \textbf{[Trivial]} 

\textbf{Scrapper’s Gel:} This goo turns into a potent acid when given an electrical charge. It comes in a gel-like state and may be smeared like jelly, and may even be used in space. In acid form, scrapper’s gel does 1d10 + 5 DV per Action Turn to anything it touches, unless the material has been treated against acid. Armor will protect against this acid at first, but the acid will eat through the armor, so that it will no longer protect after its full armor value has been reached. \textbf{[Low]} 

\textbf{Slip:} This liquid is almost entirely frictionless. When spread around an area (commonly used in splash grenades), anyone attempting to walk or run on the affected surface must make a COO Test or fall down. Likewise, any coated surface becomes extremely hard to grip onto, requiring a SOM Test to hang on. Anyone attempting to grapple a slip-soaked character suffers a $-$30 modifier. \textbf{[Low]} 

\textbf{Tracker Dye:} This liquid is colorless at normal light but becomes recognizable under pre-specified different wavelengths (such as infrared or ultraviolet). \textbf{[Trivial]} 



\subsection{Toxins} \label{sec:toxins} 

Chemical warfare involves using the toxic properties of biological and chemical substances to kill, injure, or incapacitate an enemy. Note that an antidote can be constructed for most toxins if a sample is acquired and an appropriate Medicine or Academics Test is made. This is considered a Task Action with a timeframe of 1 hour. These toxins only affect biomorphs; synthmorphs are immune. 

\textbf{BTX:} BTX-squared (also called Frog Bite) is a genetically-enhanced variant of the extremely potent cardiotoxic and neurotoxic batrachotoxin. It leads to fast paralysis and cardiac arrest that usually kills the target within a few Action Turns. Affected characters suffer 2d10 + 10 damage a turn for 3 Action Turns; medichines reduce this damage by half. They must also make a SOM $\times$ 2 Test (+30 with medichines) or be paralyzed for 1 hour. \textbf{[High]} 

\textbf{CR Gas:} This potent incapacitating agent causes eye twitching and temporary blindness, severe coughing and breathing difficulty, skin irritation, and panic. Affected characters suffer 1d10 $\div$ 2 damage, a $-$30 modifier to sight-based Perception Tests, and a $-$20 modifier to all other actions for 20 minutes (5 minutes if the character has medichines). \textbf{[Low]} 

\textbf{Flight:} This drug is derived from human pheromones released due to fear, and is intended to instill alarm or even terror in the character. Affected characters must make a WIL $\times$ 3 Test (+30 with medichines) or suffer a panic attack, flooring 1d10 stress. Dosed characters also suffer a $-$30 modifier for resisting intimidation or fear-based emotional manipulations. Flight affects last for 1 hour (5 minutes with medichines). \textbf{[Low]} 

\textbf{Nervex:} Derived from deadly nerve agents like cyclosarin, VX, and novichok, this genetically-modified toxin is deployed as a colorless, odorless gas that turns safely inert 10 minutes after deployment. It causes involuntary contraction of the muscles, seizures, and death by respiratory failure. One minute after exposure, the character must make a SOM Test or be incapacitated by seizures, paralysis, or nausea and vomiting; unaffected characters still suffer a $-$20 modifier to all actions. After 10 minutes, the character will die unless an antidote (such as atropine, p. 323) is applied. Characters with medichines suffer the initial effects, but recover after 5 minutes. \textbf{[High]} 

\textbf{Oxytocin-A:} A genetically-improved variant of oxytocin, this drug induces trust in the recipient. Drugged characters suffer a $-$30 modifier on all WIL and Kinesics Tests where trust is a factor. Medichines provide immunity. \textbf{[Low]} 

\textbf{Twitch:} Twitch is a convulsive agent, a nonlethal nerve gas. Affected characters must succeed in a SOM Test (+30 with medichines) or become incapacitated with severe muscle tremors. Unaffected characters still suffer a $-$20 on all actions. The effects of Twitch last for 10 minutes, 5 if the character has medichines. \textbf{[Low]} 



\subsection{Nanotoxins} \label{sec:nanotoxins} 

\textbf{Disruption:} This nanotoxin attacks the myelin sheath on nerves, disrupting nerve impulses and flooring symptoms of multiple sclerosis. Every hour the morph suffers a $-$5 modifier to COO, REF, and COG. If any aptitudes are reduced to zero,the morph is effectively paralyzed and catatonic. \textbf{[Moderate]} 

\textbf{Necrosis:} Necrosis nanobots attack the walls of cells inside the body, killing tissue. This nanotoxin floors 1d10 $\div$ 2 damage per Action Turn for one minute, after which the nanobots disable and flush from the body. Necrosis only affects biomorphs. \textbf{[Moderate]} 

\textbf{Neuropath:} These nanobots are designed to stimulate the pain receptors of a morph on a systemic level to cause agony and impairment. While most neuropaths target biological receptors, variants are available that induce comparable (phantom) pain stimulations in the cyberbrains of synthmorphs to create an equivalent effect. The affected character must succeed in a WIL $\times$ 3 Test or become incapacitated. Even if they succeed, they suffer $-$30 from the floored agony. Any form of pain resistance that allows a character to ignore wound modifiers will negate the neuropath pain modifier by an appropriate amount. \textbf{[Moderate]} 

\textbf{Nutcracker:} Nutcrackers are nanobots designed to locate, migrate, and decompose the synthdiamond case of a cortical stack within a morph by attacking its crystal lattice. This process takes approximately 6 hours, after which the cortical stack is destroyed. These nanobots also attack the cortical stack’s connections to the (cyber)brain and brain-mapping nanobots. After 1 hour, the victim will be aware that their cortical stack is threatened. After 3 hours, all connections will be severed and the cortical stack will no longer be able to back up the character. \textbf{[High]} 



\subsection{Pathogens} \label{sec:pathogens} 

A pathogen is an infectious biological agent that causes disease or illness to its host. While natural pathogens rarely strive to kill their hosts, germ warfare programs revived during the Fall --- or instigated by the TITANs --- sought to modify and use pathogens as a weapon of war. The ideal characteristics of lethal biological agents are high infectivity, high potency, availability of vaccines, and delivery as an aerosol. Most biomorphs are immune to standard pathogens thanks to their basic bio-mods, and medichines will protect against most others. However, even these defenses may not protect against diseases left by the TITANs or a new terrorist cell’s biowar bug. It is largely recommended that pathogens be handled as a plot device, rather than an active threat to the characters. Pathogens have no effect on synthmorphs. 

\textbf{Degen:} Characters exposed to this degenerative neurological disease must make a DUR $\times$ 2 Test or become infected. Medichines will defeat the disease, but others will not show signs of infection for 1 week, when the symptoms of a rapidly progressing dementia will become clear: memory loss, personality changes, and hallucinations. If untreated, Degen will progress for another week with more serious symptoms, including speech impediments, jerky movements, loss of balance and coordination, and even seizures. This is reflected by a 5 point loss in all aptitudes per day (after the first week). When any aptitude reaches 0, the character dies. Degen is notorious for its effect in corrupting cortical stack backups before infection symptoms manifest. \textbf{[Expensive]} 

\textbf{Trigger:} Trigger is a designer virus that selectively targets and infects mast cells to trigger a hyper-allergic reaction. The resulting anaphylactic shock due to systemic vasodilatation (associated with a sudden drop in blood pressure) and bronchial swelling (resulting in constriction and difficulty breathing) usually leads to death in a matter of minutes after onset, if not treated. Infected characters must succeed in a DUR Test (using their current Durability score minus damage) or die quickly. Even medichines have difficulty reacting in time against this virus; characters with medichines must make a DUR $\times$ 2 Test to survive. \textbf{[Expensive]} 



\subsection{Psi drugs} \label{sec:psi-drugs} 

Research into the Watts-MacLeod strain has resulted in several exceptional breakthroughs involving the creation of psi-impacting drugs. Each of these drugs is in the experimental stage, but they are already finding some use among Firewall and similar secretive groupings. 

\textbf{Inhibitor:} Inhibitor is a cocktail of neurochemicals that block some brain receptor and transmitter functions in an attempt to reduce psi-waves and block or impair sleights. This drug is commonly used to restrain async prisoners from using their abilities. A drugged character must make a WIL $\times$ 2 Test. If they fail, they lose all psi abilities for the drug’s duration. If they succeed, they suffer a $-$30 impairment on Psi skills and all strain is doubled. Inhibitor has an unfortunate side effect of doping the character down, however; apply a $-$10 modifier to their COG. Inhibitor-influenced characters tend to have a glazed, dopey expression and have difficulty getting excited or emotional. \textbf{[High]} 

\textbf{Psi-Opener:} Psi-opener drugs are variants of the Watts-MacLeod strain with a temporary effect and which do not permanently alter the user’s brain. Psiopener temporarily imbues the user with the ability to use one particular sleight, regardless of whether or not they have the Psi trait. Each type of Psi-opener is customized for a particular sleight. While primarily intended for non-asyncs, non-asyncs may not possess Psi skills, so they must default to WIL. For this reason, Psi-Opener is often doubled up with Psike-out. Using Psi-opener is a mind-wrenching experience. Users are occasionally subject to hallucinations (gamemaster discretion). When the drug wears off, it floors 1d10 points of mental stress, +2 if the drug imbues a psi-gamma sleight. \textbf{[Expensive]} 

\textbf{Psike-Out:} Psike-out bolsters an async’s psi abilities. Apply a +20 modifier to the async’s Psi skill tests for the drug’s duration. However, also apply +2 to all strain DVs for the drug’s duration. Psike-out is mentally addictive, with an Addiction modifier of $-$10. \textbf{[Expensive]} 

\begin{table} 

\begin{tabular}{|l|l|l|l|l|} \hline

\hline{5}{|c|}{\textbf{Toxins}} \\ \hline

&\textbf{Type}	&\textbf{Application}	&\textbf{Onset time}	&\textbf{Duration} \\ \hline

\hline{5}{|l|}{Chemical toxins} \\ \hline

BTX2	&Chem	&D, Inj, O	&1 Action Turn	&3 Action Turns/1 hour \\ \hline

CR Gas	&Chem	&D, Inh	&1 Action Turn	&20 minutes \\ \hline

Flight	&Bio	&Inh	&3 Action Turns	&1 hour \\ \hline

Nervex	&Chem	&D, Inh, Inj, O	&1 minute	&death \\ \hline

Oxytocin-A	&Bio	&Inh, Inj	&3 minutes	&2 hours \\ \hline

Twitch	&Chem	&D, Inh, Inj, O	&3 Action Turns	&10 minutes \\ \hline

\hline{5}{|l|}{Nanotoxins} \\ \hline

Degeneration	&Nano	&Inj, O	&Immediate	&8 hours \\ \hline

Necrosis	&Nano	&Inj, O	&3 Action Turns	&1 minute \\ \hline

Neuropath	&Nano	&D, Inj, O	&3 Action Turns	&8 hours \\ \hline

Nutcracker	&Nano	&Inj, O	&Immediate	&6 hours \\ \hline

\hline{5}{|l|}{Psi Drugs} \\ \hline

Inhibitor	&Chem	&Inj, O	&3 Action Turns	&6 hours \\ \hline

Psi-Opener	&Bio	&Inj, O	&20 minutes	&1 hour \\ \hline

Psike-Out	&Chem	&Inj, O	&1 minute	&1 hour \\ \hline

\label{tab:Toxins} \label{tab:Toxins} \end{table} 



\section{Everyday technology} \label{sec:everyday-tech} 

The following devices are all exceptionally common and can be acquired in almost any habitat. Almost everyone in \emph{Eclipse Phase} either owns these devices or knows several people who do. 

\textbf{Ecto:} Ectos are the external version of basic mesh inserts (p. 300), minus the medical sensors. These colorful devices serve as a wearable mesh terminal, PDA, locator, and camera-phone. The devices are flexible (often worn as bracelets), dirt-resistant, self-cleaning, and may be stretched out to increase screen size. They may project holographic displays and are typically equipped with wireless-enabled glasses or contact lenses and decorative earpieces or earrings so that the user may access augmented reality. Given the ubiquity of mesh inserts, ectos are growing less common, but they are still used by bioconservatives, others without implants, and those who prefer to access the mesh via an external device for security concerns. \textbf{[Low]} 

\textbf{Holographic Projectors:} These devices are capable of projecting high-definition, ultra-realistic three-dimensional images and movies. From a distance (20+ meters), such holograms can be difficult to distinguish as fake, but up close they are easier to see for what they are (+20 Perception Test modifier). Holograms do not appear wavelengths other than visual light, and so are easily identified by anyone with enhanced vision. [Low] Micrograv Shoes: These shoes are equipped with velcro and/or a magnetic system, allowing the wearer to walk normally on appropriate surfaces in micrograv and zero-G environments, rather than floating or bouncing. \textbf{[Trivial]} 

\emph{Portable Sensor:} This is a small portable (possibly even wearable) sensor system. The type of sensor must be chosen (for example: infrared, lidar, radar, x-ray). Combined sensor systems are also available, at a cumulative cost. See \emph{Radio and Sensor Ranges}, p. 299. and \emph{Using Enhanced Senses}, p. 302. \emph{[Moderate]} 

\textbf{Smart Clothing:} Smart clothing can change its color, texture, and even its cut, taking only a minute or two to transform from a solid color jumpsuit to a plaid party dress or a replica of a pinstriped, late 20th century business suit. It can also camouflage the wearer, providing a +20 bonus to Infiltration Tests to avoid being seen, as long as the wearer is stationary or not moving faster than a slow walk, and as long as the wearer is completely covered or also using chameleon skin (p. 303) of the same color/pattern. If incompletely camouflaged, or if moving faster, reduce the modifier to +10. Smart clothing also keeps the character warm or cool, allowing the character to exist comfortably in environments from $-$40 to 70$^{\circ}$ C. \textbf{[Low]} 

\textbf{Smart Vac Clothing:} Just like regular smart clothing, this outfit can also transform into a light vacsuit (p. 333). It also functions as armor with a rating of 2/4. \textbf{[Moderate]} 

\textbf{Specs:} Specs are vision-enhancing glasses. They deliver sensory data directly into the wearer’s visual cortex by connecting with their basic mesh inserts (p. 300), though visual displays are available for bioconservatives and other characters without implants. Specs extend the range of the wearer’s vision from terahertz waves to gamma rays (p. 302). Specs include a t-ray emitter (p. 306), however, using x-rays, or gamma rays for visual purposes requires a separate emitter, since neither of these sorts of radiation are common inside habitats, or in any safe environments. Specs have a variable focus equivalent to 5 power magnifiers and provide the wearer with a +10 bonus to all Perception Tests involving vision. \textbf{[Low]} 

\textbf{Tools:} Tools come in kits (portable), shops (can fit into a large vehicle), and facilities (large, non-mobile). Each set of tools applies to a particular skill, such as Hardware: Electronics or Hardware: Groundraft. \textbf{[Low (Kit), High (Shop), Expensive (Facility)]} 

\textbf{Utilitool:} This hand tool includes a specialized small nanobot generator. In its basic form, a utilitool is the size and shape of a large fountain pen. It can transform into almost any tool, however, from a wrench, knife, or powered screwdriver to a rotary grinder or pair of pliers. Some inexpensive utilitools are optimized for specialized tasks, like cooking or wilderness survival, but more expensive models become almost any imaginable hand tool. Utilitools are normally mentally controlled using the character’s basic mesh inserts. Characters without such implants can control the tool via voice commands and touch controls. Characters using a utilitool gain a +10 modifier to skills involving repairing or modifying devices with mechanical parts, opening locks, disarming alarm systems, or performing first aid. \textbf{[Low]} 

\textbf{Viewers:} These small and highly advanced binoculars possess all the visual enhancement of specs (p. 325), but also provide 50x magnification. They also include a directional microphone that magnifies sound from the direction the viewers are pointed by a factor of 50. Viewers provide the user with a +30 bonus to all Perception Tests involving vision or hearing for the target they are aimed at. This bonus is not cumulative with bonuses from any other device or augmentation. \textbf{[Low]} 



\section{Nanotechnology} \label{sec:nanotech} 

Nanotechnology is the precise manipulation of matter at the atomic level, typically using millions of microscale nanomachines. Nanotechnology transformed manufacturing, enabling new techniques and materials. The advent of nanofabrication --- building objects from the molecular level up --- transformed economies, allowing people to simply manufacture whatever they needed from raw materials. Nanotechnology is still a growing field, however, and has its limitations. While the TITANs unleashed self-replicating nanoswarms with the ability to transform or destroy anything through the power of geometric growth, such technology remains far beyond transhumanity’s grasp. 



\subsection{Basic nanotechnology} \label{sec:basic-nanotech} 

Basic nanotechnology is exceedingly widespread and used throughout the solar system, serving as the primary method of manufacturing for decades. The nanobots of basic nanotech are confined to delicate and specially-maintained environments like the insides of cornucopia machines or healing vats and cannot operate elsewhere. 

\subsubsection{Healing vats} 

Healing vats were the first type of nanotech medicine developed and remain the most powerful medical devices in common use. With the exception of a few exceptionally deadly nanoplagues, a healing vat can cure any disease and heal any injury. As long as the patient is alive when they are place in the healing vat, they will not only survive, but emerge without a scratch. A healing vat can even take a severed head (as long as it has been stabilized by medichines or nanotech first aid) and regrow an entire body based on the head’s genetics. If the patient’s body or medical records contain information about their implants, bioware, or advanced nanotechnology, all of these modifications are also fully restored. 

Few people suffer injuries serious enough to require a healing vat. Most are used as a safe and easy way to perform bodysculpting or to install implants or bioware. Healing vats use specialized nanomachines to either alter the patient’s body or integrate implants or bioware. One advantage of using a healing vat is that no additional healing time is needed, the patient leaves the vat fully recovered from the augmentation and ready to go. Every hospital, clinic, bodyshop, and augmentation parlor has several healing vats. The time required by a healing vat varies with the severity of the damage it is healing or the extent of the modification being made, as noted on the Healing Vat table, p. 327. \textbf{[High]} 

\begin{table} \begin{tabularx}{\hline}{|X|X|} \hline

\hline{2}{|c|}{Healing vat table} \\ \hline

\textbf{Injury}	&\textbf{Healing time} \\ \hline

Healing normal damage to a character who has taken 3 or fewer wounds.	&2 hours per wound (min. 1 hour for 0 wounds) \\ \hline

Restoring major lost body parts like arms or legs, or healing dying or nearly dead character who has taken 4 wounds.	&12 hours per wound \\ \hline

Restoring recently dead character who was placed in medical stasis to avoid death, but who is mostly intact.	&1 day per wound \\ \hline

Restoring recently dead character who is placed in medical stasis to avoid death, and who is missing most of their body.	&3 days per wound \\ \hline

\hline{2}{|l|}{Augmentation} \\ \hline

Minor implants and bioware, minor cosmetic changes like alterations in skin color, eye color or shape, or hair color, texture or distribution, minor alterations to face shape or body fat distribution.	&1 hour \\ \hline

Major brain and neural implants, nanoware or bioware, sex changes, changing height by no more than 5\% or weight by no more than 20\%.	&12 hours \\ \hline

Major physical modifications like adding limbs or radical changes to height and weight.	&3 days \\ \hline

\label{tab:healing-vat} \label{tab:healing-vat} \end{table} 

\subsubsection{Nanodetectors} 

Nanodetectors are small devices that suck in air and micro debris in order to scan for and detect nanobots. Given that nanobots are so small, the density of nanobots in the area has a large impact on its success. The nanodetector has a base skill of 30 for detecting nanobots, modified by +30 if an active nanoswarm or hive is present, +0 if a nanoswarm or hive was active recently, and $-$10 for the presence of nanobots outside of a swarm or hive. Once a nanobot is detected it may be analyzed either by the user or the nanodetector’s AI, using Academics: Nanotechnology 30 skill. Nanodetectors are often worn and left on, set to alert the user if a hostile nanoswarm is detected. [Low] 

\subsubsection{Nanofabricators} 

Nanofabrication machines are universal assemblers that perform almost all of the manufacturing in the solar system. The user loads in raw materials and electronic plans and it can produce literally any manufactured good, from a weapon to an ultralight plane to a hot and delicious dinner. Many nanofabricators come equipped with a library of common-use blueprints (basic foods, standard clothing, common tools, etc.). Other blueprints must either be purchased online, selfprogrammed, or acquired through some other method (see \emph{Nanofabrication}, p. 284). The largest nanofabrication units are more than 10 meters on a side and are used to produce small consumer goods in bulk as well as building large devices like orbital transfer vehicles. 

The availability and legality of nanofabricators varies widely throughout the system. In the inner system and Jovian Republic, cornucopia machines are commonly restricted and sometimes illegal, with licenses only available to hypercorps, military units, and other officials and elites. In these habitats, only more limited fabbers are available to the general populace. Additionally, blueprints are licensed and protected by copyright laws, and many nanofabricators feature pre-programmed restrictions that prevent them from using unlicensed blueprints as well as from manufacturing weapons, explosives, or other restricted items. Among the autonomists of the outer system, however, nanofabricators are commonly accessible, shared by everyone, and unrestricted. 

For rules on creating goods in a nanofabricator, see \textit{Nanofabrication}, p. 284. 

\textbf{Desktop Cornucopia Machine:} Cornucopia machines (CMs) are general-purpose nanofabricators. The smallest CMs are desk-sized cubes approximately half a meter on a side with a volume of at least 40 liters. They can produce any small object, from tools to well-folded suits of clothing to handguns or a bowl of cereal. It is sometimes possible to assemble larger items, but they must be manufactured in smaller pieces and then assembled (likely requiring an appropriate Hardware Test). 

While users can purchase bulk raw materials, CMs also come equipped with a disassembler. The user loads garbage and other objects into the disassembler so that they can be turned into raw materials for the CM. All legally-available disassemblers only deconstruct non-living material. \textbf{[Expensive]} 

\textbf{Fabber:} Fabbers are specialized nanofabricators, portable and considerably smaller than CMs. There are a wide variety of portable fabbers, including ones that can make any hand tool or small piece of personal electronics, ones that can turn any organic material into food and drink, and ones that can create any drug or medicine as well as bandages and specialized dressings. The most common fabbers have a volume of 4 liters. Larger hand tools and devices are produced as 2 or 3 separate parts that must be fitted together. Like CMs, fabbers also contain miniature disposal units. \textbf{[Moderate]} 

\textbf{Maker:} Makers are specially-designed to produce food and drink for the user. Raw materials can be provided by the addition of any water-containing liquid and collected biomass like leftover food, grass, dirt, dead animals, or transhuman waste. Some models are built into standard vacsuits. Makers can produce water and various flavored beverages, as well as ration bars or thick pudding-like edible gels. With adequate raw material, a maker can indefinitely provide food and drink for up to three transhumans. Most units, however, have a very limited range of flavors and textures that are widely considered to be fairly bad. Models with a wider and better range of flavors and textures are more expensive, but produce food that is considered adequate or occasionally good. \textbf{[Low to Moderate]} 

\emph{Blueprints:} If you want a nanofabricator to make something, you need to instruct the device how to create it from the molecular level up. Such blueprints are available for almost every conceivable item out there. The cost of such blueprints typically exceeds the cost of purchasing the item, though factors like legality and quality may affect the cost as usual (see \emph{Acquiring Gear}, p. 296). \textbf{[One Cost Category Higher Than Item Cost]} 



\subsection{Advanced nanotechnology} \label{sec:advanced-nanotech} 

Advanced nanotechnology includes more recent developments. Like basic nanotech, advanced nanotechnology cannot self-replicate but the nanobots can function normally in most environments and are highly resistant to bacterial attacks and other environmental problems. Typical advanced nanotech consists of a generator --- known as a ``hive'' --- that produces nanobots as long as it is supplied with raw materials. Every such hive also includes a miniature disassembly unit and/or specialized nanomachines that collect raw materials for the generator. These hives produce nanobot swarms that are set loose to perform some function in the world. 

Examples of advanced nanotech include COTs (p. 315), medichines (p. 308), smart dust (p. 316), and utilitools (p. 326), among others. 

\textbf{General Hive:} General hives are capable of producing any conceivable type of nanobot with the right blueprints and/or programming. Even at their smallest size they are not really portable, with a minimum size being cubes 30 centimeters on a side and a volume of 25 liters. \textbf{[Expensive]} 

\textbf{Specialized Hive:} Specialized hives are far more common than general hives, though they can produce only one type of nanomachines (i.e., choose one type of nanoswarm per hive). The smallest specialized hives are approximately the size of a 12-gauge shotgun shell or a large cherry tomato. \textbf{[Moderate, plus Cost of Programmed Nanoswarm]} 

\subsubsection{Ego bridges} 

Ego bridges are vat devices used for uploading and downloading minds. See \textbf{Backups and Uploading}, p. 268, and \emph{Resleeving}, p. 271. \textbf{[Expensive]} 

\subsubsection{Nanoswarms and microswarms} 

Swarms are colonies of nanobots or larger microbots created in a hive, programmed with specific instructions, and then set free to perform a set task. Each swarm is composed of hundreds or thousands of nanobots or microbots, ranging in size from a microbe to a small insect. Nanobots are typically invisible to the naked eye, though they can be detected with a nanodetector (p. 326) or nanoscopic vision (p. 311). Microbots are more noticeable but still quite small, usually the size of a grain of sand or a dust mote, or occasionally as big as a flea. Individual bots in a swarm are directed by nanocomputers, with behavioral routines modeled on biological insect and animal swarms. These swarms stick together and work as a whole, communicating with nanoradios, nanolasers, or chemical cues, and sharing information between each bot in the swarm. Note that nanoswarms don’t invade inside living bodies (though they may attack externally) --- internal nano is handled by nanoware (p. 308), nanodrugs (p. 321), and nanotoxins (p. 324). 

Nanobots and microbots may be designed with all manner of miniaturized propulsion systems (see \emph{Mobility Systems}, p. 310), with the exception of ionic drives. They are powered by tiny batteries or solar cells. Their tiny sensors are very effective at allowing them to identify materials and objects, and so to target discriminatingly. Nanobots or microbots could, for example, be programmed to ignore metal objects, certain types of plants, specific morphs, females, or specific individuals. Swarms may either be released directly from a hive or from pre-packaged programmable canisters. 

Swarms must be programmed before they are released. The programming first determines how long the swarm is active. This timeframe is open-ended, though most swarms deteriorate into ineffectiveness after 2 weeks unless they are replenished by a hive. The programming then sets what area the swarm is to occupy. This is also open to interpretation and can vary from ``coat this person'' to ``spread out to a diameter of 20 meters'' to ``find the nearest chemical traces and track them to their source.'' Finally, programming sets any other parameters for the swarm’s mission --- for example, if it should ignore certain materials, if it should send a report at a predetermined time, or if it should self-destruct into harmless dust when it has completed a certain task. 

Programming is generally handled as a Simple Success Test using Programming (Nanoswarm) skill. Failure simply infers that the programming is imperfect, and so the swarm may not operate completely as planned. An actual Programming (Nanoswarm) Success Test is only called for if the swarm’s programming is substantially complex or if the character seeks to have the swarm act outside of its usual set functions. The bots in each swarm are specially equipped for the task they are designed for, however, so attempting to drastically repurpose a swarm may be difficult or pointless at the gamemaster’s discretion. 

Swarms may also be teleoperated, controlled, and/ or (re)programmed once they are released, via radio or laser link. 

Swarms are treated as a whole. The standard swarm size is enough to cover a 10 $\times$ 10 $\times$ 10 meter cube, and this is the standard ``unit'' of swarm released by a canister or hive. Swarms may be larger, but they are treated as individual swarm units. Each swarm has a Durability of 50 and is immune to wounds. Most attacks against a swarm simply floor 1 point of damage. Area-effect weapons, plasma rifles, and fire floor 1d10 damage, plasma grenades do full damage. EMP weapons (p. 340) are very effective against swarms, flooring 2d10 + 5 damage and a -10 modifier to all tests due to their damaging effects on the swarm’s communication abilities until repaired. Swarms are not affected by vacuum. 

\textbf{Cleaners:} This nanoswarm cleans, polishes, and removes dirt and stains. It may be used on an area, specific objects, or people. Some facilities employ permanent cleaner swarms to keep their area spotless. Cleaners may also be programmed to remove specific toxins, chemicals, or other hazardous substances in order to decontaminate an area. Covert operatives and criminals sometimes use cleaners to eliminate any evidence they may have left at a scene usable for forensics purposes, such as blood, hair, or anything that could be DNA-typed. \textbf{[Low]} 

\textbf{Disassemblers:} Also known as smart corrosives, these nanobots break down any matter. Their advantage over common acids is that not only are they able to break down any material by using energy to disrupt chemical bonds, but that they can be programmed to take apart certain components while ignoring others, leaving them intact. Disassemblers are a common weapon used against synthmorphs, eating away their components without having to worry about accidentally splashing biomorphs. Upon contact, these nanobots floor 1d10 $\div$ 2 damage (round up) per Action Turn. Accumulated damage counts as a wound when the Wound Threshold is reached. Both Energy and Kinetic armor protect against this damage, but these armors are eaten away as well, so the Armor Value is reduced by the soaked DV. \textbf{[High]} 

\textbf{Engineers:} Engineer microswarms are used for various construction purposes: erecting walls, digging tunnels, sealing holes, reinforcing foundations, and so on. \textbf{[Moderate]} 

Fixers: This is the nanoswarm version of repair spray (p. 333). \textbf{[Moderate]} 

\textbf{Injectors:} Injector microswarms are equipped with tiny needles and a drug payload. A biological target affected by an injector swarm suffers 1 point of damage and the effects of the carried drug, chemical, or toxin. \textbf{[Moderate]} 

\textbf{Gardeners:} This microswarm is useful for a number of agricultural purposes: killing weeds, planting seeds, trimming plants, pollinating, and even harvesting small items. It may also be programmed to simply defoliate an area. \textbf{[Moderate]} 

\textbf{Guardians:} Guardians watch for and attack other unauthorized swarms. Guardians floor 1d10 $\div$ 2 damage (round up) on other swarms they come into contact with per Action Turn. \textbf{[Moderate]} 

\textbf{Proteans:} This nanoswarm is designed to disassemble other materials and objects and to create a single specific, pre-programmed device from the components (much like a specialized nanofabricator). The proteans must be able to scavenge appropriate raw materials (for example, to create a metallic device the nanobots must transform something else made of metal). The construction time takes 1 hour per cost category of the item (1 hour for a Trivial cost item, 2 hours for Low, etc.). \textbf{[High]} 

\textbf{Saboteurs:} Sab nanobots are designed to infiltrate electronics or machinery and sabotage them in small but difficult to discern ways: severing connections, disabling components, gumming up moving parts, etc. Saboteurs floor damage on devices similar to disassemblers, but the target is not destroyed and such damage is not immediately obvious. They floor 1d10 $\div$ 2 points of damage to synthmorphs, bots, and other devices every Action Turn. Armor has no effect, but accumulated damage counts as a wound when the Wound Threshold is reached. \textbf{[High]} 

\textbf{Scouts:} A scout nanoswarm will systematically map and explore an area, collecting samples of all materials and substances it encounters. The samples are carried back to the hive or canister and chemically analyzed. Scouts can also be used for forensic purposes, collecting DNA samples, analyzing chemical residues, and examining other evidence. \textbf{[High]} 

\textbf{Taggants:} Taggants seek to lodge themselves onto everything in their area of dispersal. Each carries a unique identifier, so that if it is found later, the tagged person or object can be linked back to the point they were tagged. Taggants can be programmed to remain silent, only responding to query broadcasts made with the proper crypto codes, or they can be programmed to broadcast their location back to the deployer via the mesh. \textbf{[Low]} 



\subsection{Pets} \label{sec:pets} 

These partially-uplifted and bio-engineered animals have rudimentary intelligence and limited communication skills. They make for fine companions and helpers. 

\textbf{Fur Coat:} A so-called ``fur coat'' is outerwear made from a living primitive organism. The creature’s skin, fur, or scales are real. The organism is cultivated from transgenic stocks and grown around molds into clothing shapes, often with actual usefulness: polar bear parkas, seal diving suits, porcupine coats, etc. Fur coats are modified with wireless controls and haptic systems, so they can be made to move, shiver, massage, or prickle up on command. \textbf{[Low]} 

\textbf{Smart Dogs:} Commonly used as discriminatory guardians, smart dogs are sometimes enhanced with combative bioware or cybernetics. \textbf{[Moderate]} 

\textbf{Smart Monkey:} Commonly used by criminal groups for minor larceny such as pickpocketing, smart monkeys can be useful and intelligent aides. \textbf{[Moderate]} 

\textbf{Smart Rats:} These upgrades of the common Norwegian rat are clever and dexterous, and they easily fit into a pocket or hood. \textbf{[Low]} 

\textbf{Space Roach:} Grown to the size of a small dog, these insects are often biosculpted for bright colors and patterns. They are useful for minor janitorial duties. \textbf{[Low]} 

\subsection{Scavenger tech} \label{sec:scavenger-tech} 

This technology is often employed by gatecrashers, space scavengers, and Firewall teams during missions. 

\textbf{Disassembly Tools:} These tools are useful for salvage ops, breaking down wrecks, or dissembling anything from a habitat room to a vehicle or synthmorph. They include plasma torches, laser cutters, pneumatic jaws, and smart tools like spanners and wrenches that can be adapted to a wide array of connections and fittings. \textbf{[High]} 

\textbf{Mobile Lab:} The mobile lab is a handheld device that contains all different types of sensors to investigate organic and inorganic liquid, gaseous, and solid components (from soil to tissue samples) and compositions. It performs material analysis using different methods of spectrometry and biochemical testing, comparing results to a built-in database of element and compound spectra. Its built in AI comes equipped with Academic: Chemistry 30. \textbf{[Moderate]} 

\textbf{Specimen Container:} This capsule container is designed to hold samples of any sort (chemical, biological, etc.) in near stasis. It can be programmed to reproduce whatever conditions the user specifies, from cryogenic freezing to extreme heat, or even vacuum or high-pressure atmosphere. \textbf{[Low]} 

\textbf{Superthermite Charges:} These powerful and highly stable demolition charges are made from a combination of nanometals and metal oxides. A single charge can be used to create an explosive blast flooring 2d10+5 damage. This charge can be shaped with a successful Demolitions Test, focusing the blast in a particular 90-degree direction (for example, to blow through a door). This triples the damage of the blast in the focused direction; in all other directions, the damage is reduced to 1/3rd (round down). Multiple charges apply a cumulative effect. \textbf{[Moderate]} 

\subsection{Services} \label{sec:services} 

\emph{Anonymous Accounts:} These accounts are crucial for anyone who wants to be discreet with their online transactions. See \emph{Anonymous Account Services}, p. 252. [Moderate] 

\textbf{Backup:} A single, one-time backup without insurance is sometimes all the poor can afford, hoping that they can buy backup insurance later or that someone that cared about them will see to a resleeving. \textbf{[Moderate]} 

\textbf{Backup Insurance:} In the event of verifiable death, or after a set period of being missing, backup insurance will arrange for your cortical stack to be retrieved and your ego downloaded into another morph. If the cortical stack cannot be retrieved, your most recent backup is used. Most policies require that the holder provide a backup to be uploaded into secure storage at least twice a year. This industry works in a manner similar to insurance underwriting in terms of cost and individuals engaged in high risk professions can expect to pay a premium for the service. Additionally, attempts to retrieve a cortical stack are minimal unless one wants to pay for some extra effort (a thriving industry of paramilitary ego-repo operatives exists for this purpose). \textbf{[Low to Moderate per month]} 

\textbf{Body Bank:} People who are egocasting to another station but whom hope to download back into the same body they have before when they return may put the morph on ice for the duration of their absence. \textbf{[Moderate per month]} 

\textbf{Bot/Pod Rental:} When you need a helping hand or a personal companion for a day or two, renting a bot or pod is often the way to go. \textbf{[Moderate per day]} 

\textbf{Egocasting:} This is the use of a farcaster to transmit an ego/infomorph. Farcasting is not cheap, and the cost is impacted by factors such as distance to receiver station and priority service (paying extra to get bumped ahead in line). \textbf{[Expensive]} 

\textbf{Fake Ego ID:} This forged ID will pass in most inner system and Jovian Republic habitats, and sometimes others. \textbf{[High]} 

\emph{Morph Brokerage:} Acquiring a new morph is not always easy and is affected by factors such as the type of morph, sought-after enhancements/customizations, and local availability. Numerous brokerage services exist to find you what you need, or close to it. With enough lead-time, it may be possible to grow a pod that closely imitates your morph of choice. A willingness to accept used/traded-in morphs helps to reduce costs. For more details, see \emph{Morph Brokerage}, p. 276. 

\emph{Psychosurgery:} A character can purchase time in an immersive high-fidelity simulspace with expert care from psychosurgeons and AIs in order to cope with derangements and disorders that build up as a result of existing in a transhuman universe. For an additional price the procedure can be time shifted to speed up the relative time within the simulspace. For more details, see \emph{Mental Healing and Psychotherapy}, p. 215, and \emph{Psychosurgery}, p. 229. \emph{[Moderate per month]} 

\textbf{Simulspace Subscription:} This will by you access to the simulspace of your choice, whether you want it for a private meeting/vacation or to play the latest and hottest VR game. \textbf{[Low (single use/1 day) to Moderate (monthly subscription)]} 

Space Travel: Space transport cost depends on a number of factors like distance, quality of lodgings, and how much cargo you’re bringing with. At the low end, an intra-habitat shuttle trip within the same cluster, or a trip to or from a planetary body’s surface and orbit, is not cheap but affordable \textbf{[High]}. Just about anything else is progressively more costly. \textbf{[Expensive]} 

\subsection{Software} \label{sec:software} 

For information on using software, see the \emph{Mesh} chapter, p. 234. 

\subsubsection{Programs} 

These programs can be run on any computerized device. 

\emph{AR Illusions: }These databases of AR clips can be used to create realistic illusions in someone’s entoptic display. See \emph{Augmented Reality Illusions}, p. 259. \textbf{[Moderate]} 

\textbf{Exploit:} Exploits are hacker tools that take advantage of known vulnerabilities in other software. They are required for intrusion attempts (p. 254). \textbf{[High]} 

\textbf{Facial/Image Recognition:} This program can be used to take an image and run a pattern-matching search among public archives. Similar version of this program exist for other biometrics: gait recognition, vocal recognition, etc. \textbf{[Low]} 

\textbf{Firewall:} This program protects a device from hostile intrusion. Every system comes with a standard version of this software by default. \textbf{[Low]} 

\emph{Sniffer:} Sniffer programs collect all of the transmission that pass to, from, or through the device they are running on. See \emph{Sniffing}, p. 252. \textbf{[Moderate]} 

\emph{Spoof:} Spoof is a hacker tool used to fake commands and transmissions, making them seem as if they came from another source. See \emph{Spoofing Authentication}, p. 255. \emph{[Moderate]} 

\emph{Tactical Networks:} These programs allow people in the same squad to share tactical data in real-time. See \emph{Tactical Networks}, p. 205. \textbf{[Moderate]} 

\emph{Tracking:} This software is used to track people by their presence online. See \emph{Scanning, Tracking, and Monitoring}, p. 251. \textbf{[Moderate]} 

\textbf{XP:} Experience playback recordings are clips of someone else’s experiences. Depending on the content, some XP (porn, snuff, crime, etc.) may be restricted in certain jurisdictions. Some XP clips are intentionally modified so that their emotive tracks are more intense, giving the viewer a greater thrill. \textbf{[Low to High]} 

\subsubsection{AIs and muses} 

Every character starts with a personal muse for free. Many devices also come with pre-installed AIs, capable of helping the user, responding to commands, or even operating the device on their own. Rules for AIs can be found on p. 264. 

Below are some commonly available AI programs. Unless otherwise noted, these AIs have aptitudes of 10. These AIs may also be equipped with skillsofts (p. 332). 

\textbf{Bot/Vehicle AI:} These AIs are designed to be capable of piloting the robot/vehicle without transhuman assistance. REF 20. Skills: Hardware: Electronics 20, Infosec 20, Interests: [Bot/Vehicle] Specs 80, Interface 40, Research 20, Perception 40, Pilot: [appropriate field] 40. \textbf{[High]} 

\textbf{Device AI:} These AIs are designed to operate a particular device without transhuman assistance. Skills: Infosec 20, Interests: [Device] Specs 80, Interface 30 (Device Specialization), Programming 20, Research 20, Perception 20. \textbf{[Moderate]} 

\textbf{Kaos AI:} Kaos AIs are used by hackers and covert ops teams to create distractions and sabotage systems. REF 20. Skills: Hardware: Electronics 40, Infosec 40, Interface 40, Professional: Security System 80, Programming 40, Research 20, Perception 30 plus one weapon skill at 40. \textbf{[Expensive]} 

\textbf{Security AI:} Security AIs provide overwatch for electronic systems. Skills: Hardware: Electronics 30, Infosec 40, Interface 40, Professional: Security Systems 80, Programming 40, Research 20, Perception 30, plus one weapon skill at 40. \textbf{[High]} 

\emph{Standard Muse:} Muses are digital entities that have been designed as personal assistants and lifelong companions for transhumans (see \emph{AIs and Muses}, p. 264). INT 20. Skills: Academics: Psychology 60, Hardware: Electronics 30, Infosec 30, Interface 40, Professional: Accounting 60, Programming 20, Research 30, Perception 30, plus three other Knowledge skills at 40. \textbf{[High]} 

\subsubsection{Scorchers} 

Scorchers are damaging neurofeedback programs used to torment hacked cyberbrains (p. 261). 

\textbf{Bedlam:} Bedlam programs assault the ego with traumatic mental input, inflicting mental stress. Victims are overwhelmed with horrific, monstrous, sanity-ripping sensory and emotional input. Each attack floors 1d10 SV. \textbf{[High]} 

\textbf{Cauterizer:} This scorch program rips into the ego with destructive neurofeedback routines. Each attack with a cauterizer inflicts 1d10 + 5 DV on the target ego. This damage is reflected as digitized neurological damage. \textbf{[High]} 

\textbf{Nightmare:} Nightmare programs trigger anxiety and panic attacks within the victim by stimulating the neural circuitry representing the amygdala and hippocampus. The target ego must make a WIL $\times$ 2 Test. 

If they succeed, they are shaken but otherwise unaffected, suffering a $-$10 modifier to all actions until the end of the next Action Turn. If they fail, they suffers 1d10 $\div$ 2 stress damage and are overcome with panic. This causes them either to blindly flee, have a nervous breakdown, or cower in frozen shock (gamemaster’s discretion). This panic episode lasts for 1 Action Turn per 10 points of MoF. \textbf{[High]} 

\textbf{Shutter:} Shutters target the victim’s sensory cortices, inflicting a $-$30 modifier to one chosen sense. Double this modifier if the attacking hacker scored an Excellent Success. This modifier reduces at the rate of 10 points per Action Turn. \textbf{[High]} 

\textbf{Spasm:} Spasm programs are design to incapacitate the ego with excruciating pain. Affected targets must immediately make a WIL $\times$ 2 Test. If they fail, they immediately convulse, are disabled, and writhe in agony for 1 Action Turns per 10 full points of MoF. If they succeed, they still suffer a $-$30 modifier to all actions, which reduces at the rate of 10 points per Action Turn. Due to the nature of the delivery, pain tolerance of any sort has no effect. \textbf{[High]} 

\subsubsection{Skillsofts} 

Skillsofts are used with skillware implants (p. 309). 

\textbf{Standard Skillsoft:} These programs provide the character with a rating of up to 40 in a single Active skill. \textbf{[High]} 

\subsection{Survival gear} \label{sec:survival-gear} 

The following gear is often critical to the survival of soldiers, spies, criminals, gatecrashers, emergency service personnel, and others who regularly venture into unsafe or unfamiliar regions. 

\textbf{Breadcrumb Positioning System:} This worn device leaves micro ``breadcrumbs'' behind as the character moves. These devices interact with mesh inserts (or ectos) as long as they are within range (50 meters), allowing the user to map their position in relation to the breadcrumb trail. This is useful in derelict habitats, wilderness, and other areas where there is no local functioning mesh, and is helpful both for mapping and for finding one’s way back. \textbf{[Low]} 

\textbf{Electrogravitics Net:} Also called a safety net, this failsafe system uses electric fields to counter gravity when falling. While the system is not able to actually levitate heavy objects, it will slow down a fall enough that the user can land safely if the gravitational force is not too high (the fall height is not greater than 50 meters in 1G). Generating these electric fields consumes a lot of energy, so the net is only charged for one use only and needs to be recharged afterwards. \textbf{[Moderate]} 

\textbf{Electronic Rope:} The fibers in this rope can be controlled electronically, making it move in a snakelike fashion, stiffen up, and even wrap around objects. Typically comes in a 50- meter length capable of supporting 250 kg. \textbf{[Low]} 

\textbf{Emergency Bubble:} Commonly used as a last resort ``life raft'' on spaceships, an emergency bubble is made of advanced smart materials and comes in a portable package that can be quickly inflated (1 Action Turn) around the user, usually inside an airlock. The bubble has a 5-meter diameter and can comfortably accommodate 4 people. It maintains 1 atmosphere of pressure in a vacuum, protect the inhabitants from temperatures ranging from $-$175 to 140$^{\circ}$ C, and provide light, breathable air and water and food recycling for up to four human-sized inhabitants, using its built in maker (p. 327). It features a simple airlock, carries an emergency distress beacon (below), and can be transparent, opaque, or polarized. It is powered by a small nuclear battery and also includes comfortable inflatable furniture. \textbf{[Moderate]} 

\textbf{Emergency Distress Beacon:} This small but powerful transmitter is powered by a nuclear battery and will broadcast any programmed distress call for years. Though portable and medium-sized, this beacon has a range of 500 km in urban areas and 5,000 km elsewhere. \textbf{[Moderate]} 

\textbf{Flashlight:} These handheld, wearable, or portable lights can display light in the normal visual spectrum, infrared, or ultraviolet, as desired. \textbf{[Trivial]} 

\emph{Nanobandage:} Characters without medichines must rely on external sources of healing. The most common option is the nanobandage --- a plum-sized advanced nanotechnology generator built into a reusable, selfsterilizing bandage. It can treat all forms of injury and illness, from poisoning to burns to trauma. Characters simply apply the bandage to the wound and let the nanobots do the work. It removes pain and discomfort and speeds healing (see \emph{Biomorph Healing}, p. 208). For especially severe injuries, physical first aid such as setting bones and removing projectiles may be necessary (gamemaster’s choice). If the wounds are too severe (the patient has suffered more than five wounds), the unit places the patient in medical stasis and radios for emergency services. \textbf{[Trivial]} 

\emph{Repair Spray:} This nanobot generator creates nanobots designed to repair synthmorphs, vehicles, and other common objects. Repair spray contains the specifications and plans for almost all commonly used synthmorphs and devices and is a ubiquitous household item. If it does not contain the specifications for something it is being used to repair, it must query the object’s voice for these details, otherwise it cannot repair it. Simply touch it to the damaged area, push the button on top, and it sprays out a number of nanobots sufficient to make repairs. These nanobots repair 1d10 points of damage per 2 hours. Once all damage is restored, the nanobots repair wounds at the rate of 1 per day. Repair spray also cleans and polishes items and returns them to a pristine and new state. Repair spray is not effective on any object with more than 3 wounds, but it provides a +30 to all repair rolls on anything too badly damaged for it to fully repair (see \emph{Synthmorph and Object Repair}, p. 208). \textbf{[Low]} 

\textbf{Shelter Dome:} A variant of the emergency bubble, this package unfolds into a dome with a 2.5-meter ceiling and a floor 4 meters across. To safely use this shelter, it must be staked down to the surface it is placed on. \textbf{[Moderate]} 

\textbf{Spindle:} A spindle is an advanced nanotechnology generator that produces a super-strong cable. It can produce up to 2 kilometers of 0.2 millimeter diameter line than can support up to 250 kilograms before it needs more raw materials. The spindle can produce up to 20 meters of cable every second. It can produce line in a continuous length or cut the cable it produces to any length. Spindles can also reabsorb their cable, retracting it at a rate of 5 m per second. As long as it is recharged and has small amounts of additional material added every 1,000 hours of use, a spindle can keep producing and retracting cable indefinitely. By setting the maximum production speed at 10 m/ second a character with a spindle can safely jump off a building and land safely, using the cable to slow their descent. \textbf{[Moderate]} 

\textbf{Spindle Climber:} This device attaches to a spindle and transforms it into a highly effective climbing device. The spindle climber has two functions. First, it attaches hardened tips to the spindle’s cable and fires it at high speed, up to 50 meters, with sufficient force to imbed the tip into almost any sufficiently durable surface. Second, the spindle climber can pull itself and up to 250 kg up the cable at a speed of up to 2 m/sec. A spindle climber has enough power to shoot and pull up the cable 50 times before it must be recharged. A spindle fits inside a spindle climber. \textbf{[Low]} 

\subsubsection{Vacuum suits} 

Most vacuum suits are skin-tight garments that use the pressure of their advanced smartfabrics on the wearer’s body to resist vacuum. When the wearer is in a breathable atmosphere, the smartfabric also loosens the suits to serve as ordinary clothing or be easily put on or taken off. In all cases, the suits can become skin-tight within 3 Action Turns. All vacsuits contain advanced rebreather units capable of maintaining a breathable atmosphere for several hours or days. 

\textbf{Light Vacsuit:} Everyone living in a sealed habitat owns at least one of these suits. They come in a variety of forms. Inexpensive versions are typically lightweight jumpsuits made of simple smart fabric that adjusts to fit and folds up small enough to fit into a coat pocket. The best models include suits of high-end smart clothing that can transform into a vacsuit and an advanced nanotech generator the size of a large orange that deploy nanobots that cover the user and fit together into a vacuum suit. Both can transform into a vacsuit in 2 full Action Turns and do so either on command or if their sensors reveal that life support is needed. 

All models include a lightweight belt or torc containing a miniature oxygen tank and advanced rebreather unit that provides 3 hours of air. However, the suits contain no food or water recycling. All models include an ecto (p. 325) and a headlight, but typically little else beyond atmosphere sensors to let the wearer know when it is safe to take off the suit. They protect the wearer from temperatures from $-$75 to 100$^{\circ}$ C. These vacuum suits also provide an Armor rating of 5/5 and instantly self-seal breaches unless more than 20 points of damage are inflicted at once. \textbf{[Low, Moderate for smartfabric suits]} 

\textbf{Standard Vacsuit:} These suits resemble light vacsuits made from thicker and more durable materials that resist tearing and provides the wearer with light armor. They are fitted with more substantial life support belts that includes a maker (p. 327) capable of recycling all wastes and producing air for up to 48 hours and food and water indefinitely. The best suits are made of smart materials that can transform from standard clothing to vacuum suits in a single Action Turn, and will do so automatically if life support is needed. Each suit also contains an ecto (p. 325), a radio booster (p. 313), and sensors equal to specs (see p. 325). These suits have an Armor rating of 7/7 and protect the wearer from temperatures from $-$175 to 140$^{\circ}$ C. They can almost instantly seal any hole unless more than 30 points of damage are inflicted at once. \textbf{[Moderate, High for smartfabric suits]} 

\textbf{Hard Suit:} This heavy-duty suit can almost be considered a miniature space ship. Hard Suits look like large metallic ovals with jointed arms and legs. They are quite heavy, but the user can move relatively easily by using servo assist motors in all the major joints of the arms and legs. Unlike other vacsuits, they are solid and can resist both vacuum and up to 100 atmospheres of external pressure. Characters wearing hard suits can safely explore the upper atmosphere of a gas giant. They are well armored against punctures and radiation and possess miniature plasma thrusters capable of delivering 0.01G for 10 hours. A built-in high quality maker produces sufficient food, air, and water that a user can remain in a hard suit indefinitely. Explorers have used them continuously for up to 2 months. Their gloves incorporate smart materials that allow each hand to use the equivalent of a utilitool (p. 326). Hard suits also contain radios and sensors equivalent to those on standard vacsuits. These suits have an Armor rating of 15/15, are maintained by a fixer nanohive (p. 329), and are instantly self-sealing of any breach unless more than 30 points of damage are inflicted at once. They protect the wearer from temperatures of $-$200 to 180$^{\circ}$ C. \textbf{[High]} 



\section{Weapons} \label{sec:weapons} 

A wide range of weapons are available in \emph{Eclipse Phase}, from the primitive to the technologically advanced. 



\subsection{Melee weapons} \label{sec:melee-weapons} 

Melee weapons are those wielded by hand (or foot) in melee combat. They are divided by the skill be which they are used. 

\subsubsection{Blades} 

These weapons are wielded with Blades skill. 

\textbf{Diamond Axe:} Commonly found on many habitats for fire and emergency purposes, axes require two hands to wield. Their blades are diamond-coated for superior cutting ability. \textbf{[Low]} 

\textbf{Flex Cutter:} The blade of this machete-like weapon is made of a memory polymer. When deactivated, the blade is limp and flexible, and may even be rolled up or otherwise easily concealed. When activated, however, the blade stiffens and sharpens into a vicious slashing weapon. \textbf{[Low]} 

\textbf{Knife:} A standard cutting implement, still carried by many. \textbf{[Trivial]} 

\textbf{Monofilament Sword:} Though swords are rather archaic in the time of Eclipse Phase, a few eccentrics take advantage of modern versions with a selfsharpening near-monomolecular edge, easily capable of slicing through metal or limbs. \textbf{[Low]} 

\textbf{Vibroblade:} These buzzing electronic blades vibrate at a high frequency for extra cutting ability. This has little extra effect when stabbing or slashing, but provides an extra $-$3 AP and +2d10 damage when carefully sawing through something. \textbf{[Low]} 

\textbf{Wasp Knife:} Wasp knives are equipped with a canister in their handle. The common use is to fill these canisters with pressured air, which inflates inside the target. This is potentially lethal in vacuum or pressurized environments (like underwater), as the gas bursts out of the body cavity to escape (+2d10 damage in such situations). Wasp knives may also be loaded with chemicals, drugs, or nanobots. The target must be damaged for the canister’s contents to affect them. \textbf{[Low]} 

\subsubsection{Clubs} 

Characters use Clubs skill when using these weapons. 

\textbf{Club:} Clubs encompasses a wide range of one-handed blunt objects, from saps to sticks to pipes. \textbf{[Trivial]} 

\textbf{Extendable Baton:} This hardened composite baton retracts into its handle for easy carrying, storage, or concealment. Extending it simply requires a flick or an electronic signal. \textbf{[Trivial]} 

\textbf{Shock Baton:} Shock batons are standard clubs used for policing duties, but when activated they also deliver an electric shock to struck targets (see Shock Attacks, p. 204). \textbf{[Low]} 

\subsubsection{Exotic melee weapons} 

Unusual weapons requires a specific Exotic Melee field skill to use. 

\textbf{Monowire Garrote:} This assassin’s weapon features a dangerous monomolecular wire wrapped around a contained spool with two handles. One handle grips the spool, while the other extends the wire so that it may be used to wrap around targets (typically necks or limbs) and slice through them when pulled. Monofilament tensile strength is weak, however, usually breaking after one use. \textbf{[Moderate]} 

\subsubsection{Unarmed} 

These weapons are wielded using Unarmed Combat skill. 

\textbf{Densiplast Gloves:} These gloves extra-harden when activated, for extra punch. \textbf{[Trivial]} 

\textbf{Shock Gloves:} When activated, these gloves deliver an incapacitating shock along with every punch or grab. Note that the effect is the same whether wearing one glove or two. \textbf{[Low]} 







\begin{table} \begin{tabularx}{\hline}{|l|X|l|l|} \hline

\hline{4}{|c|}{\textbf{Melee weapons --- Blades, Clubs, Exotic, Unarmed}} \\ \hline

&\textbf{Armor penetration (AP)}	&\textbf{Damage value (DV)}	&\textbf{Average DV} \\ \hline

\hline{4}{|l|}{\emph{Blades}} \\ \hline

Diamond Ax	&$-$3	&2d10 + 3 + (SOM $\div$ 10)	&14 + (SOM $\hline$ 10) \\ \hline

Flex Cutter	&$-$1	&1d10 + 3 + (SOM $\div$ 10)	&8 + (SOM $\hline$ 10) \\ \hline

Knife	&$-$1	&1d10 + 2 + (SOM $\div$ 10)	&7 + (SOM $\hline$ 10) \\ \hline

Monofilament Sword	&$-$4	&2d10 + 2 + (SOM $\div$ 10)	&13 + (SOM $\hline$ 10) \\ \hline

Vibroblade	&$-$2	&2d10 + (SOM $\div$ 10)	&11 + (SOM $\hline$ 10) \\ \hline

Wasp Knife	&$-$1	&1d10 + 2 + (SOM $\div$ 10)	&7 + (SOM $\hline$ 10) \\ \hline

\hline{4}{|l|}{\emph{Clubs}} \\ \hline

Club	&--- &1d10 + 2 + (SOM $\div$ 10)	&7 + (SOM $\hline$ 10) \\ \hline

Extendable Baton	&--- &1d10 + 2 + (SOM $\div$ 10)	&7 + (SOM $\hline$ 10) \\ \hline

Shock Baton	&--- &1d10 + 2 + (SOM $\div$ 10) + shock (p. 204)	&7 + (SOM $\hline$ 10) \\ \hline

\hline{4}{|l|}{\emph{Exotic melee weapons}} \\ \hline

Monowire Garrote	&$-$8	&3d10	&16 \\ \hline

\hline{4}{|l|}{\emph{Unarmed}} \\ \hline

Bioware Claws (p. 304)	&$-$1	&1d10 + 1 + (SOM $\div$ 10)	&6 + (SOM $\hline$ 10) \\ \hline

Cyberclaws (p. 307)	&$-$2	&1d10 + 3 + (SOM $\div$ 10)	&8 + (SOM $\hline$ 10) \\ \hline

Densiplast Gloves	&--- &1d10 + 2 + (SOM $\div$ 10)	&7 + (SOM $\hline$ 10) \\ \hline

Eelware (p. 304)	&--- &shock (p. 204)	--- \\ \hline

Shock Gloves	&--- &1d10 + (SOM $\div$ 10) + shock (p. 204)	&5 + (SOM $\hline$ 10) \\ \hline

Unarmed	&--- &1d10 + (SOM $\div$ 10)	&5 + (SOM $\hline$ 10) \\ \hline

\label{tab:meleeweapons} \label{tab:meleeweapons} \end{table} 



\subsection{Kinetic weapons} \label{sec:kinetic-weapons} 

Kinetic weapons damage the target by firing a hard impact projectile at high-velocities. Slugthrowers have evolved from the mechanical firearms of the early 21st century, however, and now fall into two categories: chemical firearms and railguns. Though their mechanisms for firing are different, they are roughly similar in effect. Railguns have a higher penetration and inflict more damage, which is offset by more limited ammunition choices. While modern beam weapons have their uses, they rarely match the punch of kinetic weapons, therefore slugthrowers are still perceived as the most versatile and effective weapon system. 

Kinetic weapons are constructed from lightweight, reinforced plastoceramic materials, which are easily produced even without nanofabrication. By default, modern kinetic weapons are ambidextrous but more importantly feature safety and smartlink systems (p. 342) that automatically connect to the wielder’s mesh inserts for firing assistance, target recognition, and tactical networking. 

The wielder of a firearm or railgun uses Kinetic Weapons skill. For information on firing modes, see p. 198. For different ammunition types, see p. 336. Ranges are listed on p. 203. 

\subsubsection{Firearms} 

Modern chemical firearms use caseless ammunition that is auto-loaded from a magazine. They are effectively recoilless (thanks to rheological smart fluid mechanisms) and electronically fired (an electric charge vaporizes the propellant, using the expanding steam and plasma to eject and accelerate the projectile). Note that older, pre-Fall firearms still exist and are traded by black marketeers, though they use outdated system such as liquid propellants or cased ammunition. At the gamemaster’s discretion, these relics may suffer shorter ranges, less penetration, fewer firing modes, or reduced damage. 

\textbf{Pistols:} Pistols are small-sized (p. 297) and designed for one-hand use. Light pistols sacrifice penetrating ability for concealability. Heavy pistols focus on stopping power, with medium pistols occupying a middle ground. All versions fire in semi-automatic, burst-fire, and full-auto modes. \textbf{[Low]} 

\textbf{Submachine Guns:} SMGs use pistol ammunition, but are medium-sized (p. 297) and may fire in semi-auto, burst fire, or full auto modes. They typically are designed in a bullpup configuration for close quarters operations and are ideal for tactical and strike teams. \textbf{[Moderate]} 

\textbf{Automatic rifles:} Automatic rifles use rifle ammunition and have greater range and penetration than SMGs. They fire in semi-auto, burst fire, or full auto modes. They are two-handed weapons. \textbf{[Moderate]} 

\textbf{Sniper rifle:} Sniper rifles are optimized for range, accuracy, penetration, and stopping power. They fire in semi-auto, burst fire, or full auto modes, and are two-handed weapons. \textbf{[High]} 

\textbf{Machine Gun:} Machine guns are heavy weapons, typically mounted, and intended to provide continuous fire for support or suppressive purposes. They fire in burst fire or full auto modes, and are twohanded weapons. \textbf{[High]} 

\begin{table} \begin{tabularx}{\hline}{|l|X|X|X|l|l|} \hline

\hline{6}{|c|}{\textbf{Kinetic weapons --- Firearms}} \\ \hline

&\textbf{Armor penetration (AP)}	&\textbf{Damage value (DV)}	&\textbf{Average DV}	&\textbf{Firing modes}	&\textbf{Ammo} \\ \hline

Light Pistol	&--- &2d10	&11	&SA, BF, FA	&10 \\ \hline

Medium Pistol	&$-$2	&2d10 + 2	&13	&SA, BF, FA	&12 \\ \hline

Heavy Pistol	&$-$4	&2d10 + 4	&15	&SA, BF, FA	&16 \\ \hline

Submachine Gun	&$-$2	&2d10 + 3	&14	&SA, BF, FA	&20 \\ \hline

Automatic rifle	&$-$6	&2d10 + 6	&17	&SA, BF, FA	&30 \\ \hline

Sniper rifle	&$-$12	&2d10 + 10	&21	&SA	&40 \\ \hline

Machine Gun	&$-$6	&2d10 + 6	&17	&BF, FA	&50 \\ \hline

\label{tab:kinetic-firearms} \label{tab:kinetic-firearms} \end{table} 

\subsubsection{Railguns} 

Railguns use a pair of electromagnetic rails to slide and accelerate a non-explosive conductive projectile at extremely high velocities (Mach 6+) to create an overwhelming, penetrating attack. The kinetic energy of the projectile exceeds that of an explosive-filled shell of greater mass and creates shock and heat waves upon impact that shatter and incinerate the target, or portions of it. While railguns are more potent than firearms, the ammunition choices are limited as the projectile must be conductive and able to survive both acceleration and heat created in the process due to friction. Nanofabrication allows railguns to be manufactured on the personal weapons scale while high-energy portable batteries provide the power to fire them. Railgun operation is silent except for the supersonic crack of the projectile. 

Railguns are available in the same models as firearms (pistols through machine guns), with the following modifications: 

\end{itemize} \item Increase AP by $-$3 \item Increase damage by +2 \item Increase the maximum for each range category by x1.5 \item Increase Cost category by one \item Railguns may only use regular and armor-piercing ammunition \item Railguns also require battery power for each shot. Standard railgun batteries hold enough power for 200 shots, after which they must be recharged at the rate of 20 points per hour. \end{itemize} 

\begin{table} \begin{tabularx}{\hline}{|l|X|X|X|X|l|} \hline

\hline{6}{|c|}{\textbf{Kinetic weapons --- Railguns}} \\ \hline

&\textbf{Armor penetration (AP)}	&\textbf{Damage value (DV)}	&\textbf{Average DV}	&\textbf{Firing modes}	&\textbf{Ammo} \\ \hline

Light Pistol	&$-$3	&2d10 + 2	&13	&SA, BF, FA	&10 \\ \hline

Medium Pistol	&$-$5	&2d10 + 4	&15	&SA, BF, FA	&12 \\ \hline

Heavy Pisto	&$-$7	&2d10 + 6	&17	&SA, BF, FA	&16 \\ \hline

Submachine Gun &$-$5	&2d10 + 5	&16	&SA, BF, FA	&20 \\ \hline

Automatic Rifle	&$-$9	&2d10 + 8	&19	&SA, BF, FA	&30 \\ \hline

Sniper Rifle	&$-$15	&2d10 + 12	&23	&SA	&40 \\ \hline

Machine Gun	&$-$9	&2d10 + 8	&19	&BF, FA	&50 \\ \hline

\label{tab:kinetic-railguns} \label{tab:kinetic-railguns} \end{table} 

\subsubsection{Kinetic ammunition} 

Ammunition is defined by its various types (standard, gel, APDS, etc.) and by the class of gun (light pistol, heavy pistol, SMG, etc.). For simplicity, each gun can trade ammunition with another gun of its class, though ammunition for firearms and railguns is not exchangeable. For example, all railgun SMGs can share ammo. 

The ammunition’s Damage Value and Armor Penetration modifiers are added to the weapon’s base DV and AP. With the exception of regular and armorpiercing rounds, none of this ammunition may be used with railguns. Listed costs are per 100 rounds of ammunition. 

\textit{Armor-Piercing:} This tungsten-carbide ammunition penetrates armor effectively. \textit{[Low]} 

\textbf{Bug:} Bug rounds are equipped with a microbug and medical sensor nanobots. They attempt to gather information on the target’s location (via standard mesh tracking), health (querying the target’s medichines), and surroundings (typically hindered by being inside the target’s body). They will transmit status reports in a pre-programmed manner via the mesh or a prechosen frequency band either continuously or in preset intervals. \textbf{[Low]} 

\textbf{Capsule:} Capsule ammo carries a payload (drug, toxin, nanobots) that is released inside the target after the round penetrates. [Trivial plus payload cost] Flux: Flux ammo is made from rheological materials that allow each bullet to be ``programmed'' so that they may change from regular rounds to less-lethal soft plastic-like rounds. This allows the firer to choose the type of round (regular or plastic) made with each shot or burst, and then change with the next one. \textbf{[Low]} 

\textbf{Hollow-Point:} Hollow-point bullets are designed to deform and widen once they penetrate a target, thus inflicting more damage. \textbf{[Trivial]} 

\textbf{Jammer:} Jammers stick to the target and pulse out jamming electromagnetic signals, jamming the target’s wireless communications. If an Opposed Test is called for, these devices have an Interface of 30. See Radio Jamming, p. 262. \textbf{[Low]} 

\textbf{Plastic:} Plastic ammo is designed to hurt but not wound targets, and is commonly used for crowd control purposes. \textbf{[Trivial]} 

\textbf{Reactive:} The casing on these projectiles is made of reactive materials that release a large amount of energy when subjected to a sudden shock or impact --- such as striking a target. In other words, they explode or superheat when they hit. \textbf{[Low]} 

\textbf{Reactive Armor-Piercing (RAP):} This is a tungsten- carbide armor-piercing round with a reactive casing, allowing the ammunition to penetrate even further. \textbf{[Moderate]} 

\textbf{Regular Ammo:} This standard metal projectile is designed to put holes into morphs. \textbf{[Trivial]} 

\textbf{Splash:} Splash rounds carry a payload like capsule ammo, but are designed to break upon impact rather than penetrating, splashing their contents on the target’s exterior. Splash rounds are typically loaded with paint, taggant nanobots, tracker dye, and similar substances. \textbf{[Trivial plus payload cost]} 

\textbf{Zap:} Zap rounds are rubber or gel bullets that create an electric charge upon firing in a piezoelectric like manner to stun the target effectively with both the bullet and the electric shock. \textbf{[Trivial]} 

\subsubsection{Smart ammo} 

Smart ammunition takes advantage of nanotechnology to produce bullets that can alter their flight path, home in the target, and correct aim. Smart ammo may not be used with railguns. With the exception of biter, flayer, and proximity rounds, smart ammo may be combined with other ammo types (accushot armorpiercing, for example). 

\textbf{Accushot:} Accushot bullets change shape within flight to keep dead on course, countering the effects of wind, drag, and gravity over distance. Attacks made with accushot bullets ignore all range modifiers. \textbf{[Low]} 

\textbf{Biter:} Biters are specially-designed to fragment in opposite proportion to the hardness of the target they strike. For hard targets (synthmorphs), they fragment very little, blasting a big hole. For soft targets (biomorphs), they fragment and tumble in multiple directions within the body. \textbf{[Low]} 

\textbf{Flayer:} Flayers have nanosensors to detect an oncoming impact, shooting out monomolecular barbs as they are about to strike a target. These monowires cut through the target along with the bullet, inflicting additional damage. \textbf{[Low]} 

\textbf{Homing:} When fired with a smartlink system, the bullet identifies the target and uses nanosensors to lock on, correcting the bullet’s trajectory with surface alterations and tiny vectored nozzles. Apply a +10 modifier to the Attack Test, cumulative with aiming and smartlink modifiers. Homing bullets may also be used for indirect fire (p. 195). \textbf{[Low]} 

\textbf{Laser-Guided:} These bullets function like homing smart rounds (apply the +10 attack modifier), except rather than requiring a smartlink system, they lock onto the reflection of the laser sight used to paint the target. Laser-guided bullets may also be used for indirect fire (p. 195). \textbf{[Low]} 

\textbf{Proximity:} Proximity is an explosive ammunition that identifies the target when fired via smartlink. If the round determines that it will miss the target, it will still explode if it reaches the close proximity of the target. If the attack misses with an MoF of 10 or less, the round explodes 1d10 meters away from the target and inflicts 1d10 area effect damage (see Blast Effect, p. 193) in the proximity of the target. \textbf{[Moderate]} 

\textbf{Zero:} Similar to homing smart rounds, zero bullets identify the target when fired via smartlink. Whether the round hits or misses, however, it sends telemetry data back to the next zero bullet, allowing it to course-correct and ``zero in'' to hit the target (or hit more accurately). Apply a +10 modifier to each shot (or burst) fired after the first against the same target in the same Action Turn. \textbf{[Low]} 

\begin{table} \begin{tabular}{|l|l|l|} \hline

\hline{3}{|c|}{\textbf{Kinetic Ammunition}} \\ \hline

Armor-Piercing	&$-$5	&$-$2 \\ \hline

Bug	&+1	&$-$1d10 \\ \hline

Capsule	&+1	&$-$half \\ \hline

Flux	&as ammo type	&as ammo type \\ \hline

Hollow-Point	&+2	&+1d10 \\ \hline

Jammer	&--- &no damage \\ \hline

Plastic	&(AV doubled)	&$-$half \\ \hline

Reactive	&$-$2	&+2 \\ \hline

Reactive Armor-Piercing	&$-$6	&$-$1 \\ \hline

Regular	&--- &--- \\ \hline

Splash	&--- &no damage \\ \hline

Zap	&+2	&$-$half + shock (p. 204) \\ \hline

\hline{3}{|l|}{\emph{Smart ammo}} \\ \hline

Accushot	&--- &--- \\ \hline

Biter	&--- &+1d10 \\ \hline

Flayer	&--- &+2 \\ \hline

Homing	&--- &--- \\ \hline

Laser-Guided	&--- &--- \\ \hline

Proximity	&$-$1	&+2 \\ \hline

Zero	&--- &--- \\ \hline

\label{tab:kinetic-ammo} \label{tab:kinetic-ammo} \end{table} 



\subsection{Brand name weapons and combined arms} \label{sec:brand-weapons-combined} 

The weapons listed in this book define generic samples of each weapon. Gamemasters are encouraged to offer brand name versions of each weapon, each with its particular idiosyncrasies and small variations. For example, a Direct Action A30 SMG might lack a semi-automatic setting but come equipped with an extra ammo capacity of 35. Likewise, a Medusan Arms Longinus sniper rifle may inflict an extra +2 damage but have an AP of only $-$12. 

Similarly, many of the weapons listed here are available as combined arms weapons systems. A police-issue assault rifle may also feature a stunner --- all built into the same weapon. For combined arms, simply add together the individual weapon component costs. 



\subsection{Beam weapons} \label{sec:beam-weapons} 

Beam weapons is a broad category for a number of electromagnetic weapons with a wide range of effects. With a few exceptions, energy weapons are primarily used for less-than-lethal purposes, designed to impair the target rather than kill it. Their poor performance against armor, lesser ability to damage targets, and high power requirements make them less versatile than kinetic weapons. The wielders of such weapons use Beam Weapons skill. Beam weapons are powered by nuclear batteries. This battery is good for a list number of shots before it is depleted. Batteries may be recharged at the rate of 20 shots per hour; they have a Cost of \textbf{[Low]}. 

\textbf{Laser Pulsers:} Laser weapons use focused beams of light to inflict damage on the target by burning into it and causing it’s outer surface to vaporize and expand, creating an explosive effect. The laser beam is pulsed in order to bite into the target before the beam is diffused. Pulsers are vulnerable to atmospheric effects like dust, mist, smoke, or rain, however --- the gamemaster should reduce their effective range categories as appropriate. Note that laser pulses are invisible in the normal visual spectrum (but are visible to characters with enhanced vision). Pulsers are medium-sized (p. 297) and fire in semi-auto mode. \textbf{[Moderate]} 

One advantage to the pulser is that it can be placed in less-lethal mode. In this case, it first fires a pulse at the target to create a ball of plasma, quickly fired by a second pulse that strikes the plasma and creates a flash-bang shockwave to stun and disorient the target. This blast has an area of effect with a 1-meter radius. Anyone caught in the blast must make a SOM $\times$ 2 Test (SOM $\times$ 3 for synthmorphs or biomorphs with any form of pain tolerance). Failure means the target is temporarily stunned and disoriented and loses their next action. A critical failure means the target is knocked down and paralyzed for 1 Action Turn per 10 points of MoF. In this stun setting, the pulser fires only in single-shot mode. 

\textbf{Microwave Agonizer:} The agonizer fires millimeterwave beams that create an unpleasant burning sensation in skin (even through armor) and to metals. Agonizers have two settings. The first is an active denial setting that causes extreme burning pain in the target, inflicting $-$20 to the target’s actions and forcing them to move away from the beam on their next action unless they succeed in a WIL Test (targets with Level 1 Pain Tolerance or the equivalent only suffer a $-$10 modifier and roll WIL $\times$ 2). Synthetic morphs and biomorphs with Level 2 Pain Tolerance (or the equivalent) are immune to this weapon. The second setting (colloquially known as the ``roast'' setting) has the same effect of the first, but also actually burns the target, inflicting the listed damage. Originally developed for crowd control, the agonizer is also useful for repelling animals. The agonizer is small-sized (p. 297) and fires in semi-auto mode. \textbf{[Moderate]} 

\textbf{Particle Beam Bolter:} This weapon shoots a bolt of accelerated particles at near light speed that transfer massive amounts of kinetic energy to the target, superheating and creating an explosion when striking. The bolter’s beam is not diffused by the cloud that occurs when it strikes, and so it has greater penetration than the laser pulser. Likewise, the bolter is not affected by smoke, fog, or rain. The bolter’s beam is invisible. Note that bolters are designed for either atmospheric or exoatmospheric (vacuum) operation, and will not function in the opposite environment (though bulkier dual models, combining both models, are also available). Bolters fire in semi-auto mode and are rifle-sized two-handed weapons. \textbf{[High]} 

\textbf{Plasma Rifle:} This bulky, heavy, two-handed weapon blasts a stream of nova-hot plasma at the target, inflicting severe burns and thermal damage, possibly melting or evaporating the target entirely. Plasma rifles are perhaps the deadliest man-portable weapons in use. Plasma guns suffer from dangerous overheating, however, and so require 1 full Action Turn of cool-down time after every 2 shots. Plasma rifles only fire in single shot mode. [Expensive] Stunner: The stunner is an electrolaser that creates an electrically-conductive plasma channel to the target, down which it transmits a powerful electric current, shocking the target. Stunners do not work in vacuum. Stunners fire in semi-auto mode. \textbf{[Moderate]} 

\begin{table} \begin{tabularx}{\hline}{|X|X|X|X|l|l|} \hline

\hline{6}{|c|}{\textbf{Beam weapons}} \\ \hline

&\textbf{Armor penetration (AP)}	&\textbf{Damage value (DV)}	&\textbf{Average DV}	&\textbf{Firing modes}	&\textbf{Ammo} \\ \hline

Cybernetic Hand Laser (p. 308)	&--- &2d10	&11	&SA	&50 \\ \hline

Laser Pulser	&--- &2d10	&11	&SA	&100 \\ \hline

Stun Mode	&--- &1d10	&5	&SS	&--- \\ \hline

Microwave Agonizer	&--- &pain (see description)	&--- &SA	&100 \\ \hline

Roast Mode	&$-$5	&2d10	&11	&SA	&50 \\ \hline

Particle Beam Bolter	&$-$2	&2d10 + 4	&15	&SA	&50 \\ \hline

Plasma Rifle	&$-$8	&3d10 + 12	&28	&SS	&10 \\ \hline

Stunner	&--- &(1d10 $\div$ 2) + shock (p. 204)	&--- &SA	&200 \\ \hline

\label{tab:beam-weapons} \label{tab:beam-weapons} \end{table} 



\subsection{Seekers} \label{sec:seekers} 

Seekers are a combination of automatic grenade launcher, micromissile, coilgun, and smart munitions technology. Unlike traditional launchers of the past, miniaturization allows the manufacture of seeker micromissile launchers in personal weapon sizes. Seeker rounds are fired at high-velocity via rings of magnetic coils, after which the micromissile or minimissile uses scramjet technology to propel itself and maintain high velocities over great distances. Seekers are wielded using Seeker Weapon skill. 

Seeker missiles are detailed on p. 340. Like grenades, seekers may be programmed for a variety of trigger events (see Grenades and Seekers, p. 199). All seeker weapons are smartlink-equipped (p. 342). 

\textbf{Disposable Launcher (Standard Missile):} This launcher is pre-packed with one standard missile. \textbf{[Moderate (includes missile)]} 

\textbf{Seeker Armband (Micromissile):} This weapons unit is worn on the arm, allowing the user to point and fire using an entoptic smartlink system. Though highly portable, the armband’s micromissile supply is low. It fires in single-shot mode. \textbf{[Moderate]} 

\textbf{Seeker Pistol (Micromissile):} This pistol-sized seeker launcher fires micromissiles in semi-auto mode. \textbf{[Moderate]} 

\textbf{Seeker rifle (Micromissile/Minimissile):} The seeker rifle comes in a bullpup configuration and fires either micromissiles or minimissiles in semi-auto mode. It is a two-handed weapons. \textbf{[High]} 

\textbf{Underbarrel Seeker (Micromissile):} This seeker micromissile launcher is commonly attached to the underbarrel of SMGs or assault rifles. It fires in semiauto mode. \textbf{[Moderate]} 

\begin{table} \begin{tabular}{|l|l|l|} \hline

\hline{3}{|c|}{\textbf{Seeker weapons}} \\ \hline

&\textbf{Firing modes}	&\textbf{Ammo} \\ \hline

Disposable Launcher	&SS	&1 \\ \hline

Seeker Armband	&SS	&4 \\ \hline

Seeker Pistol	&SA	&8 \\ \hline

Seeker Rifle	&SA	&12 micromissile/6 minimissile \\ \hline

Underbarrel Seeker	&SA	&6 \\ \hline

\label{tab:seeker-weapons} \label{tab:seeker-weapons} \end{table} 



\subsection{Spray weapons} \label{sec:spray-weapons} 

Spray weapons blast their ammunition outwards in a widening cone, allowing them to strike several targets at once. These weapons are wielded with Spray Weapons skill. Spray weapon ammunition has a flat cost of Low per 100 shots (with the exception of buzzers, which use nanoswarms). 

\textbf{Buzzer:} Equipped with a specialized nanobot hive, Buzzers are used to spray a nanoswarm (p. 328) on a target or area. They have a limited capacity of swarms, though the nanohive can construct one new swarm each hour. This weapon is two-handed. \textbf{[Moderate]} 

\textbf{Freezer:} Freezers spew out a fast-hardening foam that immediately begins to harden. They are primarily used as a non-lethal method of immobilizing or securing a target. Struck characters must immediately make a REF $\times$ 3 Test or become trapped. Apply a $-$30 modifier to this test if the attacker scored an Excellent Success (MoS 30+). The foam allows characters to breath even if their mouth and nose are covered, but it may impede sight. Freezer foam can be spiked with contact toxins or drugs to additionally sedate the target. It can also be used to construct temporary barricades or cover. Hardened foam has an Armor of 10 and Durability of 20. It slowly breaks down and degrades over a 12 hour period. Freezers are twohanded. \textbf{[Moderate]} 

\textbf{Shard Pistol:} The shard pistol is a flechette weapon, firing a stream of of diamondoid monomolecular shards at high velocities. These micro flechettes are very good at penetrating armor, but they do not disperse kinetic energy well and so do not cause as much tissue damage as kinetic weapons. Shard ammunition is often coated with drugs or toxins for extra efficiency. \textbf{[Low]} 

\textbf{Shredder:} A heavier version of the shard pistol, the shredder fires a larger cloud of lethal flechettes, enough to shred a portion of the target into a fine mist. \textbf{[Moderate]} 

\textbf{Sprayer:} This is a general-purpose two-handed squirtgun, loaded with tanks filled with the chemical or drug of the wielder’s choice. \textbf{[Low]} 

\textbf{Torch:} This modern flamethrower uses condensed ammunition capsules rather than fuel tanks, scorching targets and setting them on fire. Any hit that is an Excellent Success (MoS 30+) sets the target on fire, where they will continue to take 2d10 damage per Action Turn. These chemical fires are particularly difficult to put out unless they are deprived of oxygen. Torches are two-handed. \textbf{[Moderate]} 

\begin{table} \begin{tabularx}{\hline}{|l|X|l|l|l|l|} \hline

\hline{6}{|c|}{\textbf{Spray weapons}} \\ \hline

&\textbf{Armor penetration (AP)}	&\textbf{Damage value (DV)}	&\textbf{Average DV}	&\textbf{Firing modes}	&\textbf{Ammo} \\ \hline

Buzzer	&--- &nanoswarm	&--- &SS	&3 \\ \hline

Freezer	&--- &incapacitation	&--- &SA	&20 \\ \hline

Shard	&$-$10	&1d10 + 6	&11	&SA, BF, FA	&100 \\ \hline

Shredder	&$-$10	&2d10 + 5	&16	&SA, BF, FA	&100 \\ \hline

Sprayer	&as chemical/drug	&as chemical/drug	&as chemical/drug	&SA	&20 \\ \hline

Torch	&$-$4	&3d10	&16	&SS	&20 \\ \hline

\label{tab:spray-weapons} \label{tab:spray-weapons} \end{table} 

\subsection{Grenades and seekers} \label{sec:grenades-seekers} 

Grenades and seeker missiles come in similar munitions packages and with similar trigger mechanisms, though their packaging, physical form, and methods of application differ. Seeker missiles are fired from a seeker launcher (p. 339) using Seeker Weapons skill. Grenades are thrown at targets using Throwing Weapons skill. If a grenade or seeker misses, use the rules for scatter (p. 204). 

Grenades are available in standard form or as microgrenades. Similarly, missiles are available in standard, minimissile, or micromissile sizes. Standard grenades and minimissiles are the baseline standard for listed effects. All are area effect weapons (p. 193). Minigrenades and micromissiles inflict $-$1d10 damage (or have another decreased effect as noted). Standard missiles double the listed DV. For weapons with a uniform blast effect or other static blast area, divide the base listed radius in half for minigrenades and micromissiles and double it for standard missiles. Listed costs are for 10 grenades/missiles. 

Each seeker has one smart ammo option (p. 338) other than biter or flayer. 

\textbf{Concussion:} These devices emit a concussive blast designed to knock opponents off their feet and stun them. Any character caught within a base blast radius of 10 meters must make a SOM $\times$ 2 Test. If they fail, they are knocked down. If their MoF is 30+, they are additionally stunned until the end of the next Action Turn. Anyone caught in the blast radius suffers a $-$10 action modifier for the rest of that Action Turn. \textbf{[Moderate]} 

\emph{EMP:} EMP munitions fire off a strong electromagnetic pulse when they ``detonate.'' Since most electronics in \emph{Eclipse Phase} are built with optical technology, and power supplies and sensitive microcircuits are shielded and surge-protected, this has no major damaging effect. Antennas, however, are vulnerable, especially finer wires like those used with mesh inserts. As a result, the primary effect of EMP is to disable radio communications --- every radio within range of the blast is reduced to 1/10th the normal range. The base blast radius for EMP is 50 meters. \textbf{[High]} 

\textbf{Frag:} Fragmentation explosives spread a cloud of lethal flechettes over the area of effect. They are resisted with kinetic armor. \textbf{[Moderate]} 

\textbf{Gas/Smoke:} Gas/smoke munitions emit a cloud of their contained substance. Smoke impedes sight by releasing thick fumes upon ignition of the seeker. The smoke can be of any color and is often heated (called thermal smoke) to obfuscate heat signatures moving through the smoke as cover. Note that gases dissipate much more quickly under certain environmental conditions (wind, rain, etc.) \textbf{[Low]} 

\textbf{High-Explosive:} High-explosive seekers and grenades are designed to create a very destructive shock and heat wave. This damage is resisted with energy armor. \textbf{[Moderate]} 

\textbf{High-Explosive Armor-Piercing (HEAP):} A design only available for seekers (not grenades), HEAP warheads use high explosives to blast a path for a penetrating round. HEAPs lose $-$4 damage per meter distance from the blast, as opposed to the usual $-$2. \textbf{[Moderate]} 

\textbf{Overload:} Overload grenades and seekers launch an all-out assault on the target’s sensory spectrum. This attack includes blinding by intense flashing light, a deafening thunderclap followed by intense ultrasonic screaming, nausea-inducing malodorants, and infrasonic frequencies that can trigger unpleasant emotional responses (anxiety, uneasiness, extreme sorrow, nervous feelings of revulsion or fear). For an extra kick, overloads are also packed with ``stingballs'' --- rubber pellets that inflict pain when detonated near an underarmored target. Anyone caught in the base 10-meter blast radius must make a SOM + WIL Test. If they fail, they must immediately leave the area of effect. If they fail with an MoF of 30+, they are incapacitated for 3 Action Turns with disorientation and/or vomiting, after which they must roll again. Overload munitions remain in effect for 1 full minute. Anyone in the area of effect suffers a $-$30 action modifier, which reduces by 10 per Action Turn after they leave the area. Additionally, anyone facing the direction of the overload round suffers a $-$10 glare modifier (neutralized by anti-glare systems). \textbf{[Moderate]} 

\textbf{Plasmaburst:} Also called ``hellballs,'' these munitions release a burst of plasma upon detonation that causes searing heat and fire damage across the area of effect without the devastating shockwaves of explosions that might rebound in an enclosed environment and/or breach a habitat’s infrastructure. \textbf{[High]} 

\textbf{Splash:} Splash rounds spread a contained substance (drug, chemical, nanoswarm, paint) over a base 10- meter blast radius when they detonate. \textbf{[Low plus payload cost]} 

\textbf{Thermobaric:} Thermobaric grenades and seekers utilize a more deadly form of explosion. When they detonate, they disperse a cloud of aerosol explosive over an area and then ignite, literally setting the air on fire, generating a devastating pressure wave, and sucking the oxygen out of the area. Thermobarics use the rules for uniform blast (p. 194). \textbf{[High]} 

\subsubsection{Sticky grenades} 

Sticky grenades have a special coating that when triggered becomes a sticky adhesive, allowing the grenade to be stuck to almost any surface. Sticky grenades can even be wielded in melee combat, smacking them on an opponent to be detonated later. \textbf{[Trivial]} 

\begin{table} \begin{tabular}{|l|l|l|l|l|} \hline

\hline{5}{|c|}{\textbf{Grenades and seekers}} \\ \hline

\textbf{Type}	&\textbf{AP}	&\textbf{DV}	&\textbf{Average DV}	&\textbf{Armor used to resist} \\ \hline

Concussion	&--- &1d10 $\div$ 2	&5	&E \\ \hline

Frag	&$-$4	&3d10 + 6	&22	&K \\ \hline

EMP	&--- &--- &--- &--- \\ \hline

Gas/Smoke	&--- &--- &--- &--- \\ \hline

High-Explosive	&--- &3d10 + 10	&26	&E \\ \hline

HEAP	&$-$8	&3d10 + 12	&28	&K \\ \hline

Overload	&(AV $\times$ 2)	&1d10 $\div$ 2	&5	&K \\ \hline

Plasmaburst	&$-$6	&3d10 + 10	&26	&E \\ \hline

Splash	&--- &--- &--- &--- \\ \hline

Thermobaric	&$-$10	&3d10 + 5	&21	&E \\ \hline

\label{tab:grenades-seekers} \label{tab:grenades-seekers} \end{table} 

\subsection{Exotic ranged weapons} \label{sec:exotic-ranged-weapons} 

These weapons are either rare or distinctly separate from other weapons types. These weapons are wielded with an Exotic Ranged Weapon skill of the appropriate field. 

\textbf{Vortex Ring Gun:} This less-lethal two-handed weapon detonates a blank cartridge and accelerates the explosive pressure down a widening barrel so that it develops into a high-speed vortex ring --- a spinning, donut-shaped blast vortex. This concussive blast is used to knock down and incapacitate close-range targets. Struck targets suffer a $-$10 action modifier for the rest of that Action Turn and must must succeed in a SOM $\times$ 2 Test or fall down. If their MoF is 30+, they are additionally stunned and unable to act until the end of the next Action Turn. Drugs, chemicals, and similar agents may be loaded into the charge as well. \textbf{[Moderate]} 

\subsection{Weapon accessories} \label{sec:weapon-accessories} 

The following accessories are available for various weapons. 

\textbf{Arm Slide:} This slide-mount can hold a pistolsized weapon under a character’s sleeve, pushing the weapon into the character’s hand with an electronic signal or specific sequence of arm movements. \textbf{[Low]} 

\textbf{Extended Magazine:} This ammunition case has an increased capacity. Increase the weapon’s ammo capacity by +50\%. Only available for firearms and seekers. \textbf{[Low]} 

\textbf{Gyromount:} This weapon harness features a gyrostabilized weapon mount that keeps the weapon steady. Negates all modifiers from movement. \textbf{[Moderate]} 

\textbf{Imaging Scope:} Imaging scopes attach to the top of the weapon and act like specs (p. 325). Scopes may also bend like a periscope, along a character to point the weapon and target around corners without leaving cover. \textbf{[Low]} 

\textbf{Flash Suppressor:} This device obscures the muzzle flash on firearms, applying a $-$10 modifier on Perception Tests to locate a firing weapon by its flash. \textbf{[Low]} 

\textbf{Laser Sight:} This underbarrel laser emits a beam that places a glowing red dot on the target to assist targeting. Apply a +10 modifier to Attack Tests (not cumulative with a smartlink modifier). Laser sights may also be used to paint a target for laser-guided smart ammo or seekers. Infrared and ultraviolet lasers are also available, so that the dot is only visible to characters able to see in those spectrums. \textbf{[Low]} 

\textbf{Safety System:} A biometric (palmprint or voiceprint) or ego ID (p. 279) sensor is embedded in the weapon, disabling it if anyone other than an authorized user attempts to fire it. \textbf{[Low]} 

\textbf{Shock Safety:} Just like a safety system, except that an unauthorized user is zapped with an electric shock. Treat as a shock baton (p. 334). \textbf{[Moderate]} 

\textbf{Silencer/Sound Suppressor:} This barrel-mounted accessory reduces the sound of a firearm’s discharge. Apply a $-$10 modifier on hearing-based Perception Tests to hear or locate the gun’s firing. \textbf{[Moderate]} 

\textbf{Smartlink:} A smartlink system connects the weapon to the user’s mesh inserts, placing a targeting bracket in the character’s field of vision and providing range and targeting information. Apply a +10 modifier to the Attack Test. Smartlinks also incorporate a microcamera that allows the user to see what the weapon is pointed at, fire around corners, etc. Smartlinks also allow certain other types of weapon system control, such as changing flux ammo (p. 337) or programming seeker trigger conditions (p. 199). \textbf{[Moderate]} 

\textbf{Smart Magazine:} A smart magazine allows the character to pick and choose what ammo round will be fired with each shot. This system leaves less room for bullets, however, so reduce the weapon’s ammunition capacity by half (round up). Smart magazines may be combined with extended magazines, in which case ammo capacity is normal. \textbf{[Moderate]} 

\section{Robots and vehicles} \label{sec:robots-vehicles} 

The following is a small selection of the many vehicles in use in the solar system. Almost all of the vehicles in current use, including all of the vehicles listed here, have built-in AIs capable of piloting the vehicle under almost all circumstances. In most cases, passengers simply state their destination and the vehicle takes them there. Manual piloting is primarily used in emergencies or by people who prefer the exotic thrill of controlling their own vehicle. 

Rules for handling robots and vehicles are detailed on p. 195. Any of these shells may be modified for use as a synthetic morph by adding a cyberbrain system (p. 300). Each of the shells listed here comes with a puppet sock (p. 307) for remote-control operation. 

\subsection{Aircraft} \label{sec:aircraft} 

On Mars, Venus, and within large open-space habitats like O’Neil cylinders, aircraft of various kinds see regular use. This includes modern version of rotorcraft (helicopters, autogyros, tilt-rotors), fixed-wing planes, and zeppelins and other lighter-than-air craft. These are typically propelled by turbofan or jet engines, rotors, or vectored thrust. These vehicles are piloted with Pilot: Aircraft skill. Microlight: This ultra-light personal aircraft is not much more than a strut-based wing, an airframe, and an electric propeller engine. They are ideal for getting around inside large habitats with enclosed airspace. \textbf{[Low]} 

\textbf{Portable Plane:} Powered by superconducting batteries and with an exceedingly small but powerful electric motor, this light but durable propeller plane is made of smart materials that allow it to be swiftly folded up into a small portable package. Different versions are designed for flight on Mars, Titan, or Venus, each taking 10 minutes to assemble or disassemble. The Martian version unpacks into an airplane with a wingspan of 11 m with a top speed of 250 kph and a cruising speed of 220 kph and a range of 1,300 km. The Venusian version has a wingspan of 9 m, a top speed of 200 kph and a range of 1,000 km. The version designed for use on Titan has a wingspan of 8 m and has a top speed of 200 kph and a range of 2,000 km. In all versions, the two occupants ride in an inflatable and insulated pressurized bubble with a life support system capable of providing clean air and comfortable temperatures for 20 hours on Mars and Venus, and 15 hours on Titan. \textbf{[High]} 

\textbf{Rocket Buggy:} This vehicle is the most common form of medium to long distance personal transport on Luna, and is in common on most other moons and large asteroids. On these airless worlds, a rocket buggy can reach orbit and return or take a parabolic path to any destination on that moon in less than an hour. This vehicle is also regularly used to travel between habitats that are less than 30,000 km apart. The vehicle is pressurized, but is designed for short duration travel only. The seats are relatively small and the life support system contains no provisions for recycling food or water and can only support the passengers for an absolute maximum of 50 uncomfortable hours. Rockets buggies come equipped with headlights, radio boosters, and radar with a range of up to 250 km. 

A version of this vehicle is also used on both Mars and Titan, but here the frame has been modified to act as a lifting body, and it has a top speed in the thin Martian atmosphere of 2,500 km/hour and a range of 8,000 km on Mars. On Titan is has a top speed of 3,000 kph in the atmosphere, but it can also reach orbit. \textbf{[Expensive]} 

\textbf{Small Jet:} Methane-powered jet planes are one of the most common forms of fast transport on Mars and Venus. Similar planes are used on Titan, except that they carry both liquid methane and liquid oxygen. These jets range in size from huge vehicles the size of late 20th-century airliners to small planes designed to carry half a dozen passengers. All jets are made using smart materials, so that their wings and frames can adapt to a wide range of speeds and altitudes. One common small jet has similar versions in use on Venus, Mars, and Titan, has a single jet engine and has a life support system capable of providing air for up to 100 hours. The Venusian and Martian versions both have a top speed of 900 kph, a wingspan of 11 m, and a maximum range of 5,000 km. The version designed for Titan has a wingspan of 8 m, a top speed of 650 kph, and a range of 4,000 km. Jets are equipped with headlights, radio boosters, and radar with a range of up to 250 km. \textbf{[Expensive]} 

\begin{table} \begin{tabularx}{\hline}{|X|X|X|X|X|l|l|X|} \hline

\hline{8}{|c|}{\textbf{Vehicles --- Aircraft}} \\ \hline

&\textbf{Passenger capacity}	&\textbf{Handling}	&\textbf{Movement rate}	&\textbf{Max velocity}	&\textbf{Armor}	&\textbf{Durability}	&\textbf{Wound threshold} \\ \hline

Microlight	&1	&+20	&8/40	&100	&--- &30	&10 \\ \hline

Portable Plane	&2	&+10	&--- &200$-$250	&10/6	&50	&10 \\ \hline

Rocket Buggy	&4	&$-$10	&8/32	&2,500$-$3,000	&24/16	&100	&20 \\ \hline

Small Jet	&6	&+20	&--- &650-900	&30/20	&200	&30 \\ \hline

\label{tab:aircraft} \label{tab:aircraft} \end{table} 



\subsection{Exoskeletons} \label{sec:exoskeletons} 

Exoskeletons are powered mechatronic skeleton frameworks worn by a person. Servo-hydraulic joints allow the exoskeleton to be maneuvered by mimicking the wearer’s own movements, as well as enhancing their strength. Exoskeletons may also be piloted electronically. A character wearing an exoskeleton (other than the trike or transporter) maneuvers as normal, because the exoskeleton is like an extension of their own body. A character jamming an exoskeleton remotely uses Pilot: Walker skill (except for the trike and transporter). 

\textbf{Battle Suit:} The battle suit powered exoskeleton features a military-grade fullerene armor shell with flexible aerogel for thermal insulation and a diamond-hardened exterior designed to resist even potent ballistic and energy-based weapons. The suit also enhances the wearer’s strength and mobility, applying a +10 bonus to strength-based tests, inflicting an extra +1d10 damage and AP of $-$2 on melee attacks, and doubling the distance by which the character may jump. Battlesuits are completely sealed to protect the wearer from environmental factors, and fitted with life support features and a maker (p. 327) capable of recycling all wastes and producing air for up to 48 hours and food and water indefinitely. Battle suits are equipped with each an ecto (p. 325), a radio booster (p. 313), and sensors equal to specs (see p. 325). These suits have an Armor Value of 18/18 (not cumulative with any other armor) and protect the wearer from temperatures from $-$175 to 140$^{\circ}$ C. \textbf{[Expensive]} 

\textbf{Exowalker:} Exowalkers are minimal framework exoskeletons, primarily designed to bolster the wearer’s strength and movement. They provide a an Armor Value of 2/4, a +10 modifier to strength-based tests, and double the distance by which the character may jump. \textbf{[Moderate]} 

\textbf{Hyperdense Exoskeleton:} These powered exoskeletons are larger (roughly twice human-sized) and built for heavy-use industrial purposes, such as handling heavy/large objects. The wearer is partially encapsulated to protect them from debris and industrial accidents. Hyperdense exoskeletons provide no movement bonus, but provide a +30 bonus to strength-based tests and inflict an extra +3d10 damage and $-$5 AP on physical attacks. They have an Armor Value of 6/12. \textbf{[Expensive]} 

\textbf{Transporter:} This exoskeleton framework includes a pair of vector-thrust turbofan engines, giving the user flight capabilities in gravity and increased maneuverability in zero-G. It provides partial protection to the wearer with an Armor Value of 2/4. Piloted with Pilot: Aircraft skill. \textbf{[High]} 

\textbf{Trike:} The trike exoskeleton is a three-wheeled personal motorcycle design, rather than a walker. It provides partial protection to the wearer with an Armor Value of 2/4. Piloted with Pilot: Groundrcraft skill. \textbf{[Moderate]} 



\begin{table} \begin{tabularx}{\hline}{|X|X|X|X|X|X|X|X|} \hline

\hline{8}{|c|}{\textbf{Vehicles --- Exoskeletons}} \\ \hline

&\textbf{Passenger\newline capacity}	&\textbf{Handling}	&\textbf{Movement rate}	&\textbf{Max velocity}	&\textbf{Armor}	&\textbf{Durability}	&\textbf{Wound threshold} \\ \hline

Battlesuit	&1	&--- &8/32	&30	&18/18	&60	&12 \\ \hline

Exowalker	&1	&--- &8/40	&40	&2/4	&30	&6 \\ \hline

Hyperdense Exoskeleton	&1	&--- &8/20	&30	&6/12	&100	&20 \\ \hline

Transporter	&1	&+10	&8/40	&200	&2/4	&50	&10 \\ \hline

Trike	&1	&+10	&8/40	&120	&2/4	&50	&10 \\ \hline

\label{tab:exoskeletons} \label{tab:exoskeletons} \end{table} 









\subsection{Groundcraft} \label{sec:groundcraft} 

In Eclipse Phase, trains and bicycles remain the most common form of ground transportation, especially on habitats. In larger habitats and on moons and planets, cycles and cars are used as well. 

\textbf{Cycle:} Because of the high cost of enclosing a habitat and providing life support, space is at a premium in all cities except some of the newest cities on Mars. As a result, there is rarely room for large roads or the cars that once carpeted the roads of Earthly cities. Instead, the ubiquitous modern vehicle is the cycle, which is designed to drive down narrow streets only a little wider than sidewalks in Earth cities. 

There are many different varieties of cycle. Some have only a single wheel and are gyro-stabilized, but most have two wheels and resemble old Earth motorcycles. In some, the driver and passenger are enclosed by a streamlined pod. These vehicles are powered by superconducting batteries, have a range of 600 km and a top speed of 120 kph, but must usually drive more slowly in crowded streets. Cycles are all equipped with radio boosters, headlights, and a portable radar sensor. Tires are solid state (not inflated), or in some cases smart spokes capable of handling stairs. Some luxury versions have limited life-support in the small cabin, capable of providing air for the passengers for up to 10 hours. \textbf{[Moderate]} 

\textbf{Mars Buggy:} One of the most ubiquitous vehicles on Mars is the so-called Mars buggy, a four-wheeled vehicle with large balloon tires that is designed for use both on roads and on almost any terrain. Mars buggies can travel at speed of up to 110 kph on roads, 90 kph over relatively flat terrain, and up to 40 kph on jagged and rocky terrain. They can maintain these speeds because smart materials in both the suspension and the tires reshape themselves to adapt to uneven conditions and their nuclear batteries give them an effectively unlimited range. Most Mars buggies are enclosed but unpressurized. Similar vehicles are used on Luna and Titan, however, though the passenger compartments of these vehicles includes life support gear that provides the occupants with air for at least 100 hours. Buggies are powered by nuclear batteries and come in a variety of sizes, from small two-person buggies to large trucks. Mars buggies come equipped with headlights, radio boosters, and a vehicle radar system. \textbf{[High]} 

\subsection{Personal vehicles} \label{sec:personal-vehicles} 

These one-person movement aids primarily are used in space, but do not count as spacecraft per se. 

\textbf{EVA Sled:} This small sled uses air impellers to maneuver in zero-G. It is commonly used to carry attached gear, but may also pull along 1 human-sized morph. \textbf{[Low]} 

\textbf{Rocket Pack:} This is a miniature metallic hydrogen rocket that the wearer straps to their back, with two rocket exhausts extending out to either side, away from the wearer’s body or legs. Biomorphs and pod morphs can only safely use this vehicle when wearing a vacuum suit or some garment that is similarly heat resistant. Also, to prevent harm to the wearer, the thrust must be kept sufficiently low that it can only take off on Mars or moons with even lower gravity. A rocket pack can keep the wearer airborne for up to 15 minutes in Mars gravity, or 30 minutes on Luna, Titan, or any of the four large Jovian moons. On Mars, it has a maximum speed of 700 kph. It can be used to reach orbit and land again on Luna, Titan, and other similarly small bodies like the Jovian moons. Rocket packs are equipped with radio boosters but no other sensors or communication devices. \textbf{[Low]} 

\textbf{Thruster Pack:} Worn for EVA duties, this thruster pack uses vectored thrust nozzles, allowing a character to maneuver in open space. This is not a jetpack and does not produce enough thrust for atmospheric movement. \textbf{[Low]} 

\begin{table} \begin{tabularx}{\hline}{|l|X|X|X|X|X|X|X|} \hline

\hline{8}{|c|}{\textbf{Vehicles --- Groundcraft, personal vehicles}} \\ \hline

&\textbf{Passenger capacity}	&\textbf{Handling}	&\textbf{Movement rate}	&\textbf{Max velocity}	&\textbf{Armor}	&\textbf{Durability}	&\textbf{Wound threshold} \\ \hline

\hline{8}{|l|}{\emph{Groundcraft}} \\ \hline

Cycle	&1$-$3	&+20	&4/40	&120	&12/10	&50	&10 \\ \hline

Mars Buggy	&2$-$6	&+10	&8/32	&40/90/110	&30/20	&150	&30 \\ \hline

\hline{8}{|l|}{\emph{Personal vehicles}} \\ \hline

EVA Sled	&1	&$-$30	&4/16	&16	&5 40	&&8 \\ \hline

Rocket Pack	&1	&$-$20	&--- &700	&+5/+5	&40	&8 \\ \hline

Thruster Pack	&1	&$-$10	&4/20	&40	&+4/+4	&30	&6 \\ \hline

\label{tab:groundcraft-personal} \label{tab:groundcraft-personal} \end{table} 



\subsection{Robots} \label{sec:robots} 

Robots are a common sight and accepted fact of daily life within Eclipse Phase. Numerous varieties exist, from robo-pets to mechanical workers to warbots. If a job can be done more cheaply (and sometimes safely) by a bot, it usually is. The robots listed here are not generally used as synthetic shells by transhuman egos, often for cultural reasons (sleeving a case is bad enough, sleeving a creepy is just ... wrong), and they are not equipped to be sleeved into (though the may be jammed; see p. 196). Any of these bots may be modified for use as a synthetic morph, however, by adding a cyberbrain system (p. 300). 

\textbf{Automech:} Automechs are general purpose repair drones, found just about everywhere. Each particular automech tends to specialize in a particular type of repair work and so carries the appropriate tools and AI skills, whether it be habitat waste recyclers, outer hull integrity, or servitor systems. Standard automechs are wheeled cubes with articulated limbs, though they are also equipped with vectored-thrust drives for zero-G work. \textbf{[Moderate]} 

\textbf{Creepy:} Creepies are small crawler bots that come in an eclectic variety of shapes and forms, from robosquirrels to insectoids to bizarre and artsy mechanical creatures. Creepies were originally designed as a sort of robotic pet, but they are commonly used as a general purpose household minion, like a more beloved servitor. Many people in fact wear a creepy on their person, dropping it to handle small tasks for them and letting it crawl up and down and over their body. \textbf{[Low]} 

\textbf{Dr. Bot:} These wheeled medical robots are designed to tend to and transport injured or sick people. They carry a healing vat (p. 326), a specialized pharmaceuticals maker, miscellaneous medical gear, and articulated arms for conducting remote surgery. \textbf{[High]} 

\textbf{Dwarf:} These large industrial bots are named not just for their primary use --- mining, excavation, tunneling, and construction --- but because the default AIs they shipped with had a programmed tendency to happily whistle as they worked. Dwarfs are quadrapedal walkers, equipped with massive modular industrial tools like boring drills, shovels, hydraulic jacks, jackhammers, scooping arms, acid sprays, and so on. \textbf{[Expensive]} 

\textbf{Gnat:} Gnats are small rotorcraft camera/surveillance drones. Many people use gnats for personal lifelogging, while socialites and media use them to capture the glamor or hottest news. \textbf{[Low]} 

\textbf{Guardian Angel:} Similar to gnats, guardian angel rotorcraft hover around their charges, keeping a watchful eye out to protect them from threats. \textbf{[Moderate]} 

\textbf{Saucer:} These disc-shaped drones are lightweight and quiet. They are typically launched by throwing them like a frisbee, after which they propel themselves with an ionic drive (p. 310). Saucers make excellent ``eye in the sky'' monitors and scouts. \textbf{[Low]} 

\textbf{Servitor:} Servitors are the most common robot, acting as cooks, janitors, universal helpers, movers, and personal aides. Every home has one, if not several. Servitors are intentionally built in non-humanoid forms so as not to confuse them with common synthmorphs and in order to defuse bad feelings at ordering them around. However, they all have some form of ``face'' to interact with, so as not to be too machine-like. \textbf{[Low]} 

\textbf{Speck:} Specks are tiny insectoid spy drones, 2.5 mm long and 2 mm wide, about the size of a small fruit fly. They fly with tiny wings, carry a microbug, and are excellent for surveillance purposes or otherwise being a ``speck on a wall.'' Specks are difficult to notice ($-$30 Perception modifier) and almost impossible to distinguish from an actual insect. \textbf{[Low]} 

\begin{table} \begin{tabular}{|l|l|l|l|l|l|l|} \hline

\hline{7}{|c|}{\textbf{Vehicles --- Robots}} \\ \hline

&\textbf{Movement}	&\textbf{Max}	&\textbf{Armor}	&\textbf{Durability}	&\textbf{Wound}	&\textbf{Mobility} \\ &\textbf{Rate}	&\textbf{Velocity}	&&&\textbf{threshold}	&\textbf{system} \\ \hline

\textbf{Automech}	&4/8	&8	&4/4	&30	&6	&Wheeled/Vector-Thrust \\ \hline

\hline{7}{|l|}{Enhancements: Access Jacks, Electrical Sense, Extra Limbs (4), Headlights, Magnetic System, } \\ \multicolumn{7}{|l|}{Radiation sense, Utilitool, misc. tools} \\ \hline

\textbf{Creepy}	&4/12	&12	&2/2	&25	&5	&Walker or Hopper \\ \hline

\hline{7}{|l|}{Enhancements: +5 COO, Access Jacks, Chameleon Skin, Extra Limbs (2--8), Grip Pads} \\ \hline

\textbf{Dr. Bot}	&4/16	&16	&--- &40	&8	&Wheeled \\ \hline

\hline{7}{|l|}{Enhancements: Access Jacks, Enhanced Smell, Fabber, Fractal Digits, Healing Vat, Nanoscopic Vision} \\ \hline

\textbf{Dwarf}	&4/12	&20	&16/12	&150	&30	&Walker \\ \hline

\hline{7}{|l|}{Enhancements: +10 SOM, Access Jacks, Extra Limbs (4), Industrial Armor, Radar, Sonar, misc. tools} \\ \hline

\textbf{Gnat}	&8/40	&60	&2/2	&25	&5	&Rotor \\ \hline

\hline{7}{|l|}{Enhancements: 360-Degree Vision, Access Jacks, Enhanced Hearing, Enhanced Vision, Radar} \\ \hline

\textbf{Guardian Angel}	&8/40	&80	&14/12	&40	&8	&Rotor \\ \hline

\hline{7}{|l|}{Enhancements: +5 REF, 360-Degree Vision, Access Jacks, Chameleon Skin, Eelware, Enhanced Hearing, } \\ \multicolumn{7}{|l|}{Enhanced Smell, Enhanced Vision, Lidar, Light Combat Armor, Neurachem, T-Ray Emitter} \\ \hline

\textbf{Saucer}	&8/40	&200	&2/2	&25	&5	&Ionic \\ \hline

\hline{7}{|l|}{Enhancements: 360-Degree Vision, Access Jacks, Chameleon Skin, } \\ \multicolumn{7}{|l|}{Enhanced Hearing, Enhanced Vision, Radar} \\ \hline

\textbf{Servitor}	&4/20	&20	&4/4	&30	&6	&Walker or Wheeled \\ \hline

\hline{7}{|l|}{Enhancements: Access Jacks, Extra Limbs (2-6)} \\ \hline

\textbf{Speck}	&1/5	&5	&--- &5	&1	&Winged/Hopper \\ \hline

\hline{7}{|l|}{Enhancements: +5 REF, +5 COO, $-$10 SOM, Access Jacks, Grip Pads, } \\ \multicolumn{7}{|l|}{Enhanced Hearing, Enhanced Vision, Synthetic Mask} \\ \hline

\label{tab:robots} \label{tab:robots} \end{table} 

\subsection{Spacecraft} \label{sec:spacecraft} 

Though egocasting is a common method of personal transport and it’s often easier to simply transmit the specifications for various goods and to allow nanofactories to create duplicates, spacecraft play an important role in the solar system, carrying both passengers and valuable cargo. Both in terms of materials and propulsion, spacecraft in the post-Fall era are far superior to the primitive vessels used in the 20th and early 21st centuries, but they are still based on the same principles. 

Spacecraft have few stats in \emph{Eclipse Phase}, as they are primarily handled as setting rather than vehicles. Note also that no stats are given for spacecraft weaponry. It is highly recommended that space combat be handled as a plot device rather than a combat scene, given the extreme lethality and danger involved. If you absolutely must know the DV of a spacecraft weapon, treat it as a a standard weapon with a DV multiplier of x3 for small craft (fighters and shuttles), x5 for medium craft, and x10 for larger craft. 

\subsubsection{Spacecraft propulsion} 

The most important part of any spacecraft is its engine, and the most important features of any engine are the exhaust velocity, which determines how much fuel the rocket requires to reach a given speed, and the engine’s thrust, which determines how high the acceleration can be. Any rocket that has a thrust of less than approximately twice the gravity of a planet or moon cannot take off from that planet or moon. Sample thrusts and gravities are listed on the \emph{Escaping Gravity Wells} table, p. 346. 

\textbf{Hydrogen-Oxygen Rocket (HO):} Though optimized with improved engine design and light-weight materials, these are essentially the same primitive rockets that humanity used to first reach the moon in the 20th century. These are rarely used and only common with groups too poor or primitive to safely manufacture metallic hydrogen. 

\textbf{Metallic Hydrogen Rocket (MH):} Metallic hydrogen is a solid form of hydrogen created using exceedingly high pressures. Although naturally unstable, it can be stabilized with carefully controlled electrical and magnetic fields, and these field generators are an integral part of every metallic hydrogen fuel tank. By selectively reducing these fields near the exhaust nozzle, small amounts of metallic hydrogen can be made to swiftly and explosively revert to conventional hydrogen gas, propelling the rocket with great force in an easily controlled fashion. Metallic hydrogen engines are used in most planetary landers and short range vehicles. 

\textbf{Plasma Rocket (P):} This drive heats hydrogen into plasma and accelerates it using a powerful electrical field. This type of rocket was very common in the mid 21st century, but has been superseded by fusion rockets and is only used in older and more primitive spacecraft, notably scum barges. 

\textbf{Fusion Rocket (F):} Similar to a plasma rockets, fusion rockets require significantly higher temperatures and pressures, and the rocket also produces large amounts of power for the spacecraft. Fusion rockets are now the most common form of propulsion for spacecraft designed for long-distance voyages. 

\textbf{Anti-Matter Rocket (AM):} Anti-matter rockets work mixing small amounts of anti-matter into the hydrogen fuel, producing enormous amounts of energy and an exceptionally fast and powerful exhaust. These rockets typically carry a heavily shielded magnetically contained anti-matter storage vessel carrying a mass of anti-matter equal to 1\% of the mass of the hydrogen fuel used by the rocket. The magnetic containment vessels needed to safely contain antimatter usually weight at least 10 times the mass of the antimatter used. 

Though anti-matter storage is exceptionally safe, the vast energy release possible if there was an accident means that anti-matter rockets are forbidden from coming closer than 25,000 km from any inhabited planet or moon. Also, very few habitats will allow an anti-matter rocket to dock with them, and instead require the spacecraft to remain at least 10,000 km away and for all cargo and passengers to be transferred using a small craft like a small LOTV. Anti-matter is exceedingly expensive to produce and so anti-matter rockets are only used in military vessels and in fast couriers designed to carry critical cargoes across the solar system in short periods of time. 

\begin{table} \begin{tabular}{|l|r} \hline

\hline{2}{|c|}{\textbf{Escaping gravity wells}} \\ \hline

\textbf{Spacecraft engine}	&\textbf{Thrust (in GS)} \\ \hline

Hydrogen-Oxygen Rocket	&4+ \\ \hline

Metallic Hydrogen	&3 \\ \hline

Plasma Rocket	&0.01 \\ \hline

Fusion Rocket	&0.05 \\ \hline

Anti-Matter	&0.2 \\ \hline

Rocket Buggy	&0.5 \\ \hline

\textbf{Planets, moons etc.}	&\textbf{Gravity} \\ \hline

Earth	&1 \\ \hline

Europa	&0.13 \\ \hline

Jupiter	&2.53 \\ \hline

Luna	&0.17 \\ \hline

Mars	&0.38 \\ \hline

Mercury	&0.38 \\ \hline

Neptune	&1.14 \\ \hline

Pluto	&0.06 \\ \hline

Saturn	&0.91 \\ \hline

Titan	&0.14 \\ \hline

Uranus	&0.89 \\ \hline

Venus	&0.9 \\ \hline

\label{tab:excaping-gravity} \label{tab:excaping-gravity} \end{table} 

\subsubsection{Sample spacecraft} 

The following is a representative sample of the most common type of spacecraft used in the solar system today. 

\textbf{Bulk Carrier:} This vessel is simply a standard transport refitted to carry large amounts of cargo in external cargo grapples. Used for carrying refined ores, ice, and similar forms of large, useful, but low priority cargos, bulk carriers transport large cargos at relatively low velocities. They also offer an inexpensive, reliable, and slow method for passengers to travel from one habitat to another and are not infrequently used by individuals who wish to disappear for a while. Unlike the standard transport, the bulk carrier lacks the rotating habitat pods. 

\textbf{Courier:} In a standard transport, a typical journey from Luna to Mars requires approximately three weeks, while a journey from Mars to Jupiter requires approximately four months. This is sufficient for most purposes, but occasionally characters need to take themselves or sufficiently valuable cargoes across the solar system in a matter of days or weeks, instead of weeks or months. 

Anti-matter drive fast couriers are vessels designed for this specific purpose. This vessel can travel from Venus to Mars in a week and from Mars to Jupiter in a month. The fast courier is the swiftest vessels currently made and is able to reach at much as one half of one percent of the speed of light. To manage this, this spacecraft must also carry 6 tons of antimatter in a 100 ton magnetic containment vessel. In an emergency, this containment facility can be jettisoned. 

\textbf{Destroyer:} One of the largest military spacecraft in common use, destroyers use an antimatter drive holding 150 tons of antimatter in a 2,000-ton magnetic containment vessel. This antimatter can also be used to provide the spacecraft’s missiles with anti-matter for devastatingly powerful anti-matter warheads. This spacecraft is also armed with railguns, nuclear and high explosive missiles, and point defense lasers. In addition, all destroyers carry a contingent of 20 fighters. 

\textbf{Fighter:} This small, short range military vessel is designed to be crewed by an infomorph or AI. If needed, however, it can hold a single synthmorph or vaccumadapted biomorph as a pilot. It carries 3 lasers and 2 railguns mounted on small pods placed around the middle of ship that can fire in any direction. A single missile launcher is located in the nose of the fighter and typically holds 6 small high explosive missiles or tactical nuclear missiles (or even anti-matter missiles if facing high-threat targets). 

\textbf{General Exploration Vehicle (GEV):} A GEV is one of the standard vehicles used for exploration beyond the Pandora Gates. It is specifically designed to handle almost any environment. It is a boxy vehicle, 6 meters long, 2.2 meters wide, and 2 meters high. It makes extensive use of smart matter in the lower part of the chassis, and can create wheels or short legs (primarily useful for exceedingly rough terrain). It can even produce limited hull streamlining and propulsion suitable for travel both on and underwater. In addition, it contains a small metallic hydrogen engine that allow it to maneuver in space with an acceleration of up to 0.1 G. GEVs have a Maximum Velocity of 200 (wheeled)/40 (walker)/60 (sea)/40 (submerged). 

The GEV also has a closed cycle life support system that can support up to 6 (fairly cramped) living occupants for up to one month and limited electromagnetic shielding against charged particle radiation. All models are fitted with advanced AI piloting and navigation as well as limited self-repair capacity. In addition, GEV’s have an extensible airlock, a single healing vat, several desktop CMs, and a variety of sensors, including both radar and telescopic full spectrum cameras. 

\textbf{Large Lander and Orbit Transfer Vehicle (LLOTV):} This common vehicle is used for transporting passengers and cargo between a planet or moon and orbit and for short distance transfers between habitats less than 100,000 km apart. This conical vehicle has a curved heat shield on the base and smart material landing legs and grapples so that it can rest securely on any stable terrain and link up with all forms of docking clamps. It comes in variants designed to use either a hydrogen-oxygen chemical rocket or a metallic hydrogen rocket. The use of light-weight smart materials allows the interior to be easily and rapidly reconfigured to accommodate different amounts of fuel, passenger seats, and cargo space. LLOTVs that are not designed for planetary landing or which are designed only to land on airless moons are unstreamlined and look considerably blockier. 

LLOTVs come in two configurations: high or low velocity. High velocity configuration allows the vehicle to land and take off again on Venus or Earth without refueling and for rapid transport between nearby habitats. Low velocity configuration is designed to land and take off again on Mars or various large moons without refueling and for slower and more fuel efficient transport between nearby habitats. The extensive use of smart materials in this vehicle means that LLOTVs that use metallic hydrogen engines can be easily converted between the high and low velocity configurations, requiring less than a day in a wellequipped maintenance facility. However, vessels using hydrogen oxygen engines cannot be converted. Since metallic hydrogen is a much more efficient propellant, landers using it always include significant amounts of extra propellant for emergencies. 

\textbf{Scum Barge:} These huge craft were originally designed for use during the first stages of the evacuation of Earth. They were built to carry up to 20,000 people and to allow them to survive for months or even years, in relatively cramped conditions, until more suitable habitats could be constructed. A number of these vessels are still in service, primarily used as mobile habitats by various anarchic subcultures. The best have had their plasma rockets replaced by modern fusion rockets and carry 5-10,000 in relative comfort. The worst use aging plasma rockets and stretch their life support systems and living spaces to the limit, carrying up to 25,000 poor and desperate residents. 

\textbf{Small Lander and Orbit Transfer Vehicle (SLOTV):} This vehicle is identical in use and design to the LLOTV, except that it is one third the total mass and correspondingly less expensive to build and refuel. Some exceptionally wealthy individuals own private small LOTVs. Using a small LOTV with a hydrogenoxygen engine to take off and land on Venus or for other high velocity uses is exceptionally cramped and allows for absolutely no room for error. Like the LLOTV, this vehicle can be easily converted between low and high velocity configurations and is made in both streamlined and non-streamlined versions. 

\textbf{Standard Transport:} This vessel is one of the most common freighter and passenger vessel in the solar system. While egocasting is by far the most common form of inter-habitat transport, some people prefer to travel by ship and others do not wish to leave their current morph behind. In addition, some goods are easier or cheaper to physically transport rather than duplicating their templates. As a result, standard transports regularly travel to and from every large habitat and inhabited planet and moon in the solar system. These are modern fusion-drive ships that offer fast and comfortable travel for passengers as well as relatively swift transport for small cargoes. 

One of the additional benefits of the standard transport is the fact that it contains four separate passenger compartments, each of which is mounted on a 90 meter-long booms that can extend and rotate to simulate gravity. When rotating at a comfortable 2 rpm, passengers experience Mars level gravity. Typically, the gravity maintained in these pods starts at the local gravity (or Mars gravity, if the local gravity is higher) and over the course of the journey gradually increases or decreases to the gravity of the destination. However, these pods cannot rotate to produce gravity higher than that found on Mars. 

\begin{table} \begin{tabularx}{\hline}{|l|X|l|X|l|l|} \hline

\hline{6}{|c|}{\textbf{Vehicles --- Spacecraft}} \\ \hline

&\textbf{Passenger capacity}	&\textbf{Handling}	&\textbf{Armor}	&\textbf{Durability}	&\textbf{Wound threshold}\\ \hline

Bulk Carrier	&110	&--- &20	&750	&150 \\ \hline

Courier	&13	&--- &15	&500	&100 \\ \hline

Destroyer	&90	&--- &30	&2,000	&500 \\ \hline

Fighter	&1	&+30	&20	&240	&60 \\ \hline

GEV	&6	&$-$10	&15	&200	&40 \\ \hline

LLOTV (HO)	&20	(high-velocity)/100 (low-velocity)	&$-$10	&20	&800	&160 \\ \hline

LLOTV (MH)	&250 (high-velocity)/350 (low-velocity)	&$-$10	&20	&800	&160 \\ \hline

Scum Barge	&20,000	&--- &20	&1,500	&150 \\ \hline

SLOTV (HO)	&3 (high-velocity)/30 (low-velocity)	&$-$10	&20	&400	&80 \\ \hline

SLOTV (MH)	&70	(high-velocity)/100 (low-velocity)	&$-$10	&20	&400	&80 \\ \hline

Standard Transport	&200	&--- &20	&750	&150 \\ \hline

\label{tab:spacecraft} \label{tab:spacecraft} \end{table} 



