



\chapter{Équipement} \label{cha:gear} 

Les niveaux de technologie de \emph{Eclipse Phase} permettent d'utiliser de nombreux appareils pour l'amélioration personnelle, la survie ou d'autres choses. 



\section{Règles d'équipement} \label{sec:equipment-rules} 

Les règles suivantes s'appliquent à tous les objets technologiques dans \emph{Eclipse Phase}. 



\subsection{Acquérir du matériel} \label{sec:acquiring-gear} 

Pendant la création de personnage, les joueurs achètent le matériel de leur personnage en utilisant les crédits acquis pendant le processuss de la création de personnage. Cependant, une fois que le jeu commence, les personnages doivent obtenir tout l'équipement dont ils ont besoinb en utilisant les méthodes classiques: acheter, emprunter, fabriquer ou voler. 

Dans le système intérieur, dans les abris hypercorporatiste et de la République Jovienne ... et dans les autres endroits ou règne le capitalisme ... il s'agît simplement de trouver un vendeur et de lui acheter le matériel recherché. Chaque objet à ubn prix, de Trivial à Cher, comme noté sur la tablede Prix du Matériel. En raison de la disponibilité des ressources locales, de l'offre et de la demande ainsi que des lois, ces prix sont approximatifs.  Lorsqu'aucun autre facteur ne s'applique, le Prix Moyen indiqué pour la catégorie peut-être utilisé. Le maître de jeu devrait modifier la valeur de l'objet de manière appropriée à la situation, en prenant en compte les facteurs économiques locaux, tout en gardant le prix dans les limites de la catégorie de l'objet. La table des Modificateurs de Prix liste suggère quelqueséléments pouvant changer le prix d'unobjet, mais il s'agît de simple recommandations et ils peuvent êtreignorés ou suivi selon la volonté du maître de jeu. Les conditions locales exactes sont entièrement laissées à l'appréciation du maître de jeu. 

Dans certaines circonstances, les personnages peuvent essayer de marchander le prix du matériel. Cela est généralement géré par l'interprétation, mais lemaître de jeu peut aussi demander un Test en Opposition de Persuasion (ou éventuellement un Test d'Intimidation). Le personnage qui l'emporte peut faire varier le prix de 10\% par tranche de 10 points de MdR. 

Dans le système extérieur, les habitats anarchistes, Titaniens, de la racaille et les autres endroits qui utilisent une économie réputationnelle, les personnages doivent se reposer sur leur scores de rep pour acquérir les biens et services dont ils ont besoin. Le fonctionnement de ce système est décrit auchapitre \emph{Réputation et Réseaux Sociaux }, p. 285. 

Bien entendu, les personnages sont libre d'utiliser d'autres moyens pour mettre la main sur du matériel tel que les arnaques, l'emprunt à un ami et le vol, aec toutes les conséquences et tests appropriés. Dans certains casn acquérir du matériel peut être une aventureen soi. 

\begin{table} \begin{tabular}{|l|l|l|} \hline

\multicolumn{3}{|c|}{\textbf{Coût de l'équipement}}	\\ \hline

\textbf{Catégorie}	&\textbf{Gamme (en crédits)}	&\textbf{Moyenne (en crédits)} \\ \hline

Trivial	&1-99	&50 \\ \hline

Bas	&100-499	&250 \\ \hline

Modéré	&500-1499	&1000 \\ \hline

Élevé	&1500-9999	&5000 \\ \hline

Cher	&10000+	&20000 \\ \hline

\end{tabular} \label{tab:gear-costs} \end{table} 

\begin{table} \begin{tabular}{|l|l|} \hline

\multicolumn{2}{|c|}{\textbf{Modificateur du prix de l'équipement}}	\\ \hline

\textbf{Facteur Économique}	&\textbf{Modificateur de prix suggéré} \\ \hline

Objet Volé	&-50\% \\ \hline

Objet d'occasion	&-25\% \\ \hline

Objet Réglementé	&+25\% \\ \hline

Objet Illégal	&+50\% \\ \hline

Objet Rare	&+25\% \\ \hline

Objet Extrèmement Rare	&+50\% \\ \hline

Objet Commun	&-25\% \\ \hline

\end{tabular} \label{tab:gear-cost-modifiers} \end{table} 

\subsubsection{Fabriquer l'équipement} 

Grâce à la technologie de nanofabrication, les personnages peuvent également créer leur propre équipement en utilisant des machines d'abondances et autres appareils de nanofabrication similaire (p. 327). Le personnage doit avoir le plan approprié, qu'il vienne du faber, qu'il soit acheté légalement ou sur le marché noir. Les personnages peuvent aussi coder les plans qu'ils désirent en utilisant la compétence Programmer: Nanofabrication. 



\subsection{Modificateurs d'équipement} \label{sec:gear-modifiers} 

Dans le futur technologique, l'équipement est unenécessité. Dans de nombreux cas, l'utilisation d'équipement ne fournitaucunbonus, il permet simplement aupersonnage d'effectuer une tâche qu'il n'aurait pas pu accomplir sinon. Par exemple, il est impossible de crocheter une serrure mécanique sans un crocheteur ou un minimum d'outil. 

Dans d'autres situation, l'équipement fournit un bonus à la tâche en cours. Escalader un mur est possible sans outils, mais si vous avez des gants adhésifs ou du matériel d'escalade, cela sera bien plus facile. Le modificateur spécifiquement appliqué est généralement noté dans la description du matériel, et est généralement compris entre +10 et +30. 

\subsubsection{Qualité de l'équipement} 

Dans les deux situations décrites ci-dessus, il est possible d'avoir des objets qui son de qualité exceptionnelle ou de qualité inférieure, ayant des modificateurs correspondant. Le matériel peut-être extr€mementbien fabriqué, à la pointe de la technologie, un prototype expérimental ou simplement haut-de-gamme, fournissant un modificateur supplémentaire de +10 à +30. Inversement, l'objet peut-être dépassé, vieilot ou non-entretenus, infligeant un modificateur de -10 à -30 (et annulant le bonus de base dans certains cas). 

\subsubsection{Taille du matériel} 

À l'occasion, vous pourriez avoir besoin de connaître la taille d'un élément d'équipement particulier. Bien que ce soit quelquechose que le maître de jeu peut généralementimproviser en utilisant le senscommun, nous avons inclus la taille de nombreux éléments qui sont inhabituels ou tellement futuristes que le joueur moyen aurait du mal à apréhender. Les catégories de tailles sontlistées dans la table de Taille de l'Équipement (p. 297). Ces tailles sont des approximations, et doivent être ajustée en fonction du fabricant et de la façon de les fabriquer, certains objets pouvant être plsugros ou plus petit que les objets similaires. Il est également important de garder en tête que au fur et àmeusre que la technologie progresse, la taille et les composants de difféents objets réduisent, donc lorsque vous hésitez sural taille d'unobjet,choisissez la plus petite. 



\begin{table} \begin{tabularx}{\textwidth}{|l|X|} \hline

\multicolumn{2}{|c|}{\textbf{Tailles d'équipement}}	\\ \hline

\textbf{Catégorie de taille}	&\textbf{Dimensions générales et remarques} \\ Nano	&Tellement petit que l'objet ne peut pas être vu sans l'aide d'un microscope ou de vision nanoscopique (p. 311), et qu'il ne peut être manipulé sans doigts fractals (p. 311) ou autre outil similaire. \\ \hline

Micro	&Tout ce qui va de la taille d'un point à peine visible à un insecte moyen. \\ \hline

Mini	&Les objets mini peuvent être dissimulés dans la main de quelqu'un ou dans une petite poche. \\ \hline

Petit	&Les petits objets peuvent être tenus dans une main, et cachés dans les poches standard.\\ \hline

Moyenne	&Les objets de taille moyenne sont difficile à tenir à unemain, allant de la taille d'une bouteille de 2 litres à celle d'un chien de taille moyenne. Ils ne tiennent pas dans lespoches, mais ils peuvent être dissimulés par des couverts plus gros. \\ \hline

Grand	&De la taille d'un humain. \\ \hline

Gros	&Les véhicules et tous les autres objets massifs. \\ \hline

\end{tabularx} \label{tab:gear-sizes} \end{table} 



\subsubsection{Masse et encombrement} 

Un personnage quitransporte trop d'équipement devrait être ralentis, subissant des modificateurs négatifs à leur allure de déplacement et à leurs tests decompétences. Plutôt que de microgérer le poid de chaque pièce d'équipement, cela reste essentiellement àla discrétion du maître de jeu qui doit utiliser son sens commun. Si un personnage s'équipe au-delà de la raison, appliquez les modificateurs qui semblent appropriés. Le maître de jeu devrait cependant garder en tête que la plupart des matériau de fabrication utilisé dans \emph{Eclipse Phase} rendant les objets bienplus léger que les standards actuel et sans perte de solidité ou de fonctionnamité (voir \emph{Matériau du Futur}, p. 298). De la même manière, les personnages dans des environnements en faible ou en micro-gravité peuvent porter bien plus de charge. 

\subsubsection{Dissimuler du matériel} 

Les personnages peuvent essayer de dissimuler desobjets sur eux,spérant au moins les soustraire à l'attention des passants voire d'une fouille intense. Pour déterminer l'efficacité avec laquelle le personnage dissimule l'équipement, faites un Test de Manipulation et notez la MdR (le maître de jeu peut vouloir faire ce jet en secret). Lorsqu'un autre personnage a une chance de remarquer l'objet dissimulé, ils doivent réussir un Test de Perception et obtenir une MdR suppérieure à celle du Test de Manipulation. Le maître de jeu devrait appliquer des modificateurs aux deux tests de manière appropriée. Par exemple, dissimuler un gros objet tel qu'une épée sera difficile (-30), alors que porter des vêtements facilitant la dissimulation comme un manteau long ou une combincaison de saut avec beaucoup de poche aidera (+20). De manière identique, un personnage qui ne cherche pas activement  aura moins de chance de remarquer du matériel caché (-30) alors que quelqu'un faisant unefouille physique (+30) ou ayant une vision améliorée permettant de percer les couches de protection s'en sortira bien mieux. 



\subsection{Conception et mode} \label{sec:design-fashion} 

De nombreux objets de \emph{Eclipse Phase} ressemble beaucoup à leurs équivalent du début du 21° siècle ... une bouteille de soda est toujoursun container transparent renfermant un liquide coloré, les habits sont bien entendus des choses que vous portez,et uncouteau est toujourscomposé d'une lame et d'une poignée. Les matériaux, procédés et état d'esprits qui entrent en compte dans leurfabrication sont cependant relatiment différents. Pour commencer, trés peu d'objet ont une apparence uniforme et de production de masse, même si ils ont été effectivement été produits en quantité. Les opérations de miniffacture et de nanofabrication permettent à chaqueobjet d'être fabriqué avec une apparenceunique (ou au moins sensiblement différente). Dans les zones anarchistes/d'économie réputationnelles, où les possessions personnellesont trés peu de valeur intrinsèque, l'expression et la créativité sont encouragés, et énormément d'objet sont personnalisés artistiquement (et les objets fabriqués à la main sont rare et précieux). De la même façon, presque tous les équipements sont conçus en priorisant la simplicité d'utilisataion et l'ergonomie, le matériel avec des courbes douces, des couleurs plaisantes et des formes ergonomiques est fréquent. Beaucoup d'objet personnels technologique, tels que les lampes torche ou les petitsoutils, ont une forme ovoïde qui tiens confortablement dans la main de l'utilisateur  ou une forme similaire permettant de l'attacher facilement aux habits. Pour quelqu'un du 20° siècle, de nombreux appareils courant ressemblent à d'étrange rochers colorés ou à des éléments de décorations en plastique u en céramique (en fait, de tels objets sont appelés "blobjets" par les plus vieux transhumains). 

Les matériaux utilisés pour créer les objets du quotidiens sont également évolués, allant de l'aérogel et du graphène aux matériaux intelligents (p. 298) en passant par les métamatériaux exotiques aux propriétés physique inhabituelles. En pratique, cela signifie que la plupart des objets sont légers, solide (avec à la fois une résistance à la traction et/ou de la flexibilité en fonction des besoins), résistants à l'eau, résistants à la poussière et auto-nettoyant. La plupart du matériel est également conçus pour fonctionner en microgravité ou en zéro-G, et ils peuvent facilement clipsés, attachés ou collés à une surface ayant des coussinets adhésifs. 

Pratiquement tout le matériel disponible dans \emph{Eclipse Phase} est également disponible dans des formes utilisables parles animaux élevés et les morphs non-humanoïde telle que les novacrabes, les slithéroïdes et les autres. Même si ce type de matériel adapté n'est pas immédiatement disponible, il n'est pas dur à nanofabriquer. Les matériau intelligents (p. 298) facilitent également l'interopérabilité entre différentes morphs. 

\subsubsection{Interface} 

Il n'est pas rare que les appareils du quotidien n'aient pas de contrôles visible car ils sont conçus pourêtre opréré par des émissions radios en provenance de l'ecto ou de l'insert de mesh de l'utilisateur. Tout objet fabriqué pour être utilisé dans des conditions d'urgence, de combat, de survie ou d'explorations auront généralement des contrôles de base, juste au cas où. Les interfaces physiquesq sont typiquement contrôllées par des zones tactiles qui ne sont rien de plus que des zones de couleurs à la surface del'appareil, bien que certains peuvent également projeter des affichages holographique. La plupart des équipements de ce type peuvent également être activés et contrôlés par la voix. 

Quasiment tous les appareils sont fournis avec une documentation complète et des guides d'utilisation. La plupart des objets électroniques sont également capable de se connecter au mesh et sont équipé d'IA spécialisées (voir \emph{Équipement Meshé}, page suivante). 

\subsubsection{Matériaux intelligents} 

Beaucoup d'objets technologiques courants sont composé de matériaux dits intelligents. Ces appareils contiennent - ou sont parfois entièrement composés de - nombreuses nanomachines qui peuvent à la fois se déplacer et se réorganiser afin demodifier la forme, la couleur et la texture du matériau. Par exemple, des vêtements intelligents peuvent transformer une combinaison adaptée au froid polaire des pôles Martiens à une tenue à la dernière mode en raison des centaines de milliers de petites nanomachines dans les habits qui permuttent et se déplacent pour réorganiser la garde robe. De la même manière, un outil fait de matériau intelligent peut basculer d'un tournevis de puissance en une pince ou un marteau, grâce aux mouvements des nanomachines qui restructurent complètement l'outil. Les matériaux intelligents contiennent tous des générateurs de nanomachines spécialisées et avancées (p. 328) qui les maintiennent en parfait étât tant qu'ils sont régulièrement rechargés. 



\subsection{Matériau futuristes} \label{sec:future-materials} 

Beaucoup de matériau disponible et couramment utilisés dans \emph{Eclipse Phae} sont encore rare, théorique ou inconnus de nos jours. Les entrés suivantes notent ceux qui sont les plus intéressants. 

\subsubsection{Aérogel} 

La "fumée gelée" solide et à faibledensité est faites en faisant pracuationneusement mousser différents matériaux, générallement du verre ou des céramiques, à un état de densité extrêmement faible. L'aérogle est semi-transparent et extrêmement légé, unpeu comme le polystyrène expansé, mais est un excellent isolant thermique. Il est fréquemment utilisé dans les habitats. 

\subsubsection{Diamant} 

Le diamant artificiel estléger et extrêmemnt solide, a un point de fusion très élevé et à une conductivité thermique presque parfaite. Cela en fait la substance idéal pour renforcer les surfaces traités (armure)et créer une machinerie en diamant extrêment solide. 

\subsubsection{Fullerènes/Fullerites} 

Les fullerènes sont des structures d'atomes de carbone (appelées buckyballs, nanotubes de carbones et graphènes) qui sont extrêmement solides (bien plus solide à poid égal que l'acier), thermo-résistante et qui peuvent être soit isolantes soit supraconductrices. Ces propriétés les rendent extrêmement utiles dans des équipements aussi variés que lesarmures, l'électroniques, les systèmes de capteur ou les câbles des ascensseur spatiaux. 

\subsubsection{Mousse métallique} 

La mousse métallique est obtenue en ajoutant des agents moussants à des métaux liquides, ce qui permet d'avoir des structures métalliques extrêmement légère - suffisament légère pour flotter sur l'eau. Elle est idéale pour la construction d'habitat et les cités flottantes. 

\subsubsection{Verre métallique} 

Les verres métalliques sont des alliages métalliques conçus pour avoir une structure atomique désordonnées (au lieu d'une structure cristalline) possédant une combinaison unique de souplesse et de force, en faisant une bonne surface de vêtement et une alternative à la céramique pour les armures. Ils sont également utilisés pour leur résitivité inhabituellement élevé (pour des métaux). 

\subsubsection{Métamatériaux} 

Les métamatériaux ont des propriétés physiques inhabituelles (généralement électromagntéique) en raison de leur structure, telle que le fait d'avoir un indice de réfraction négatif. Les métamatériaux sont utilisé pour créer des manteaux d'invisibilité (p. 316), des superlentilles, des grille optiques phasique et des hologrammes 2D impressionant. 

\subsubsection{Métaux refactorés} 

Ces alliages métalliques ont  des points de fusion extrêmement élevés, en faisant les matériaux idéaux pour les moteurs extrêmement chaud, les véhicules d'entrée atomsphérique et les avions hypersonique. 

\subsubsection{Alumine transparente} 

Sous forme transparente, cette céramique est souvent apelée saphir. L'alumine transparente étant plus dure que l'acier et les technique de jetté en zéro-G permettent la réalisation d'intriguantes construction transparente, tant que la faible résistance à la traction de ce matériau est respectée. 



\section{Équipement meshé} \label{sec:meshed-gear} 

Almost all technology in \emph{Eclipse Phase} is designed to be operated via radio signals from the user’s basic implant, although models usable by characters without basic implants are also available. De plus, tous les appareils contiennent un ordinateur quasiment microscopique et unlien radio (appelé la "voix") qui permet àl'utilisateur de localiser facilement l'objet et d'obtenir un état de l'objet ou de l'appareil, de savoir comment l'utiliser et l'entretenir correctement,ainsi que de savoir quand l'appareil doit être réparé et comment. La plupart des voix sont discètes et extrêmement utiles, mais les objets de mauvaisequalité ont des voix partiuclièrement ennuyante. 

Cela signifie qu'il est possible d'accéder à presque tout lesapapreilsparle mesh ou directement si ils sont à portée de signal radio. Cela les rend vulnérables au piratage et aux tentatives d'intrusions (p. 254) ainsi qu'à l'interception radio (p. 262). Cependant, denombreux appareils sont publiquement accessibles (voir \emph{Spimes},p 238). L'équipement meshé peut également être pisté par le mesh (p. 251). D'un point de vue sécurité et intimité, ces appareils sont souvent asservis à d'autres systèmes (voir \emph{Asservir des Appareils}, p. 248); les appareils portés par les personnages sont généralement intégrés au réseau personnel et asservi aux inserts de mesh du personnage. Pour plus d'informations sur les appareils meshés, voir le chapitre sur le \emph{Mesh}, p. 234. 

Beaucoup d'appareils sont fournis avec une IA, qui est équipée de logiciels de compétences qui lui permet d'utiliser elle-même l'appareil, en fonction des instructionsvocales et des commandes reçues par réseau. Les IA sont décrites aux p. 264 et p. 331. 



\subsection{Portée de signal et de capteur} \label{sec:radio-sensor-ranges} 

Dans \emph{Eclipse Phase}, quasiment tous les appareils sont équipées de petites radios leur permettant d'être meshés. Denombreuses pièces d'équipement sont également équipées de capteurs tels que caméras, micros ou autres détecteurs. La table de Portée de Signal et de Capteur note la portée à laquelle opère ces appareils. 

\begin{table} \begin{tabularx}{\textwidth}{|l|l|l|X|} \hline

\multicolumn{4}{|c|}{\textbf{Portée de signal et de capteur}}	\\ \hline

\textbf{Catégorie de taille}	&\textbf{Portée urbaine}	&\textbf{Portée urbaine}	&\textbf{Exemples} \\ \hline

Nano &20 mètres &100 mètres	&Poussière Intelligente, Essaime de nanobots/microbots \\ Micro	&50 mètres	&500 mètres	&Microinsectes \\ Mini	&1 kilomètre	&20 kilomètres	&Inserts de Mesh\\ Petit	&5 kilomètres	&50 kilomètres	&Ectos, Farcasteurs Radio Miniature, Capteurs Portables \\ Moyenne	&25 kilomètres	&250 kilomètres	&Boosters de Signaux, Capteurs de Véhicules \\ Grand	&500 kilomètres	&5000 kilomètres	&\\ \hline

\end{tabularx} \label{tab:radio-sensor-ranges} \end{table} 



\subsection{Énergie} \label{sec:power} 

Tous les appareils à énergie d'\emph{Eclipse Phase} ont besoin d'électricité pour fonctionner. A de rares exceptions, la plupart d'entre eux dépendent soit de panneaux solaires, soit de batteries. Ces batteries sont des supraconducteurs à température ambiante à haute densité et avec 25 fois plus de capacité que lesmeilleures batteries d'utilisation courante au début du 21° siècle. De telles batteries peuvent également être construite pour être souple, imprimée sur les appareils ou tissés dans le tissu. Elles permettent 100 à 500 heures d'utilisation, et préviennent l'utilisateur lorsqu'elles commencent à s'épuiser. Des batteries nucléaire à isotopes radioactifs plus puissantes sont également disponible, lourdement blindées pour qu'elles n'émettent aucune radiation et sont fonctionnelles pendant 3 ans ou plus. 

En résumé, l'énergie ne devrait être que rarement un problème lors des parties d'\emph{Eclipse Phase}, sauf si cela sert l'histoire. Les pannes de courant peuvent aussi être la conséquence d'un échec critique. 



\section{Augmentation Personnelle} \label{sec:personal-augmentation} 

Presque tous les citoyens du système solaire, qu'ils soient humains, IA ou élevés, utilisent différentes formes d'augmentation biologique, cybernétique ou nanotechnologique. La liste suivante décrit les augmentations les plus communes.  

Sauf mention contraire, tous les bonus tirés d'augmentation personnelle sont compatibles et cumulatifs avec les bonus fournit par d'autre améliorations. 



\subsection{Augmentations standard} \label{sec:std-augmentations} 

La plupart des morphs produites dans le système solaire incluent lesaugmentations suivantes. 

\subsubsection{Biomods de base} 

Quasiment universellement  installés dans les biomorphs, de nombreux habitats n'autoriseront pas les individus à visiter/immigrer l'habitat si leur biomorph ne possèdent pas ces biomodifications dans un but de préservation de la santé publique. Les biomods de base consistent en une série de bidouille génétique, de virus adaptés et de bactéries qui accélèrent la guérison, augmentant grandement la résistance aux maladies et ralentissant le vieillissement. Une morph équipés des biomods de base guérit deux fois plus vite qu'un humain du début du 21° siècle, fait graduellement repousser les zones corporelles perdues, est immunisé à toutes les maladies normales (du cancer à la grippe, en passant par le SIDA), et est essentiellement immunisé au vieillissement. De plus, la morph ne nécessite pas plus de 3-4 heures de sommeil parnuit, est immunisé aux effets secondaires liés à une exposition prolongée à unegravité faible ou nulle et ne souffre pas naturellement de problème biologique tels que la dépression, les états de chocs suite aux blessures ou les allergies. \textbf{{Modérées, mais incluses gratuitement dans la plupart desbiomorphs}} 

\subsubsection{Inserts de mesh basique} 

Les inserts de mesh sontomniprésent parmi les morphs modernes. Ce réseau de cerveaux cybernétiques implantées est un équipement essentiel pour tous ceux qui veulent rester connectées et utiliser pleinement lemesh sans-fil. Les composants interconnectés de ce système incluent: 

\begin{itemize} \item \textbf{Ordinateur cranien:} Cet ordinateur sert de concentrateur pour le réseau personnel du personnage et héberge sa muse (p. 264). Il a toutes les fonctionnalités d'un smartphone ou d'un PDA, se comportant comme un lecteur multimédia, un navigateur de mesh, un réevil/calendrier, un système de positionnement et de cartographie, une calculatrice avancée, un système de stockage de fichier, un moteur de recherche, un client de réseau social, un programme de messagerie et un bloc note. Il gère toutes les entrées de réalité augmentée de l'utilisateur et peut exéctuer tous les logiciels que le personnage désire (voir \textbf{Logiciels}, p. 331). Il traite aussi les données XP, permettant à l'utilisateur de vivre les souvenirs enregistrés d'autres personnes, et permettant également à l'utilisateur de partager en temps réel ses propres entrées sensorielles XP avec d'autres. Des logiciels de reconnaissance faciale/d'image et de chiffrementr (p. 331) sont inclus par défaut \item \textbf{Transmetteur radio:} Ce transmetteur connecte l'utilisateur au mesh et à d'autres personnages/appareils à portée. Il a une portée effective de 20 kilomètre dans l'espace ou à d'autres endroits éloignés des interférences radio, et d'e 1kilomètre dans les habitats bondés. \item \textbf{Capteurs médicaux:} Cettes suite d'implant supervise le status médical de l'utilisateur, incluant le rythme cardiaque, la respiration, la pression sanguine, la tempoérature, l'activité neuronale et bien plus. Un systèpe de diagnostique médical sophistiqué interpètent les données et prévient l'utilisateur de tous les problèmes ou dangers. \end{itemize} 

Utiliser n'importe laquelle de ces fonction est aussi simple que de penser. \textbf{{Modéré, mais inclus gratuitement dans la plupart des morphs}} 

\subsubsection{Pile corticale} Une pile corticale est une petite unité de stockage cybernétique et protégée par un étui en synthdiamant de la taille d'un grain de raison, implantée à la base du crâne là où le cerveau et la moëlle épinière se connectent. Elle contient une sauvegarde numérique de l'ego du personnage. Partiellement nanoware, l'implant mainitiens un réseau de nanomachines qui supervisent lesconnexions synaptiques et l'architecture cérébrale, notant tous les changement et mettant à jour la sauvegarde de l'ego en temps réel, jusqu'au moment de la mort. Si le personnage meurt et que la pile corticale peut être récupérés, il peut être restaurés depuis la sauvegarde (voir Réincarnation, p. 271). t r Stack, p.
Les piles corticales n'ont pas d'accès externe ou sans-fil (par sécurité), elles doivent être supprimée chirurgicalement (voir Récupérer une Pile Corticale, p. 268). Les piles corticales sont extrêmement solides, nécessitant un effort particulier pour être abîmées ou détruites. Elles sont fréquement récupérées sur des corps qui ont étés démembrés ou réduit en bouillie. Les piles corticales sont intentionnellement isolées des inserts de mesh et des autres implants, comme mesure de sécurité pour prévenir le piratage ou la falsification externe. \textbf{{Modéré, mais inclus gratuitement dans la plupart des morphs}} 

\subsubsection{Cybercerveau} 

Les cerveaux cybernétiques sont le lieu de résidence de l'ego (ou de l'IA résidente) dans les synthmorphs et les pods. Imitant les cerveauxbilogiques, les cybercerveaux ont une architecture holistique et servent de nœud de commande et de centre de traitement central pour gérer les entrées sensorielles et le système de prise de décision. Un seul ego ou IA ne peut "habiter" un cybercerveau à la fois; pour en héberger plus, des inserts de mesh (p. 300) ou un module ghostrider (p. 307) doivent être utilisés. Puisque lescybercerveaux stockent les souvenirs numériquement, ils ont l'équivalent d'une augmentation mnémonique (p. 307). Ils ont aussi un port de marionettiste inclus (p. 307) et qui peut-être contrôlé à distance, bien que cette option peut-être désactivée par ceux qui attachent de l'importance à leur sécurité. Les cybercerveaux sont vulnérables au piratage de cerveau (p. 261) et à d'autres formes d'inflitration/attaque électronique. Les cybecerveaux sont fournis équipé avec deux  (ou plus) paires de port d'accès externe (p. 306), généralement localisés à la base du crâne et qui permettent une connexion filaire directe. \textbf{{Modéré, mais inclus gratuitement dans la plupart des morphs synthétique ou des pods}} 

\subsection{Bioware} \label{sec:bioware} 

Les augmentations de bioware peuvent être acquise soit comme un génémod lorsque la morph est conçue et en croissance, ou comme une modification tardive d'une morph existencesoit en utilisant des nanomachines pour modifier les tissus de la morph ou en développant l'organe en cuve puis en l'implantant. Le bioware peut être utilisé pour améliorer les biomorphs (y compris les pods et les élevées), mais pas les synthmorphs. Le bioware peut être utilisé pour améliorer les biomorphs (y compris les pods et les élevées), mais pas les synthmorphs (voir Synthmorphs et Bioware, p. 306). 

\subsubsection{Sens augmentés} 

La liste suivante est une liste des sens les plus courament augmentés. Chacun d'entre eux est également disponible sous forme d'implant cybernétique, mais le bioware est bien plus courant 

\textbf{Sens Directionnel:\textbf{ Le personnage a un sens innée de la direction et de la distance grâce à un système de navigation inertiel avancé. Le personnage peut définir arbitrairement n'importe quel point comme étant le "nord" et il peut garder une trace de la direction de ce dernier, ainsi que de savoir approximativement à quelle distance cepoint se situe. Les personnages avec cette augmentation peuvent toujours resuivre n'importe quelle route qu'ils ont prit, ne rencontrant des difficultés que dans les routes tri-dimensionnelles et manquant de marqueurs de navigation (tels que l'espace ou le milieu sous-marin; appliquez unmodificateur de -30). Le positionnement à lintérieur de habitats par quiconque ayant des inserts de mesh de base étant automatique, seuls les personnages s'aventurant dans des lieux éloignés ont besoin de cette augmentation. \textbf{[Bas]} 

\textbf{Écholocalisation:}Le personnage possède un sonar similaire à celui d'une chauve-souris ou d'un dauphin. Le personnage envoiede brèves implusions ultrasonique sur leur voisinage et les utilise pour former une image de ce voisinage grâce aux motifs de réflections des implusions reçues par les oreilles du personnage. Pour plus de détails, voir Utilisation de Sens Augmentés, p. 302. Cette augmentation fonctionne à la fois dans l'air et dans l'eau, et a une portée de 20 mètres dans l'air et de 100 mètres dans l'eau. \textbf{[Bas]} 

\textbf{Ouïe Augmentée:} Les oreilles de la morph sont améliorées pour entendre à la fois les fréquences sonores plus élevées et plus basses - la gamme de sons qu'ellespeuvent percevoir est le double des oreilles humaines normales (voir Utilisation de Sens Augmentés, p. 302). De plus, leur ouïe est considérablement plus sensible, leur permettant d'entendre des sons comme si ils étaient cinq fois plus proche qu'ils ne le sont réellement. Un personnage avec cette augmentation peut facilement entendre même une conversation à voix basse à une autre table d'un petit restaurant. Cette augmentation fournit un modificateur de +20 à tous les Tests de Perception impliquant l'audition. \textbf{[Bas]} 

\textbf{Odorat Augmenté:} Le sens de l'odorat de la morph est équivalent à celui des chiens de chasse. L'utilisateur peut identifier à la fois les produits chimiques et les individus par l'odeur, et ils peuvent également pister des personnes et des produits chimiques par l'odeur du moment que la piste a étét faites lors des dernières heures et qu'elle n'a pas été obscurcie. Le personnage peut aussi avoir un sens général des émotions et de la santé de tout personnage dans les 5 mètres (+20 aux Tests de Perception ou de Kinésique dans ce but). \textbf{[Bas]} 

\textbf{Vue Augmentée:} Les yeux de la morph ont une vision tétrachromatique capable d'une différenciation des couleurs exceptionnelle. Ces yeux sont également capable de percevoir le spectre électromagnétique allant des fréquences terahertz aux rayons gamma, leur permettant de voir un total de plusieurs douzaines de couleurs, au lieu des sept ordinairement perçues par les yeux humains. De plus, ces yeux ont une miseaupoint variable équiavelente à un zoom de puissance 5 ou à des jumelles. Cette augmentation fournit un modificateur de +20 à tous les Tests de Perception basés sur la vue. Pour plus de détail, voir Utilisation de Sens Augmentés, p. 302. \textbf{[Bas]} 

\subsubsection{Augmentations mentales} 

Les augmentations mentales sont extrêmement courantes. 

\textbf{Mémoire Éidétique:} Le personnage peut se rappeler de tout ce qu'il leur est arrivé, en détail, et sans perte de souvenir. Par exemple, ils peuvent réciter une page qu'ils ont lus dans unlivre il y a unmois, se rappeler d'une chaîne de 200 caractères aléatoire qu'ils ont vus il y a un an, ou même vous dire ce qu'ils ont eu au petit-déjeuner un jour précis il y a dix ans. Ils ne peuvent cependant se rappeler que les choses auxquelles ils font attention. Le personnage ne se rappelera pas le contenu d'une note sur le bureau de quelqu'un si ils l'ont à peine regardé; ils doivent l'avoir lue en détail. Aucun effort n'est nécessaire pour utiliser cette augmentation, le personnage a seulement besoin d'essayer de se souvenir d'un fait spécifique. \textbf{[Bas]} 

\textbf{Hyper Linguiste:} Le cerveau de la morph entretient la flexibilité linguistique d'un petit enfant, peremttantau personnage d'apprendre de nouvelles langues avecune facilité déconcertante. Cette augmentation fonctionne comme le trait Hyper Linguiste, p. 146. \textbf{[Bas]} 

\textbf{Amélioration Mathématique:} Cet implant fonctionne comme le trait Génie des Maths, p. 146. \textbf{[Bas]} 

\textbf{Personnalités Multiples:} Le cerveau du personnage est intentionnellement partitionné pour s'accomoder d'une personnalité supplémentaire. Cette multiplicité n'est pas vue comme un désordre, mais comme un outil cognitif servant à aider les personnes à gérer leurs environnements hypercomplexe. Cette personnalité supplémenatire peut-être un PNJ joué par le maître de jeu, un personnage distcint (sous forme d'ego uniquement) créé par le personnage, ou un fork d'un autre personnage. De tous les points de vue, cette personnalité supplémentaire doit être considérés comme un ego séparé (i.e., elle doit forker séparément), excepté que les deux personnalités sont sauvegardées dans la même pile corticale et que si elles sont téléchargées, elles doivent être placée dans des morphs séparées ou dans une autre moprh avec le même implant. 

Un seul ego peut cobntrôller la morph à la fois. L'autre réside en arrière-plan,toujours actif, mais pas à la surface. Chaque ego est complètement conscient de ce que fait l'autre, de ses pensées, etc. Si, pour une raison ou une autre, la personnalité d'arrière plan veut prendre le contrôle, mais que l'autre persnnalité ne veut pas abandonner le contrôle, faites un Test en Opposition de VOL $\times$ 3. Chaque ego a sa propre Lucidité et son Seuil de Trauma, et elles gardent une trace distincte de leur stress et de leur trauma. Les attaquespsi et les influences social/mentales n'affectent que la personnalité au premier plan. Avoir un ego deplus dans votre tête, travaillant en tâche de fond, est utile pour faire du multi-tâche. Le personnage reçoite une Action Complexe supplémentaire à chauqe tour et qui ne peut être utilisée que pour les actions mentales ou de mesh. \textbf{[Élevé]} 

\subsubsection{Augmentations physique} 

La plupart des augmentations physique en bioware sont dérivées des possibilités des animaux. 

\textbf{Adrénaline Améliorée:} Cette amélioration de la glande adrénale surcharge la réponse adrénale dupersonnageaux situations impliquant le stress, la douleur ou desémotions violentes (peur, angoissen, plaisir, haine). Lorsqu'elle est activé, la décharge concentrée de norépinéphrine accélère le rythme cardiaque et lelux sanguin et brûle les glucides. En terme de jeu, cela permet au personnage d'ignorer le modificateur de -10 d'1 blessure et d'augmenter temporairement ses REF de +10 (améliorant aussi les compétences liées aux REF). Ces modificateurs s'appliquentjusqu'à ce que le personnage se calme (si il a aussi le contrôle endocrinien, p. 304, l'amélioration d'adrénaline peut alors être activée ou désactivée à volonté, et les blessures annulées sont cumulative). \textbf{[Élevé]} 

\textbf{Armure Biotissée (Légère):} Les armuresbiotissées implique le laçage de la peau de la morph avec des fibres de sois d'araignées artificielles. Cela fournit une Armure de 2/3 sans modifier l'apparence, la texture ou la sensitivité de la peau. Cette armure est cumulative avec l'armure portée. \textbf{[Bas]} 

\textbf{Armlure Biotissée (Lourde):} L'armure biotissée lourde implique le laçage de la peau de la morph avec un réseau plus dense et plus épais des mêmes fibres. La peau de la morph devient plus épaisse et relativement moins flexible, sauf au niveau des articulations. La peau de la morph a aussi une apparence inhabituellement lisse, et une texture particulièrement lisse et dure au toucher. Cela fournit une Armure de 3/4 sans réduire la mobilité de la morph. Le sens du toucher du personnage est cependant réduit significativement (modificateur de -20) à l'exception de leurs mains, de leurs pieds et de leur visages. Cette armure est cumulative avec l'armure portée. \textbf{[Modéré]} 

\textbf{Armure Carapace:} Les armure carapaces combinent des armures biotissés avec des plaques de dure mais flexible d'un matériau hybride chitine-céramique modélisé d'après les structures microscopique et les textures des exosquelettes des arthropodes. Cette armure est visible et à une apparence quelque peu crocodilesque ou insectoïde (au choix du personnage). La morph ne possède également aucune pilosité. Cela fournit une Armure de 11/11. Cette armure n'est pas cumulative avec l'armure portée. \textbf{[Modéré]} 

\textbf{Peau Caméléon:} La peau de la moprh est augmentée avec des chromatophores complexes qui lui permettent de changer de couleur comme la peau d'un caméléon ou d'une pieuvre. La morph peut prendre l'apparence de presque toutes les couleurs et de beaucoup de motifs. Cela fournit un modificateur de +20 aux Tests d'Infiltration pour éviter d'être vu ou remarqué, tant que le personnageest stationnaire ou qu'il ne se déplace pa à une allure supérieure à une marche lente. Le personnage doit être nu ou porter des vêtements intelligents (p. 325) de même couleur/motif. Si le personnage n'est pas complètement camouflé ou si il se déplace trop vite, réduisez le modificateur à +10. En plus de se fondre dans le décor, le personnage peut également changer consciemment la couleur et les motifs de sa peau pour se faire remarquer délibéremment (+20 aux Test de Perception pour remarquer) ou simplement pour produire des couleurs et motifs attractifs ou intéressants. \textbf{[Bas]} 

\textbf{Régulation Circadienne:} La morph n'a besoin que de 2 heures de sommeilpour se maintenir en forme et pour fonctionner à capacités maximales. ly.
Le personnage rêve constamment lorsqu'il dort et peut s'endormir et se réveiller presque instantanément. De plus, le personnagepeut afcilement et sans effet secondaire passer sur un cyclede 2 jour, restant éveillé 44 heures et dormant pendant 4. \textbf{[Modéré]} 

\textbf{Griffes:} La morph a des griffesrétracetiles comme celles des chats. Ces griffes n'interfèrent pas avec la dextérité manuelle du personnage et sont tranchantes commedes rasoirs. Elles restent cependant relativement petites et ne font que 1d10 + 1 + (SOM $\div$ 10) dégats avec une PA de -1. Elles sont donc légales dans la plupart des habitats et sont plutôt considérées comme des outils que comme des armes. \textbf{[Bas]} 

\textbf{Métabolisme Propre:} Les bactéries symbiotiques de la morph, sa flore intestinale et ses glandes ont été génétiquement retravaillées pour garder la morph "propre." La morph produit également des antibiotiques intelligents qui préviennent le développement de bactérie ou de parasites en interne ou sur sa peau. La morph est donc complètement immunisée aux infections, aux cariesdentaires et à la mauvaise haleine, sa transpiration n'a pas d'odeur, et le processuss digestif efficace de la morph produit moins de déchets slides et des composés chimiques moins odorants. \textbf{[Modéré]} 

\textbf{Glandes à Drogues:} La morph a des glandes spécifiquement conçues pour produire des hormones spécifiques ou des produits chimiques et les libérer dans le corps. Le personnage a le contrôle de ces glandes et peut libérer lescomposés chimiques à volonté. Chaque type de glandes à drogues est considérées comme une amélioration distincte. Pour les drogues et composés chimiques potentiels, voir p. 317. \textbf{{Une Catégorie de Prix Au-Dessus de Celle de la Drogue}} 

\textbf{Implant Anguille:} Dérivés des génome des anguilles électriques, unpersonnage peut avoir un implant d'anguille pour qu'il se connecte à un réseau de bioconducteur dans les mains et les pieds (ou d'autres membres), permettant au personnage de générer des chocs étourdissant au toucher. Les implants d'anguille inflige des dégats étourdissant (p. 204) exactement de la même façon qu'une paire de gants à chocs. Les implants d'anguilles peuvent être utilisés pour alimenter les implants et les périphériques portables spécialement étudiés pour au simple toucher. \textbf{[Bas]} 

\textbf{Amortisseur Émotionnels:} Cette altrernatif bonmarché au contrôle endocrinien (p.

304) allows the user to voluntarily damp their morph���s emotional resp 304) permet à l'utilisateur de volontairement atténuer les réponses émotionnelles de sa morph ainsi que différents signes non-verbaux tels quela dilatation de la pupille, lesmouvements occulaires ou l'intonation vocale. L'utilisation de cette augmentyation permet à l'utilisateur de lmentir et de dissimuler ses émotions de telle manière qu'ils peuvent tromper même l'observateur le plus attentifs; appliquez un modificateur de +30 aux Tests de Supercherie et d'Imposture. Cette modification n'affecte paslesméthodes de détection du mensonge et des émotions reposantssur la lecture de l'état neuronal du personnage, exploits psi-gamma inclus. Cette augmentation réduit cependant toutes les réponses émotionnelles et réduit la capacité du personnage à convaincre lors d'une interaction en temps-réel, imposant un modificateur de -10 aux autres Tests de compétences Sociales comme Persuasion. Les personnages peuvent activer et désactivercette augmentation à volonté. \textbf{[Bas]} 

\textbf{Contrôle Endocrinien:} Cette augmentation modifie le systèmeendocrinien de la morph, donnant aupersonnage un contrôle précis de ses flux hormonaux. Cela permet au personnage de contrôler entièrement son apétit et émotions et de réguler la douleur. Ils reçoivent un modificateur de +30 contre les effets de la faim, de la peur et contre toute forme de manipulation émotionnelle, telle que l'exploit Déclencher une Émotion. Cette augmentation permet également aux personnages de mentir avec conviction et de tromper complètement toutes les méthodes de détection de mensonge qui ne se base pas sur les signaux neuronaux de la cible; appliquez un modificateur de +20 aux Tests de Persuasion. Elle permet également aupersonnage de rester éveillé pendant 48 heures sanspénalité, mais passé cette période, le personnage commence à subir une fatigue normale. Enfin, la capacité à réguler la perception de la douleur permet au personnage d'ignorer le modificateur de -10 lié à 1 blessure. \textbf{[Élevé]} 

\textbf{Phéromones Améliorées: }La biochimie de la morph a été modifiée pour produire des phéromones améliorés qui affectennt de manière subconsciente le comportement des autres humains dans le voisinage. Ces phéromones rendent le personnage plus attirant et digne de confiance pour la cible; appliquez un modificateur de +10 aux Tests decompétence Social appropriés, tels que Persuasion. Cette augmentation n'affectent que les personnages qui peuvent sentir les phéromones, et elle n'affecte ni les élevés, ni les xénomorphs. [Bas] Respiration Améliorée: En améliorant à la fois l'efficacité des poumons et la capacité de transport de l'oxygène du sang, le personnage peut vivre confortablement dans les environnements ensurpression ou en dépression, de 0,2 à 5 atmosphères, sans vertiges ou besoin de paliers de décompression. De plus, le personnage peut garder sarespiration pendant 30 minutes en effectuant une activité minimale ou pour 10 minutes en effectuant une activité extrêmement fatiguante. \textbf{[Bas]} 

\textbf{Branchies:} Les tissus pulmonaires de la morph ont été adaptés pourfonctionner comme des branchies, permettant à la morph de respirer à lafois dans l'air et dans l'eau, tant que l'eau n'est ni toxique ni stagnante. Les personnages ayant cette augmentation respirent sous l'eau et expulsent ensuite l'eau grâce à des ouïes situées juste sous la dernière paire de côte et qui se verouillent lorsque le personnage n'est pas sous l'eau. \textbf{[Bas]} 

\textbf{Coussinets Antidérapants:} La morph possède des coussinetsspécialisés sur la paume, les bras, les jambes et sous les pieds. Concçus pour imiter les coussinets des pieds desgeckos, les personnages peuvent se maintenir sur un mur ou un plafond en plaçant deux de ces padscontre toute surface composé de matériau qui n'ont pas été spécialement conçus pour résister à cette augmentation. Les personnages peuvent escalader n'importe quelle surface et se déplacer facilement à travers un plafond qui supporte leur poid. Appliquez un modificateur de +30 aux Tests d'Escalade. Les coussinets doivent être encontact direct avec la surface qu'escalade le personnage (pas de gants,pas de chaussures). La nature de ces coussinets est évidente pour quiconque les regarde, mais ils ne gènent pas le sens du toucher du personnage ou sa dextérité. Si ces coussinets sont combinées avec l'augmentation d'étanchéité au vide, le personnage peut même coller aux surfaces dans le vide spatial. \textbf{[Bas]} 

\textbf{Hibernation:} Le personnage peut réduire volontairement le métabolisme de la morph au point qu'elle ne nécessite que 5\% de laquantité normale denourriture, d'eau et d'air. Le personnageparaît sombrer dans un étât de sommeil profond, mais il peut maintenir un état de conscience faible, tactile et auditive, et peut donc être facilement réveillé. Entrer ou quitter cet état nécessite 3 minutes pendant lesquelles le personnage est relativement impuissant. Avecsuffisament d'oxygène, un personnage peut hiberner sans danger pendant 40 jours sans manger ni boire. \textbf{[Bas]} 

\textbf{Augmentation Musculaire:} La masse musculaire de la morph a été améliorée et tonifiée, et les myofibres ont été renforcées. Appliquez un modificateur de +5 à la SOM. \textbf{[High]} 

\textbf{Drogues Neurales:} Cette modification bioware améliore les synapses du personnage et surcharge ses neurotransmetteurs, améliorant drastiquement la vitesse de ses connexions neurales. Les drogues neuralespeuvent être activées mentalement ou déclenchées par une émotion forte. Les drogues neurales de niveau 1 augmentent la stat Vitesse du personnage de +1,sans effets secondaires. Leniveau 2 augmente la stat Vitesse de +2, mais à chaque utilisation, le personnage subit une redescente de 1 heure après que l'amélioration se termine et qui se traduit par une fatigue du système nerveux (appliquez un modificateur de -20 à toutes les actions). \textbf{[Élevé (Niveau 1), Cher (Niveau 2)]} 

\textbf{Glande à Poison:} Similaire aux glandes à drogue, cette morph possède des glandes spéciales qui produisent des toxines, un peu comme les glandes à venin des serpents. La morph a des glandes à poisons dans ses doigts et dans la bouchen, pour qu'elle puisse libérer le poison en griffant quelqu'un avec un ongle, en le mordant jusqu'au sang ou même en partageant une boisson ou en crachant dans le verre d'une personne. La morph est immunisée au poison qu'elle produit. Ces glandes ne peuvent produire de nanotoxines. \textbf{[Bas]} 

\textbf{Pieds Prehensiles:} Les pieds et les articulations des jambessont modifiées pour que ses orteils soient plus long et plus souples et le gros orteils est transformé en pouce opposable. Physiquement, les pieds de lamorphs ressemble à une main longue et fine ou à unpiedhumain avec des orteils en forme de doigts (et de pouces). Le personnage peut marcher normalement mais doit porter des chaussures spécialement adaptées. Cette morph court cependant moin vite qu'une morph ayant des pieds non modifiés (-1 mètre par Tour d'Action). Les hanches de la morph sont également légèrement modifiées pour peremttre une meilleure mobilitée. Dans un fauteuil adapté, ou en flottant en zéro-G, le personnage peut utiliser à la fois ses mains et ses pieds pour manipuler le même objet. La plupart des morphs utilisées par lespersonnages vivant en zéro-G possèdent cette augmentation. \textbf{[Bas]} 

\textbf{Queue Préhensile:} Une longue queue préhensile (1,5 mètre) est ajoutée dans le dos de la morph, prolongeant la colonne vertébrale. Cette queue est préhensile et peut être utilisée pour attraper, tenir et même manipuler des objets. Le personnage peut contrôler les mouvements de la queue en se concentrant, mais elle bouge d'elle même dans le cas contraire. La queue améliore également l'équilibre du personnage; appliquez un modificateur de +10 à tout Tests de compétence Physique pour lesquel l'équilibre a une importance. \textbf{[Bas]} 

\textbf{Changeur de Sexe:} Un ensemble complexe de modifications permet aupersonnage de changer leursexe physique vers mâle, femelle, hermaphrodite ou neutre. Ce changement est déclenché mentallement mais prends une semaine pour se terminer. \textbf{[Modéré]} 

\textbf{Poche de Peau:} Cette morph a une poche capable de tenir et de cahcer (+30) les petits objets située dans son épiderme. \textbf{[Trivial]} 

\textbf{Tolérance Thermique:} La régulation thermique et le système circulatoire de la morph ont été sibstantiellement améliorés permettant au personnage de survivre aux températures allant de -30 degrés Celsius jusqu'à 60 degrés Celsius sans inconfort ou effets secondaires. \textbf{[Bas]} 

\textbf{Filtres à Toxine:} La morph possède un foie et des reins améliorés ainsi que des filtres biologiques dans ses poumons. Les personnages avec cette augmentation son immunisés à toutes les toxines chimiques oubiologique,incluant tout ce qui va desdrogues récréatives aux agents neurochimique ou à la nourriture avariée. De plus, le personnage peut respirer de la fumée et boire de l'eau salée sans aucune gène ni aucun danger. Contrairement aux machinaments, l'immunité aux toxines empêche le personnage de faire l'expérience des plus petites blessures et incomforts générés par les toxines (les machinaments réparant rapidement les dégats causés parla toxin puis les suppriment de la morph). Cette augmentation ne fournit aucune résistance aux acidesconcentrés, aux attaquesnanotechnologiques ou aux agents similairement destructifs. Certains personnages avec cette augmentations apprennent à apprécier le goût de différentes toxines chimiques telles que le cyanure ou l'arsenic. \textbf{[Modéré]} 

\textbf{Étanchéité au Vide:} Pour posséder cette augmentation, le personnage doit également posséder une forme d'armure bioware ou d'armure carapace. La morph a été spécialement conçue pour survivre aux effets du vide. La peau du personnage résiste au vide et protège le porteur des températures allant de -75 à 100$^{\circ}$ C. De plus, la bouche du personnage,son nez et tous les autres orifices peuvent se verrouiller suffisament pour résister au vide, et la morph possède une membrane spéciale qui se déploie sur ses yeux, permettant aupersonnage de voir dans le vide sans risque de dégâts occulaires. Cette augmentation est généralement combinée avec la respiration améliorée ou une augmentation du stockage d'oxygène, ou les deux. \textbf{[Élevé]} 

\subsubsection{Synthmorphs et bioware} 

Même si le bioware est préféré et plus courant, de nombreux éléments de bioware peuvent être imitée par de la cybernétique. Cela est particulièrement vrai pour les synthmorphs/robots qui ne peuvent être améliorés avec du bioware. Les pièces de bioware suivantes peuvent être répliquées en cybernétique pour les synthmorpsh et les robots: 

\begin{itemize} \item Peau Caméléon \item Glandes à Drogues \item Matériel d'Anguille \item Amortisseurs Émotionnels \item Sens Améliorés (Tous) \item Coussinets Adhésifs \item Augmentations Mentales (Toutes) \item Augmentation Musculaire \item Drogue Neuronale \item Glande à Poison \item Pieds Préhensile \item Queue Préhensile \end{itemize} 

\subsection{Cyberware} \label{sec:cyberware} 

Trés peu de cyberware est implanté physiquement. La morph est en fait mise dans une cuve de soin (p. 326) et les nanomachines de la cuve construisent le cyberware à l'intérieur du corps de la biomorph. Le cyberware est rarement utilisé pourquoi que ce soit qui peut déjà être fait par du bioware. 

Les synthmorphs et les bots peuvent également utiliser du cyberware. 

\subsubsection{Sens augmentés} 

En plus d'être capable de dupliquer les effets de tous les sens améliorés bioware, il y a quelques sens améliorés qui ne peuvent être produits que par le cyberware. 

\textbf{Anti-Flash:} Cette modification visuelle élimine les pénalités d'éblouissement. \textbf{[Bas]} 

\textbf{Sens Électrique:} Le personnage peut percevoir les champs électriques. Sur un rayon de 5 mètre, le personnage peut déterminer instantanément si un appareil électrique est allumé ou éteint et il peut voir la localisation d'un cabalge électrique derrière un mur ou dans un appareil. Ce sens donne au personnage unmodificateur de +10 à tous les tests impliquant l'analyse, la réparation ou la modification d'équipements électrique. \textbf{[Bas]} 

\textbf{Sens Radiocatif:} Le personnage peut percevoir la présence et la source approximative de toutes ls formes de radiations dangereuses, incluant les neutrons, lesparticules chérgées et les rayons cosmiques. \textbf{[Bas]} 

\textbf{Émetteur de Rayons-T:} Installés sous la peu du front du personnage, cet implant génère un rayonnement térahertz de basse puissance (rayons T) qui permet aupersonnage de voir en utilisant la réflection des rayons-T. Comme expliqué dans Utilisation des Sens Améliorés, p. 302, cet implant combiné avec l'amélioration de la vision (ou un capteur térahertz) permet à l'utilisateur de voir à travers le tissu, le plastique, le bois, les murs, les compositeset les céramiques ainsi que de déterminer lacompositions de divers matériaux. Cet implant permet à l'utilisateur de voir en utilisant les rayons-T sur 20 mètres en atmosphère normale et sur 100 mètres dans le vide. \textbf{[Bas]} 

\subsubsection{Augmentations mentales} 

Ces augmentations cybernétiques améliorent le cerveau et les fonctions cognitives. 

\textbf{Prise d'Accès:} Généralement situé à la base du crâne ou sur la nuque,cet implant est un connecteur externe doté d'une interface neurale directe. Cela permet au personnage d'étbalir une connexion filaire directe en utilisant un câble en fibre optique vers des appareils externe ou d'autres personnage, qui peuvent être utilisés dans des endroits ou les liens sans-fil ne sont pas fiable ou lorsqu'une intimité toitale est nécessaire. Deux personnages connectés par des prises d'accès peuvent "parler" d'esprit à esprit et transférer des informations entre leurs inserts de mesh et d'autres implants. Toutes les synthmorphs en sont équipées par défaut. \textbf{[Bas]} 

\textbf{Coupe Mort:} Cet accessoire de pile corticale (p. 300) est conçu pour empêcher la pile de tomber entre de mauvaisesmains. Si la morph est tuée, le coupe mort efface et détruit complètement la pile corticale, empéchant ainsi la récupération de l'ego. Cette option n'est généralement utilisées que par les agents clandestins ayant des sauvegardes récentes. \textbf{[Bas]} 

\textbf{Farcast d'Urgence:} Seuls les personnages ayant une pile cprticale peuvent avoir cette augmentation. La morph possède un farcast quantique implanté (p. 314) relié à une installation de stockage de haute sécurité. Le prix élevé de cet  implant couvre également le prix de ce stockage. En utilisant des transmissions standard et du chiffrement quantique, le farcast envoie une sauvegarde complète de l'ego du personnage (extrait de la pile corticale) toutes les 48 heures. À la discrétion du maître de jeu, l'intervalle de sauvegarde peut-être planifié avec une fréquence plus ou moins élevées, en gardant à l'esprit que les émissions d'ego sont générallement limitées dans une optique de sécurité et parcequ'ils consomment de la bande passante. Ces émissions ne fonctionnent que lorsque le personnage est en contact radio avec l'installation de stockage et n'est généralement utilisé qu'à l'intérieur d'un habitat pour envoyer des sauvegardes vers un vaisseau spatial à proximité. Si les émissions radios sont bloquées ou interceptées, cet appareil ne peut pas faire de sauvegarde. 

En cas d'éched du farcast, cette augmentation inclut également un émetteur à neutrino d'urgence utilisable une fois (p. 314). Cet émetteur contient approximativement 10 nanogrammes d'antimatière stockés dans un containeur à triple redondance magnétique de la taille d'une orange. Si le personnage meurt ou a besoin d'abandonner sa morph en urgence, cette petite quantité d'antimatière est mise en contact avec unequantité identique de matière de manière ontrôllée afin de générer une seule émission de neutrino rapide et mnutieusement codée contenant la sauvegarde la plus récente de l'ego. La chaleur générée par l'opération cuit littéralemnt la morph, la tuant et détruisant tous les implants et l'électronique interne et externe. 

Le processus entier prend moins de 0,1 secondeet l'émission peut-être reçue tant que le récepteur à neutrinosest situés dans les 100 unités astronomiques du personnage. À l'intérieur du système solaire, cet implant garanti la sauvegarde effective du personnage. Il est beauoup moins utile sur les exoplanètes où le personnage est hors de portée de son installation de sauvegarde. La quantité d'antimatière véhiculée par cet implant est suffisament petite pour qu'ellene produise pas d'explosion et elle n'endommageras aucun objet alentour. La plupart des habitats scannent attentivement  tous les visiteurs pour déterminer si ils ont cet implant et si la quantité d'antimatière impliquée est sufisament basse pour ne pas mettre en danger l'habitat et ses habitants, certains vont jusqu'à bannir comlètement cet implant. \textbf{[Cher]} 

\textbf{Module Ghostrider:} Cet implant permet au personnage d'embarqueruneautre infomroph dans leur tête. Cette infomroph peut-être uneautre muse, uneIA, un ego sauvegardé ou un fork. Le module est connecté aux inserts de mesh du personnage, le passager peut accéder au mesh. Le personnage peut limiter les accès du passager, ou l'autoriser à obtenir un accès direct à leurs informations sensorielles, pensées, communications et autres implants. \textbf{[Bas]} 

\textbf{Augmentation Mnémonique:} Un personnage doté de cette augmentationet une pile corticale peut accéder aux enregistrements numériques de toutes les données sensorielles dont ils ont fait l'expérience au format XP (et ils peuvent partager ces enregistrements avec d'autres). L'augmentation mnémonique diffère du bioware mémoire éidétique car elle permet aux personnages de partager numériquement toutes leurs données sensorielles avec les autres. Elle permet également d'examiner précisément les données sensorielles auquelle un personnage n'avait pas forcément fait attention. Par exemple, si le personnage a jeté un œil à une note mais ne l'as pas lue, il peut utiliser plus tard des logiciels de traitements d'image pour améliorer cette image et réellement lire ce que dit la note. L'augmentation mnémonique permet aux personnages de percevoir clairement tous les bruits de fonds, comme une conversation à la table d'à côté dont le personnage n'a initiallement entendus que quelques mots. Utiliser l'augmentation mnémonique pour retrouver une information spécifique est relativement facile, mais requiert généralement entre 2 et 20 minutes de concentration. \textbf{[Bas]} 

\textbf{Multi Tâche:} Seuls les personnages ayant une pile corticale peuvent avoir cette augmentation. Lepersonnage possède un ordinateur avancé installé dans leur cerveau qui utilise les données dans la pile corticale pour créer simultanément différents forks à court-terme pour gérer diverses tâches mentale. Cet ordinateur est conçus pour réintégrer tous ces forks dans la personnalité principale du personnage après un maximum de 4 heures, ou plus tôt si désiré. Cette augmentation peremt au personnage de préparer un discours tout en se livrant à une exploration intense du mesh et en participant simultanément à un échange de tir ou en étant poursuivi, chaque fork opérant indépendament. Cependant, ces forks ne peuvent être utilisés que pour effectuer des tâchs uniquement mentales ou en-ligne. Cette augmentation peut produire un maximum de deux forks à la fois, donnant aupersonnage deux Actions Complexes supplémentaire à chaque Phase d'Action pour desactions mentales ou en-ligne. Cet implant ne peut-être utilisé simultanément avec d'autrs augmentations autorisant des actions supplémentaire, ou avec l'augmentation de voitesse mentale (p. 308). \textbf{[Élevé]} 

\textbf{Prise Marionette:} Cet ordinateur implanté permet au corps de la biomorph (la "marionnette") d'être contrôlée par un autre personnage (le "marionnetiste"). Lorsque la prise est activé, la marionnette n'a aucun contrôle sur sonc orps et est juste passager (à la discrétion du maître de jeu, les marionettes qui sont tourmentées par des pertes de contrôle répétées ou étendues peuvent subir du stress mental). Le marionetiste peut directement "intercepter" la marionette ou la contrôler àdistance de la mêam manière que les robots et les pods sont téléopérés (p. 196). Le marionettiste doit soit être passager de lamarionette (voir le Module Ghostrider,p. 307) ou avoir un lien de communication diret (via le mesh, la radio, le laser, etc.). \textbf{[Modéré]} 

\subsubsection{Augmentations physique} 

Ces implants améliorent le corps physique de la morph. 

\textbf{Cybergriffes:} Les os sur le dos des mains de la morpsh sont reliées à ds griffes en matériaux intelligents. Ces griffes peuvent se déployer à travers des ports dissimulés sous la peau de la morph et s'étendent sur 15 centimètres au delà des phalanges de la morph. Ces armes aiguisées comme des rasoirs infligent 1d10 + 3 + (SOM $\div$ 10) points de dégats et ont une PA de -2. Si elles sont combinées avec le matériel d'anguille (p. 304), elles peuvent aussi infliger des chocs électriques. De manière similaire, elles peuvent injecter poisons et nanotoxines secrétés par une glande à poison (p. 305) ou des nanotoxines implantées. \textbf{[Bas]} 

\textbf{Cybermembre:} À une époque où les bras et les jambes peuvent repousser facilement, beaucoup de personnes considèrent les protèses cybernétiques comme vulgaire et disgracieuses. La Racaille et d'autres les considèrent cependant comme des symboles iconiques d'auto-expression. Des remplacement standards de cybermembre fonctionne de la même manière que leurs équivalents biologiques, bien que ces mmebres particuliers reçoivent un bonus d'Armure de +3/+3 lorsqu'ils sont ciblés spécifiquement (ce bonus ne s'applique pas aus synthmorphs). Les cybermembres peuvent être masqués pour paraître réel (voir Masque Synthétique, p. 311), et peuvent aussi avoir de petits compartiments pour cacher/stocker de petits objets. \textbf{[Modéré]} 

\textbf{Cybermembre Plus:} Des modèles de cybermembres plus extravagants sont également disponible, bien qu'ils nécessitent des modificatins corporelles plus importantes pour s'adapter. Ces membres fournissent un bonus de +5 à la SOM par membre (maximum +10). Il peut s'agir de remplacement ou de membres "supplémentaire" accrochés sur la structure squeletique du corps. Ces cybermembres ne peuvent être masqués. \textbf{[Élevé]} 

\textbf{Main Laser:} La morp posède un laser de grade militaire implanté dans l'avant bras, équiopé d'un flexible guidant la lumière à une lentille située entre les deux premières phalanges de la main directrice de la morph. Le laser tire à travers ce guide, infligeant 2d10 points de dégats avec 0 de PA. Le laser est alimenté apr unepetite batterie nucléaire situé dans le torse de la morph, permettant 50 tirs avant de devoir être rechargé comme toutes les autres batteries d'armes à rayons (p. 338). \textbf{[Modéré]} 

\textbf{Squelette Renforcé:} Le squelette renforcé a été tissé avec des matériaux renforcés. APpliquez un bonus de +5 en SOL et +5 en SOM. \textbf{[Élevé]} 

\textbf{Réserve d'Oxygène:} La morph a un réservoir d'oxygène miniature et un recycleur d'air installé dans son torse. Cet implant fournit l'équivalent du système de surviedans une exocombi légère (p. 333), peremttant au personnage de respirer confortablement pendant 3 heures. L'imlant alimente lamorph en oxygène directement dans le système sanguin, évitant les problèmes liés aux changements de pression. Des capteurs implantés permettent au personnage d'automatiquement utiliser l'oxygène stocké si ils détectent une atmosphère toxique ou insuffisante. Sans étanchéité au vide, le personnage ne peut survivre que 5 minutes dans le vide, mais il reste conscient et actif pendant tout ce temps, lui donnant beaucoup plus de temps pour trouver un abri ou une exocombi que les personnages ne dsposant pas de cet implant. Chaque heure passée dans une atmosphère respirable permet à cet implant de récupérer une heure d'oxygène. L'implant peut-être complètement rechargé en 15 minutes si lepersonnage est dans une atmosphère à haute-pression composée essentiellement d'oxygène. \textbf{[Bas]} 

\textbf{Accélérateurs de Réflexe:} La colonne vertébrale et le système nerveux de la moprh ont été recabél avec dematériaux supraconducteurs, accélérant leur vitesse de transmission. Cela augmente les REF du personnage de +10 et augmente sa Vitesse de +1. \textbf{[Cher]} 

\subsection{Utiliser les sens augmentés} \label{sec:using-enhanced-senses} 

Les augmentations personnelles et les aides technologique ont drastiquement amélioré les capacités sensorielles de la plupart des transhumains. Les notes suivantes fournissent certains détails sur les possibilités de ces fonctions sensorielles. Les capacités sont typiquemnt les mêmes qu'il s'agisse de sensbiologique ou de capteur technologique, bien que les capteurs techno peuvent "éteindre" certaines longueurs d'onde et percevoir seulement certaines fréquences spécifiques, alors que les sens biologiques perçoivent le spectre complet sans possibilité d'en filtrer certains éléments. 

\subsubsection{Bases de données senorielle} 

Les capteurs technologiques et les sens biologiques augmentés sotnfournis avec des bases de données de "signatures" qui facilitent l'identification de ce que perçoit l'utilisateur (dans le cas du bioware, ces bases de données sont stockés dans et accessible par l'insert de mesh du personnage). Par exemple, les capteurs infrarouges possèdent des bases de données listant les signatures thermiques de différents anilmaux et objets, facilitant l'identification de telles choses. Dans les situations appropriées, appliquez un modificateur de +20 pour identifier les cibles perçues de cette façon. 

\subsubsection{Actif contre passif} 

Un scanneur actif doit en fait émetre une fréquence particulière puis en mesurer le reflet; cela signifie qu'un capteur similaire peut détecter et localiser la source d'émission. Par exemple, un personnage avec une vision améliorer peut voir les radiations terahertzémisent par quelqu'un utilisant un capteur terahertz actif, un peu comme quelqu'un avec une vision normalepeut voir lalumière émise par une lampe torche. 

Un scanneur passif ne fait qu'analyser les fréquences émises naturellement - il n'y a rien qui signale l'existence du capteur. 

\subsubsection{Spectre électromagnétique} 

Dans le cadre des règles de \textit{Eclipse Phase}, le spectre électromagnétique est divisé en longueur d'onde et fréquence selon ces différents catégories: radio, micro-onde, térahertz, infrarouge, lumière visible, ultraviolet, rayons X et rayons gamma. 

\textbf{Radar (Radio/Micro-onde): }Les capteurs Radar donctionennt en emettant activement des ondes et micro-onde radio et en mesurant leur rebond sur la cible. Les radars fonctionnent au mieux pour détecter desobjets métallique et est beaucoup moins efficace (modificateur de -20) contre les biomorphs et les petits objets. La résolution n'est cependant pas trés élevée, il peut donc voir des formesmais ni les couleurs, ni les détails. Le radar peut être utilisé pour détecter à la fois la vitesse et leur mouvement, peut "voir" à travers lesmurs (jusqu'à un score d'Armure + Solidité cumulé de 100) et peut détecter les implantscybernétique ou les objets dissimulés. À courte portée (1 à 2 mètres), il peut détecter le rythmecardiaque et la respiration en mesurant le mouvement des cavitations thoraciques. 

\textbf{Térahertz:} Les capteurs térahertzémittent desrayons T, mesure leur réfléchissement et les comparent à une bases de données de signature térahertz que possèdent les différents objets/matériaux. La résolution est plus élevées que le radar, mais avec sensiblement moins de détails que la vision normale. De manière similaire au radar, les capteurs terahertz peuvent voir à travers les murs et d'autres matériaux, mais de manière moins intrusive (jusqu'à une Armure + Solidité cumulées de 50). Les rayons T existent à l'état naturel, mais les capteurs terahertz ont généralement besoin d'un émetteur car cesrayons sont absorbés par l'atmosphère (ainsi que par l'eau et les métaux). Dans le vide spatial, unmétteur n'est cependant pas nécessaire. De manière similaire, des analyses térahetz passive en atmosphère opnt une portée effective de25 mètres. Les rayons T ne pénètrent pas la peau et sont doncinefficace pour localiser des implants. 

\textbf{Infrarouge:} Des longueurs d'ondes prochedesinfrarouges sont utilisés pour la visionnocturne, fournissant une résolution et un niveau de détail équivalent à la vision normale sous faible lumière. Les infrarouges de longueur moyenne sont excellents pour détecter les sources de chaleurs (nonobstruées par le brouilalrd ou la fumée) et les différences de températures (aussi petite que 0,1°C), et une imagerie tehrmique de ce type percevra les traces de dissipation thermique laissée par une source chaude sur une plus froide, permettant à l'utilisateur de voir où s'est assis quelqu'un, de pister les empreintes de pas se dissipant ou de voir sur quels boutons on a appuyé, sil'utilisateur est suffisament rapide. Les infrarouges détectent également les aflux snaguins dans le visage d'unebiomorph, ce qui peut être utile pour évaluer un état émotionnel (modificateur de +20 aux Tests de Kinésique) et peut détecter les implants sous-cutanés. Certaines surface normalement blanche sont réflexive (elles agissent comme des miroirs) en infrarouge, et peuvent permettre à un observateur de regarder autour desangles ou derrière eux. D'un autre côté, certains verres sont opaques à la lumière infrarouge. Lesinfrarouges sont également utiles pour déterminer la composition chimique (permettant de faire un Test de Chimie par obsrevation visuelle pure). Les capeturs infrarouges sont des capteurs passifs. 

\textbf{Lidar (Lumière Visible):} Similaire au radar, mais avec une bien meilleure résolution, le lidar envoie activement de la lumière allant des infrarouge à l'utlravioletsur une cible et en mesure la dispersion, la fluorescence et d'autres propriétés. Le lidar est extrêmement utile pour détecter les propriétés chimique de l'atmosphère ainsi que la météo. Comme le radar, le lidar peut être utilisé pour mesurer la distance et la vitesse d'uneciblen, ou pour réaliser une image tri-dimensionnelle. Une utilisation intelligente du lidar est de "cartographie"préciséement la position de tout ce qu'il y a dans une pièce (nécessitant plusieurs tours d'analyse) puis de vérifier le positionnement plus tard pour voir si quelque chose a été déplacé. 

\textbf{Ultraviolet:} Certains objets sont fluorescent dans la lumière ultraviolette, incluant certains animaux, fleurs, insectes, urines et minéraux (qui se voient nettement mieux en ultraviolet qu'en lumière normale). Certaines plantes et animaux ont des motifs qui ne peuvent être vus qu'en ultraviolet. De manière identique, des teintures chimiques qui ne se voient que grâce aux ultraviolets ou qui peuvent rendre certaines substances (comme le sang) fluorescentes sous lumière ultraviolette, ont denombreux applications en sécurité. Certains verres sont opaques aux longueurs d'ondes ultraviolettes. 

\textbf{Rayons X/Rayons Gamma:} Les systèmes d'imagerie par dispersion qui utilisent les fréquences des rayons X et gamma, produsieent des iamges tri-dimensionnelles à haute résolution et sont trés utiles pour détecter des armes dissimulées et des implants. De tels systèmes d'imagerie sont trés bon pour pénétrer les murs et les métaux (jusqu'à un score d'Armure + Solidité cumulé de 200, du moins à des niveaux de rayonnements sûrs pour les transhumains). Ces capteurs peuvent, bien entendu, également détecter la présence de radiations dangereuses. 

\subsubsection{Ondes sonores} 

La transmission de vibrations ou de sons à travers un médium est divisé entre les infrasons (les fréquences inférieures à l'ouïe humaine normale), la gamme accoustique normale, et les ultrasons (fréquence supérieures à l'ouïe humaine normale). Les ondes sonores ne se propagent pas dans le vide. 

\textbf{Ultrasons:} Les sonars à ultrasons fonctionnent un peu comme les radars, envoyant des ondes rebonir sur une cible et en mesurant l'écho renvoyé. Les systèmes d'imagerie à ultrasons sont également en basse réoslution, affichant les formes et les mouvements, mais aucune couleur et trés peu de détails, à moins d'être trés près (1 à 2 mètres). Les ultrasons restent très bons pour identifier la densité d'un matériau et peut détecter des matériau plus dense cachés derrières d'autres moins denses. Beaucoup d'appareils médicaux utilisent les ultrasons et lescapteurs à ultrasons peuvent également détecter les fuites de gaz, les bruits de frictions de moteur et d'autres émissions mécaniques similaire. Les capteurs à ultrasons ne sontgénél=ralement pas effecté par les générateurs de bruits provenant de fréquences acoustiques standard. 

\textbf{Infrasons:} Les infrasons voyagent bien plus loin que les fréquences sonores classique (sur des centaines de kilomètres). Les machineries mécanique, les perturbations sismiques, les tornades, explosions, chutes d'eaux et certains phénomènes météorologiques créent des infrasons. Les grands animaux tels que les éléphants et les baleines utilisent les infrasons pourcommuniquer par le sol sur de grande distances, bien que le transfert de données par infrasons soit troplent pour des communications complexes. 

\subsubsection{Systèmes de capteurs combinés} 

Lorsqu'ils sont combinés, ces technologies de capetirs peuvent être trés puissantes. Par exemple, l'utilisation du lidar, d'imagerie thermographique et du radar peut fournir une carte tri-dimensionnelles dun bâtiment et de tous les objets et personnes présents à l'intérieurs. 



\subsection{Nanoware} \label{sec:nanoware} 

Toutes les augmentations de nanoware sont des éléments de technologie avancée (p. 328), composés de générateurs de nanomachines de la taille d'une noix qui produisent des nanomachines spécialisées. Le nanoware est disponible pour les synthmorphs et les bots en plus d'être disponible pour les biomorphs. 

\textbf{Nanotoxine Implantée:} La morph a une ruche denanomachine implantée qui produit des nanotoxines (p. 324). Cet implant est conçu pour que le personnage puisse déployer ces nanomachines instantanément en griffant, en crachant ou simplement en maintenantr un contat direct sur la peau. Les personnages peuvent choisir de déployer ou non ces nanomachines. Chaque générateur de nanotoxine ne peut produire qu'une variété de nanomachine, les types les plus courant étant ceux pouvant tuer ou incapacités presque toute les cibles vivantes ou ceux conçus pour détruire les machineries délicates. Les personnages sont immunisés à leur propres nanotoxines. Les nanotoxines sont hautement restreintes et de nombreux habitats n'autoriseront personen à entrer avec cet implant. \textbf{[Modéré]} 

\textbf{Machinaments:} C'est la forme la plus courante de nanoware. Ces nanobots supervisent le corps del'utilisateur à un niveau cellulaire et réparent tous les problèmes qui apparaissent. Les machinaments éliminent la plupart des maladies, drogues et toxines (mais ni les nanodrogues, ni les nanotoxines) avant qu'elles ne causent plus que des dégâts mineur pour l'hœte (voir Effets des Drogues, p. 318). Si désiré, l'utilisateur peut temporairement contourner cette protection p�ur permettre une intoxication ou d'autres effets, mais à moins que le personnage n'active un second contournement sépcifiquement dédié, les machinaments empécherons les toxines de s'accumuler à des niveaux létaux ou de dégâts permanents. Dans ce cas, elles peuvent aussi être activés un peu plus tard pour réduire la durée des effets de la drogue ou de la toxine de moitié 

Les machinaments permettent au personnage d'ignorer les effets d'1 blessure. Elles accélèrent également la récupération naturelle telles que notées sous Guérison des Biomorphs, p. 208. Si l'utilisateur subit 5 blessure ou plus en une seule fois, ou plus de 6 blessure en une heure, les dégâts ont surpassés la capacité de réparation des machinaments. Dans ce cas, les machinaments placent le personnage en stase médicale, un état dans lequel l'esprit et le corps sont parfaitement préservés, mais où le personnage ne peut pas agir. Dans ces circonstances, les machinaments envoient également un appel prioritaire pour des services d'urgence en utilisant l'insert de mesh du personnage. Les machinaments pour synthmorphs et bots sont composés de nanomachines qui supervisent et réparent l'intégrité de la coque et le fonctionnement des systèmes internes. Notez que la version des machinaments pour synthmorphs leur permettent de s'auto réparer de la même manière que les biomorphs avec des machinaments se soignent naturellement (p. 208). \textbf{[Bas]} 

\textbf{Accélérateur Cognitif:} Avec ce système de nanoware, les nanomachines altèrent l'architecture nuerale du personnage et augmentent le fonctionnement de sesneurones. Lepersonnage peut accélérer délibérément sonesprit pour penser mais aussi pour recevoir et traiter les informations sensorielles bien plus rapidement qu'un humain ordinaire. Le temps subjectif semble ralentir pour le personnage, lui permettant de planifier minutieusement sa prochaine action, même si ils n'ont qu'une fraction de seconde pour le faire. Lorsque ce système est actif, le personnage peut discerner des choses qui se produisent trop rapidement pour qu'un humain normal le perçoive, comme la distinction de chaque image d'un vieux film analogique ou la compréhension de son qui ont été accélérés à une vitesse bien plus élevé que la normale. Le personnage peut également lire 10 fois plus vite que la normaleet peut suivre la trajectoire des balls et d'autresobjets se déplaçant à une vitesse similaire en réussissant un Test de Perception. 

En utilisant cette augmentation, le personnage gagne deux Actions Complexe supplémentare par Phase d'Action, mais elles ne peuvent être utilisées que pour des actions mentales. Le personnage reçoitg également un bonus d'Initiative de +30. Le personnage pense à vitesse normale lorsque ce nanoware est inactif. Ce nanoware est incompatible avec tout les autres types d'augmentations qui fournissent une forme d'actions supplémentaires tels que le multi tâche. Cette augmentation peut-être utilisé aussi souvent que désiré, mais l'utiliser activement rend les conversations et les interactions sociales ordinaire difficile et nécessite un effort de concentration supplémenbtaire. \textbf{[Élevé]} 

\textbf{Nanophages:} Ces nanomachines patrouillent le corps, déclenchant des alertes lors de signes de nanodrogues ou nanotoxinesintruse et en les détruisant avant qu'elles aient plus qu'un effet mineur. Les nanophages fournissent une immunité automatique contre les nanodrogues et les nanotoxines sauvent si elles ont été spécifiquement instruite de se tenir en retrait par l'utilisateur, via ses implants de mesh. \textbf{[Modéré]} 

\textbf{Oracles:} Ces nanomachines neuronales macrocapteurs font attention aux entrées sensorielles sur lesquelles le personnage ne se concentre pas, le prévenant des choses importantes qui pourraient échapper au personnage. Les oracles agissent également comme une sorte de zone tampon de souvenirs et d'aide à la recherche, aidant le personage a se rappeler de souvenirs et des détails, et en les vérifiants vis à vis d'autres souvenirs. Les oracls annulent les modificateurs de Perception liés à la distraction, appliquent un modificateur de +10 aux Tests d'Investigation et ajoutent un bonus de +30 aux tests liés à la mémoire. \textbf{[Modéré]} 

\textbf{Respirocytes:} Ces nanomachines se comportent comme des globules rouges artificiel et extrêmement efficace, augmentant les propriétés de transfert de l'oxygène et du dioxyde de carbone. Ils permettent à la morp de ne pas respirer pendant 4 heure et augmentent sa SOL de +5. \textbf{[Modéré]} 

\textbf{Skillware:} Le cerveau  de la morph a été cablé avec un réseau de neurones artificiels qui peuvent être formattés avec des informations téléchargées. Cela permet à l'utilisateur de télécharger des skillsofts (p. 332) directement dans le cerveau, permettant d'utiliser la compétence logicielle jusqu'à ce que le skillsoft soit effacé ou remplacé. Les systèmes skillware ne sont capable de gérer que l'équivalent de 100 points de compétence en skillsoft à la fois. \textbf{[Élevé]} 

\textbf{Peau changeante:} Cet implant de déguisement permet à l'utilisateur de restructurer ses caractéristiques faciales et musculaire et de modifier son teint de peau et sa couleur de cheveux. Le processus complet prend 20 grosses minutes. La peau changeante ajoute +30 aux Tests de Déguisements. \textbf{[Modéré]} 

\textbf{Interface dermale:} Les nanomachines de l'interface dermalevivent dans l'épiderme externe de la morph oiu de la coque, se regroupant automatiquement et créant des connexions physiques avec tous les appareils électroniques que touche l'utilisateur. Cet interface exploite également le champ électrique de la peau d'une biomorph pour communiquer. Elle permet à l'utilisateur de communiquer et de mesher avec tous les appareils simplement en les touchants. Cela est considéré comme un lien filaire, et n'est donc pas sujet à l'interception ou aux interférences sans-fil. Deux personnes connectées par interface dermale peuvent également communiquer et se mesher par simple contact. \textbf{[Modéré]} 

\textbf{Outils de Poignet:} La morph possède des bandes métalliques de 6 centimètres de large implantées autour de chaque poignet et contenant des générateurs de nanomachines.  Ces nanomachine se connectent enemble pour dupliquer les fonction d'un outilitaire (p. 326) en créant de petitsbras flexible se terminant chacun en un outil spécialisé. Ces nanomachines peuvent également produire de toute petite fibresoptiques permettant au personnage de voir à travers de petitesouvertures, ainsi que d'être capable de créer de petites armes assimilables aux griffes bioware. Le fait que ces outils soient mentalement contrôlléesdonne au personnage un modificateur de +20 aux compétences impliquant la réparation ou la modification d'appareils ayant des partiesmécanique, l'ouverture de serrure ou la désactivation des systèmes d'alarmes, ainsi que l'exécution des gestes de premier secours. \textbf{[Modéré]} 



\subsection{Modifications cosmétiques} \label{sec:cosmetic-mods} 

A une ère de beauté universelle, les modifications artistiques corporelles sont courament effectuées par de nombreux transhumains. Les modificateurs corporelles autrefois considérées comme dangereuses ou excentrique sont maintenant sûre et courante, en particulier parmi les factions telles que les anarchistes, la racaille ou les bordés. 

\textbf{Sculpture corporelle:} Si les améliorations physique de votre morph ne vous suffisent pas, vous pouvez aller un cran au-dessus grâce aux sculpture corporelle telle que desoreilles ou des doigts allongés, des modifications du nez, l'ajout/la suppression de cheveux ou de poils, de plumes, d'yeux exotiques, d'une peau de seprent, de parties génitales exacerbée et toutes autres modifications physique plus inhabituelles. \textbf{[Bas]} 

\textbf{Nanotatouages:} Les tatouages créés avec des nanomachines peuvent se déplacer autour du corps, changer de forme, de couleur, de luminosité ou de texture, peuvent alterner textes et images et/ou créer dseffets holographiques mineurs à la surface de la peau, le tout étant contrôlable par des inserts de mesh. \textbf{[Bas]} 

\textbf{Piercings:} Choisissez n'importe quelle partie du corps et quelqu'un aura trouvé une manière de la piercer, probablement de nombreuses fois. Anneaus, piques, plugs et chaînes sont extrèmement courant, souvent fait à partir de matériaux intelligents changeants de forme. \textbf{[Trivial]} 

\textbf{Scarification:} Étant donné les capacité médicalemoderne, les cicatrices de toutes sortes sont purement cosmétiques et à peine une gène. \textbf{[Trivial]} 

\textbf{Modification d'Odeur:} Des changements mineurs à la biochimie d'un corps peuvent modifier l'odeur naturelle d'un personnage ou le parfumer constamment. \textbf{[Bas]} 

\textbf{Teinture Épidermique:} Des teintures de toutes formes et de toutes couleurs sont disponibles. \textbf{[Trivial]} 

\textbf{Implants Subdermiques:} Ajouter de petits implants sous la peau peut créer des bosses, des crêtes, des attaches pourpiercings et des altérations et texturs similaires. \textbf{[Trivial]} 



\subsection{Amélioration Robotique} \label{sec:robotic-enhancements} 

Les modifications suivantes ne sont disponibles que pour les synthmorphs/robots. 

\subsubsection{Armure} 

Ces modifications d'armure remplacent le niveau d'Armure inclut dans la synthmorph 

\textbf{Armure de Combat Lourde:} La structure de la synthmorph est chargée d'armure qui offrent une protection contre les armes lourdes pour les opérations de combat violent. La modification est lourde et visible; la structure du bot est enchassé dans une carapace épaisse. Cela élève l'Armure incluse dans le bot à 16/16. Les systèmes de mobilité et les sorties énergétiques de la coque sont égalment renforcés pour gérer avec cette charge supplémentaire. \textbf{[Élevé]} 

\textbf{Armure Industrielle:} La coque est équippée avec des protections contre les collisions, la météo extrême, les accidents industriels et tout accident de même type. Cela élève l'Armure incluse dans le bot à 10/10. [Modéré] Armure de Combat Légère: La structure de la synthmorph est protégées par une armure conçue pour les fonctions de police et de sécurité. Cela élève l'Armure incluse dans le bot à 14/12. \textbf{[Modéré]} 

\subsubsection{Systèmes de mobilité} 

Les coques sont conçues avec une large gamme de système de propulsion, et sont parfois construites pour un environnement ou une gravité spécifique. Certaines synthmorphs peuvent avoir de nombreux systèmes de mobilité. De tels systèmes sont rétractables, ils peuvent être repliés dans la structure de la coque. 

\textbf{Sauteur:} Lessauteurs ont deux jambes ou plus conçues pour propulser la morph en avant ou vers le haut, un peu comme les grenouilles ou lessauterelles. \textbf{[Modéré]} 

\textbf{Aéroglisseur:} La coque utilise un ventilateur pour envoyer un coussin d'air à haute pression vers le sol, propulsant la structure au-dessus du sol (les aéroglisseurs modernes n'utilisent plus de jupe en caoutchou) Laplupart des aéroglisseurs voyagent à environ un mètre du sol, mais ils peuvent temporairement léviter plus haut pendant de courtes périodes. \textbf{[Bas]} 

\textbf{Ioniseur:} La coque utilise les principes de la magnétohydrodynamique pour léviter et voler, en ionisant l'air alentour en plasma pour créer propulsion et moment. La coque tourne sur elle-même pour améliorer la stabilité. Ce système ne fonctionne pas dans le vide, mais une version sous-marine utilise les même mécanismes pour la propulsion en environnement liquide. \textbf{[Élevé]} 

\textbf{Ultra Léger:} Populaire dans lesenvironnements en faible ou enmicro gravité, les ultra léger regroupent différent type de systèmes ultra léger ou plus léger que l'air, telles que les parapentes propulsés, les autogyres, lesballons, les aérostats et les dirigeables. Ces systèmes ne fonctionnent pas dans le vide. [Bas] Rouleur: Ce système,disponible uniquement pour les coques circulaires, permettent à la synhtmorphs de rouler comme une balle. La coque roule autour d'un axe interne et est propulsée par un pendule motorisé. \textbf{[Modéré]} 

\textbf{Rotor:} Des hélices rotatives crééent une traction, permettant à la coque de se déplacer comme un hélicoptère. La plupart des modèles utilisent des rotors ou des ailes basculables pour que les hélices puissent être inclinéesvers l'avant (pour une propulsion plus rapide) et pour une meilleuremanœuvrabilité en zéro-G. Ce système ne fonctionne pas dans le vide. \textbf{[Bas]} 

\textbf{Serpent:} Couramment utilisés par lesslitheroïdes, ces coques utilisent l'ondulation latérale, la flexion de leur corps de gauche à droite et l'ondulation de leur structure vers l'avant. Ces coques peuvent également utiliser un balayage latéral ou un mouvement enaccordéon (s'étendre vers l'avant puis rétracter l'arrière) pour se déplacer. Elles ont aussi des stabilisateurs gyroscopiques pour qu'elles puissent s'enrouler en une boucle et rouler comme une roue. [Modéré] Sous-marin: Conçu pour la mobilité sub-aquatique, les coquessous-marines utilisent des turbines ou des jets d'eau compressée pour se déplacer dans l'eau. \textbf{[Modéré]} 

\textbf{Tenseurs:} Les coques avec des tenseurs utilisent des utilisent des treuils lintelligents pour se déplacer à travers des surfaces qui gèneraient les autres véhicules terrestres. Elles peuvent se propulser vers le haut afin de franchir de plus grands obstacle ou s'allonger pour créer un pont au-dessus d'un gouffre ou d'une crevasse. \textbf{[Bas]} 

\textbf{Poussée Vectorielle:} Ces coques utilisent des turbines pour créer une poussée atmosphérique et sont équippées d'un ensemble d'ailes. Lesmoteurs peuvent être manœuvrés pour orienter et générer une poussée dans différentes directions afin de permettre l'atterissage et le décollage vertical et pour améliorer la maniabilité en zéro-G. \textbf{[Modéré]} 

\textbf{Marcheur:} Les marcheurs utilisent deux jambes ou plus pour marcher ou ramper à travers une surface. Beaucoup d'entre eux utilisent des coussinets adhésifs (p. 305) ou des systèmes magnétiques (p. 310) pour coller aux surfaces. \textbf{[Bas]} 

\textbf{Roues:} Laplupart des coques à roues possèdent des essieux intelligents permettant aux roues d'adapter leur formes aux obstacles et même de pouvoir grimper des escaliers. Certaines coques pour faible gravité possèdent des pneus à air comprimé résistant à la crevaison et autoréparants. \textbf{[Bas]} 

\textbf{Ailé:} Principalement utilisés par de plus petites coques, ce système de quatre ailes indépendament contrôlées permettent à la coque de voler ou de sedéplacer rapidement dans toutes les directions. \textbf{[Bas]} 

\subsubsection{Modifications physique} 

Ces modifications sont appliqués à la structure physique de la coque. 

\textbf{Membres Supplémentaire:} La coque est équippée de membres supplémentaires. Un personnage utilisant ces membres souffrent du malus de mauvaise main (p. 193). Ces membres peuvent être des bras (avec des mains,des griffes ou autres), des jambes, des tentacules ou d'autres membres articulés et/ou préhensile. Certaines coques ont des structures rotatives qui leur permettent de déplacer leurs membres autour de leur corps. \textbf{[Bas]} 

\textbf{Doigts Fractals:} La synthmorph a des doigts en "buissons robotisés" qui sontcapable de se diviser en plusieurs doigts plus petits, et ces petits doigts peuvent sediviser en micro doigts, et ainsi de suite jusqu'à une échelle micrométrique, permettant des manipulations ultrafine. Appliquez un modificateur de +20 en COO lorsqu'une fine manipulation est un facteur (tels que du travail de réparation de précision). Le bot doit avoir une vision nanoscopique fonctionnelle (p. 311) pour obtenir ce bonus. \textbf{[Modéré]} 

\textbf{Compartiment Caché:} La coque possède une ouverture dissimulé vers un compartiment blindé interne, idéal pour stocker des valeurs ou pour faire passer en contrebande desobjets. Appliquez un modificateur de -30 pour détecter ce compartiment manuellement ou par capteurs. \textbf{[Bas]} 

\textbf{Système Magnétique:} Un système magnétique permet à la coque de s'accrocher à la plupart des matériau ferreux. Cela permet au personnage de marcher en situations zéro-G en adhérant magnétiquement aux surfaces, à s'accrocher la tête en bas, et à garder des appareils sans les laisser tomber ou dériver au loin. La coque reçoit unmodificateur de +30 lorsqu'il maintient une prisemagnétique sur quelque chose. \textbf{[Bas]} 

\textbf{Conception Modulaire:} Cette coque est conçue pour s'accrocher avec d'autres morphs modulaires dans différents motifs architecturaux et pour créer un ensemble plus gros. Lorsqu'unie avec d'autres module, le groupe est considéré comme une seule unité/morph, avec des capacités partagées. Si elle est endomagée puis séparées, les dégâtset les blessures sont distribuées équitablement entre tous les modules; les restes sont alloués aléatoirement. Les capacités exactes de différentes formesdépendent de sa composition, et sont largement laissées entre les mains du maître de jeu. \textbf{[Élevé]} 

\textbf{Membres Pneumatiques:} Les membres sont équipés avec des systèmes pneumatiques cylindriques qui peuvent générer jusqu'à 750 kilos de poussée. Cela permet à la coque de se propulser etde faire des sauts impressionants (une synth de taille et de poid humain peut franchir 2 mètres de hauteurs). Appliquez un modificateur de +20 aux Tests de Parkour. Un membre pneumatique utilisé pour frapper un adversaire en combat à mains nues inflige 1d10 de dégats supplémentaire. \textbf{[Bas]} 

\textbf{Membres Rétractable/Téléscopique:} Les membres de la coque peuvent soit être complètement rétractés à l'intérieur de la structure, soit être étendus au-delà deleur longueur normale (généralement pour 1 ou 2 mètres de plus). Les membres téléscopiques peuvent donner à la coque l'avantage de l'allonge en combat de mélée (p. 204). \textbf{[Bas]} 

\textbf{Changement de Forme:} Cette coque est construite dans des matériaux intelligents qui luipermettent de modifier sa forme, altérant sa taille, sa largeur, sa circonférence et ses caractéristiques extenre,tout en conservant la même masse. Cette modification est typiquement employée pour changer la forme de la morph dans des configurations particulières, adaptées à des tâches spécifiques (par exemple, s'allonger pour ramper dans un tunnel, élragir sa base pour se stabiliser, s'étendre pour atteindre et s'attacher à de multiples points d'accès, et ainsi de suite). Cette modification permet à la morph dechanger ses caractéristiques dans un but dé déguisement; appliquez un modificateur de +30 aux Tests de Déguisements. \textbf{[Élevé]} 

\textbf{Amélioration Structurelle:} Cette modification renforce l'intégrité structurelle de la coque, améliorant sa capacité à subir des dégâts. Augmentez la Solidité de 10 et le Seuil de Blessure de 2. \textbf{[Modéré]} 

\textbf{Structure en Essaim:} Lacoque n'est pas une seuleunité, mais un essaim de centaines de microdrônes de la taille d'un insecte. Chaque "insecte"individuel est capable deramper, de rouler, de sauter sur plusieurs mètres ou d'utiliser des pales de nanocoptère pour se propulser dans les airs. Le cybercerveau, les systèmes de capteurs et les implants sont distribués dans l'essaim. Même si l'essaim peu "fusionenr" en une forme grossière de la taille d'un enfant, elle est incapable de gérer des tâches physiques telle que attraper, tirer ou tenir un objet en tant qu'unité physique. Chaque insecte est cependant capable de s'intercafer avec des systèmes électroniques. L'essaim ne peut pas porter la plupart du matériel ni porter une armure, et ne peutr pas faire de tests de compétences liés à SOM et basés sur la force. Dans une optique de combat, utilisez les mêmes règles que celles des nuées de nanites,p. 328. Les dégâts et les blessures sont reflétés par des insectes endommagés/massacrés. La nuée peut être "soignée" en fabricant d'autres insectes. \textbf{[Élevé]} 

\textbf{Masque Synthétique:} La synthmorph est équippée avec un habillage réaliste de fausse peau et est minutieusement sculptée pour passer pour une biomorph (et éventuellement pour une personne particulière). La morph peut pleurer, cracher, avoir des rapports sexuelset saignersi elle est coupée. Seule un examen phgysique détaillé ou une analyse par radar, rayons T ou rayons X détectera la nature de la morph, et même ainsi de tels scans subissent un modificateur de -30. [Modéré] Monture d'Armes: La coque possède des systèmes d'armes embarqués. Cette monture d'arme peut aussi bien être interne (dissimulée, seules les armes de petites tailles pour la coque peuvent être adaptée, -30 aux Tests de Perception pour les détecter) ou externe (visible). Elle peut-être fixe (une seule direction), rotative (champ de tir limité) ou sur une monture articulée (toute les directions de tir possible). \textbf{[Bas; Modéré pour les dissimulées/articulées]} 

\subsubsection{Capteurs} 

\textbf{{Vision à 360$^\circ}$:} Les capteurs visuels de la morphs sont orientés pour un champ visuel de 360 degrés. \textbf{[Bas]} 

\textbf{Renifleur Chimique:} Ce capteur détecte les molécules dans l'air et analyse leur composition chimique. Il permet de faire des Tests de Chimie pour déterminer la présence de gazs, incluant les toxines et autres vapeurs. Il peut également détecter la présence d'explosifs ou d'armes à feu. \textbf{[Modéré]} 

\textbf{Lidar:} Ce capetur émets des faisceaux laser et mesure leur réfléexion pour évaluer la distance, la vitesse et pour réaliser une image de la cible. Voir Utilisation des Sens Améliorés, p. 302. \textbf{[Bas]} 

\textbf{Vision Nanoscopique:} Les senseurs visuels de la coque peuvent faire le point comme unmicroscope, en utilisant des superlentille évoluées pour annuler les limites de la diffraction optique et rerésentyer des objets de taille nanométriques. Cela permet au personnage de voir et analyser des objets aussi petits que les cellules sanguines et de distinguer les nanomachines individuels. La synthmorph doit rester relativement stable pour voire les objets de cette taille. \textbf{[Modéré]} 

\textbf{Radar:} Ce système de capteurs envoie des ondes radio ou des microondes sur une cible et mesure les ondes réfléchient pour évaluer la taille, la composition et le mouvement de la cible. Voir Utilisation des Sens Améliorés, p. 302. \textbf{[Bas]} 

\subsection{Armure} \label{sec:armor} 

Les systèmes d'armures personnelles modernes ont évolués à partir des thermoplastiques en polyéthilène à haute modulation et des tissus aramides du début du 21° siècle. Les armures dans \emph{Eclipse Phase} sont dérivées de la biotechnologie, sous la forme de fibres de tissus organiques et de plaques poussant sur des crystaux, et de la nanotechnologie, sous la forme de fullérène (p. 298) absorbant les chocs. Occasionnelement d'autres matériaux peuvent être utilisés, comme des plaques de verres métalliques ou des fluides non newtoniens qui se solidifient à l'impact. De telles armures protègent contre les balles (perces armures) et les impacts cinétiques ainsi que contre les armes tranchantes et les objets pointus. Elles incluent également une protection contre la chaleur explosive des armes à énergie et contre les chocs électriques. Alors que de telles armures protègent contre les balles, les couches de matériaux prenant la balle et redistribuant son énergie cinétique sur l'ensemble du corps, qui peut toujours amener à de sérieux ématomes et à des traumas. 

Les règles pour l'armure en combat peuvent être trouvées p. 194. Les exosquellette blindés sont listés à la p. 343. 

\textbf{Vêtement Renforcés:} Les fibres organiques renforcés et les matériau fullérènes offrent une protection de base contre les armes cinétiques ou à énergie peuvent être tissées avec des matériaux intelligents standard pour créer une large gamme de vêtements renforcés discrets fournissant un niveau de sécurité correct. De tels orenemnts de protection ne sont pas distinguable des vêtements normaux et sont disponibles dans de nombreux styles. Les vêtements renforcés fournissent une Valeur d'Armure de 3/4. \textbf{[Trivial]} 

\textbf{Veste Blindée:} Les vestes blindés fournissent une protection plus solide à des zones vitales du corps, couvrant entièrement l'abdomen et le torse,protégeant la nuque avec un collier rigide et fournissant même une protection pelvienne. Bien que les vestes blindées ne soient pas anguleuse, elles sont perçues comme des armures. Les vestes blindées peuvent être portées avec des vêtements renforcés sans pénalités. Elles fournissent une valeur d'Armure de 6/6. \textbf{[Bas]} 

\textbf{Armure Corporelle (Légère):} Ces combinaisons d'armure à haute performance protègent le porteur des pieds à la tête. Une veste blindée intégrés est supplée avec une protection améliorée sur les membres et les articulations, tout en parvenant à être flexible et non-contraignant. Les armures corporelles sont généralement portée par ls forces de police et de sécurité, et sont associées à un casque. Les vêtements renforcés fournissent une Valeur d'Armure de 10/10. \textbf{[Bas]} 

\textbf{Armure Corporelle (Lourde):} Similaire à une armure corporelle légère, mais avec des couches de protection similaire, souvent argonomiquement assemblées pour se conformer au corps d'un personnage spécifique, et avec une isolation environnementale et d'un contrôleur climatique pour protéger le porteur des environnements hostiles. Les vêtements renforcés fournissent une Valeur d'Armure de 13/13. \textbf{[Modéré]} 

\textbf{Combinaison de Crash:} Conçues à la fois pour la sécurité sur les lieux de travail et pour la protection des collisions accidentelles en zéro-G, les combinaisons de crash sont également appréciées des sportifs et des explorateurs. Les combinaisons de sauts de base disposent de protection comfortable de ce type. Lorsqu'elles sont activées avec des signaux électroniques, des polymères élastiques à l'intérieurs de la combinaison se rigidifient et forment une protection à l'impact des zones vitales. Les combinaisons de crash fournissent une Valeur d'Armure de 3/4 lorsqu'elles sont inactives et de 4/6 lorsqu'elles sont activées. \textbf{[Bas]} 

\textbf{Casque:} Cet accessoire d'armure est généralement porté avec une armure corporelle ou une combinaison de bataille. Les casques léger sont ouvert, alors que les casques complets se verouillent et fournissent un joint environnemental avec une réserve d'air de 12 heures. Les casques légers fournissent un bonus de +2/+2 à la  Valeur d'Armure,alors que les casques complets ajoutent un +3/+3. Les casques sont souvent équippés avec un ecto (p. 325), un booster  radio (p. 313), et des capteurs équivalents à des poussières (see p. 325). \textbf{[Trivial]} 

\textbf{Bouclier Anti-Émeute:} Utilisé pour le contrôle de foule, les boucliers anti-émeutes sont légers, solides et peuvent être configurés pour s'électrifier à la demande, étourdissant tout ceux qui sont en contact avec la surface externe (traitez ça comme des effets de gants à chocs, p. 334). Les boucliers anti-émeutes fournissent un bonus de +3/+2 à la Valeur d'Armure. \textbf{[Bas]} 

\textbf{Seconde Peau:} Cette combinaison corporelle légère, tissée avec des soies d'araignées et des fullérènes, est généralement portée comme une sous-couche,bien que certains athlètes l'utilisent comme uniforme. Elle fournit une protection minimale, mais elle ne peut-être portée avec d'autres armure sans pénalité. Elle fournit une Valeur d'Armure de 1/3. \textbf{[Bas]} 

\textbf{Peau Intelligente:} La peau intelligente est un nanofluide avancé qui couvre la peau du porteur. Il ressemble à du mercure liquide mais conserve la texture et la souplesse de la peau normale jusqu'à ce qu'elle soit activée, moment oùles matériau deviennent suffisament rigides pour prtéger le porteur et distribuer l'énergie cinétqiue (bien que toujours suffisament souple aux articulations pour ne pas géner le mouvement). Une ruche spécialisée, portée par le personnage, maintiens les nanomachines et les stockent lorsqu'ils ne sont pas disponibles. Déployer les nanomachines sur tout le corps prend un Tour d'Action complet. La peau intelligente a une Valeur d'Armure de 3/2 et peut être portée sans pénalité avec d'autres armures. \textbf{[Bas]} 

\textbf{Armure en Spray:} Cette application rapide d'armure est fourni sous forme d'aérosol et se disperse comme un polymère chimique intelligent qui se colle à la chair nue (mais n'adhère ni aux yeux, ni aux cheveux). Le polymère se solidifie en une armure corporelle moulante lorsqu'il est exposé à une température corporelle avec l'apparence et la sensation d'une combinaison latex. L'armure en spray ne fonctionne pas sur les morphs synthétiques, ni sur les vêtements ou d'autres armures. La couleur et la sensation de l'armure peuvent être ajustée avec des courants élmectroniques et des polymères additionnels, la rendant populaire parmi les célébrité et dans le milieu de la scène nocturne. L'armure en spray ne se lave pas, mais se dégrade au rythme d'1 pont d'armure (cinétique et énergétique) toutes les 12 heures. Elle peut être enlevée avec un solvant nanotechnologique spécial. L'armure en spray a une Valeur d'Armure de 2/2. \textbf{[Bas]} 

\begin{table} \begin{tabular}{|l|l|l|l|} \hline

\multicolumn{2}{|c|}{\textbf{Valeurs d'armures}}	\\ \hline

\textbf{Armure}	&\textbf{Énergétique} &\textbf{Cinétique}	&\textbf{Page} \\ \hline

Vêtement Renforcés	&3	&4	&311	\\ \hline

Veste Blindée	&6	&6	&312	\\ \hline

Armure de Combat Propulsée par Exosquelette 	&18	&18	&344	\\ \hline

Armure Biotissée (Légère)	&2	&3	&302	\\ \hline

Armure Biotissée (Lourde)	&3	&4	&302	\\ \hline

Armure Corporelle (Légère)	&10	&10	&312	\\ \hline

Armure Corporelle (Lourde)	&13	&13	&312	\\ \hline

Armure Carapace	&11	&11	&303	\\ \hline

Combinaison de Crash (Inactive)	&3	&4	&312	\\ \hline

Combinaison de Crash (Active)	&4	&6	&312	\\ \hline

Exoarmure	&2	&4	&344	\\ \hline

Combinaison Renforcée	&15	&15	&334	\\ \hline

Casque (Léger)	&+2	&+2	&312	\\ \hline

Casque (Complet)	&+3	&+3	&312	\\ \hline

Exosquelette Hyperdense	&6	&12	&344	\\ \hline

Bouclier Anti-Émeute	&+3	&+2	&312	\\ \hline

Seconde Peau	&1	&3	&312	\\ \hline

Peau Intelligente	&3	&2	&312	\\ \hline

Vêtement de Vide Intelligent	&2	&4	&325	\\ \hline

Armure en Spray	&2	&2	&312	\\ \hline

Armure Industrielle de Synthmorph	&10	&10	&310	\\ \hline

Armure de Combat de Synthmorph (Légère)	&14	&12	&310	\\ \hline

Armure de Combat de Synthmorph (Lourde)	&16	&16	&310	\\ \hline

Exosquelette de Transport	&2	&4	&344	\\ \hline

Exosquelette de Voyage	&2	&4	&344	\\ \hline

Exocombinaison (Légère)	&5	&5	&333	\\ \hline

Exocombinaison (Standard)	&7	&7	&333	\\ \hline

\end{tabular} \label{tab:armor-values} \end{table} 



\subsection{Modification d'armure} \label{sec:armor-mods} 

Les modifications d'armure ajoutent des matériau ou revêtements supplémentaires qui améliorent la résistance de l'armure à certains dangers ou qui fournissent d'autres effets. Les modifications d'armures peuvent être facilement ajoutés ou supprimés avec l'applicateurs de nanomachines adéquat. 

\textbf{Patchs Ablatifs:} Ces patchs autocollant, fins et légers sont conçus pourabsorber lachaleur et l'énergie des rayons et explosions, se vaporisant et envoyant les gazs chauds au loin. Les patchs ablattifs augmentent la Valeur d'Armure par +4/+2, mais chaque touche réduit la valeur d'armure éergétique et cinétique dupatch de 1. \textbf{[Trivial]} 

\textbf{Revêtement Caméléion:} Cela fournoit à l'armure les mêmes effets qu'une cape caméléon (p. 315). \textbf{[Trivial]} 

\textbf{Traitement Ignifuge:} Le traitement ignifuge inclut l'ajout de céramique résistante à la chaleur ou de couches résistants au feu, toutes deux capables de supporter des température extrêmement élevées. Le traitementignifuge augmente la Valeur d'Armure de +2/+0 et fournit 10 points d'armure additionelle contre la chaleur ou le feu. \textbf{[Trivial]} 

\textbf{Système Immunogène:} La modification immunogénique ajoute un essaim de nanomachine active,maintenues parune ruche spécialisées, qui recouvre la couche externe de l'armure ainsi que les parties non-armurées de la morph du porteur. Elle agît comme un système immunitaire externe conçu pour neutraliser les agents toxiques et les nanotoxines avec lesquels elle entre en contact. Cela fournit uneimmunité aux drogues,toxines et nanotoxines devant être appliqués sur la peau, par exemple par un patch ou une éclaboussure. Elle n'a aucun effet sur les drogues inhalés,absorbées ou injectées (y compris les armes enrobées). \textbf{[Bas]} 

\textbf{Revêtement Lotus:} L'armure a été imprégnées avec un revêtement superhyrophique (avec un angle de contact d'approximativement 170$^{\circ}$) qui repousse tous les liquides. Si l'armure est éclaboussée par une toxine liquide ou par des produits chimiques, l'effet est réduit puisque les liquides glissent hors de l'armure. Appliquez un modificateur de +30 lors de la défense contre des attaques basées sur des liquides. \textbf{[Trivial]} 

Armure Offensive: Lorsqu'activée, la couche externe de cette armure est cablée pour envoyer une décharge électrique à qui ou quoi que ce soit entrant en contact. Coinsidérez sa VD et ses effets comme ceux d'un bâton de chocs (p. 334). \textbf{[Bas]} 

\textbf{Revêtement Réactifs:} Une fine couche de nanotechnologie avancée est appliquée à l'armure, la protégeant avec une colonnie de nanomcahinesconçus pour percevoir les attaques en approche. Lorsqu'une attaque frappe le revêtement, il détonne pour interrompre l'attaque. Les rafales et les tir automatiques sont considérés comme uneattaque simple. Un revêtement réactif augmente la Valeur d'Armure de +5/+5, mais chaque détonation inflige automatiquement 1points de dégats au porteur. Les armures réactives fonctionnentaussi contre les attaques de mélée et infligentalors 1d10 $\div$ 2 (arrondissez au supérieur) points de dégats par attaque (l'armure protège de cesdégats) à cause des microexplosions. Le revêtement réactif nefonctionne que contre 5 attaque, après quoi la ruche de nanomachine spécialisée remplit le revêtements au rythme de 1 utilisation par heure. \textbf{[Modéré]} 

\textbf{Mirroirs Réfractifs:} Une combinaison de réflecteurs, de métamatériaux réfractifs et d'un système de transfert d'énergie avec des radiateursthermiques fournit une protection supplémentaire contre les armes à énergie. Augmentez la Valeur d'Armure de +3/+0. \textbf{[Bas]} 

\textbf{AUto-Réparante:} L'armure est équippé avec une ruche de nanites qui agissentr commeun spray réparant (p. 333). \textbf{[Modéré]} 

\textbf{Résistance aux Décharge:} Les armures résistantes aux décharges soint életroniquement isolées auxdéchargeset réduisent l'effets desarmes à chocs. Appliquez un modificateur additionel de +10 lorsque vousrésistez aux VD et aux effets desarmes à chocs (p. 204). \textbf{[Bas]} 

\textbf{Amortisseur Thermique:} L'amortisseur thermique obscurcit les signatures thermiques en convertissant la chaleur corporelle en énergie électrique. Il rend la cible plus difficile a détecter avec des capteurs thermiques: appliquez un modificateur de -30 aux Tests de Perception. \textbf{[Modéré]} 

\subsection{Communications} \label{sec:communications} 

La plus vieille technologie de communication et la plus répandue est la radio. Chaque habitat et monde habité par la transhumanité est baigné dans un traffic radio, les humains, les machines et les élevés communiquant les uns avec les autres constament. Les plus petites radios ne sonot pas plus grosse qu'un grain de poussière et ont une portée inférieure à 20 mètres, alors que les plus grosses sont de la taille d'un camion et ont une portée de plusieurs milliers de kilomètres. Peu importe la taille des radios, elles sont omniprésentes et la plupart des appareils contiennent au moins des raidos à courte portée pour qu'ils puissent intergair avec le mesh. La plupart des morphs sont équipés d'insert de mesh basique (p. 300) qui incluent une radio implantée. Pour les portées de la radio, voir p. 296. 

\textbf{Câble en Fibre Optique:} Les câbles de fibre optique sont utilisés pour établir une connexions filaire entre deux appareils. Étant donné l'omniprésence des radios et l'entremèlement de câbles que les liaisons filaires peuvent causer, ils ne sont généralement utilisés que pour des besoins d'intimité (contrairement aux communications radios, les signaux sur fibre optique ne peuvent pas être interceptés) ou dans des zones subissant de fortes interférences radios. \textbf{[Trivial]} 

\textbf{Lien Laser/Micro Onde:} Ces apapreils portables sont utilisés pour établir un canal de communication avec un autre lien de même type avec une ligne de vue. La portée de ces transmetteur varient largement en fonction des facteurs environnementaux mais est approximativement égale à 50 kilomètres dans l'atmosphère et à 500 kilomètres dans l'espace (bien que la ligne d'horizon doive être gardée à l'esprit, elle s'étend à 5 kilomètre au niveau du sol sur Terre et est plus courte sur les corps plus petits). Les lasers sont sujets aux interférence venant du brouillard, de la poussière, de la fumée et d'autres gènes visuelle, alors que les micro-ondes sont génées par les obstructions métalliques. Ces liens ne peuvent être interceptés qu'en se plaçant directement sur le rayon. Certaines équipes embarquent une version miniature de cee système, permettant des communications à vue intra-équipe ne pouvant être interceptées comme la radio. \textbf{[Modéré]} 

\textbf{Radio Booster:} This device boosts the range and sensitivity of short-range radios, like those from implants, ectos, or microbugs. The booster must be with the shorter-ranged device’s range (or directly linked via fiberoptic cable). It will repeat any transmissions received from that device, but at its extended range of 25 kilometers in urban areas (250 kilometers remote areas). Broadcasts from a radio booster are easy to receive by anyone looking for broadcasts (see Wireless Scanning, p. 251), though transmissions may be stealthed (p. 252). Boosters are commonly used by characters traveling far from habitats or other civilized regions. \textbf{[Bas]} 



\subsection{Neutrino communicators} \label{sec:neutrino-communicators} 

Neutrinos are particles that can pass through any solid matter with ease and are impossible to block. As a result, they make an ideal medium for communications. Unfortunately, they are also easy to intercept. Even a tight beam of neutrinos sent between two locations can be intercepted simply by placing another receiver behind the location the broadcaster is sending to. Neutrino communicators require a large power plant to power the high energy particle interactions required to generate the neutrino broadcast. Neutrino receivers are also relatively large, with the smallest occupying 100 cubic meters. In most cases, neutrino communicators are designed to broadcast neutrinos in all directions, though tight-beam transmissions are also possible. Quite often neutrino communications take advantage of quantum farcasting for security. 

\textbf{Neutrino Transceiver:} This transceiver is capable of generating and receiving neutrino signals at a range of at least 100 astronomical units. It is large, with a size of 8 cubic meters (in a cube 2 meters on a side), but they can be loaded onto large vehicles. To function, it must be connected to a large power plant, such as one found in habitats or large spacecraft. The cost and size of this device includes the computer necessary for quantum farcasting. \textbf{[Cher]} 

\subsection{Quantum farcasters} \label{sec:quantum-farcasters} 

Quantum farcasters are special computers designed to protect a communications channel (such as fiberoptic, radio, laser/microwave, or neutrino) with unbreakable encryption. To function, two or more quantum farcaster computers must first be entangled together (on a quantum level) in the same physical location. The farcasters may then be separated, at which point they may continue to exchange encrypted data via quantum teleportation. This data exchange requires a standard communications link (fiberoptic, radio, laser/microwave, or neutrino), and so is limited by the speed of light, but it is a high bandwidth form of communications. The quantum encryption used by these entangled farcasters is unbreakable, and any attempted interception is immediately detected and neutralized. A quantum farcaster may not be used to securely communicate with any farcasters other than the ones it is entangled with. 

Because it is exceptionally safe and secure, quantum farcasting via neutrino communications is the primary means of both long-distance communication between habitats and egocasting (p. 276). The neutrino signal cannot be blocked and it can only be decrypted if a character has access to the computer that is sending or receiving the signal. 

\textbf{Miniature Radio Farcaster:} Miniature farcasters communicate with each other using standard radio transceivers. As noted above, they may only securely communicate with the other farcasters with which they are entangled. Most miniature farcasters are worn as jewelry or fitted into clothing or other equipment. \textbf{[Bas]} 



\subsection{Quantum entanglement communication} \label{sec:quantum-entanglement-communication} 

The rarest form of communications is quantum entangled (QE) communication. QE communication is instantaneous and works over any distance, but is also very limited. QE communication requires pairs of entangled particles known as qubits. To use QE, large number of pairs of qubits are created and then separated from each other. Millions of these separated pairs of particles are stored in special containers known as qubit reservoirs. If two QE communicators each have a qubit reservoir containing qubits that are each entangled with qubits in the other communicator’s qubit reservoir, then characters can use the two QE communicators to commutate with one another instantaneously. Characters can use QE to instantly communicate between any two locations, even if one character is in the solar system and the other has passed through a Pandora gate and is standing on a planet 500 light years away. 

Each bit of data transmitted between these two QE comms uses up one qubit. Once all of the qubits are used up, the two QE comms can no longer communicate with each other until they each get a new batch of entangled qubits. Qubits are expensive to produce, contain, and transport, making this an exceedingly expensive form of communication. As a result, extremely high bandwidth communications like full sensory AR and egocasting cannot be performed using QE communication. 

\textbf{Portable QE Comm:} This is a handheld FTL communications device. The actual communications unit can be made as small as desired, but must be large enough to connect to or hold a qubit reservoir. Because qubit reservoirs are relatively large and must be replaced, they are rarely implanted. Some miniature farcasters are designed so that users can also attach qubit reservoirs to enable them to be used for both light speed and FTL communication. \textbf{[Bas]} 

\textbf{Low-Capacity Qubit Reservoir:} Low-capacity qubit reservoirs can be used for 10 hours of high-resolution video conferencing or meshbrowsing and 100 hours of voice or text only communications. \textbf{[Élevé]} 

High-Capacity Qubit Reservoir: High-capacity qubit reservoirs can be used for 100 hours of high-resolution video conferencing or meshbrowsing and 1,000 hours of voice or text only communications. \textbf{[Cher]} 



\subsection{Covert and espionage technologies} \label{sec:covert-espionage-tech} 

These technologies allow characters to acquire protected information and to gain access to places that others try to keep them out of. Many of these devices are mesh-capable and equipped with radios, see p. 296 for radio ranges. 

\textbf{Chameleon Cloak:} This loose, poncho-like cloak contains a network of sensors that perceive wavelengths from microwave to ultra-violet. A similar network of miniature emitters precisely replicate the information its sensors receive, making the wearer seem transparent to those wavelengths. A chameleon cloak allows a character to effectively become invisible as long as they are stationary or not moving faster than a slow walk. When worn by someone moving faster, the cloak still provides a +30 modifier to Infiltration Tests to avoid being seen or noticed. 

Chameleon cloaks are not effective against radar, x-ray, or gamma-ray sensors. They do hide the character from thermal infrared, however, by absorbing the character’s body heat into its heat sink. The cloak can only absorb a character’s body heat for one hour before it must emit this heat. Heat emission also requires one hour, during which time the character is easily visible in the thermal infrared spectrum. \textbf{[Bas]} 

\textbf{Covert Operations Tool (COT):} This handheld device is the ultimate in infiltration technology. It contains both smart matter micromanipulators, cutting tools, and an advanced nanotechnology generator capable of producing nanobots that can bore or cut through almost any material and disable or open almost any electronic lock. 

Cutting out a lock or boring a 1-millimeter hole in a wall with a COT requires ((Durability + Armor) $\div$ 10) seconds. Cutting out a 1-meter diameter hole in a wall requires ((Durability + Armor) $\div$ 10) minutes. These same nanobots can later be used to repair this damage so that it is invisible to any but the most careful and detailed examination. 

A COT can easily open any old-fashioned mechanical lock simply by analyzing it and shaping an appropriate key, though this takes a full Action Turn. It can also open electronic locks by infiltrating them with nanobots that influence the lock’s electronics, no matter what authentication system the lock uses. Opening electronic locks takes a full Action Turn, but success is practically guaranteed. Opening an electronic lock in this manner will, however, trigger an alarm and/or be logged as an event. For more details, see \emph{Electronic Locks}, p. 291. \textbf{[Élevé]} 

\textbf{Cuffband:} This smart plastic loop restricts around a prisoner’s limbs when activated. If the prisoner struggles, it will tighten more. Cuffbands will inform the user if they are cut or loosened and are electronically- controlled, so the user can release the prisoner remotely. Some cuffband variants including a shock system (treat as a shock baton, p. 334) to zap and restrain unruly prisoners. \textbf{[Bas]} 

\textbf{Dazzler:} The dazzler is a tiny laser system set on a rotating ball. When activated, it consistently spins and emits laser pulses in all directions. These laser pulses are not dangerous, but they detect the lenses of camera systems (including specs, viewers, and bot/ synthmorph sensors) and repeatedly zap them with laser pulses of varying strength to overload and dazzle them. For as long as a dazzler is active, any camera system (visual, infrared, and ultraviolet) within line of sight and within 200 meters is blinded. \textbf{[Modéré]} 

\textbf{Disabler:} This handy device emits an overloading surge that completely incapacitates and disables a synthetic morph or pod (anything with a cyberbrain) when it is plugged into an access jack and activated. The affected cyberbrain will be unable to function until the signal is deactivated, effectively shutting down the ego (or AI). In order to plug a disabler into an unwilling target, the target must first be grappled or a called shot must be successfully made in melee combat. This device does not work on larger synthetic morphs (like vehicles) or on cyberbrainless robots. \textbf{[Élevé]} 

\textbf{Fiber Eye:} This is a flexible and electronically-controllable length of fiberoptic cable and viewer, which can be worked through cracks, under doors, and around corners to peep unobtrusively. \textbf{[Bas]} 

\textbf{Invisibility Cloak:} This cloak is made of metamaterials with a negative refractive index, so that light actually bends around it, making it and anything it covers invisible. This invisibility works from the microwave to ultraviolet spectrums, but not against radar or x-rays. The drawback is that anything concealed within the cloak can’t see out. This is easily overcome by using external sensor feeds (if available) and entoptics to navigate. Alternately, a small piece of anti-cloak, which cancels the cloak’s invisibility properties when touched together, can be used to create a small window to peep out of, though this increases the chance of being spotted. Noticing such a window requires a Perception Test with a $-$30 modifier. \textbf{[Élevé]} 

\textbf{Microbug:} This device is a tiny camera and microphone 1 millimeter across. It has the visual capabilities of a set of specs (p. 325). It can hear everything within 20 meters and see everything within the same range that is in its line of sight. A microbug can record up to 100 hours of information. Microbugs can be set to broadcast continuously, at set intervals, or only when they receive a special signal. If desired, they can also be set to only record if there is movement or voices in the room they are in. Microbugs have adhesive backs and can stick to almost any surface. Microbugs can also establish their location via mesh positioning or GPS, and so double as tracking devices. To avoid being detected by their radio transmissions, some microbugs are attached to miniature quantum farcasters (p. 314). These microbugs are much larger (1 centimeter) and easy to see, but their transmissions cannot be detected or blocked. \textbf{[Trivial, Low for quantum farcaster bugs]} 

\textbf{Prisoner Mask: }This hood tightens around the head of a prisoner, blocks all vision frequencies, and engages in low-level jamming in order to prevent any wireless communication via mesh inserts. \textbf{[Medium]} 

\textbf{Psi Jammer:} This device jams frequencies used by brainwaves within a 20-meter radius. This has no effect on brain functions, but it does prevent any ranged used of psi sleights within this area of effect. \textbf{[Modéré]} 

\textbf{Quantum Computer:} These advanced devices make use of quantum computation, allowing them to handle extremely large numbers with ease. This makes them especially useful for codebreaking, as noted on p. 254. \textbf{[Cher]} 

\textbf{Smart Dust:} This device is a walnut-sized specialized nanobot generator that creates tiny sensor nanobots, each one of which is a tiny sphere the diameter of a human hair. A packet of smart dust nanobots is sufficient to perform detailed surveillance on a large room like an auditorium has a volume of 1 cubic centimeter and contains 3 million nanobots. Each nanobot contains tiny cameras, microphones, a tiny computer, a radio, and chemical sensors, as well as short legs that allow them to walk and climb at a rate of 5 cm per second. 

When a character dumps a packet of smart dust in a room, it will cover every surface in the room within 20 minutes, including all furniture and the insides of every drawer and other space that is not airtight. At this point, the smart dust has recorded all data about the room that can be obtained by exceedingly detailed observation, including the DNA of everyone who has visited the room in the last week or two. The smart dust can then either broadcast a brief, highly compressed signal, or it can send all of its information to a few hundred nanobots that then walk to a pre-arranged destination for pickup and downloading by their user. The user need only find a single nanobot with a nanodetector to acquire the information obtained by the smart dust. If ordered to do so, the remaining nanobots can either power down and await further orders or self-destruct in a fashion that turns them into a tiny amount of dust made mostly of metal and silicon. \textbf{[Moderate]} 

\textbf{Traction Pads:} This set of specialized fingerless gloves, shoes, and kneepads is designed to emulate the pads on geckos’ feet. Characters can support themselves on a wall or ceiling by placing any two of these pads against any surface not made from a material specially designed to resist such devices. Characters can climb any surface and move easily across walls and ceilings that can support their weight (+30 to Climbing Tests). In addition to climbing, these devices are also very popular in zero-g environments. Wearing this item does not impair the user’s agility or manual dexterity. \textbf{[Bas]} 

\textbf{White Noise Machine:} This small and wearable device generates masking sounds that protect a conversation from being audibly recorded or overheard by anyone not in the immediate vicinity. \textbf{[Trivial]} 

\textbf{X-Ray Emitter:} This device is designed to be used with either the enhanced vision augmentation (p. 301) or specs (p. 325). It emits a focused beam of low-powered x-rays that allows the user of either device to both see and see through most objects using backscatter x-ray radiation (p. 303). This allows the character to literally see through walls and into containers, including ones made of metal. \textbf{[Bas]} 

\subsection{Bugs and surveilance} \label{sec:bugs-surveilance} 

Though surveillance technologies are pervasive and easy to come by in Eclipse Phase, secretly obtaining information on someone who wants to retain privacy can be quite difficult. Microbugs, smart dust, and similar recording devices that are all but invisible may be exceptionally easy to put into place, but once they begin actively transmitting, they are easy to to detect (see Wireless Scanning, p. 251). An eavesdropper may attempt to stealth the signal (see Stealthed Signals, p. 252), but this is not guaranteed to work. Once a signal is detected, locating the broadcasting device is usually just a matter of time (see \textit{Tracking}, p. 251). 

Some recording devices attempt to avoid this problem by using miniature quantum farcasters (p. 314), but those are far larger and more difficult to hide. Often the most effective way to acquire discrete information is to plant a surveillance device, set to record but not transmit, and then retrieve it later. While doing this is often difficult and risky, the recording device never reveals its presence by broadcasting and so is more difficult to detect. 

\section{Drugs, chemicals and toxins} \label{sec:drugs-chemicals-toxins} 

In \emph{Eclipse Phase}, the transhuman desire to enhance the body and mind --- especially with chemicals --- merges right into humanity’s popular pastime of recreational substance abuse. Drugs of all kinds, whether they be chemical, nano-based, or electronic, are not only popular but widespread. While advances in biotechnology have eliminated many of the side effects that once plagued drug users, transhuman bodies remain complicated environments, and so side effects (especially with long-term use) are still a factor. Additionally, addiction is always a consideration for anyone who gets comfortable with popping the same pills too often, though there are also drugs for addiction of course. 

Drug descriptions include benefits, side effects, noticeable signs that a person is using the drug, addictiveness, and effects from long-term use). Descriptions also include the drug’s Duration and its Addiction Modifier (see \emph{Addiction and Substance Abuse}). 

\subsection{Substance rules} \label{sec:substance-rules} 

These rules explain how to handle drugs and toxins. 

\subsubsection{Classification of substances} 

Substances fall into four categories: 

\textbf{Chemicals:} These are pharmacologically-active small chemical compounds (toxins, pharmaceuticals, chemical drugs) that have been produced by chemical synthesis, nanotech fabrication, or enzymatic biosynthesis in (transgenic) organisms. They include naturally- occurring drugs from known species of (exo-)flora and fauna, endotoxins produced by biological organisms, enhancements of endogenic substances (designer drugs), and de novo developments designed for a specific medical or recreational application. Chemical drugs affect only biological morphs and pods. 

\textbf{Biologicals:} These include peptides, hormones, and biologically-based substances like biotoxins, bacteria, and viral organisms --- drugs devised or based on naturally-occurring endogenic biological substances. This category also includes infectious biological organisms that can produce drug-like effects, like virii and bacteria. Biologicals affect biomorphs and pods but not synthetic morphs or infomorphs. 

\textbf{Nanodrugs:} These are temporary nanobot colonies programmed to create a certain effect. While nanobots are generally able to target or infect all morph types except infomorphs, exactly which morphs are affected usually depends on the pre-programmed effect (i.e., whether it targets a biological or mechanical mechanism). 

\textbf{Electronic:} Electronic drugs include software and technology that affect the brain directly, such as manipulative XP programs or retro-tech like transcranial magnetic stimulation or cranial electrotherapy. It also includes narcoalgorithms --- programs that reproduce drug-like effects for AIs, infomorphs, and egos residing in cyberbrains. 

\subsubsection{Application methods} 

There are number of vectors by which a substance may be applied to a morph. 

\textbf{Dermal (D):} This drug or toxin is absorbed via the skin (or exterior hull with some nanotoxins) as either a gas, liquid, or solid (e.g., paste). Slap patches and slap bands are commonly used, loaded with the chemical DMSO, which transfers the drug through the skin. 

\textbf{Inhalation (INH):} This is a gas that is breathed into the lungs or snorted nasally. Used for inhalers, aerosols, powders, and gas grenades/seekers. 

\textbf{Injected (INJ):} This liquid is applied via either an intramuscular or intravenous injection. Used for needles and piercing weapons. 

\textbf{Oral (O):} This is a liquid or solid that is absorbed through the stomach or oral cavity (eating or drinking). Used with pills and liquids. 

\subsubsection{Drug effects} 

If a character is exposed to a drug via its method of application --- for example, they pop a pill, slap on a dermal patch, are soaked with a splash grenade, breathe in gas, or get stabbed with a coated weapon --- then they are subject to the drug’s effects. The onset time determines how long these effects take to kick in, and the duration determines how long they last. While there is no resistance test to ignore a drug or toxin’s effects once exposed, in some cases (especially toxins) a test might be called for to determine the \emph{severity} of the effects. 

Unless otherwise noted or specifically overridden, medichines (p. 308) will protect a character from drug/toxin effects (but not nanodrugs/nanotoxins). Enhancements like toxin filters (p. 305) may also impede a drug’s effect or provide complete resistance. If an antidote is taken in advance or before the effects kick in, the drug will not work. 

\subsubsection{Addiction and substance abuse} 

Some drugs are addictive, either physically (affecting the morph) or mentally (affecting the ego) --- and sometimes both. Every time a character uses the drug (or after an appropriate amount of use, as determined by the gamemaster), they must make a WIL $\times$ 3 Test to avoid addiction. Each drug has an Addiction Modifier that will modify this test. 

Failure indicates that the character has become addicted --- they immediately acquire the Addiction negative trait (p. 148). Addiction is measured in three levels: Minor, Moderate, and Major. The severity determines how often an addicted character needs the drug and what the negative effects of not using the drug are. 

An addicted character must continue to make WIL $\times$ 3 Tests as they use the drug, as determined by the gamemaster. Failure indicates the character’s addiction severity increases. 

The negative effects from not using a drug end whenever the character does the drug again. Durability and Lucidity penalties are not damage, but temporary decreases to the character’s maximum values; the character immediately regains the lost Durability or Lucidity when they do the drug again. 

Addiction is of indefinite duration. To clean up, the character must stay off the drug for 1 week for each level of addiction. Resisting this craving is difficult, and should at least require another WIL $\times$ 3 Test, modified by the drug’s Addiction modifier. Players and gamemasters are encouraged to roleplay an attempt to kick a habit. Each week the character is off the drug, the addiction drops by one level. When it reaches 0, the character is clean ... though there is always danger of a relapse. 

Physical addictions do not carry over to a new morph if the character resleeves, but mental addictions do. If the character uploads and resleeves, the mental addictions persist, and the morph the character leaves behind remains physically addicted. This means that poor or unlucky characters may occasionally find themselves resleeved into a morph that has a physical addiction. In this case, the character is subject to the physical addictiveness of the drug but not the mental addiction, although if they break down and indulge in the drug, they may themself become physically addicted. 

Characters who resleeve as infomorphs can remain mentally addicted to a substance despite no longer having a body. The market is always happy to provide, though; a wide variety of narcoalgorithms mirroring the effects of most of the drugs described below are available for infomorphs and AIs. For the infomorphported narcoalgorithm version of any physicallyonly addictive drug described below, consider the Addictiveness to be effectively physical. The character remains addicted as long as they are an infomorph, but they do not remain addicted if they sleeve into a physical morph. 

\subsection{Drugs} \label{sec:drugs} 

The drugs described here are usually (but not always beneficial), and are typically taken intentionally. Drugs and chemicals used offensively are described under Chemicals and Toxins, both on p. 323. Note that the drugs here are just a representative sampling. There are thousands if not millions of drugs in circulation in \emph{Eclipse Phase} --- gamemasters are encouraged to introduce their own, using these as guidelines. 

\subsubsection{Cognitive drugs} 

Nootropics and similar drugs are intended to boost the user’s mental faculties. 

\textbf{Drive:} This nootropic speeds up left-right brain hemisphere communication, stimulates idea production, and improves concentration, with no usual side effects. Users receive a +5 bonus to COG while the drug lasts. \textbf{[Bas]} 

\textbf{Klar:} Klar boosts alertness and enhances clarity and perception. Users report a feeling of being ``elevated'' to a higher level. They receive +5 INT while the drug lasts. \textbf{[Bas]} 

\textbf{Neem:} Neem is a mnemonic drug that works by ``tagging'' experiences and mental input with a set of unique sensations that contribute to the formation of state-based memories. Neem gummy chews come in a variety of fruit flavors shaped like extinct old Earth animals. Neem gives characters a +20 bonus on COG Tests to recall information they learned while on Neem (see \emph{Memorizing and Remembering}, p. 176). The drawback to Neem is that memories they accumulate while under the drug’s influence have no emotional association. For example, a character who witnessed something horrible happening to a friend or who had a fight with a romantic partner while on Neem would feel no emotional connection whatsoever to what happened. \textbf{[Modéré]} 

\subsubsection{Combat drugs} 

Combat drugs are an easy way of evening the odds in a fight. 

\textbf{BringIt:} In some respects more a social than a combat drug, BringIt stimulates massive bursts of aggression pheromones designed to make the user the center of attention in a fight. In combat, opponents within 3 meters of the character not already in unarmed or melee combat with another character must pass a WIL $\times$ 3 Test or attack the character using BringIt. The nature of airborne pheromones is imprecise, however, so if the character using BringIt is within 1 meter of another character hostile to the character affected, the affected character may opt to attack the proximate character instead of the BringIt user. Characters using this drug suffer a $-$20 modifier on social skill tests. \textbf{[Bas]} 

\textbf{Grin:} Grin is an effective opiate and pain suppressant. Users may ignore the $-$10 modifiers from 2 wounds (not cumulative with similar effects), and in fact may not even be aware they are injured. Grin users suffer from tunnel vision, however, and so suffer a $-$10 modifier on Perception Tests. \textbf{[Bas]} 

\textbf{Kick:} Kick is a strong stimulant that increases the user’s response time and puts them on edge. The character gains +10 REF and +1 Speed for the duration of the drug. Characters under the influence of Kick are twitchy, however, reacting in a jumpy, cat-like fashion to sudden or unexpected stimuli. At the gamemaster’s discretion, they must make a WIL $\times$ 2 Test or react without thinking towards unexpected noises or other surprises. Long-term users suffer $-$5 COO. \textbf{[Modéré]} 

\textbf{MRDR:} MRDR is a straightforward and brutal combat drug. It increases pain tolerance, speed, and strength. The character receives +10 SOM, +1 Speed, +10 Durability, and may ignore the $-$10 modifier of one wound. Any damage incurred while under the effects of the drug is taken from the bonus Durability first. MRDR users are easily identifiable by the broken blood vessels in their eyes, tense posture, and visible tension in the muscles of the face, arms, and legs. Long-term users suffer $-$5 SOM. \textbf{[Low]} 

\textbf{Phlo:} Phlo increases alertness and coordination, making the user more graceful and nimble in a fray. The character gains +5 COO and +10 on Perception Tests for the duration of the drug. Everything feels possible to a character on Phlo, and so they are vulnerable to being goaded into actions that might be foolish or dangerous (apply a $-$10 modifier to appropriate Social Skill Tests). \textbf{[Modéré]} 

\subsubsection{Health drugs} 

Pharma-foods that boost the consumer’s health and physical state are common. 

\textbf{Bananas Furiosas:} This drug reverses some of the effects of de-ionizing radiation on the cells of the body. Although a pill form is available, it most commonly comes in large bunches of bright orange-red bananas. Bananas reduce the severity of a radiation dosage (gamemaster determines effect). \textbf{[Bas]} 

Comfurt: This tasty yogurt treat blocks stress hormones, stabilizes mood, and relieves anxiety, allowing them to ignore the effect of 1 trauma and temporarily boosting Lucidity by +5. Any stress suffered while the drug is in effect is taken from the bonus Lucidity first. Comfurt also provides a +10 bonus when resisting attempts to manipulate the user’s emotions. Excessive use of Comfurt can lead to chronic itchiness caused by histamine release. [Low] 

\subsubsection{Recreational drugs} 

These drugs compete with petals (p. 321) and black market XP for wasting people’s time and lives away. 

\textbf{Buzz:} This gene-modified variant of BZ is an odorless, invisible, extremely powerful hallucinogen. Users or affected characters will undergo extremely realistic hallucinations for the duration, and may even ``share'' hallucinations with other affected characters. Characters will suffer a $-$30 modifier to any tests to remember what occurred while under the influence. \textbf{[Modéré]} 

\textbf{Mono No Aware:} Taken from the Japanese term for sadness at the ephemerality of worldly things, this drug, typically ingested as a tea, is a depressant that induces a meditative state. Mono No Aware gives the character a +10 bonus on Art and Sense Tests. With frequent use, Mono No Aware reacts with pigments in the skin to create a pallor with a slight bluish tinge, even in darker-skinned morphs. \textbf{[Bas]} 

\textbf{Orbital Hash:} Good ol’ reefer --- but grown in space using powerful lighting and post-singularity hydroponics. Because space is at a premium in habitats and scum barges, blocks of hashish are the preferred mode of transport and delivery. However, for the wealthy and on planets, buds in leaf form are not uncommon. Hash allows the character to ignore the effects of 1 trauma, but inflicts a $-$10 penalty on all memory-related tests and Knowledge Skill Tests. Hash users exhibit bloodshot eyes, lethargic behaviors, and the munchies. \textbf{[Low]}\subsubsection{Social drugs}These social lubricants affect the user’s interactions with others. 

\textbf{Alpha:} Alpha is a more subtle version of BringIt, popular with hypercorp execs, street thugs, and anyone else who wants to come across as a domineering asshole. The pharm designer who invented it had a retro sensibility (and maybe a sick sense of humor); Alpha is typically synthesized as a sparkling white powder designed to be snorted. Alpha stimulates production of threat pheromones, but less bluntly than BringIt. Alpha imparts confidence, a feeling of power, and alertness. Users can function without sleep for 4 days, after which point they need to catch up with at least 4 hours of sleep (remember morphs with basic biomods require less sleep). Dosed characters receive a +20 modifier on Intimidation Tests and +10 on Persuasion and Networking Tests where attitude is a factor (gamemaster discretion). These bonuses only apply to characters within 2 meters of the Alpha user. 

On the downside, alpha users are impatient, unfocused assholes. At the gamemaster’s discretion, Social skill modifiers may be reversed to penalties with certain types of people. Additionally, Alpha users suffer $-$10 on all COG skill tests related to memory and coherent or logical thinking. Long-term users may suffer the COG penalty even when not on the drug; on it, they may be worse. \textbf{[Élevé]} 

\textbf{Hither:} Want to ooze sexy like a pleasure morph on a hot tin roof? For those desiring that slinky je-nesais- quoi, Hither is the tool. Hither is a clear, slippery gel, sometimes with a faint, musky, floral scent. Hither is applied to parts of the body with large concentrations of sweat glands, where the skin quickly absorbs it. Hither is a mild euphoriant, imparting a feeling of confidence and you-know-you-want-it-ness to the user. It also stimulates abundant production of lust pheromones. The character gains a +10 bonus on Persuasion Tests against targets who are possible to seduce. At the gamemaster’s discretion, this extends to Deception, Impersonate, and Networking Tests. \textbf{[Bas]} 

\textbf{Juice:} This potent anti-depressant makes it almost impossible to have bad feelings or negative thoughts. The character is unnaturally happy --- often irritatingly or strangely so. The character receives a +30 bonus against fear or attempts to manipulate their emotions in a negative direction, but is also likely to act inappropriately, like giggling over the massive amount of spilled blood or cheerfully changing the subject to inane topics when someone else is freaking out. \textbf{[Bas]} 



\begin{table} \begin{tabularx}{\textwidth}{|X|l|l|l|l|X|X|} \hline

\multicolumn{7}{|c|}{\textbf{Drugs}}	\\ \hline

&\textbf{Type}	&\textbf{Application}	&\textbf{Onset}	&\textbf{Duration}	&\textbf{Addiction\newline mod}	&\textbf{Addiction\newline type} \\ \hline

\multicolumn{7}{|l|}{\emph{Cognitive Drugs}}	\\ \hline

Drive	&Chem	&O	&20 min &8 hours	&$-$	&Mental	\\ \hline

Klar	&Chem	&O	&20 min	&8 hours	&$-$	&Mental	\\ \hline

Neem	&Chem	&O	&20 min	&12 hours	&$-$	&Mental	\\ \hline

\multicolumn{7}{|l|}{\emph{Combat drugs}}	\\ \hline

BringIt	&Bio	&Inh, Inj, O	&1 min	&15 min	&+10	&Physical	\\ \hline

Grin	&Chem	&Inh, Inj, O	&3 turns	&3 hours	&$-$10	&Physical	\\ \hline

Kick	&Chem	&Inh, Inj, O	&3 turns	&2 hours	&$-$10	&Physical	\\ \hline

MRDR	&Chem	&O	&20 min	&1 heure	&$-$10	&Physical	\\ \hline

Phlo	&Chem	&O	&20 min	&1 heure	&$-$10	&Physical	\\ \hline

\multicolumn{7}{|l|}{\emph{Health drugs}}	\\ \hline

Bananas Furiosas	&Chem	&O	&20 min	&1 jour	&$-$	&$-$	\\ \hline

Comfurt	&Bio	&O	&20 min	&12 hours	&$-$10	&Mental	\\ \hline

\multicolumn{7}{|l|}{\emph{Recreational drugs}}	\\ \hline

Buzz	&Chem	&Inh, O	&1 heure	&36 hours	&$-$	&Mental	\\ \hline

Mono No Aware	&Chem	&O	&20 min	&8 hours	&$-$10	&Mental	\\ \hline

Orbital Hash	&Chem	&Inh	&3 min	&3 hours	&$-$	&Mental	\\ \hline

\multicolumn{7}{|l|}{\emph{Social drugs}}	\\ \hline

Alpha	&Bio	&Inh	&1 minute	&2 hours	&$-$10	&Mental	\\ \hline

Hither	&Bio	&D	&1 minute	&6 hours	&$-$10	&Physical	\\ \hline

Juice	&Chem	&O, Inh	&20 min	&8 hours	&$-$	&Mental	\\ \hline

\end{tabularx} \end{table} 

\clearpage



\label{tab:drugs} 



\subsection{Nanodrugs} \label{sec:nanodrugs} 

Nanodrugs are temporary nanobot infestations that apply a specific effect. 

\textbf{Frequency:} Frequency (or Freeq) is a nanodrug designed as a tool for scientific visualization. It releases a small swarm of nanobots into the character’s bloodstream that settle in the epidermis, where they act as sensors of electromagnetic radiation. This sensory input is then injected into the character’s visual and tactile sensoria, hitting the user with a sequence of novel stimuli, typically a light show or weird tactile sensations. Aside from its recreational uses, Frequency is good at picking up on localized field radiation with a standard Perception Test. A character can take advantage of this to spot sensors and hidden electronics. Similar to now-obsolete 20th-century hallucinogens like LSD and psilocybin, however, a Frequency trip can be disorienting and upsetting (the gamemaster should apply any modifiers, mental stress, or even trauma as they feel appropriate). Characters typically experience a period about 1/3 of the way through their trip in which sensory input is extremely intense; during this period, which usually lasts about 2 hours, they are unable to read. \textbf{[Modéré]} 

\textbf{Gravy:} Gravy assists characters in acclimating to high gravity environments. It comes in a variety of flavors and is often added as a sauce to food. For Gravy to be 100\% effective, the character must begin using it in advance. Reduce penalties for high-gravity acclimation by 20. \textbf{[Bas]} 

\textbf{Schizo:} Schizo is a nanodrug that mirrors the effects of paranoid schizophrenia. It is popular in some hyperelite social circles as a truly daring and intriguing experience. A dose of schizo looks like a disposable antique razor blade. Making an incision in the skin releases a swarm of nanobots that travel to the central nervous system and induce the effects of the drug. While in effect, the character is severely paranoid and hears voices. How this plays out is at the discretion of the gamemaster, but should include irrational fears, unusual compulsions based on the instructions of the voice or voices, and a strong possibility that the character will behave in a violent or destructive fashion. The character may make WIL $\times$ 3 Tests to avoid violent acts against objects or strangers. Friends and trusted acquaintances are probably less likely to be targets of violence (+30 modifier to avoid hurting people the character cares about or destroying important possessions). Note that the character’s muse is unaffected by Schizo and can make efforts to babysit the character. Characters who take Schizo suffer 1d10 mental stress. \textbf{[Bas]} 

\subsubsection{Petals} 

Petals is a term for a type of narrative hallucinogen, a nanodrug that hijacks the senses and takes the user on a game-like, highly immersive trip. Known by a myriad of intriguing names --- Forgotten Hand, Darkly Selving, Inquisitive Green, to name a few --- Petals are post-Fall society’s heroin --- the drug of choice for the desperate and fucked. Petals almost always appear as nanopharmaceutical flowers, potted or with a nutrient pack attached to the stem. Plucking and swallowing the petals from the flower triggers the effects immediately. Flowers have 5-10 petals. Multiple users may share the experience if they take the Petals within 1 minute of the first one being plucked; after this all petals remaining on the flower fade to translucent white and become inert. 

Petal experiences are like entire scenarios in and of themselves. Some take place entirely in the user’s mesh inserts (the user must cede control of their implants voluntarily; if they do not, the drug has no effect other than producing very low-intensity LSD-like visual hallucinations), taking control of the character’s entoptic displays, linking to secretive mesh servers and other trippers, and invading the character’s sensorium with AR ``hallucinations.'' Others put the character into a near-comatose state during which they go on a head trip. Normally there is some kind of well-developed theme or plot to a Petal experience, although in some cases they just experience a stream of images. 

Though most societies seek to suppress Petals, new ones appear constantly, fueled by a persistent subculture of crafters and users. Petalcrafters view their work as an art form (or at least as really good entertainment), and the better Petals are lovingly crafted, hauntingly beautiful experiences --- even if they’re also terrifying. The subculture of Petal use ranges from casual users who occasionally do an easy, short-duration flower to hardcore addicts who spend much of their time not on Petals trying to hunt down the most intense and esoteric varieties. From this subculture comes a lot of information on what various Petals look like and their effects. Because Petals combine custom nanobots with tailored chemical payloads and sometimes connections to mesh servers, duplicating them using fabricators is impossible, leading to an active market of crafters, dealers, and traders. 

Petals sometimes contain easter eggs and rewards, called ``sweets'' by petal users. Getting the sweets usually requires fulfilling certain conditions within the trip, such as correctly answering questions or fulfilling goals. Typical sweets include skillsofts, new clothing or product designs, and custom infomorph sleeves. 

On the negative side, some Petal trips go bad, flooring 1d10 mental stress or more on the user. Perhaps worse, some Petals are loaded with malware that takes over the user’s mesh inserts and worse --- some sentinels even whisper of Petals carrying strains of the Exsurgent virus. \textbf{[Trivial to High]} 

\subsubsection{Sample petals} 

A few examples of Petal experiences: 

\textbf{Forgotten hand} 

One of the character’s hands detaches and makes a run for it. The character is conscious and able to interact normally with the real world, but they cannot perceive the ``escaped'' hand and firmly believe that it’s getting away. The hand will lead the character a merry chase, but at some point, a new hand appears on the character’s wrist. It may be glittery and opalescent, demonic and clawed, or bestial. Eventually, after an hour or two, the character will catch up to their hand, but to get rid of their new hand and re-attach the old, they must answer cryptic questions posed by a gnome-like being. 

\textbf{Darkly selving} 

This petal is believed to achieve many of its effects by connecting to the mesh, where an AI observes and controls some of the event flow, and only works for multiple trippers. Like Forgotten Hand, it works by overlaying AR perceptions on the real world, but because of the effects, it’s highly inadvisable to take in places where any non-trippers will be present. Darkly Selving creates an epsilon fork of each character tripping and sleeves the fork in an infomorph that looks like a demonic version of themself, using visual input from the character’s co-trippers. AR overlays cause the characters to perceive themselves as angelic beings, while the realseeming demonic infomorphs appear as AR overlays on their real world perceptions. What happens next varies, but generally both the characters and their forks are subjected to a series of strong chemical and narcoalgorithmic stimuli, ranging from Hitherlike effects to massive doses of MRDR (or sometimes both). The effects directed against the forks are generally much more intense. The objective --- hinted at via environmental clues --- is to merge with one’s fork, which can be accomplished in a variety of ways, ranging from hunting them down and eating their heart to solving a puzzle or reaching a goal before their forks can. 

\textbf{Delphinium six} 

The last and rarest in a series of petals, Delphinium Six is the Grail of petal users, a supposedly transcendental experience that might not even exist. Delphinium One is scarce, Two and Three are quite rare, Four is an amazing find, and Five and Six are only rumors. Hints of what Six might hold are based largely on extrapolation from the little that is known about the lower-numbered petals. The following facts are generally accepted. It is a group experience, but not all members of the tripping group are rewarded equally. It is intensely surreal, yet in a purposeful way, as are all of the Delphinium series. It concludes the loosely-built narrative of a drugged-out version of a fairy tale princess and her quest for enlightenment begun in Delphinium One, replete with strange omens and mythological creatures. Rumors of what the ending might hold are more fanciful, and range from the trippers being resleeved in god-like infomorphs to them being trapped forever in an ego prison. Delphinium Six is completely virtual, leaving the characters comatose for the duration, and probably lasts a long time, perhaps 40 hours. 

\subsubsection{Other nanodrugs} 

Nanodrugs have the capability of making fundamental changes to a body’s biochemistry and mental state. The potential effects are too numerous to list, but gamemasters should consider allowing nanodrugs that temporarily apply certain traits, such as Brave, Direction Sense, Math Wiz, Pain Tolerance, Psi Chameleon, Psi Defense, Situational Awareness, Tough, Feeble, Frail, Low Pain Tolerance, Mental Disorder, Mild Allergy, Neural Damage, Psi Vulnerability, Severe Allergy, Timid, VR Conditioning, VR Vertigo, Weak Immune System, or Zero-G Nausea. Similarly, the nanodrug could force the character into a particular mental emotional state, such as a bad mood, edginess, contentment, or overconfidence. Gamemasters are encouraged to experiment with different possibilities and effects. 

\hspace{1cm} 

\begin{tabular}{|l|l|l|l|l|l|l|} \hline

\multicolumn{6}{|c|}{\textbf{Nanodrugs}}	\\ \hline

\textbf{Nanodrugs}	&\textbf{Type}	&\textbf{Application}	&\textbf{Duration}	&\textbf{Addiction mod}	&\textbf{Addiction type} \\ \hline

Frequency	&Nano	&Inj, O	&8 hours &$-$10	&Mental \\ \hline

Gravy	&Nano	&Inj, O	&special &--- &--- \\ \hline

Petals	&Nano	&O	&2 hours--1 day	&+10 to $-$20	&Mental \\ \hline

Schizo	&Nano	&Inj	&1 jour	&--- &Mental \\ \hline

\end{tabular} \label{tab:nanodrugs} 



\subsection{Narcoalgorithms} \label{sec:narcoalgorithms} 

Narcoalgorithms are software programs that simulate the effects of drugs on biological bodies. Almost all bio, chemical, and nano drugs can be replicated as narcoalgorithms, with corresponding effect (gamemaster discretion). Narcoalgorithms may be run by infomorphs, egos encased in cyberbrains (pods and synthmorphs), simulmorphs, and even AIs. 

\textbf{DDR:} Originally crafted by prankster hackers and distributed as a virus, DDR (for ``Dance Dance Robot'') triggers impulses in the target’s motor control circuits. Primary targeting robot AIs, the effect is that targets ``dance'' in jerky, automated movements. Pleasure receptors are also activated so that dancing --- and movement of any kind --- feels good. Different software variants invoke different motions and styles. The target suffers a $-$20 modifier on other actions while dancing, but the dancing may be overridden with a WIL $\times$ 3 Test. \textbf{[Bas]} 

\textbf{Linkstate:} This software actually connects the user to a peer-to-peer network, where it randomly connects to other linkstate users and samples a bit of their XP feed and randomly accessed memories --- typically just enough to provide context, but not enough to acquire private personal details. These inputs are spliced together, their emotional inputs amplified, and then the entire package is spiked with some hormonal circuit triggers and artificial synaesthesia. The effect is a mind-blowing mixed sampling of people’s lives, mashed together in a sensory soup, that hits the mind with a euphoric rush. Linkstate users are catatonic while under the effects (typical sessions run 3-4 hours), but afterwards they often report that they have flashbacks of events in other people’s lives. \textbf{[Bas]} 



\subsection{Chemicals} \label{sec:chemicals} 

\textbf{Atropine:} Though poisonous in large doses, atropine is an effective antidote against nerve agents like BTX2 and Nervex. Easily synthesized in a maker, atropine will avert the effect whether taken soon before or after dosage by a nerve agent. \textbf{[Trivial]} 

\textbf{DMSO:} This chemical acts as a carrier, allowing other chemicals to be absorbed through the skin. It allows any chemical agent to be applied dermally. \textbf{[Trivial]} 

\textbf{Liquid Thermite:} Similar to scrapper’s gel, liquid thermite comes in a gel form that is easily applied under all environmental conditions (by the nature of its chemical reaction, thermite is oxygenated and will burn underwater or in space). It is ignited with an electric charge, burning at temperatures exceeding 2,500 degrees Celsius and melting through whatever it is touching. Liquid thermite floors 3d10 + 5 DV per Action turn to whatever it is touching. Armor will also be burnt through, offering no protection once the full Armor rating has been reached. \textbf{[Modéré]} 

\textbf{NotWater:} NotWater is an effective liquid fire retardant that does not get objects wet, no matter how absorbent they are --- it simply beads up and slides right off. \textbf{[Trivial]} 

\textbf{Scrapper’s Gel:} This goo turns into a potent acid when given an electrical charge. It comes in a gel-like state and may be smeared like jelly, and may even be used in space. In acid form, scrapper’s gel does 1d10 + 5 DV per Action Turn to anything it touches, unless the material has been treated against acid. Armor will protect against this acid at first, but the acid will eat through the armor, so that it will no longer protect after its full armor value has been reached. \textbf{[Bas]} 

\textbf{Slip:} This liquid is almost entirely frictionless. When spread around an area (commonly used in splash grenades), anyone attempting to walk or run on the affected surface must make a COO Test or fall down. Likewise, any coated surface becomes extremely hard to grip onto, requiring a SOM Test to hang on. Anyone attempting to grapple a slip-soaked character suffers a $-$30 modifier. \textbf{[Bas]} 

\textbf{Tracker Dye:} This liquid is colorless at normal light but becomes recognizable under pre-specified different wavelengths (such as infrared or ultraviolet). \textbf{[Trivial]} 



\subsection{Toxins} \label{sec:toxins} 

Chemical warfare involves using the toxic properties of biological and chemical substances to kill, injure, or incapacitate an enemy. Note that an antidote can be constructed for most toxins if a sample is acquired and an appropriate Medicine or Academics Test is made. This is considered a Task Action with a timeframe of 1 hour. These toxins only affect biomorphs; synthmorphs are immune. 

\textbf{BTX:} BTX-squared (also called Frog Bite) is a genetically-enhanced variant of the extremely potent cardiotoxic and neurotoxic batrachotoxin. It leads to fast paralysis and cardiac arrest that usually kills the target within a few Action Turns. Affected characters suffer 2d10 + 10 damage a turn for 3 Action Turns; medichines reduce this damage by half. They must also make a SOM $\times$ 2 Test (+30 with medichines) or be paralyzed for 1 hour. \textbf{[Élevé]} 

\textbf{CR Gas:} This potent incapacitating agent causes eye twitching and temporary blindness, severe coughing and breathing difficulty, skin irritation, and panic. Affected characters suffer 1d10 $\div$ 2 damage, a $-$30 modifier to sight-based Perception Tests, and a $-$20 modifier to all other actions for 20 minutes (5 minutes if the character has medichines). \textbf{[Bas]} 

\textbf{Flight:} This drug is derived from human pheromones released due to fear, and is intended to instill alarm or even terror in the character. Affected characters must make a WIL $\times$ 3 Test (+30 with medichines) or suffer a panic attack, flooring 1d10 stress. Dosed characters also suffer a $-$30 modifier for resisting intimidation or fear-based emotional manipulations. Flight affects last for 1 hour (5 minutes with medichines). \textbf{[Bas]} 

\textbf{Nervex:} Derived from deadly nerve agents like cyclosarin, VX, and novichok, this genetically-modified toxin is deployed as a colorless, odorless gas that turns safely inert 10 minutes after deployment. It causes involuntary contraction of the muscles, seizures, and death by respiratory failure. One minute after exposure, the character must make a SOM Test or be incapacitated by seizures, paralysis, or nausea and vomiting; unaffected characters still suffer a $-$20 modifier to all actions. After 10 minutes, the character will die unless an antidote (such as atropine, p. 323) is applied. Characters with medichines suffer the initial effects, but recover after 5 minutes. \textbf{[Élevé]} 

\textbf{Oxytocin-A:} A genetically-improved variant of oxytocin, this drug induces trust in the recipient. Drugged characters suffer a $-$30 modifier on all WIL and Kinesics Tests where trust is a factor. Medichines provide immunity. \textbf{[Bas]} 

\textbf{Twitch:} Twitch is a convulsive agent, a nonlethal nerve gas. Affected characters must succeed in a SOM Test (+30 with medichines) or become incapacitated with severe muscle tremors. Unaffected characters still suffer a $-$20 on all actions. The effects of Twitch last for 10 minutes, 5 if the character has medichines. \textbf{[Bas]} 



\subsection{Nanotoxins} \label{sec:nanotoxins} 

\textbf{Disruption:} This nanotoxin attacks the myelin sheath on nerves, disrupting nerve impulses and flooring symptoms of multiple sclerosis. Every hour the morph suffers a $-$5 modifier to COO, REF, and COG. If any aptitudes are reduced to zero,the morph is effectively paralyzed and catatonic. \textbf{[Modéré]} 

\textbf{Necrosis:} Necrosis nanobots attack the walls of cells inside the body, killing tissue. This nanotoxin floors 1d10 $\div$ 2 damage per Action Turn for one minute, after which the nanobots disable and flush from the body. Necrosis only affects biomorphs. \textbf{[Modéré]} 

\textbf{Neuropath:} These nanobots are designed to stimulate the pain receptors of a morph on a systemic level to cause agony and impairment. While most neuropaths target biological receptors, variants are available that induce comparable (phantom) pain stimulations in the cyberbrains of synthmorphs to create an equivalent effect. The affected character must succeed in a WIL $\times$ 3 Test or become incapacitated. Even if they succeed, they suffer $-$30 from the floored agony. Any form of pain resistance that allows a character to ignore wound modifiers will negate the neuropath pain modifier by an appropriate amount. \textbf{[Modéré]} 

\textbf{Nutcracker:} Nutcrackers are nanobots designed to locate, migrate, and decompose the synthdiamond case of a cortical stack within a morph by attacking its crystal lattice. This process takes approximately 6 hours, after which the cortical stack is destroyed. These nanobots also attack the cortical stack’s connections to the (cyber)brain and brain-mapping nanobots. After 1 hour, the victim will be aware that their cortical stack is threatened. After 3 hours, all connections will be severed and the cortical stack will no longer be able to back up the character. \textbf{[Élevé]} 



\subsection{Pathogens} \label{sec:pathogens} 

A pathogen is an infectious biological agent that causes disease or illness to its host. While natural pathogens rarely strive to kill their hosts, germ warfare programs revived during the Fall --- or instigated by the TITANs --- sought to modify and use pathogens as a weapon of war. The ideal characteristics of lethal biological agents are high infectivity, high potency, availability of vaccines, and delivery as an aerosol. Most biomorphs are immune to standard pathogens thanks to their basic bio-mods, and medichines will protect against most others. However, even these defenses may not protect against diseases left by the TITANs or a new terrorist cell’s biowar bug. It is largely recommended that pathogens be handled as a plot device, rather than an active threat to the characters. Pathogens have no effect on synthmorphs. 

\textbf{Degen:} Characters exposed to this degenerative neurological disease must make a DUR $\times$ 2 Test or become infected. Medichines will defeat the disease, but others will not show signs of infection for 1 week, when the symptoms of a rapidly progressing dementia will become clear: memory loss, personality changes, and hallucinations. If untreated, Degen will progress for another week with more serious symptoms, including speech impediments, jerky movements, loss of balance and coordination, and even seizures. This is reflected by a 5 point loss in all aptitudes per day (after the first week). When any aptitude reaches 0, the character dies. Degen is notorious for its effect in corrupting cortical stack backups before infection symptoms manifest. \textbf{[Cher]} 

\textbf{Trigger:} Trigger is a designer virus that selectively targets and infects mast cells to trigger a hyper-allergic reaction. The resulting anaphylactic shock due to systemic vasodilatation (associated with a sudden drop in blood pressure) and bronchial swelling (resulting in constriction and difficulty breathing) usually leads to death in a matter of minutes after onset, if not treated. Infected characters must succeed in a DUR Test (using their current Durability score minus damage) or die quickly. Even medichines have difficulty reacting in time against this virus; characters with medichines must make a DUR $\times$ 2 Test to survive. \textbf{[Cher]} 



\subsection{Psi drugs} \label{sec:psi-drugs} 

Research into the Watts-MacLeod strain has resulted in several exceptional breakthroughs involving the creation of psi-impacting drugs. Each of these drugs is in the experimental stage, but they are already finding some use among Firewall and similar secretive groupings. 

\textbf{Inhibitor:} Inhibitor is a cocktail of neurochemicals that block some brain receptor and transmitter functions in an attempt to reduce psi-waves and block or impair sleights. This drug is commonly used to restrain async prisoners from using their abilities. A drugged character must make a WIL $\times$ 2 Test. If they fail, they lose all psi abilities for the drug’s duration. If they succeed, they suffer a $-$30 impairment on Psi skills and all strain is doubled. Inhibitor has an unfortunate side effect of doping the character down, however; apply a $-$10 modifier to their COG. Inhibitor-influenced characters tend to have a glazed, dopey expression and have difficulty getting excited or emotional. \textbf{[Élevé]} 

\textbf{Psi-Opener:} Psi-opener drugs are variants of the Watts-MacLeod strain with a temporary effect and which do not permanently alter the user’s brain. Psiopener temporarily imbues the user with the ability to use one particular sleight, regardless of whether or not they have the Psi trait. Each type of Psi-opener is customized for a particular sleight. While primarily intended for non-asyncs, non-asyncs may not possess Psi skills, so they must default to WIL. For this reason, Psi-Opener is often doubled up with Psike-out. Using Psi-opener is a mind-wrenching experience. Users are occasionally subject to hallucinations (gamemaster discretion). When the drug wears off, it floors 1d10 points of mental stress, +2 if the drug imbues a psi-gamma sleight. \textbf{[Cher]} 

\textbf{Psike-Out:} Psike-out bolsters an async’s psi abilities. Apply a +20 modifier to the async’s Psi skill tests for the drug’s duration. However, also apply +2 to all strain DVs for the drug’s duration. Psike-out is mentally addictive, with an Addiction modifier of $-$10. \textbf{[Cher]} 

\begin{table} 

\begin{tabular}{|l|l|l|l|l|} \hline

\multicolumn{5}{|c|}{\textbf{Toxins}} \\ \hline

&\textbf{Type}	&\textbf{Application}	&\textbf{Onset time}	&\textbf{Duration} \\ \hline

\multicolumn{5}{|l|}{Chemical toxins} \\ \hline

BTX2	&Chem	&D, Inj, O	&1 Action Turn	&3 Action Turns/1 hour \\ \hline

CR Gas	&Chem	&D, Inh	&1 Action Turn	&20 minutes \\ \hline

Vol	&Bio	&Inh	&3 Action Turns	&1 heure \\ \hline

Nervex	&Chem	&D, Inh, Inj, O	&1 minute	&death \\ \hline

Oxytocin-A	&Bio	&Inh, Inj	&3 minutes	&2 hours \\ \hline

Twitch	&Chem	&D, Inh, Inj, O	&3 Action Turns	&10 minutes \\ \hline

\multicolumn{5}{|l|}{Nanotoxins} \\ \hline

Degeneration	&Nano	&Inj, O	&Immediate	&8 hours \\ \hline

Necrosis	&Nano	&Inj, O	&3 Action Turns	&1 minute \\ \hline

Neuropath	&Nano	&D, Inj, O	&3 Action Turns	&8 hours \\ \hline

Nutcracker	&Nano	&Inj, O	&Immediate	&6 hours \\ \hline

\multicolumn{5}{|l|}{Psi Drugs} \\ \hline

Inhibitor	&Chem	&Inj, O	&3 Action Turns	&6 hours \\ \hline

Psi-Opener	&Bio	&Inj, O	&20 minutes	&1 heure \\ \hline

Psike-Out	&Chem	&Inj, O	&1 minute	&1 heure \\ \hline

\end{tabular} \label{tab:Toxins} \end{table} 



\section{Everyday technology} \label{sec:everyday-tech} 

The following devices are all exceptionally common and can be acquired in almost any habitat. Almost everyone in \emph{Eclipse Phase} either owns these devices or knows several people who do. 

\textbf{Ecto:} Ectos are the external version of basic mesh inserts (p. 300), minus the medical sensors. These colorful devices serve as a wearable mesh terminal, PDA, locator, and camera-phone. The devices are flexible (often worn as bracelets), dirt-resistant, self-cleaning, and may be stretched out to increase screen size. They may project holographic displays and are typically equipped with wireless-enabled glasses or contact lenses and decorative earpieces or earrings so that the user may access augmented reality. Given the ubiquity of mesh inserts, ectos are growing less common, but they are still used by bioconservatives, others without implants, and those who prefer to access the mesh via an external device for security concerns. \textbf{[Bas]} 

\textbf{Holographic Projectors:} These devices are capable of projecting high-definition, ultra-realistic three-dimensional images and movies. From a distance (20+ meters), such holograms can be difficult to distinguish as fake, but up close they are easier to see for what they are (+20 Perception Test modifier). Holograms do not appear wavelengths other than visual light, and so are easily identified by anyone with enhanced vision. [Low] Micrograv Shoes: These shoes are equipped with velcro and/or a magnetic system, allowing the wearer to walk normally on appropriate surfaces in micrograv and zero-G environments, rather than floating or bouncing. \textbf{[Trivial]} 

\textbf{Portable Sensor:} This is a small portable (possibly even wearable) sensor system. The type of sensor must be chosen (for example: infrared, lidar, radar, x-ray). Combined sensor systems are also available, at a cumulative cost. See \emph{Radio and Sensor Ranges}, p. 299. and \emph{Using Enhanced Senses}, p. 302. \textbf{[Modéré]} 

\textbf{Smart Clothing:} Smart clothing can change its color, texture, and even its cut, taking only a minute or two to transform from a solid color jumpsuit to a plaid party dress or a replica of a pinstriped, late 20th century business suit. It can also camouflage the wearer, providing a +20 bonus to Infiltration Tests to avoid being seen, as long as the wearer is stationary or not moving faster than a slow walk, and as long as the wearer is completely covered or also using chameleon skin (p. 303) of the same color/pattern. Si le personnage n'est pas complètement camouflé ou si il se déplace trop vite, réduisez le modificateur à +10. Smart clothing also keeps the character warm or cool, allowing the character to exist comfortably in environments from $-$40 to 70$^{\circ}$ C. \textbf{[Low]} 

\textbf{Smart Vac Clothing:} Just like regular smart clothing, this outfit can also transform into a light vacsuit (p. 333). It also functions as armor with a rating of 2/4. \textbf{[Modéré]} 

\textbf{Specs:} Specs are vision-enhancing glasses. They deliver sensory data directly into the wearer’s visual cortex by connecting with their basic mesh inserts (p. 300), though visual displays are available for bioconservatives and other characters without implants. Specs extend the range of the wearer’s vision from terahertz waves to gamma rays (p. 302). Specs include a t-ray emitter (p. 306), however, using x-rays, or gamma rays for visual purposes requires a separate emitter, since neither of these sorts of radiation are common inside habitats, or in any safe environments. Specs have a variable focus equivalent to 5 power magnifiers and provide the wearer with a +10 bonus to all Perception Tests involving vision. \textbf{[Bas]} 

\textbf{Tools:} Tools come in kits (portable), shops (can fit into a large vehicle), and facilities (large, non-mobile). Each set of tools applies to a particular skill, such as Hardware: Electronics or Hardware: Groundraft. \textbf{[Low (Kit), High (Shop), Expensive (Facility)]} 

\textbf{Utilitool:} This hand tool includes a specialized small nanobot generator. In its basic form, a utilitool is the size and shape of a large fountain pen. It can transform into almost any tool, however, from a wrench, knife, or powered screwdriver to a rotary grinder or pair of pliers. Some inexpensive utilitools are optimized for specialized tasks, like cooking or wilderness survival, but more expensive models become almost any imaginable hand tool. Utilitools are normally mentally controlled using the character’s basic mesh inserts. Characters without such implants can control the tool via voice commands and touch controls. Characters using a utilitool gain a +10 modifier to skills involving repairing or modifying devices with mechanical parts, opening locks, disarming alarm systems, or performing first aid. \textbf{[Bas]} 

\textbf{Viewers:} These small and highly advanced binoculars possess all the visual enhancement of specs (p. 325), but also provide 50x magnification. They also include a directional microphone that magnifies sound from the direction the viewers are pointed by a factor of 50. Viewers provide the user with a +30 bonus to all Perception Tests involving vision or hearing for the target they are aimed at. This bonus is not cumulative with bonuses from any other device or augmentation. \textbf{[Bas]} 



\section{Nanotechnology} \label{sec:nanotech} 

Nanotechnology is the precise manipulation of matter at the atomic level, typically using millions of microscale nanomachines. Nanotechnology transformed manufacturing, enabling new techniques and materials. The advent of nanofabrication --- building objects from the molecular level up --- transformed economies, allowing people to simply manufacture whatever they needed from raw materials. Nanotechnology is still a growing field, however, and has its limitations. While the TITANs unleashed self-replicating nanoswarms with the ability to transform or destroy anything through the power of geometric growth, such technology remains far beyond transhumanity’s grasp. 



\subsection{Basic nanotechnology} \label{sec:basic-nanotech} 

Basic nanotechnology is exceedingly widespread and used throughout the solar system, serving as the primary method of manufacturing for decades. The nanobots of basic nanotech are confined to delicate and specially-maintained environments like the insides of cornucopia machines or healing vats and cannot operate elsewhere. 

\subsubsection{Healing vats} 

Healing vats were the first type of nanotech medicine developed and remain the most powerful medical devices in common use. With the exception of a few exceptionally deadly nanoplagues, a healing vat can cure any disease and heal any injury. As long as the patient is alive when they are place in the healing vat, they will not only survive, but emerge without a scratch. A healing vat can even take a severed head (as long as it has been stabilized by medichines or nanotech first aid) and regrow an entire body based on the head’s genetics. If the patient’s body or medical records contain information about their implants, bioware, or advanced nanotechnology, all of these modifications are also fully restored. 

Few people suffer injuries serious enough to require a healing vat. Most are used as a safe and easy way to perform bodysculpting or to install implants or bioware. Healing vats use specialized nanomachines to either alter the patient’s body or integrate implants or bioware. One advantage of using a healing vat is that no additional healing time is needed, the patient leaves the vat fully recovered from the augmentation and ready to go. Every hospital, clinic, bodyshop, and augmentation parlor has several healing vats. The time required by a healing vat varies with the severity of the damage it is healing or the extent of the modification being made, as noted on the Healing Vat table, p. 327. \textbf{[Élevé]} 

\begin{table} \begin{tabularx}{\textwidth}{|X|X|} \hline

\multicolumn{2}{|c|}{Healing vat table} \\ \hline

\textbf{Injury}	&\textbf{Healing time} \\ \hline

Healing normal damage to a character who has taken 3 or fewer wounds.	&2 hours per wound (min. 1 hour for 0 wounds) \\ \hline

Restoring major lost body parts like arms or legs, or healing dying or nearly dead character who has taken 4 wounds.	&12 hours per wound \\ \hline

Restoring recently dead character who was placed in medical stasis to avoid death, but who is mostly intact.	&1 day per wound \\ \hline

Restoring recently dead character who is placed in medical stasis to avoid death, and who is missing most of their body.	&3 days per wound \\ \hline

\multicolumn{2}{|l|}{Augmentation} \\ \hline

Minor implants and bioware, minor cosmetic changes like alterations in skin color, eye color or shape, or hair color, texture or distribution, minor alterations to face shape or body fat distribution.	&1 heure \\ \hline

Major brain and neural implants, nanoware or bioware, sex changes, changing height by no more than 5\% or weight by no more than 20\%.	&12 hours \\ \hline

Major physical modifications like adding limbs or radical changes to height and weight.	&3 days \\ \hline

\end{tabularx} \label{tab:healing-vat} \end{table} 

\subsubsection{Nanodetectors} 

Nanodetectors are small devices that suck in air and micro debris in order to scan for and detect nanobots. Given that nanobots are so small, the density of nanobots in the area has a large impact on its success. The nanodetector has a base skill of 30 for detecting nanobots, modified by +30 if an active nanoswarm or hive is present, +0 if a nanoswarm or hive was active recently, and $-$10 for the presence of nanobots outside of a swarm or hive. Once a nanobot is detected it may be analyzed either by the user or the nanodetector’s AI, using Academics: Nanotechnology 30 skill. Nanodetectors are often worn and left on, set to alert the user if a hostile nanoswarm is detected. [Low] 

\subsubsection{Nanofabricators} 

Nanofabrication machines are universal assemblers that perform almost all of the manufacturing in the solar system. The user loads in raw materials and electronic plans and it can produce literally any manufactured good, from a weapon to an ultralight plane to a hot and delicious dinner. Many nanofabricators come equipped with a library of common-use blueprints (basic foods, standard clothing, common tools, etc.). Other blueprints must either be purchased online, selfprogrammed, or acquired through some other method (see \emph{Nanofabrication}, p. 284). The largest nanofabrication units are more than 10 meters on a side and are used to produce small consumer goods in bulk as well as building large devices like orbital transfer vehicles. 

The availability and legality of nanofabricators varies widely throughout the system. In the inner system and Jovian Republic, cornucopia machines are commonly restricted and sometimes illegal, with licenses only available to hypercorps, military units, and other officials and elites. In these habitats, only more limited fabbers are available to the general populace. Additionally, blueprints are licensed and protected by copyright laws, and many nanofabricators feature pre-programmed restrictions that prevent them from using unlicensed blueprints as well as from manufacturing weapons, explosives, or other restricted items. Among the autonomists of the outer system, however, nanofabricators are commonly accessible, shared by everyone, and unrestricted. 

For rules on creating goods in a nanofabricator, see \textit{Nanofabrication}, p. 284. 

\textbf{Desktop Cornucopia Machine:} Cornucopia machines (CMs) are general-purpose nanofabricators. The smallest CMs are desk-sized cubes approximately half a meter on a side with a volume of at least 40 liters. They can produce any small object, from tools to well-folded suits of clothing to handguns or a bowl of cereal. It is sometimes possible to assemble larger items, but they must be manufactured in smaller pieces and then assembled (likely requiring an appropriate Hardware Test). 

While users can purchase bulk raw materials, CMs also come equipped with a disassembler. The user loads garbage and other objects into the disassembler so that they can be turned into raw materials for the CM. All legally-available disassemblers only deconstruct non-living material. \textbf{[Cher]} 

\textbf{Fabber:} Fabbers are specialized nanofabricators, portable and considerably smaller than CMs. There are a wide variety of portable fabbers, including ones that can make any hand tool or small piece of personal electronics, ones that can turn any organic material into food and drink, and ones that can create any drug or medicine as well as bandages and specialized dressings. The most common fabbers have a volume of 4 liters. Larger hand tools and devices are produced as 2 or 3 separate parts that must be fitted together. Like CMs, fabbers also contain miniature disposal units. \textbf{[Modéré]} 

\textbf{Maker:} Makers are specially-designed to produce food and drink for the user. Raw materials can be provided by the addition of any water-containing liquid and collected biomass like leftover food, grass, dirt, dead animals, or transhuman waste. Some models are built into standard vacsuits. Makers can produce water and various flavored beverages, as well as ration bars or thick pudding-like edible gels. With adequate raw material, a maker can indefinitely provide food and drink for up to three transhumans. Most units, however, have a very limited range of flavors and textures that are widely considered to be fairly bad. Models with a wider and better range of flavors and textures are more expensive, but produce food that is considered adequate or occasionally good. \textbf{[Low to Moderate]} 

\textbf{Blueprints:} If you want a nanofabricator to make something, you need to instruct the device how to create it from the molecular level up. Such blueprints are available for almost every conceivable item out there. The cost of such blueprints typically exceeds the cost of purchasing the item, though factors like legality and quality may affect the cost as usual (see \emph{Acquiring Gear}, p. 296). \textbf{[One Cost Category Higher Than Item Cost]} 



\subsection{Advanced nanotechnology} \label{sec:advanced-nanotech} 

Advanced nanotechnology includes more recent developments. Like basic nanotech, advanced nanotechnology cannot self-replicate but the nanobots can function normally in most environments and are highly resistant to bacterial attacks and other environmental problems. Typical advanced nanotech consists of a generator --- known as a ``hive'' --- that produces nanobots as long as it is supplied with raw materials. Every such hive also includes a miniature disassembly unit and/or specialized nanomachines that collect raw materials for the generator. These hives produce nanobot swarms that are set loose to perform some function in the world. 

Examples of advanced nanotech include COTs (p. 315), medichines (p. 308), smart dust (p. 316), and utilitools (p. 326), among others. 

\textbf{General Hive:} General hives are capable of producing any conceivable type of nanobot with the right blueprints and/or programming. Even at their smallest size they are not really portable, with a minimum size being cubes 30 centimeters on a side and a volume of 25 liters. \textbf{[Cher]} 

\textbf{Specialized Hive:} Specialized hives are far more common than general hives, though they can produce only one type of nanomachines (i.e., choose one type of nanoswarm per hive). The smallest specialized hives are approximately the size of a 12-gauge shotgun shell or a large cherry tomato. \textbf{[Moderate, plus Cost of Programmed Nanoswarm]} 

\subsubsection{Ego bridges} 

Ego bridges are vat devices used for uploading and downloading minds. See \emph{Backups and Uploading}, p. 268, and \emph{Resleeving}, p. 271. \textbf{[Expensive]} 

\subsubsection{Nanoswarms and microswarms} 

Swarms are colonies of nanobots or larger microbots created in a hive, programmed with specific instructions, and then set free to perform a set task. Each swarm is composed of hundreds or thousands of nanobots or microbots, ranging in size from a microbe to a small insect. Nanobots are typically invisible to the naked eye, though they can be detected with a nanodetector (p. 326) or nanoscopic vision (p. 311). Microbots are more noticeable but still quite small, usually the size of a grain of sand or a dust mote, or occasionally as big as a flea. Individual bots in a swarm are directed by nanocomputers, with behavioral routines modeled on biological insect and animal swarms. These swarms stick together and work as a whole, communicating with nanoradios, nanolasers, or chemical cues, and sharing information between each bot in the swarm. Note that nanoswarms don’t invade inside living bodies (though they may attack externally) --- internal nano is handled by nanoware (p. 308), nanodrugs (p. 321), and nanotoxins (p. 324). 

Nanobots and microbots may be designed with all manner of miniaturized propulsion systems (see \emph{Mobility Systems}, p. 310), with the exception of ionic drives. They are powered by tiny batteries or solar cells. Their tiny sensors are very effective at allowing them to identify materials and objects, and so to target discriminatingly. Nanobots or microbots could, for example, be programmed to ignore metal objects, certain types of plants, specific morphs, females, or specific individuals. Swarms may either be released directly from a hive or from pre-packaged programmable canisters. 

Swarms must be programmed before they are released. The programming first determines how long the swarm is active. This timeframe is open-ended, though most swarms deteriorate into ineffectiveness after 2 weeks unless they are replenished by a hive. The programming then sets what area the swarm is to occupy. This is also open to interpretation and can vary from ``coat this person'' to ``spread out to a diameter of 20 meters'' to ``find the nearest chemical traces and track them to their source.'' Finally, programming sets any other parameters for the swarm’s mission --- for example, if it should ignore certain materials, if it should send a report at a predetermined time, or if it should self-destruct into harmless dust when it has completed a certain task. 

Programming is generally handled as a Simple Success Test using Programming (Nanoswarm) skill. Failure simply infers that the programming is imperfect, and so the swarm may not operate completely as planned. An actual Programming (Nanoswarm) Success Test is only called for if the swarm’s programming is substantially complex or if the character seeks to have the swarm act outside of its usual set functions. The bots in each swarm are specially equipped for the task they are designed for, however, so attempting to drastically repurpose a swarm may be difficult or pointless at the gamemaster’s discretion. 

Swarms may also be teleoperated, controlled, and/ or (re)programmed once they are released, via radio or laser link. 

Swarms are treated as a whole. The standard swarm size is enough to cover a 10 $\times$ 10 $\times$ 10 meter cube, and this is the standard ``unit'' of swarm released by a canister or hive. Swarms may be larger, but they are treated as individual swarm units. Each swarm has a Durability of 50 and is immune to wounds. Most attacks against a swarm simply floor 1 point of damage. Area-effect weapons, plasma rifles, and fire floor 1d10 damage, plasma grenades do full damage. EMP weapons (p. 340) are very effective against swarms, flooring 2d10 + 5 damage and a -10 modifier to all tests due to their damaging effects on the swarm’s communication abilities until repaired. Swarms are not affected by vacuum. 

\textbf{Cleaners:} This nanoswarm cleans, polishes, and removes dirt and stains. It may be used on an area, specific objects, or people. Some facilities employ permanent cleaner swarms to keep their area spotless. Cleaners may also be programmed to remove specific toxins, chemicals, or other hazardous substances in order to decontaminate an area. Covert operatives and criminals sometimes use cleaners to eliminate any evidence they may have left at a scene usable for forensics purposes, such as blood, hair, or anything that could be DNA-typed. \textbf{[Bas]} 

\textbf{Disassemblers:} Also known as smart corrosives, these nanobots break down any matter. Their advantage over common acids is that not only are they able to break down any material by using energy to disrupt chemical bonds, but that they can be programmed to take apart certain components while ignoring others, leaving them intact. Disassemblers are a common weapon used against synthmorphs, eating away their components without having to worry about accidentally splashing biomorphs. Upon contact, these nanobots floor 1d10 $\div$ 2 damage (round up) per Action Turn. Accumulated damage counts as a wound when the Wound Threshold is reached. Both Energy and Kinetic armor protect against this damage, but these armors are eaten away as well, so the Armor Value is reduced by the soaked DV. \textbf{[Élevé]} 

\textbf{Engineers:} Engineer microswarms are used for various construction purposes: erecting walls, digging tunnels, sealing holes, reinforcing foundations, and so on. \textbf{[Modéré]} 

Fixers: This is the nanoswarm version of repair spray (p. 333). \textbf{[Moderate]} 

\textbf{Injectors:} Injector microswarms are equipped with tiny needles and a drug payload. A biological target affected by an injector swarm suffers 1 point of damage and the effects of the carried drug, chemical, or toxin. \textbf{[Modéré]} 

\textbf{Gardeners:} This microswarm is useful for a number of agricultural purposes: killing weeds, planting seeds, trimming plants, pollinating, and even harvesting small items. It may also be programmed to simply defoliate an area. \textbf{[Modéré]} 

\textbf{Guardians:} Guardians watch for and attack other unauthorized swarms. Guardians floor 1d10 $\div$ 2 damage (round up) on other swarms they come into contact with per Action Turn. \textbf{[Modéré]} 

\textbf{Proteans:} This nanoswarm is designed to disassemble other materials and objects and to create a single specific, pre-programmed device from the components (much like a specialized nanofabricator). The proteans must be able to scavenge appropriate raw materials (for example, to create a metallic device the nanobots must transform something else made of metal). The construction time takes 1 hour per cost category of the item (1 hour for a Trivial cost item, 2 hours for Low, etc.). \textbf{[Élevé]} 

\textbf{Saboteurs:} Sab nanobots are designed to infiltrate electronics or machinery and sabotage them in small but difficult to discern ways: severing connections, disabling components, gumming up moving parts, etc. Saboteurs floor damage on devices similar to disassemblers, but the target is not destroyed and such damage is not immediately obvious. They floor 1d10 $\div$ 2 points of damage to synthmorphs, bots, and other devices every Action Turn. Armor has no effect, but accumulated damage counts as a wound when the Wound Threshold is reached. \textbf{[Élevé]} 

\textbf{Scouts:} A scout nanoswarm will systematically map and explore an area, collecting samples of all materials and substances it encounters. The samples are carried back to the hive or canister and chemically analyzed. Scouts can also be used for forensic purposes, collecting DNA samples, analyzing chemical residues, and examining other evidence. \textbf{[Élevé]} 

\textbf{Taggants:} Taggants seek to lodge themselves onto everything in their area of dispersal. Each carries a unique identifier, so that if it is found later, the tagged person or object can be linked back to the point they were tagged. Taggants can be programmed to remain silent, only responding to query broadcasts made with the proper crypto codes, or they can be programmed to broadcast their location back to the deployer via the mesh. \textbf{[Bas]} 



\subsection{Pets} \label{sec:pets} 

These partially-uplifted and bio-engineered animals have rudimentary intelligence and limited communication skills. They make for fine companions and helpers. 

\textbf{Fur Coat:} A so-called ``fur coat'' is outerwear made from a living primitive organism. The creature’s skin, fur, or scales are real. The organism is cultivated from transgenic stocks and grown around molds into clothing shapes, often with actual usefulness: polar bear parkas, seal diving suits, porcupine coats, etc. Fur coats are modified with wireless controls and haptic systems, so they can be made to move, shiver, massage, or prickle up on command. \textbf{[Bas]} 

\textbf{Smart Dogs:} Commonly used as discriminatory guardians, smart dogs are sometimes enhanced with combative bioware or cybernetics. \textbf{[Modéré]} 

\textbf{Smart Monkey:} Commonly used by criminal groups for minor larceny such as pickpocketing, smart monkeys can be useful and intelligent aides. \textbf{[Modéré]} 

\textbf{Smart Rats:} These upgrades of the common Norwegian rat are clever and dexterous, and they easily fit into a pocket or hood. \textbf{[Bas]} 

\textbf{Space Roach:} Grown to the size of a small dog, these insects are often biosculpted for bright colors and patterns. They are useful for minor janitorial duties. \textbf{[Bas]} 

\subsection{Scavenger tech} \label{sec:scavenger-tech} 

This technology is often employed by gatecrashers, space scavengers, and Firewall teams during missions. 

\textbf{Disassembly Tools:} These tools are useful for salvage ops, breaking down wrecks, or dissembling anything from a habitat room to a vehicle or synthmorph. They include plasma torches, laser cutters, pneumatic jaws, and smart tools like spanners and wrenches that can be adapted to a wide array of connections and fittings. \textbf{[Élevé]} 

\textbf{Mobile Lab:} The mobile lab is a handheld device that contains all different types of sensors to investigate organic and inorganic liquid, gaseous, and solid components (from soil to tissue samples) and compositions. It performs material analysis using different methods of spectrometry and biochemical testing, comparing results to a built-in database of element and compound spectra. Its built in AI comes equipped with Academic: Chemistry 30. \textbf{[Modéré]} 

\textbf{Specimen Container:} This capsule container is designed to hold samples of any sort (chemical, biological, etc.) in near stasis. It can be programmed to reproduce whatever conditions the user specifies, from cryogenic freezing to extreme heat, or even vacuum or high-pressure atmosphere. \textbf{[Bas]} 

\textbf{Superthermite Charges:} These powerful and highly stable demolition charges are made from a combination of nanometals and metal oxides. A single charge can be used to create an explosive blast flooring 2d10+5 damage. This charge can be shaped with a successful Demolitions Test, focusing the blast in a particular 90-degree direction (for example, to blow through a door). This triples the damage of the blast in the focused direction; in all other directions, the damage is reduced to 1/3rd (round down). Multiple charges apply a cumulative effect. \textbf{[Modéré]} 

\subsection{Services} \label{sec:services} 

\textbf{Anonymous Accounts:} These accounts are crucial for anyone who wants to be discreet with their online transactions. See \emph{Anonymous Account Services}, p. 252. [Moderate] 

\textbf{Backup:} A single, one-time backup without insurance is sometimes all the poor can afford, hoping that they can buy backup insurance later or that someone that cared about them will see to a resleeving. \textbf{[Modéré]} 

\textbf{Backup Insurance:} In the event of verifiable death, or after a set period of being missing, backup insurance will arrange for your cortical stack to be retrieved and your ego downloaded into another morph. If the cortical stack cannot be retrieved, your most recent backup is used. Most policies require that the holder provide a backup to be uploaded into secure storage at least twice a year. This industry works in a manner similar to insurance underwriting in terms of cost and individuals engaged in high risk professions can expect to pay a premium for the service. Additionally, attempts to retrieve a cortical stack are minimal unless one wants to pay for some extra effort (a thriving industry of paramilitary ego-repo operatives exists for this purpose). \textbf{[Low to Moderate per month]} 

\textbf{Body Bank:} People who are egocasting to another station but whom hope to download back into the same body they have before when they return may put the morph on ice for the duration of their absence. \textbf{[Moderate per month]} 

\textbf{Bot/Pod Rental:} When you need a helping hand or a personal companion for a day or two, renting a bot or pod is often the way to go. \textbf{[Moderate per day]} 

\textbf{Egocasting:} This is the use of a farcaster to transmit an ego/infomorph. Farcasting is not cheap, and the cost is impacted by factors such as distance to receiver station and priority service (paying extra to get bumped ahead in line). \textbf{[Cher]} 

\textbf{Fake Ego ID:} This forged ID will pass in most inner system and Jovian Republic habitats, and sometimes others. \textbf{[Élevé]} 

\textbf{Morph Brokerage:} Acquiring a new morph is not always easy and is affected by factors such as the type of morph, sought-after enhancements/customizations, and local availability. Numerous brokerage services exist to find you what you need, or close to it. With enough lead-time, it may be possible to grow a pod that closely imitates your morph of choice. A willingness to accept used/traded-in morphs helps to reduce costs. For more details, see \emph{Morph Brokerage}, p. 276. 

\textbf{Psychosurgery:} A character can purchase time in an immersive high-fidelity simulspace with expert care from psychosurgeons and AIs in order to cope with derangements and disorders that build up as a result of existing in a transhuman universe. For an additional price the procedure can be time shifted to speed up the relative time within the simulspace. For more details, see \emph{Mental Healing and Psychotherapy}, p. 215, and \emph{Psychosurgery}, p. 229. \textbf{[Moderate per month]} 

\textbf{Simulspace Subscription:} This will by you access to the simulspace of your choice, whether you want it for a private meeting/vacation or to play the latest and hottest VR game. \textbf{[Low (single use/1 day) to Moderate (monthly subscription)]} 

Space Travel: Space transport cost depends on a number of factors like distance, quality of lodgings, and how much cargo you’re bringing with. At the low end, an intra-habitat shuttle trip within the same cluster, or a trip to or from a planetary body’s surface and orbit, is not cheap but affordable \textbf{[High]}. Just about anything else is progressively more costly. \textbf{[Cher]} 

\subsection{Logiciels} \label{sec:software} 

For information on using software, see the \emph{Mesh} chapter, p. 234. 

\subsubsection{Programs} 

These programs can be run on any computerized device. 

\textbf{AR Illusions: }These databases of AR clips can be used to create realistic illusions in someone’s entoptic display. See \emph{Augmented Reality Illusions}, p. 259. \textbf{[Modéré]} 

\textbf{Exploit:} Exploits are hacker tools that take advantage of known vulnerabilities in other software. They are required for intrusion attempts (p. 254). \textbf{[Élevé]} 

\textbf{Facial/Image Recognition:} This program can be used to take an image and run a pattern-matching search among public archives. Similar version of this program exist for other biometrics: gait recognition, vocal recognition, etc. \textbf{[Low]} 

\textbf{Firewall:} This program protects a device from hostile intrusion. Every system comes with a standard version of this software by default. \textbf{[Bas]} 

\textbf{Sniffer:} Sniffer programs collect all of the transmission that pass to, from, or through the device they are running on. See \emph{Sniffing}, p. 252. \textbf{[Modéré]} 

\emph{Spoof:} Spoof is a hacker tool used to fake commands and transmissions, making them seem as if they came from another source. See \emph{Spoofing Authentication}, p. 255. \emph{[Moderate]} 

\textbf{Tactical Networks:} These programs allow people in the same squad to share tactical data in real-time. See \emph{Tactical Networks}, p. 205. \textbf{[Modéré]} 

\textbf{Tracking:} This software is used to track people by their presence online. See \emph{Scanning, Tracking, and Monitoring}, p. 251. \textbf{[Modéré]} 

\textbf{XP:} Experience playback recordings are clips of someone else’s experiences. Depending on the content, some XP (porn, snuff, crime, etc.) may be restricted in certain jurisdictions. Some XP clips are intentionally modified so that their emotive tracks are more intense, giving the viewer a greater thrill. \textbf{[Low to High]} 

\subsubsection{AIs and muses} 

Every character starts with a personal muse for free. Many devices also come with pre-installed AIs, capable of helping the user, responding to commands, or even operating the device on their own. Rules for AIs can be found on p. 264. 

Below are some commonly available AI programs. Unless otherwise noted, these AIs have aptitudes of 10. These AIs may also be equipped with skillsofts (p. 332). 

\textbf{Bot/Vehicle AI:} These AIs are designed to be capable of piloting the robot/vehicle without transhuman assistance. REF 20. Skills: Hardware: Electronics 20, Infosec 20, Interests: [Bot/Vehicle] Specs 80, Interface 40, Research 20, Perception 40, Pilot: [appropriate field] 40. \textbf{[Élevé]} 

\textbf{Device AI:} These AIs are designed to operate a particular device without transhuman assistance. Skills: Infosec 20, Interests: [Device] Specs 80, Interface 30 (Device Specialization), Programming 20, Research 20, Perception 20. \textbf{[Modéré]} 

\textbf{Kaos AI:} Kaos AIs are used by hackers and covert ops teams to create distractions and sabotage systems. REF 20. Skills: Hardware: Electronics 40, Infosec 40, Interface 40, Professional: Security System 80, Programming 40, Research 20, Perception 30 plus one weapon skill at 40. \textbf{[Cher]} 

\textbf{Security AI:} Security AIs provide overwatch for electronic systems. Skills: Hardware: Electronics 30, Infosec 40, Interface 40, Professional: Security Systems 80, Programming 40, Research 20, Perception 30, plus one weapon skill at 40. \textbf{[Élevé]} 

\textbf{Standard Muse:} Muses are digital entities that have been designed as personal assistants and lifelong companions for transhumans (see \emph{AIs and Muses}, p. 264). INT 20. Skills: Academics: Psychology 60, Hardware: Electronics 30, Infosec 30, Interface 40, Professional: Accounting 60, Programming 20, Research 30, Perception 30, plus three other Knowledge skills at 40. \textbf{[Élevé]} 

\subsubsection{Scorchers} 

Scorchers are damaging neurofeedback programs used to torment hacked cyberbrains (p. 261). 

\textbf{Bedlam:} Bedlam programs assault the ego with traumatic mental input, inflicting mental stress. Victims are overwhelmed with horrific, monstrous, sanity-ripping sensory and emotional input. Each attack floors 1d10 SV. \textbf{[Élevé]} 

\textbf{Cauterizer:} This scorch program rips into the ego with destructive neurofeedback routines. Each attack with a cauterizer inflicts 1d10 + 5 DV on the target ego. This damage is reflected as digitized neurological damage. \textbf{[Élevé]} 

\textbf{Nightmare:} Nightmare programs trigger anxiety and panic attacks within the victim by stimulating the neural circuitry representing the amygdala and hippocampus. The target ego must make a WIL $\times$ 2 Test. 

If they succeed, they are shaken but otherwise unaffected, suffering a $-$10 modifier to all actions until the end of the next Action Turn. If they fail, they suffers 1d10 $\div$ 2 stress damage and are overcome with panic. This causes them either to blindly flee, have a nervous breakdown, or cower in frozen shock (gamemaster’s discretion). This panic episode lasts for 1 Action Turn per 10 points of MoF. \textbf{[Élevé]} 

\textbf{Shutter:} Shutters target the victim’s sensory cortices, inflicting a $-$30 modifier to one chosen sense. Double this modifier if the attacking hacker scored an Excellent Success. This modifier reduces at the rate of 10 points per Action Turn. \textbf{[Élevé]} 

\textbf{Spasm:} Spasm programs are design to incapacitate the ego with excruciating pain. Affected targets must immediately make a WIL $\times$ 2 Test. If they fail, they immediately convulse, are disabled, and writhe in agony for 1 Action Turns per 10 full points of MoF. If they succeed, they still suffer a $-$30 modifier to all actions, which reduces at the rate of 10 points per Action Turn. Due to the nature of the delivery, pain tolerance of any sort has no effect. \textbf{[Élevé]} 

\subsubsection{Skillsofts} 

Skillsofts are used with skillware implants (p. 309) pour communiquer. 

\textbf{Standard Skillsoft:} These programs provide the character with a rating of up to 40 in a single Active skill. \textbf{[Élevé]} 

\subsection{Survival gear} \label{sec:survival-gear} 

The following gear is often critical to the survival of soldiers, spies, criminals, gatecrashers, emergency service personnel, and others who regularly venture into unsafe or unfamiliar regions. 

\textbf{Breadcrumb Positioning System:} This worn device leaves micro ``breadcrumbs'' behind as the character moves. These devices interact with mesh inserts (or ectos) as long as they are within range (50 meters), allowing the user to map their position in relation to the breadcrumb trail. This is useful in derelict habitats, wilderness, and other areas where there is no local functioning mesh, and is helpful both for mapping and for finding one’s way back. \textbf{[Bas]} 

\textbf{Electrogravitics Net:} Also called a safety net, this failsafe system uses electric fields to counter gravity when falling. While the system is not able to actually levitate heavy objects, it will slow down a fall enough that the user can land safely if the gravitational force is not too high (the fall height is not greater than 50 meters in 1G). Generating these electric fields consumes a lot of energy, so the net is only charged for one use only and needs to be recharged afterwards. \textbf{[Modéré]} 

\textbf{Electronic Rope:} The fibers in this rope can be controlled electronically, making it move in a snakelike fashion, stiffen up, and even wrap around objects. Typically comes in a 50- meter length capable of supporting 250 kg. \textbf{[Bas]} 

\textbf{Emergency Bubble:} Commonly used as a last resort ``life raft'' on spaceships, an emergency bubble is made of advanced smart materials and comes in a portable package that can be quickly inflated (1 Action Turn) around the user, usually inside an airlock. The bubble has a 5-meter diameter and can comfortably accommodate 4 people. It maintains 1 atmosphere of pressure in a vacuum, protect the inhabitants from temperatures ranging from $-$175 to 140$^{\circ}$ C, and provide light, breathable air and water and food recycling for up to four human-sized inhabitants, using its built in maker (p. 327). It features a simple airlock, carries an emergency distress beacon (below), and can be transparent, opaque, or polarized. It is powered by a small nuclear battery and also includes comfortable inflatable furniture. \textbf{[Modéré]} 

\textbf{Emergency Distress Beacon:} This small but powerful transmitter is powered by a nuclear battery and will broadcast any programmed distress call for years. Though portable and medium-sized, this beacon has a range of 500 km in urban areas and 5,000 km elsewhere. \textbf{[Modéré]} 

\textbf{Flashlight:} These handheld, wearable, or portable lights can display light in the normal visual spectrum, infrared, or ultraviolet, as desired. \textbf{[Trivial]} 

\textbf{Nanobandage:} Characters without medichines must rely on external sources of healing. The most common option is the nanobandage --- a plum-sized advanced nanotechnology generator built into a reusable, selfsterilizing bandage. It can treat all forms of injury and illness, from poisoning to burns to trauma. Characters simply apply the bandage to the wound and let the nanobots do the work. It removes pain and discomfort and speeds healing (see \emph{Biomorph Healing}, p. 208). For especially severe injuries, physical first aid such as setting bones and removing projectiles may be necessary (gamemaster’s choice). If the wounds are too severe (the patient has suffered more than five wounds), the unit places the patient in medical stasis and radios for emergency services. \textbf{[Trivial]} 

\textbf{Repair Spray:} This nanobot generator creates nanobots designed to repair synthmorphs, vehicles, and other common objects. Repair spray contains the specifications and plans for almost all commonly used synthmorphs and devices and is a ubiquitous household item. If it does not contain the specifications for something it is being used to repair, it must query the object’s voice for these details, otherwise it cannot repair it. Simply touch it to the damaged area, push the button on top, and it sprays out a number of nanobots sufficient to make repairs. These nanobots repair 1d10 points of damage per 2 hours. Once all damage is restored, the nanobots repair wounds at the rate of 1 per day. Repair spray also cleans and polishes items and returns them to a pristine and new state. Repair spray is not effective on any object with more than 3 wounds, but it provides a +30 to all repair rolls on anything too badly damaged for it to fully repair (see \emph{Synthmorph and Object Repair}, p. 208). \textbf{[Bas]} 

\textbf{Shelter Dome:} A variant of the emergency bubble, this package unfolds into a dome with a 2.5-meter ceiling and a floor 4 meters across. To safely use this shelter, it must be staked down to the surface it is placed on. \textbf{[Modéré]} 

\textbf{Spindle:} A spindle is an advanced nanotechnology generator that produces a super-strong cable. It can produce up to 2 kilometers of 0.2 millimeter diameter line than can support up to 250 kilograms before it needs more raw materials. The spindle can produce up to 20 meters of cable every second. It can produce line in a continuous length or cut the cable it produces to any length. Spindles can also reabsorb their cable, retracting it at a rate of 5 m per second. As long as it is recharged and has small amounts of additional material added every 1,000 hours of use, a spindle can keep producing and retracting cable indefinitely. By setting the maximum production speed at 10 m/ second a character with a spindle can safely jump off a building and land safely, using the cable to slow their descent. \textbf{[Modéré]} 

\textbf{Spindle Climber:} This device attaches to a spindle and transforms it into a highly effective climbing device. The spindle climber has two functions. First, it attaches hardened tips to the spindle’s cable and fires it at high speed, up to 50 meters, with sufficient force to imbed the tip into almost any sufficiently durable surface. Second, the spindle climber can pull itself and up to 250 kg up the cable at a speed of up to 2 m/sec. A spindle climber has enough power to shoot and pull up the cable 50 times before it must be recharged. A spindle fits inside a spindle climber. \textbf{[Bas]} 

\subsubsection{Vacuum suits} 

Most vacuum suits are skin-tight garments that use the pressure of their advanced smartfabrics on the wearer’s body to resist vacuum. When the wearer is in a breathable atmosphere, the smartfabric also loosens the suits to serve as ordinary clothing or be easily put on or taken off. In all cases, the suits can become skin-tight within 3 Action Turns. All vacsuits contain advanced rebreather units capable of maintaining a breathable atmosphere for several hours or days. 

\textbf{Light Vacsuit:} Everyone living in a sealed habitat owns at least one of these suits. They come in a variety of forms. Inexpensive versions are typically lightweight jumpsuits made of simple smart fabric that adjusts to fit and folds up small enough to fit into a coat pocket. The best models include suits of high-end smart clothing that can transform into a vacsuit and an advanced nanotech generator the size of a large orange that deploy nanobots that cover the user and fit together into a vacuum suit. Both can transform into a vacsuit in 2 full Action Turns and do so either on command or if their sensors reveal that life support is needed. 

All models include a lightweight belt or torc containing a miniature oxygen tank and advanced rebreather unit that provides 3 hours of air. However, the suits contain no food or water recycling. All models include an ecto (p. 325) and a headlight, but typically little else beyond atmosphere sensors to let the wearer know when it is safe to take off the suit. They protect the wearer from temperatures from $-$75 to 100$^{\circ}$ C. These vacuum suits also provide an Armor rating of 5/5 and instantly self-seal breaches unless more than 20 points of damage are inflicted at once. \textbf{[Low, Moderate for smartfabric suits]} 

\textbf{Standard Vacsuit:} These suits resemble light vacsuits made from thicker and more durable materials that resist tearing and provides the wearer with light armor. They are fitted with more substantial life support belts that includes a maker (p. 327) capable of recycling all wastes and producing air for up to 48 hours and food and water indefinitely. The best suits are made of smart materials that can transform from standard clothing to vacuum suits in a single Action Turn, and will do so automatically if life support is needed. Each suit also contains an ecto (p. 325), un booster  radio (p. 313), et des capteurs équivalents à des poussières (see p. 325). These suits have an Armor rating of 7/7 and protect the wearer from temperatures from $-$175 to 140$^{\circ}$ C. They can almost instantly seal any hole unless more than 30 points of damage are inflicted at once. \textbf{[Moderate, High for smartfabric suits]} 

\textbf{Hard Suit:} This heavy-duty suit can almost be considered a miniature space ship. Hard Suits look like large metallic ovals with jointed arms and legs. They are quite heavy, but the user can move relatively easily by using servo assist motors in all the major joints of the arms and legs. Unlike other vacsuits, they are solid and can resist both vacuum and up to 100 atmospheres of external pressure. Characters wearing hard suits can safely explore the upper atmosphere of a gas giant. They are well armored against punctures and radiation and possess miniature plasma thrusters capable of delivering 0.01G for 10 hours. A built-in high quality maker produces sufficient food, air, and water that a user can remain in a hard suit indefinitely. Explorers have used them continuously for up to 2 months. Their gloves incorporate smart materials that allow each hand to use the equivalent of a utilitool (p. 326). Hard suits also contain radios and sensors equivalent to those on standard vacsuits. These suits have an Armor rating of 15/15, are maintained by a fixer nanohive (p. 329), and are instantly self-sealing of any breach unless more than 30 points of damage are inflicted at once. They protect the wearer from temperatures of $-$200 to 180$^{\circ}$ C. \textbf{[High]} 



\section{Armes} \label{sec:weapons} 

A wide range of weapons are available in \emph{Eclipse Phase}, from the primitive to the technologically advanced. 



\subsection{Melee weapons} \label{sec:melee-weapons} 

Melee weapons are those wielded by hand (or foot) in melee combat. They are divided by the skill be which they are used. 

\subsubsection{Blades} 

These weapons are wielded with Blades skill. 

\textbf{Diamond Axe:} Commonly found on many habitats for fire and emergency purposes, axes require two hands to wield. Their blades are diamond-coated for superior cutting ability. \textbf{[Bas]} 

\textbf{Flex Cutter:} The blade of this machete-like weapon is made of a memory polymer. When deactivated, the blade is limp and flexible, and may even be rolled up or otherwise easily concealed. When activated, however, the blade stiffens and sharpens into a vicious slashing weapon. \textbf{[Bas]} 

\textbf{Knife:} A standard cutting implement, still carried by many. \textbf{[Trivial]} 

\textbf{Monofilament Sword:} Though swords are rather archaic in the time of Eclipse Phase, a few eccentrics take advantage of modern versions with a selfsharpening near-monomolecular edge, easily capable of slicing through metal or limbs. \textbf{[Bas]} 

\textbf{Vibroblade:} These buzzing electronic blades vibrate at a high frequency for extra cutting ability. This has little extra effect when stabbing or slashing, but provides an extra $-$3 AP and +2d10 damage when carefully sawing through something. \textbf{[Bas]} 

\textbf{Wasp Knife:} Wasp knives are equipped with a canister in their handle. The common use is to fill these canisters with pressured air, which inflates inside the target. This is potentially lethal in vacuum or pressurized environments (like underwater), as the gas bursts out of the body cavity to escape (+2d10 damage in such situations). Wasp knives may also be loaded with chemicals, drugs, or nanobots. The target must be damaged for the canister’s contents to affect them. \textbf{[Bas]} 

\subsubsection{Clubs} 

Characters use Clubs skill when using these weapons. 

\textbf{Club:} Clubs encompasses a wide range of one-handed blunt objects, from saps to sticks to pipes. \textbf{[Trivial]} 

\textbf{Extendable Baton:} This hardened composite baton retracts into its handle for easy carrying, storage, or concealment. Extending it simply requires a flick or an electronic signal. \textbf{[Trivial]} 

\textbf{Shock Baton:} Shock batons are standard clubs used for policing duties, but when activated they also deliver an electric shock to struck targets (see Shock Attacks, p. 204). \textbf{[Bas]} 

\subsubsection{Exotic melee weapons} 

Unusual weapons requires a specific Exotic Melee field skill to use. 

\textbf{Monowire Garrote:} This assassin’s weapon features a dangerous monomolecular wire wrapped around a contained spool with two handles. One handle grips the spool, while the other extends the wire so that it may be used to wrap around targets (typically necks or limbs) and slice through them when pulled. Monofilament tensile strength is weak, however, usually breaking after one use. \textbf{[Modéré]} 

\subsubsection{Unarmed} 

These weapons are wielded using Unarmed Combat skill. 

\textbf{Densiplast Gloves:} These gloves extra-harden when activated, for extra punch. \textbf{[Trivial]} 

\textbf{Shock Gloves:} When activated, these gloves deliver an incapacitating shock along with every punch or grab. Note that the effect is the same whether wearing one glove or two. \textbf{[Bas]} 







\begin{table} \begin{tabularx}{\textwidth}{|l|X|l|l|} \hline

\multicolumn{4}{|c|}{\textbf{Melee weapons --- Blades, Clubs, Exotic, Unarmed}} \\ \hline

&\textbf{Armor penetration (AP)}	&\textbf{Valeur de dégats (VD)}	&\textbf{Average DV} \\ \hline

\multicolumn{4}{|l|}{\emph{Blades}} \\ \hline

Diamond Ax	&$-$3	&2d10 + 3 + (SOM $\div$ 10)	&14 + (SOM $\div$ 10) \\ \hline

Flex Cutter	&$-$1	&1d10 + 3 + (SOM $\div$ 10)	&8 + (SOM $\div$ 10) \\ \hline

Knife	&$-$1	&1d10 + 2 + (SOM $\div$ 10)	&7 + (SOM $\div$ 10) \\ \hline

Monofilament Sword	&$-$4	&2d10 + 2 + (SOM $\div$ 10)	&13 + (SOM $\div$ 10) \\ \hline

Vibroblade	&$-$2	&2d10 + (SOM $\div$ 10)	&11 + (SOM $\div$ 10) \\ \hline

Wasp Knife	&$-$1	&1d10 + 2 + (SOM $\div$ 10)	&7 + (SOM $\div$ 10) \\ \hline

\multicolumn{4}{|l|}{\emph{Clubs}} \\ \hline

Club	&--- &1d10 + 2 + (SOM $\div$ 10)	&7 + (SOM $\div$ 10) \\ \hline

Extendable Baton	&--- &1d10 + 2 + (SOM $\div$ 10)	&7 + (SOM $\div$ 10) \\ \hline

Shock Baton	&--- &1d10 + 2 + (SOM $\div$ 10) + shock (p. 204)	&7 + (SOM $\div$ 10) \\ \hline

\multicolumn{4}{|l|}{\emph{Exotic melee weapons}} \\ \hline

Monowire Garrote	&$-$8	&3d10	&16 \\ \hline

\multicolumn{4}{|l|}{\emph{Unarmed}} \\ \hline

Bioware Claws (p. 304)	&$-$1	&1d10 + 1 + (SOM $\div$ 10)	&6 + (SOM $\div$ 10) \\ \hline

Cyberclaws (p. 307)	&$-$2	&1d10 + 3 + (SOM $\div$ 10)	&8 + (SOM $\div$ 10) \\ \hline

Densiplast Gloves	&--- &1d10 + 2 + (SOM $\div$ 10)	&7 + (SOM $\div$ 10) \\ \hline

Eelware (p. 304)	&--- &shock (p. 204)	--- \\ \hline

Shock Gloves	&--- &1d10 + (SOM $\div$ 10) + shock (p. 204)	&5 + (SOM $\div$ 10) \\ \hline

Unarmed	&--- &1d10 + (SOM $\div$ 10)	&5 + (SOM $\div$ 10) \\ \hline

\end{tabularx} \label{tab:meleeweapons} \end{table} 



\subsection{Kinetic weapons} \label{sec:kinetic-weapons} 

Kinetic weapons damage the target by firing a hard impact projectile at high-velocities. Slugthrowers have evolved from the mechanical firearms of the early 21st century, however, and now fall into two categories: chemical firearms and railguns. Though their mechanisms for firing are different, they are roughly similar in effect. Railguns have a higher penetration and inflict more damage, which is offset by more limited ammunition choices. While modern beam weapons have their uses, they rarely match the punch of kinetic weapons, therefore slugthrowers are still perceived as the most versatile and effective weapon system. 

Kinetic weapons are constructed from lightweight, reinforced plastoceramic materials, which are easily produced even without nanofabrication. By default, modern kinetic weapons are ambidextrous but more importantly feature safety and smartlink systems (p. 342) that automatically connect to the wielder’s mesh inserts for firing assistance, target recognition, and tactical networking. 

The wielder of a firearm or railgun uses Kinetic Weapons skill. For information on firing modes, see p. 198. For different ammunition types, see p. 336. Ranges are listed on p. 203. 

\subsubsection{Firearms} 

Modern chemical firearms use caseless ammunition that is auto-loaded from a magazine. They are effectively recoilless (thanks to rheological smart fluid mechanisms) and electronically fired (an electric charge vaporizes the propellant, using the expanding steam and plasma to eject and accelerate the projectile). Note that older, pre-Fall firearms still exist and are traded by black marketeers, though they use outdated system such as liquid propellants or cased ammunition. At the gamemaster’s discretion, these relics may suffer shorter ranges, less penetration, fewer firing modes, or reduced damage. 

\textbf{Pistols:} Pistols are small-sized (p. 297) and designed for one-hand use. Light pistols sacrifice penetrating ability for concealability. Heavy pistols focus on stopping power, with medium pistols occupying a middle ground. All versions fire in semi-automatic, burst-fire, and full-auto modes. \textbf{[Bas]} 

\textbf{Submachine Guns:} SMGs use pistol ammunition, but are medium-sized (p. 297) and may fire in semi-auto, burst fire, or full auto modes. They typically are designed in a bullpup configuration for close quarters operations and are ideal for tactical and strike teams. \textbf{[Modéré]} 

\textbf{Automatic rifles:} Automatic rifles use rifle ammunition and have greater range and penetration than SMGs. They fire in semi-auto, burst fire, or full auto modes. They are two-handed weapons. \textbf{[Modéré]} 

\textbf{Sniper rifle:} Sniper rifles are optimized for range, accuracy, penetration, and stopping power. They fire in semi-auto, burst fire, or full auto modes, and are two-handed weapons. \textbf{[Élevé]} 

\textbf{Machine Gun:} Machine guns are heavy weapons, typically mounted, and intended to provide continuous fire for support or suppressive purposes. They fire in burst fire or full auto modes, and are twohanded weapons. \textbf{[Élevé]} 

\begin{table} \begin{tabularx}{\textwidth}{|l|X|X|X|l|l|} \hline

\multicolumn{6}{|c|}{\textbf{Kinetic weapons --- Firearms}} \\ \hline

&\textbf{Armor penetration (AP)}	&\textbf{Valeur de dégats (VD)}	&\textbf{VD moyenne}	&\textbf{Firing modes}	&\textbf{Ammo} \\ \hline

Pistolet Léger	&--- &2d10	&11	&SA, BF, FA	&10 \\ \hline

Pistolet	&$-$2	&2d10 + 2	&13	&SA, BF, FA	&12 \\ \hline

Pistolet Lourd	&$-$4	&2d10 + 4	&15	&SA, BF, FA	&16 \\ \hline

Submachine Gun	&$-$2	&2d10 + 3	&14	&SA, BF, FA	&20 \\ \hline

Automatic rifle	&$-$6	&2d10 + 6	&17	&SA, BF, FA	&30 \\ \hline

Sniper rifle	&$-$12	&2d10 + 10	&21	&SA	&40 \\ \hline

Mitrailleuse	&$-$6	&2d10 + 6	&17	&BF, FA	&50 \\ \hline

\end{tabularx} \label{tab:kinetic-firearms} \end{table} 

\subsubsection{Railguns} 

Railguns use a pair of electromagnetic rails to slide and accelerate a non-explosive conductive projectile at extremely high velocities (Mach 6+) to create an overwhelming, penetrating attack. The kinetic energy of the projectile exceeds that of an explosive-filled shell of greater mass and creates shock and heat waves upon impact that shatter and incinerate the target, or portions of it. While railguns are more potent than firearms, the ammunition choices are limited as the projectile must be conductive and able to survive both acceleration and heat created in the process due to friction. Nanofabrication allows railguns to be manufactured on the personal weapons scale while high-energy portable batteries provide the power to fire them. Railgun operation is silent except for the supersonic crack of the projectile. 

Railguns are available in the same models as firearms (pistols through machine guns), with the following modifications: 

\begin{itemize} \item Increase AP by $-$3 \item Increase damage by +2 \item Increase the maximum for each range category by x1.5 \item Increase Cost category by one \item Railguns may only use regular and armor-piercing ammunition \item Railguns also require battery power for each shot. Standard railgun batteries hold enough power for 200 shots, after which they must be recharged at the rate of 20 points per hour. \end{itemize} 

\begin{table} \begin{tabularx}{\textwidth}{|l|X|X|X|X|l|} \hline

\multicolumn{6}{|c|}{\textbf{Kinetic weapons --- Railguns}} \\ \hline

&\textbf{Armor penetration (AP)}	&\textbf{Valeur de dégats (VD)}	&\textbf{VD moyenne}	&\textbf{Firing modes}	&\textbf{Ammo} \\ \hline

Pistolet Léger	&$-$3	&2d10 + 2	&13	&SA, BF, FA	&10 \\ \hline

Pistolet	&$-$5	&2d10 + 4	&15	&SA, BF, FA	&12 \\ \hline

Heavy Pisto	&$-$7	&2d10 + 6	&17	&SA, BF, FA	&16 \\ \hline

Submachine Gun &$-$5	&2d10 + 5	&16	&SA, BF, FA	&20 \\ \hline

Automatic Rifle	&$-$9	&2d10 + 8	&19	&SA, BF, FA	&30 \\ \hline

Fusil de Précision	&$-$15	&2d10 + 12	&23	&SA	&40 \\ \hline

Mitrailleuse	&$-$9	&2d10 + 8	&19	&BF, FA	&50 \\ \hline

\end{tabularx} \label{tab:kinetic-railguns} \end{table} 

\subsubsection{Kinetic ammunition} 

Ammunition is defined by its various types (standard, gel, APDS, etc.) and by the class of gun (light pistol, heavy pistol, SMG, etc.). For simplicity, each gun can trade ammunition with another gun of its class, though ammunition for firearms and railguns is not exchangeable. For example, all railgun SMGs can share ammo. 

The ammunition’s Damage Value and Armor Penetration modifiers are added to the weapon’s base DV and AP. With the exception of regular and armorpiercing rounds, none of this ammunition may be used with railguns. Listed costs are per 100 rounds of ammunition. 

\textbf{Armor-Piercing:} This tungsten-carbide ammunition penetrates armor effectively. \textit{[Bas]} 

\textbf{Bug:} Bug rounds are equipped with a microbug and medical sensor nanobots. They attempt to gather information on the target’s location (via standard mesh tracking), health (querying the target’s medichines), and surroundings (typically hindered by being inside the target’s body). They will transmit status reports in a pre-programmed manner via the mesh or a prechosen frequency band either continuously or in preset intervals. \textbf{[Bas]} 

\textbf{Capsule:} Capsule ammo carries a payload (drug, toxin, nanobots) that is released inside the target after the round penetrates. [Trivial plus payload cost] Flux: Flux ammo is made from rheological materials that allow each bullet to be ``programmed'' so that they may change from regular rounds to less-lethal soft plastic-like rounds. This allows the firer to choose the type of round (regular or plastic) made with each shot or burst, and then change with the next one. \textbf{[Bas]} 

\textbf{Hollow-Point:} Hollow-point bullets are designed to deform and widen once they penetrate a target, thus inflicting more damage. \textbf{[Trivial]} 

\textbf{Jammer:} Jammers stick to the target and pulse out jamming electromagnetic signals, jamming the target’s wireless communications. If an Opposed Test is called for, these devices have an Interface of 30. See Radio Jamming, p. 262. \textbf{[Bas]} 

\textbf{Plastic:} Plastic ammo is designed to hurt but not wound targets, and is commonly used for crowd control purposes. \textbf{[Trivial]} 

\textbf{Reactive:} The casing on these projectiles is made of reactive materials that release a large amount of energy when subjected to a sudden shock or impact --- such as striking a target. In other words, they explode or superheat when they hit. \textbf{[Bas]} 

\textbf{Reactive Armor-Piercing (RAP):} This is a tungsten- carbide armor-piercing round with a reactive casing, allowing the ammunition to penetrate even further. \textbf{[Modéré]} 

\textbf{Regular Ammo:} This standard metal projectile is designed to put holes into morphs. \textbf{[Trivial]} 

\textbf{Splash:} Splash rounds carry a payload like capsule ammo, but are designed to break upon impact rather than penetrating, splashing their contents on the target’s exterior. Splash rounds are typically loaded with paint, taggant nanobots, tracker dye, and similar substances. \textbf{[Trivial plus payload cost]} 

\textbf{Zap:} Zap rounds are rubber or gel bullets that create an electric charge upon firing in a piezoelectric like manner to stun the target effectively with both the bullet and the electric shock. \textbf{[Trivial]} 

\subsubsection{Smart ammo} 

Smart ammunition takes advantage of nanotechnology to produce bullets that can alter their flight path, home in the target, and correct aim. Smart ammo may not be used with railguns. With the exception of biter, flayer, and proximity rounds, smart ammo may be combined with other ammo types (accushot armorpiercing, for example). 

\textbf{Accushot:} Accushot bullets change shape within flight to keep dead on course, countering the effects of wind, drag, and gravity over distance. Attacks made with accushot bullets ignore all range modifiers. \textbf{[Bas]} 

\textbf{Biter:} Biters are specially-designed to fragment in opposite proportion to the hardness of the target they strike. For hard targets (synthmorphs), they fragment very little, blasting a big hole. For soft targets (biomorphs), they fragment and tumble in multiple directions within the body. \textbf{[Bas]} 

\textbf{Flayer:} Flayers have nanosensors to detect an oncoming impact, shooting out monomolecular barbs as they are about to strike a target. These monowires cut through the target along with the bullet, inflicting additional damage. \textbf{[Bas]} 

\textbf{Homing:} When fired with a smartlink system, the bullet identifies the target and uses nanosensors to lock on, correcting the bullet’s trajectory with surface alterations and tiny vectored nozzles. Apply a +10 modifier to the Attack Test, cumulative with aiming and smartlink modifiers. Homing bullets may also be used for indirect fire (p. 195). \textbf{[Bas]} 

\textbf{Laser-Guided:} These bullets function like homing smart rounds (apply the +10 attack modifier), except rather than requiring a smartlink system, they lock onto the reflection of the laser sight used to paint the target. Laser-guided bullets may also be used for indirect fire (p. 195). \textbf{[Bas]} 

\textbf{Proximity:} Proximity is an explosive ammunition that identifies the target when fired via smartlink. If the round determines that it will miss the target, it will still explode if it reaches the close proximity of the target. If the attack misses with an MoF of 10 or less, the round explodes 1d10 meters away from the target and inflicts 1d10 area effect damage (see Blast Effect, p. 193) in the proximity of the target. \textbf{[Modéré]} 

\textbf{Zero:} Similar to homing smart rounds, zero bullets identify the target when fired via smartlink. Whether the round hits or misses, however, it sends telemetry data back to the next zero bullet, allowing it to course-correct and ``zero in'' to hit the target (or hit more accurately). Apply a +10 modifier to each shot (or burst) fired after the first against the same target in the same Action Turn. \textbf{[Bas]} 

\begin{table} \begin{tabular}{|l|l|l|} \hline

\multicolumn{3}{|c|}{\textbf{Kinetic Ammunition}} \\ \hline

Armor-Piercing	&$-$5	&$-$2 \\ \hline

Bug	&+1	&$-$1d10 \\ \hline

Capsule	&+1	&$-$half \\ \hline

Flux	&as ammo type	&as ammo type \\ \hline

Hollow-Point	&+2	&+1d10 \\ \hline

Jammer	&--- &no damage \\ \hline

Plastic	&(AV doubled)	&$-$half \\ \hline

Reactive	&$-$2	&+2 \\ \hline

Reactive Armor-Piercing	&$-$6	&$-$1 \\ \hline

Regular	&--- &--- \\ \hline

Splash	&--- &no damage \\ \hline

Zap	&+2	&$-$half + shock (p. 204) \\ \hline

\multicolumn{3}{|l|}{\emph{Smart ammo}} \\ \hline

Accushot	&--- &--- \\ \hline

Biter	&--- &+1d10 \\ \hline

Flayer	&--- &+2 \\ \hline

Homing	&--- &--- \\ \hline

Laser-Guided	&--- &--- \\ \hline

Proximity	&$-$1	&+2 \\ \hline

Zero	&--- &--- \\ \hline

\end{tabular} \label{tab:kinetic-ammo} \end{table} 



\subsection{Brand name weapons and combined arms} \label{sec:brand-weapons-combined} 

The weapons listed in this book define generic samples of each weapon. Gamemasters are encouraged to offer brand name versions of each weapon, each with its particular idiosyncrasies and small variations. For example, a Direct Action A30 SMG might lack a semi-automatic setting but come equipped with an extra ammo capacity of 35. Likewise, a Medusan Arms Longinus sniper rifle may inflict an extra +2 damage but have an AP of only $-$12. 

Similarly, many of the weapons listed here are available as combined arms weapons systems. A police-issue assault rifle may also feature a stunner --- all built into the same weapon. For combined arms, simply add together the individual weapon component costs. 



\subsection{Armes à Rayons} \label{sec:beam-weapons} 

Beam weapons is a broad category for a number of electromagnetic weapons with a wide range of effects. With a few exceptions, energy weapons are primarily used for less-than-lethal purposes, designed to impair the target rather than kill it. Their poor performance against armor, lesser ability to damage targets, and high power requirements make them less versatile than kinetic weapons. The wielders of such weapons use Beam Weapons skill. Beam weapons are powered by nuclear batteries. This battery is good for a list number of shots before it is depleted. Batteries may be recharged at the rate of 20 shots per hour; they have a Cost of \textbf{[Low]}. 

\textbf{Laser Pulsers:} Laser weapons use focused beams of light to inflict damage on the target by burning into it and causing it’s outer surface to vaporize and expand, creating an explosive effect. The laser beam is pulsed in order to bite into the target before the beam is diffused. Pulsers are vulnerable to atmospheric effects like dust, mist, smoke, or rain, however --- the gamemaster should reduce their effective range categories as appropriate. Note that laser pulses are invisible in the normal visual spectrum (but are visible to characters with enhanced vision). Pulsers are medium-sized (p. 297) and fire in semi-auto mode. \textbf{[Modéré]} 

One advantage to the pulser is that it can be placed in less-lethal mode. In this case, it first fires a pulse at the target to create a ball of plasma, quickly fired by a second pulse that strikes the plasma and creates a flash-bang shockwave to stun and disorient the target. This blast has an area of effect with a 1-meter radius. Anyone caught in the blast must make a SOM $\times$ 2 Test (SOM $\times$ 3 for synthmorphs or biomorphs with any form of pain tolerance). Failure means the target is temporarily stunned and disoriented and loses their next action. A critical failure means the target is knocked down and paralyzed for 1 Action Turn per 10 points of MoF. In this stun setting, the pulser fires only in single-shot mode. 

\textbf{Microwave Agonizer:} The agonizer fires millimeterwave beams that create an unpleasant burning sensation in skin (even through armor) and to metals. Agonizers have two settings. The first is an active denial setting that causes extreme burning pain in the target, inflicting $-$20 to the target’s actions and forcing them to move away from the beam on their next action unless they succeed in a WIL Test (targets with Level 1 Pain Tolerance or the equivalent only suffer a $-$10 modifier and roll WIL $\times$ 2). Synthetic morphs and biomorphs with Level 2 Pain Tolerance (or the equivalent) are immune to this weapon. The second setting (colloquially known as the ``roast'' setting) has the same effect of the first, but also actually burns the target, inflicting the listed damage. Originally developed for crowd control, the agonizer is also useful for repelling animals. The agonizer is small-sized (p. 297) and fires in semi-auto mode. \textbf{[Modéré]} 

\textbf{Particle Beam Bolter:} This weapon shoots a bolt of accelerated particles at near light speed that transfer massive amounts of kinetic energy to the target, superheating and creating an explosion when striking. The bolter’s beam is not diffused by the cloud that occurs when it strikes, and so it has greater penetration than the laser pulser. Likewise, the bolter is not affected by smoke, fog, or rain. The bolter’s beam is invisible. Note that bolters are designed for either atmospheric or exoatmospheric (vacuum) operation, and will not function in the opposite environment (though bulkier dual models, combining both models, are also available). Bolters fire in semi-auto mode and are rifle-sized two-handed weapons. \textbf{[Élevé]} 

\textbf{Plasma Rifle:} This bulky, heavy, two-handed weapon blasts a stream of nova-hot plasma at the target, inflicting severe burns and thermal damage, possibly melting or evaporating the target entirely. Plasma rifles are perhaps the deadliest man-portable weapons in use. Plasma guns suffer from dangerous overheating, however, and so require 1 full Action Turn of cool-down time after every 2 shots. Plasma rifles only fire in single shot mode. [Expensive] Stunner: The stunner is an electrolaser that creates an electrically-conductive plasma channel to the target, down which it transmits a powerful electric current, shocking the target. Stunners do not work in vacuum. Stunners fire in semi-auto mode. \textbf{[Modéré]} 

\begin{table} \begin{tabularx}{\textwidth}{|X|X|X|X|l|l|} \hline

\multicolumn{6}{|c|}{\textbf{Beam weapons}} \\ \hline

&\textbf{Armor penetration (AP)}	&\textbf{Valeur de dégats (VD)}	&\textbf{VD moyenne}	&\textbf{Firing modes}	&\textbf{Ammo} \\ \hline

Cybernetic Hand Laser (p. 308)	&--- &2d10	&11	&SA	&50 \\ \hline

Laser à Impulsion	&--- &2d10	&11	&SA	&100 \\ \hline

Stun Mode	&--- &1d10	&5	&SS	&--- \\ \hline

Agoniseur à Micro Ondes	&--- &pain (see description)	&--- &SA	&100 \\ \hline

Roast Mode	&$-$5	&2d10	&11	&SA	&50 \\ \hline

Bolter à Particule	&$-$2	&2d10 + 4	&15	&SA	&50 \\ \hline

Fusil à Plasma	&$-$8	&3d10 + 12	&28	&SS	&10 \\ \hline

Étourdisseur	&--- &(1d10 $\div$ 2) + shock (p. 204)	&--- &SA	&200 \\ \hline

\end{tabularx} \label{tab:beam-weapons} \end{table} 



\subsection{Seekers} \label{sec:seekers} 

Seekers are a combination of automatic grenade launcher, micromissile, coilgun, and smart munitions technology. Unlike traditional launchers of the past, miniaturization allows the manufacture of seeker micromissile launchers in personal weapon sizes. Seeker rounds are fired at high-velocity via rings of magnetic coils, after which the micromissile or minimissile uses scramjet technology to propel itself and maintain high velocities over great distances. Seekers are wielded using Seeker Weapon skill. 

Seeker missiles are detailed on p. 340. Like grenades, seekers may be programmed for a variety of trigger events (see Grenades and Seekers, p. 199). All seeker weapons are smartlink-equipped (p. 342). 

\textbf{Disposable Launcher (Standard Missile):} This launcher is pre-packed with one standard missile. \textbf{[Moderate (includes missile)]} 

\textbf{Seeker Armband (Micromissile):} This weapons unit is worn on the arm, allowing the user to point and fire using an entoptic smartlink system. Though highly portable, the armband’s micromissile supply is low. It fires in single-shot mode. \textbf{[Modéré]} 

\textbf{Seeker Pistol (Micromissile):} This pistol-sized seeker launcher fires micromissiles in semi-auto mode. \textbf{[Modéré]} 

\textbf{Seeker rifle (Micromissile/Minimissile):} The seeker rifle comes in a bullpup configuration and fires either micromissiles or minimissiles in semi-auto mode. It is a two-handed weapons. \textbf{[Élevé]} 

\textbf{Underbarrel Seeker (Micromissile):} This seeker micromissile launcher is commonly attached to the underbarrel of SMGs or assault rifles. It fires in semiauto mode. \textbf{[Modéré]} 

\begin{table} \begin{tabular}{|l|l|l|} \hline

\multicolumn{3}{|c|}{\textbf{Seeker weapons}} \\ \hline

&\textbf{Firing modes}	&\textbf{Ammo} \\ \hline

Disposable Launcher	&SS	&1 \\ \hline

Seeker Armband	&SS	&4 \\ \hline

Seeker Pistol	&SA	&8 \\ \hline

Seeker Rifle	&SA	&12 micromissile/6 minimissile \\ \hline

Underbarrel Seeker	&SA	&6 \\ \hline

\end{tabular} \label{tab:seeker-weapons} \end{table} 



\subsection{Spray weapons} \label{sec:spray-weapons} 

Spray weapons blast their ammunition outwards in a widening cone, allowing them to strike several targets at once. These weapons are wielded with Spray Weapons skill. Spray weapon ammunition has a flat cost of Low per 100 shots (with the exception of buzzers, which use nanoswarms). 

\textbf{Buzzer:} Equipped with a specialized nanobot hive, Buzzers are used to spray a nanoswarm (p. 328) on a target or area. They have a limited capacity of swarms, though the nanohive can construct one new swarm each hour. This weapon is two-handed. \textbf{[Modéré]} 

\textbf{Freezer:} Freezers spew out a fast-hardening foam that immediately begins to harden. They are primarily used as a non-lethal method of immobilizing or securing a target. Struck characters must immediately make a REF $\times$ 3 Test or become trapped. Apply a $-$30 modifier to this test if the attacker scored an Excellent Success (MoS 30+). The foam allows characters to breath even if their mouth and nose are covered, but it may impede sight. Freezer foam can be spiked with contact toxins or drugs to additionally sedate the target. It can also be used to construct temporary barricades or cover. Hardened foam has an Armor of 10 and Durability of 20. It slowly breaks down and degrades over a 12 hour period. Freezers are twohanded. \textbf{[Modéré]} 

\textbf{Shard Pistol:} The shard pistol is a flechette weapon, firing a stream of of diamondoid monomolecular shards at high velocities. These micro flechettes are very good at penetrating armor, but they do not disperse kinetic energy well and so do not cause as much tissue damage as kinetic weapons. Shard ammunition is often coated with drugs or toxins for extra efficiency. \textbf{[Bas]} 

\textbf{Shredder:} A heavier version of the shard pistol, the shredder fires a larger cloud of lethal flechettes, enough to shred a portion of the target into a fine mist. \textbf{[Modéré]} 

\textbf{Sprayer:} This is a general-purpose two-handed squirtgun, loaded with tanks filled with the chemical or drug of the wielder’s choice. \textbf{[Bas]} 

\textbf{Torch:} This modern flamethrower uses condensed ammunition capsules rather than fuel tanks, scorching targets and setting them on fire. Any hit that is an Excellent Success (MoS 30+) sets the target on fire, where they will continue to take 2d10 damage per Action Turn. These chemical fires are particularly difficult to put out unless they are deprived of oxygen. Torches are two-handed. \textbf{[Modéré]} 

\begin{table} \begin{tabularx}{\textwidth}{|l|X|l|l|l|l|} \hline

\multicolumn{6}{|c|}{\textbf{Spray weapons}} \\ \hline

&\textbf{Armor penetration (AP)}	&\textbf{Valeur de dégats (VD)}	&\textbf{VD moyenne}	&\textbf{Firing modes}	&\textbf{Ammo} \\ \hline

Buzzer	&--- &nanoswarm	&--- &SS	&3 \\ \hline

Geleur	&--- &incapacitation	&--- &SA	&20 \\ \hline

Shard	&$-$10	&1d10 + 6	&11	&SA, BF, FA	&100 \\ \hline

Déchiquetteur	&$-$10	&2d10 + 5	&16	&SA, BF, FA	&100 \\ \hline

Spray	&as chemical/drug	&as chemical/drug	&as chemical/drug	&SA	&20 \\ \hline

Torche	&$-$4	&3d10	&16	&SS	&20 \\ \hline

\end{tabularx} \label{tab:spray-weapons} \end{table} 

\subsection{Grenades et chercheurs} \label{sec:grenades-seekers} 

Grenades and seeker missiles come in similar munitions packages and with similar trigger mechanisms, though their packaging, physical form, and methods of application differ. Seeker missiles are fired from a seeker launcher (p. 339) using Seeker Weapons skill. Grenades are thrown at targets using Throwing Weapons skill. If a grenade or seeker misses, use the rules for scatter (p. 204). 

Grenades are available in standard form or as microgrenades. Similarly, missiles are available in standard, minimissile, or micromissile sizes. Standard grenades and minimissiles are the baseline standard for listed effects. All are area effect weapons (p. 193). Minigrenades and micromissiles inflict $-$1d10 damage (or have another decreased effect as noted). Standard missiles double the listed DV. For weapons with a uniform blast effect or other static blast area, divide the base listed radius in half for minigrenades and micromissiles and double it for standard missiles. Listed costs are for 10 grenades/missiles. 

Each seeker has one smart ammo option (p. 338) other than biter or flayer. 

\textbf{Concussion:} These devices emit a concussive blast designed to knock opponents off their feet and stun them. Any character caught within a base blast radius of 10 meters must make a SOM $\times$ 2 Test. If they fail, they are knocked down. If their MoF is 30+, they are additionally stunned until the end of the next Action Turn. Anyone caught in the blast radius suffers a $-$10 action modifier for the rest of that Action Turn. \textbf{[Modéré]} 

\textbf{EMP:} EMP munitions fire off a strong electromagnetic pulse when they ``detonate.'' Since most electronics in \emph{Eclipse Phase} are built with optical technology, and power supplies and sensitive microcircuits are shielded and surge-protected, this has no major damaging effect. Antennas, however, are vulnerable, especially finer wires like those used with mesh inserts. As a result, the primary effect of EMP is to disable radio communications --- every radio within range of the blast is reduced to 1/10th the normal range. The base blast radius for EMP is 50 meters. \textbf{[Élevé]} 

\textbf{Frag:} Fragmentation explosives spread a cloud of lethal flechettes over the area of effect. They are resisted with kinetic armor. \textbf{[Modéré]} 

\textbf{Gas/Smoke:} Gas/smoke munitions emit a cloud of their contained substance. Smoke impedes sight by releasing thick fumes upon ignition of the seeker. The smoke can be of any color and is often heated (called thermal smoke) to obfuscate heat signatures moving through the smoke as cover. Note that gases dissipate much more quickly under certain environmental conditions (wind, rain, etc.) \textbf{[Low]} 

\textbf{High-Explosive:} High-explosive seekers and grenades are designed to create a very destructive shock and heat wave. This damage is resisted with energy armor. \textbf{[Modéré]} 

\textbf{High-Explosive Armor-Piercing (HEAP):} A design only available for seekers (not grenades), HEAP warheads use high explosives to blast a path for a penetrating round. HEAPs lose $-$4 damage per meter distance from the blast, as opposed to the usual $-$2. \textbf{[Modéré]} 

\textbf{Overload:} Overload grenades and seekers launch an all-out assault on the target’s sensory spectrum. This attack includes blinding by intense flashing light, a deafening thunderclap followed by intense ultrasonic screaming, nausea-inducing malodorants, and infrasonic frequencies that can trigger unpleasant emotional responses (anxiety, uneasiness, extreme sorrow, nervous feelings of revulsion or fear). For an extra kick, overloads are also packed with ``stingballs'' --- rubber pellets that inflict pain when detonated near an underarmored target. Anyone caught in the base 10-meter blast radius must make a SOM + WIL Test. If they fail, they must immediately leave the area of effect. If they fail with an MoF of 30+, they are incapacitated for 3 Action Turns with disorientation and/or vomiting, after which they must roll again. Overload munitions remain in effect for 1 full minute. Anyone in the area of effect suffers a $-$30 action modifier, which reduces by 10 per Action Turn after they leave the area. Additionally, anyone facing the direction of the overload round suffers a $-$10 glare modifier (neutralized by anti-glare systems). \textbf{[Modéré]} 

\textbf{Plasmaburst:} Also called ``hellballs,'' these munitions release a burst of plasma upon detonation that causes searing heat and fire damage across the area of effect without the devastating shockwaves of explosions that might rebound in an enclosed environment and/or breach a habitat’s infrastructure. \textbf{[Élevé]} 

\textbf{Splash:} Splash rounds spread a contained substance (drug, chemical, nanoswarm, paint) over a base 10- meter blast radius when they detonate. \textbf{[Low plus payload cost]} 

\textbf{Thermobaric:} Thermobaric grenades and seekers utilize a more deadly form of explosion. When they detonate, they disperse a cloud of aerosol explosive over an area and then ignite, literally setting the air on fire, generating a devastating pressure wave, and sucking the oxygen out of the area. Thermobarics use the rules for uniform blast (p. 194). \textbf{[Élevé]} 

\subsubsection{Sticky grenades} 

Sticky grenades have a special coating that when triggered becomes a sticky adhesive, allowing the grenade to be stuck to almost any surface. Sticky grenades can even be wielded in melee combat, smacking them on an opponent to be detonated later. \textbf{[Trivial]} 

\begin{table} \begin{tabular}{|l|l|l|l|l|} \hline

\multicolumn{5}{|c|}{\textbf{Grenades and seekers}} \\ \hline

\textbf{Type}	&\textbf{PA}	&\textbf{DV}	&\textbf{VD moyenne}	&\textbf{Armor used to resist} \\ \hline

Concussion	&--- &1d10 $\div$ 2	&5	&E \\ \hline

Frag	&$-$4	&3d10 + 6	&22	&K \\ \hline

EMP	&--- &--- &--- &--- \\ \hline

Gas/Smoke	&--- &--- &--- &--- \\ \hline

High-Explosive	&--- &3d10 + 10	&26	&E \\ \hline

HEAP	&$-$8	&3d10 + 12	&28	&K \\ \hline

Overload	&(AV $\times$ 2)	&1d10 $\div$ 2	&5	&K \\ \hline

Plasmaburst	&$-$6	&3d10 + 10	&26	&E \\ \hline

Splash	&--- &--- &--- &--- \\ \hline

Thermobaric	&$-$10	&3d10 + 5	&21	&E \\ \hline

\end{tabular} \label{tab:grenades-seekers} \end{table} 

\subsection{Exotic ranged weapons} \label{sec:exotic-ranged-weapons} 

These weapons are either rare or distinctly separate from other weapons types. These weapons are wielded with an Exotic Ranged Weapon skill of the appropriate field. 

\textbf{Vortex Ring Gun:} This less-lethal two-handed weapon detonates a blank cartridge and accelerates the explosive pressure down a widening barrel so that it develops into a high-speed vortex ring --- a spinning, donut-shaped blast vortex. This concussive blast is used to knock down and incapacitate close-range targets. Struck targets suffer a $-$10 action modifier for the rest of that Action Turn and must must succeed in a SOM $\times$ 2 Test or fall down. If their MoF is 30+, they are additionally stunned and unable to act until the end of the next Action Turn. Drugs, chemicals, and similar agents may be loaded into the charge as well. \textbf{[Modéré]} 

\subsection{Weapon accessories} \label{sec:weapon-accessories} 

The following accessories are available for various weapons. 

\textbf{Arm Slide:} This slide-mount can hold a pistolsized weapon under a character’s sleeve, pushing the weapon into the character’s hand with an electronic signal or specific sequence of arm movements. \textbf{[Bas]} 

\textbf{Extended Magazine:} This ammunition case has an increased capacity. Increase the weapon’s ammo capacity by +50\%. Only available for firearms and seekers. \textbf{[Bas]} 

\textbf{Gyromount:} This weapon harness features a gyrostabilized weapon mount that keeps the weapon steady. Negates all modifiers from movement. \textbf{[Modéré]} 

\textbf{Imaging Scope:} Imaging scopes attach to the top of the weapon and act like specs (p. 325). Scopes may also bend like a periscope, along a character to point the weapon and target around corners without leaving cover. \textbf{[Bas]} 

\textbf{Flash Suppressor:} This device obscures the muzzle flash on firearms, applying a $-$10 modifier on Perception Tests to locate a firing weapon by its flash. \textbf{[Bas]} 

\textbf{Laser Sight:} This underbarrel laser emits a beam that places a glowing red dot on the target to assist targeting. Apply a +10 modifier to Attack Tests (not cumulative with a smartlink modifier). Laser sights may also be used to paint a target for laser-guided smart ammo or seekers. Infrared and ultraviolet lasers are also available, so that the dot is only visible to characters able to see in those spectrums. \textbf{[Bas]} 

\textbf{Safety System:} A biometric (palmprint or voiceprint) or ego ID (p. 279) sensor is embedded in the weapon, disabling it if anyone other than an authorized user attempts to fire it. \textbf{[Bas]} 

\textbf{Shock Safety:} Just like a safety system, except that an unauthorized user is zapped with an electric shock. Treat as a shock baton (p. 334). \textbf{[Modéré]} 

\textbf{Silencer/Sound Suppressor:} This barrel-mounted accessory reduces the sound of a firearm’s discharge. Apply a $-$10 modifier on hearing-based Perception Tests to hear or locate the gun’s firing. \textbf{[Modéré]} 

\textbf{Smartlink:} A smartlink system connects the weapon to the user’s mesh inserts, placing a targeting bracket in the character’s field of vision and providing range and targeting information. Apply a +10 modifier to the Attack Test. Smartlinks also incorporate a microcamera that allows the user to see what the weapon is pointed at, fire around corners, etc. Smartlinks also allow certain other types of weapon system control, such as changing flux ammo (p. 337) or programming seeker trigger conditions (p. 199). \textbf{[Modéré]} 

\textbf{Smart Magazine:} A smart magazine allows the character to pick and choose what ammo round will be fired with each shot. This system leaves less room for bullets, however, so reduce the weapon’s ammunition capacity by half (round up). Smart magazines may be combined with extended magazines, in which case ammo capacity is normal. \textbf{[Modéré]} 

\section{Robots and vehicles} \label{sec:robots-vehicles} 

The following is a small selection of the many vehicles in use in the solar system. Almost all of the vehicles in current use, including all of the vehicles listed here, have built-in AIs capable of piloting the vehicle under almost all circumstances. In most cases, passengers simply state their destination and the vehicle takes them there. Manual piloting is primarily used in emergencies or by people who prefer the exotic thrill of controlling their own vehicle. 

Rules for handling robots and vehicles are detailed on p. 195. Any of these shells may be modified for use as a synthetic morph by adding a cyberbrain system (p. 300). Each of the shells listed here comes with a puppet sock (p. 307) for remote-control operation. 

\subsection{Aircraft} \label{sec:aircraft} 

On Mars, Venus, and within large open-space habitats like O’Neil cylinders, aircraft of various kinds see regular use. This includes modern version of rotorcraft (helicopters, autogyros, tilt-rotors), fixed-wing planes, and zeppelins and other lighter-than-air craft. These are typically propelled by turbofan or jet engines, rotors, or vectored thrust. These vehicles are piloted with Pilot: Aircraft skill. Microlight: This ultra-light personal aircraft is not much more than a strut-based wing, an airframe, and an electric propeller engine. They are ideal for getting around inside large habitats with enclosed airspace. \textbf{[Bas]} 

\textbf{Portable Plane:} Powered by superconducting batteries and with an exceedingly small but powerful electric motor, this light but durable propeller plane is made of smart materials that allow it to be swiftly folded up into a small portable package. Different versions are designed for flight on Mars, Titan, or Venus, each taking 10 minutes to assemble or disassemble. The Martian version unpacks into an airplane with a wingspan of 11 m with a top speed of 250 kph and a cruising speed of 220 kph and a range of 1,300 km. The Venusian version has a wingspan of 9 m, a top speed of 200 kph and a range of 1,000 km. The version designed for use on Titan has a wingspan of 8 m and has a top speed of 200 kph and a range of 2,000 km. In all versions, the two occupants ride in an inflatable and insulated pressurized bubble with a life support system capable of providing clean air and comfortable temperatures for 20 hours on Mars and Venus, and 15 hours on Titan. \textbf{[Élevé]} 

\textbf{Rocket Buggy:} This vehicle is the most common form of medium to long distance personal transport on Luna, and is in common on most other moons and large asteroids. On these airless worlds, a rocket buggy can reach orbit and return or take a parabolic path to any destination on that moon in less than an hour. This vehicle is also regularly used to travel between habitats that are less than 30,000 km apart. The vehicle is pressurized, but is designed for short duration travel only. The seats are relatively small and the life support system contains no provisions for recycling food or water and can only support the passengers for an absolute maximum of 50 uncomfortable hours. Rockets buggies come equipped with headlights, radio boosters, and radar with a range of up to 250 km. 

A version of this vehicle is also used on both Mars and Titan, but here the frame has been modified to act as a lifting body, and it has a top speed in the thin Martian atmosphere of 2,500 km/hour and a range of 8,000 km on Mars. On Titan is has a top speed of 3,000 kph in the atmosphere, but it can also reach orbit. \textbf{[Cher]} 

\textbf{Small Jet:} Methane-powered jet planes are one of the most common forms of fast transport on Mars and Venus. Similar planes are used on Titan, except that they carry both liquid methane and liquid oxygen. These jets range in size from huge vehicles the size of late 20th-century airliners to small planes designed to carry half a dozen passengers. All jets are made using smart materials, so that their wings and frames can adapt to a wide range of speeds and altitudes. One common small jet has similar versions in use on Venus, Mars, and Titan, has a single jet engine and has a life support system capable of providing air for up to 100 hours. The Venusian and Martian versions both have a top speed of 900 kph, a wingspan of 11 m, and a maximum range of 5,000 km. The version designed for Titan has a wingspan of 8 m, a top speed of 650 kph, and a range of 4,000 km. Jets are equipped with headlights, radio boosters, and radar with a range of up to 250 km. \textbf{[Cher]} 

\begin{table} \begin{tabularx}{\textwidth}{|X|X|X|X|X|l|l|X|} \hline

\multicolumn{8}{|c|}{\textbf{Vehicles --- Aircraft}} \\ \hline

&\textbf{Passenger capacity}	&\textbf{Handling}	&\textbf{Movement rate}	&\textbf{Max velocity}	&\textbf{Armure}	&\textbf{Solidité}	&\textbf{Wound threshold} \\ \hline

Microlight	&1	&+20	&8/40	&100	&--- &30	&10 \\ \hline

Portable Plane	&2	&+10	&--- &200$-$250	&10/6	&50	&10 \\ \hline

Rocket Buggy	&4	&$-$10	&8/32	&2,500$-$3,000	&24/16	&100	&20 \\ \hline

Small Jet	&6	&+20	&--- &650-900	&30/20	&200	&30 \\ \hline

\end{tabularx} \label{tab:aircraft} \end{table} 



\subsection{Exoskeletons} \label{sec:exoskeletons} 

Exoskeletons are powered mechatronic skeleton frameworks worn by a person. Servo-hydraulic joints allow the exoskeleton to be maneuvered by mimicking the wearer’s own movements, as well as enhancing their strength. Exoskeletons may also be piloted electronically. A character wearing an exoskeleton (other than the trike or transporter) maneuvers as normal, because the exoskeleton is like an extension of their own body. A character jamming an exoskeleton remotely uses Pilot: Walker skill (except for the trike and transporter). 

\textbf{Battle Suit:} The battle suit powered exoskeleton features a military-grade fullerene armor shell with flexible aerogel for thermal insulation and a diamond-hardened exterior designed to resist even potent ballistic and energy-based weapons. The suit also enhances the wearer’s strength and mobility, applying a +10 bonus to strength-based tests, inflicting an extra +1d10 damage and AP of $-$2 on melee attacks, and doubling the distance by which the character may jump. Battlesuits are completely sealed to protect the wearer from environmental factors, and fitted with life support features and a maker (p. 327) capable of recycling all wastes and producing air for up to 48 hours and food and water indefinitely. Battle suits are equipped with each an ecto (p. 325), un booster  radio (p. 313), et des capteurs équivalents à des poussières (see p. 325). These suits have an Armor Value of 18/18 (not cumulative with any other armor) and protect the wearer from temperatures from $-$175 to 140$^{\circ}$ C. \textbf{[Expensive]} 

\textbf{Exowalker:} Exowalkers are minimal framework exoskeletons, primarily designed to bolster the wearer’s strength and movement. They provide a an Armor Value of 2/4, a +10 modifier to strength-based tests, and double the distance by which the character may jump. \textbf{[Modéré]} 

\textbf{Hyperdense Exoskeleton:} These powered exoskeletons are larger (roughly twice human-sized) and built for heavy-use industrial purposes, such as handling heavy/large objects. The wearer is partially encapsulated to protect them from debris and industrial accidents. Hyperdense exoskeletons provide no movement bonus, but provide a +30 bonus to strength-based tests and inflict an extra +3d10 damage and $-$5 AP on physical attacks. They have an Armor Value of 6/12. \textbf{[Cher]} 

\textbf{Transporter:} This exoskeleton framework includes a pair of vector-thrust turbofan engines, giving the user flight capabilities in gravity and increased maneuverability in zero-G. It provides partial protection to the wearer with an Armor Value of 2/4. Piloted with Pilot: Aircraft skill. \textbf{[Élevé]} 

\textbf{Trike:} The trike exoskeleton is a three-wheeled personal motorcycle design, rather than a walker. It provides partial protection to the wearer with an Armor Value of 2/4. Piloted with Pilot: Groundrcraft skill. \textbf{[Modéré]} 



\begin{table} \begin{tabularx}{\textwidth}{|X|X|X|X|X|X|X|X|} \hline

\multicolumn{8}{|c|}{\textbf{Vehicles --- Exoskeletons}} \\ \hline

&\textbf{Passenger\newline capacity}	&\textbf{Handling}	&\textbf{Movement rate}	&\textbf{Max velocity}	&\textbf{Armure}	&\textbf{Solidité}	&\textbf{Wound threshold} \\ \hline

Battlesuit	&1	&--- &8/32	&30	&18/18	&60	&12 \\ \hline

Exoarmure	&1	&--- &8/40	&40	&2/4	&30	&6 \\ \hline

Exosquelette Hyperdense	&1	&--- &8/20	&30	&6/12	&100	&20 \\ \hline

Transporter	&1	&+10	&8/40	&200	&2/4	&50	&10 \\ \hline

Trike	&1	&+10	&8/40	&120	&2/4	&50	&10 \\ \hline

\end{tabularx} \label{tab:exoskeletons} \end{table} 









\subsection{Groundcraft} \label{sec:groundcraft} 

In Eclipse Phase, trains and bicycles remain the most common form of ground transportation, especially on habitats. In larger habitats and on moons and planets, cycles and cars are used as well. 

\textbf{Cycle:} Because of the high cost of enclosing a habitat and providing life support, space is at a premium in all cities except some of the newest cities on Mars. As a result, there is rarely room for large roads or the cars that once carpeted the roads of Earthly cities. Instead, the ubiquitous modern vehicle is the cycle, which is designed to drive down narrow streets only a little wider than sidewalks in Earth cities. 

There are many different varieties of cycle. Some have only a single wheel and are gyro-stabilized, but most have two wheels and resemble old Earth motorcycles. In some, the driver and passenger are enclosed by a streamlined pod. These vehicles are powered by superconducting batteries, have a range of 600 km and a top speed of 120 kph, but must usually drive more slowly in crowded streets. Cycles are all equipped with radio boosters, headlights, and a portable radar sensor. Tires are solid state (not inflated), or in some cases smart spokes capable of handling stairs. Some luxury versions have limited life-support in the small cabin, capable of providing air for the passengers for up to 10 hours. \textbf{[Moderate]} 

\textbf{Mars Buggy:} One of the most ubiquitous vehicles on Mars is the so-called Mars buggy, a four-wheeled vehicle with large balloon tires that is designed for use both on roads and on almost any terrain. Mars buggies can travel at speed of up to 110 kph on roads, 90 kph over relatively flat terrain, and up to 40 kph on jagged and rocky terrain. They can maintain these speeds because smart materials in both the suspension and the tires reshape themselves to adapt to uneven conditions and their nuclear batteries give them an effectively unlimited range. Most Mars buggies are enclosed but unpressurized. Similar vehicles are used on Luna and Titan, however, though the passenger compartments of these vehicles includes life support gear that provides the occupants with air for at least 100 hours. Buggies are powered by nuclear batteries and come in a variety of sizes, from small two-person buggies to large trucks. Mars buggies come equipped with headlights, radio boosters, and a vehicle radar system. \textbf{[Élevé]} 

\subsection{Personal vehicles} \label{sec:personal-vehicles} 

These one-person movement aids primarily are used in space, but do not count as spacecraft per se. 

\textbf{EVA Sled:} This small sled uses air impellers to maneuver in zero-G. It is commonly used to carry attached gear, but may also pull along 1 human-sized morph. \textbf{[Bas]} 

\textbf{Rocket Pack:} This is a miniature metallic hydrogen rocket that the wearer straps to their back, with two rocket exhausts extending out to either side, away from the wearer’s body or legs. Biomorphs and pod morphs can only safely use this vehicle when wearing a vacuum suit or some garment that is similarly heat resistant. Also, to prevent harm to the wearer, the thrust must be kept sufficiently low that it can only take off on Mars or moons with even lower gravity. A rocket pack can keep the wearer airborne for up to 15 minutes in Mars gravity, or 30 minutes on Luna, Titan, or any of the four large Jovian moons. On Mars, it has a maximum speed of 700 kph. It can be used to reach orbit and land again on Luna, Titan, and other similarly small bodies like the Jovian moons. Rocket packs are equipped with radio boosters but no other sensors or communication devices. \textbf{[Bas]} 

\textbf{Thruster Pack:} Worn for EVA duties, this thruster pack uses vectored thrust nozzles, allowing a character to maneuver in open space. This is not a jetpack and does not produce enough thrust for atmospheric movement. \textbf{[Bas]} 

\begin{table} \begin{tabularx}{\textwidth}{|l|X|X|X|X|X|X|X|} \hline

\multicolumn{8}{|c|}{\textbf{Vehicles --- Groundcraft, personal vehicles}} \\ \hline

&\textbf{Passenger capacity}	&\textbf{Handling}	&\textbf{Movement rate}	&\textbf{Max velocity}	&\textbf{Armure}	&\textbf{Solidité}	&\textbf{Wound threshold} \\ \hline

\multicolumn{8}{|l|}{\emph{Groundcraft}} \\ \hline

Cycle	&1$-$3	&+20	&4/40	&120	&12/10	&50	&10 \\ \hline

Mars Buggy	&2$-$6	&+10	&8/32	&40/90/110	&30/20	&150	&30 \\ \hline

\multicolumn{8}{|l|}{\emph{Personal vehicles}} \\ \hline

EVA Sled	&1	&$-$30	&4/16	&16	&5 40	&&8 \\ \hline

Rocket Pack	&1	&$-$20	&--- &700	&+5/+5	&40	&8 \\ \hline

Thruster Pack	&1	&$-$10	&4/20	&40	&+4/+4	&30	&6 \\ \hline

\end{tabularx} \label{tab:groundcraft-personal} \end{table} 



\subsection{Robots} \label{sec:robots} 

Robots are a common sight and accepted fact of daily life within Eclipse Phase. Numerous varieties exist, from robo-pets to mechanical workers to warbots. If a job can be done more cheaply (and sometimes safely) by a bot, it usually is. The robots listed here are not generally used as synthetic shells by transhuman egos, often for cultural reasons (sleeving a case is bad enough, sleeving a creepy is just ... wrong), and they are not equipped to be sleeved into (though the may be jammed; see p. 196). Any of these bots may be modified for use as a synthetic morph, however, by adding a cyberbrain system (p. 300). 

\textbf{Automech:} Automechs are general purpose repair drones, found just about everywhere. Each particular automech tends to specialize in a particular type of repair work and so carries the appropriate tools and AI skills, whether it be habitat waste recyclers, outer hull integrity, or servitor systems. Standard automechs are wheeled cubes with articulated limbs, though they are also equipped with vectored-thrust drives for zero-G work. \textbf{[Modéré]} 

\textbf{Creepy:} Creepies are small crawler bots that come in an eclectic variety of shapes and forms, from robosquirrels to insectoids to bizarre and artsy mechanical creatures. Creepies were originally designed as a sort of robotic pet, but they are commonly used as a general purpose household minion, like a more beloved servitor. Many people in fact wear a creepy on their person, dropping it to handle small tasks for them and letting it crawl up and down and over their body. \textbf{[Bas]} 

\textbf{Dr. Bot:} These wheeled medical robots are designed to tend to and transport injured or sick people. They carry a healing vat (p. 326), a specialized pharmaceuticals maker, miscellaneous medical gear, and articulated arms for conducting remote surgery. \textbf{[Élevé]} 

\textbf{Dwarf:} These large industrial bots are named not just for their primary use --- mining, excavation, tunneling, and construction --- but because the default AIs they shipped with had a programmed tendency to happily whistle as they worked. Dwarfs are quadrapedal walkers, equipped with massive modular industrial tools like boring drills, shovels, hydraulic jacks, jackhammers, scooping arms, acid sprays, and so on. \textbf{[Cher]} 

\textbf{Gnat:} Gnats are small rotorcraft camera/surveillance drones. Many people use gnats for personal lifelogging, while socialites and media use them to capture the glamor or hottest news. \textbf{[Bas]} 

\textbf{Guardian Angel:} Similar to gnats, guardian angel rotorcraft hover around their charges, keeping a watchful eye out to protect them from threats. \textbf{[Modéré]} 

\textbf{Saucer:} These disc-shaped drones are lightweight and quiet. They are typically launched by throwing them like a frisbee, after which they propel themselves with an ionic drive (p. 310). Saucers make excellent ``eye in the sky'' monitors and scouts. \textbf{[Bas]} 

\textbf{Servitor:} Servitors are the most common robot, acting as cooks, janitors, universal helpers, movers, and personal aides. Every home has one, if not several. Servitors are intentionally built in non-humanoid forms so as not to confuse them with common synthmorphs and in order to defuse bad feelings at ordering them around. However, they all have some form of ``face'' to interact with, so as not to be too machine-like. \textbf{[Bas]} 

\textbf{Speck:} Specks are tiny insectoid spy drones, 2.5 mm long and 2 mm wide, about the size of a small fruit fly. They fly with tiny wings, carry a microbug, and are excellent for surveillance purposes or otherwise being a ``speck on a wall.'' Specks are difficult to notice ($-$30 Perception modifier) and almost impossible to distinguish from an actual insect. \textbf{[Bas]} 

\begin{table} \begin{tabular}{|l|l|l|l|l|l|l|} \hline

\multicolumn{7}{|c|}{\textbf{Vehicles --- Robots}} \\ \hline

&\textbf{Muovement}	&\textbf{Max}	&\textbf{Armure}	&\textbf{Solidité}	&\textbf{Wound}	&\textbf{Mobility} \\ &\textbf{Rate}	&\textbf{Velocity}	&&&\textbf{threshold}	&\textbf{system} \\ \hline

\textbf{Automech}	&4/8	&8	&4/4	&30	&6	&Wheeled/Vector-Thrust \\ \hline

\multicolumn{7}{|l|}{Enhancements: Access Jacks, Electrical Sense, Extra Limbs (4), Headlights, Magnetic System, } \\ \multicolumn{7}{|l|}{Radiation sense, Utilitool, misc. tools} \\ \hline

\textbf{Creepy}	&4/12	&12	&2/2	&25	&5	&Walker or Hopper \\ \hline

\multicolumn{7}{|l|}{Enhancements: +5 COO, Access Jacks, Chameleon Skin, Extra Limbs (2--8), Grip Pads} \\ \hline

\textbf{Dr. Bot}	&4/16	&16	&--- &40	&8	&Wheeled \\ \hline

\multicolumn{7}{|l|}{Enhancements: Access Jacks, Enhanced Smell, Fabber, Fractal Digits, Healing Vat, Nanoscopic Vision} \\ \hline

\textbf{Dwarf}	&4/12	&20	&16/12	&150	&30	&Walker \\ \hline

\multicolumn{7}{|l|}{Enhancements: +10 SOM, Access Jacks, Extra Limbs (4), Industrial Armor, Radar, Sonar, misc. tools} \\ \hline

\textbf{Gnat}	&8/40	&60	&2/2	&25	&5	&Rotor \\ \hline

\multicolumn{7}{|l|}{Enhancements: 360-Degree Vision, Access Jacks, Enhanced Hearing, Enhanced Vision, Radar} \\ \hline

\textbf{Guardian Angel}	&8/40	&80	&14/12	&40	&8	&Rotor \\ \hline

\multicolumn{7}{|l|}{Enhancements: +5 REF, 360-Degree Vision, Access Jacks, Chameleon Skin, Eelware, Enhanced Hearing, } \\ \multicolumn{7}{|l|}{Enhanced Smell, Enhanced Vision, Lidar, Light Combat Armor, Neurachem, T-Ray Emitter} \\ \hline

\textbf{Saucer}	&8/40	&200	&2/2	&25	&5	&Ionic \\ \hline

\multicolumn{7}{|l|}{Enhancements: 360-Degree Vision, Access Jacks, Chameleon Skin, } \\ \multicolumn{7}{|l|}{Enhanced Hearing, Enhanced Vision, Radar} \\ \hline

\textbf{Servitor}	&4/20	&20	&4/4	&30	&6	&Walker or Wheeled \\ \hline

\multicolumn{7}{|l|}{Enhancements: Access Jacks, Extra Limbs (2-6)} \\ \hline

\textbf{Speck}	&1/5	&5	&--- &5	&1	&Winged/Hopper \\ \hline

\multicolumn{7}{|l|}{Enhancements: +5 REF, +5 COO, $-$10 SOM, Access Jacks, Grip Pads, } \\ \multicolumn{7}{|l|}{Enhanced Hearing, Enhanced Vision, Synthetic Mask} \\ \hline

\end{tabular} \label{tab:robots} \end{table} 

\subsection{Spacecraft} \label{sec:spacecraft} 

Though egocasting is a common method of personal transport and it’s often easier to simply transmit the specifications for various goods and to allow nanofactories to create duplicates, spacecraft play an important role in the solar system, carrying both passengers and valuable cargo. Both in terms of materials and propulsion, spacecraft in the post-Fall era are far superior to the primitive vessels used in the 20th and early 21st centuries, but they are still based on the same principles. 

Spacecraft have few stats in \emph{Eclipse Phase}, as they are primarily handled as setting rather than vehicles. Note also that no stats are given for spacecraft weaponry. It is highly recommended that space combat be handled as a plot device rather than a combat scene, given the extreme lethality and danger involved. If you absolutely must know the DV of a spacecraft weapon, treat it as a a standard weapon with a DV multiplier of x3 for small craft (fighters and shuttles), x5 for medium craft, and x10 for larger craft. 

\subsubsection{Spacecraft propulsion} 

The most important part of any spacecraft is its engine, and the most important features of any engine are the exhaust velocity, which determines how much fuel the rocket requires to reach a given speed, and the engine’s thrust, which determines how high the acceleration can be. Any rocket that has a thrust of less than approximately twice the gravity of a planet or moon cannot take off from that planet or moon. Sample thrusts and gravities are listed on the \emph{Escaping Gravity Wells} table, p. 346. 

\textbf{Hydrogen-Oxygen Rocket (HO):} Though optimized with improved engine design and light-weight materials, these are essentially the same primitive rockets that humanity used to first reach the moon in the 20th century. These are rarely used and only common with groups too poor or primitive to safely manufacture metallic hydrogen. 

\textbf{Metallic Hydrogen Rocket (MH):} Metallic hydrogen is a solid form of hydrogen created using exceedingly high pressures. Although naturally unstable, it can be stabilized with carefully controlled electrical and magnetic fields, and these field generators are an integral part of every metallic hydrogen fuel tank. By selectively reducing these fields near the exhaust nozzle, small amounts of metallic hydrogen can be made to swiftly and explosively revert to conventional hydrogen gas, propelling the rocket with great force in an easily controlled fashion. Metallic hydrogen engines are used in most planetary landers and short range vehicles. 

\textbf{Plasma Rocket (P):} This drive heats hydrogen into plasma and accelerates it using a powerful electrical field. This type of rocket was very common in the mid 21st century, but has been superseded by fusion rockets and is only used in older and more primitive spacecraft, notably scum barges. 

\textbf{Fusion Rocket (F):} Similar to a plasma rockets, fusion rockets require significantly higher temperatures and pressures, and the rocket also produces large amounts of power for the spacecraft. Fusion rockets are now the most common form of propulsion for spacecraft designed for long-distance voyages. 

\textbf{Anti-Matter Rocket (AM):} Anti-matter rockets work mixing small amounts of anti-matter into the hydrogen fuel, producing enormous amounts of energy and an exceptionally fast and powerful exhaust. These rockets typically carry a heavily shielded magnetically contained anti-matter storage vessel carrying a mass of anti-matter equal to 1\% of the mass of the hydrogen fuel used by the rocket. The magnetic containment vessels needed to safely contain antimatter usually weight at least 10 times the mass of the antimatter used. 

Though anti-matter storage is exceptionally safe, the vast energy release possible if there was an accident means that anti-matter rockets are forbidden from coming closer than 25,000 km from any inhabited planet or moon. Also, very few habitats will allow an anti-matter rocket to dock with them, and instead require the spacecraft to remain at least 10,000 km away and for all cargo and passengers to be transferred using a small craft like a small LOTV. Anti-matter is exceedingly expensive to produce and so anti-matter rockets are only used in military vessels and in fast couriers designed to carry critical cargoes across the solar system in short periods of time. 

\begin{table} \begin{tabular}{|l|r} \hline

\multicolumn{2}{|c|}{\textbf{Escaping gravity wells}} \\ \hline

\textbf{Spacecraft engine}	&\textbf{Thrust (in GS)} \\ \hline

Hydrogen-Oxygen Rocket	&4+ \\ \hline

Metallic Hydrogen	&3 \\ \hline

Plasma Rocket	&0.01 \\ \hline

Fusion Rocket	&0.05 \\ \hline

Anti-Matter	&0.2 \\ \hline

Rocket Buggy	&0.5 \\ \hline

\textbf{Planets, moons etc.}	&\textbf{Gravity} \\ \hline

Earth	&1 \\ \hline

Europa	&0.13 \\ \hline

Jupiter	&2.53 \\ \hline

Luna	&0.17 \\ \hline

Mars	&0.38 \\ \hline

Mercury	&0.38 \\ \hline

Neptune	&1.14 \\ \hline

Pluto	&0.06 \\ \hline

Saturn	&0.91 \\ \hline

Titan	&0.14 \\ \hline

Uranus	&0.89 \\ \hline

Venus	&0.9 \\ \hline

\end{tabular} \label{tab:excaping-gravity} \end{table} 

\subsubsection{Sample spacecraft} 

The following is a representative sample of the most common type of spacecraft used in the solar system today. 

\textbf{Bulk Carrier:} This vessel is simply a standard transport refitted to carry large amounts of cargo in external cargo grapples. Used for carrying refined ores, ice, and similar forms of large, useful, but low priority cargos, bulk carriers transport large cargos at relatively low velocities. They also offer an inexpensive, reliable, and slow method for passengers to travel from one habitat to another and are not infrequently used by individuals who wish to disappear for a while. Unlike the standard transport, the bulk carrier lacks the rotating habitat pods. 

\textbf{Courier:} In a standard transport, a typical journey from Luna to Mars requires approximately three weeks, while a journey from Mars to Jupiter requires approximately four months. This is sufficient for most purposes, but occasionally characters need to take themselves or sufficiently valuable cargoes across the solar system in a matter of days or weeks, instead of weeks or months. 

Anti-matter drive fast couriers are vessels designed for this specific purpose. This vessel can travel from Venus to Mars in a week and from Mars to Jupiter in a month. The fast courier is the swiftest vessels currently made and is able to reach at much as one half of one percent of the speed of light. To manage this, this spacecraft must also carry 6 tons of antimatter in a 100 ton magnetic containment vessel. In an emergency, this containment facility can be jettisoned. 

\textbf{Destroyer:} One of the largest military spacecraft in common use, destroyers use an antimatter drive holding 150 tons of antimatter in a 2,000-ton magnetic containment vessel. This antimatter can also be used to provide the spacecraft’s missiles with anti-matter for devastatingly powerful anti-matter warheads. This spacecraft is also armed with railguns, nuclear and high explosive missiles, and point defense lasers. In addition, all destroyers carry a contingent of 20 fighters. 

\textbf{Fighter:} This small, short range military vessel is designed to be crewed by an infomorph or AI. If needed, however, it can hold a single synthmorph or vaccumadapted biomorph as a pilot. It carries 3 lasers and 2 railguns mounted on small pods placed around the middle of ship that can fire in any direction. A single missile launcher is located in the nose of the fighter and typically holds 6 small high explosive missiles or tactical nuclear missiles (or even anti-matter missiles if facing high-threat targets). 

\textbf{General Exploration Vehicle (GEV):} A GEV is one of the standard vehicles used for exploration beyond the Pandora Gates. It is specifically designed to handle almost any environment. It is a boxy vehicle, 6 meters long, 2.2 meters wide, and 2 meters high. It makes extensive use of smart matter in the lower part of the chassis, and can create wheels or short legs (primarily useful for exceedingly rough terrain). It can even produce limited hull streamlining and propulsion suitable for travel both on and underwater. In addition, it contains a small metallic hydrogen engine that allow it to maneuver in space with an acceleration of up to 0.1 G. GEVs have a Maximum Velocity of 200 (wheeled)/40 (walker)/60 (sea)/40 (submerged). 

The GEV also has a closed cycle life support system that can support up to 6 (fairly cramped) living occupants for up to one month and limited electromagnetic shielding against charged particle radiation. All models are fitted with advanced AI piloting and navigation as well as limited self-repair capacity. In addition, GEV’s have an extensible airlock, a single healing vat, several desktop CMs, and a variety of sensors, including both radar and telescopic full spectrum cameras. 

\textbf{Large Lander and Orbit Transfer Vehicle (LLOTV):} This common vehicle is used for transporting passengers and cargo between a planet or moon and orbit and for short distance transfers between habitats less than 100,000 km apart. This conical vehicle has a curved heat shield on the base and smart material landing legs and grapples so that it can rest securely on any stable terrain and link up with all forms of docking clamps. It comes in variants designed to use either a hydrogen-oxygen chemical rocket or a metallic hydrogen rocket. The use of light-weight smart materials allows the interior to be easily and rapidly reconfigured to accommodate different amounts of fuel, passenger seats, and cargo space. LLOTVs that are not designed for planetary landing or which are designed only to land on airless moons are unstreamlined and look considerably blockier. 

LLOTVs come in two configurations: high or low velocity. High velocity configuration allows the vehicle to land and take off again on Venus or Earth without refueling and for rapid transport between nearby habitats. Low velocity configuration is designed to land and take off again on Mars or various large moons without refueling and for slower and more fuel efficient transport between nearby habitats. The extensive use of smart materials in this vehicle means that LLOTVs that use metallic hydrogen engines can be easily converted between the high and low velocity configurations, requiring less than a day in a wellequipped maintenance facility. However, vessels using hydrogen oxygen engines cannot be converted. Since metallic hydrogen is a much more efficient propellant, landers using it always include significant amounts of extra propellant for emergencies. 

\textbf{Scum Barge:} These huge craft were originally designed for use during the first stages of the evacuation of Earth. They were built to carry up to 20,000 people and to allow them to survive for months or even years, in relatively cramped conditions, until more suitable habitats could be constructed. A number of these vessels are still in service, primarily used as mobile habitats by various anarchic subcultures. The best have had their plasma rockets replaced by modern fusion rockets and carry 5-10,000 in relative comfort. The worst use aging plasma rockets and stretch their life support systems and living spaces to the limit, carrying up to 25,000 poor and desperate residents. 

\textbf{Small Lander and Orbit Transfer Vehicle (SLOTV):} This vehicle is identical in use and design to the LLOTV, except that it is one third the total mass and correspondingly less expensive to build and refuel. Some exceptionally wealthy individuals own private small LOTVs. Using a small LOTV with a hydrogenoxygen engine to take off and land on Venus or for other high velocity uses is exceptionally cramped and allows for absolutely no room for error. Like the LLOTV, this vehicle can be easily converted between low and high velocity configurations and is made in both streamlined and non-streamlined versions. 

\textbf{Standard Transport:} This vessel is one of the most common freighter and passenger vessel in the solar system. While egocasting is by far the most common form of inter-habitat transport, some people prefer to travel by ship and others do not wish to leave their current morph behind. In addition, some goods are easier or cheaper to physically transport rather than duplicating their templates. As a result, standard transports regularly travel to and from every large habitat and inhabited planet and moon in the solar system. These are modern fusion-drive ships that offer fast and comfortable travel for passengers as well as relatively swift transport for small cargoes. 

One of the additional benefits of the standard transport is the fact that it contains four separate passenger compartments, each of which is mounted on a 90 meter-long booms that can extend and rotate to simulate gravity. When rotating at a comfortable 2 rpm, passengers experience Mars level gravity. Typically, the gravity maintained in these pods starts at the local gravity (or Mars gravity, if the local gravity is higher) and over the course of the journey gradually increases or decreases to the gravity of the destination. However, these pods cannot rotate to produce gravity higher than that found on Mars. 

\begin{table} \begin{tabularx}{\textwidth}{|l|X|l|X|l|l|} \hline

\multicolumn{6}{|c|}{\textbf{Vehicles --- Spacecraft}} \\ \hline

&\textbf{Passenger capacity}	&\textbf{Handling}	&\textbf{Armure}	&\textbf{Solidité}	&\textbf{Wound threshold}\\ \hline

Bulk Carrier	&110	&--- &20	&750	&150 \\ \hline

Courier	&13	&--- &15	&500	&100 \\ \hline

Destroyer	&90	&--- &30	&2,000	&500 \\ \hline

Fighter	&1	&+30	&20	&240	&60 \\ \hline

GEV	&6	&$-$10	&15	&200	&40 \\ \hline

LLOTV (HO)	&20	(high-velocity)/100 (low-velocity)	&$-$10	&20	&800	&160 \\ \hline

LLOTV (MH)	&250 (high-velocity)/350 (low-velocity)	&$-$10	&20	&800	&160 \\ \hline

Scum Barge	&20,000	&--- &20	&1,500	&150 \\ \hline

SLOTV (HO)	&3 (high-velocity)/30 (low-velocity)	&$-$10	&20	&400	&80 \\ \hline

SLOTV (MH)	&70	(high-velocity)/100 (low-velocity)	&$-$10	&20	&400	&80 \\ \hline

Standard Transport	&200	&--- &20	&750	&150 \\ \hline

\end{tabularx} \label{tab:spacecraft} \end{table} 



