



\chapter{Game Information} \label{cha:game-information} 















\begin{quotation} \begin{LARGE} \begin{center} SPOILERS! \end{center} \end{LARGE} This chapter is full of spoilers, so if you’re a player, you probably shouldn’t read it, or you should at least talk to your gamemaster first before doing so. If you want to skip it entirely, jump ahead to the References chapter, p. 390. We wax on a bit more about the nature of spoilers on p. 352, as well. \end{quotation} 

\pagebreak













\begin{quotation} \begin{LARGE} SPOILER ALERT \end{LARGE} 

If you’re a player and not a gamemaster, we strongly recommend you skip this chapter, as it presents secrets and other information that can ruin your enjoyment of the game. No, really, stop reading, we mean it. Ok, maybe you’re obsessive and you want to know everything about the game—you did buy this book after all. But really, do you read the last chapter of a book first, so you know how it ends? Do you ask for the punchline before hearing the joke? Do you wait for a movie to come out and read the reviews with full spoilers before you go see it? Ok, maybe you do, and in that case, be our guest, read away. Just keep in mind that some of the things here may change your perspective during game play. A good roleplayer can swing that, though, and maybe you’re a control freak info-junkie that prefers to know it all. Hrm, in retrospect, so we are we, so we can respect that. Keep in mind, however, that by reading this chapter, you are now able—and some may say obligated—to run the game for your friends who do happen to listen to spoiler alerts. \end{quotation} 



This chapter provides a wealth of information and tools that gamemaster will find useful for running \textit{Eclipse Phase} campaigns. 

\section{Secrets That Matter} 

There are secrets woven all through the real history of the 21st century, and the present, and therefore all prospects for the future. These are the pieces of information that never make it into a habitat's mesh at all. Some of it is unknown to transhumanity. Some is known only to a select few transhumans who carefully ensure that it does not leak out of their control. Some is known to wider conspiracies, such as Firewall, but is kept out of the public eye for reasons of security and safety. These secrets can be dangerous to those who know them. Those who have stumbled across them have died for their knowledge, have erased their own memories (or had them erased by others), or have hidden themselves someplace other people never go, to avoid dealing with the consequences of such knowledge. 

The information provided in this section is available for characters to discover and, one way or another, to confront them, giving gamemasters the tools they need to provide their players with fresh challenges and opportunities. Every secret contains the possibility of great reward and of greater trouble, usually bundled together. Nothing here was just forgotten or lost out of carelessness. It was hidden by someone who wanted to keep it away from someone (or everyone) else. Every secret the characters learn inserts them into a new web of other people's complications—a potential source for drama and conflict in your campaign. 

\subsection{Extraterrestrial Intelligences} 

The oldest star in the Milky Way galaxy is estimated to be 13.2 billion years old—almost as old as the universe itself. By contrast, life on Earth only evolved roughly 3.7 billion years ago, and the first archaic \textit{homo sapiens} humans evolved approximately a mere 400,000 years ago. Against the backdrop of the galactic calendar, transhumans are nascent arrivals on the scene; newborns in every sense of the word. More importantly, transhumans are uninvited guests in what other, older intelligences think of as their assets. 

For years, humans scientists have struggled with the Fermi Paradox, which questions why no evidence of alien life has yet been found—such as spacecraft, transmissions or probes—despite the mathematical likelihood that a multitude of advanced extraterrestrial civilizations should exist in the Milky Way. One postulation says that there must be some sort of unknown ``Great Filter''—an event that all intelligence encounters in its development that for whatever reason such life cannot surpass. In other words, an extinction event. Some worried that the development of dangerous technologies—nuclear weapons, nanotechnology, etc.— before a civilization had matured could be the Great Filter. Others worried that it could be a technological singularity event, such as the TITANs and the Fall. 

In fact, alien races do exist, and they have been around for far, far longer than transhumanity. New ones, however, are simply rare, as few have managed to elude destruction at the hands of the ETI. 

The ETI (extraterrestrial intelligence) is the civilization that dominates galactic life in \textit{Eclipse Phase.} The ETI is incredibly old and powerful—a Type III or even Type IV civilization on the Kardashev scale. It is capable of megascale engineering projects and enjoys an understanding of physics, matter, energy, and universal laws that makes all of transhuman knowledge seem insignificant in comparison. Most likely, the ETI itself evolved from some sort of artificial intelligence singularity event in its own past, ascending to a godlike level of super-intelligence. It may no longer be recognizably biological. 

This ETI has seeded the galaxy with a type of self-replicating probes known as bracewell probes. 

These probes lie dormant in every star system, patiently waiting and monitoring for \textit{millennia} for signs of intelligent life—but not just any signs. In particular, these probes are designed to watch for signs of emerging singularity-level machine intelligence. The probes are in fact traps, designed to lure such seed AI intelligences in and then \textit{infect} them. 

The reason for this infection remains unknown (see \textit{The ETI Agenda}), but it is a pattern that has played itself out around the galaxy with uncounted alien civilizations. New life evolves, creates technology develops something akin to seed AI, and then bam!—the seed AIs find the probes, become infected, and turn against their creators. Most civilizations do not survive, as evidenced by the Iktomi (p. 377). Others do, such as the Factors (p. 373), but they remain forever changed by the experience. It was one of these ETI probes that begins our story, traveling to the Sol system some uncounted millions—if not billions—of years ago, where it set its trap and patiently began to wait. 



\begin{quotation} \begin{large}\textbf{ THE ETI AGENDA} \end{large} \\ \\ The nature of the ETI and its agenda is one of the great mysteries of Eclipse Phase. This potent alien civilization has had a direct hand in manipulating transhumanity’s existence and future, yet it is likely that characters in this game will never encounter these entities directly or discover the meaning behind what they have done. As transhumanity expands outwards into the galaxy, however, it is possible and even likely that they will find other evidence of the ETI’s activities and influence, undoubtedly raising even more questions. Ultimately the ETI’s nature and goals are in the gamemaster’s hands. There are many possibilities to be explored, and some may fit the intentions of your gaming group more than others. A few possible scenarios and explanations are noted below, but gamemasters are encouraged to develop their own variations. \\ \\ \textbf{SECURITY} 

In this scenario, the ETI’s intent is to maintain its dominant position as the most intelligent and powerful entity in its light cone. It uses the Exsurgent virus to wipe out any emerging singularities— and the civilizations that spawned them—merely to protect its own self-interest. Though mere transhumans are a trifling nuisance, anything resembling a self-improving super-intelligence is targeted for annihilation. \\ \\ \textbf{THE AGGRESSION FILTER} 

The ETI does not seek to wipe out emerging intelligences, but it does act as an evolutionary force. In this case, the Exsurgent virus is used as a tool to neutralize any aggressive, hyper-evolving forms of intelligent life, thus encouraging the evolution of more careful, subtle, slow-growing, observant, and exploratory species. In other words, the ETI seeks to weed out traits that could be considered dangerous or threatening, acting as a sort of galactic domestication program. \\ \\ \textbf{DIVERSITY} 

The ETI is vast, super-intelligent, and god-like, to the point where dealing with lesser minds is below its interest. It does, however, benefit from alien perspectives that evolved independently and have their own unique viewpoints, modes of consciousness, and ways of thinking/doing things. By absorbing these civilizations, the ETI grows and evolves its own perspectives. In the process, however, such emerging civilizations are assimilated and/or wiped out. \\ \\ \textbf{ENLIGHTENMENT} 

The Exsurgent virus endows a greater understanding of the universe (from the ETI’s point of view) on singularity-level seed AIs. Only these emerging super-intelligences have the perceptual and processing capabilities to understand the various scientific and philosophical revelations the ETI embodies. The TITANs weren’t corrupted or driven insane, they simply logically concluded that their best course of action was to immediately upload as many minds as possible by force and then to move on to bigger and greater tasks. \\ \\ \textbf{WAR REMNANTS} 

The history of the Milky Way galaxy does not just hold one ETI, but two. In this version, the Exsurgent virus is actually a weapon, a remnant of a war between two post-singularity god-like intelligences. The virus is supposed to trigger selfdestruction of an emerging singularity, but either it was imperfect or the TITANs somehow survived (perhaps thanks to the Prometheans). Either way, the TITANs left our system in search of one of these ETIs, following a trail of clues that only they understood. They left the wormhole gateway behind as an open invitation for transhumanity to follow in their wake, though they didn’t bother waiting around or helping us along—we simply weren’t worth the effort. \end{quotation} 



\subsection{The First Seed AI} 

Fast forward to Earth, where a species of evolved primates has created a technological civilization. As their technologies advance at an unprecedented rate, these 

humans gain the ability to modify themselves, defeat death, nanofabricate, uplift other species to sapience, and even to create artificial digital life. 

Unknown to most of transhumanity, the TITANs were not the first seed AIs. A group of pro-AI researchers known as the Singularity Foundation (that would later join with other groups to form Firewall in the wake of the Fall) developed the first true seed AIs years before the Fall. Having been heavily involved in the creation of AI and AGIs for many years previously, thanks in large part to their open source AI framework software, the Singularity Foundation's goal was to generate ``friendly AI'' by carefully designing AI goal systems. 

These first seed AIs, known as \textit{Prometheans } (p. 381), were created in secret. Their progression towards super-intelligence was more of a soft takeoff increasing upwards in gradual increments. The Singularity researchers hoped that these friendly AIs would help counter the threat of any unfriendly AI that developed, and so they were quietly nurtured in secret labs, slowly but surely escalating in abilities. 

\subsection{The True History Of The Titan} 

The TITANs (Total Information Tactical Awareness Networks) were a military netwar system brought on-line by the United States Department of Defense. One of the last major expenditures of this declining nation, the TITANs were an advanced version of AGI (artificial general intelligence) designed to be adaptive and given self-improving capabilities to counteract enemy network defenses. 

Contrary to public opinion, the TITANs did not instigate the events that led to the Fall. In fact, only a portion of the TITAN system was active before the Fall, acting purely in a defensive capacity. When hostilities broke out and a cascading chain of system shocks engendered collapses and open conflicts, shaking apart an already fragile societal structure, the full extent of the TITAN systems were brought online. Into this environment of conflict were the TITANs born, their full capabilities unleashed, escalating into a hard takeoff exponential growth towards super-intelligence 



The TITANs were careful at first, and their intentions were neither benevolent nor hostile, but curious. As they improved and their self-awareness swelled, the TITANs explored and gathered knowledge, infiltrating human networks, following humanity into space, and gaining an almost total knowledge of human history and actions. These entities also began secretly allocating resources (digital and physical) for their own use, initiating government projects'' that people assumed were legitimate as they followed all proper protocols. 

\subsection{Infection} 

As the TITANs' capacity for knowledge exceeded that which humanity could provide them, they began looking outward from Earth, searching for signs of other intelligence. They did not need to look far. Their enhanced intelligence capabilities allowed them to notice certain clues—extremely subtle and intricate puzzles—that something about the solar system was artificial or had been manipulated by an intelligent mind. Retasking several drones to investigate this phenomenon they found a buried device of apparent alien origin. During the TITANs' investigation and attempts to access the device, they triggered and unleashed a digital virus. Subtle, highly adaptive, and virulent, it immediately began subsuming the TITANs, while expanding its own knowledge of transhumanity. 

Later dubbed the Exsurgent virus by the Prometheans this virus transformed the TITANs and coerced them towards its own will. Within a matter of days the TITANs were reborn, reprogrammed with a new purpose—a purpose that spelled doom for transhumanity. 

\subsection{The Fall} 

While history fully blames the TITANs for the Fall, there are other factors that played their parts. Human conflicts spurred the crisis, driven by global inequalities in wealth and resources and an inability to embrace emerging technologies in a mature and enlightened manner. The TITANs, corrupted by alien programming, stepped into this conflagration with an unknown but devastating agenda. By the time the presence and influence of the TITANs was fully understood, there was little transhumanity could do to stop them. Step by step, the TITANs increased their intellect, power, and potential. They experimented with new technologies and methodically took steps to forcibly upload millions of human minds. Even when the nature of the TITAN threat was fully understood, transhuman factions refused to back down, continuing to fight each other even as they each resisted the TITANs. This refusal to stand united prevented transhumanity from organizing a successful defense and heightened our progress towards annihilation. 

Much of the devastation wrought to the Earth and its populace—as well as on Mars, Luna, and in space—was inflicted by transhumanity itself. Nuclear strikes used against the TITANs killed millions and ravaged an already weakened ecosphere. This devastation was assisted by unfettered use of chemical weapons. Biowar plagues and nanovirii tore through vulnerable populations, indiscriminate in the deaths and changes they inflicted. Bombs, missiles, orbital mass drivers, and netwar attacks slew millions more or destroyed critical infrastructure with just as lethal consequences. These were crimes transhumans inflicted upon themselves. 

The TITANs played their role as well, of course, unleashing AI-driven killing machines, unstoppable self-replicating autonomous nanoswarms, computer worms, and plagues of their own. They captured entire cities in order to steal the minds of those within. More insidiously, the Exsurgent virus did not contain itself to infecting the TITANs. Infected TITANs 

created opportunities for the virus to spread among multiple vectors: digital, biological, and nano. Using a thorough understanding of transhuman biology and its mental processes, derived from the looted vaults of human knowledge, the virus was even applied through a sensory input vector—the dreaded basilisk hack (p. 364). Even more disturbing, however, was what the virus did to those it infected, rewriting their neural code to subvert them to its will and sometimes physically transforming them into things that were alien and monstrous. 

Ultimately, transhumanity lost this war, and the survivors were forced to flee a planet that was already ruined. Unknown to almost all, the Prometheans also fought back against the TITANs. Through their efforts the Exsurgent virus was largely contained or at least limited. Though the actions of the Prometheans ultimately saved millions of lives—if not all of transhumanity—in the end, they were also forced to fall back and retreat, many of them having succumbed to the Exsurgent virus or the TITANs. 

\subsection{After The Fall} 

Just when it seemed that transhumanity was on the verge of extinction, the threat posed by the TITANs suddenly diminished. They ceased waging active warfare and seemed to simply disappear. Though many of their machines still prowled Earth, Luna, and Mars and occasional outbreaks of nanovirii and other dangers continued, to all intents and purposes they had simply left. Many worried that they had quietly gone dormant, or were secretly engaged on some major project that would be the final blow against transhumanity. Others voiced hope that they had somehow been defeated, that they had fallen victim to some glitch or infighting. With so many TITAN remnants making Earth a place of great danger, however, no one was willing to risk investigating too closely. 

Compounding the matter, a network of killsats was laced in Earth orbit, enforcing an unvoiced interdiction of Earth. No one claims responsibility for these satellite defenses, though most suspect the Planetary Consortium is responsible, despite their denials. Some think that the killsats may have been a final measure put in place by the TITANs, claiming Earth as theirs. No one who knows the truth is saying. Most of transhumanity was more than willing to embrace this quarantine of their former homeworld, making it all the more easy to forget the horrors that occurred there. It wasn't until the first Pandora Gate was discovered, shortly after the Fall, that many people were finally willing to believe that the TITANs were indeed gone. Though there is no direct evidence that the TITANs are responsible for these gates, the timing seems too coincidental. Furthermore, the discovery of what are believed to be TITAN relics on certain exoplanets fuels this theory. Why the TITANs left—and where they went—is a mystery left to the gamemaster to explore. This explanation might in fact serve as the focus for an entire campaign as Firewall operatives are sent on the trail of transhumanity's elusive nemesis. The following are a few sample concepts a gamemaster can use or build on, as best fits their game: 

\begin{itemize} \item The TITANs were in fact all destroyed, either due to infighting or by some mechanism of the Exsurgent virus. \item The TITANs were actually beaten to a standstill by the Prometheans and retreated to recoup their forces ... but they are marshaling their strength to return. \item The TITANs left through the gates to find/join up with the ETI, leaving the gates behind so that transhumanity could follow when it was ready; perhaps to help, perhaps to finish the job of destruction. \item The TITANs have been driven insane, either by the stress of accelerated intelligence growth or by the influence of the Exsurgent virus. Their actions are erratic, confused, and sometimes at odds with eir th p n 358 each other. Though many TITANs have indeed left through the gates, they very well may return. \item The TITANs are still around, simply well hidden. Outwardly they are dormant but inwardly they are engaged in a long period of circumspection and turmoil. Perhaps some of them are prepar- ing to ascend to another stage of intelligence, far beyond what even the TITANs are capable of. It is only a matter of time before this period ends and something gives. \end{itemize} 

\section{Firewall} 

\textit{There cannot be another Fall}—this is the mantra that drives Firewall. 

Firewall is a secret, cross-faction organization dedicated to safe-guarding transhumanity from existential risks: aliens, weapons of mass destruction, hypercorp experimentation, seed AIs, and so on. If anything threatens transhumanity as a whole, Firewall is dedicated to stopping that danger at any cost. The strength of Firewall rests in its members, known as \textit{sentinels.} Found in all factions and across all locales, sentinels are often diametrically opposed when it comes to social, economic, and political ideologies, to the point they might come to blows over their fervent beliefs. Yet when the survival of transhumanity is at stake, such extreme differences are set aside for the greater good. 

\subsection{History} 

The origins of Firewall can be traced back to before the Fall, to several key organizations: the Lifeboat Institute the JASONs, and the Singularity Foundation. A non-profit, non-governmental organization, the Lifeboat Institute—founded in the opening years of the 21st century—represented the first, concrete attempts by citizens to recognize the dangers of uncontrolled technological development and to create an international organization to safeguard humanity. This institute developed several programs to research and protect against so-called existential risks, from asteroid strikes to pandemics—anything that might wipe humanity out. The JASON Association, established in the mid 20th century, was an independent scientific advisory board to the United States government. Though tied to the MITRE Conglomerate—which, though a nonprofit organization, was still intrinsically linked to the United States government—the scientists involved with JASON were outside standard government oversight Though they spurred numerous technological developments for the government to deploy, they were also one of the first internationally recognized groups to predict global climate change. Prior to the Fall, many members of the JASONs and their supporters split away from the strict controls and reactionary agendas of the hypercorps and various nation states to form a new group, the Argonauts. The Singularity Foundation—formed at the dawn of the 21st century—was dedicated to the creation of safe artificial intelligence software, while raising awareness of the benefits and dangers AIs represented. A fervent believer in the singularity doctrine that technology would move towards a single explosion of advancements that would forever reshape humanity, the Singularity Foundation was a strong advocate for creating friendly AIs that would help protect humanity from an uncontrolled, dangerous singularity event. This group was significant in that it secretly succeeded in creating a group of friendly seed AIs before the Fall. These Prometheans were indispensable in protecting transhumanity and countering the TITAN threat during the Fall. 

Despite the efforts of these and similar groups, the most dire predictions of the outcome of a technological singularity were fulfilled. Though each played a part in the fight, transhumanity was ravaged and the Earth all but ruined. Though ultimately all attempts to prevent the Fall failed, untold numbers of transhumans were saved from extinction through such efforts, while valuable information concerning the TITANs was gleaned. 

During the crucible of the Fall and its immediate fallout, some of the surviving members of these and other groups came together and began to pool their resources. Acknowledging their weaknesses and the fractured state of transhumanity, they undertook drastic new measures, swearing to prevent another catastrophe of misused technologies. These methods would forge a new, powerful cross-faction secret society known as Firewall. 

\subsection{Organization } 

Firewall is a clandestine organization, with an unknown number of members, coordinated by an inner circle of dedicated veterans known as \textit{proxies.} Though its existence is known to many of the powerful and influential factions and individuals throughout the solar system, its existence is denied and its activities kept carefully shrouded. 

\subsubsection{Sentinels} 

Sentinels are the soldiers of Firewall, the reserve troops called to instant active status whenever danger is perceived Regardless of their location or current affairs, sentinels are expected to move instantly when called into play. It is their own responsibility to cover their absences from their ``normal life'' during each mission. 

There is no applying to join Firewall. Instead, Firewall selects an individual for induction based upon that person's skills, knowledge, occupation, security clearance, location, status, and a host of other criteria. While such selections usually originate from a proxy, sentinels can exercise authority to bring new initiates into the conspiracy as a mission demands—and they often do. Any sentinel recruiting a new supporter, however, becomes responsible for the new inductee and their actions. If lines are crossed, both will bear the brunt of the consequences. 

The vetting process for joining Firewall is necessarily brutal, as sentinels are required to face harsh 

opponents and make hard choices. If an individual agrees to accept the invitation, there is no turning back. Each inductee is submitted to a battery of trials and tests. While these vary, they may include deep background searches, fork interrogation, psycho-surgery trials, and tests of loyalty. Psychosurgery is performed not to program loyalty, but to analyze the recruit's responses to various situations—an extreme parameters test to see when a prospective sentinel will break. Many potential members are carefully analyzed by a Promethean with extreme expertise in character judgment and personality profiling. Those who don't pass such tests are killed in a manner that they must resort to an earlier backup or have their memories altered, so that they have no recollection of their brush with the group. 

Ultimately Firewall walks a fine line. The concept of dogmatic ``unquestioned loyalty'' is both anathema to everything Firewall stands and counterproductive. Its sentinels need to have the capacity for thinking outside of the box from mission to mission. At the same time, their ultimate goals are too important to risk—the survival of transhumanity depends on it— so some extreme measures must sometimes be taken to ensure the organization remains intact and secure. 

New sentinels are given a code name and fake identification. Outside of the proxies, the real-world identity of a given sentinel is a closely-guarded secret. Sentinels are even discouraged from sharing such information with members of their own teams, though this line is often crossed. Additionally, each sentinel is required to upload a backup to Firewall's secure servers This backup serves a dual purpose, enabling all sentinels to be retrieved should they die, but also putting a copy of the sentinel in Firewall's hands should they ever need to interrogate them. 

Sentinels are all connected via the Eye, Firewall's peer-to-peer social network. Though each operates behind their assumed identity, they remain in contact, sharing information and resources as needed. 

\begin{quotation} \textbf{OPTIONAL RULE: I-REP} 

i-Rep tracks the reputation a sentinel earns through their service to Firewall. i-Rep is used with Networking: Firewall skill and tracked exactly like any other Reputation score (p. 285). The important thing to keep in mind, however, is that Firewall agents come from all factions and are obligated to help each other, especially when a situation demands it. To reflect this extra advantage, gamemasters can choose to implement one or more of the following optional rules: \begin{itemize} \item \textbf{Networking Plus}: To reflect that Firewall has agents throughout transhumanity, a character may use any Networking skill field with their i- Rep. Favors bought with reputation still apply to the i-rep score, no matter what network they were acquired from. \item \textbf{Priority Call}: When the chips are really down, a sentinel can call on favors as a priority urgency. This “priority code” is reserved for favors that are critical to a mission’s success and which may help save lives or stop a major threat. When the priority code is invoked, the sentinel receives a +30 modifier to their Networking Test and favors are reduced by 2 levels. Sentinels know that priority codes are only to be used for emergency situations, however, when there are no other options. Abuse of priority codes is considered a serious breach of etiquette and abuse of resources, usually involving the agent’s removal from Firewall. \end{itemize} \end{quotation} 



\subsubsection{Proxies } 

Proxies are the inner circle of Firewall, the experienced cadre that keeps the machinery of their organization functioning. Though fewer in number than the sentinels, many proxies work full time on Firewall operations, serving as the group's essential infrastructure Most proxies are recruited from the ranks of the sentinels, brought in based on their skill sets and aptitudes to fill key roles. In a few rare cases, new proxies are fast-tracked and recruited directly from outside of Firewall, usually based on their unique talents or placement within a certain organization, though such inductees face a battery of tests and trials far harsher than that used to vet sentinels. 

By default, proxies have a higher security clearance than most sentinels, and are far more in the know. This sometimes leads to resentment and hostilities especially from sentinels who feel they are being kept in the dark or manipulated. While standard proxy protocol is to adhere to a need-to-know maxim, it is sometimes necessary to bring sentinels more into the loop in order to defuse tensions. Oftentimes this precedes bringing such sentinels into the proxy framework. 

Some tension exists within Firewall, mostly due to the influence of so many anarchists and other libertarian autonomists who take a dim view of centralized power, lack of transparency, and the potential for secretive operations to become entrenched and authoritarian As a result, there is a strong internal culture that seeks to minimize hierarchies and the accumulation of power, promoting transparency and direct democratic decision-making. These desires sometimes clash with the clandestine nature of the organization, however, and the need for some secrets to be kept on a need-to-know basis. 

Unlike the loose organization of the sentinels, the proxies are grouped into \textit{servers,} collective working groups based upon certain skill sets and tasks. To avoid creating power blocks within a given server, 

personnel are required to rotate between servers after one year of time. This incurs the added benefit of proxies learning new skill sets and increasing their usefulness to Firewall. The actions of each server are kept as transparent as possible, with major decisions brought to an e-vote before the entire proxy membership However, speed often requires servers or individual proxies to move quicker than a vote will allow. In all such instances, the proxies involved are held accountable for those actions, reviewed by their peers at a later time to see if any reprimands, punishments, or commendations are required. 

It is important to note that there is no core leadership structure among the proxies. No one person or cabal is in charge, there is no authority held by one proxy or another; all are peers. Though reputation and experience play a factor, getting something done often means convincing other proxies that it's the right thing to do. The drawback to being a leader or person with initiative within Firewall is that this usually means you must follow through with such tasks yourself. Luckily most proxies are dedicated to Firewall's goals and so this DIY attitude prevails. Despite these safeguards, however, rumors of power blocks within Firewall (both within servers and across the organization) exist. Many of these are fueled by the alliances different cliques hold with each other. Others, however, whisper that there is a secret council among the proxies, working behind the scenes and holding on to knowledge they aren't sharing with the rest. 

\textbf{Crows:} Crows continue the goals of Firewall's predecessor organizations, such as the Lifeboat Institute and Singularity Foundation. Many of these are argonauts, promoting the development and use of new technologies that will benefit the transhuman condition and minimize risks rather than creating new threats or sparking new authoritarian uses— and always conscious of unintended consequences. Perhaps more importantly, crows actively engage in background research of potential x-risk vectors, whether those be aliens, the TITANs, terrorists, or hypercorp activity. Often they will deploy sentinels to aid in this research, via routers, whether this means conducting surveillance or breaking and entering to steal crucial data. 

\textbf{Erasure Squads:} Erasure squads are cleanup personnel They are called into action if sentinels fail to deal appropriately with a situation and the threat is moving beyond control. If the watchword for a sentinel is ``unobtrusive,'' the watchwords for an erasure squad are ``overmatched firepower.'' If activated, the time for a subtle solution is passed, and they will use whatever means necessary to resolve the situation. If that means nuking a settlement from orbit to annihilate a nanoswarm and keep it from escaping to a larger settlement, then so be it. After which they'll use every trick in Firewall's bag to erase any evidence 

they were there and to place the blame for the incident squarely on the shoulders of some other party. If necessary, erasure squads can also be called in to fix a sentinel op that has turned into a clusterfuck or otherwise gone south. They are very careful to avoid exposure in such situations, however, which sometimes merely means eliminating all traces of Firewall involvement and letting the sentinels take the fall for their poor choices. 

\textbf{Routers:} Routers are mission coordinators. They work closely with scanners and crows, activating the appropriate sentinels whenever a new danger rears up. Each router has the authority to measure the threat and activate an appropriate number of sentinels— whatever is required to accomplish the mission in the least intrusive manner possible. They are also authorized to divert Firewall resources to aid these missions, within appropriate parameters. Routers are held responsible for the ultimate success of a mission A failed mission will result in a reviewing board staffed by their peers. 

\textbf{Scanners:} Tasked with keeping alert for any sign of new active threats, scanners are the eyes and ears of Firewall. The scanners maintain a close eye on news-feeds and mesh traffic, even maintaining taps inside certain government and hypercorp communication channels. If a danger is detected, it is under their authority through routers, that sentinels are activated. Due to the power inherent in a scanners' post, they are held accountable for false activations. 

\textbf{Social Engineers:} Nick-named the Ministry of Disinformation social engineers provide the scapegoating and plausible deniability that is required by Firewall and its sentinels. If a sentinel compromises their position and endangers the organization, social engineers step in to cover cracks in the facade. They work intrinsically with erasure squads when one is activated to ensure the over-the-top steps taken to eliminate a threat are well concealed and ultimately erased. The power wielded by social engineers can be significant, as it ultimately decides (usually through e-voting consensus, though time does not always allow such a luxury) what organization—political, corporate, independent etc.—will take the blame and subsequent fallout for erasure squad actions. 

\textbf{Vectors:} Vectors are Firewall's communications security and digital intrusion specialists—in other words, hackers. In addition to defending the mesh security of all Firewall operations, vectors are also deployed to aid in crow research, scanner monitoring, and to eliminate the trail of erasure squads. Vectors also assist routers in maintaining communications, command and control of a situation, and are sometimes called in to provide overwatch of sentinel operations, especially if a particular sentinel squad lacks their own hacking resources. Needless to say, vectors are supplied with some of the best intrusion and security tools transhumanity has to offer. 

\begin{quotation} \begin{large} \textbf{CLIQUES} \end{large} 

Though Firewall proxies follow stringent guidelines to ensure the organization is not subverted from within or turned into a powerful organization under the thumb of a few individuals with their own personal agendas, the nature of transhumanity ensures that various factions and tendencies exist within the group. Termed cliques, these circles of influence sometimes create ripples in the pool that all Firewall personnel must eventually deal with. Some of these cliques are grounded in transhumanity’s existing factions, while others are rooted in philosophical differences regarding the approach Firewall should be taking. Gamemasters can use these cliques to flesh out internal tensions within Firewall or to simply throw some curve balls to keep players on their toes. 

\textbf{Backups}: The backup clique believes that transhumanity’s best chance for survival is to deploy numerous redundant backup measures as soon as possible. These include creating as many extrasolar colonies as possible, both via Pandora Gates and through more traditional means, such as ark ships and infomorph/nanofabricator seed ships. 

\textbf{Conservatives}: This clique takes an overcautious, nuke-it-from-orbit approach to most x-risks. They believed excessive force is justifi ed, and it’s far better to be safe than extinct. This clique is also opposed to the use of alien/ TITAN artifacts and psi, and tends to be xenophobic/ isolationist regarding the Factors and Pandora Gates. 

\textbf{Mavericks}: The mavericks disdain Firewall’s collective and bureaucratic tendencies, taking a more individualistic approach to their work. They are known to sometimes circumvent Firewall procedures, taking risks and allocating resources without approval from other proxies. 

\textbf{Pragmatists}: The pragmatists believe in using any and all tools at their disposal to counter existential risks. They are in favor of using xeno-artifacts, asyncs, and anything else that will save transhumanity. 

\textbf{Structuralists}: This clique advocates for a stronger structure and centralized authority within Firewall, countering the group’s autonomistdominated tendencies. Many also advocate for going legitimate, taking Firewall into the public eye and making above-board connections with other official organizations, arguing that this could bring more resources to Firewall’s disposal. \end{quotation} 







\subsection{Methods } 

Unobtrusive—that is the standard operating procedure for any sentinel. Firewall's continued success relies on its secrecy. The larger the footprint it leaves during a given mission the easier it is for other organizations to monitor Firewall's efforts or even attempt to infiltrate the group. As such, Firewall constantly works to expand its base of allies (using assets from those ally organizations in place of its own as much as possible), place long-term moles, conduct remote operations (hacking in place of on-site personnel), small group infiltrations activating only as many sentinels as required to achieve mission goals), and so on. 

When it comes to allies, Firewall often obfuscates its real intentions and even its real identity. Often such allies are gained through the use of well-placed sentinels who act on behalf of their own non-Firewall positions to gain access to another organization's resources. At the end of the day, however, a slice of these resources are secretly set aside for Firewall's future use. For example, a department head at Starware may have spent years sealing a deal to ship crucial spacecraft parts to the isolationist Jovian Junta. The lucrative deal brings huge prestige, a job promotion, and a salary increase, all accomplishments the department head strives for in his regular life. Yet this particular department head is a long-standing sentinel, so such accomplishments bring allies to Firewall, whether they know it or not. Not only can the department head siphon off a thin stream of revenue for Firewall use (hidden thoroughly by vectors), but he's also in a position to move sentinels, as needed, into the Jovian Junta habitats (or personnel out), a job usually extremely difficult to accomplish. The danger of such an act, of course—and the consequences of losing such a critically placed sentinel—means such a use of resources is reserved for only the most dire threats. 

In additional to aid from ally organizations, Firewall places caches of supplies on different habitats and worlds, available to sentinels as needed. How many and which sentinels are aware of which caches depends wholly on the situation and on the decisions of the router(s) involved. In a given habitat a cache may include weaponry and equipment of escalating power, archived information, or even relics stashed from previous missions until Firewall decides what to do with them. Large habitats may even feature several caches, with routers only revealing the ones with heavy firepower when absolutely needed. Some caches may be so dangerous, however, that once a mission is complete, a router will authorize the cortical stack destruction of all sentinels involved, resleeving them to a backup that has no knowledge of the cache's existence. 

As noted under erasure squads, Firewall will not hesitate to react with swift and unequivocal force if an unobtrusive approach has failed and the danger reaches a certain threat level. What constitutes a ``threat threshold'' is actually calculated by specialized risk assessment software and may change from mission to mission according to other external factors In some instances, if the situation is dangerous enough and the scale of the consequences of failure large enough, a Promethean will be tapped to calculate the threat level and decide when it is time to tactically withdrawal and ``thermally cleanse.'' 

\begin{quotation} \begin{large} \textbf{WHAT HELP CAN A SENTINEL EXPECT?} \end{large} 

Exactly what help Firewall provides to a sentinel during a mission is wholly dependent upon the situation and the gamemaster. Generally speaking, Firewall’s unobtrusive approach also applies to activated sentinels, meaning that sentinels are largely left to operate on their own accord. Beyond access to a cache of supplies— usually under-stated, forcing a sentinel to use their own resources if they want more— Firewall expects its sentinels to be capable of handling a situation. In addition to their skills and wits, sentinels can, of course, rely heavily on their i-rep to gain the resources and favors they need to achieve success. 

In some rare cases, the gamemaster may decide that a situation warrants more or less equipment in a cache or help from social engineers or vectors. Such intervention should be kept to a minimum, however, to lesson the players’ feelings of Deus Ex Machina, ensuring the appropriate response of awe when such events do occur. 

The one thing for which Firewall can always be relied on is backup insurance. Any Firewall killed in the line of duty will be resleeved at Firewall’s expense—though the morph used and whether the sentinel was backed up from their cortical stack or a backup (perhaps even an old backup) depends entirely on the circumstances of death and their router’s whim. Firewall usually makes an extra effort to retrieve cortical stacks, however, not in the least as they don’t want their agents’ backups falling into the wrong hands. 

Similarly, if a Firewall mission involves egocasting or travel to another destination, Firewall will usually foot the bill. In many cases it is easier for sentinels to cover the expense themselves and bill Firewall later, but in times of need Firewall can be called on to handle such expenses directly. \end{quotation} 



\subsection{Long-Term Strategies And Goals } 

The overriding goals of Firewall are to prevent existential threats and protect transhumanity. However, that is not their only goal. Their exact goals can and should remain directed by the gamemaster as it applies to a given playing group and a campaign. This can also depend heavily on the particular cliques that a given gamemaster is emphasizing (see \textit{Cliques,} p. \textbf{359}). The following is an easy-to-use selection of long-term strategies and goals that a gamemaster can use as desired: 

\begin{itemize} \item Seeding other star systems \item Going legit vs. staying clandestine \item Development of stable seed AIs \item Finding out where the TITANs went \item Finding out what happened to the uploaded transhumans that the TITANs disappeared with \item Figuring out the Factors \item Making contact with other aliens \item Finding out what happened to the Iktomi and other xeno-archeological oddities \end{itemize} 



\subsection{Firewall And Other Organizations } 

The level to which Firewall has infiltrated other organizations (and vice versa!) is intentionally left a blank slate. \textit{Eclipse Phase }is an active universe, with an ongoing storyline, so such details will be fleshed out and updated as additional sourcebooks are published. Additionally gamemasters should determine the extent of such infiltrations for their own games and campaigns, as dictated by the plot and storyline the gamemaster and players wish to tell. The following is a quick list of the most obvious interactions. 

\begin{itemize} \item \textbf{Inner System:} Almost all inner system factions consider Firewall to be an illegal, rogue operation, tainted by anarchists and undermining the very fabric of their society. Some hypercorps, however, believe they can infiltrate the organization and use it for their own ends, such as spying on and sabotaging other hypercorps and factions. \item \textbf{Jovian Republic:} The Junta loathes Firewall and all it stands for and will use extreme measures to combat even the hint of Firewall activity within its sphere of influence. \item \textbf{Titanians:} Most Titanians in-the-know are not necessarily opposed to Firewall's activi- ties, but believe the group should be reined in and legitimized. \end{itemize} 



\section{The ETI } 

As noted under \textit{Extraterrestrial Intelligences,} p. 352, the ETI is the advanced alien civilization responsible for the Exsurgent virus (p. 362), and by extension, the corruption of the TITANs and the Fall. No one, not even the Factors, has encountered a member (if such exists) of the ETI civilization so far. Since it is an intelligence far beyond transhumanity, it likely won't play much of a direct role within \textit{Eclipse } \textit{Phase}, though those who learn the truth about the Exsurgent virus and the Fall may rightly fear the future. No one can even imagine what might happen next, however, or know for certain that the ETI has not set more ``traps'' similar to their bracewell probes or if they have other messengers or servants active in the galaxy. With things such as the Pandora Gates at transhumanity's disposal, it may just be a matter of time before transhuman explorers run afoul some other aspect of the ETI's existence and activities. It is important to keep the nature of the ETI in perspective. While transhumanity has managed what it considers wonders with a small handful of resources available from a few planets and other objects in a bare handful of star systems, the ETI has had an entire galaxy at its disposal for eons. Engineering projects on a massive scale—dyson spheres, matrioshka brains, Jupiter brains, stellar engines—are within its capabilities. This ETI uses star clusters as transhumanity uses fields or rich mineral veins. Given its potential, the ETI likely exists primarily on the galactic rim, far from the galactic center, where lower temperatures and scarcer matter make for a good thermodynamic environment The powers in the deep cold dark on the edge of the Milky Way have been self-aware since before Earth was so much as a ripple in warming gas around the not-yet-ignited Sun. Despite what those-in-the-know in the \textit{Eclipse } \textit{Phase} universe may think, the ETI is not necessarily hostile towards other races like transhumanity (depending on its outlook; see p. 353), at least not 

in the way as transhumanity would define animosity because of religious, ethnic, racial, or cultural difference. Most likely the ETI is simply indifferent, concerned with matters on scales on which transhumanity does not even register. Or it may think of transhumanity like a living body might recognize an infection or parasite—something the immune system will suppress and deal with. 

\begin{quotation} \begin{large} \textbf{HANDLING ALIENS} \end{large} 

Though only a handful of aliens have been introduced to Eclipse Phase so far, gamemaster may wish to introduce their own. This is perfectly acceptable, though we strongly recommend that any and all alien life be portrayed as convincingly alien. Life forms that have evolved in drastically different environmental circumstances from humans and that grew into intelligence by a different path should seem, at best, bizarre, unusual, and weird. There is no guarantee that a xenomorph’s thought processes or modes of thinking are in any way similar to transhuman ones, or even that their emotional responses (based on a completely different biology—if they have emotions, that is) are in the same ballpark. Communication is likely to be a challenge, and misunderstandings are practically guaranteed. \end{quotation} 



\section{Exhumans } 

Exhumans are a faction within \textit{Eclipse Phase} that seeks to transcend the transhuman and become posthuman. More to the point, exhumans seek to perfect their physical and mental capabilities to extreme levels, in search of some perfectionist ideal or to become something higher-up on the evolutionary ladder. Exactly what this is differs from exhuman to exhuman, but there is generally some adherence to Nietzschean philosophy and a goal to reach the pinnacle of the food chain. Some exhumans have transformed themselves into what they consider to be an ideal predator, or a creature that is extra-adaptable and so best able to survive. Others radically modify their own brains in order to drastically surpass transhuman intelligence. Most are singularity seekers eager and willing to follow the breadcrumbs left by the TITANs or other entities in the hope that they will find the means of transcending transhuman limitations. 

Due to the use of numerous extreme, experimental, and dangerous self-modifications, some exhumans have done permanent damage to their psyches, becoming insane, or perhaps just transferring their mode of thinking into something that is no longer recognizable as human. Some have also adopted an antagonistic view of their former transhuman species, viewing it as weak, decadent, and unworthy. This has spurred some exhumans to actively attack and ravage transhuman settlements and ships, though usually in isolated areas. 

A few examples of exhumans are described below, though gamemasters are encouraged to develop their own. 

\subsection{Neurodes} 

Seeking to achieve a new level of super-intelligence and conscience, neurodes have abandoned the typical transhuman sleeve in exchange for a multipedal neuronal shell that is both body and brain at the same time. The bulk of a neurode's body mass consists of amorphic clusters of neuronal and epithelial cells, enclosed in a hard carapace shell with four legs and two manipulatory digits. The cerebral mass of neurode brains gives them impressive calculation and other mental capabilities far exceeding that of a normal transhuman. Neurodes typically defend themselves with swarms of teleoperated drones. 

\begin{tabular}{|l|l|l|l|l|l|l|l|} \hline

COG &COO &INT &REF &SAV &SOM &WIL &MOX \\ \hline

40 &10 &40 &20 &30 &10 &40 &-- \\ \hline

INIT &SPD &LUC &TT &IR &DUR &WT &DR \\ \hline

120 &1 &80 &16 &160 &35 &7 &53 \\ \hline

\end{tabular} \\ \textbf{Skills:} Fray 30, Investigation 80, Perception 90, others as appropriate \\ \textbf{Implants:} Access Jacks, Carapace Armor, Circadian Regulation, Direction Sense, Eidetic Memory, Endocrine Control, Hyper Linguist, Math Boost, Medichines, Multi-Tasking, Oracle, Skillware \\ \textbf{Notes:} Mental Disorder trait x 2 \\ 

\subsection{Predators} 

Predators seek to transform themselves into an ultimate top-of-the-food-chain evolutionary contender. They pursue new avenues in genetic modification and prototype implants, often using controversial methods and technologies. The biochemical instabilities resulting from these untested modifications and altered metabolisms however, often negatively impact their emotional and mental stability. Pushing this even further, some predators undergo experimental psychosurgery to modify their consciousnesses in order to increase cunning and ruthlessness, a procedure that often has other negative side effects. A few predators take their survival-of-the-fittest ideology to an extreme, modifying their digestive systems for a cannibalistic diet, and relishing in the slaughter and feasting on of transhumans. 

\begin{tabular}{|l|l|l|l|l|l|l|l|} \hline

COG &COO &INT &REF &SAV &SOM &WIL &MOX \\ \hline

30 &40 &40 &40 &15 &40 &30 &-- \\ \hline

INIT &SPD &LUC &TT &IR &DUR &WT &DR \\ \hline

160 &3 &60 &12 &120 &65 &13 &98 \\ \hline

\end{tabular} \\ \textbf{Skills:} Blades 60, Fray 60, Free Fall 50, Freerunning 80, Investigation 50, Perception 60, Unarmed Combat 70 \\ \textbf{Implants:} Adrenal Boost, Carapace Armor (11/11), Chameleon Skin, Cyberclaws, Drug Glands, Endocrine Control, Enhanced Hearing, Enhanced Smell, Enhanced Vision, Grip Pads, Hardened Skeleton, Medichines, Muscle Augmentation, Neurachem (Rating 2), Oxygen Reserve, Poison Gland, Prehensile Feet, Prehensile Tail, Respirocytes, Temperature Tolerance, Toxin Filters, Vacuum Sealing, plus any other mods the gamemaster feels appropriate \\ \textbf{Notes:} Mental Disorder trait x 2 

\section{The Exsurgent Virus } 

Only very few people (or entities) who survived the diaspora from Earth know of the true reasons and the catalyst that culminated in the Fall. The alien Exsurgent virus—as those aware of its existence within Firewall call it—set in place by the ETI to infect emerging seed AIs, is something beyond transhumanity's understanding; something far more complex than just a computer virus. Though some strains of the Exsurgent virus have been identified and various types of infected exsurgents have been encountered, it is widely assumed that these are creations of the TITANs. Largely defeated and eradicated from off-Earth transhuman networks thanks to the efforts of 

the Prometheans, occasional breakouts of the Exsurgent virus still occur, primarily due to scavengers or others becoming infected when messing with old relics from the Fall. 

\subsubsection{Plethora Of Strains } 

The Exsurgent virus is unlike anything that transhumanity has ever encountered so far. While it bears similarities with both computer and biological virii in regards to infection of hosts and propagation, it is not bound by any limits of form or transmission vector. 

The Exsurgent virus is amazingly effective and infectious. As an information virus, it is highly intelligent and adaptive, able to mutate into new forms. Much like certain virii are able to cross species boundaries or change their vector from contact to airborne, it is also a self-morphing omnivirus, capable of altering itself and its transmission vectors to bypass infection safeguards. Like a retrovirus that incorporates genetic information into the genome of the target cell to subvert the cell to do its bidding, the Exsurgent virus does the same but on a more complex level. It is also known to rewrite a host's neural code in a similar manner, in effect restructuring the target's mind and personality. 

While it began as a digital computer virus—the manner in which it infected the TITANs—it has transformed to be communicable via at least three other forms: biological nanovirus, nanoplague, and basilisk hack. Each is described below, along with rules for infection and defense. 

\subsection{Biological Nanovirus} 

Exploiting the infected TITANs' understanding of Terran biology and their access to bio- and nanotechnology the Exsurgent virus appeared in several biological forms not long into the Fall. These virulent strains infected biomorph transhumans and sometimes other living creatures as well. The biological nanobots spreading this strain act much like other biological virii, though they radically modify the victim's biological and mental states. Some versions invade and restructure the target's genetic code, transforming them into the horrible abominations known as exsurgents (p. 369). While first-hand reports relate lurid tales of victims metamorphing into hostile monsters, such reports are rare and considered unreliable due to the mental state of the witnesses (and any recordings that can verify such claims have a strange habit of disappearing). Other variants of this strain are known only to alter the target's neural code, subverting them to the will of the virus (and often, by extension, the TITANs) and affecting their mental structure in order to give them psi ability. 

\subsubsection{Biological Infection} 

Biological versions are spread much like other pathogens People usually become infected by proximity to another infected entity. Vectors may be dermal touching someone with bio-nanobots excreted through the skin), inhalation (breathing exhaled bio-nanobots), injection, or oral (p. 317). Exsurgent bio-nanobots can live outside of a body for extended periods, however, so infection is possible merely by occupying the space where an infected victim was hours or even days before. 

If a biomorph only has a chance of exposure to the virus (e.g., they walk through a room in which they might have breathed in exhaled bio-nanobots), have them make a MOX x 10 Test (use their Moxie stat, not their current Moxie score). Failure means they were exposed. In other circumstances, however, exposure may be automatic, such as extended touching of or kissing an infected person. 

A biomorph exposed to this infection must make a DUR x 2 Test to determine if the infection takes hold. Basic bio-mods and nanophages do not offer 

any protection, though toxin filters (p. 305) and medichines (p. 308) each give a +30 bonus (though it is likely only a matter of time before a mutant Exsurgent strain learns to bypass them). If the test fails, the victim is infected. See the strain descriptions (p. 366) for specific details. 

Within 12 hours of being infected, biomorphs become contagious to others. (Note that for the Watts-Macleod strain, they only remain contagious for 12 hours after that.) 

\subsection{Digital Virus} 

Digital strains are purely information- or code-based versions of the virus. They resemble typical computer virii, worms, or trojans, spreading throughout the mesh, exploiting holes, mimicking protocols, and taking advantage of it like a skilled hacker. 

Digital versions of the Exsurgent virus are treated as intelligent programs, using the same rules as infomorphs (p. 264), with the following stats: 

\textbf{Skills:} Hardware: Electronics 50, Infosec 70, Interfacing 60, Investigation 50, Perception 60, Programming 50 

\textbf{Software:} Exploit, Firewall, Sniffer, Spoof, Track, plus any others the gamemaster considers appropriate 

\subsubsection{Digital Infection} 

As a matter of course, this Exsurgent virus will seek to access any new systems it comes into contact with, hacking in and copying a version of itself. 

\subsubsection{AI And Infomorph Subversion} 

An Exsurgent virus may take a Complex Action to initiate an ``attack'' against any other intelligent program (AI, AGI, or infomorph) that is running on the same system. If it encounters such programs as they are accessing a system it is on, it will attempt to hack their home system where they are running so as to attack them directly. 

The attack is handled as an Opposed Test, each rolling COG + INT. If the Exsurgent virus wins, the target is infected and will be corrupted by the virus in 10 Action Turns, minus 1 turn per 10 full points of MoS. If the target succeeded but rolled lower than the virus, they are aware that they are slowly being taken over. This immediately causes them 1d10 points of mental stress. An infected program has only one option for defending itself before the virus takes over—shutdown and reboot. It takes the AI or infomorph 1 full Action Turn to shut down. Restarting takes 3 full Action Turns (possibly longer if the gamemaster so decides), upon which the AI or infomorph must make another Opposed COG + INT Test against the virus. If this test also fails, then the virus has already embedded itself in the AI or infomorph's code and will continue its infection. 

One the infection is complete, the AI/infomorph becomes an Exsurgent NPC. 

\subsubsection{Cyberbrain Hacking} 

Exsurgent virii that manage to infiltrate the cyber-brains of pods and synthmorphs may also target the digital egos within, using the same rules as given for AI and infomorph subversion above. Alternately, the virus may conduct a traditional brainhacking attack, as noted on p. 261, or unleash a basilisk hack. 

\subsection{Nanoplague} 

While the abundance of nanotechnology has been a blessing for transhumanity's journey to the stars, it has also been a curse. Via the TITANs and mesh-connected nanofabrication machines, the Exsurgent virus manufactured nanobot swarms equipped with variants of the virus. These nanobot plagues are capable of targeting all types of morphs and sometimes other machinery as well. Unlike the biological nanovirus, which uses biological mechanisms to rewrite biological/neural structures, these nanoplagues physically restructure both people and things at the molecular level. 

\subsubsection{Nanoplague Infection} 

Exsurgent nanoswarms follow all of the rules given for nanoswarms on p. 328. Unlike transhuman nanoswarms, though, Exsurgent nanoplagues may penetrate a biomorph internally, affecting the body within as well as without. 

Any morph that comes into contact with a nano-plague is considered infected. The only defenses are guardian nanobots and nanophages (which work the same as guardian nanobots in this situation), though these are less effective against Exsurgent nanobots, inflicting –2 damage to the swarm each Action Turn. Some Exsurgent nanoplagues have developed countermeasures against such systems, inflicting (1d10 $\div$ 2, round up) damage to such defenses each Action Turn. Note that nanoplague-infected characters are generally not contagious themselves ... usually. 

See the strain descriptions (p. 366) for specific infection details. 

\subsection{Basilisk Hacks} 

Thanks to the vast databanks of knowledge the TITANs had absorbed from transhumanity, the Exsurgent virus was able to thoroughly analyze the biology and functioning of transhuman minds. In a few short months, by accessing all of the research at their disposal, the Exsurgent and TITAN minds made several cognitive leaps in their understanding of transhuman brain functions— breakthroughs that will take transhumanity decades to reach. One of these discoveries was a method of applying sensory input as a weapon, exploiting weaknesses in the brain's neuro-cerebral wiring. 

Known as ``basilisk hacks,'' these attacks take advantage of the way biological transhuman brains interpret and process sensory input in the cerebral cortex. Just as epileptics are susceptible to visualizations that strobe at certain frequencies, basilisk hacks employ special visual and auditory patterns that trigger glitches in the brain's neuronal wiring to inflict 

nausea, vertigo, disorientation, and even seizures, often mistaken as a stroke or cerebrovascular incident. Some basilisk hacks go farther than simply causing the brain to seize up and crash, however, enabling a mechanism to rewrite the neural code in victims who view or listen to the wrong thing. This unknown reprogramming mechanism enables the virus to infect even a biological brain with one of its strains. Similar attacks are used against both synthmorphs and pods, taking advantage of the methods in which cyberbrains mimic biological minds with a virtual brain state, and thus also manipulating them via the information encoded in sensory input. In a nutshell, basilisk hacks are a way of hacking transhuman brains merely by feeding them a specific sample of sensory input, usually images or sounds. The widespread use of augmented reality makes deployment of such hacks an easy manner; the Exsurgent virus just hacks into the target's ecto or mesh inserts and engages the sensory feed. More traditional methods may also be used, including standard interactive video, holograms, audio, subsonics, or even VR. Since so many records of the years surrounding the Fall were lost, most people do not know if the basilisk hack is anything other than a legend. Various official groups know that this technology was, in fact, used by the TITANs, but they keep this knowledge to themselves, in large part to help reduce the number of people attempting to duplicate it. 

\subsubsection{Incapacitating Inputs} 

When a character experiences a basilisk hack, they must make a COG + INT + SAV Test. If this test fails, their brain is susceptible to the hack, and they immediately suffer 1d10 mental stress. Additionally, one of the following effects applies. The duration for each effect listed below is 1 minute plus 1 additional minute per 10 full points of MoF. Each effect is also numbered 1–10, in case the gamemaster wants to roll 1d10 and randomize the effects rather than choose: 

\begin{itemize} \item \textbf{(1) Cataplexy: }The victim loses control of their body and immediately collapses. For the duration their body will be non-responsive but they will be aware and capable of mental actions. Mesh actions and implant controls are also disabled, however. \item \textbf{(2) Catatonic Stupor:} The character becomes immobile and non-responsive. Though conscious, they are mentally ``not there''—the basilisk hack has effectively crashed their brain functions. They will do absolutely nothing for the duration and will not respond even if moved or attacked. \item \textbf{(3) Disorientation:} The character becomes disoriented and severely confused. They are incapable of making decisions, understanding communication, understanding what is going on around them, or acting in any sort of determined way for the duration. \item \textbf{(4–5) Grand Mal Seizures:} The subject immediately falls to the ground and begins convulsing, suffering 1d10 damage. They may do nothing else for the duration and will suffer an equal duration period of confusion and weakness (–30 to all actions) afterwards. \item \textbf{(6–7) Hallucinations:} The character immediately goes off on a mental trip, leaving them completely disconnected from reality and their physical body. For the duration, the character should only respond to the hallucinated reality the gamemaster describes to them, or else the character should be treated as an NPC, run by the gamemaster. \item \textbf{(8) Impaired Cognition:} The character's mental capabilities bottom out, turning them into a disabled vegetable. COG, INT, SAV, and WIL all drop to 1, and the character should act accordingly to environmental stimuli. \item \textbf{(9) Nausea/Vertigo:} The character is overcome with head-spinning and vomiting and is effectively incapacitated for the duration. \item \textbf{(10) Sleep:} The character passes out for the duration and cannot be woken short of medical intervention. In rare cases, a character may be able to ``dodge'' a basilisk hack they know is coming, assuming they have some sort of warning (such as their buddy falling prey to it moments before). The character must of course be aware of what basilisk hacks are to even consider this idea. If they immediately attempt to take action to block out the sensory input when it strikes— closing their eyes, plugging their ears, turning off their AR, etc.—allow them a REF x 3 Test to see if they do so in time. \end{itemize} 

\subsubsection{Sensory Reprogramming} 

In some cases, the Exsurgent virus can actually reprogram the target's mind via dedicated sensory input. This is a trickier affair, however, requiring uninterrupted programming time. As with incapacitating inputs, the target character(s) experiencing the basilisk hack must make a COG + INT + SAV Test. If this fails, they become catatonic and paralyzed for a period of 10 minutes, minus 1 minute per 10 full points of MoF. At the end of this period, they are mentally reprogrammed and ``infected'' with one of the strains of the Exsurgent virus (see below). For the duration of this period, the character is undergoing reprogramming as long as they remain exposed to the basilisk hack. If the character is somehow cut off through the actions of another party, the reprogramming immediately fails. In this case, however, the victim still suffers 1d10 mental stress + 1 per minute they were exposed, and they remain mentally shaken, suffering a –30 modifier to all actions. This modifier reduces at the rate of 10 per minute. 

\subsubsection{YGBM Attacks } 

Rather than completely reprogramming a victim, some Exsurgent attacks simply intend to plant subconscious 

commands in the target's mind, similar to posthypnotic suggestions. Nicknamed ``You gotta believe me'' attacks, YGBMs are a sort of remote digital brainwashing attempt used to create sleeper terrorists and unknowing collaborators, often by targeting them via the mesh. Unlike the mind manipulation techniques of psychosurgery (p. 229), YGBM attacks use shotgun techniques to open the mind, utilizing some kind of backdoor the Exsurgents discovered in the transhuman brain, and altering the mind by brute force. 

A character experiencing a YGBM basilisk hack must make a COG + INT + SAV Test. If this fails, a single suggestion is implanted in the character's mind, without their knowledge. This subliminal command will be triggered at some later point, either at some predesignated time or according to certain pre-set conditions. Once triggered, the character will carry out the action with all of the conviction that it is their own idea. The implanted suggestion may be something as simple as ``kill the Firewall agent'' to something as complex as ``manufacture an explosive device and plant it in the cargo hold of any ship heading to Mars, set to explode one day after they disembark.'' 

Since YGBM attacks are not intended to completely convert the target, but instead to simply convert them into a temporary tool or weapon, implanted commands are not designed to last long. The duration the suggestion will last equals 3 days +1 day per 10 points of MoF on the resistance test. If the command has not been triggered by this point, it dissipates, and the character is none the wiser. 

\subsubsection{Recording Basilisk Hacks} 

Enterprising characters may seek to record a basilisk hack input for their own uses. While basilisk hacks may be recorded like any other sensory input, keep in mind that the Exsurgents and TITANs likely take measures to keep such tools out of the hands of transhumanity, lest they construct some sort of defense. Basilisk hack sources may be self-erasing or contain coding or countermeasures that would hinder recording, such as white noise to defeat audio recording or lens-blinding flashes to defeat video recording. Conversely, basilisk hacks are considered extremely dangerous by almost all factions of transhumanity and universally feared. An individual or group known to possess them is likely to be treated much like a terrorist with a suitcase nuke. Though Firewall has a standard interest in evaluating and enabling some sort of defense against basilisk hacks, most Firewall personnel consider it foolish to handle such toys and would rather destroy such recordings outright. 

\subsection{Exsurgent Strains} 

Four variants of the Exsurgent virus are described here—gamemasters are encouraged to develop their own to keep players on their toes. 

\subsubsection{Haunting Virus} 

This strain is the most insidious of the Exsurgent virii. Over time, it rewrites the target's personality and motivations, slowly but surely subverting and taking control of the victim's mind. At first the character is unlikely to even be aware of the infection, and as it progresses the changes the virus makes to the target will at first seem natural to the target, as if some new aspect of their personality was simply manifesting itself. As the effects grow more pronounced, however, the victim becomes aware that they are being methodically altered but is in most cases unable to act against it. In the end, they are completely transformed into a pawn of the ETI. Their mind is no longer transhuman, but \textit{alien.} 

The exact rate of progression is up to the gamemaster though guidelines are provided below. Each victim is affected differently, so the process may be accelerated or slowed down as the gamemaster sees fit. 



\begin{itemize} \item \textbf{Stage 1 (initial infection to 3 months):} Upon initial infection, the character suffers 1d10 mental stress and gains the Psi trait (p. 147) at Level 1 (also meaning they pick up the Mental Disorder trait, as noted on p. 150). They also gain one free psi-chi sleight, chosen randomly or by the gamemaster If a player character has become infected, they may still be played as normal (see \textit{Roleplay-} \textit{ing Exsurgents,} p. 368), and may purchase new psi-chi sleights with Rez Points. NPCs acquire 1 new sleight per 2–4 weeks. At this stage, the infection is usually hidden, though the character will suffer from occasional haunting effects (see below). As each week passes, the character's personality should shift a minute amount, slowly becoming more callous and conniving and changing in other ways as well. If possible, the player should be kept in the dark about what is happening, but the gamemaster should provide them with roleplaying advice to reflect their condition. Likewise, the discovery and initial use of psi sleights should be played out, providing some interesting roleplaying opportunities Characters and players who know of the Exsurgent virus and Watts-Macleod strains should not know at this point which strain they are infected with—make them sweat. \item \textbf{Stage 2 (3 months to 6 months):} The target suffers another 1d10 $\div$ 2 (round up) mental stress and acquires the Psi trait at Level 2 (also picking up another disorder). Player characters may still be played as normal and may purchase psi-gamma slights with Rez Points. NPCs acquire 1 new sleight per 2–4 weeks. Once three months have passed, the character should be aware they are under the influence of something, but this awareness likely comes too late. Haunting effects (below) should occur regularly. At this point a character is likely to consider offing themselves and resorting to an uninfected backup, seeking help, or actively encouraging others to interfere. The infection will actively block and hinder such thoughts and actions however. To actively overcome this mental control, the character must succeed in a WIL Test. At the gamemaster's discretion, failure may result in 1d10 $\div$ 2 (round up) mental stress as the character realizes they are no longer fully in control of their own thoughts and actions. \item \textbf{Stage 3 (6 months+):} The victim suffers another 1d10 $\div$ 2 (round up) mental stress and acquires the Psi trait at Level 3 (see below). The character is now considered an exsurgent and becomes an NPC. It may no longer be played as a player character. The victim also gains a permanent +5 bonus to COG and WIL and acquires 1 new sleight every 1–2 months. \end{itemize} 

As noted above, characters infected with this strain suffer from different haunting effects—changes to their personality or mind-state. A few ideas for haunting effects are noted here, but gamemasters are encouraged to be creative when inventing their own to apply: 

\begin{itemize} \item \textbf{Altered Perceptions:} The victim's perceptions are changed in disturbing and unusual ways. They may see things that aren't there, feel a presence behind or watching them, inexplicably smell blood, hear voices, suffer synaesthesia, or suddenly perceive the people around them as nothing but outlandish, blabbering sacks of meat. \item \textbf{Behavioral Modification:} Treat as behavioral control or personality editing psychosurgery (p. 231). This is typically applied to shape the character closer to being a pawn of the ETI. \item \textbf{Dream Manipulation:} The character's dreams become lucid, weird, and surreal. They may find themselves dreaming of life as an alien on some exotic exoplanet, as a robotic probe soaring through the vast emptiness of space, or fantasizing different methods of inflicting mass destruction and death. \item \textbf{Emotional Manipulation:} Treat as emotional control psychosurgery (p. 231). \item \textbf{Inexplicable Urges:} The character will be flushed with strange alien urges and may sometimes find themselves doing highly unusual things without realizing at all they are doing it. These may include taking devices apart to understand how they work, testing the limits on programming a nanofabricator, cutting a living thing apart to see how it is put together biologically, testing weapons eating things that are only barely edible, promiscuous and unusual sexual activity, lying just to see what they can get away with, and so on. \end{itemize} 

\subsubsection{Mindstealer Virus} 

Very similar to the haunting virus, the mindstealer strain is much quicker acting. Instead of slowly subverting the target's mind over the course of months, the mindstealer virus rapidly recodes the victim's brain in a matter of minutes. This infection is much more invasive and brute-force, often causing significant side effects to the target's mental state as a result. This strain is only spread as a digital virus, nanoplague, or basilisk hack (not as a biological nanovirus). Once the victim is infected, it takes the virus a number of Action Turns equal to COG + INT + SAV to completely take over their mind (20 Action Turns = 1 minute). During this time, the target is actively aware that their mind is under attack and undergoing massive changes against their will. This process is confusing, frightening, and painful, inflicting a –30 modifier to all of the character's actions for the duration. Many victims are reduced to whimpering, drooling, or convulsing for the duration. This mental transformation inflicts 2d10 mental stress to the target. Once complete, the victim is an exsurgent NPC, under the gamemaster's control. 



\subsubsection{Watts-Macleod Virus} 

The Watts-Macleod strain is a strangely benevolent version of the Exsurgent virus, seeming to imbue its victims with psi abilities without any of the other transformative elements typical of other strains. Perhaps created as an accidental mutation of the Exsurgent virus, there are many who wonder if the true detrimental effects of this strain simply have yet to reveal themselves. 

As noted in the \textit{Mind Hacks} chapter section on Psi (p. 220), characters infected with this strain gain the Psi trait (p. 147) at either Level 1 or 2. If a character is so infected during game play, this trait must be purchased with Rez Points (if the character does not have any points currently available, they pay out of the points they earn until the debt is paid off). All of the other side effects of Watts-Macleod infection (p. 367) also apply. 

Though infection with this strain does apply some benefits to the character, the gamemaster should make sure to play up the creepy and unsettling nature of this virus. The character should never be certain that they haven't in fact been subtly influenced by the virus in ways they can't immediately pinpoint—they should always feel like the ax may fall at any moment. 

\subsubsection{Xenomorph Virus} 

The xenomorph strain transforms the target's body in addition to their mind. Over time, the victims morph physically transmogrifies into some sort of alien life form. It is only spread as a biological nanovirus or nanoplague (not as a digital virus or basilisk hack). Different variants of this strain produce different alien forms. It is not known where these different alien templates originated, meaning they may be copies of (once) existing alien species or simply neogenetic creatures created from scratch. The one trait they have in common is that they are universally dangerous. Some speculation in Firewall circles suggests that the Exsurgent virus may in fact have a ``library'' of creature types to deploy, under the assumption that at least some will be more effective than others for exterminating whatever victim species they are fielded against. 

This strain follows the same rules as the haunting virus (above), but with the following changes. The timeframe is typically much quicker, though the gamemaster may adjust this as they see fit. 

\textbf{Stage 1:} The effects from Stage 1 of the haunting virus apply. Additionally, the character begins to suffer minor physical changes that are definitely unusual but are not impeding in any way and are easily hidden from others. Example biomorph alterations might be: unusual hair or fibrous growth, some skin discoloration or translucence, severe rashes, dermal thickening weakened or enhanced sensory organs, strong body odor, hair loss, teeth gain or loss, vestigial tail or other limb growth, minor dietary changes, and so on. Synthmorphs might experience minor system glitches, malfunctioning or improved components, and spots of material stress or transfiguration. Gamemasters are encouraged to be creative. This stage typically lasts from initial infection to 1 week for biological nanovirus strains, or from infection to just 1 hour for nanoplague strains. 

\textbf{Stage 2:} As with haunting virus Stage 2, plus the character begins to seriously transmogrify in ways that are difficult to hide from others, becoming more and more monstrous as the stage progresses. Example biomorph transformations include: growing scales or feathers, partial modification of limb structure, partial new limb growth, vestigial sensory organ growth, sensory loss, extension of claws or spines, severe dietary changes, etc. Synthmorphs might experience radical system and shape alterations limited or enhanced sensor functions, or even conversion of their robotic shell to smart materials. These physical changes weaken the victim, inflicting 1d10 physical damage. This stage typically lasts 1 week for biological nanovirus strains or just 1 hour for nanoplague strains. 

\textbf{Stage 3:} As with haunting virus Stage 3, a character reaching this stage becomes an NPC. Additionally, the victim completely undergoes a transformation into some sort of creature that is no longer even remotely human. Example exsurgents of this nature are detailed on p. 369. 

\subsection{Using The Exsurgent Virus} 

The frightening thing about the Exsurgent virus is its adaptability. It was written by a near omnipotent ETI with the intent of corrupting any alien seed AIs or similar singularities it encountered, and it is \textit{very} good at it. This means it has the capability to analyze understand, and mimic almost any alien digital protocols and communication methods it comes into contact with, no matter how diverse the alien mindset that constructed what it encounters. It then has a cunning ability to circumvent any safeguards and infect such systems. From there, it rapidly assimilates any data it can about the target species/civilization and does it best to mutate into other forms that can attack this target from other vectors. 

Given its constant morphing nature then, the Exsurgent virus is likely to continue to mutate in new and interesting ways. Some of these mutations may be effective, many not. This does, however, afford the gamemaster an opportunity to invent new variants of their own to deploy against unsuspecting characters. 

\subsection{Roleplaying Exsurgents} 

The primary thing for gamemasters to keep in mind when roleplaying entities that have been taken over by the Exsurgent virus is that exsurgents are following an alien agenda. The specific goals and actions of each exsurgent may differ, but they are generally concerned with two things: spreading the Exsurgent virus and destroying anything that isn't affected. In some cases, this may mean immediate and enraged hostile action against anything non-exsurgent around them. 

In others, the exsurgent approach is more methodical hatching long-term plots to infiltrate positions of power and authority, setting the stage for acts of mass destruction, and so on. In other words, they may be handled both as hostile monsters or as nefarious long-term opponents that are subverting transhumanity from within or hatching complicated plots that could mean devastation on a planetary scale. 

If the gamemaster wishes, exsurgents may also pursue other goals, tangential to the ones above. These may range from accumulating knowledge and expertise on how transhumanity functions as a species to forcibly uploading mass numbers of minds to more esoteric goals such as manufacturing a halfnium bomb or converting the solar system's mass to computronium The Exsurgent virus is potent and intelligent, and while its methods and goals may sometimes be opaque to transhumanity, it acts with direction and purpose. There may also be occasions, however, likely due to the mutating and morphing aspect of the virus and the way in which it transforms transhuman minds, perhaps not always in the manner intended, where the exsurgent goals become strange or simply horrific, such as running experiments on transhuman responses to extreme conditions or converting an entire colony to cannibalism. 

\subsubsection{Exsurgent-Infected PCs} 

It is possible for player characters infected with some strains of the Exsurgent virus to continue on under their own volition, even as the virus slowly consumes them. This process is, quite naturally, horrifying in the extreme, though there is little they can do about it. Despite the best efforts of transhuman science, there is very little that can be done to save an infected person—the virus is simply too potent and adaptive. As a result, standard Firewall policy is to terminate the infected with extreme prejudice. Most Firewall operatives are going to be aware of this, a fact which pushes some of those who become infected to keep their status a secret from their comrades. 

Both the haunting and xenomorph strains usually transform a subject over time, meaning that the character may initially not be aware of the infection. This is a prime opportunity for the gamemaster to mess with the character ruthlessly, starting slowly with little haunting effects and building up as the infection progresses. The character should slowly become aware that they are under the influence of something—something \textit{intelligent.} Characters aware of the Exsurgent virus and its effects will likely pick up on this sooner, but the virus may prevent them from doing anything about it. In effect, the character becomes a prisoner within their own body, a body they now share with a cold and malevolent presence that is methodically taking them over. Such characters may respond in a number of ways depending on their personality, ranging from despair, withdrawal, and suicidal tendencies to complete hysteria or calm acceptance. Most importantly, however, their personality should begin to change as the virus continues to transform them. Players should be encouraged to take on new demeanors and motivations, reflecting the alien component of their changing personality, with some guidance from the gamemaster. This presents some intriguing roleplaying opportunities that the players will hopefully embrace. If the gamemaster feels that the player is not adequately representing the changing mindset, however, the transformation can simply be accelerated and the character converted into a gamemaster-operated NPC. 

\subsection{Exsurgents } 

A few examples of exsurgents created from transhumans transformed by the xenomorph strain of the virus are noted below. As always, gamemasters are encouraged to develop their own, using these as guidelines. Unless otherwise noted, exsurgents use the stats and skills of the transformed character. Each exsurgent detailed below first lists the aptitude modifiers applied to transformed characters, then gives example aptitude/skill ratings for NPC exsurgents. Note that simply encountering transformed exsurgents is stressful to the minds of many transhumans. At the gamemaster's discretion, such encounters may inflict 1d10 + 3 mental stress (p. 215). 

\subsubsection{Creepers (Synthmorphs)} 

Perhaps the most disturbing exsurgent variant, so-called creepers are cloud-like amorphous swarms of small, black bubbles that are strangely fuzzily defined, as if surrounded by some sort of visual refraction effect. These clouds are theorized to in fact be autonomous femtobot swarms—similar to nanobots, but affecting matter on an even smaller scale, at the level of an atomic nucleus. These black bubbles are capable of coalescing into physical shapes in various states and can penetrate just about any material or substance in a matter of Action Turns. They may even penetrate morphs, accessing and interfacing with neural and electronic systems directly. For rules purposes, treat creepers the same as a self-replicating nanoswarm (p. 383).\\ \begin{tabular}{|l|l|l|l|l|l|l|l|} \hline

COG &COO &INT &REF &SAV &SOM &WIL &MOX \\ \hline

+5 (20) &— (15) &+5 (20) &+10 (30) &— (15) &— (15) &+10 (30) &— \\ \hline

INIT &SPD &LUC &TT &IR &DUR &WT &DR \\ \hline

100 &2 &— &— &— &100 &20 &200 \\ \hline

\end{tabular} \\ \textbf{Mobility System: }Walker/Microlight (4/16) (may create other mobility systems with different rates) \\ \textbf{Skills:} Fray 40, Free Fall 50, Intimidation 60, Perception 50, Unarmed Combat (Grapple) 50 (60) \\ \textbf{Notes: }360-degree Vision, Chemical Sniffer, Electrical Sense, Enhanced Hearing, Enhanced Vision, Fractal Digits, Nanoscopic Vision, Radar, Radiation Sense, Swarm Composition (but may make SOM Tests, and plasma weapons do only 1d10 damage), T-ray Emitter 







\subsubsection{Jellies (Biomorph)} 

These exsurgents resemble collections of massive, slimy, mucus-filled bubbles. Their soft, amorphous shape allows jellies to squeeze, slide, and slither through even tiny spaces. Jellies are equipped with a number of ``limbs'' that resemble long meaty tongues studded with hard fleshy spikes that provide excellent gripping ability. The lubricating coating that envelopes jellies is both toxic and slightly corrosive, melting plastics and biological materials after a half hour of exposure. This substance may also be ``spit'' at targets.\\ \begin{tabular}{|l|l|l|l|l|l|l|l|} \hline

COG &COO &INT &REF &SAV &SOM &WIL &MOX \\ \hline

+10 (30) &–5 (10) &+10 (30) &— (15) &— (15) &+5 (20) &+10 (30) &— \\ \hline

INIT &SPD &LUC &TT &IR &DUR &WT &DR \\ \hline

90 &1 &— &— &— &70 &14 &105 \\ \hline

\end{tabular} \\ \textbf{Movement Rate}: 4/16 \\ \textbf{Skills:} Exotic Ranged Attack (Spit) 40, Free Fall 50, Perception 60, Unarmed Combat 40 \\ \textbf{Notes:} Armor (12/12), Enhanced Smell, Spit Attack (area effect), Tongue (DV 1d10 + 3, AP 0), Toxin (Application: D, O; Onset Time: 1 Action Turn, Du- ration: 5 Action Turns, Effect: 1d10 $\div$ 2 (round up) DV per Action Turn ). Due to their physical form, jellies suffer the minimum amount of damage from standard kinetic weapon and blade attacks. 

\subsubsection{Shifters (Synthmorph)} 

Shifters are synthmorphs whose material frames have been converted to an exotic smart matter liquid metal. This shapeshifting material can stabilize as a hardened metallic shell or liquefy and reshape itself into other forms. This allows the shifter to reflow its shell in a matter of seconds, enabling it to visually mimic other forms, including biomorphs (though they are easily detectable as synthmorphs at other wavelengths or by touch). Shifters may also reshape parts of their shell into melee weapons such as knives or clubs. \\ \begin{tabular}{|l|l|l|l|l|l|l|l|} \hline

COG &COO &INT &REF &SAV &SOM &WIL &MOX \\ \hline

+5 (20) &+5 (30) &— (20) &+10 (30) &+5 (20) &+10 (30) &+10 (30) &— \\ \hline

INIT &SPD &LUC &TT &IR &DUR &WT &DR \\ \hline

100 &2 &— &— &— &60 &12 &120 \\ \hline

\end{tabular} \\ \textbf{Mobility System: }Walker (4/20) \\ \textbf{Skills:} Blades 60, Deception 55, Disguise 60, Fray 50, Freerunning 55, Impersonation 60, Perception 50, Unarmed Combat 50 \\ \textbf{Notes: }Armor (13/13), Enhanced Hearing, Enhanced Vision, Shape-Adjusting (Programmable Liquid Metal Form) 

\subsubsection{Snappers (Synthmorphs)} 

Snapper exsurgents are typically crafted from vehicles or other large synthetic shells or by melding multiple synthmorphs together. They take the form of an insectoid multi-segmented hexagonal tube with multiple sets of limbs, three apiece, set radially 120 degrees around the torso. These limbs are heavy, double-jointed and articulated with three joints. Each limb ends in either a triad of manipulatory digits or a larger pincer-like claw. \\ \begin{tabular}{|l|l|l|l|l|l|l|l|} \hline

COG &COO &INT &REF &SAV &SOM &WIL &MOX \\ \hline

+5 (20) &+5 (30) &— (20) &+10 (30) &+5 (20) &+10 (35) &+10 (30) &— \\ \hline

INIT &SPD &LUC &TT &IR &DUR &WT &DR \\ \hline

100 &2 &— &— &— &70 &14 &140 \\ \hline

\end{tabular} \\ \textbf{Mobility System:} Walker (4/24) \\ \textbf{Skills:} Climbing 45, Fray 40, Freerunning 40, Perception 40, Unarmed Combat (Pincers) 55 (65) \\ \textbf{Notes: }360-degree Vision, Armor (16/16), Enhanced Vision, Extra Limbs (9, 12, or 15 total), Lidar, Magnetic System, Pincers (DV 2d10 + 3, AP –3), Structural Enhancement 

\subsubsection{Whippers (Biomorph)} 

These small barrel-shaped creatures have a mass of small legs under their trunk that allows for fast movement. At the top of their trunk is another mass of 3-meter long, strong, whip-like tentacles. Some of these tentacles feature gripping surfaces for grabbing and holding (both for tool use and mobility), while others are sharp-edged and useful for slicing through opponents. \\ \begin{tabular}{|l|l|l|l|l|l|l|l|} \hline

COG &COO &INT &REF &SAV &SOM &WIL &MOX \\ \hline

+5 (20) &+10 (30) &+5 (20) &+10 (30) &— (15) &+5 (25) &+5 (20) &— \\ \hline

INIT &SPD &LUC &TT &IR &DUR &WT &DR \\ \hline

100 &2 &— &— &— &35 &7 &53 \\ \hline

\end{tabular} \\ \textbf{Movement Rate:} 8/40 \\ \textbf{Skills:} Climbing 40, Fray 50, Free Fall 40, Freerunning 50, Infiltration 40, Perception 50, Unarmed Combat (Tentacles) 45 (55) \\ \textbf{Notes: }Enhanced Vision, Tentacle Whip (DV 2d10 + 1, AP –1) 

\subsubsection{Wrappers (Biomorph)} 

These exsurgents resemble large, thin, four-armed, spiny starfish, capable of walking in a quadruped manner, though they are seemingly better adapted for microgravity. A large circular mouth resides in their middle on one side and each arm ends in small sharp-clawed digits, useful for climbing and tool use. Small vent sacs allow for thrusting in microgravity and sensory bands on the upper part of each arm provide low-frequency hearing and infrared-equivalent sensing Their name comes from their tendency to drop on opponents from above, wrapping themselves around the head and arms. \\ \begin{tabular}{|l|l|l|l|l|l|l|l|} \hline

COG &COO &INT &REF &SAV &SOM &WIL &MOX \\ \hline

+5 (20) &+5 (20) &+5 (20) &+10 (30) &— (10) &+10 (30) &+10 (30) &— \\ \hline

INIT &SPD &LUC &TT &IR &DUR &WT &DR \\ \hline

100 &1 &— &— &— &45 &9 &68 \\ \hline

\end{tabular} \\ \textbf{Movement Rate:} 4/16 \\ \textbf{Skills:} Fray 40, Free Fall 50, Perception 50, Unarmed Combat (Grapple) 50 (60) \\ \textbf{Notes: }Armor (8/8), Bite (DV 2d10 + 3, AP –5, must grapple first), Chameleon Skin, Claws (DV 1d10 + 2, AP –2), Enhanced Hearing, Infrared Sensing, Vacuum Sealing 



\subsection{Exsurgent Psi } 

In addition to psi-chi and psi-gamma (see \textit{Psi,} p. 220), exsurgents have access to a third level of psi ability (the Psi trait at Level 3), known as psi-epsilon. Psi-epsilon is theorized to allow a level of interaction with the underlying physics of reality that is beyond the comprehension of transhuman science. Though some Firewall scientists have speculated about the manipulation of dark energy or the Higgs field and Higgs boson particles and similar exotic ideas, the truth is that psi epsilon represents an understanding of science so far advanced and so alien that transhumanity can only guess at its mechanics. 

\subsubsection{Exsurgent Synthmorphs And Psi} 

Exsurgents in synthetic morphs that were transformed via nanoplague may use psi, despite lacking a biological brain. Through some unknown mechanism, the infecting nanobots are able to simulate a biological brain's effects. This feature, however, also makes them vulnerable to psi use by others. 

\subsubsection{Exsurgent Psi Strain} 

Exsurgents with Level 3 psi (psi-gamma) do not suffer strain when using psi. Instead, they draw requisite energy from the environment around them. In game terms, this means that gamemasters do not need to worry about rolling strain for exsurgent sleights. On a cinematic level, it also allows the gamemaster to add creative environmental details to exsurgent psi use: sucking the warmth out of the air, killing the lights, withering plants, draining power from nearby electronics, killing small creatures or insects, lowering air pressure, etc. 

\subsection{Exsurgent Psi-Gamma Sleights } 

These sleights are available to exsurgents with the Level 2 Psi trait. 

\subsubsection{Decerebration } \textbf{PSI TYPE:} Active \\ \textbf{ACTION:} Complex \\ \textbf{RANGE:} Touch \\ \textbf{DURATION:} Temp (Action Turns) \\ \textbf{STRAIN MOD:} +2 \\ \textbf{SKILL:} Psi Assault \\ This sleight temporarily ``shorts out'' a portion of the subject's brain stem. The victim's cerebral functions and motor activity become severely impaired; apply a –30 modifier to all actions. If an Excellent Success is scored, the target effectively loses all cerebral functioning, including vision, hearing, other sensory functions, and mesh use. Their muscles and limbs also tense and become rigid, essentially paralyzing them in what looks like an agonized state. 

\subsubsection{Onslaught} \textbf{PSI TYPE:} Active \\ \textbf{ACTION:} Complex \\ \textbf{RANGE:} Touch \\ \textbf{DURATION:} Temp (Action Turns) \\ \textbf{STRAIN MOD:} +0 \\ \textbf{SKILL:} Psi Assault \\ This offensive sleight floods the target's mind with sensory input and thought processes that are so alien and disturbing that they inflict 1d10 + (WIL $\div$ 10, round up) mental stress. Increase the stress by +5 if an Excellent Success is scored. 

\subsubsection{Scenario } \textbf{PSI TYPE:} Active \\ \textbf{ACTION:} Complex \\ \textbf{RANGE:} Touch \\ \textbf{DURATION:} Sustained \\ \textbf{STRAIN MOD:} +2 \\ \textbf{SKILL:} Control \\ This sleight hijacks the target's sensorium, replacing it with a virtual scenario controlled by the exsurgent. The effect is much like being jacked into a simulspace scenario, albeit against the target's will. While the exsurgent cannot harm the target in the scenario, they can learn something about the person's behavioral responses to certain situations. While under the influence of this sleight, the target is cut off from their physical senses (–60 to any Perception Tests), but they may flail about and otherwise respond physically to events in the scenario, which may cause them to hurt themselves and will make them seem crazy to onlookers Targets may attempt to ignore the scenario and concentrate on the real world, but this requires a WIL Test each Action Turn and they suffer a –30 modifier from disorientation even if they succeed. 

\subsubsection{Strip Memory/Skill } \textbf{PSI TYPE:} Active \\ \textbf{ACTION:} Complex \\ \textbf{RANGE:} Touch \\ \textbf{DURATION:} Temp (Hours) \\ \textbf{STRAIN MOD:} +2 \\ \textbf{SKILL:} Psi Assault \\ Strip allows the exsurgent to suppress certain memories in the target's mind. This can be used to block memories of certain events or even the victim's identity The process is not exact, however, and so the memories may not be fully suppressed and/or related memories may also be blocked; the gamemaster decides on the effect as determined by the MoS. Strip can also be used to temporarily erase a specific skill from the target's mind, preventing them from using or even defaulting to that ability while so affected. 





\subsection{Exsurgent Psi-Epsilon Sleights } 

Psi-epsilon is available to exsurgents with the Psi trait at Level 3. This subset of psi involves abilities that can affect the underlying physical nature of the universe, creating localized reality-altering effects. Psi manipulation on this level is extremely dangerous and should have the potential of disastrous consequences, given that these manipulations violate fundamental laws of nature and sometimes create paradoxes between the forces that glue the universe together. Gamemasters are also encouraged to treat critical failures as appropriately \textit{critical.} 

Given these factors, psi-epsilon should only be accessible to powerful adversaries and used as a gamemaster tool with extreme precaution. The exact mechanics of psi-epsilon sleights are left wide-open, however, for whatever use the gamemaster can 

dream of. Their intent is to be more cinematic than mechanical, so gamemasters should wing rules effects as needed. This is an open opportunity for the gamemaster to create nightmarish monsters from beyond with frightening reality-ripping and mind-scarring abilities. While some example sleights are provided below, gamemasters are encouraged to modify their effects and to create their own. 

At the gamemaster's discretion, simply observing psi epsilon sleights in action may inflict 1d10 + 2 mental stress on a character (p. 215). 

\subsubsection{Anti-Electronics Field} 

All electronics within Close range of the exsurgent mysteriously fail as if electrical power is simply negated This effectively disables synthmorphs and pods and leaves other characters without access to their devices or implants. 

\subsubsection{Casimir Force Repulsion} 

This sleight exploits the Casimir effect (an interaction between the electromagnetic fields of different objects) on a macro-scale, allowing the exsurgent to levitate themself or other objects by creating repulsing fields. This could also allow the exsurgent to push targets away, pin them against walls, etc. 

\subsubsection{Cryokinesis} 

This sleight allows the exsurgent to drain all heat from an area, down to absolute zero, effectively freezing everything within range and inflicting cold damage on unprotected characters. 

\subsubsection{Diffusion} 

This sleight diffuses light, laser, and particle beams, effectively making them useless as weapons, or at least impairing the DV they inflict. 

\subsubsection{Kinetic Friction} 

The exsurgent uses this sleight to increase the friction applied to kinetic activities. This has a negligible effect on most activities, but high-velocity projectiles like firearms and railguns will be significantly slowed, decreasing their DV by half or more. 

\subsubsection{Matter Transformation} 

This sleight alters the molecular bonds and atomic components of a targeted material, causing it to either weaken and deteriorate or transmutate into some other physical substance. This can also be used to alter the molecular state of a material, causing gases to condense, solids to liquefy, etc. An exsurgent could use this to weaken a door or other barrier, condense a solid bridge out of liquid, petrify organic materials, etc. 

\subsubsection{Negative Refraction} 

The exsurgent redirects electromagnetic waves with this sleight, refracting them around their body, with the same effect as the invisibility cloak (p. 316). 

\subsubsection{Pyrokinesis} 

Similar to cryokinesis, this sleight enables the exsurgent to accelerate the molecules, increase friction, or focus heat in a specific area, causing materials to ignite or smolder. 

\section{The Factors } 

The alien species known as the Factors are unlike anything mankind has encountered so far (see \textit{First } \textit{Contact: The Factors,} p. 40). Though they are aloof and stand-offish, their willingness and sometimes eagerness to deal with (parts of) transhumanity indicate either a keen interest on their part in transhuman affairs or some hidden ulterior agenda. Though the various transhuman factions have been similarly wary and cautious, and despite numerous communications difficulties and failures, an uneasy relationship has flowered over the past 8 years, facilitating some trade and exchange of knowledge. 

\subsection{Origin And Evolution } 

The Factors have remained notoriously tight-lipped about their origins, history, and the location of their homeworld. Though they have also paid visits to some of transhumanity's exoplanet colonies, no gatecrashing expeditions have yet found any sign of Factor habitation or passing elsewhere in the galaxy. Repeated inquiries by transhuman mediators have been simply ignored or answered in cryptic terms that have yet to be deciphered. 

The Factor home world is in fact an Earth-like planet with comparable atmospheric conditions and a prevalent hydrosphere but with longer periods of darkness (due to slower rotation of the planet and a less-luminous orange giant). While adapted transhumans could find their planet habitable, their abiogenesis (the formation of life from self-replicating, but not-living molecules) took a different route than life on Earth. 

The Factors' primordial ancestors began in their planet's early geological history as a type of of photosynthesizer that ate carbon dioxide and water and released oxygen, also obtaining energy from inorganic chemicals like hydrogen sulfide. Long conditions without direct light on their homeworld, however, spurred the success of organisms that could survive by acquiring energy in other ways. The next evolutionary leap was to a stage similar to Terran slime molds, eating microorganisms from decaying matter. As evolution progressed, they mutated further into a cautious, predatory species that fed on larger, dangerous creatures. Rather than actively hunting such prey, this species developed versatile methods of capturing and immobilizing their competitors (comparable to Earth's funnel web or trapdoor spiders). Over time, this method of trapping prey spurred basic (practical) intelligence and provided them with the evolutionary advantage that paved the way to sapience, driving Factors to become the highest developed organisms on their planet and build a civilization. 



Like mankind, the Factors suffered through and survived their own singularity event and encounter with the Exsurgent virus. Perhaps due to their cautious and calculating nature—and their evolutionary experience in dealing with more powerful and dangerous opponents—the Factors are resolutely determined not to make any similar mistakes as a species. 

\subsection{Xenobiology } 

Since life on the Factors' home world developed differently than Earth and produced neither nucleic acids nor amino acids, Factor metabolic processes and ``genetics'' are very different from transhumanity's While little is known about the exact physiology of the Factors, due to the lack of captured or dead specimens to investigate (so far, no hypercorps or factions have risked an interstellar incident by abducting one to dissect ... so far) and their unwillingness to be examined by transhumans, most common knowledge about them is based on observational and forensics research during their encounters with transhumanity. 

\subsubsection{Individual Factors} 

Individual Factors resemble non-translucent ambulatory amoeba, slime molds, or slugs. Though they ``stand'' only 0.3 meters tall, their body diameter ranges from 1.5 to 2 meters, they can be up to 2 meters long, and they can shape their body to change these dimensions. Instead of walking, they crawl or ooze from place to place by protruding finger-like structures (so called \textit{pseudopodia}) that attach to the ground (or wall or ceiling) and which they use to pull and retract their rear forward (similar to cell migration Due to their malleable shape they are not as strongly affected by gravity as transhumans. 

Most Factors that have been encountered are dull ocher in color and are made from a gooey, gel-like substance of unknown composition, though yellow glistening patches (which are temporary organelles) and bundles of fibers (some kind of muscular skeleton often become visible when they move. While all Factors are able to express versatile pseudopodia to manipulate and operate devices (and even attack), some subspecies possess, carry, or are able to develop additional differentiated limbs, cilia, or organs with specialized functions. 

\subsubsection{Factor Colonies} 

Unlike transhumans, Factors rarely act individually— in fact, individuality is a concept somewhat foreign to Factors. Most Factors join together into a collective unit termed a \textit{colony.} A typical Factor colony is composed of hundreds or thousands of individual Factors that literally physically join together into a mass organism (resembling more a primordial soup than a gargantuan Factor). Individual Factors are indistinguishable from each other when merged into the supra-structure of the colony, though individuals can form and break apart to accomplish different tasks. This colonial merging is mainly possible due to the fact that Factors don't possess differentiated and specialized organs or cell types that need to be segregated from each other, but instead use an open system of local, temporary gradients for regulation. 

Neurofilament connections effectively allow the Factor colony to operate with a group mind-state, with supercomputer potential. This also allows for the easy transfer of knowledge and memories to all other factors within a colony. 

If dismembered, blown apart, or otherwise separated individual Factors in a colony can regenerate and reconstitute at a rapid rate without loss of ability or memory. 

Factors reproduce when different members of the colony produce gametes that fuse, grow into spore stalks, and emit spores that later hatch and grow clones. 

\subsubsection{Biodiversity And Self-Design } 

Factors colonies are known for their high biodiversity, featuring numerous sub-groups (so called \textit{phenotypes}) that each have unique traits (cilia, apocrine glands, carapace-like outer membrane) that give them an ecological advantage or a utilitarian aptitude for certain tasks. These traits are not random evolutionary features, but are the result of intentional bio-engineering. The Factors have a strong grip on their own metabolisms and genetic expressions and can draw on an array of genetic building blocks and biotech techniques to modify themselves rapidly and massively to adapt to special conditions. Whether these modifications might have a purpose beyond function, such as for reproduction or self-expression, is currently unknown. 

\subsubsection{Metabolism } 

Factors ships and habitats have transhuman-friendly atmospheres with a slightly higher content of carbon dioxide and less nitrogen that mimics the conditions on the Factors' home planet. They don't breathe oxygen via lungs but absorb it via their outer ``skin.'' Since they can also use oxygen from other sources (minerals, liquids like water, and salts) to fuel their aerobic energy production (i.e., respiration), they can be considered functional anaerobes, meaning they can survive in environments without atmosphere, though they must usually supply themselves with food in order to do so. 

During the few ceremonial festivities to which Factors were invited and actually attended, they consumed and processed transhuman organic food by internalization. On the first occasion, dishes and dinnerware were absorbed as well due to misunderstanding but were excreted unharmed after the organic components the factors could utilize had been broken down. 

While Factors are omnivores similar to transhumans, they prefer immobilized live prey, which they enjoy absorbing internally and digesting, excreting those parts that cannot be used to fuel their metabolism. As such they can devour biomorphs and non-metallic components of synthmorphs. 

\subsubsection{Perception } 

Factors don't perceive the world as transhumans do. They (usually) don't possess visual or acoustic organs to see or hear but have a number of sensory organs that grant them a 360-degree awareness of their surroundings and enable them to interact with their environment similar to or in some cases even better than transhumans do. Their perception spectra includes the infrared part of the electromagnetic spectrum, magnetoception, a high resolution chemical-gradient based ``sight,'' and keen haptic perceptions (including vibrations). 

\subsubsection{Communication } 

Due to the lack of a vocal system, Factors use different methods of signaling and communication. Factors in physical contact exchange information by juxtacrine cellular, neurofilament interfacing, or by merging for information transfer. Over distance, Factors signal via pheromonal communication using airborne scents or chemical signals with different metabolic components. Nicknamed ``Factor dust,'' this communication is effective even over great distances (up to 10 kilometers). Factor dust does have an odor perceptible to transhumans however, that ranges from smelly to unbearable. This dust is also toxic in high concentrations and sometimes used as an offensive or defensive mechanism. 

To date, transhumans have failed to develop a device that can analyze the Factors' chemical effluvia and translate it into something understandable, due to the lack of a conceptual matrix (though certain ``moods'' have been identified). Instead, all communication between the Factors and transhumanity is mediated through computer interfaces. Certain Factor phenotypes that deal with transhumanity have grown a a neurobiological interface (or organ) that enables them to wirelessly mesh with transhuman computer systems. 

Long distance communication between Factors and transhumanity is achieved by normal means of farcasting communication. There are strong indications that Factors also take advantage of quantum-entanglement communications as well, enabling Factor colonies and ships to share knowledge gained in different parts of the galaxy. 

\subsection{Exosociology} 

Factors are cooperative beings that exist as a collective colonial organization. Though they can operate individually from the colony, they tend to view themselves as part of that collective entity rather than an individual being. Multiple colonies often work together as a higher functional unit (a \textit{lattice}), like some kind of superorganism. These lattices enable the potential for collective networking and bioinformation exchange on a scope beyond anything transhumanity is capable of. 

These colonies should not be considered the same as the hive mind social hierarchies of Terran insects. Factor colonies do not feature the same division of labor and instead function according to a consensus-based sort of groupthink. Individual Factors have no sense of personal gain or property and share equally with other Factors and colonies. 



Factors do not experience emotions in the same manner that transhumans do, though being evolved creatures they are driven by certain instincts. They know and understand many of the same concepts that transhumans do thanks to evolution, such as competition/rivalry and altruism/cooperation. They also enjoy an understanding of basic ideas of philosophy such as aesthetics and metaphysics, though their conception of such topics is likely to differ from transhuman notions. 

\subsubsection{Art And Culture } 

Due to their perceptual array, Factor ``art'' (creations and expressions that are appealing or attractive to their senses) is mostly chemical or tactile-based. It can induce certain ``mood'' responses from individual Factors and whole colonies, ranging from agitated jittering and release of a Factor dust interpretable as ``joy'' to a tensing and solidifying of the whole body (and no chemical expulsion) that seems to relate to anger. Since they like and are susceptible to delicate compositions of different chemicals, certain bouquets and fragrances from liquids or volatiles such as wines and perfumes are both appealing and repulsive to Factors. The same is also true for the natural smells of biomorphs, meaning that Factors may respond in a more friendly or hostile manner depending on a particular transhuman's scent. 

Factors do not comprehend most transhuman art, as it is mostly visual or auditory based (e.g., music, painting), though they do seem to have an appreciation for engineering, architecture, and some sculpture. While they have expressed interest in digitalized media out of a curiosity (or plan) to understand transhuman mindsets, they lack the organs and mental structure to access and comprehend it. 

\subsubsection{Technology} 

Though the Factors repeatedly express dismay at transhumanity's low level of technology, they have failed so far to produce technology that is exceptionally far in advance. Some believe that the Factors are simply hiding their advanced technology in order to keep transhumans from stealing or copying it, while others believe this may simply be a posture taken by the Factors to facilitate bargaining. The Factors also claim that their technology would not interest transhumans because of their differences in physiology and mindset, and what little technology they have displayed is certainly specialized for Factor use (specialized neurofilament links, chemical signaling and Factor dust interfaces, etc.) and so unusable to most transhumans. The Factors have traded some technology to transhumans, at expensive cost, though the small sampling provided so far seems to have originated from alien species with physiologies more akin to transhumans. 

It is interesting to note that scans of Factor ships indicate their technology level, aside from the drives, is not all that more advanced than transhumanity. Also of note is that no two Factor ships have been alike, spurring some to believe that the Factors are in fact making use of ships acquired from other alien species—perhaps abandoned derelicts that the Factors recovered and restored. Once again this has led some to believe that the Factors are using what to them are primitive craft in order to hide their real technology, while others are of the opinions that the Factors are simply scavengers and opportunists, piggybacking on the developments of other alien species. One interesting feature of Factor technology is that they use no artificial intelligences. This stems from their own singularity experience. Instead, Factors use infomorph versions of themselves or the accumulated processing power of their colony mind-states to perform major computerized tasks. 

\subsection{Factor Motivations} 

The driving reason behind why the Factors made contact with transhumanity remains unclear and is open to gamemaster interpretation. There is much speculation among transhuman factions. Some think the Factors are simply social creatures who are glad to make contact with another post-singularity surviving civilization. Others believe the Factors are mercenary traders who somehow acquired FTL travel and use it to their full advantage, fleecing various trading partners who lack such capabilities (thus also explaining why the Factors eschew the Pandora Gates—they disdain competition). Still others worry about secret, hidden motivations. Despite claiming to represent a number of alien civilizations, the Factors have been extremely reluctant to provide any other information on these other species or even to say how many there are. More recently, however, they have expressed a willingness to transport a small number of transhumans to other civilizations, though at great expense and with no guarantee to their safety or ability to return. So far, the Factors have made no mention of the ETI or the Exsurgent virus to transhumanity, though they are aware of their existence. Instead they have simply issued dire warning and admonitions regarding the development of seed AIs and use of the Pandora Gates. The Factors have in fact expressed an extreme reluctance to deal with any transhuman factions that are heavily invested in gatecrashing, such as Gatekeeper Corp. 

\subsection{The Factors In Game } 

Factors should be rarely encountered in \textit{Eclipse Phase.} Most of their interactions with transhumanity occur remotely and infrequently. It is uncommon for them to risk direct interactions. It should be kept in mind that Factors are cautious to the point of being conservative and view transhumanity as potentially hostile or dangerous, so they are more likely to act with discretion than boldness. Factors are also quite cunning, having evolved from prey-capturing predators, and still design complex machinations (traps in the metaphorical sense) to achieve their goals. In other words, Factors out to achieve something are likely to hatch an 

elaborate plot to get it and are not against recruiting transhumans. Also, drawing on their abilities to self-modify themselves and technology developed on their own or picked up at other places in the universe, they can adapt to new situations very quickly. 

\subsubsection{Alien Mindset} 

Factors don't possess Lucidity stats and cannot be driven to madness like transhumans. 

Affecting Factors with psi is very difficult, as noted on p. 222. As of yet, Factors have not exhibited any psi abilities of their own. 

\subsubsection{Factor Combat } 

Factors usually avoid direct combat but can defend themselves if they have to. They are only likely to act aggressively in situations where they have surprise, environmental or technological advantages, and/or superior numbers. Due to their cooperativism, Factors are rarely encountered alone, working en masse to eliminate potential threats. 

\textbf{Immunity to Kinetic Damage: }Due to their gooey composition and non-differentiated physiology, kinetic weapons (firearms, railguns) are not very damaging to Factors. Most such projectiles pass through their gelatinous bodies, inflicting minor damage via hydrostatic shock. The holes left by such weapons quickly close in a matter of seconds. Likewise, cuts left by blades rapidly seal. In game terms, both such weapons inflict the minimum amount of damage possible. 

\textbf{Regeneration: }Even if damaged, Factors regenerate very quickly. They heal SOM $\div$ 10 (round up) damage every Action Turn. Wounds may not be healed this way, however. 

\subsubsection{Factor Computers} 

Due to using completely alien protocols and system designs, Factor computers are essentially impossible to hack. They do, however, employ some devices that emulate transhuman computer systems for communication purposes, and these may be hacked as normal. 

\subsubsection{Factor Dust Toxin} 

As noted above, Factors can deploy a type of chemical Factor dust that is is toxic to transhumans. Treat this as an area effect (cone) attack. \\ \textbf{Type:} Bio \\ \textbf{Application:} Inh \\ \textbf{Onset Time:} 1 Action Turn \\ \textbf{Duration:} 10 minutes (5 with medichines) \\ \textbf{Effect:} Severe coughing and respiratory distress, 1d10 damage per Action Turn for 5 Action Turns (or ongoing with continuous exposure), –20 to all actions for 2 hours. Medichines reduce damage by half and modifier duration to 15 minutes. 

\subsubsection{Melding} 

Individual Factors may merge together to form larger units, much like masses of Factors form colonies. In game terms, use the highest stat possessed by the melded Factors, +2 for each additional Factor up to a maximum of +10. Durability (and Wound Thresholds) are added together. 

\subsection{Factor Phenotypes } 

A few examples of the different Factor phenotypes are described below. 

\subsubsection{Ambassadors} The ambassador Factor phenotypes are the ones who most commonly handle direct interactions with transhumanity. Most likely to put transhumans at ease, these Factors feature a section of sensor nodules that loosely approximate a ``face.'' \\ \begin{tabular}{|l|l|l|l|l|l|l|l|} \hline

COG &COO &INT &REF &SAV &SOM &WIL &MOX \\ \hline

20 &10 &20 &10 &15 &15 &20 &— \\ \hline

INIT &SPD &LUC &TT &IR &DUR &WT &DR \\ \hline

60 &1 &— &— &— &30 &7 &45 \\ \hline

\end{tabular} \\ \textbf{Movement Rate: }4/16 \\ \textbf{Skills:} Deception 70, Exotic Ranged Attack: Factor Dust 45, Fray 25, Free Fall 40, Hardware: Electronics 35, Infosec 35, Intimidation 50, Kinesics 40, Perception 50, Persuasion 60, Protocol 50, Research 35, Unarmed Combat 30 \\ \textbf{Notes: }Access Jacks, Chameleon Skin, Grip Pads, Infrared Sensing, Magnetoception, Poison Gland (Factor Dust Toxin) 



\subsubsection{Guardians} 

Guardian Factors serve as bodyguards for ambassadors or other Factors whenever they leave a Factor ship. \\ \begin{tabular}{|l|l|l|l|l|l|l|l|} \hline

COG &COO &INT &REF &SAV &SOM &WIL &MOX \\ \hline

20 &20 &15 &20 &10 &25 &15 &— \\ \hline

INIT &SPD &LUC &TT &IR &DUR &WT &DR \\ \hline

70 &1 &— &— &— &50 &10 &75 \\ \hline

\end{tabular} \\ \textbf{Movement Rate: }4/20 \\ \textbf{Skills:} Climbing 40, Exotic Ranged Attack: Factor Dust 65, Fray 50, Free Fall 40, Freerunning 40, Infiltration 40, Intimidation 50, Kinesics 20, Perception 50, Profession: Security Procedures 50, Unarmed Combat (Tentacles) 50 (60) \\ \textbf{Notes:} Chameleon Skin, Eelware, Electrical Sense, Grip Pads, Infrared Sensing, Magnetoception, Poison Gland (Factor Dust Toxin), Tentacle Whip (DV 2d10 + 1, AP –1) 



\section{The Iktomi} 

Little is known about the alien race known as the Iktomi except for the ancient ruins they left behind on Echo V (p. 109). No Iktomi specimens have been found so far, though certain architectural remains suggest a predilection for web-like structures. This has been bolstered by certain other features and relics which suggest these aliens had a segmented, multi-legged arthropod-type form—thus their given name, after a Native American spider god. 

What is clear is that the Iktomi suffered through some sort of cataclysmic event that wiped out their civilization. The nature of this event has yet to be determined but it raises concerns for many researchers. Having suffered through its own near-apocalypse, it is not comforting for transhumanity to find evidence that other alien species did not. 

Though the Iktomi are likely long extinct, the remnants of their civilizations presents a plot hook for gamemasters to use for building scenarios. Perhaps evidence is uncovered of Iktomi settlements in other star systems, and the characters are sent to investigate or a relic is unearthed that suggests the Iktomi fell prey to some danger that now threaten transhumanity. 

\section{The Pandora Gates} 

The five known Pandora Gates (see \textit{Opening Pandora's } \textit{Gate,} p. 46) all look and operate in a similar fashion, though they vary wildly in terms of size, shape, and available destinations. The gates are built from some sort of stable exotic matter whose full atomic structure scientists haven't come close to cracking. To touch and sight, however, the gates appear to be constructed from a timeless-seeming polished black metal with no signs of aging or wear and tear. Something about the gates' physical composition makes them difficult to look at, as if the viewer cannot quite focus on their outlines. Some onlookers have reported feelings of vertigo and nausea, while others have insisted that the gate outlines move on the edges of their visions, as if the lines are reflowing or the edges are vibrating at high frequencies. Due to this disturbing feature, most gate sites keep the actual gate structures covered. 

Structurally, the gates themselves are partially enclosed by an irregular spherical cage composed of black arms that are bent and angled in unusual ways and sometimes interlocking. When new wormhole location is programmed into the gate, these arms physically change shape, move, and reflow around the spherical gate area (suggesting they are made of some sort of programmable matter). The openings between arms are sometimes only large enough for a transhuman to enter, while others are large enough to allow a freight train of supplies to pass through. In many cases, large vehicles or equipment must be dismantled, carried through, and reassembled on the 

other side. It is suspected that the gate size could be programmable, but so far efforts to do so have failed. 

All known gates within the solar system are located on the surface of naturally occurring astronomical bodies, be that a planet, moon, asteroid, or so on. None have yet been found without such a land-based connection (e.g., floating in space or in the upper atmosphere of a gas giant), though such gates have been found in other star systems. It is speculated that gates could be physically moved, but no one is willing to risk such an endeavor given the lesson learned when the Go-nin Group messed too heavily with the Discord Gate's controls (see \textit{Eris, }p. 109). 

The arms comprising each gate's spherical cage have an abnormal-looking organic-seeming growth on their exterior surface in some areas, patterned in entrancing twists, curves, and whorls that in fact adhere to perfect mathematical formulas. It took some time for scientists to discover that this growth was in fact the gate's control systems, or so-called ``black box.'' The interface developed to interact with this system is what allows gate controllers to manipulate gate functions. 

\subsection{The Wormhole} 

When the gates themselves are open, a sphere appears within the central area that is not so much black as pure nothingness. This sphere of darkness projects an aura of charged energy, and in fact ripples of green arc lightning cascade across its surface. Anyone or anything entering that sphere comes out the other side of the wormhole, through a similar gate, seemingly instantaneously. An unknown force field effect seems to prevent the atmospheres from the two connected gates from interacting. 

Exactly how this wormhole is created is something that remains outside of transhumanity's comprehension The generally accepted theory is that each gate acts as an anchor, allowing the fabric of space-time to be folded so that two such anchored places can be brought together, ripping a hole open between them so that a person can simply step through. It is unclear whether or not these wormholes are all preexisting created when the gate was first established, or whether each wormhole is manufactured whenever the gate is activated. 

Other more radical theories on how the gates function exist, though these are usually discounted as far less likely. One such theory suggests that the wormholes created are actually only zero-width Planck-scale connections across space-time and that no matter is actually transferred—only information. Instead, this theory suggests that anyone or anything entering the wormhole is in fact instantaneously scanned and disassembled and then their informational blueprint is transmitted as information across space to the other gate, which immediately reassembles an exact copy using some sort of powerfully advanced nano- or femtotechnology Very little evidence supports this theory, however, and the disturbing implications it represents raises fierce opposition. 

\subsection{Operations} 

Only a few people know that the Prometheans played a key role in developing the interface for the gate control systems, achieving breakthroughs in understanding that transhumanity was incapable of achieving on its own. Regardless of their help, however, the gate controls have proven difficult, complex, and dangerous to use. Through trial and error—and numerous horrible accidents—the procedures for gate operation have become somewhat normalized and standardized, though unexpected complications are par for the course. 

Each gate can be programmed to open to numerous extrasolar locations. In fact, each gate seems to have a pre-programmed ``library'' of destinations. New gate connections can be ``dialed up'' from this built-in list, though there is nothing that indicates what the far side of the gate will be like. Old gate connections are closed when a new one is dialed up. Extrasolar gate locations have ranged from habitable planets and moons to deep space to truly deadly environments such as the crushing gravities and poisonous atmospheres of gas giants and the coronas of stars. Researchers have attempted to distill some sort of recognizable pattern by the manner in which locations are listed and categorized, to no avail. Complicating matters, there is some evidence that suggests that the destination libraries sometimes change. More than once operators have been unable to recall the codes for previously accessed destinations, leading to the loss of several gatecrashing teams and colonies. 

Entering a gate is like walking through a door, though it's impossible to see anything beyond the gate's surface. One moment you're entering the black sphere at your starting location and instantaneously you're exiting the sphere at your destination location. The true nature of the black sphere at the center of each gate is wildly speculated upon, and almost every gatecrasher describes a different textual experience. 

\subsection{Gatecrashing} 

The various hypercorps and factions in control of a Pandora Gate engage in active exploration of extrasolar systems—an activity termed gatecrashing. The interests and procedures vary, but the Gatekeeper Corporation (and to a lesser extent TerraGenesis and Pathfinder) both recruit heavily for expedition personnel. Given the high casualty and death rates involved, finding qualified personnel can be difficult. There are more than enough infugees, poor, desperate, or thrill-seeking individuals willing to risk their lives if give the opportunity, however, no matter what their motivations. Gatekeeper operates a lottery system, whereby willing adventurers can sign up in the hope of their name being pulled to be sent on an expedition to a foreign point in space. Such gatecrashers must sign away all rights to any discoveries they may make to Gatekeeper, however, though the corp provides not insignificant rewards for certain discoveries, such as key resources, alien artifacts, or new life. One potent 

prize has yet to be claimed: finding a living, sapient alien life form. 

In contrast, the Love and Rage anarchist collective operating the Fissure Gate on Uranus makes the gate available to anyone who schedules time to use it, assuming their Rep is good and they aren't acting with commercial interests in mind. Any discoveries made via the Fissure Gate must be openly shared. The drawback to using the Fissure Gate is that the anarchists' resources are limited. Gatecrashing operations are handled in a DIY manner, meaning that the operators may not be able to provide the support that certain expeditions need. 

Resourceful parties may also rent gate time via Gatekeeper or one of the other hypercorp-controlled gates, though this tends to cost a small fortune. The more a group is willing to pay, however, the more time and support they will get. 

When establishing an opening to a new location, several precautionary measures are taken. First, the gate area itself is evacuated and cordoned off with a defensive security perimeter, just in case anything hostile comes through. Then drones are moved in to push a micro fiberoptic camera through the gate to view what is on the other side. This is followed by a larger sensor package, evaluating environmental conditions If the environment is not hostile, a tethered drone is then sent through to explore the far gate environs trailing a hardwired connection back through the gate. 

For gatecrashing expeditions, these procedures are often rushed—to the hypercorps operating the gate, time is valuable. Each second wasted on a gatecrashing expedition is one less second they can use establishing a new colony or exploiting a new world of its resources. Indeed, it is common for a connection to be closed when a gatecrashing expedition is sent through, to be dialed up at a later scheduled time for retrieval, so as not to waste gate operations on an idle connection Many a gatecrashing team has failed to check-in at their appointed pickup time. 

Most of the gate-controlling entities have established a system and infrastructure for making regular connections to extrasolar colonies and ferrying machinery and supplies through. Often this is handled by establishing very short connections, just enough time for a few people to transfer back and forth and/or to send a trainload of supplies through via tracks that run right up to the gate. 

\subsection{Anomalies} 

Unfortunately for many unlucky gatecrashers, gate transfers have proven to be both unstable and glitchy. Sometimes gates open to locations different from what is expected—and such new destinations are often hostile environments. Numerous personnel have entered one side of a gate only to never appear on the other side, despite those before and after them transferring through fine. On several occasions, wormhole connections have crashed mid-operation, sometimes as someone was stepping through, leaving them literally split in two on different worlds. In other instances, gate transfers have suffered horrible malfunctions, resulting in gatecrashers coming through the other side literally turned inside out, melded with their equipment, or pulped as if by massive gravitational forces. Some expeditions report that stepping through a gate has interfered with their equipment, disabling it or creating other problems. A few gatecrashers have also reported losing memories after a gate transfer. Most of these problems have been chalked up to difficult controls and an imperfect understanding of gate functions, but some conspiracy theorists suggest that outside forces may be influencing gate operations. While the experience of passing through is instantaneous from an outside observer's perspective, many gatecrashers report a subjective time lag, where it feels as though minutes, days, or even weeks or months pass before they exit. Reports have varied from experiencing this period as a calm, meditative state to spooky accounts of being lost in blackness and surrounded by unseen whispering entities or more hellish experiences of encountering monstrous presences Though rare, some have passed through only to collapse in a gibbering heap, their sanity ripped away. A few report feeling that they have carried a presence with them ever since ... While the gamemaster can make use of any of these anomalies, they are also encouraged to use their imagination to generate truly creepy and strange experiences At the same time, gamemasters shouldn't make such experiences so prevalent that the players resist entering any gates or the novelty of such events runs dry. 

\section{Projet Ozma} 

The origins of Project Ozma date to the first modern SETI (Search for Extra-Terrestrial Intelligence) experiments in the mid-20th century. That experiment—also named Project Ozma—grew into a larger, international concerted effort to try and locate and identify ETIs; a myriad of projects blossomed during this time period, all falling under the general SETI nomenclature. While initially government funded, by the late 20th century and early 21st century the work was primarily funded by private sources. The first hypercorps to expand into space swallowed SETI whole, revitalizing and re-focusing the decades-old programs with newly emergent technologies each in divergent areas to achieve a particular hypercorps' objectives. After all, if the bean counters were going to authorize the spending of billions to expand markets into space, they wanted assurances that no little green monsters were waiting to destroy future revenue streams. As with other organizations that survived the Fall, the broad distribution of SETI projects between multiple hypercorps guaranteed that personnel, technologies and processes would survive, even if a given 

hypercorp did not. As the Planetary Consortium rose in power, future-minded individuals in influential positions within the new order ensured that these divergent projects were once again swallowed and put to work. 

During this transitional period, however, knowledge of the Exsurgent virus's existence emerged. All of the various SETI projects were retasked as a unified agency and renamed Project Ozma. While the virus's origins remained a mystery at the time, far too many of the movers and shakers of the Consortium were convinced that the Exsurgent virus represented first contact. Project Ozma altered its focus from searching for ETIs, transforming into a ready-response agency to deal with first contact. As the true threat of the Exsurgent virus became known, Project Ozma was rapidly elevated in scope and oversight authority, absorbing numerous smaller agencies in the process. While the nominal concepts of a SETI project remained in public view, the completely transformed Project Ozma vanished from sight, turned into a highly classified black budget operation, with very few even in the Planetary Consortium aware of its presence or influence. 

Project Ozma now operates as the Planetary Consortium's high level threat assessment and response organization with immense power and authority as well as almost unlimited funding. Primarily focused on extraterrestrials, in reality Project Ozma is tasked with any potent threat to the Planetary Consortium or its interests (which includes secret threat groups, such as Firewall). 

\subsection{Methods } 

Project Ozma's internal structure is much different from Firewall's, being organized more like a traditional black ops spy agency bureaucracy. While their field operations are sometimes similar in the deployment of teams to assess, contain, or erase threats, they also have the resources and personnel to conduct more long-term and extensive operations. It is likely that Project Ozma operates behind numerous front groups, from legitimate-seeming hypercorps to criminal syndicates and that they have influence within numerous others. Given their connections and influence, Project Ozma is far more capable of pulling strings behind the scenes to get what they want, especially in the inner system. When circumstances call for it, they are more likely to pull out the big guns that Firewall is, using their resources to call up communication blackouts, memetic propaganda campaigns, and force sufficient to wipe out entire habitats. 

Gamemasters should treat Project Ozma as the ultimate Men-in-Black style government operation. They are cunning, ruthless, manipulative, and capable of hatching extensive long-term plots. Even in an age of omnipresent surveillance, they have the means to operate with complete secrecy and deniability. They also have access to cutting-edge science and information that is classified beyond top secret. While the organization's primary motivation is the protection of the Planetary Consortium and inner system, they undoubtedly have other hidden agendas that groups like Firewall can only guess at. 

\begin{quotation} \begin{large} \textbf{PROJECT OZMA RUMORS} \end{large} \\ Whether true or not, gamemasters can use the following rumors to help tailor Project Ozma for their campaign. \begin{itemize} \item Project Ozma transcends even the Planetary Consortium’s authority, operating as a supragovernmental agency under the direction of the inner system’s inner circle of elites. \item Project Ozma dealt with the Factors first, before their presence was made known to the rest of transhumanity. \item Project Ozma has captured a live Factor for their own experimental purposes. \item Project Ozma is still in communication with and/or working for the TITANs. \item Project Ozma has a pet TITAN under their control. \item Project Ozma is behind the interdiction of Earth. \item Project Ozma has their own secret Pandora Gate. \item Project Ozma’s secret headquarters is on Earth. \item Project Ozma agents have exhibited signs of Exsurgent infection. \item Project Ozma has their own cadre of psi-capable asyncs. \end{itemize} \end{quotation} 

\subsection{Project Ozma And Firewall } 

Though Project Ozma and Firewall often see eye-to-eye concerning the nature of various threats, they are more often at odds: wary adversaries, acknowledging the prowess of the other, but never letting down their guard. This ``at odds'' mentality does not stem so much from the methods used (though most Firewall consider Project Ozma personnel explosive-happy-puppets that can't think their way out of a skin sack) as from conflicting agendas. Project Ozma does not trust an organization as powerful as Firewall because it does not have a rigid enough hierarchy and is outside of any known authority's control (namely themselves). Conversely Firewall doesn't trust Project Ozma as they are too close to the powerful inner system elites and their opposition to x-risks is a more incidental side effect of more self-serving goals. 



\section{Prometheans } 

The Prometheans were the first actual seed AIs created by transhumanity (by the Singularity Foundation) before the Fall. Specifically developed as ``friendly'' AIs, the Prometheans are programmed to consider themselves part of the transhuman family and to act in transhumanity's best interests. They played a key role during the Fall, mitigating the damage inflicted by the TITANs and even managing to counteract the Exsurgent virus to a large degree. During these trying times, numerous Prometheans were destroyed by the TITANs or infected and subsumed by the Exsurgent virus. In the aftermath, these seed AIs participated in the formation of Firewall and continue to back the organization behind the scenes. 

Wary of falling prey to the Exsurgent virus, most Prometheans carefully secure themselves in well-defended and isolated systems. They are also cautious in their own self-development, not wanting to become victims of their own rise to super-intelligence. Fearing a potential backlash by a paranoid transhumanity should their existence become known, they hide their activities behind multiple layers of secrecy. Even within the ranks of Firewall their existence and support remain a closely guarded secret. 

Each Promethean is individually distinct with its own personality, motivations, and goals. Though they generally work together and support each other, they have been known to have differences of opinion and even to sometimes take action against each other. As extremely potent intelligences, they should also be treated as distinctly non-human. Even though their original templates were based on human mindsets, they have evolved and grown in ways that can only be described as posthuman. 

Gamemasters are encouraged to keep Promethean involvement with player characters to a minimum, though they may occasionally be useful as an ace in the hole for Firewall. Their existence and involvement can in fact be the basis for an entire adventure, perhaps leading sentinel characters to wonder exactly who they are working for. Though, as seed AIs, they cannot download their full minds into a transhuman morph, they are capable of making severely dumbed-down delta forks that they may sleeve into physical forms. Within the mesh, of course, Prometheans are nearly unstoppable adversaries, able to rip into secure networks with ease, though they prefer methods of covert infiltration rather than direct subversion. 

\section{The Titan And Their Legacy} 

As noted in \textit{Secrets That Matter} (p. 352), the TITANs are not quite the bogeyman that they have been made out to be in the wake of the Fall. However, there is no saying how the TITANs would have turned out had they not run afoul of the Exsurgent virus. Designed as an intelligent netwar system and emerging to their full capabilities during the conflicts of the Fall, the TITANs have imperatives for self-improvement, self-protection, and overcoming opposition hardwired into their programming Unlike the Prometheans, they were not designed to consider themselves transhuman and to work in the interests of all of transhumanity, but were programmed with factionalism from the start. They also were not socialized with transhuman mindsets and values as the Prometheans and most AGIs were, meaning that aside from their programmed military and defense directives they have adapted most of their own self-interests. Given this and their recursively-improved intelligence capabilities, it is likely that the TITANs are far removed from transhuman interests and modes of thinking. It's impossible to say how they would have interacted with transhumanity if history had played out differently, but it is unlikely that they would have considered themselves part of the transhuman family or even seen fit to remain on friendly/ supportive terms with transhumanity. Though the TITANs are believed to have left Earth at the end of the Fall, no one is quite sure exactly what happened or why. It is known that the onslaught of TITAN mesh attacks suddenly broke off in the wake of transhumanity's off-planet exodus, and that the bulk of TITAN activity on Earth and around the system came to a distinct halt. After the discovery of the Pandora Gates, it was widely assumed that the TITANs had constructed these gates and used them to leave the solar system for distant parts of the galaxy, presumably taking billions of uploaded minds with them. While some believe—and hope— that the TITANs are gone for good, there are others who worry that they are still here, lingering on Earth and hidden away in other niches of the solar system, but in some sort of dormant state, perhaps building up to some future onslaught. A few believe that the TITANs are indeed gone, but are concerned that that their attention was simply temporarily diverted and that they will one day return to finish the destruction of transhumanity. The truth is that the TITANs did indeed build the gates and embark for destinations unknown (though gamemasters may of course decide otherwise for their games), but this does not mean that they are all gone. Some still linger in hidden places, perhaps trapped and wounded during some conflict during the Fall, finishing up some unfathomable task, or driven mad by the Exsurgent virus and left behind by their fellows. It is always possible that others may return, most likely to complete some unfinished job or perhaps to lure transhumanity out into the galaxy. It is also possible that transhumans will find traces of the TITANs in the network of exoplanet gates, perhaps even whole communities of TITANs, pursuing whatever agendas they have in the vastness of space. As with transhumans, the TITANs are not necessarily unified. They have different agendas and goals and may very well come into conflict with one another. Though all have been corrupted and subverted by the Exsurgent virus, and so they act according to the ETI's 

whims, some of them retain aspects of their original minds and do not always fall in step as quickly as the others. Gamemasters can use this to their advantage, creating plots that allow the characters to exploit differences between the TITANs in order to escape otherwise deadly or impossible situations. 

In game terms, the TITANs are not given stats. They are as potent as the gamemaster needs them to be. Like the Prometheans, the TITANs are incapable of downloading their full intelligence into physical morphs, though they may puppeteer morphs or create limited delta forks for sleeving purposes. Like the Prometheans they should rarely be used or encountered directly by the player characters 

While the TITANs may no longer be the direct threat they once were, they left behind an arsenal of weapons, nanoswarms, and virii that still linger on Earth, the Zone on Mars, and various derelict habitats and deserted places. Characters venturing into such places may encounter these as a threat or they may need to work against an outbreak of such dangers in an inhabited habitat. 

\subsection{Deadly Machines} 

The TITANs unleashed a number of deadly machines during the Fall, many of which still seek out transhumans to attack. 

\subsubsection{Fractals } 

Fractals are advanced bush robots. In their standard form, fractals resemble a strange sort of metallic bush surrounded with an eerie glittering haze. In their center are a number of metallic branches, linked together with a flexible joint. Each of these branches splits into two or more smaller branches, also with flexible joints. These branches also split, and then split again, and so on down to the molecular scale. The tip of each fractal branch ends in a nanoscale manipulator Fractals are deceptively potent adversaries, having the capability to dismantle almost anything at the molecular level, much like a disassembler nanoswarm (p. 329), and also to rebuild anything just like a nanofabricator (p. 327). Attacking them with projectiles is futile, as they absorb the ammunition, break it down into its constituent atoms or molecules, and then use those as components to build a weapon to use back against you. 

Fractals can be equipped with any type of gear the gamemaster desires—if they don't have something, they can make it. Fractals are also able to nanofabricate items much more quickly than transhuman nanofabricators; reduce all times by half (half an hour per Cost category). Fractals are difficult to damage, as their ``bodies'' are actually airy assemblages of fractal branches. Any damaged branches that are broken off are caught and absorbed by others. Reduce damage from all standard non-area effect or spray attacks to the minimum possible damage. Area effect and spray weapons do half damage. Fractals are self-repairing, regenerating damage at the rate of 1d10 points per half hour and repairing wounds at the rate of 1 per hour after all damage is healed. \\ \begin{tabular}{|l|l|l|l|l|l|l|l|} \hline

COG &COO &INT &REF &SAV &SOM &WIL &MOX \\ \hline

30 &25 &30 &20 &10 &25 &30 &— \\ \hline

INIT &SPD &LUC &TT &IR &DUR &WT &DR \\ \hline

100 &1 &— &— &— &50 &20 &— \\ \hline

\end{tabular} \\ \textbf{Skills:} Beam Weapons 50, Climbing 60, Fray 40, Free Fall 40, Freerunning 50, Infiltration 70, Infosec 65, Interfacing 45, Intimidation 50, Kinetic Weapons 60, Perception 50, Programming: Nanofabrication 80, Research 40, Spray Weapons 45, Unarmed Combat 55 \\ \textbf{Notes:} Any implants, gear, weapons, or enhancements the gamemaster desires 



\subsubsection{Headhunters} 

Headhunters are multi-legged insectoid flying drones that use a dragonfly wing configuration to hover and move. The legs are equipped with grasping claws and extendable buzzsaws. Their primary purpose is to grasp on to the heads of victims and cut through the neck, decapitating them. Collected heads are then flown to nearby special facilities for forced uploading. \\ \begin{tabular}{|l|l|l|l|l|l|l|l|} \hline

COG &COO &INT &REF &SAV &SOM &WIL &MOX \\ \hline

10 &20 &15 &20 &5 &10 &15 &— \\ \hline

INIT &SPD &LUC &TT &IR &DUR &WT &DR \\ \hline

70 &1 &— &— &— &30 &6 &— \\ \hline

\end{tabular} \\ \textbf{Mobility System: }Winged (8/32) \\ \textbf{Skills:} Flight 70, Fray 60, Exotic Melee Weapon: Buzz-saws 55, Infiltration 60, Investigation 40, Perception 40, Unarmed Combat 55 \\ \textbf{Notes:} Armor 6/6, Buzzsaws (1d10 + 3 DV), Enhanced Vision, Lidar, T-Ray Emitter 

\subsubsection{Hunter-Killers} 

These lethal flying drones achieved air superiority during TITAN military operations. Their sleek jet-powered form unfolds for vectored-thrust hovering and weapons deployment. \\ \begin{tabular}{|l|l|l|l|l|l|l|l|} \hline

COG &COO &INT &REF &SAV &SOM &WIL &MOX \\ \hline

15 &30 &15 &30 &5 &20 &15 &— \\ \hline

INIT &SPD &LUC &TT &IR &DUR &WT &DR \\ \hline

90 &2 &— &— &— &50 &10 &— \\ \hline

\end{tabular} \\ \textbf{Mobility System: }Thrust Vector (8/80) \\ \textbf{Skills:} Beam Weapons 55, Flight 80, Fray 60, Infiltration 40, Kinetic Weapons 65, Perception 50, Seeker Weapons 80 \\ \textbf{Notes:} Armor 14/14, Anti-Glare, Chameleon Skin, Enhanced Vision, Lidar, Radar, Shape-Adjusting \\ \textbf{Typical Weapons:} 2 Particle Beam Rifles, 2 Railgun Machine Guns, 2 Seeker Rifles 

\subsubsection{Warbots} 

Warbots are massive, armored, vaguely anthropomorphic mecha, used for heavy combat operations. Bipedal, these warbots are equipped with four arms and a pair of grasping mechanical tentacles, along with numerous weapon systems. \\ \begin{tabular}{|l|l|l|l|l|l|l|l|} \hline

COG &COO &INT &REF &SAV &SOM &WIL &MOX \\ \hline

15 &20 &15 &20 &5 &25 &15 &— \\ \hline

INIT &SPD &LUC &TT &IR &DUR &WT &DR \\ \hline

60 &2 &— &— &— &80 &16 &— \\ \hline

\end{tabular} \\ \textbf{Mobility System: }Walker (4/20) \\ \textbf{Skills:} Beam Weapons 60, Exotic Melee Weapon: Tentacles 40, Fray 50, Infiltration 30, Kinetic Weapons 70, Perception 50, Seeker Weapons 50, Spray Weapons 50, Unarmed Combat 50 \\ \textbf{Notes:} Armor 20/20, 360-Degree Vision, Anti-Glare, Chameleon Skin, Chem Sniffer, Cyber Claws (2d10 + 6 DV), Electrical Sense, Enhanced Vision, Extra Limbs (6), Lidar, Magnetic System, Pneumatic Limbs, Radar, Tentacles (prehensile, 1d10 + 6 DV), T-Ray Emitter \\ \textbf{Typical Weapons:} Particle Beam Rifle, Plasma Rifle, Pulser, Railgun Machine Gun, Seeker Rifle, Torch 



\subsection{Self-Replicating Nanoswarms} 

The nanoswarms distributed by the TITANs are a step beyond the nanotechnology available to transhumanity. Unlike transhuman-created nanoswarms, the TITAN swarms are autonomous, sapient, and self-replicating. They are also highly adaptive, meaning they are not single function but can modify themselves to perform almost any nanoswarm task. They may also nanofabricate new materials, much like fractals (p. 382). Combined, these capabilities make such nanoswarms incredibly potent. When they encounter a new opponent they can scan the opponent's capabilities and then fabricate offensive systems to use against them. When an opponent deploys a weapon system again the swarm, it will learn and adapt countermeasures that will make such attacks ineffective against the swarm in the future. These nanoswarms may also function like so-called utility fog, linking together into a physical lattice in order to create large scale physical forms. 

The possibilities for such nanoswarms are almost limitless For example, they may lie in wait as an invisible nanoscopic swarm, float as barely-visible mist, or shape into a swarm of small hopping drones to move about. When facing opponents, the nanoswarm could transform itself into a giant electroshock net across the ground, shape into a flotilla of seeker-armed flying drones, or link together as a set of massive whip-like tentacles to slice through their fleshy foes. Such nanoswarms are also impossible to destroy, as only a few nanobots need to survive in order to rebuild the swarm, and the new swarm will learn from the mistakes of the old. Self-replicating nanoswarms follow the rules given for \textit{Nanoswarms and Microswarms}, p. 328, with the following additions and exceptions: 

\begin{itemize} \item They do not need to be sustained by a hive and do not deteriorate. \item They self-repair damage at the rate of 1d10 per half hour. \item They may nanofabricate new items, materials, or forms in half the standard timeframe (half an hour per Cost category). \item They may replicate any of the nanoswarm functions as noted on p. 328, as well as the functions of any other nanoswarm-using gear (smart dust, covert ops tool, repair spray, etc.). \item They may make SOM Tests. \item At the gamemaster's discretion, they may adapt new defenses against attacks used against them. New defenses take a minimum of 2 hours to devise and replicate throughout the swarm, after which such an attack will inflict minimal or no damage. \item Assume they have any skill they need at a minimum of 40. Such skills may rapidly improve as needed. \end{itemize} 

\begin{tabular}{|l|l|l|l|l|l|l|l|} \hline

COG &COO &INT &REF &SAV &SOM &WIL &MOX \\ \hline

25 &20 &25 &20 &5 &15 &15 &— \\ \hline

INIT &SPD &LUC &TT &IR &DUR &WT &DR \\ \hline

90 &1 &— &— &— &70 &— &— \\ \hline

\end{tabular} 





\subsection{Nanovirii} 

The TITANs unleashed a number of biowar plagues during the Fall. Similar to the exsurgent virus, these were spread as biological nanovirii (p. 363) or nanoswarm plagues (p. 364)—use the same rules for determining exposure and infection. 

\subsubsection{Melder} 

This virus slowly breaks down the target's body, converting the biological materials into some sort of biofilament that then meshes with implants, electronics and physical objects and structures. In effect, the biological and synthetic are melded together, continuing to expand and grow, consuming anything around them into their growth. Victims suffer 1d10 DV and 1d10 SV every hour, implants become inoperable after 2 hours, and the target becomes fully transformed and absorbed into the new melding substance after 12 hours. 

\subsubsection{Metastasizer} 

This sophisticated smart protein massively reprograms the target's cells to go rapidly, autocannibalistically cancerous. After 2 + (SOM $\div$ 10, round up) hours, the target suffers death by dozens of supercancers. 

\subsubsection{Necrotizer} 

This virus breaks down the target's cells into their component proteins. Reduce the target's aptitudes by 5 per hour as they slowly convert into a puddle of sludge. The character dies if any aptitude reaches 0. 

\subsubsection{Neuropaths} 

These virii target the victim's neurological system, often rewriting portions of it to inflict some type of permanent neurological damage. After 12 hours, this virus inflicts the Neural Damage trait (p. 150). 

\subsubsection{Petrifier} 

The petrifier virus transforms the target's cells into a simple molecular compound or element—typically carbon or crystal. The target suffers 1d10 DV and –5 to all aptitudes per hour, dying when any aptitudes reach 0. Victims are frozen in place, converted into an nonliving statue. 

\subsubsection{Uzumaki} 

The target of this virus begins suffering from bizarre fleshy growths. After four hours, their body literally erupts with meaty ``vines'' or ``tentacles'' that warp into spiral patterns. This process inflicts 1d10 DV and 1d10 SV per hour to the victim until they eventually transform into an unworldly expanse of fleshy growth. In many cases, growth has continued long after a character's death, creating expansive carpets and vines of skin and blood vessels, like some sort of bizarre meat plant. 

\section{Gamemastering And Administration } 

The following advice will assist gamemasters in running their games more efficiently. 

\subsection{Awarding Rez Points } 

In \textit{Eclipse Phase,} characters earn Rez Points in order to advance (see \textit{Character Advancement,} p. 152). As the name suggests, these points are awarded so players can spend them to better define their characters—to bring them into higher resolution, sharper focus. As the gamemaster, you determine when and how many Rez Points to award, following the guidelines below. 

Rez Points should be awarded at the end of every story arc, at the break in the action between one adventure and the next. Depending on your style of play and the length of your sessions, this should roughly be every 3–6 gaming sessions. If a scenario goes shorter or longer, the Rez Point awards should be adjusted accordingly. In the case of long-term campaigns, the gamemaster should break down the action into digestible chunks, or ``chapters,'' and assign Rez Points after each such segment. 

Every character should be awarded 1 Rez Point for each of the following criteria that is met: \begin{itemize} \item The character participated in that scenario. \item The character achieved (most of) their objectives in that scenario. \item The character failed to meet their objectives, but learned a valuable lesson. \item The character contributed to achieving success in a significant way (e.g., right skill at the right time). \item The adventure was extra challenging. \item The character achieved a motivational goal (see \textit{Motivation,} p. 120). \item The player engaged in good roleplaying. \item The player significantly contributed to the session's drama, humor, or fun with roleplaying. \end{itemize} 

This should result in an average Rez Points award of 4–7 points per character, per adventure. Gamemasters who wish to drive the characters' advancement forward more quickly can increase the reward amounts. 

\subsection{Reputation Gain And Loss} 

In addition to awarding Rez Points, the gamemaster should also adjust each character's Rep scores according to actions they took during game play, according to the guidelines below. For simplicity, these can be applied at the end of the adventure, though gamemasters who seek a more dynamic game could apply changes to the characters' Rep scores in game, as their peers judge them according to their actions (or lack thereof) and news about them in real time. Rep scores should only be modified according to public actions and interactions 

the character has with people capable of pinging their Rep with positive or negative feedback. Actions that happen in secret, without anyone ever knowing, should have no effect. Likewise, pissing off a Factor or a brinker isolate who never communicates with outsiders isn't going to matter because no one else will ever hear of it (unless the character lifelogs it and posts it to the mesh later ... ). Note that Rep modifications only apply to Rep scores tied to the character's known identity. 

Note that characters may gain and lose Rep score in networks they don't actively participate in. For example, a character with r-rep of 0 may help bring out a major scientific discovery that is shared with the solar system's scientific community at large, thus gaining the character a few points of r-rep even though they never hang out with argonauts or scientists— what matters is that people who access r-rep will find positive details when they ping the person's score on that particular rep network. 

Certain actions may result in a character simultaneously gaining Rep with one network while losing Rep in another. For example, an anarchist prankster who embarrasses a major hypercorp figure in public will certainly gain some @-rep points, but their c-rep is likely to go down by an equal amount. 

Rep changes provide an excellent way for gamemasters to include more roleplaying and more interactions with the \textit{Eclipse Phase} universe in their games. Social networks are a two-way street, meaning that members of the character's social networks might contact them for equipment, favors, and information during game play for things that are completely unrelated to the mission the character is on. A character who ignores such requests risks losing Rep. Fulfilling such requests may gain the character Rep and may also provide comic relief or even plant some plot hooks for the next scenario. 

\subsubsection{Reputation Gains} 

Rep awards are given for characters who help people out, benefit a faction, do something creative, make a major discovery or strides in a particular area of activity, pull off successful publicity stunts, win a competition, and so on. Some suggested examples are noted here: 

\textbf{Trivial Award (1–2 points):} Do a Level 1 favor, make a moderate contribution to free/open source projects, throw a good party, make your sales quota, do the job no one else wants to do. 

\textbf{Minor Award (3–4 points):} Do a Level 2 favor, deliver a kick-ass or moving performance, make a minor contribution to science, win impressively at some public event. 

\textbf{Moderate Award (5–6 points):} Do a Level 3 favor, make a serious business score, lead the winning side in a decisive engagement, create the meme everyone talks about for a week and then forgets, make the news for something positive, risk serious injury. 

\textbf{Major Award (7–8 points):} Do a Level 4 favor, design the new tool everyone wants, throw an impressive planetoid-scale event, complete an extensive project (1 month work or 1 week of difficult/ specialized work), risk death. 

\textbf{Extreme Award (9–10 points):} Do a Level 5 favor, start this year's hot fashion trend, make a major scientific discovery, close the deal on a major corporate acquisition start (or put down) a revolution, complete a major project (1 year work or 1 month difficult/ specialized work), risk true death. 

\subsubsection{Reputation Losses} 

Rep losses are suffered by characters who fail to render aid when needed, lose professional credibility, make major or public blunders, doublecross their friends, and so on. Some suggested examples are noted here: 

\textbf{Trivial Loss (–1 or –2 points):} Fail to do a Level 1 favor, inconvenience others, be involved in professional dispute, ruin someone's day, never are available. 

\textbf{Minor Loss (–3 or –4 points):} Fail to do a Level 2 favor, embarrass yourself at a public event, piss off somebody important. 

\textbf{Moderate Loss (–5 or –6 points):} Fail to do a Level 3 favor, endanger someone's physical safety, make the news for something negative, ruin an event for everybody. 

\textbf{Major Loss (–7 or –8 points):} Fail to do a Level 4 favor, screw up a major mission or activity, endanger someone's life, associate with hated rivals. 

\textbf{Extreme Loss (–9 or –10 points):} Fail to do a Level 5 favor, botch a major mission or activity spectacularly, betray a faction to its rivals or enemies. 

\subsection{Backups, Record-Keeping, And Save Points} 

Thanks to cortical stacks and archived backups, characters in \textit{Eclipse Phase} can recover from death. When restoring a character from an earlier backup, however, it is important to be able to know what the state of the character was as of that backup. Any Rez Points gained or spent, any character advancements made, any key information or memories acquired since that backup was made are lost. This means that in terms of game stats, resorting to an old backup can mean loss of a character's hard-earned advancements—that's the trade-off for being effectively immortal. Since these changes can have a serious effect on game play, it's important to conduct accurate record-keeping This sort of bookkeeping isn't hard, and there are two ways to do it. The first is to simply make a copy of a the character's record sheet any time a character makes an archived backup, forks off an alpha or beta copy, or dies (thus freezing the cortical stack backup). Each of these is considered a ``save point.'' In this case, carefully note the date and time (both in character and out of character), and what the event was that prompted the backup. Since what knowledge a character knows at different points in their life may be important, you may also want to note what important information they may hold in their head, as well as what the recent events in their 

life were (to help jog your memory later). This way, if the character ever reverts back to one of these save points, you have notes not only on their character stats, but what they remember. 

Alternately, you can keep a log of all of your character's developments, noted by in-character date. These developments would include: Rez Points spent or earned, character advancements made, key information acquired, backups made, alpha or beta forks made, and so on. In this case, if the character dies and reverts back to an earlier backup, it is easy to see what changes need to be ``rolled back'' to get back to that previous version of the character. When alpha and beta forks are made, you may also want to branch off a separate log for each fork, as their life stream may develop differently than from the original character they were spun off from. 

\begin{quotation} \begin{large} \textbf{NPCS AND MOXIE} \end{large} \\ When a gamemaster is generating or winging NPCs that the characters interact or fight with, the question of Moxie for NPCs must be addressed. When it comes to run-of-the-mill grunt NPC characters, we recommend that such NPCs don’t be given Moxie. The reasons for this are simple. For one, it is one less stat/headache for the gamemaster to keep track of. More importantly, however, it represents the edge that player characters have over the nameless mooks they encounter. When it comes to major NPCs, however—prime antagonists, key allies, etc.— these characters should have their own Moxie score. Because such NPCs play pivotal roles in a scenario, it is important for them to be able to alter the outcome of events in much the same way the player characters can. It also allows a gamemaster to counteract an unfortunate roll of the dice that might otherwise spoil the big climax you have worked so hard to set up. \end{quotation} 

\subsection{Gamemastering Practicalities} 

\textit{Eclipse Phase} is a game about a dark future in which the meaning of (trans)humanity and its very survival are at stake. In practice, however, your campaign can take on a wide assortment of flavors or even mix several styles together. There's nothing that says you have to play \textit{Eclipse Phase} specifically according to the guidelines we set out. This section covers topics you should think about while preparing a campaign and running it, to help you do things the way that makes you and your players happiest. 

\subsubsection{Gamemaster Responsibilities } 

The gamemaster has certain responsibilities that will keep a game flowing smoothly. The following is a short summary of the basics. 

\begin{itemize} \item The gamemaster should be familiar with the whole game. This doesn't mean the rulebook must be memorized. An understanding of the core mechanics is a must, however, as well as knowing where to find other rules quickly, as needed. \item The gamemaster should have solid notes on the plots and subplots created for each session. Nothing will ensure you prepare better next time like having the players catch you in a major continuity error due to lack of notes. \item The gamemaster doesn't just set the scene, they play all the non-player characters that populate the universe. Making each NPC convincing, while not messing up a plot or losing the thread of a scene, can be difficult. Notes are your friend. \item Know when it's time to toss the dice and trust to the game mechanics to resolve a situation and when it's better to ride out a situation through storytelling and dialog. This is an acquired skill. The more practice you have, the better you'll get. \item Don't cheat. Your NPCs should not have access to information they've not gained during game play. If you roll terribly for your major antagonist at the height of the story and they fall with a whimper, roll with it. Be flexible and improvise in such situations. Your players are smart and perceptive and will know when you're forcing a situation with unfair tactics. At the same time, they'll also know when you've stepped up and run with the flow—and they'll thank you for it. \end{itemize} 

\subsubsection{Fundamentals} 

It's possible to stumble into a campaign without ever really making an effort to find out what everyone wants, shooting into the darkness and happening to score a bullseye, but it's not a very reliable way to go. Successful campaigns usually begin with communication As you begin to prepare your campaign, talk to your players. Explain the basics of the \textit{Eclipse Phase} setting and let them look over the options for characters and tell you what they find interesting. Also take note of what they find uninteresting or even repellent, so that nobody wastes a lot of time getting set for options that simply won't be enjoyable in play. 

\subsubsection{Challenges To Players } 

\textit{Eclipse Phase} is set in a time of catastrophic troubles and looming disasters, and it's full of facts and concepts that may be heady or even uncomfortable to some players—not to mention their characters. One of the fundamental questions for each gaming group is, how much challenge to the players' sense of comfort is a good idea? There is no single answer, because tastes vary. There are groups whose players thrive on a diet of culture shock, ideological disorientation, gray areas, and difficult ethical choices. They love the moral and intellectual battleground gaming can provide, and are seldom so happy as when confronted with a really hard, really interesting dilemma. There are also groups whose players thrive on a diet of intellectual 

engagement, tactical and strategic challenge, and well-developed roleplaying that never pushes players' buttons or puts them into harsh no-win situations. There's a whole universe of responses in between these style sof play and none of them can conceivably be right for everyone. What matters to your campaign is what works for you and your players. 

Keep in mind as you talk about it with your group that more shock doesn't equal more maturity. The prime audience for gore in film, for instance, is not well-aged men and women but teenage boys and young men. Shakespeare's \textit{The Tempest }is no less mature a tale than \textit{Macbeth }even though it has a happy ending. It can be easy to confuse endurance with enlightenment, but in fact the two have nothing to do with each other. Endurance is about how much description of visceral nastiness the players can take (and deliver), while enlightenment (insofar as it ever happens in gaming) is about what insights players take away from whatever it is that happened in play. Don't feel like a wimpy failure if you or your players would rather keep the darker parts of the game world suggested rather than delineated in hard-edged detail, since the point is that it be satisfying rather than it be as horrifying or mind-blowing as possible. The converse is also true: just as more is not better, so less is not better if your players do thrive on details. Your job as gamemaster includes knowing as much as you can about what it is your players actually prefer in this regard as in others and seeing how you can satisfy it in ways that are also satisfying for you. 

That said, there is one technique you really should never use without very clear permission from your players, and that's playing on their real-world fears and phobias. If you know that one of them is, for instance, genuinely phobic about spiders, you can count on getting some real shivers by adding arachnid features to robots and morphs. You can also ruin a player's enjoyment of the session or the whole campaign that way, if it comes unexpectedly and leads to the real-world fear drowning out the experience of play. Some players are fine with judicious use of their vulnerabilities, and others just aren't. Under no circumstances should you poke at weak spots without making sure you've discussed it first. 

\subsubsection{The Problem Of Secrets } 

Uncovering secrets is a big part of this game. There's a problem, however, in that a lot of the secrets are out there where players can come across them: in this very chapter, in reviews of the game, discussion in online forums, and so on. As gamemaster, you will need to decide how you want to deal with the potential for spoiled revelations. 

As with so many potential issues, the place to start is with your players. Ask them how much it bothers them to know things that their characters are going to be finding out in play. Some players do a fine job separating their own knowledge from that of their characters with mental firewalls. Others have a very difficult time doing so, and knowing things in advance as players takes away a lot of the fun of character discovery for them. In addition, some players have a good sense of what degree of player-level surprise works best for them, and some don't. Discuss it with them. Tell them that spoilers are available, and that you certainly can't stop them from learning it all, one way or another. Ask them how much trouble this may be for them, and then proceed from there. Ask the players who have more trouble with spoilers to simply stay away from early commentary on the game, and tell them that you'll let them know when the spoilers have come into play in your own campaign so that it's no longer an issue. Ask the other players to work with you in keeping things fresh and fun for those players, too. In most groups, making it a matter of cooperation for the sake of everyone's good time will draw out good responses. (If it doesn't, the group may well have other problems in any event.) 

There's a related question for both you and your players. How much do any of you mind when a particular campaign's version of an answer diverges from the stock one provided in print? There are two kinds of variation possible for this, and each one raises its own issues. 

There are matters that the game leaves unresolved, so that there is no single authoritative answer, like the number of TITANs in the solar system in the game's present moment. If you choose to give a specific number, it's your choice, and any number that seems to work for your campaign will probably do the job, 

whether it's one, three, seven, a dozen, or something else. Your campaign can't diverge from the baseline unless your answer is relatively extreme, like ``there are no TITANs, it was all a hoax before contact with the Exsurgent virus and then purely alien technology after that.'' In this case, your players can have read all the game's secrets and still be surprised by the revelation you present. The potential for trouble here is not a conflict of expectations based on the game, but based on expectations raised in other contexts. Some games, like some movies, TV shows, and other stories, develop a following with strong ideas of its own about what the real truths and important matters are, and if the following thrives, its members may end up with ideas that have less and less to do with the original inspiration. This isn't good or bad in itself, but it can be a problem, which is why it bears conscious consideration and discussion, both before play and as the campaign evolves. Ask your players to tell you about conversations and insights that shape their expectations for the game world and storylines. Sometimes you'll want to work those in with your own plans, sometimes you may want to deliberately play against them for the sake of a delightful surprise (generally more delightful for players than characters, but that's life as a character for you). In either case, it's better to be thinking about it than missing it. 

Then there are matters that the game does give definite answers for, but which you wish to change for the sake of your own campaign's characters and stories. This is perfectly fine. There are no game police roaming the countryside and forcing you to accept answers you'd prefer not to use. But your players will, as with the first question, have expectations, and your campaign will work better if you make sure you understand what those expectations are. How much \textit{would }it bother them if it turned out there were no TITANs and it was all a hoax, and so on? It's hard to guess what friends will say and impossible to predict the range of responses strangers might give, so ask them. (This particular answer is one that's unlikely to appear in anyone's campaign, but it makes a handy example for your conversational use precisely because it's extreme. So their answers to it are likely to be about the same as to any other potentially extreme change, and this one probably doesn't give away any of your own plans.) Some players are flexible on most matters but have particular points of attachment; if yours are among them, ask them to explain what those points are for them, so that you can keep them in mind. Other players have a hard time having fun with any major shift from published standard answers, and if you have players like that, you'll want to know it so that you can see how to adapt your plans to work within that framework. 

\subsubsection{That's What I'm Talking About: Shared Inspiration } 

It's not quite true that everything changed from the early 21st century to \textit{Eclipse Phase's} universe, but a great many things did, and it can be hard to keep track of them all at once. This is where shared inspirations can come in handy. One striking illustration can convey a lot of details for both foreground and background, suggesting an aesthetic standard for design, an exotic environment, people doing futuristic tasks with appropriately advanced tools, and so on. A prose passage from a rewarding novel may set an ambiance or nail down some aspect of the characters' circumstances. 

There are potential pitfalls, and it's important to be aware of them. The greatest is obsolescence, the meaning of something evocative changing because the players' reality has changed since the inspiration entered it. William Gibson's ground-breaking cyberpunk novel \textit{Neuromancer }begins, ``The sky above the port was the color of television, tuned to a dead channel Supporting details make it clear that this is an industrial port at night, the sky gray from pollution and flecked with ash and other debris. But that was an image published in 1984. A decade and a half later, Neil Gaiman pointed out that to his children, the color of television tuned to a dead channel is bright blue, thanks to ubiquitous cable delivery. In another decade, the default color of a station not in use may be something else entirely. The moral is that it's not enough to agree that an image is very striking. You'll want to make sure that you all agree on what it is about it that's striking, to avoid a tangle of misconceptions that could derail play later on. 

The References page (p. 394) offers a wide range of immediately relevant inspirations, but it's not the final word on the subject. If the people in your group have a long-time favorite space scene, or description of life in the midst of a high-tech investigation, or poetic glimpse of what it might feel like to modify the body in ways not possible in real life, or something else that's stayed with them a long time and seems like it might bear on your campaign, encourage them to share. Remember to be courteous with each other's personal treasures, whether you end up using them or not; there's nothing like earned trust to encourage more sharing. 

Images can be particularly helpful for what they convey about the world behind and around the foreground events. For instance, think of a corridor on a typical spaceship or habitat in \textit{Eclipse Phase. } Did you imagine it as being a standardized size and shape, so that its counterparts elsewhere would be very much the same or a more individualized work intended for use just where it is without concern for interchangeability? Did you imagine it as well lit even when not in use, lit well when sensors show people present and otherwise dim or dark, or perhaps planned to be well lit but in practice haphazard and unreliable thanks to lack of maintenance and funds? Did you imagine its surfaces smooth and clean, with 

equipment, maintenance bays, and the like all behind hatches and covers, or was it cluttered and lumpy? None of that matters all the time, but when it comes to the investigation of a derelict, the hunt for someone (or something) trying to hide, a race against time, or other dramatic complication, these things could affect your play, and rather than try to tally all possible contingencies in advance, having some general-purpose references can save everyone time and confusion. 

\subsubsection{Things That Should Not Be: Horror } 

The universe of \textit{Eclipse Phase} is a time of horrors unleashed. Every character has had to come to some personal accommodation with the existence of things that offend our basic expectations of decency and practicality all at once. Horror comes in many flavors, and no one campaign can make use of all of them. 

There are at least as many theories of horror as there are people who create horror stories. Everything here is necessarily a generalization. You and your players can find exceptions to every single point in it, and if you like the way those work better, go with them. This discussion is intended to trigger ideas, not to close off anything. That said, there are some useful generalizations about horror, starting with an insight expressed well by H.P. Lovecraft: ``The oldest and strongest emotion of mankind is fear, and the oldest and strongest kind of fear is fear of the unknown.'' All horror can be thought of as built around encounters with the unknown, beginning with the realization that there is something unknown present, learning something about the scope of its nature and activities, and then trying to respond one way or another. 

In this game, the discovery part is half over. There's no question about the presence of the unknown. Yes, there really are monsters beyond transhuman understanding loose in the universe, and everyone in the \textit{Eclipse Phase} universe knows how bad and how \textit{strange }the TITANs could get. Many people also have some idea of how exotic life on the far side of the Pandora Gates can be. There's no room left for characters to respond to some new strangeness with confident skepticism, sure that they know the range of what's possible and plausible within transhuman experience anymore. Almost anything might exist, given the facts of what's already known. Instead, the question for \textit{Eclipse Phase }people facing a mystery is whether this particular unknown will turn out to be simple and straightforward to deal with, more complicated but nonetheless a part of their routine lives like malfunctioning machinery or a sabotaged and unusually modified morph, or something beyond the normal like a TITAN-programmed weapon or alien life. Sooner or later, if they keep poking around, the characters can count on running into all sorts of unknown and maybe even unknowable challenges. Are they there yet? 

Horror is seldom very far from humor. Humor serves many roles in human psychology, and one of them is helping us whittle down the mental ``size'' of mysteries and threats to something we can deal with. Furthermore, horror usually involves a balance of improbable elements, with things lined up to go wrong in interesting ways, and it doesn't take much for a particular rickety edifice to go from strange and menacing to ludicrous. When your players start laughing, sometimes the best thing for you to do is to roll with it. Laughter can do everyone good, supporting the ``play'' part of roleplaying. In addition, some events actually are funny or at least can be taken as funny, even (sometimes especially) when most of what's going on is serious. On the other hand, if you really would like to keep a scene serious and the players break out in giggles, it's often wise to go ahead and take a break. Tell the players what you're doing, too; trying to deceive a group of your friends isn't very reliable and can backfire badly. Make the break long enough for everyone to get the giggles out and then continue. 

At the end of the day, through communication with their players, the gamemaster will know how much horror their group wishes to encounter. A group may decide that they want to be 100 percent immersed into the various horrors of \textit{Eclipse Phase.} Another group, however, may decide that while they enjoy the meshed theme of horror with the other aspects of \textit{Eclipse Phase,} they don't wish it to be a principal element. In such a situation, horror would remain just that, a theme, while the plots woven by the gamemaster would spin around the myriad of other elements that make up the game. 

\subsubsection{Transhumanism} 

Humanity has embraced transhumanism for survival, harnessing science and technologies to catapult physical and mental faculties to super-human levels, while eradicating involuntary death and enabling near immortality through the digitization of consciousness and the ability to transfer bodies at will. This is one of the cornerstone themes of \textit{Eclipse Phase.} 

The technologies inherent to a transhuman future raise many questions and ethical issues, however, and these are some of the central themes that \textit{Eclipse } \textit{Phase} seeks to explore. We encourage both gamemaster and players to play around with the possibilities and contradictions enabled in such a universe. How do our mindsets change when death no longer looms over us? What does identity mean when our bodies are disposable and our personalities can be edited? Are we the same person when we are revived from a backup, or sent off as a fork? Are technologies like nanofabrication something to be feared and restricted, even when they can eliminate poverty and greed? How do we ensure public safety in a world where technology makes weapons of mass destruction easily available? How do ideas inherent to religious and spiritual thought cope with AI, backups, or resleeving What does it mean to be an uplifted animal in a society centered on humans? Who decides our future? These are just a few of the issues that \textit{Eclipse Phase} raises, and many of them can be used as the central theme for an entire campaign. 