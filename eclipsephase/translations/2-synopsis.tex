\chapter{La singularité arrive.} \label{chap:enter-the-singularity} 

Nous autres humains avons une manière très particulière de nous hisser vers le haut tout en se mettant des bâtons dans les roues. Nous avons accompli plus de progrès que jamais, au prix de la destruction de notre planète et de la déstabilisation de nos gouvernements. Mais ça n'était que le début. 

Avec une technologie progressant à une vitesse exponentielle, nous avons atteint les limites du système solaire. Nous avons re-forgé nos corps et nos esprits, abandonnant la maladie et la mort. Nous avons atteint l'immortalité grâce à la numérisation de nos esprists, passant d'un corps biologique ou synthétique au suivant. À volonté. Nous avons donné la conscience à des animaux et aux IA afin d'en faire nos égaux. Nous avons acquis les moyens d'assembler au niveau moléculaire tout ce que nous désirions, pour que plus personne ne soit jamais dans le besoin. 

Et cependant notre trajectoire vers l'extinction n'as pas été ralentie, et à même reçu une assistance des machines pour nous pousser dans le précipice. Des milliard sont mort alors que quelque chose hors de contrôle est sorti de l'œuf de notre technologie ... transformant encore l'humanité en quelque chose d'autre, nous dispersant à travers le système solaire, et rallumant les anciens conflits. Frappes nucléaires, armes biologiques, nuées de nanites, upload de masse ... un millier d'horreurs faillirent supprimer l'humanité de l'univers. 

Nous avons survécu, divisé entre le patchwork d'olligarchie restricitive des hypercorp du système intérieur d'un côté et des habitats libertariens et collectivistes, des réseaux tribaux et de nouveau modèles de sociétés expérimentaux dans le système extérieur de l'autre. Nous nous sommes propagés jusqu'à la limite extérieure du système solaire et nous avons même acquis quelques têtes de ponts dans la galaxie. Mais nous ne sommes plus seulement "humains" ... nous avons évolué en quelque chose qui est à la fois mieux et différent - quelque chose de transhumain. 



\section{Commencer.} \label{sec:starting-out} 

Eclipse Phase est un jeux de rôle post-apocalyptique de conspiration et d'horreur transhumaniste. Les humains sont renforcés et améliorés, mais l'humanité est battue et profondément divisée. La technologie nous permet de retravailler nos corps et nos esprits et nous a libérée de nos besoins matériels, tout en créant des opportunités d'oppression et donnant à tout le monde des capacités de destruction massive. Des menaces rôdent dans les habitats dévastés par la Chute, des dangers à la fois familiers et étranger. 



\subsection{Qu'est-ce que le jeu de rôle?} \label{sec:what-roleplaying} 

Avez-vous déjà lu un livre ou vu un film ou une série télévisée où l'un des personnages fait quelque chose de vraiment stupide, comme descendre à la cave de nuit quand le personnage sait qu'il y a un tueur en série dans le coin? A chaque fois, vous pensez: "Je ne descendrait jamais ces escaliers terrifiant dans cette cave obscure, surtout sans une lampe de poche. Je ferais plutôt ça, ou ça!" En étant un simple spectateur de l'intrigue que vous lisez ou regardez, vous ne pouvez que vous asseoir et laisser se dérouler l'histoire. 

Et si vous pouviez prendre le contrôle? Et si vous pouviez diriger l'intrigue dans la direction que vous choisiriez? C'est l'essence même du jeu de rôle. 

Un jeu de rôle (ou JDR) est un mélange de théatre d'improvisation, d'art du conte et de jeu. Une personne (le maître de jeu) anime la partie pour un groupe de joueurs prétendant être des personnages dans un monde fictif. Le monde peu aussi bien être un décor mystérieux dans les années 1920 qui vous emmènerait dans des aventures autour du globe, un royaume mystérieux habité par des dragons, des trolls et des barbares maniant l'épée ou un cadre de science fiction avec des aliens, des vaisseaux spatiaux et une artillerie capable de détruire des planètes. Les joueurs choisissent un cadre qu'ils trouvent intéressant et dans lequel ils veulent jouer. Ils créent ensuite leurs propres personnages, leur fournissant un historique détaillé et une personnalité pour leur donner la vie. Ces personnages ont un ensemble de statistqiues (des valeurs numériques) qui représentent les compétences, attributs et autres capacités. Le maître de jeu explique ensuite la situation dans laquelle se trouve les personnages. Les joueurs, par l'intermédiaire de leur personnage, interagissent avec les évènements et les personnages des autres, agissant sur l'intrigue. Alors que les joueurs interpètent leur rôle à travers le scénario, le maître de jeu demandera probablement à l'un des joueurs de lancer quelques dés dont le résultat déterminera la réussite ou l'échec de l'action qu'à tenté le personnage. Le maître de jeu utilise les règles du jeu pour interpréter le jet de dé et les conséquences de l'action du personnage. 

En tant qu'exercice de groupe, les joueurs contrôlent la suite d'évènement (l'aventure), qui évolue à peu près comme n'importe quel film ou livre mais à l'inétrieur de l'intrigue flexible créée par le maître de jeu. L'intrigue du maître de jeu fournit un cadre et des idées pour déterminer les possibilités d'actions et leurs impact, mais c'est plus simplement un aperçu de ce qui pourrait arriver - cela devient concret uniquement au moment où les joueurs sont impliqués. Si vous ne voulez pas descendre ces escaliers, vous ne les descendez pas. Si vous pensez pouvoir vous sortir d'une situation en discutant calmement plutôt qu'en dégainant une arme, alors allez-y et tentez votre chance. Le script de chaque session de jeu est écrit par les joueurs, et l'histoire, basée sur les actions des personnages et leur réponse aux évènement de l'intrigue, changera et évoluera en permanence. 

Le meilleur c'est qu'il n'y a pas de "bonne" ou "mauvaise" façon de jouer à un JDR. Certains jeu impliqueront plus de combat et de situation liées à des jets de dés, là où d'autre impliquerontplus d'interprétation et de dialogue improvisés pour résoudre une situation. Chaque groupe de joueur décide le type et le style de jeu auquel ils ont envie de jouer! 



\subsection{Qu'est-ce que le transhumanisme?} \label{sec:what-transhumanism} 

Le transhumanisme est un mot utilisé comme synonyme de "amélioration de l'humain." C'est un mouvement culturel et intellectuel international qui approuve l'utilisation de la science et de la technologie pour améliorer la condition humaine, aussi bien mentalement que physiquement. Pour supporter cette théorie, le transhumanisme englobe également l'utilisation de la technologie pour éliminer les éléments indésirables de la condition humaine tels que le vieillissement, les maladies et la mort non volontaire. Beaucoup de transhumains croient que ces technologies arriveront dans un futur proche suivant un rythme exponentiel et travaillent à promouvoir l'accès universel et le contrôle démocratique de ces technologies. Sur le long terme, le transhumanisme peut aussi être considéré comme la période de transition entre l'humain actuel et une entité qui aurait tellement évoluée dans ses capacités (aussi bien mentales que physiques) qu'elle mérite d'être appellée "posthumain". 

En tant que thème, la transhumanité brasse des questions lourde de significations. Qu'est-ce qui défini l'humain? Qu'est ce que signifie vaincre la mort? Si l'esprit est un logiciel, où se situe la limite avec leur programmation? Si les machines et les animaux peuvent être amenée à ressentir, quelles sont nos responsabilité vis à vis d'eux? Si vous pouvez vous copier, où se situe la limite entre "vous" et quelqu'un de nouveau? Quels sont les potentiels de ces technologie en terme de contrôle oppressif et de libération? Comment ces technologies changeront nos société, nos cultures et nos vies? 



\subsection{Les thèmes post-apocalyptiques, de conspirations et d'horreur} \label{sec:post-apoc-consp} 

Plusieurs thèmes imprègnent Eclipse Phase, le lecteur pourrait ne pas être familier avec certains d'entre eux. Les aides suivantes définissent ces thèmes afin de permettre aux joueurs lisant ce livre de règle de gagner une compréhension solide de la manière dont Eclipse Phase est construit autour de ces thèmes pour créer son cadre unique. 

Post-apocalyptique est un terme utilisé pour décrire les fictions se déroulant après un évènement catacyclismique qui a mis fin à la civilisation humaine telle que nous la connaissont actuellement (habituellement accompagnée par la perte de vie humaine sur une échelle impensable). L'exacte mécanisme du désastre est habituellement sans importance. Il peut s'agir d'une guerre nucléaire, d'une épidémie, de la chute d'un astéroïde, etc. La part importante du thème est la condition humaine. Si le monde tel que nous le connaissons nous a été arraché et que les humains on subit des horreurs au-delà de l'imaginable lors de cette transformation à un cadre post-apocalyptique, comment l'humanité y fait face? Survivons-nous? Prospérons-nous? Surmontons-nous ces difficultés? Ou perdons-nous notre humanité dans le processus, succombant finalement à l'extinction? Ce sont les questions qui sont au cœur du thème. 

Conspirer signifie "joindre une organisation secrète pour atteindre un but illégal ou mauvais ou utiliser ces moyens à un but légal." En tant que telle, une théorie conspirationiste attribue la cause ultime d'un évènement ou d'une succession d'évènements (qu'ils soient politique, sociaux ou historiques) à un groupe secret d'individus possédant un pouvoir immense (incluant politique et financier entre autres) et qui dissimulent leurs activités au public tout en manipulant les évènements afin d'atteindre leurs buts, sans se soucier des conséquences. Beaucoup de théories conspirationnistes prétendent qu'une grande partie des évènements historiques ont été initiés et finallement contrôllés par de telles organisations secrètes. Sur le même plan d'importance, on trouve la lutte silencieuse entre des groupes clandestins, s'affrontant dans une guerre occulte afin de déterminer qui influencera le futur. 

L'horreur prend beaucoup de formes, mais dans Eclipse Phase, il s'agît plus d'horreur psychologique que de gore. C'est la survie incertaine, le suspens de trouver des choses malveillantes parmi les étoiles, la peur de l'inconnu, l'angoisse d'affronter des Choses Qui Ne Devraient Pas Être, la révulsion lors de la rencontre de choses étrangères, et la prise de conscience maladive des choses horribles et mauvaises que les transhumains sont capable de s'infliger. L'horreur apparaît aussi du fait qu'il y a des choses effrayantes au delà de notre compréhension qui habitent notre univers et du fait que la transhumanité est probablement son pire ennemi. En dépit de tous les outils technologiques et des avancées disponibles aux futurs transhumains, ils doivent toujours affronter des horreurs comme la perte de contrôle de leur propre identité, de leur perceptions et de leur faculté mentale - sans parler du contrôle de leur futur en tant qu'espèce. 

Eclipse Phase mélange tout ces thèmes dans un contexte transhumain. L'angle postapocalyptique couvre la compréhension de tout ce que la transhumanité à perdu, la lutte contre l'extinction et la part de cette lutte qui est un combat contre notre propre nature. L'angle conspirationniste se situe dans la nature des organisations secrètes qui jouent un rôle clef dans la déterminatuion du futur de la transhumanité et comment les actions d'individus déterminés peut influer sur la vie de beaucoup. La perspective de l'horreur permet d'explorer les résultats des transformations que c'est infligé l'humanité, et comment certains de ces changements nous a effectivement transformés en non-humain. Mélanger ces thématiques permet une sensibilisation à l'indifférence massive et à la terrible étrangeté qui imprègne l'univers et souligne à quel point la transhumanité est insignifiante dans un tel contexte. 

Au delà de ces thèmes, cependant, Eclipse Phase affirme également qu'il y a toujours de l'espoir, qu'il y a quelque chose quimérite d'être défendu, et que la transhumanité peut se frayer son propre chemin vers le futur. 



\subsection{Mais comment jouons-nous?} \label{sec:but-how-do} 

Pour jouer une partie d'Eclipse Phase, vous aurez besoin du matériel suivant: 

\end{itemize} \item Un groupe de joueur et un endroit où se rencontrer (dans le monde réel ou en-ligne!) \item Un joueur pour tenir le rôle de maître de jeu \item Le contenu de ce livre \item Quelque chose pour que les personnes puisse prendre des notes (bloc notes, ordinateur portable, peu importe!) \item Deux dés à dix-faces (ou un équivalent numérique) \item De l'imagination \end{itemize} 

\subsubsection{Un groupe de joueur et un endroit pour se rencontrer} \label{sec:group-players} 

Bien que jouer au jeu de rõle soit suffisament flexible pour permettre à n'importe quel nombre de personne de participer, la plupart des groupes de jeu sont composés de quatre à huits joueurs. Ce nombre de participant permet un mélange interessant des personnalités autour de la table et assure une bonne coopération pendant la partie. 

Une fois qu'un groupe de joueurs a déterminé qu'ils allaient jouer à Eclipse Phase, ils auront besoin de désigner quelqu'un pour être le maître de jeu (voir plus bas). Ils auront ensuite besoin de déterminer la date, l'heure et le lieu de la partie 

La plupart des groupes de jeu se retrouvent une fois par semaine à un intervalle de temps régulier: 19:00, Jeudi soir, chez Rob par exemple. Chaque groupe détermine cependant où, quand et comment ils veulent jouer. Un groupe peu décider qu'ils ne peuvent se rassembler qu'une fois par mois, alors qu'un autre est tellement excité à l'idée de se plonger dans les histoires potentielles d'Eclipse Phase, qu'ils voudront se rencontrer deux fois par semaine (ils décident d'une rotation entre leurs domiciles cependant, pour ne pas surcharger un joueur en particulier). Si un groupe à suffisament de chance pour que leur boutique de jeu préférée acceuille des parties, ils pourront décider de se retrouver là-bas. D'autres groupes se rencontrent dans des bibliothèques, dans des salles communes dans leurs écoles, dans des librairies qui possèdent des "salles de lecture" généreuses. Quelque soit ce qui convient à votre groupe de jeu, faites le! 

Lorsque les joueurs se retrouvent pour jouer, la plupart des JDR parlent de "session de jeu." La durée de chaque session de jeu dépend autant du consensus établi par le groupe de joueur que des limitations du local dans lequel ils jouent. L'histoire particulière qui se déroule dans une session donnée peut aussi impacter sur la durée de celle-ci. En jouant dans une boutique de jeu, le groupe de joueur peut n'avoir qu'un créneau de quatre heures ou bien le maître de jeu et le groupe peuvent avoir déterminés - après plusieurs sessions de jeu - que c'est un laps de temps parfait pour profiter de l'histoire à laquelle ils participent chaque semaine. Un autre groupe, cependant, pourrait vouloir un temps de jeu encore plus court. Un groupe encore différent pourrait décider que, même si d'habitude ils jouent par session de quatre heures, une fois par mois ils se regrouperont pour jouer tout le Samedi pour une super session de jeu étalée sur toute la journée. Les joueurs devraont s'immerger et commencer à jouer et faire preuve de souplesse pour décider ce qui leur procurera l'amusement ultime pour leur groupe de jeu. 

Alors que la camaraderie d'une expérience partagée en jouant face à face avec un groupe d'amis reste la force du jeu de rôle, les groupes ont besoin de ne pas se confiner à un seul mode de jeu. Des myriades d'options peuvent être utilisées. E-mail, messageires instantanées, forums, chat vidéo, appel téléphonique/en VoIP, SMS, wikis, (micro-)blogs: chacun d'entre eux peut être utilisé pour jouer sans se tenir chaud directement assis autour d'une table les uns en face des autres. 

Finalement, lorsqu'un groupe de jeu se rencontre pour la première fois, les joueurs devraient créer leurs personnage (par opposition à la création de personnage chacun de son côté). Alors qu'un groupe de joueur peut décider de générer leur personnages individuellement, il est souvent bien plus facile de le faire une fois que les joueurs sont rassemblés. Cela permet à ceux qui ont plus d'expérience dans le jeu de rôle d'aider ceux qui débutent. Encore plus important, cela permet au groupe entier de dimensionner les personnages pour qu'il n'y ait pas trop de doublons au niveau des compétences et des styles. Après tout, avec la richesse des types de personnages disponible, vous ne voulez pas arriver à la table de jeu avec un personnage presque identique à celui de votre voisin. 

\subsubsection{Le maître de jeu} \label{sec:gamemaster} 

Une fois qu'un groupe s'est organisé, quelqu'un doit s'avancer et prendre les rênes du maître de jeu. Certains groupes ont un seul maître de jeu qui anime toute leurs sessions de jeu mois après mois. D'autre groupes changent régulièrement de maître de jeu, avec l'un d'entre eux animant une portion donnée du déroulement de l'histoire pendant plusieurs session avant de passer le rôle à un autre joueur. Encore une fois, les participants doivent faire preuve de souplesse. Certains groupe peuvent avoir la personne idéale qui aime le travail induit par le rôle et qui est plus que volontaire pour animer une session après l'autre, alors que d'autres peuvent décider qu'ils veulent tous alterner entre être le maître de jeu et être un joueur. 

Le maître de jeu contrôle l'histoire. Il garde une trace de tout ce qui est supposé arrivé et de quand cela arrive, il décrit les évènements au fur et à mesure qu'ils se produisent afin que les jouers (en tant que personnages) puissent réagir, il garde une trace des autres personnages de la partie (appelés personnages non joueur, ou PNJs), et résolve les actions en utilisant le système de jeu. Le sytème de jeu intervient lrosque les personnages cherchent à utiliser leur compétences ou à faire quelque chose qui nécessite un test pour détermienr si ils ont réussit ou pas. Des règles spécifiques sont présentées pour les situations qui impliquent de lancer des dés pour déterminer la réussite (voir Mécanique de Jeu, p.  112). 

Le maître de jeu décrit le monde tel que le voient les personnages, en utilisant leurs yeux, leurs oreilles, et leurs autres sens. Maîtriser n'est pas facile, mais les frissons de la création d'une aventure qui engage l'imagination des autres joueurs, et confrontent leur compétence de jeu et les compétences de leurs personnages dans l'univers de jeu, vaut largement l'investissement. Posthuman Studios et Catalyst Game Labs suivront la publication d'Eclipse Phase en publiant des suppléments et des aventures pour faciliter le processus, mais les maîtres de jeu expérimentés peuvent toujours adapter l'univers de jeu à leur propre style. En fait, puisque Eclipse Phase est publiée sous licence Creative Commons (voir p. 5), les joueurs sont encouragés à adapter l'univers à leur style de jeu et à partager leurs modifcications avec les autres joueurs. Vous ne saurez jamais quand un choix spécifique que vous faites en maîtrisant une campagne est exactement ce qu'un autre maître de jeu et son groupe recherchent. 

\subsubsection{Contenu de ce livre} \label{sec:contents-this-book} 

Que vous vous soyez procuré la version imprimée ou électronique, ce livre est organisé spécifiquement pour présenter les informations que vous avez besoin de connaître pour commencer à raconter votre histoire dans l'univers d'Eclipse Phase. Vosu trouverez ci-dessous un résumé de chaque chapitre de ce livre 

\paragraph{Une Époque d'Éclipse:} Une histoire globale et un cadre entièrement détaillé décrivant l'univers d'Eclipse Phase et racontant comment l'humanité a effectué une transition d'ici à là-bas. Voir p. 30. 

\paragraph{Mécaniques de Jeu:} Les actions voulues par les joueurs deviennent réalité dans l'univers grâce à des mécaniques de jeu simple et facile à utiliser. Voir p. 112. 

\paragraph{Création de Personnage et Avancement:} Créer un personnage unique peut-être l'une des expéreinces les plus intéressantes du jeu de rôle. Encore plus gratifiant est de voir ce personnage évoluer et grandir au travers de nombreuses sessions de jeu, bien au-delà de tout ce que votre imagination avait envisagé. Voir p. 128. 

\paragraph{Compétences:} Au delà des capacités innées d'un personnage, les compétences sont ce qui les distinguent. C'est ce que votre personnage sait et ce qu'ils savent faire. Voir p. 170. 

\paragraph{Action et Combat:} Qu'est-ce qu'une histoire dramatique sans action ni violence? Lorsque les mots échouent, les armes s'enflamment. Voir p. 186. 

\paragraph{Piratages Cognitifs:} Les possibilité inahbituelles offertes par les capacités psi et la reprogrammation mentale. Voir p. 216. 

\paragraph{Le Mesh:} La nature omniprésente du mesh assure que c'est un élément clef de toutes les histoires racontées. Voir p. 234. 

\paragraph{Futur Accéléré:} Les merveilles de la technologies avancées et comment elle fonctionne. Voir p. 266. 

\paragraph{Équipement:} Des améliorations personnelles, aux armes en passant par les robots. Voir p. 294. 

\paragraph{Informations de jeu:} La quintessence de l'ensemble des secrets destinés aux maîtres de jeu. Voir p. 350. 



\subsubsection{Prendre des notes} \label{sec:taking-notes} 

Que vous soyez maître de jeu ou joueur, vous avez besoin d'un moyen quelconque de suivre l'information. Lesjoueurs créeront des personnages et leur apporteront des changements d'une session à l'autre. Entre-temps, le maître de jeu devra suivre un paquet d'information: des notes sur la façon dont se déroule l'histoire suite aux interactions des personnages joueurs et dont vous devrez tenir compte lors de la session de la semaine suivante; des changements apportés aux PNJs; des changements apportés aux personnages joueurs que les joueurs ignorent encore (comme le fait qu'un personnage se soit fait pirater le cerveau mais qu'il ne le sait pas encore); et ainsi de suite. 

Additionellement, certains groupes apprécient un résumé de chaque session qui puisse être compilé et lu un peu plus tard pour appréicer et partager leurs exploits, de la même façon dont vour partageriez des extraits vidéos de votre jeu vidéo préféré pour montrer vos capacité à battre le méchant (traditionnellement cela s'appelle "tenir un journal"). Cela se révèle particulièrement utile si un joueur n'as pas pu partciiper à une session donnée, fournissant un résumé rapide qu'il peut lire avant de venir à la prochaine session de jeu et s'éviter ainsi de tomber dans un piège pendant qu'il essaye de se raccrocher aux évènements de la session de jeu actuelle. La prise de note est une responsabilité qui peut être partagée entre tous les joueurs ou assignée à un seul, tout dépend de ce qui fonctionne le mieux au sein du groupe. De la même manière, certains groupe de jeu réalisent un enregistrement audio de toute la session de jeu, à la fois pour pouvoir s'y reporter plus tard et pour effectuer des podcasts de "jeu live". 

Les vieux standard du crayon et du papier font encore des miracles. Un tas de techonolgies additionelles fournissent cependant plein de nouvelles options aux joueurs. D'un fichier texte sur un portable à un wiki partagé, la possibilité de suivre une somme d'information suffisament importante d'une manière simple et rapide - tout en fournissant des informations appropriées à chaque joueur d'une session de jeu à l'autre - réduit de manière significative le temps que chacun passe à suivre ces informations. Ce temps peut maintenant être réallouer au plaisir de participer à une historie grandiose. 



\subsubsection{Dés} \label{sec:dice} 

Comme décrit dans la section Mécanique de Jeu (p. 112), deux dés à dix faces sont nécessaires pour jouer à Eclipse Phase. Bien que la plupart des joueurs prennent plaisir à lancer les dés, il existe un nombre important de mécanisme pour parvenir à un résultat compris entre 00 et 99. Les joueurs qui font un usage intensif de techonolgies online pour jouer - tels que les discussions en ligne ou les blogs vidéos - peuvent trouver plus simple de créer et implémenter rapidement un petit programme de lancer de dés. 



\subsubsection{Imagination} \label{sec:imagination} 

Bien trop souvent, il est facile pour quelqu'un qui regarde un JDR d'être intimidé. Il y a tellement de concepts à saisir, tellement d'idées qui semblent écrasantes. Cependant, comme décrit à la section Qu'est-ce qu'un Jeu de Rôle?, combien de fois avez-vous lus un livre ou vu un film et décidés que vous auriez fait mieux? C'est votre imagination qui est à l'œuvre. Laissez-vous simplement aller et vous serez étonnés de la vitesse à laquelle vous pouvez vous immerger dans l'univers d'Eclipse Phase. Bientôt vous serez au cœur d'histoires avec le meilleur de l'univers. 

N'oubliez pas non plus d'exploiter vos ressources. Votre groupe de jeu est la meilleure d'entre elles. Ce qu'il s'y passe, les idées pour gérer une situation ou pour vaincre le méchant: ce sont juste quelques unes des choses qui peuvent et doivent être disuctées par le groupe de joueur entre les sessions, et chacune de ces discussions est une opportunité de développer votre imagination. 

Une autre ressource est de simplement regader la télé ou de lire un bon livre. Faites attention à la manière dont l'histoire est assemblée, comment les personnages sont construits, et comment l'intrigue est dévoilée. Travaillez votre imagination et bientôt vous devinerez les sous-intrigues et qui est réellement le méchant bien avant qu'ils ne soient révélés. Savoir comment une histoire est construite vous permets d'assembler les votres lors de chaque session de jeu. 

Enfin, eclipsephase.com est le site officiel pour Eclipse Phase. Si vous avez des questions à propos du jeu, ou que vous voulez savoir comment un autre groupe de joueur gèrerai une situation, postez sur les forums. La communauté en ligne peut-être tout autant utile et plaisante qu'un groupe de jeu local. 



\subsection{Que font les joueurs?} \label{sec:what-do-players} 

Les joueurs peuvent remplir toute une variété de rôles dans Eclipse Phase. Suite à l'avancée dans les technologies de l'émulation digitale de l'esprit, l'upload et le download dans de nouvelles morphs (corps physiques, biologiques ou synthétique), il est possible d'être, littéralement, une nouvelle personne d'une session à l'autre. Avec les corps qui se retrouvent réduit au rôle d'équipement, les joueurs peuvent personnaliser leur apparence pour la tâche à venir. 



\subsubsection{La campagne par défaut} \label{sec:default-campaign} 

Dans l'histoire par défaut (aussi appellée "cadre de campagne"), chaque personnage joueur est une "sentinelle", un agent disponible (ou une recrue potentielle) pour un réseau paralégal appelé "Firewall". Firewall est dédié à contrer les "risques existentiels" - des menaces à l'existence de la transhumanité. Ces risques peuvent inclure des fléaux de la guerre biologique, des invasions d'essaim de nanites, de la prolifération nucléaire, des terroristes avec des ADMs, des attaques informatiques destructrices, des IAs maligne, des rencontres aliens, et ainsi de suite. Firewall ne se contente évidement pas de simplement contrer ces menaces au moment où elles apparaissent, les personnages peuvent donc être également envoyés sur des missions de renseignement ou pour mettre en place des mesures de prévention ou de sécurité. Les personnages peuvent être chargés d'enquêter sur des personnes ou des lieux apparement innofensifs (et qui se rèveleront ne pas l'être), de négocier des arrangements avec d'obscurs réseaux criminels (qui se réveleront ne pas être digne de confiance), ou de voyager à travers le trou de ver d'une Porte de Pandore pour annalyser des reliques d'une ruine alien quelconque (et de vérifier si la menace qui les as tués est toujours là). Les sentinelles sont recrutés dans toutes les factions de la transhumanité; ceux qui ne sont pas idéologicallement loyaux à la cause sont recrutés en tant que mercenaires. Ces campagnes ont tendance à se méler à un peu de mystère et d'investigation avec des scènes d'actions et de combats achanés, se baignant dans une bonne dose de crainte et d'horreur. 



\subsubsection{Campagnes alternatives} \label{sec:alternate-campaigns} 

Quand ils ne sont pas en train de sauver le système solaire, les sentinelles sont libres de poursuivre leurs propres buts. Le maître de jeu et les joueurs peuvent utiliser ce livre de règle pour générer tout type d'histoire qu'ils ont envie de raconter. Cependant, les exemples suivants fournissent un rapide aperçu des opportunités les plus évidentes pour des aventures dans Elcipse Phase. 

Après chaque variantes de campagnes ci-dessous, une liste "d'archétypes" pour Eclipse Phase est fournie entre parenthèses. Les archétypes sont les noms donnés au type de personnage le plus commun rencontrés dans ces scénarios. Par exemple, dans une histoire de détective classique, les archétypes seraient le Détective, la Demoiselle en Détresse, le Flic Dur à cuir, etc. Dans un film de cowboy, les archétypes seraient le Pistolero, le Barman, le Shériff, le Brave Indien, etc. Les joueurs noterons que plusieurs archétypes correspondant à de multiple cadre de scénario. Le système de création de personnage (p. 128) permet aux joueurs de créer tous les types d'archétype suggérés. De la même manière que les jeus de rôle sont conçus pour que les joueurs construisent leurs propres histoires, ces archétypes sont juste des suggestions et les joueurs peuvent les mélanger et choisir ceux qu'ils veulent. 

\paragraph{Mission de Récupération et de Sauvetage:} la Chute a laissé deux mondes et de nombreux habitats à l'état de ruines - mais ces cités et stations dévastées contiennent des trésors cachés pour ceux qui sont suffisament courageux et téméraires. Les trouvailles peuvent inclure: des systèmes d'armement; des ressources physiques; des banques de données perdues; des uploads abandonnés d'amis, de membres de la famille ou de personnes importantes; de nouvelles technologies développées et perdues lors du décollage brutal de la singularité; des héritages de valeur pour des oligarques immortels; et bien plus. À l'extérieur de ces royaumes autrefois habités, l'espace est en soi un endroit trés grand et beaucoup de personnes et de choses sont perdus là-bas dehors. Quelqu'uns doivent être sauvés et d'autres sont au delà du sauvetage. Cette option laisse les joueurs explorer l'inconnu ou chercher des cibles spécifiques pour un contrat. (Archéologiste/Charognard/Pirate/Libre-Échangiste/Contrebandier/Marchand Noir) 

\paragraph{Exploration:} Il y a pas mal d'opportunités à être un explorateur, un colon, ou un éclaireur avancé - peut-être même l'un des rare individus suffisament chanceux ou suicidaires et qui explorent via une Porte de Pandorre non testée. Même la Ceinture de Kuiper, à la limite de notre système solaire, est toujours partiellement explorée; il peut y avoir des trésors et des mystères à découvrir. Il y a aussi de nombreux dangers qui rôdent dans les recoins étranges du système, des factions posthumaines isolationnistes aux cartels criminels cachés, en passant par les pirates, les aliens et d'autres choses qui veulent demeurer hors de vue. (Explorateur/Archéologue/Charognard/Adepte de la Singularité/Techie/Médecin) 

\paragraph{Commerce:} Alors que la majorité du commerce dans le système intérieur est contrôlé par de brillantes hypercorporations, la plupart des stations les plus petites ou indépendantes dépendent des petits négociants. Dans les systèmes post-abondance à l'extérieur, le commerce prend une forme différente, avec l'information, les faveurs et la créativité qui servent de monnaie parmi ceux qui n'ont plus besoin de quoi que ce soit grâce à la disponibilité des machines d'abondances. (Libre-Échangistes/Contrebandier/Marchand Noir/Pirate) 

\paragraph{Crime:} L'assemblage hétéroclite d'habitats de la taille d'une ville et les lois variant grandmenet à travers le système ont créés d'amples opportunités pour ceux qui voudraient faire de cette situation leur gagne-pain. Les marchandises et activités du marché noir incluent l'échange d'esclave infomorph, les industries du plaisir et des sex pods, l'échange et le vol de données, l'exfiltration et le convoyage de technologies avancées et de scientifiques, el'spionnage politique et économique, l'assassinat, la vente de drogues et d'XP, l'échange d'âme, et bien d'autres choses. Que ce soit en tant qu'indépendant ou en tant que membre d'une organisation criminelle, il y a toujorus des opportunitées pour ceux qui ont soif d'aventure ou de profit et à la moralité douteuse. (Criminel/Contrebandier/Pirate/Arrangeur/Marchand Noir/GénoHacker/Hacker/Infiltrateur) 

\paragraph{Mercenaires:} Les manœuvres permanente des factions à caractère idéologique, les disputes autour de ressources contestées, et la ruée vers la colonisation de nouvelles exoplanètes au delà des Portes de Pandorre sont à l'origine de conflits sur des bases régulières. Certains d'entre eux couvent pendant des années en tant que conflits de faible intensité, dégénérant occasionnellement en raids et affrontements. D'autres choisissent de s'affronter dans une guerre sans pitié. Des femmes et des hommes voulant prendre les armes pour des crédits sont toujours à la recherche de bonnes paies. Les joueurs peuvent s'engager dans des campagne commando et militaire dans des habitats, au milieu des étoiles, ou dans l'environnement hostile d'une planète. (Mercenaire/Consultant en Sécurité/Arrangeur/Chasseur de prime/Ex-Flic/Médecin) 

\paragraph{intrigue Socio-Politique:} Les corporations et les factions politiques qui couvrent tout le système solaire ne respectent pas toujours les règles du jeu en jouant avec les autres, mais il n'est pas simple pour autant pour eux de se confronter ouvertement les uns aux autres sauf en de rares circonstances. Beaucoup de batailles ont été remportées par des manœuvres diplomatiques et politiques, en utilisant des mots et des idées plus puissantes que des armes. A l'intérieur même des factions, différents groupes sociaux peuvent se mener une concurrence sans pitié, ou les luttes de classes surchauffées  peuvent arriver à ébullition, déchirant une société de l'intérieur. Dans cette campagne, les joueurs peuvent commencer en tant que pions d'une entité quelconque et qui gravirons les échellons alors qu'ils seront de plus en plus impliqués dans les intrigues de leur soutien, ils peuvent joueur un groupe d'ambassadeur et d'espions stationnés dans la capitale de l'opposition, ou encore jouer un groupe d'activistes et de radicaux combattant pour des changements sociaux. (Politicien/Socialite/Infiltrateur/Hacker/Consultant en Sécurité/Journaliste/Memeticien) 



\subsection{Où cela se passe-t-il?} \label{sec:where-does-it} 

Alors qu'Eclipse Phase se déroule dans un futur proche, les changements qui ont été effectués suite à l'avancée technologique ont transformés la Terre et ses habitants de manière méconnaissable. Alros que les joueurs plongent dans l'univers, ils rencontrerons généralement l'un des cadres suivants. 



\subsubsection{Habitats de l'humanité} \label{sec:humanitys-habitats} 

La Terre est devenue une ruine écologiquement dévastée, mais l'humanité s'est envolée vers les étoiles. Lorsque la Terre a été abandonnée,  il en fut autant des derniers états-nation; la transhumanité manque maintenant d'un corps gouvernemental uni et est maintenant assujettie aux lois et aux régulations de quiconque contrôle un habitat donné. 

La majorité de la transhumanité est confinée dans des habitats orbitaux ou des stations sattelites éparpillées à traver tout le système Sol. Certains d'entre eux ont été construits à partir de rien en orbite ou aux points de Lagrange des corps planétraires, d'autres ont été creusés dans des sattelites et de gros astéroïdes. Ces stations possèdent des myriades de but, du commerce à la guerre, en passant par l'espionnage et la recherche. 

Mars continue d'être l'une des plus grandes colonnies de la transhumanité, même si elle a également été durement impactée par la Chute. De nombreuses cités et colonnies persistent, même si la terraformation de la planète n'est que partielle. Vénus, la Lune et Titan abritent aussi une pouplation significative. Il faut également compter un petit nombre de colonnies établies sur des exoplanètes (de l'autre côté des Porte de Pandorre) qui possèdent des environnement peu hostile envers l'humanité. 

Certains des transhumains préfèrent vivre sur de grands vaisseau coloniaux ou sur des essaims de plus petit vaisseaux liée entre eux, en bougeant nomadiquement. Certains de ces vagabond s'exilent intenstionnelement aux confins du système solaires, loin de tout le reste du monde, alors que d'autres commercent activement d'habitat en stations, de stations en habitats, servant de marché noirs mobiles. 



\subsubsection{Le grand inconnu} \label{sec:great-unknown} 

Les zones de la galaxie dans lesquelles l'homme a posé le pied sont peu nombreuses et trés éloignées les unes des autres. Reposant entre ces avant-postes occasionnels de civilisations parfois douteuse se trouvent des mystères à la fois dangereux et merveilleux. Depuis la découverte des Portes de Pandore, il n'y a pas eu pénurie d'avnturiers suffisament courageux ou intrépides pour s'aventurer seuls dans les régions inconnues de l'espace dans l'espoir de trouver un artefact étranger, voire d'établir le contact avec l'une des autres espèces consciente de l'univers. 



\subsubsection{Le mesh} \label{sec:mesh} 

Bien que n'étant pas un "cadre" dans le sens traditionnel, contrairement aux sections décrites ci-dessus, le réseau informatique connu sous le nom de "mesh" est omniprésent. La nature omniprésente de l'environnement informatique a été rendu possible grâce aux techonogies informatique avancées et à la nanofabrication qui permettent un stockage de données illimité et des capacités de transmissions quasi-instantanée. Avec des émetteurs-récepeteurs sans fil microscopique, et peu cher à fabriquer et en surabondance, absolument tout possède une connexion sans-fil et est connecté. Via des implants ou de petits ordinateurs personnels, les personnages ont accès aux archives d'information qui éclipsent l'ensemble de l'internet du 21° siècle et à des systèmes de capteurs qui imprègent chaque lieu public. La vie entière de personnes est enregistrée et commentée, partagée avec d'autres sur l'un des nombreux réseaux sociaux qui relient les gens entre eux dans une toile de contact, de faveur et de systèmes réputationnels. 



\subsection{Ego contre Morph} \label{sec:ego-vs.-morph} 

La distinction entre ego (votre esprit et votre personnalité, incluant les souvenirs, les connaissances et les compétences) et morph (votre corps physique et ses capacités) et l'une des caractéristique d'Eclipse Phase. Une bonne compréhension de ce concept dés le départ offrira un aperçu de toutes les possibilités narratives aux joueurs. 

Votre corps est jetable. Si il devient vieux, malade ou trop gravement abimé, vous pouvez numériser votre conscience et la télécharger dans un nouveau. Le processuss n'est ni bon marché, ni simple, mais il vous garanti une immortalité effective - tant que vous vous rappelez de vous sauvegarder et que vous ne devenez pas fou. Le terme de morph est utilisé pour décrire tout type de forme que votre esprit habite, qu'il s'agisse d'une enveloppe clonée cultivé en cuve, d'une coquille robotique et synthétique, d'un "pod" en partie bio et en partie synthétique, ou même de l'état purement logiciel d'une informoph. 

La morph d'un personnage peu mourir, mais l'ego du personnage continuera à vivre, du moment que les mesures de sauvegardes nécessaires ont été prises. Les morphs sont immuables, mais l'ego de votre personnage représente la continuité des chemins pris par son esprit et sa personnalité tout au long de sa vie. Cette continuité peut-être interrompue par une mort inattendue (dépendant de la date de la dernière sauvegarde), mais elle représente la somme de l'état mental du personnage et de ses expériences. 

Des apects de votre personnage - en particulier les compétences, ainsi que quelques traits et statistiques - appartiennent à l'ego de votre personnage et ainsi l'acompagnent tout au long du développement du personnage. D'autres statistiques et traits sont cependant déterminés par une morph, comme noté précédemment, et changeront donc si votre personnage quitte son corps et en prend un autre. Les morphs peuvent aussi affecter d'autres compétences et statistiques, comme détaillé dans la description des morphs. 



\subsection{Où aller maintenant?} \label{sec:where-go-from} 

Maintenant que vous savez de quoi parle ce jeu, nous vous suggérons de lire le chapitre Une Époque d'Eclipse (p. 30), pour avoir une idée du cadre de jeu par défaut (que vous êtes, bien entendu, libre de changer pour  l'adapter à vos envies). Lisez ensuite le chapitre Mécaniques de Jeu (p. 112) pour avoir une idée des règles. Après ça, vous pouvez passer à la Création et Évolution de Personnage (p. 128) et créer votre premier personnage! 



\subsection{Terminologie} \label{sec:terminology} 

Eclipse Phase utilise tout un jargon pour transmettre simplement les nombreux concepts couverts par ce livre. Bien que non exhaustive, cette liste de terme permettra aux joueurs de s'acclimater rapidement à leur voyage dans Eclipse Phase. Si vous lisez quelque chose et que vous êtes perdus, ne vous inquiétez pas. Ces concepts sont entièrement détaillés dans d'autres sections du livre. 

Notez que plusieurs des mots sur la liste sont des termes scientifiques standard, souvent utilisé en astronomie. Comme Eclipse Phase essaye de rester aussi proche que possible du "hard science" - tout en permettant aux joueurs d'interagir avec les passionantes histoires qui attendent d'être révélées - de tels termes sont utilisés librement. 

\end{itemize} \item Aérostat: Un habitat conçu pour flotter comme un ballon dans la haute atmosphère d'une planète. \item AF: (After the Fall) Après la Chute (utilisé comme date de référence). \item IAG: Intelligence Artificielle Généraliste Une IA qui possède des facultés cognitives comparables ou supérieures à celle d'un humain. Aussi connues sous le nom de "IA forte" (par opposition aux "IA limitées" plus spécialisée). Voir aussi "IA germe." \item IA: Intelligence Artificielle. Généralement utilisé pour se référer à une IA faible; c'est à dire des IAs qui n'englobent pas (ou dans certains cas, qui sont complètement hors de) toute la portée des capacités cognitives humaines. Les IAs diffèrent des IAG dans le fait qu'elles sont généralement spécialisée et/ou intentionnellement bridée/limitée. \item Anarchiste: Quelq'un qui croit que le gouvernement n'est pas nécessaire, que le pouvoir corromp, et que les gens doivent contrôller leur propre vie à travers l'auto-organisation individuelle et l'action collective. \item Arachnoïde: Une synthmorph robotique ressemblant à une araignée. \item Argonautes: Une faction de scientifique tecno-progressistes qui font la promotion d'une utilisation responsable et éthique de la technologie. \item RA: Réalité Augmentée. Informations du mesh (réseau de donnée universel) qui sont superposées à vos sens du monde réel. Les données RA sont habituellement entoptique (visuelles), mais peuvent aussi être auditives, tactiles, olfactives, kinesthésique (conscience corporelle), émotionnelles ou tout autre type d'entrée. \item Async: Une personne avec des pouvoirs psi. \item UA: Unité Astronomique La distance entre la Terre et le Soleil, équivalent à 8,3 minutes lumières, ou à peu près 150 millions de kilomètres. \item Autonomistes: L'alliance des anarchistes, des Barsoomiens, des Extropiens, de la racaille et des Titaniens. \item Barsoomien: Un Martien rural, typiquement irrité par le contrôle des hypercorp. \item Piratage Basilique: Une image ou toute autre entrée sensorielle qui affecte le cortex visuel du cerveau aisni que ses capacités de recognition de motifs d'une manière à provoquer une erreur et de peut-être l'exploiter pour réécrire du code neuronal. \item Ruche: Un habitat à microgravité créé dans un astéroïde ou une lune évidée. \item BF: (Before the Fall) Avant la Chute (utilisé comme date de référence). \item Bioconservateurs: Un mouvement anti-technologie qui milite pour une régulation stricte de la nanofabrication, des IA, de l'upload, du fork, des améliorations cognitives et de toute autre techonologie perturbatrice. \item Biomorph: Un corps bilogique, qu'i ls'agisse d'un plat, d'un splicer, d'un transhumain génétiquement amélioré ou d'un pod. \item Banque de Corps: Un service pour louer, vendre, acheter  ou stocker une morph. Aussi appelé maison de poupée, morgue \item Bots: Robots. Desz coquilles synthétiques pilotée par des IA. \item Sonde Bracewell: Un type de sonde autonome de surveillance de l'espace profond conçue pour établir le contact avec des civilisations étrangères. \item Bordés: Des éxilés qui vivent aux limites du système, ansi que dans tous les autres coins et recoins bien cachés du système. Aussi appellée isolés, limités, dériveurs. \item Caisse: Une coquille synthétique bon marché, commune et produite en masse. \item Chimère: Un transgénique, contenant des traits génétiques d'autre espèces. \item Circumjovien: Orbitant autour de Jupiter. \item Circumlunaire: Orbitant autour de la Lune. \item Circumsolaire: Orbitant autour du Soleil. \item Cislunaire: Entre la Terre et la Lune. \item Clade: Une espèce ou un groupe d'organisme partageant des caractéristique. Utilisé pour se référer aux sous-espèces transhumaines et au types de morph. \item Bulle Cole: Un habitat formé d'un astéroïde ou d'une lune évidée et en rotation pour obtenir une gravité. \item Machine d'Abondance: Un nanofabeur a but généraliste. \item Pile Corticalle: Une cellulle de mémoire implantée et utilisée pour sauvegarder un ego. Localisée là où l'épine dorsale rencontre le crâne; elle peut-être extraite. \item Cybercerveau: Un cerveau artificiel, hébergeant un ego. Utilisé à la fois dans les synthmorphs et dans les pods. \item Darkcast: Services de farcast et d'egocast ilégaux et trouvés au marché noir. \item Règles de Domaine: Les règles qui régissent la réalité dans un simulspace de réalité virtuelle. \item Drône: Un robot conrtrollé par téléopération (plutôt que par des IA embarquées). \item Ecto: Périphérique de mesh personnel, souple, étirable, auto-nettoyant, translucide et alimenté par énergie solaire. De ecto-lien (lien externe). \item Ego: La part de vous qui bascule d'un corps à l'autre. Aussi appelé ghost, âme, essence, esprit, persona. \item Egocaster: Terme pour envoyer un ego par farcasting. \item Entoptiques: Images de Réalité Augmentée que vous "voyez" dans votre tête. ("Entoptique" signifie "à l'intérieur des yeux") \item ETI: Intelligence extra-terrestre. Le terme utilisé par Firewall pour faire référence aux intelligence étrangères post-singularité de niveau divin théoriquement responsable du virus Exsurgent. \item Exaltés: Humains génétiquement améliorés (entre génétiquement réparés et transhumains). Aussi connus comme génomonstre, les ascendants, les élevés. \item Exoplanète: Une planète dans un autre système solaire. \item Exsurgent: Quelqu'un infecté par le virus Exsurgent \item Virus Exsurgent: Le virus multi-vecteur créé par un ETI inconnu et répandu dans la galaxie dans des sondes Bracewell. Le virus Exsurgent est mutant et peu infecter à la fois les systèmes informatique et les créatures biologiques. \item Extrasolaire: Hors du système solaire. \item Facteurs: La race étrangère ambassadrice qui fait affaire avec la transhumanité. Aussi appelés les Courtiers. \item La Chute: L'apocalypse; la singularité et les guerres qui ont presque amenées l'extinction de l'humanité. \item Farcasting: Communication intrasolaire utilisant des technologies de communication classiques (radio, laser, etc) et la téléportation quantique. Parfois appelée Hyeprdiff. \item Long Porteur: Transport spatiaux de longue distance. \item Firewall: La conspiration secrète, multifaction qui travaille à protéger la transhumanité des "risques existentiels" (risques qui menacent l'existence de la transhumanité). \item Bas de plancher: Quelqu'un qui est né ou habitué à vivre sur une planète ou une lune avec une gravité. \item Plats: Humains de base (sans modificatiosn génétique). Aussi appelés norms. \item Flexbot: Une synthmorph capable de changer de forme ou de rejoindre d'autre trasnformers afin de créer des forms plus grande et modulaires. Aussi appelé Transformers \item Forker: Copier un ego. Tous les forks ne sont pas des copies complètes. Aussi appelés sauvegardes. \item FTL: Faster-Than-Light. Plus rapide que la lumière. \item Fury: Une morph de combat transhumaine. \item Resquilleurs: Explorateurs qui tentent leur chances en utilisant une Porte de Pandorre pour aller vers un endroit pour l'instant inexploré. \item Génohacker: Quelqu'un qui manipule le code génétique pour créer des modifications génétiques voire même de nouvelles formes de vie. \item Ghost: Une morph de combat transhumaine optimisé pour la furtivisté et  l'infiltration. \item Ghost-riding: L'acte de transporter une infomorph dans un implant sépcial à l'intérieur de votre tête. \item Grecs: Astéroîdes ou lunes troyens qui partagent la même orbite qu'une planète ou lune plus grosse, mais qui ont 60 degrés d'avance sur l'orbite, au point de Lagrange L4. Le terme Grecs fait normalement référence aux astéroïdes orbitant autour du point L4 de Jupiter. Voir aussi à "Troyens." \item Habtech: Un technicien d'habitation. \item Héliopause: Le point auquel la pression des vents solaires s'équilibre avec les moyennes interstellaires (aux allentours de 100 AU). \item Hibernoïdes: Un transhumain modifié pour l'hibernation, pour des travaux prolongés dans l'espace. \item Glacetéroïde: Un astéroïde constitué essentiellemnt de glace au lieu de roche et de métal. \item Iktomi: Le nom donné à la mystérieuse race étrangère dont les reliques ont été retrouvées au delà des Portes de Pandorre. \item Contractés: Des esclaves sous contrats synallagmatique qui ont signé pour travailer avec une hypercorp ou une autre autorité, habituellement en échange d'une morph. \item Infovie: Intelligence artificielle généraliste et IAs germe. \item Infomorph: Un égo digitalisé; un corps virtuel. Aussi connus sous les noms de datamorph, uploads, sauvegardes. \item Infugié: ``Infomorph refugié,'' ou quelqu'un qui a tout abandonné sur Terre - y compris son corp - pendant la Chute. \item Isolés: Ceux qui vivent dans des communautés isolées loin au-delà des limites du système (dans la Ceinture de Kuiper et le Nuage d'Oort); aussi appelés outsters, limités. \item Saturer: l'acte de "devenir" un drône opéré à distance grâce à la technologie XP. Également utilisé pour l'accession à un flux XP en temps-réel de lifeblogeur et autre émetteurs en temps-réel. \item Ceinture de Kuiper: Une région de l'espace partant de l'orbite de Neptune et s'étalant sur envrion 55 UA, légèrement peuplée d'astéroïdes, de comètes et de planètes naines. \item Point de Lagrange: L'une des cinqs zones relative à un petit corps planétaire orbitant autour d'un plus gros dans lesquelles les forces gravitationnelles de ces deux corps sont neutralisées. Les point de Lagrange sont considérés comme stables et sont des positions idéales pour des habitats. \item Lifeblog: L'enregistrement de toute l'expérience de la vie de quelqu'un, rendue possible grâce aux capacités mémoires des ordinateurs quasi-illimitées. \item Generation Égarée: Dans une tentative de repeupler après la Chute, une génération d'enfant fut élevée en utilisant des techniques de croissance forcée. Les résultats furent désastreux: beaucoup sont morts ou devenus fous, et le reste a été stigmatisé. \item Ceinture Principale: La principale ceinture d'astéroïdes, un anneau torique orbitant entre Mars et Jupiter. \item Meme: Une idée virale. \item Mentalistes: Transhumains optimisés pour les compétences mentales et cognitive. \item Mercuriels: Les éléments conscient non-humains de la "famille" transhumaine; incluant les IAG et les animaux éveillés. \item Mesh: L'omniprésent maillage sans-fil de réseau de données. Egalement utilisé comme verbe (mesher) et comme adjectif (meshé ou nonmeshé). \item Mesh ID: La signature unique attachée à l'activité meshée de quelqu'un. \item Microgravité: Zéro-g ou environnement quasiment sans poids. \item Mist: Les nuages de données RA qui brouillent parfois votre perception et vos affichages. \item Morph: Un corps physique. Aussi appelé costume, veste, gaine, coquille, forme. \item Muse: IA d'assitant personnel. \item Nanobot: Une machine nanoscopique. \item Nano-écologie: Mouvement écologique pro-technologie. \item Essaim de nanite: Une masse de petits nanobots libérée dans un environnement. \item Neo-Aviens: Perroquets gris et corbeaux élevés. \item Néogenèse: La création d'une nouvelle forme de vie grâce aux manipulations génétiques et à la biotechnologies. \item Neo-Hominidés: Chimpanzés, gorilles et orang-outans élevés \item Néoteniques: Transhumains modifiés pour conserver une forme enfantine. \item Novacrabe: Un pod créé à partir de crabe araignés génétiquement conçus. \item Olympien: Une biomorph transhumaine modifiée pour l'athéltisme et l'endurance. \item Cylindre O'Neill: Un habitat en forme de canette, soumis à une rotation pour créer une gravité. \item Nuage d'Oort: Le "nuage" sphérique constitué de comètes qui entoure le système solaire et qui s'étend jusqu'à une année lumière du soleil. \item PAN: Personal area network/Réseau personnel. Le réseau créé lorsque vous asservissez tous vos périphériques électroniques mineur à votre ecto ou votre insert de mesh. \item Porte de Pandorre: Les portails de trou-de-ver abandonné par les TITANs. \item Pods: Des morphs à la fois biologique et synthétiques. Les clones utilisé pour créer les pods subissent une croissance forcée et possèdent un cerveau informatique. Aussi appelé bio-bots, pelure, répliquants. \item Posthumain: Un individu, humain ou un transhumain, ou une espèce qui a été génétiquement ou cognitivement modifié à un point qu'il n'est plus réellement humain (un cran au-dessus de transhumain). Aussi appelé parahumain. \item Prométhéens: Un groupe d'IAs germes pro-transhumaine créées par le Projet Canot de Sauvetage (précurseurs des argonautes) des années avant que les TITANS ne développent une conscience d'eux et qui ont (presque) évitées l'Infection Exsurgente. Les Prométhéens travaillent secrètement en soutien de Firewall et luttent contre les menaces existentielles. \item Proxys: Membres de la structure interne de Firewall. \item Psi: Pouvoirs parapsychologique développés suite à l'infection par la souche Watts-MacLeod du virus Exsurgent. \item Reapeur: Une synthmorph de combat. \item Réclamationnistes: Une faction transhumanistes qui cherche à lever l'interdiction et à récupérer la Terre. \item Redneck: Un Martien rural. Voir Barsoomien. Aussi appelés Reds. \item Reinstantiés: Réfugiés de la Terre qui se sont échappées sous la forme d'infomorph sans corps, mais qui ont depuis été réincarné. \item Se Réincarner: Changer de corps, ou être téléchargé dans un nouveau corps. Aussi appelé remorphing, regainage, basculer, renaissance. \item Rusteur: Biomorph optimisée pour la vie sur Mars. \item Scorcheur: Programme hostile qui peut endommager ou affecter un cybercerveau. \item Racaille: La faction nomade de punks/gitans de l'espace qui voyagent de stations en stations dans des barges lourdement modifiées ou dans des nuées de vaisseaux. Connus pour être des marchés noirs errants. \item IA germe: Une IAG capable d'auto-apprentissage récursif, lui permettant d'atteindre des niveaux d'intelligences similaire à ceux des dieux. \item Sentinelles: Agents de Firewall \item Coquille: Une morph physique synthétique. Aussi appellé synthmorph. \item Simulmorph: L'avatar que vous utilisez dans les simulspace RV. \item Simulspace: Environnement de réalité virtuelle permettant une immersion sensorielle complète. \item Singularité: Un point de progrés technologique rapide, exponentiel et récursif, au-delà duquel le futur devient impossible à prévoir. Souvent utilisé pour faire référence à l'ascenssion des IAs germe à des niveaux d'intelligence divins. \item Adepte de la Singularité: Des personnes qui cherchent des reliques et des preuves que les TITANs ou d'autres super-intelligences, soit pour en apprendre plus sur eux ou pour devenir une super-intelligence. \item Peau: Une morph physique biologique. Aussi appelé viande, chair. \item Habiller: Modifier son environnement perçu par la réalité augmenté grâce à des programmes. \item Exploit Psi: Un pouvoir psi. \item Slitheroïde: Une synthmorph robotique en forme de serpent. \item Animaux Intelligents: Espèces animales partiellement élevées (incluant chiens, chats, rats et cochons). D'autres gros animaux intelligents (baleines, éléphants) sont au bord de l'extinction. \item Spimes: Périphériques meshé, conscient et localisés. \item Spliceurs: Humains qui sont génétiquement modifiés pour éliminer les maladies génétiques et quelques autres aspects. Aussi connu comme génofixé, génolavés, bidouillés. \item Swarmanoïde: Une morph synthétique composé d'un essaim de robots de la taille d'un insecte. \item Sylphes: Biomorph transhumaine d'une exotique beauté  \item Synthmorph: Morphs syntéhtiques. Coquilles robotiques possédant des égos transhumains. \item Synths: Un type spécifique de synthmorph. Les synths sont des androïdes/gynoïdes classiques; des robots conçus pour être humanoïde, bien qu'ils soit facile de remarquer qu'ils ne sont pas humains. \item Téléopération: Contrôle à distance. \item Titanien: Quelqu'un qui vient de Titan, l'une des lunes de Saturne. \item TITANs: Les IAs germes créées par l'homme, capable d'apprentissage récursifs qui ont subit un décollage abrupte de la singularité et qui ont déclenchés la Chute. La désignation militaire originelle était TITAN: Total Information Tactical Awareness Network (Réseau Cognitif d'Information Tactique Complète). \item Tore: Un habitat en forme de donut, soumit à une rotation pour générer de la gravité. \item Transgénique: Qui contient des traits génétiques d'autres espèces. \item Transhumain: Un humain largement modifié. \item Troyens: Astéroïdes ou lunes qui partagent la même orbite qu'une autre planète ou lune, mais qui la suit avec 60° de décalage, à l'avant ou à l'arrière au poinst de Lagrange L4 ou L5. Le terme de Troyens fait noramellement référence aux astéroïdes orbitant aux point de Lagrange de Jupiter, mais Mars, Saturne, Neptune et d'autres corps ont aussi des Troyens. Voir aussi "Grecs." \item Élever: Élever à la conscience un animal en le transformant génétiquement. \item Travailleur du Vide: Ouvrier de l'espace. \item Vapor: Une émulation cognitive ratée ou un fork/une infomorph criblé de défaut (dérivé de vaporware). \item VPNs: Virtual private networks/Réseaux Privés Virtuels Des réseaux transitant à travers le mesh, habituellement chiffrés pour la protection de la vie privé et pour la sécurité. \item RV: Réalité Virtuelle. Imposer une réalité hyper-réaliste construite artificiellement par-dessus les sens physique de quelqu'un. \item X-Diffeur: Quelqu'un qui transmet et vends des enregistrement XP de leurs propres expérience (dérivé de X-Diffuseurs). \item Xénomorph: Forme de vie étrangères. \item Xé: Comme dans "X-é" - quelqu'un qui est accros ou obsédé par les XP. Fait parfois également référence aux personnes qui font de l'XP. \item XP: Experience Playback/Lecture d'expérience. Faire l'expérience des entrées sensorielle de quelqu'un d'autre (en temps réel ou après enregistrement). Aussi appelé experia, sim, simsense, playback. \item Risque X: Risque existentiel. Quelque chose qui menace l'existence même de la transhumanité. \item Zéros: Personnes sans accès sans-fil au mesh. Commun chez certains contractés. \end{itemize} 

\begin{quotation} 

\textbf{Bienvenue à Firewall} 

[Message Entrant Reçus. Source: Inconnue] 

[Analyse Quantique: Pas d'interception Détectée] 

[Déchiffrement Complet] 

Salutations, 

Vos références et votre histoire ont été vérifiées trois fois et confirmées, et vous êtes maintenant validé en tant que processus sentinelle. Bienvenue à Firewall, l'ami. 

Pour ceux qui arrivent juste dans notre réseau privé, Firewall est une organisation dévouée à la protection de la transhumanité des menaces - à la fois internes et externes - et à la persistence de notre espèce. La Chute nous a peut-être rappelé que notre capacité à survivre et prospérer n'était pas garantie, mais les notres ont un spectre d'attention remarquablement réduit. En dépit de notre réalisation d'une quasi-immortalité fonctionnelle, nous continuons de à faire face à de nombreux dangers qui pourraient contribuer à notre extinction. Certains de ces risques viennent de notre propre factionnalisme et de nos divisions, combiné à de la technologie universellement disponible qui pourrait causer une destruction étenduée ou des décès indicible si elles tombaient dans les mauvaises mains. Certains viennent de notre manque de vision à long terme, incapables de voir les dangers dans lesquels nous nous sommes plongés entraînant notre nevironnement à cause d'actions imprudentes. D'autres proviennent de nos prorpres créations qui se sont retournées contre nous, comme les TITANs l'ont prouvé. D'autres risques peuvent venir d'intelligence étrangère aux motivations que nous ne pouvons pas encore deviner, et dont nous pourrions ne jamais avoir conscience. D'autres enfin pourraient nous menacer par pur hasard et la plus stupide, mais néanmoins meurtrière, causalité d'un univers dans lequel nous ne sommes rien d'autre que d'insignifiantes poussières. 

Firewall existe pour identifier, analyser et contrer ces risques. Nous sommes tous volontaires. Nous mettons tous nos vies en danger afin d'assurer la survie de la tranhsumanité. 

Firewall a existé, sous des noms et des formes différentes, bien avant la Chute. De nombreuses agences avec des plans similaires se sont regroupé à l'aube de ces évènements cataclysmique pour faire un point sur notre situation et nous préparer au pire. Maintenant, nous opérons sous une seule bannière. 

Nous sommes un réseau privé pour deux raisons. Premièrement, notre existence et nos capacités opératoires sont protégées par notre secret. Moins notre opposition sait de choses sur nous, plus nous pouvons les contrer de manière efficace. De manière similaire, certaines autorités pourraient être hostiles à une organisation telle que la notre opérant dans les territoires qu'elles proclament comme les leurs. Bien que certains d'entre eux doivent être au courant de notre existence, nous passons outre de nombreux obstacles juridiques et légaux qui pourraient sinon entraver nos actions et nos objectifs. Deuxièmement, il arrive que notre mission révèle des informations qui ne sont pas seulement dangereuses dans les mauvaises mains, mais qui pourraient en plus déclencher une panique généralisée si elles étaient rendues publique. Dans certains cas, l'existence même d'une telle connaissance peut-être problématique. En conservant ces secrets et en opérant sur lprincipe que vous savez ce que vous avez besoin de savoir, nous controns automatiquement certains risques. 

Firewall est un réseau décentralisé, de pair à pair. Nous avons une hiérarchie minimale et nous ne répondons à personne d'autre qu'à nous-même. Notre structure nodale nous permets de partager des ressources et des talents sans sacrifier la sécurité et la vie privée de nos agents de terrain. Vous avez été recrutés à cause de vos connaissances, possession ou compétences, et/ou parceque vous êtes entrés en contact avec certaines données d'accès restreint. Vous avez prouvé votre volonté à défendre nos objectifs. Nos vies et nos existences - et le futur de la transhumanité - peuvent reposer entre vos mains. 

Voici donc le futur - que nous puissions tous survivre pour le voir. 

[Fin du Message] 

[Ce message s'est auto-supprimé] 

\end{quotation} 

\begin{quotation} 

\textbf{Ce que vous avez réellement besoin de savoir} 

[Message Entrant Reçus. Source: Inconnue] 

[Analyse Quantique: Pas d'interception Détectée] 

[Déchiffrement Complet] 

Assieds toi, et prends toi un putain de verre. 

Oublies toute cette intro merdique générée par des IA que tu viens de lire. Voici la vraie affaire. 

Tu meurs sans doute d'impatience pour savoir ce dans quoi tu t'es fourré. On t'as peut-être déjà donné la ligne du parti; que nous sommes tout ce qui sépare la transhumanité de l'extinction. Ou peut-être que quelqu'un t'as murmuré que nous somme sune opération clandestine qui se mèle de sacré merdiers dans lesquels nous ne sommes pas censés intervenir, et que parfois nous faisons tuer des gens. Tu dois être curieux. Peut-être que tu as une envie de faire justice toi-même, et que tu cherches à faire couler du sang pour une bonne cause. Est-ce que ça aurai une importance pour toi que cette cause ne soit qu'une illusion? Tu es peu-être un adepte des complots et tu meurs d'envie de savoir quels secrets Firewall serre sur sa poitrine collective. Et si ces secrets éparpillés était consciencisuement assemblés en mensonges que nous nous racontons à nous-même pour préserver notre santé mentale? 

Tout ce que tu as entendu, de bon ou de mauvais, à propos de Firewall pourrait bien être vrai. Nous ne sommes pas des anges. Nous avons perdu la clarté de nos idées lorsque les TITANs ont forcé l'upload de leur premier esprit humain. En ce moment, tu dois te demander dans quel bordel tu t'es engagé. Je l'ai fait. 

Firewall est tout un tas de choses. La plupart d'entre elles sont bonnes, mais pas mal d'entre elles sont tellement monstrueuse que tu préfèreras te tirer une balle dans la pile et retourner à un précédent backup, just pour pouvoir l'oublier. Si tu avais des visions romantiques à propos de devenir un héros, oublies-les. Maintenant. Tu ne te sentiras pas héroïque lorsque tu balanceras dans le vide un gamin parcequ'il est infecté par un nanovirus. Tu ne te sentiras pas courageux lorsque tu courras à travers un truc étrange et que tu te chieras dessus. Et tu ne te sentiras même plus humain lorsque tu passeras un coup de fil qui coutera la vie à des douzaines, des centaines, ou même des milliers de gens, même si tu en sauves des millions d'autres. 

Donc. Pourquoi quelqu'un serait suffisament cinglé pour faire partie de ça? Parceque le boulot doit être fait. Notre survie en dépend. Pour certains, c'est de l'altruisme, la défense de la transhumanité. Mais en fait, il s'agît surtout de sauver ta putain de tête. Bien sûr, tu pourrais t'abstenir de prendre des responsabilité et laisser une quelconque autorité auto-proclamée s'en occuper. Mais si les anarchistes ont bien compris quelque chose, c'est que l'on ne peut pas faire confiance aux personnes qui ont le pouvoir. Ils sont, plus souvent qu'à leur tour, une partie du problème. Donc, Firewall fait les choses de manière collective. Nous sommes undergound, mais nous sommes une organisation open source. Nous partageons informations et ressources pour atteindre un but commun. Nous sommes organisés en un réseau de cellulle ad-hoc, comme une foule intelligente. Nous ne laissons personne acquérir trop de pouvoir ou de contrôle. Toutes les personnes impliquées dans une opération ont leur mot à dire. Nous nous surveillons nous-même. Nous venons de tout type d'origine et de factions, mais nous faisons face à un ennemi commun - et nous nous battons pour gagner. Il n'y a pas d'alternative. 

Tu as peut-être entendu parler du paradoxe de Fermi? La question était: pourquoi, avec une galaxie si grande, il y avait tellement peu de signe d'une autre vie? Même si nous avons rencontrés les Facteurs et trouvé des preuves d'autres étrangers, notre voisinnage galactique devrait grouiller d'intelligence - mais ce n'est pas le cas. 

Je vais te dire pourquoi. Ce putain d'univers n'est pas juste. Si la transhumanité avait été rayée de la carte, la galaxie ne l'aurai même pas remarqué. Regarde la Terre par exemple. Cette planète existe encore, acceuille toujours la vie, même si nous sommes partis depuis longtemps. La réalité est une putain indifférente. Oublies toute cette merde utopiste à propos de la vie éternelle. On sera chanceux si on survit une année de plus. Nous avons développés des technologies qui mettent les armes de destructions massives entre les mains de n'importe qui, mais nous sommes toujours une espèce adolescente incapable de passer outre les petites conneries tribales. Si tu veux réellement aller de l'avant et explorer l'univers en tant que postmortel, tu vas devoir y travailler trés dur. La survie n'est pas un droit, c'est un privilège. 

Quand tu t'engages à Firewall, tu te mets à dispo. A chaque fois qu'une merde sort des bois pour te tomber dessus ou que tu pourrais être particulièrement utile pour gérer le bordel, tu reçois un appel. On s'attend à ce que tu abandonnes tout ce que tu es en train de faire et de mettre tout le reste en attente comme si ta vie en dépendait - et c'est probablement le cas. Quand tu seras sur le terrain, pour une opé - on appelles ça "aller au docteur" - ta cellulle aura tout pouvoir pour agir comme elle le juge bon ... garde juste en tête que tu devras répondre à nos question plus tard. Tu as aussi le réseau de Firewall qui te couvre - même si les ressources sont souvent limitées, ne t'attends donc pas à ce qu'on sauve ton cul à chaque fois. D'autres sentinelles peuvent être appellées pour tirer quelques ficelles, mais à chaque fois qu'on le fait, cela menace de révéler un agent, créant un bourbier que nous devons nettoyer, et compliquant les choses de toutes manière. L'autonomie est la clef. 

Une dernière chose: ne jamais, jamais oublier que nous avons des ennemis. je ne parles pas simplement de la tête de nœud qui veut utiliser une tête nuke sur un habitat pour faire une revendication politique ou de ces néo-luddites qui pensent que les fléaux de la guerre biologique nous apprendront une leçon, je parle des agences qui connaissent l'existence de Firewall et qui le considèrent comme une menace. Si ils t'étiquettent comme sentinelle, tes jours sont comptés. Probablement ceux de tes backups aussi. Donc, surveilles tes putains d'arrières. 

Voilà le vrai bazar, aussi honnètement que je peux te le donner. Bienvenue dans notre club-house secrète, camarade. Rappelles-toi: la mort est juste la routine du boulot. 

[Fin du Message] 

[Ce message s'est auto-supprimé] \end{quotation} 



