\chapter{Mécanismes de jeu} \label{chap:game-mechanics} 

Dans tous les jeux, il vient un moment où le maître de jeu doit décider si un personnage réussi ou rate une action. C'est le moment où les joueurs lancent des dés et où les stats du personnage entrent en jeu. Ce chapitre définit les mécanismes au cœur des règles qui gouvernent l'issue des évènements dans Eclipse Phase. 

\section{Une note sur la terminologie, la traduction et le Genre.} Le cadre d'Eclipse Phase soulève nombre de questions intéressante à propos du genre et de l'identité personnelle. Que signifie le fait d'être née femelle lorsque vous occupez un corps masculin? Lorsqu'on en vient au langage et à l'édition, cela pose aussi nombre de questions intéressantes sur l'usage des pronoms. Malheureusement, la langue Française ne possédant pas de genre neutre (contrairement au it Anglais), il nous sera donc difficile d'éviter le biais linguistique en faveur du masculin dans cette traduction.   Nous parlerons donc de joueurs, de maître de jeu, et de personnages avec des pronoms neutres dans la mesure du possible (équivalent du "on") en évitant de préciser les pronom (qu'il s'agisse de "il" ou de "elle"). Cependant, et afin de ne pas rendre le tout particulièrement lourd et indigeste, le masculin restera employé lorsque la grammaire l'exige (et il s'agira donc d'un joueur, d'un maître de jeu et d'un personnage), le masculin, à défaut d'être le genre neutre, étant le genre "par défaut" de la grammaire française.   



\section{Règles de bases} \label{sec:basics} 

\subsection{La règle ultime} \label{sec:ultimate-rule} 

Une règle dans Eclipse Phase surclasse toutes les autres: éclatez-vous. Cela signifie que vous ne devriez jamais laisser les règles se mettre en travers du jeu. Si vous n'aimez pas une règle, changez-la. Si vous ne trouvez pas une règle, créez en une. Si vous n'êtes pas d'accord avec l'interpétation d'une règle, tirez à pile-ou-face. Essayez de ne pas laisser les règles interférer avec la fluidité et l'ambiance du jeu. Si vous êtes au milieu d'une trés bonne scène ou d'un moment de roleplay intense et qu'une règle pose soudainement un problème, n'arrétez pas le jeu pour la vérifier ou pour en discuter. Improvisez, prenez une décision rapide, et continuez. Vous pourez toujours revérifier la règle plus tard afin de vous en rappeler la prochaine fois. Si il y a des désaccords autour de l'interprétation d'une règle, rappelez-vous que le maître de jeu a le dernier mot. 

Cette règle signifie aussi que vous ne devriez pas laissez l'histoire être guidée uniquement par des jets de dés. L'aléas d'un jet de dé amène un sentiment d'aléatoire, d'incertitude et de surprise à une partie. Parfois c'est excitant, comem par exemple lorsqu'un personnage réussit un jet contre toute probabilité et, du coup, sauve l'équipe. A d'autres moment, c'est brutal comme, par exemple, un tir chanceux d'un adversaire tue un personnage pour de bon lors d'un combat. Si le maître de jeu veut qu'un scénario se termine par une issue dramatique planifiée et qu'un jet de dé inattendu menace son plan, il devrait se sentir libre d'ignorer ce jet et de continuer l'histoire dans la direction qu'il désire. 

\subsection{Dés} \label{sec:dice-1} 

Eclipse Phase utilise deux dés à dix faces (d10) pour les jets aléatoires. Dans la plupart des cas, les règles demanderont un jet de pourcentage, noté d100, ce qui signifie que vous lancez deux dés à dix faces, en choisissant lequel sera lu en premier, et en lisant le résultat comme compris entre 00 et 99 (avec un résultat de 00 comptant comme zéro, non pas comme 100). Le premier dé compte pour les dizaines, le second pour les unités. Par exemple, vous lancez deux dés à dix faces, un rouge et un noir, annonçant que le rouge sera lu en premier. Le rouge fait un 1 et le noir fait un 6, pour un résultat de 16. Certains ensembles de d10 sont spécifiquement marqués pour faciliter le lancement des dés et la lecture des résultats. 

De manière occasionnelle, les règles feront appel à des jets de dés seuls, chaque dés à dix face listés en temps que d10. SI les règles demandent que plusieurs d10 soient lancés, ils seront notés 2d10, 3d10 et ainsi de suite. Lorsque plusieurs dés à dix faces sont lancés de cette manière, leurs résultats sont ajoutés les uns aux autres. Par exemple un jet de 3d10 ayant pour résultat 4, 6 et 7 compte pour un 17. Sur les jets de d10, un résultat de 0 et traité comme un 10, pas comme un zéro. 

La plupart des joueurs d'Eclipse Phase s'en sortent en ayant seulement deux dés à dix-faces, mais ça ne fait pas de mal d'en avoir d'autres à portée de main. Ces dés peuvent être achetés dans votre boutique de jeu préférée ou empruntés à d'autres joueurs. 

\subsection{Faire des test} \label{sec:making-tests} 

Dans Eclipse Phase, votre personnage est destiné à se trouver impliqué dans des scènes d'actions adrénalisante, dans des situatsions sociales hyper-stressante, dans des combats létaux, dans des enquètes frissonantes et d'autres situations simillaires emplies de drame, de risque et d'aventures. Lorsque votre personnage est impliqué dans ces scénarios, vous déterminez la manière dont il s'en sort en faisant des tests - en lançant des dés pour déterminer si ils réussissent ou échouent, et dans quelle mesure. 

Vous faites des tests dans Eclipse Phase en lançant un d100 et en comparant le résultat à un seuil. Le seuil est typiquement déterminé par l'une des compétences de votre personnage (voir plus bas) et est compris entre 1 et 98. Si vous obtenez un résultat inférieur ou égal au seuil, vous réussissez votre test. Si vous dépassez le seuil, vous échouez à votre test. 

Un résultat de 00 est toujours considéré comme un succès. Un résultat de 99 est toujours considéré comme un échec. 

\begin{quotation} Le personnage de Jacqui doit faire un test de compétence. Sa compétence est de 55. Jacqui prends deux dés à dix faces et obtient un 53 - elle réussit! Si elle avait obtenu un 55, elle aurait également réussit, mais tout résultat supérieur aurait été un échec. \end{quotation} 

\subsection{Seuil} \label{sec:target-numbers} 

Comme noté précédement, le seuil pour le jet d'un d100 dans Eclipse Phase est généralement la valeur d'une compétence. Cependant et de manière occasionnelle, un nombre différent sera utilisé. Dans certains cas, un score d'aptitude sera utilisé, ce qui rend les test plus difficile car les aptitudes sont généralement bien en dessous de 50 (voir le passage Aptitudes, \ref{sec:aptitudes}). Dans d'autres test, la cible sera un score d'aptitude x2 ou x3 ou la somme de deux Aptitudes. Dans ces cas là, la description du test indiquera quel(s) score(s) utiliser. 

\subsection{Quand faire des tetst} \label{sec:when-make-tests} 

Le maître de jeu décide quand un personnage doit faire un test. Comme règle de base, les tests doivent être tentés lorsqu'il y a une chance qu'un personnage échoue une action ou lorsque la réussite ou l'échec de l'action peut avoir un impact sur l'histoire en cours. Les tests sont aussi nécessaires lorsque deux personnage ou plus agissent en opposition les uns aux autres (par exemple, si ils s'affrontent au bras de fer ou si ils négocient un prix). D'un autre côté, les utilisations routinières d'une compétence avec un score d'au moins 30 peuvent être considérées comme des réussites sans faire de test. 

Il n'est pas nécessaire de faire de jets de dés pour les actiosn de la vie quotidienne tels que s'habiller ou vérifier ses mails (particulièrement dans Eclipse Phase, où tant de choses sont gérées automatiquement par les machines autour de vous). Même une activité telle que la conduite automobile ne nécessite pas de jets de dés tant que vous avez un minimum dans la compétence. Un test peut cependant être nécessaire si vous conduisez pendant que vous vous videz de votre sang ou que vous poursuivez un gang de charognards à moto à travers les ruines d'une cité dévastée. 

Savoir quand faire appel à des jets et quand laisser l'interprétation et le roleplay se dérouler sans interruption est une compétence que chaque maître de jeu doit acquérir. Parfois, il peut-être plus simple de décider l'issue du jet arbitrairement sans lancer de dés afin de maintenir le rythme de la partie. De la même manière, dans certaines circonstances, le maître de jeu peut décider de faire des tests pour un personnage en secret, sans que le joueur ne le remarque. Si un ennemi essaye de s'infiltrer sous la vigilance d'un personnage, par example, le maître de jeu signalera au joueur que quelquechose va de travers si il lui de faire un test de perception. Cela signifie que le maïtre de jeu devrait garder en permanence une copie de chaque fiche de personnage à portée de main. 

\subsection{Difficulté et Modificateurs} \label{sec:difficulty-modifiers} 

L'évaluation de la difficulté d'un test est reflétée par ses modificateurs. Les modificateurs sont des ajustements apportés au seuil du test (pas au jet en lui-même), soit en l'augmentant, soit en le diminuant. Un test de difficulté moyenne n'aura pas de modificateur, alors que les actions plus simple auront des modificateurs positifs (augmentant le seuil de réussite et donc les chances de succès) et que les actions plus difficiles auront des modificateurs négatifs (abaissant le seuil de réussite et donc les chances de succès). C'est le boulot du maître de jeu de déterminer si un test particulier est plus difficile ou plus simple que la norme et dans quelle mesure (tel qu'illustré dans la table des Difficultés des Tests) et d'appliquer ensuite les modificateurs appropriés. 

D'autres facteurs peuvent aussi jouer un rôle dans un test, appliquant des modificateurs additionels au delà du niveau de difficulté général du test. Ces facteurs incluent l'environnement, l'équipement (ou l'absence d'équipement) ainsi que la santé du personnage parmi d'autres. Le personnage peut utiliser des outils de qualité supérieure, travailler dans des conditions lamentables ou être blessé, et chacun de ces facteurs devrait être pris en compte, appliquant des modificateurs supplémentaires au seuil et modifiant la probabilité de réussir ou de rater le jet. 

Dans un but de simplification, les modificateurs sont appliqués par multiples de 10 et viennent en trois niveaux d'intensité: Mineur (+/-10), Modéré (+/-20) et Majeur (+/-30). Tant que le maître de jeu le pense approprié, n'importe quel nombre de modificateur peuvent être appliqués, mais la valeur cumulée de ces modificateurs ne peut pas excéder + ou - 60. 

\begin{quotation} Jacqui tente d'aller d'une porte à une autre à travers une grande pièce en gravité zéro. Elle est pressée. Si elle rate la porte, elle perdra un temps précieux, donc le maître de jeu demande un Test de Compétence en Chute Libre. La compétence Chute Libre de Jaqui est de 46. Malheureusement, la pièce est emplie de débris flottant qui pourrait la géner dans ses déplacements. Le maître de jeu détermine qu'il s'agît d'un modificateur Modéré, réduisant le seuil de 20. Jacqui doit obtenir 26 ou moins pour réussir son jet. \end{quotation} 

\subsection{Critiques: obtenir des doubles} \label{sec:crit-roll-doubl} 

A chaque fois que les deux dés tombent sur le même résultat - 00, 11, 22, 33, 44, etc - vous obtenez un succès critique ou un échec critique, selon que vous ayez battus ou non le seuil de difficulté. 00 est toujour sun succès critique alors que 99 est toujours un échec critique. Obtenir des doubles signifie qu'un petit extra se produit en plus de l'issue du test, qu'il soit positif ou négatif. Les critiques ont une applications trés spécifiques sur les test de Combat (voir \ref{sec:combat}), mais dans tous les autres cas le maître de jeu décide ce qui s'est mal ou bien passé dans une situation particulière. Les critiques peuvent être utilisés pour amplifier un succès ou un échec: vous pouvez terminer avec un bonus ou rater de manière tellement spectaculaire que vous serez la cible des moqueries pendant les semaines à venir. Ils peuvent aussi amener une sorte d'effet secondaire inattendu: vous réparez l'appareil et améliorez ses performances; ou vous pouvez échouer à toucher votre ennemi et blesser un passant innocent à la place. De manière alternative, un critique peut-être utilisé pour donner un boost (ou une gène) sur une action à suivre. Par exemple, vous ne faites pas que remarquer un indice, vous supectez immédiatement qu'il s'agît d'une fausse piste; ou vous ne faites pas que ratez votre cible, mais votre arme casse vous laissant sans défense. Les maîtres de jeu sont encouragés à être inventifs dans leur utilisation des critiques et de choisir les résultats qui créent des situations comiques, dramatiques ou de tendues. 

\begin{quotation} Audrey tente d'intimider une petite frappe des triades afin d'en obtenir des informations. Malheureusement, elle obtient un 99 - un échec critique. En plus de ne pas parvenir à effrayer le mec, elle laisse filtrer une information importante qu'elle ne voulait pas que les triades obtiennent. Si elle avait obtenu un 00 - un succès critique - elle aurait fait tellement peur au caïd qu'il lui aurait balancé des informations supplémentaires importante juste pour qu'elle le laisse tranquille par la suite. \end{quotation} 

\subsection{Défausser: utilisation de compétence non-entraînée} \label{sec:defa-untr-skill} 

Certain tests peuvent demander à un personnage d'utiliser une compétence qu'il ne possède pas - ce qu'on appelle défausser. Dans ce cas, le personnage utilise le score de l'aptitude (voir p. 123) liée à la compétence en question comme seuil de réussite. 

Toutes les compétences ne peuvent pas être défaussée; certaines sont tellement complexe ou nécessite un entrainement tel qu'un personnage ne la possédant pas n'a aucune chance de réussir. Les compétences qui ne peuvent pas être défaussée sont notées dans la Liste de Compétence (p. 176) et dans la description de celles-ci. 

dans de rares cas, le maître de jeu peu autoriser un personnage à se défausser sur une compétence qui pourrait également être liée au test (voir p. 173). Lorsque c'est autorisé, défausser sur une autre compétence inclut un modificateur de -30. 

\begin{quotation} Toljek essayes de s'inflitrer dans le bâtiment d'une hypercop lorsqu'il percute accidentellement une employée hypercoproratiste. La femme qu'il croise n'a pas forcément de raison d'être suspicieuse vis à vis de la présence de Toljek, mais le maître de jeu demande à Toljek de réussir un Test de Protocole pour se faire passer pour quelqu'un qui est à sa place ici. Malheureusement, Toljek n'as pas cette compétence, il doit donc défausser sur l'aptitude liée, Astuce, à la place. Son score d'Astuce est de seulement 18, Toljek espère donc être chanceux. \end{quotation} 

\subsection{Simplifier les modificateurs} \label{sec:simpl-modif} 

Au lieu de chercher et d'accumuler une longue liste de modificateurs pour chaque action et de calculer le modificateur total, le maître de jeu peu choisir de simplement "évaluer" la situation et d'appliquer le modificateur qui résume le mieux l'effet voulu. Cette méthode est plus rapide et permet des résolutions de tests plus rapide Une manière d'évaluer la situation est de choisir le modificateur le plus sévère qui affecte la situation. 

\begin{quotation} Tyska essaye d'échapper à quelque chose qui la poursuit à travers un habitat abandonné. Le maître de jeu demande un Test de Parkour, mais il y a beaucoup de modificateurs situationnel: il fait sombre, il court avec une lampe de poche et il y a des débris un peu partout. Tyska, cependant, a une carte entoptique du meilleur trajet possible pour le sortir de là. Le maître de jeu estime la situation et décide que l'effet global se résout par un test stimulant, et un modificateur de -20 est appliqué. \end{quotation} 

\subsubsection{Modificateurs narratifs} \label{sec:narrative-modifiers} 

Si vous voulez développer une ambiance plus cinématique à vos parties, ou si vous voulez simplement encourager vos joueurs à s'investir plus dans les détails et la création narrative, vous pouvez donner des "modificateurs narratifs" aux tests des personnages lorsqu'un joueur décrit les actions de son personnage de manière exceptionnellement colorée, inventive ou dramatique. Meilleurs sont les détails ajoutés, meilleurs sera le modificateur. 

\begin{quotation} Cole ne veux pas que son personnage saute simplement par-dessus la table, il veut créer un minimum d'impact narratif. Cole annonce au maître de jeu que son personnage balance un coup de pied dans une chaise qui va valser plus loin, roule sur son épaule à travers la table à manger, attrape une fourchette dans le mouvement, s'assure d'envoyer au sol toute la porcelaine puis atterit dans une posture martiale défensive, fourchette prête à frapper. Le maître de jeu décide que cette description supplémentaire vaut bien un modificateur de +10 à son jet de Parkour. \end{quotation} 

\subsection{Travail d'équipe} \label{sec:teamwork} 

Si deux personnage ou plus se regroupent pour s'attaquer à un test ensemble, l'un des personnage doit être désigné comme l'acteur principal. Ce personnage sera habituellement (mais pas forcément automatiquement) celui qui a le plus haut score dans une compétence applicable. Le personnage principal est celui qui lancera les dés, bien qu'il reçoive un modificateur de +10 pour chaque personnage supplémentaire qui l'assite, jusqu'à un maximum de +30. Notez que les personnage assistants n'ont pas nécessairement besoin de connaître la compétence utilisé si le maître de jeu décide qu'ils peuvent suivre et comprendre les ordres du personnage dirigeant l'action. 

\begin{quotation} La jambe robotisée sur la synthmorph d'Eva a été salement abîmée, au point qu'elle ait besoin de la réparer. Max et Vic se posent et lui donnent un coup de main, lui donnant un modificateur de +20 (+10 pour chaque assistant) à son Test de Matériel: Robotique. \end{quotation} 

\section{Type de tests.} \label{sec:types-tests} 

Il y a deux types de test dans Eclipse Phase: Les tests de Résussite  et les Tests Opposés. 

\subsection{Tests de Réussite} \label{sec:success-tests} 

On demande un Test de Réussite lorsqu'un personnage agît sans opposition directe. Ce sont les tests standard utilisés pour déterminer comment un personnage utilise une compétence ou une aptitude particulière. 

Les Tests de Réussite sont gérés exactement tels que décrit dans la section Réaliser des Tests, p. 115. Le joueur lance un d100 contre un seuil égal à la compétence du personnage +/- les modificateurs. S'il obtient un résultat inférieur ou égal au seuil, le tests est une réussite et l'action se déroule comme désiré. S'il obtient un résultat supérieur au seuil, le test échoue. 

\subsubsection{Essayer encore} \label{sec:trying-again} 

Si vous ratez un test, vous pouvez le tenter votre chance une fois de plus. Chaque tentative de réussir une action après un échec subira cependant un modificateur cumulatif de -10. Ce qui signifie que le deuxième essai souffre d'un -10, le troisième d'un -20, le quatrième d'un -30 et ainsi de suite jusqu'au maximum de -60. 

\subsubsection{Prendre le temps} \label{sec:taking-time} 

La plupart des tests concernent les Actions Automatique, Rapide ou Complexes (voir pp. 119–120) et sont donc réalisées en un Tour d'Action (3 secondes, voir p. 119). Les tests faits pour des Actions de Tâches (p. 120) prennent plus de temps. 

Les joueurs peuvent choisir de prendre du temps supplémentaires lorsque leur personnage réalise une action, ce qui signifie qu'ils choisissent d'être particulièrement méticuleux lorsqu'ils accomplissent l'action afin d'améliorer leurs chances de succès. Pour chaque minute de temps supplémentaire, ils augmentent leur seuil de +10. Une fois qu'ils ont modifié leur seuil au-delà de 99, ils sont pratiquement sûr de réussir, le maître de jeu peut donc passer outre le jet de dé et leur accorder un succès automatique. Notez que la règle du modificateur maximal de +60 s'applique toujours, et que, par conséquent, si leur compétence est inférieure à 40, prendre le temps peut ne pas garantir un succès automatique. Vous pouvez prendre votre temps même lorsque vous défaussez (voir Défausser p. 116). 

Prendre du temps supplémentaires est une bonne décision lorsque le temps n'est pas un facteur décisif, car il élimine les chances d'échec critique et permet au joueur de passer outre des jets de dés inutiles. Cependant, cela est inapproprié pour certains tests si le maître de jeu décide qu'aucune quantité de temps supplémentaire n'augmentera les chances de réussite. Dans ce cas, le maître de jeu décide simplement que prendre son temps n'a aucun effet. 

Pour les test d'Actions de Tâches (p. 120), qui prennent déjà du temps pour réussir, la durée de cette tâche peut être augmentée de 50 pourcent pour chaque modificateur de +10 gagné par du temps supplémentaire. 

\begin{quotation} Srit est en train de fouiller un vaisseau abandonné, à la recherche d'un signe de ce qui a pu arriver à l'équipage manquant. Le maître de jeu lieu lui annonce que cela prendra vingt minutes pour fouiller tout le vaisseau. Cependant, elle veut être absolument sûre, elle prend donc trente minutes supplémentaires pour fouiller. Cinquante pourcent de l'intervalle originel correspondent à dix minutes, donc en prendre trente de plus signifie que Srit reçoit un modificateur de +30 à son Test d'Investigation. \end{quotation} 

\subsubsection{Tests de Réussite simple} \label{sec:simple-success-tests} 

Dans certaines situations, le maître de jeu peut ne pas être interessé par le fait qu'un personnage puisse échouer à un test, mais simplement vouloir mesurer à quel point celui-ci réussit. Dans ces cas là, le maître de jeu demande un Test de Réussite simple, qui est géré de la même manière qu'un Test de Réussite (p. 117). Au lieu de déterminer la réussite ou l'échec, on part du principe que le test est une réussite. Le résultat détermine si le personnage obtient une réussite brillante (résultat inférieur ou égal au seuil) ou une réussite de justesse (résultat au-delà du seuil). 

\begin{quotation} Dav prend un raccourci dans le vide entre deux vaisseaux stationnés là. Le maître de jeu détermine que c'est une opération de routine et demande à Dave de faire un Tests de Réussite Simple en utilisant sa compétence Chute libre. La compétence de Dav est de seulement 35. Il obtient un 76, et le maître de jeu détermine que Dav a quelques problèmes d'orientation et y passera un peu plus de temps que prévu. Si Dav avait obtenu un 77 - un échec critique - les propulseurs de sa combinaison auraient pu tomber en panne et le propulser accidentellement dans le vide spatial. \end{quotation} 

\subsubsection{Marge de réussite/d'échec} \label{sec:marg-succ} 

Par moment, il est important qu'un personnage aille au-delà de la réussite, il doit réussir avec panache et style. C'est généralement vrai dans les situations où le défi n'est pas uniquement de battre une difficulté mais aussi de s'en sortir avec finesse. Les tests de ce type peuvent nécessiter un certain degré de Marge de Réussite (MdR) - un écart sous le seuil que le joueur doit atteindre. Par exemple, un personnage faisant face à un seuil de 55 et une MdR de 20 doit obtenir un résultat inférieur ou égal à 35 pour atteindre le degré de réussite nécessaire pour la situation. 

\begin{quotation} Un ennemi a lancé un objet incendiaire non loin de Stoya. Elle n'a qu'un instant pour agir et décide de le dégager loin d'elle. Encore mieux, elle espère l'envoyer droit dans la porte ouverte du sas de maintenance à une douzaine de mètre de là. Le maître de jeu détermine que, afin de shooter l'objet dans le sas, Stoya doit obtenir une MdR de 30. Sa compétence de Combat À Mains Nues est de 66, Stoya doit donc obtenir 66 ou moins pour envoyer l'objet au loin (bien qu'elle puisse toujours subir des dégats lorsqu'il explosera) et 36 ou moins pour l'envoyer dans le sas (auquel cas, elle sera abritée totalement au moment de l'explosion). Elle obtient un 44 - ratant le sas, mais obtenant un succès critique! Elle rate sa cible, mais le maître de jeu décide que l'appareil rebondit sur une machinerie quelconque et finit par tomber dans le sas. \end{quotation} 

A d'autres moment, il peut être important de savoir à quel point un personnage échoue, déterminé par la Marge d'Échec (MdE), qui est l'écart entre le score du personnage et le seuil du jet. Dans certains cas, un test peut indiquer qu'un personnage qui échoue avec une certaine MdE peut souffrir de conséquences additionnelles pour avoir échouer si lamentablement. 

\begin{quotation} Nico essayes de dessiner de mémoire le visage de quelqu'un. Il a une mémoire éidétique, mais son dessin doit être suffisament bon pour que quelqu'un d'autre puisse identifier la personne. Il fait un jet avec sa compétence Art: Dessin de 34, et obtient un 97 - Une MdE de 63. L'illustration est tellement mauvaise que le maître de jeu détermine que quiconque utilisera ce dessin pour identifier la personne devra obtenir une MdR d'au moins 63 sur un Test de Perception pour la reconnaître. \end{quotation} 

\subsubsection{Réussites Exceptionelle/Échec Catastrophique} \label{sec:excell-succ-fail} 

Les Réussites Exceptionnelle et les Échecs Catastrophiques sont une méthode utilisée pour évaluer une réussite ou un échec avec une MdR ou une MoE de 30+. Les Réussites Exceptionnelles sont utilisés dans les situations où un jet particulièrement bon peut améliorer l'effet désiré, tels que infliger plus de dommage avec un bon coup en combat. Les Échecs Catastrophiques signalent un résultat particulièrement mauvais et à des effets pire qu'un simple échec. Ni les Réussites Exceptionnelles ni les Échecs Catastophiques sont aussi bon ou mauvais que les critiques. 

\begin{quotation} Stoya s'est retrouvée embarquée dans une négociation qui a dégénérée. Elle se déplace pour donner un coup de pied à son adversaire en utilisant sa compétence de Combats à Mains Nues de 65. Elle obtient 33 (pour une MdR de 32) et son opposant obtient un 21 (également un succès, mais inférieur au 33 de Stoya, elle l'emporte donc). Elle réussit et bat son adversaire avec une MdR de 30+, obtenant une Réussite Exceptionnelle, signifiant qu'elle infligera des dommages supplémentaires avec son coup de pied. \end{quotation} 

\subsection{Tests en Opposition} \label{sec:opposed-tests} 

Un Test en Opposition est demandé lorsque l'action d'un personnage s'oppose directement à l'action d'un autre. Indépendemment de qui commence l'action, les deux personnages doivent faire un test l'un contre l'autre, dont le résultat favorisera le gagnant. 

Pour réaliser un Test en Opposition, chaque personnage lance 1d100 contre un seuil égal à la compétence appropriée modifiée selon la situation. Si un seul des des personnages réussit (obtient un résultat inférieur ou égal à son seuil), ce personnage l'emporte. Si les deux réussissent, le personnage qui obtient le résultat le plus élevé gagne. Si les deux personnages échouent, ou qu'ils réussissent et obtiennent le même résultat, la situation est une impasse - les personnages restent au même niveau, à s'affonter, jusqu'à ce que l'un des deux tente quelque chose et parvienne à sortir de la situation ou décide de faire un autre Test en Opposition. 

Les succès critiquent surpassent le meilleur résultat dans un Test en Opposition - si les deux personnages réussissent et que l'un obtient 54 et l'autre 44, celui qui obtient le résultat critique de 44 l'emporte sur l'autre. 

Il faut apporter une attention particulière aux modificateurs à appliquer dans les Tests en Opposition. Certains modificateurs affecteront de manière égale les deux participants et doivent être appliqués aux deux tests. Si un modificateur découle d'un avantage d'un personnage sur l'autre, ce modificateur ne devrait s'appliquer qu'au bénéfice du personnage avantagé; il ne devrait pas être apliqué en temps que modificateur négatif au personnage désavantagé. 

\begin{quotation} Zhou a été employé par la République Jovienne pour infiltrer son ancien équipage pirate. Même si il s'est réincarné dans une nouvelle peau, il s'inquiète du fait que l'un de ses anciens pote, Wen, puisse reconnaître ses manières étant donné qu'ils ont vécus, partagés des filles et menés des raids ensemble pendant des années. Après que Zhou ait passé un peu de temps en compagnie de Wen, le maître de jeu fait un Test en Opposition secret, opposant la compétence Imposture de 57 de Zhou contre la Kinésique de 34 de Wen. Le maître de jeu décide de donner à Wen un bonus de +20 étant donné qu'il est extrêmemet familier de son vieux pote et qu'il le recherche activement, impatient de lui faire payer la vieille rancune qui les a séparée. Le seuil de Wen est de 54. 

La maître de jeu lance les dés pour les deux protagonistes. Zhou obtient 45 et Wen 39. Les deux réussissent, mais Zhou obtient un meilleur résultat, son mensonge est donc une réussite. Le maître de jeu décide que Wen trouve quelquechose de familier à Zhou, sans pouvoir mettre le doigt dessus. \end{quotation} 

\subsubsection{Tests en Opposition et Marge de Réussite/d'Échec} \label{sec:opposed-tests-margin} 

Dans certains cas, il peut être important de prendre en compte la Marge de Réussite ou d'Echec dans un Test en Opposition, de la même manière qu'avec un Test de réussite comme décrit ci-dessus. Dans ce cas, la MdR/MdE est toujours déterminée par l'écart entre le jet d'un personnage et son seuil - elle n'est pas calculée par la différence entre le résultat obtenus par les deux personnages. 

\subsubsection{Tests en Opposition variables} \label{sec:vari-oppos-test} 

Dans certains cas, les règles demandent un Test en Opposition Variable, qui permet des issues plus fines qu'un test en Opposition standard. Si les deux personnages réussissent leur jet lors d'un Test en Opposition Variable, alors l'issue est différente de la simple victoire d'un personnage sur l'autre. L'issue exacte est notée avec chaque Test en Opposition Variable. 

\begin{quotation} Jaqui a besoin de hacker un réseau local pour récupérer des extraits vidéos. Le réseau est activement défendu par une IA, un Test en Opposition Variable est donc demandé, opposant la Compétence Infosec de 48 de Jacqui contre l'Infosec de 25 de l'IA. Jaqui obtient un 48 - une réussite - mais l'IA obtient un 24 - également une réussite. Dans ces circonstances, Jaqui réussit a hacker le réseau, mais l'IA est consciente de l'infiltration et peut prendre des contre-meusre active contre Jaqui. \end{quotation} 

\section{Temps et Actions.} \label{sec:time-actions} 

Bien que le maître de jeu soit responsable de la gestion de la vitesse à laquelle se déroule les évènements, il y a des moments où il est important de savoir exactement qui agît quand, particulièrement si des personnages agissent avant ou après d'autres. Dans ces circonstances, le jeu dans Eclipse Phase est divisé en intervalle appelés Tour d'Action. 

\subsection{Tours d'Action} \label{sec:action-turns} 

Chaque Tour d'Action dure trois secondes, ce qui signifie qu'il y a vingt Tours d'Action par minute. L'ordre dans lequel les personnages agissent durant un tour est déterminé par un test d'Initiative (voir Initiative, p. 121). Les Tours d'Actions sont ensuite subdivisés en Phases d'Action. La stat Vitesse (p. 121) de chaque personnage détermine le total d'actions qu'ils peuvent faire en un tour, représenté par le nombre de Phases d'Action qu'ils peuvent prendre. 

\subsection{Types d'actions} \label{sec:types-actions} 

Le type d'action qu'un personnage peut prendre en un Tour d'Action sont réparties en quatre type: Action Automatique, Rapide, Complexe et de Tâches. 

\subsubsection{Actions automatiques} \label{sec:automatic-actions} 

Les actions automatiques sont "toujours actives" et ne nécessitent pas d'effort de la part du personnage, en partant du principe qu'ils sont conscient. 

Exemples: perception simple, certains exploits psi. 

\subsubsection{Actions rapide} \label{sec:quick-actions} 

Les actions rapide sont simples, elles peuvent donc être exécutées rapidement et simultanément à d'autres. Le maître de jeu décide combien d'actions Rapide un personnage peut prendre en un tour. 

Exemples: parler, basculer un cran de sureté, activer un implant, se relever. 

\subsubsection{Actions complexe} \label{sec:complex-actions} 

Les actions Complexes recquièrent concentration et effort. Le nombre d'actions Complexe qu'un personnage peut faire en un tour est déterminé par sa stat Vitesse (voir p. 121). Exemples: attaquer, tirer, faire des acrobaties, désarmer une bombe, mener un examen détaillé. 

\subsubsection{Actions de Tâche} \label{sec:task-actions} 

Les actions de Tâches sont des actions qui nécessitent plus de temps qu'un Tour d'Action pour se réaliser. Chaque action de Tâche à un intervalle de temps, habituellement listé dans la description de la tâche ou déterminée par le maître de jeu. L'intervalle de temps détermine la durée nécessaire pour terminer la tâche, bien que ce temps puisse être réduit de 10 pourcent pour chaque tranche complète de 10 points de MdR obtenues par le personnage (voir Marge de Succès/d'Échec, p. 118). Si un personnage rate son jet lors d'une action de Tâche, il travaille à la tâche pendant une période minimale égale à 10 pourcent de l'intervalle de temps pour chaque tranche complète de 10 points de MdE avant de s'apercevoir de leur échec. Pour les actions de Tâche avec un intervalle de temps de plus d'une journée, on part du principe que le personnage travaille 8 heures par jour. Un personnage travaillant plus de huit heures par jour réduit le temps nécessaire en proportion. Les personnages travaillant à des actions de Tâche peuvent interrompre leur travail pour faire quelque chose d'autre et reprendre ensuite le boulot là où ils l'ont arrété, sauf si le maître de jeu décide que l'acion nécessite une attention continue et ininterrompue. De manière similaire à celle utilisée pour Prendre son Temps (p. 117), un personnage peut précipiter le travail sur une action de Tâche, en prenant une pénalité sur le test afin de réduire l'intervalle de temps. Le personnage doit déclarer qu'il précipite son action avant de lancer les dés. Pour chaque réduction de 10\% de l'intervalle de temps, il reçoit un modificateur de -10 à son test (jusqu'à un maximum de 60 pourcent pour un odificateur maximal de -60). 

\section{Définir votre personnage} \label{sec:defin-your-char} 

Afin d'évaluer et de quantifier les domaines que votre personnage connaît à peine et ceux dans lesquels il est un exepert - ou pour déterminer ce pour quoi ils n'ont aucune idée et ce à quoi ils sont nuls - Eclipse Phase utilise un nombre de facteurs d'évaluation: stats, compétences, traits et morphs. Chacune de ces caractéristiques est enregistrée et suivie sur votre fiche de personnage (p. 399). 

\subsection{Concept} \label{sec:concept} 

Le concept de votre personnage définit qui vous êtes dans l'univers d'Eclipse Phase. Vous n'êtes pas juste un plébéien moyen avec une vie ennuyante et monotone, vous êtes un participant actif d'un futur transhumain et post-apocalyptique qui se retrouve embarqué dans des intrigues, de terribles dangers, d'innomables horreurs et qui se débat pour sa survie. Comme à peu près tous les personnages dans une aventure, un drame ou une histoire d'horreur, vous êtes une personne à qui il arrive des choses intéressantes - ou si ce n'est pas le cas, vous faites en sorte que cela vous arrive. Cela signifie que votre personnage a besoin d'une personnalité distincte et d'un sens de l'identité. A l'extrême limite, vous devriez être capable de résumer le concept de votre personnage en une phrase simple, telle que "un archéologiste renégat néoténique et aigri avec des problèmes de gestion de la colère" ou "un animal social casse-cou qui est dangereusement obsédé par les théories conspirationnsites et les mystères." Si cela vous aide, vous pouvez toujours emprunter des idées d'autres personnage que vous avez vu dans un film ou des livres, en les adapatant à vos goûts. Le concept de votre personnage sera probablement influencé par deux facteurs importants: historique et faction. Votre historique indique les circonstances dans lesquelles votre personnage a été éduqué, alors que votre faction indique le plus récent groupe post-Chute avec lequel vous avez gardé des attaches et des allégeance. Les deux éléments jouent un rôle dans la création de personnage (p. 128). 

\subsection{Motivations} \label{sec:motivations} 

Le choc des idéologies et des mêmes est une composante au cœur d'Eclipse Phase, et chaque personnage a donc trois motivations - des mêmes personnels qui dominent les buts et les intérêts du personnage. Ces mêmes peuvent être aussi abstrait que des idéologies auxquelles le personnage adhère ou qu'il supporte - par exemple, l'anarchisme social, le jihad Islamiste ou le bioconservatisme - ou aussi concret que des objectifs que le personnage désire atteindre, tels que révéler la corruption d'une hypercorp particulière, obtenir une richesse personnelle massive ou gagner des batailles pour obtenir  des droits pour les élevés. Une motivation peut aussi être définie en opposition à quelque chose; par exemple l'anti-capitalisme ou le refus de la citoyenneté pour les pods, ou rester hors de prison. Ce sont les idées qui, par essence, motivent le personnage à faire les choses qu'ils font. Les motivations sont généralement notées comme un mot ou une courte phrase sur la fiche de personnage, marquées avec un + (en faveur de) ou un - (opposé à). Les joueurs sont encouragés à développer leurs propres motivations pour leurs personnages, en coopération avec le maître de jeu. Des exemples sont fournis p. 138. En termes de jeu, les motivations sont utilisées pour aider à définir la personnalité d'un personnage et l'influence de leurs actions dans un but de rôleplay. Elles permettent aussi de regagner des points Moxie (p. 122) et de gagner des points de Rez pour l'avancement du personnage (p. 152). 

Les objectifs motivés peuvent avoir un court ou un moyen terme, et peuvent de fait évoluer au fil du temps pour un personnage. Des objectifs à court-terme sont atteignables plus immédiatement ou de simples passades, et ces buts sont plus sujet à changer une fois atteint. Même ainsi, ils doivent refléter des intentions qui prendront plus d'une session de jeu à accomplir, pouvant couvrir plusieurs semaines ou mois de jeu. Ces objectifs à court terme peuvent en fait être directement liés à la trame du maître de jeu. Les exemples incluent mener une analyse complète d'un artefact étranger, compléter un projet de recherche ou vivre la vie d'un chien élevé pendant un moment. Les objectifs à long-terme reflètent des croyances profondément implantées dans la personnalité du personnage ou des tâches qui nécessitent un effort majeur et du temps (probablement le temps d'une vie) pour s'accomplir. Par exemple, trouver la sauvegarde perdue d'un parent proche disparu depuis la Chute, faire tomber un régime autoritaire ou établir un premier contact avec une nouvelle espèce alien. Dans le cadre de l'attribution de points de Moxie ou de Res, les objectifs à long terme ont intérêt à être divisés en plusieurs parties réalisables. Quelqu'un dont le but est de pourchasser le meurtrier qui a tué ses parents lorsqu'il était un enfant, peut considérer que son objectif est atteint à chaque fois qu'il découvre des pièces manquantes au puzzle, le rapprochant de son objectif final. 

\subsection{Ego contre morph} \label{sec:ego-vs.-morph-1} 

Le cadre de jeu d'Eclipse Phase impose qu'une distinction soit faites entre l'ego d'un personnage (leur identité, leur personnalité et les traits inhérents à ces aspects qui se perpétue dans la continuité) et leur morph (leur forme physique - et parfois virtuelle - éphémère). La morph d'un personnage peut mourrir alors que son ego survit (en supposant que des mesures de sauvegardes appropriées ont été prise),transplanté dans une nouvelle morph. Les morphs sont jetables, mais l'ego de votre personnage représente la continuité de l'esprit, de la personnalité, des mémoires, des connaissances et autres de votre personnage. Cette continuité peut être interrompue par une mort inattendue (dépendant de la date de votre dernière sauvegarde), ou en forkant (voir p. 273), mais il représente la totalité de l'état mental et des expériences du personnage. 

Des apects de votre personnage - en particulier les compétences, ainsi que quelques traits et statistiques - appartiennent à l'ego de votre personnage et ainsi l'acompagnent tout au long du développement du personnage. D'autres statistiques et traits sont cependant déterminés par une morph, comme noté précédemment, et changeront donc si votre personnage quitte son corps et en prend un autre. Les morphs peuvent aussi affecter d'autres compétences et statistiques, comme détaillé dans la description des morphs. 

Il est important que vous conserviez les caractéristiques dérivées de votre ego - et de votre morph - à jour, plus particulièrement lorsque vous mettez à jour votre fiche de personnage. 

\subsection{Stats des personnages} \label{sec:character-stats} 

Les stats de votre personnage mesurent différentes caractéristiques qui sont importantes pour le jeu: Initiative, Vitesse, Solidité, Seuil de Blessure, Lucidité, Seuil de Trauma et Moxie. Certaines de ces stats sont inhérentes à l'ego de votre personnage et d'autres sont influencées ou déterminées par la morph. 

\begin{quotation} \textbf{Stats d'ego} \begin{itemize} \item Initiative \item Lucidité \item Seuil de \item Trauma\item Seuil de \item Folie \item Moxie \end{itemize} \end{quotation} 

\begin{quotation} \textbf{Stats de morph} \begin{itemize} \item Vitesse \item Solidité \item Seuil de \item Blessure \item Seuil de \item Mort \item Bonus de \item Dégat \end{itemize} \end{quotation} 

\subsubsection{Initiative (init)} \label{sec:initiative-init} 

La stat d'Initiative de votre personnage aide à déterminer à quel moment il agît par rapport aux autres personnages pendant le Tour d'Action (voir Initiative, p. 188). Votre stat d'Initiative est égal à la somme des aptitudes Intuition et Réflexes de votre personnage multipliée par 2 (voir Aptitudes, p. 123). Certains implants et d'autres facteurs peuvent modifier ce score. 

\begin{quotation} Le score d'Intuition de Lazaro est de 15 et son score de Réflexe est de 20. Cela signifie que sonInitiative est de 70 (15 + 20 = 35, 35 x 2 = 70). \end{quotation} 

\subsubsection{Vitesse (vit)} \label{sec:speed-spd} 

La stat de Vitesse détermine le nombre de fois que votre personnage peut agir en un Tour d'Action (voir Initiative, p. 188). Tous les personnages démarrent avec une stat de Vitesse de 1, signifiant qu'ils agissent une fois par tour. Certains implants et d'autres avantages peuvent améliorer cette stat jusqu'à un maximum de 4. 

\subsubsection{Solidité (sol)} \label{sec:durability-dur} 

La Solidité est la santé physique de votre morph (ou son intégrité structurelle ou système dans le cas des coques synthétique ou des infomorphs). Elle détermine le total de dégats que votre morph peut encaisser avant que vous soyez incapacité ou tué (voir Santé Physique, p. 206). 

La Solidité est illimitée, bien que l'échelle pour les humains de base (non modifié) tend à ête comprise entre 20 et 60. Votre stat de Solidité est déterminée par votre morph. 

\subsubsection{Seuil de Blessure (sb)} \label{sec:wound-threshold-wt} 

Le Seuil de Blessure est utilisé pour déterminer si vous recevez une blessure chaque fois que vous encaissez des dommages physiques (voir Santé Physique, p. 206). Plus le Seuil de Blessure est élevé, plus vous êtes résitants aux blessures sérieuses. 

Le Seuil de Blessure est calculé en divisant la Solidité par 5 (en arrondissant au supérieur). 

\subsubsection{Seuil de Mort (sm)} \label{sec:death-rating-dr} 

Le Seuil de Mort est le montant total de dommage que votre morph peut encaisser avant d'être tué ou détruit au delà de toute réparation. Le Seuil de Mort est égal à SOL x 1,5 pour les biomorpĥs et SOL x 2 pour les synthmorphs. 

\begin{quotation} Tyska est incarné dans une morph splicer standard avec une Solidité de 30. Cela lui donne un Seuil de Blessure de 6 (30 / 5) et un Seuil de Mort de 45 ( 30 x 1.5 ). Si Tyska acquiert un implant qui améliore la Solidité de +10 pour l'amener à 40, son Seuil de Blessure sera de 8 ( 40 / 5 ) et son Seuil de Mort serait de 60 ( 40 x 1.5 ). \end{quotation} 

\subsubsection{Lucidité (luc)} \label{sec:lucidity-luc} 

La Lucidité est similaire à la Solidité, excepté qu'elle mesure la santé mentale et l'état d'esprit au lieu de la santé et du bien-être physique. Votre Lucidité détermine combien de stress (dommages mentaux) vous pouvez prendre avant d'être incapacité ou de devenir fou (voir Santé Mentale, p. 209). 

La Lucidité est illimitée, mais est comprise entre 20 et 60 pour l'humain non-modifié de base. La Lucidité est déterminée par votre aptitude de Volonté x 2. 

\subsubsection{Seuil de Trauma (st)} \label{sec:trauma-threshold-tt} 

Le Seuil de Trauma détermine si vous souffrez d'un trauma (blessure mentale) chaque fois que vous prenez du stress (voir Santé Mentale, p. 209). Un Seuil de Trauma plus élevé signifie que votre état mental est plus résilient vis à vis des expériences qui peuvent infliger des troubles psychiatrique ou autre instabilité mentale sérieuse. 

Le Seuil de Blessure est calculé en divisant la Lucidité par 5 (arrondissez au supérieur). 

\subsubsection{Seuil de Folie (sf)} \label{sec:insanity-rating-ir} 

Votre Seuil de Folie est le montant total de stress que votre esprit peut prendre avant devenir complètement fou et d'être perdu pour de bon. Le Seuil de Folie est égal à LUC x 2. 

\begin{quotation} La Volonté de Cole est de 16. Cela fait que sa stat Lucidité est de 32 (16 x 2), son Seuil de Trauma est de 7 (32 / 5, arrondi au supérieur), et son Seuil de Folie 64 (32 x 2). \end{quotation} 

\subsubsection{Moxie} \label{sec:moxie} 

Le Moxie représente le talent inné de votre personnage à faire face à des challenges et à vaincre des obstacles avec une ferveur particulière. Au delà de la simple chance, le Moxie est la capacité de votre personnage à courir sur le fil du rasoir et à faire ce qu'il faut, quelle que soit les probabilités. Certaines personnes le considère comme un trait évolutionnaire qui a permis à l'humanité de ramasser des outils, de développer nos cerveaux et d'affronter le futur bille en tête, laissant les autres mammifères dans la poussière. Quand le ciel vous tombe sur la tête, que la mort est imminente et que personne ne peut vous aider, le Moxie est ce qui vous gardera en un seul morceau. 

La stat Moxie est comprise entre 1 et 10, au niveau choisi lors de la création de personnage (et peut-être augmenté plus tard). En terme de jeu, le Moxie est utilisé pour influencer le hasard en votre faveur. A chaque session de jeu, votre personnage commence avec un nombre de point de Moxie égal à leur stat Moxie. Les points de Moxie peuvent être dépensés pour obtenir l'un des effets suivants: 

\begin{itemize} \item Le personnage peut ignorer tous les modificateurs appliqués à un test. Le point de Moxie doit être dépensé avant de lancer les dés. \item Le personnage peut faire un flip-flop sur le résultat d'un d100. Par exemple, un 83 deviendrait un 38. \item Le personnage peut améliorer un succès, le transformant en critique, comme si il avait obtenu un double. Le personnage doit avoir réussi son jet avant de dépenser le point de Moxie. \item Le personnage peut ignorer un échec critique, et le considérer comme un échec normal. \item Le personnage peut agir en premier dans une Phase d'Action (p. 189). \end{itemize} 

Un seul point de Moxie peut-être dépensé sur un jet de dé. Les points de Moxie vont fluctuer lors du jeu, au fur et à mesure qu'ils seront dépensés et parfois récupérés. 

Récupérer du Moxie: À la discrétion du maître de jeu, les points de Moxie peuvent-être régénérés jusqu'au niveau de Moxie maximum du personnage, chaque fois qu'il se repose pour une durée significative. Les points de Moxie peuvent aussi être récupérés si le personnage atteint un objectif personnel tel que déterminé par leur Motivation (voir p. 121). Le maître de jeu détermine combien de Moxie est récupéré en fonction de la proportion de l'objectif atteint. 

\begin{quotation} Audrey doit faire un jet de Piloter: Engins Volants difficile. Sa compétence est de 61, mais elle doit faire face à plein de modificateurs (-30), et si elle échoue, elle se retrouvera dans une situation critique. Elle pourrait dépenser un point de Moxie avant le test pour ignorer les modificateurs, mais elle décide de tenter sa chance contre le seuil de 31. Malheureusement, elle obtient un 82. Heureusement pour elle, elle peut dépenser un point de Moxie pour faireun flip-flop et transformer le jet en 28 - un succès! \end{quotation} 

\subsubsection{Bonus de dommage} \label{sec:damage-bonus} 

La stat de Bonus de Dommage quantify le punch supplémentaire que votre personnage peut donner à ses attaques de mélées et d'armes de jet. Le Bonus de Dommage est déterminé en divisant votre aptitude Somatique (voir plus bas) par 10 et en arrondissant à l'inférieur. 

\subsection{Compétences des personnages} \label{sec:character-skills} 

Les compétences représentent les talents de votre personnage. Les compétences sont séparés entre les aptitudes (capacités innées que tout le monde possède) et les compétences apprises (capacités et connaissances assimilées au fil du temps). Les compétences déterminent le seuil utilisés pour les tests (voir Faire des Test, p. 115). 

\subsubsection{Aptitudes} \label{sec:aptitudes} 

Les Aptitudes sont les compétences de base que chaque personnage possède par défaut. Elles sont les fondations sur lesquelles les compétences apprise sont construites. Les Aptitudes sont achetées pendant la phase de création du personnage et sont comprise entre 1 et 30, avec la moyenne pour l'humain non modifié de base à 10. Elles représentent les caractéristiques innées et les talents que votre personnage a développé depuis sa naissance et qui restent avec vous même lorsque vous changez de morphs - bien que certaines peuvent modifier vos score d'aptitude. 

Chaque compétence apprise est liée à une aptitude. Si un personnage ne possède pas la compétence nécessaire à un test, il peut défausser sur l'aptitude associée (voir Défausser p. 116). 

Il y a 7 aptitudes à Eclipse Phase: 

\begin{itemize} \item \textbf{Cognition (COG)} est votre aptitude pour la résolution de problème, l'analyse logique et la compréhension. Cela peut également inclure la mémoire et la capacité à se remémorer ces souvenirs. \item \textbf{Coordination (COO)} est votre compétence à intégrer les actions de différentes parties de votre morph pour produire des mouvements fluide et efficaces. Cela inclut la dextérité manuelle, le contrôle motriciel précis, l'agilité et l'équilibre. \item \textbf{Intuition (INT)} est votre compétence à suivre vos instincts et à évoluer les situations au vol. Elle inclut la conscience, l'intelligence et la ruse physique. \item \textbf{Réflexes (REF)} est votre compétence à agir rapidement. Elle définit votre temps de réaction, votre réponse instinctive et votre capacité à penser rapidement. \item \textbf{Astuce (AST)} est votre adaptabilité, votre intuition sociale et votre maîtrise dans les interactions avec les autres. Cela inclut votre conscience sociale et votre capacité à la manipulation. \item \textbf{Somatique (SOM)} est votre compétence à pousser votre morph au meilleur de ses capacités physique, incluant l'utilisation fondamentale de la force et de l'endurance de la morph ainsi que la capacité à maintenir une position ou un mouvement sur le long terme. \item \textbf{Volonté (VOL)} est votre capacité à gader votre sang-froid, à diriger votre propre destinée. \end{itemize} 

\subsubsection{Compétences apprises} \label{sec:learned-skills} 

Les compétences apprises englobent une vaste gamme de spécialités et d'apprentissages, de l'entraînement au combat à la négociatiion en passant par l'astrophysique (pour une liste complète de compétences, voir p. 176). Les compétences apprises sont comprises entre 1 et 99, le niveau d'efficacité moyen étant de 50. Chaque compétence apprise est liée à une aptitude, qui représente la compétence sous-jacente sur laquelle elle se base. Lorsqu'une compétence apprise est développée (soit pendant la création de personnage, soit par l'avancement), elle est achetée au niveau de l'aptitude liée puis augmentée depuis ce niveau. Si l'aptitude liée est augmentée ou modifiée, toutes les compétenes basées dessus sont également modifiées de manière appropriée. 

En fonction de votre historique et de votre faction, vous pouvez recevoir des compétences gratuites à la création du personnage. Comme pour les aptitudes, les compétences apprises restent avec le personnage même lorsqu'ils changent de morph, bien que certaines morphs, certains implants ou d'autre facteurs peuvent parfois varier le niveau de votre compétence. Si vous ne possèdez pas la compétence nécessaire à un test, vous pouvez défausser sur l'aptitude associée (voir Défausser p. 116). 

\subsubsection{Spécialisations} \label{sec:specializations} 

Les Spécialisations représentent un domaine de concentration dans une compétence particulière. Un personnage qui apprend une spécialisation est quelqu'un qui va au-delà de simplement saisir les bases d'une compétence, il s'est beaucoup entraîné pour exceller dans un aspect particulier du domaine de cette compétence. Les Spécialisations ajoutent un modificateur de +10 lorsque le personnage utilise cette compétence dans le domaine de sa spécialisation. 

Les Spécialisations peuvent être achetées pendant la création de personnage ou par l'avancement pour toute compétence existante que le personnage possède au moins au niveau 30. Une seule spécialisation peut-être acquise pour chaque compétence. Les spécialisations possibles spécifiques à chaque compétences sont notés sous la description de chacune des compétences (voir Compétences p. 170). 

\begin{quotation} Toljek a une compétence Manipulation de 63 avec une spécialisation en Pick-Pocket. Lorsqu'il utilise Manipulation pour attraper quelque chose dans la poche de quelqu'un ou, de manière similaire, voler quelque chose à quelqu'un, son seuil est de 73 alors que pour tous les autres usages de Manipulation, son score standard de 63 s'applique. \end{quotation} 

\subsection{Traits de personnage} \label{sec:character-traits} 

Les Traits incluents une vaste gamme de qualités et de caractéristiques inhérentes qui aident à définir votre personnage. Certains traits sont positifs, car ils donnent à votre personnage un bonus à certaines stats, compétences ou tests, ou leur donnent un avantage dans certaines situations. D'autres sont négatifss, car ils vont géner vos capacités ou générer un défaut qui génera vos plans. Certains traits s'appliquent à l'égo du personnage, restant avec eux d'une incarnation à l'autre, alors que d'autres ne s'appliquent qu'à la morph du personnage. 

Les Traits sont achetés pêndant la création de personnage. Les traits positifs vous coûtent des points de personnalisation (PP), alors que les traits négatifs vous rapporterons des PP supplémentaires à dépenser ailleurs (voir Traits, p. 145). Le nombre maximum de PP que vous pouvez dépenser dans les traits est de 50, alors que le gain de PP que vous pouvez obtenir de traits négatifs est de 50. Dans de rares circonstances - et seulement avec l'accord du maître de jeu - des traits peuvent être acquis, rachetés ou infligés pendant le jeu (voir p. 153). 

\subsection{Morph des personnages} \label{sec:character-morph} 

Dans Eclipse Phase, votre corps est jetable. Si il devient vieux, malade ou trop gravement abimé, vous pouvez numériser votre conscience et la télécharger dans un nouveau corps. Le processuss n'est ni bon marché, ni simple, mais il vous garanti une immortalité effective - tant que vous vous rappelez de vous sauvegarder et que vous ne devenez pas fou. Le terme de morph est utilisé pour décrire tout type de forme que votre esprit habite, qu'il s'agisse d'une enveloppe clonée cultivé en cuve, d'une coquille robotique et synthétique, d'un "pod" en partie bio et en partie synthétique, ou même de l'état purement logiciel d'une informoph. 

Vous achetez votre morph de départ lors de la création de personnage (voir p. 128). Il s'agît probablement de la morph dans laquelle vous êtes né (en partant du principe que vous êtes né), bien qu'il puisse simplement s'agir d'une autre morph dans laquelle vous vous êtes déplacé. 

Au delà de l'apparence physique, votre morph a un impact important sur vos caractéristiques. Votre morph détermine certaines de vos stats physiques, telles que la Solidité et le Seuil de Blessure, et elle peut aussi influencer l'Initiative et la Vitesse. Les morphs peuvent aussi modifier certaines de vos aptitudes et de vos compétences apprises. Certaines morph sont fournies préchargées avec des traits et des implants spécifiques représentatnt la manière dont elles ont étés fabriquées, et vous pouvez toujours y ajouter un peu plus de clinquants avec d'autres implants si vous le voulez (voir Implants, p. 126). Tout ces facteurs sont notés dans les descriptions individuelles des morphs (voir p. 139). 

Si vous prévoyer de basculer d'une morph à l'autre pendant le jeu, vous pouvez vouloir vous sauvegarder d'abord (voir Sauvegardes et Uploads, p. 268). Se sauvegarder régulièrement est toujours une bonne chose au cas où vous souffriez une mort accidentelle ou non-prévue. Acquérir une nouvelle morph n'est pas toujours simple, particulièrement si vous la voulez préchargée avec certaines spécifications. Le processuss complet est détaillée sous Réincarnation, p. 271. 

\subsubsection{Maximum d'Aptitude} \label{sec:aptitude-maximums} 

Chaque morph possède un maximum d'aptitude, quelquefois modifié par les traits. Ce maximum représente la plus haute valeur à laquelle un personnage peut utiliser une aptitude pendant qu'il habite cette morph, reflétant une limitation inhérente à certaines morph. Si l'aptitude d'un personnage dépasse le maximum de leur morph, ils doivent l'utiliser à la valeur maximale pendant la durée de leur incarnation dans cette morph. Cela peut égalment affecter les compétences liées à cette aptitude, qui doivent être modifiée de manière appropriée. 

Certains implants, équipements, exploit psi ou d'autres facteurs peuvent modifier les aptitudes naturelles d'un personnage. Comme ces valeurs augmentées représentent des facteurs externe améliorant les capacités de la morph, elles peuvent dépasser le maximum d'aptitude de celle-ci. Cependant,aucune aptitude, augmentée ou non, ne peut dépasser la valeur de 40. Les capacités innées amènent une personne jusqu'à ce point - pour aller au-delà, il faut compter sur les compétences apprises. 

\begin{quotation} Eva a une aptitude Cognition de 25. Elle est malheureusement forcée de s'incarner dans une morph plate qui possède un maximum d'aptitude de 20. Pour la durée de la période pendant laquelle elle habite cette morph, sa Cognition est réduite à 20, ce qui impacte également toutes ses compétences liées à la COG, les réduisant de 5. \end{quotation} 

\section{Les choses qu'utilisent les personnages.} \label{sec:things-char-use} 

Dans le cadre technologiquement avancé d'Eclipse Phase, les personnages ne s'en sortent pas uniquement avec leur astuce et leur morphs; ils utilisent leurs crédit et leur réputation pour acquérir de l'équipement et des implants et leurs réseaux sociaux pour rassembler leurs informations. Certains personnages ont aussi la capacité d'utiliser leurs pouvoirs mentaux appelés psi. 

\subsection{Identité} \label{sec:identity} 

A une époque d'informatique ubiquitaire et de surveillance omniprésente, la vie privée est un concept qui appartient au passé - qui vous êtes et ce que vous faites est facilement accessible en ligne. Les personnages dans Eclipse Phase, sont cependant régulièrement impliqués dans des activités secrètes et para-légales, et la seule façon de garder les blogueurs, les infos, les paparazzi et la loi à l'écart est de faire un usage extensif des fausses ID. Bien que Firewall fournisse des couvertures à ses sentinelles, cela ne fait pas de mal de garder une ou deux personnas en réserve, au cas où il devienne nécessaire d'abandonner le navire. Heureusement, le patchwork d'allégeances des cités-état des habitats et des stations factionnelles signifie que les identités ne sont pas trop dures à fausser, et la possibilité de changer de morph rend la tâche encore plus facile. D'un autre côté, quiconque est en possession d'une copie de vos données biométriques ou de votre empreinte génétique aura un avantage pour vous pister ou pour trouver toute trace que vous auriez pu laisser derrière vous (pour plus d'information sur les ID, voir p. 279). 

\subsection{Réseaux sociaux} \label{sec:social-networks} 

Les réseaux sociaux représentent les personnes que connaît le personnageet les groupes sociaux avec lesquel il interagît. Ces contacts, amis et connaissances ne sont pas seulement entretenus en personne, mais aussi par un usage massif des contacts par le Mesh. Les logiciels sociaux permettent aux gens de rester au courant de ce que font leurs connaissances,où ils sont et ce qui les intéresse de manière instantanée. Les Réseaux sociaux incroporent également les projets en-ligne des membres, qu'il s'agisse d'un site sur le mesh chargé de chanson du groupe de la personne, d'une archive personnel de média archivés, d'une décennie de billets de blog recensant les endroits où trouver de l'électronique bon marché ou un dépot de publication de recherche et d'études sur lesquels quelqu'un à travaillé ou qu'il a trouvé intéressant. 

En jeu, les réseaux sociaux sont particulièrement utiles aux personnages. Leur liste d'amis est une ressource essentielle - une réserve de personnes que vous pouvez utiliser activement pour trouver des idées, pour troller sur des sujets d'actualités, pour setenir au courant des dernières rumeurs, pour acheter ou vendre du matériel, pour obtenir des conseils d'experts et même pour demander des faveurs. 

Alors que les réseaux sociaux d'un personnage sont nébuleux et en changement permanent, leur utilisation ne change pas. Un personnage utilise ses réseaux sociaux via la compétence Réseau(Domaine) (p. 182). L'utilisation exacte de cette compétence est couverte sous le chapitre Réputation et Résaux Sociaux, p. 285. 

\subsection{Cred} \label{sec:cred} 

La Chute a dévasté l'économie globale et les systèmes monétaires du passé. Dans les années de reconsolidation qui ont suivies, les hypercorps et les gouvernements ont inaugurés un nouveau système monétaire électronique à l'échelle du système solaire. Appelé credit, cette monnaie est supportée par toutes les principales factions orientées vers le capitalisme et est utilisée pour échanger des marchandises et des services aussi bien que pour d'autres transactions financières. Le credit est principalement transféré électroniquement, bien que des puces de crédits certifiées sont également fréquemment utilisées (et préférées pour leur anonymat). Des factures papiers sont même utilisées dans certains habitats. 

En fonction de votre historique et de votre faction, votre personnage peut recevoir une somme de crédit au début du jeu. Pendant le jeu, votre personnage peut gagner du credit à l'ancienne: en le gagnant ou en le volant. 

\subsection{Rep} \label{sec:rep} 

Le capitalism n'est plus le seul système économique en ville. Le développement des nanofabeurs a rendu possible l'existence d'une économie post-pénurie, un fait largement exploité par les factions anarchistes et d'autres. Lorsque n'importe qui peut fabriquer n'importe quoi, des concepts tels que la propriété ou la richesse deviennent hors sujet. L'avènement d'économies fonctionnelles basées sur le communisme et le don, parmis d'autres économies alternatives, signifie que dans de tels système vous pouvez acquérir tout bien ou tout service dont vous avez besoin via l'échange libre, la réciprocité ou le troc - en partant du fait que vous êtes un membre contributeur d'un tel système et que vous êtes respectés par vos pairs. De manière similaire, l'art, la créativité, l'innovation et différentes forme d'expression culturelle ont une valeur bien plus élevée que dans une économie capitaliste. 

Dans les économies alternatives, la monnaie n'a souvent aucun sens, c'est la réputation qui fait sens. Votre score de réputation représente votre capital social - à quel point vous êtes estimés par vos pairs. La rep peut être augmentée en influençant positivement, en contribuant ou en aidant des individus ou des groupes, et elle peut être diminuée par un comportement antisocial.  Dans les habitats anarchistes, votre capacité à obtenir les choses dont vous avez besoin est entièrement basé sur la manière dont vous êtes vus par d'autres. 

La réputation est facilement mesurée par l'un des nombreux réseaux sociaux. Vos actions sont récompensées ou punies par ceux avec qui vous interagissez, qui pingent votre score de Rep avec un retour positif ou négatif. Ces réseaux sont utilisées par toutes les factions, car la réputation peut également affecter vos activités sociales dans une économie capitaliste. Les principaux réseaux réputationnels incluent: 

\begin{itemize} \item \textbf{La liste @:} la liste Arobase est utilisée par les anarchistes, les Barsoomiens, les Extropiens, la racaille et les Titaniens, notée comme @-rep. \item \textbf{CivicNet:} utilisée par la République Jovienne, l'Alliance Lunaire-Lagrange, la Constellation Morningstar, le Consortium Planétaire et beaucoup d'hypercorp et est référencé comme c-rep. \item \textbf{EcoWave:} utilisé par les nano-écologistes, les préservationnistes et les réclamationnistes, est référencé comme e-rep. \item \textbf{Fame:} le réseau pour voir et être vu utilisés par les élites, les artistes, les célébrités et les médias, référencé comme f-rep. \item \textbf{Guanxi:} utilisé par les triades et de nombreuses entités criminelles, référencé en tant que g-rep. \item \textbf{L'Œil:} utilisé par Firewall, référencé comme i-rep. \item \textbf{RRA:} Réseau de Recherche Affilié, utilisé par les argonautes, les technologistes, les scientifiques et les chercheurs, référencé comme r-rep. \end{itemize} 

La Réputation est notée de 0 à 99. En fonction de votre historique et de votre faction, vous pouvez démarrer avec un score de Rep dans un réseau ou plus. Ils peuvent être amélioré par des points de personnalisations pendant la création du personnage. Pendant le jeu, votre score de Rep dépendra entièrement des actions de votre personnage. Pour plus d'information, voir Réputations et Réseaux Sociaux, p. 285. 

Notez que chaque score de Rep est lié à une identité particulière. 

\subsection{Matériel} \label{sec:gear} 

Le matériel regroupe tout l'équipement que votre personnage possède et qu'il garde sur eux, depuis les armes et les armures à l'habillement et à l'électronique. Vous achetez votre équipement avec les points de personnalisation pendant la création de personnage (voir p. 136) et pendant le jeu avec du Credit ou de la Rep. Certains objets restreints, illégaux ou difficiles à trouver peuvent nécessiter des efforts particulier pour les obtenirs (voir Acquérir du Matériel, p. 298). Si vous avez accès à un nanofabeur, vous pouvez simplement construire du matériel, si vous possédez le plan adéquat (voir Nanofabrication, p. 284). Pour un listing complet d'options d'équipement, voir le chapitre Matériel, p. 296. 

Même parmi les restes de l'économie capitaliste, les prix peuvent varier drastiquement. Pour représenter ce fait, tout le matériel tombe dans une catégorie de prix. Chaque catégorie définit une gamme de prix, pour que le maître de jeu puisse ajuster le prix de chaque élément de manière appropriée à la situation. Chaque catégorie liste également le prix moyen pour cette catégorie, qui est utilisé pour la création de personnage et à n'importe quel moment où le maître de jeu désire garder la gestion des prix de manière simplifiée. Voir la table des Coûts du Matériel à la p. 137. 

\subsection{Implants} \label{sec:implants} 

Les implants incluent les améliorations cybernétique et bionique, la génétech et le nanoware (ou les améliorations mécaniques dans le cas des coques synthétiques) installées dans la morph de votre personnage. Ces implants peuvent donner à votre personnage des capacités spéciales ou modifier ses stats, ses compétences ou ses traits. Certaines morphs sont pré-équipées avec des implants, tel que noté dans leur description (voir p. 139). Vous pouvez également passer des commandes spéciales pour obtenir des implants sépcifiques dans votre morph (voir Acquérir une Morph, p. 277). Si vous voulez améliorer une morph que vous occupez actuellement, vous pouvez passer par de la chirurgie ou d'autres traitements similaires pour installer une amélioration (voir Cuves de Guérison, p. 326. Pour une liste complète des implants et des améliorations disponibles, voir pp. 300-311, Matériel. 

\subsection{Psi} \label{sec:psi} 

Le Psi est un ensemble de capacités mentales rares et anormales qui sont acquises suite à l'infection par un nanovirus étrange libéré pendant la Chute. Les possibilités du Psi ne sont pas encore totallement comprises, mais elles donnent au personnages certains avantages - ainsi que certains inconvénients. Un personnage doit prendre le trait Psi (p. 147) pour faire usage des capacités psi, qui sont appelés exploits. Les utilisateurs du Psi sont appelés asyncs. Une explication détaillée du fonctionnement du Psi et des détails sur les divers exploits peuvent être trouvés dans le chapitre Piratage Cognitifs, p. 216. 

\subsection{Résumé de règle} \label{sec:game-rules-summary} 

Tout ce que vous avez besoin de savoir sur les règles - résumé en une seule page. 

\subsubsection{Faire des tests (P. 115)} 

\begin{itemize} \item Lancez 1d100 (deux dés à dix faces, lus en pourcentage, de 00 à 99). \item Le seuil est déterminé par la compétence appropriée (ou occasionnellement, par une aptitude). \item La difficulté est représentée par des modificateurs. \item 00 est toujours un succès. \item 99 est toujours un échec. \item Une Marge de Réussite de 30+ est une Réussite Exceptionnel. \item Une Marge d'Échec de 30+ est un Échec Catastrophique. \item Un double (00, 11, 22, 33, etc) équivaut à un succès critique ou à un échec critique. \end{itemize} 

\subsubsection{Tests de réussite (P. 117)} 

\begin{itemize} \item Pour réussir, lancez 1d100 et obtenez un score inférieur ou égal à la compétence +/- les modificateurs. \end{itemize} 

\subsubsection{Tests en opposition (P. 119)} 

\begin{itemize} \item Chaque personnage lance 1d100 contre sa compétence +/- les modificateurs. \item Le personnage qui réussit avec le jet le plus élevé l'emporte. Si les deux personnages échouent, ou que les deux réussissent mais obtiennent le même résultat, la situation échoue en impasse. \end{itemize} 

\subsubsection{Test de réussite simple (P. 118)} 

\begin{itemize} \item Les Tests de réussite simple réussissent automatiquement. \item La réussite ou l'échec du jet indique simplement si le personnage réussit l'action avec brio ou de justesse. \end{itemize} 

\subsubsection{Défausser (P. 116)} 

\begin{itemize} \item Si un personnage ne possède pas la compétence appropriée pour un test, il peut défausser sur l'aptitude lié à la compétence. \end{itemize} 

\subsubsection{Modificateurs (P. 115)} 

\begin{itemize} \item Les Modificateurs affectent le seuil (niveau de la compétence), pas le jet. \item Les Modificateurs (positifs ou négatifs) sont déclinés en 3 niveaux de sévérité: \begin{itemize} \item Mineur (+/-10) \item Modéré (+/-20) \item Majeur (+/-30) \end{itemize} \item Le modificateur maximum applicable est de +/- 60. \end{itemize} 

\subsubsection{Travail d'équipe (P. 117)} 

\begin{itemize} \item Un personnage est choisi comme acteur principal; il effectue le test. \item Chaque assistant ajoute un modificateur de +10 (max. +30). \end{itemize} 

\subsubsection{Prendre son temps (P. 118)} 

\begin{itemize} \item Un personnage peuvent prendre du temps supplémentaire pour terminer une action. \item Sur les actions Complexes, chaque minute prise ajoute un +10 au test. \item Sur les actions de Tâche, toutes les augmentations de 50 pourcent de l'intervalle de temps ajoutent +10 au test. \end{itemize} 

\subsubsection{Aptitudes (P. 123)} 

\begin{itemize} \item Les Aptitudes vont de 1 à 30 (la moyenne est à 15). \item Les Aptitudes sont: Cognition, Coordination, Intuition, Réflexes, Astuce, Somatique et Volonté. \end{itemize} 

\subsubsection{Compétences apprises (P. 123)} 

\begin{itemize} \item Les Compétences vont de 1 à 99 (avec une moyenne de 50). \item Chaque compétence est liée à et basée sur une aptitude. \item Les morphs, le matériel, les drogues et autres, peuvent fournir des bonus ou des pénalités à des compétences spécifiques. \end{itemize} 

\subsubsection{Spécialisations (P. 123)} 

\begin{itemize} \item Une Spécialisation ajoute un modificateur de +10 en utilisant une compétence dans le domaine de concentration choisie. \item Chaque compétence ne peut avoir qu'une seule spécialisation. \end{itemize} 

\subsubsection{Tour d'action (P. 120)} 

\begin{itemize} \item Les Tour d'Actions durent 3 secondes. \item L'ordre dans lequel les personnages agissent est déterminé par leur Initiative. \item Les Actions automatiques sont toujours "actives." \item Les personnages peuvent avoir autant d'Actions Rapides qu'ils le veulent dans un Tour (minimum 3), limité seulement par le maître de jeu. \item Les personnages ne peuvent prendre qu'un nombre d'Action Complexe égal à leur stat Vitesse. \end{itemize} 

\subsubsection{Actions de tâche (P. 120)} 

\begin{itemize} \item Les Actions de Tâches sont les actions qui nécessitent plus d'1 Tour d'Action pour se terminer. \item Chaque Action de Tâche liste un intervalle (n'importe quelle durée, pouvant aller de 2 Tour à 2 ans, et plus). \item L'intervalle est réduit de 10\% pour chaque tranche de 10 points de MdR. \item Si le personnage échoue, il travaille à la tâche purent une période mnimale de 10\% de cet intervalle pour chaque tranche de 10 points de MdE avant de réaliser leur échec. \end{itemize} 



