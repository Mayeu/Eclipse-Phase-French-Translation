\chapter{Mécanismes de jeu} \label{chap:game-mechanics} 

Dans tous les jeux, il vient un moment où le maître de jeu doit décider si un personnage réçut ou rate une action. C'est le moment où les joueurs lancent des dés et où les stats du personnage entre en jeu. Ce chapitre définit les mécnaismes au cœur des règles qui gouvernent l'issue des évènements dans Eclipse Phase. 

\section{Une note sur la terminologie, la traduction et le Genre.} Le cadre d'Eclipse Phase soulève nombre de questions intéressante à propos du genre et de l'identité personnelle. Que signifie le fait d'être née femelle lorsque vous occuper un corps masculin? Lorsqu'on en vient au langage et à l'édition, cela pose aussi nombre de questions intéressantes sur l'usage des pronoms. Malheureusement, la langue Française ne possédant pas de genre neutre (contrairement au it Anglais), il nous sera donc difficile d'éviter le biais linguistique en faveur du masculin dans cette traduction.   Nous parlerons donc de joueurs, de maître de jeu, et de personnages avec des pronoms neutres dans la mesure du possible (équivalent du "on") en évitant de préciser les pronom (qu'il s'agisse de "il" ou de "elle"). Cependant, et afin de ne pas rendre le tout particulièrement lourd et indigeste, le masculin restera employé lorsque la grammaire l'exige (et il s'agira donc d'un joueur, d'un maître de jeu et d'un personnage), le masculin, à défaut d'être le genre neutre, étant le genre "par défaut" de la grammaire française.   



\section{Règles de bases} \label{sec:basics} 

\subsection{La règle ultime} \label{sec:ultimate-rule} 

Une règle dans Eclipse Phase surclasse toutes les autres: éclatez-vous. Cela signifie que vous ne devriez jamais laisser les règles se mettre en travers du jeu. Si vous n'aimez pas une règle, changez-la. Si vous ne trouvez pas une règle, créez en une. Si vous n'êtes pas d'accord avec l'interpétation d'une règle, tirez à pile-ou-face. Essayez de ne pas laisser les règles interférer avec la fluidité et l'ambiance du jeu. Si vous êtes au milieu d'une trés bonne scène ou d'un moment de roleplay intense et qu'une règle pose soudainement un problème, n'arrétez pas le jeu pour la vérifier ou pour en discuter. Improvisez, prenez une rapide décision, et continuez. Vous pourez toujorus revérifier la règle plus tard afin de vous en rappeler la prochaine fois. Si il y a des désaccords autour de l'interprétation d'une règle, rappelez-vous que le maître de jeu a le dernier mot. 

Cette règle signifie aussi que vous ne devriez pas laissez l'histoire être guidée uniquement par des jets de dés. L'aléas d'un jet de dé amènes une impression d'aléatoire, d'incertitude et de surprise à une partie. Parfois c'est excitant, comem par exemple lorsqu'un personnage réussit contre toute attente et jet de dé et, du coup, sauve l'équipe. A d'autres moment, c'est brutal comme, par exemple, un tir chanceux d'un adversaire tue un personnage pour de bon lors d'un combat. Si le maître de jeu veut qu'un scénario se termine par une issue dramatique planifiée et qu'un jet de dé inattendu menace son plan, il devrait se sentir libre d'ignorer ce jet et de continuer l'histoire dans la direction qu'il désire. 

\subsection{Dés} \label{sec:dice-1} 

Eclipse Phase utilise deux dés à dix faces (d10) pour les jets aléatoires. Dans la plupart des cas, les règles demanderont un jet de pourcentage, noté d100, ce qui signifie que vous lancez deux dés à dix faces, en choisissant lequel sera lu en premier, et en lisant le résultat comme compris entre 0 et 99 (avec un résultat de 00 comptant comme zéro, non pas comme 100). Le premier dés compte pour els dizaines, le second pour les unités. Par exemple, vous lancez deux dés à dix faces, un rouge et un noir, annonçant que le rouge sera lu en premier. Le rouge fait un 1 et le noir fait un 6, pour un résultat de 16. Certains ensembles de d10 sont spcifiquement marqués pour faciliter le lancement des dés et la lecture des résultats. 

De manière occasionnelle, les règles feront appel à des jets de dés seuls, chaque dés à dix face listés en temps que d10. SI les règles demandent que plusieurs d10 soient lancés, ils seront notés 2d10, 3d10 et ainsi de suite. Lorsque plusieurs dés à dix faces sont lancés de cette manières, leurs résultats sont ajoutés les uns aux autres. Par exemple un jet de 3d10 ayant pour résultat 4, 6 et 7 compte pour un 17. Sur les jets de d10, un résultat de 0 et traité comme un 10, pas comme un zéro. 

La plupart des joueurs d'Eclipse Phase s'en sortent en ayant seulement deux dés à dix-faces, mais ça ne fait aps de mal d'en avoir d'autres à portée de main. Ces dés peuvent être achetés dans votre boutique de jeu préférée ou empruntés à d'autres joueurs. 

\subsection{Faire des test} \label{sec:making-tests} 

Dans Eclipse Phase, votre personnage est destiné à se trouver impliqué dans des scènes d'actions adrénalisante, dans des situatsions ociales, hyper-stressante, dans des combats létaux, dans des enquètes frissonantes et d'autres situations simillaires emplies de drame, de risque et d'aventures. Lorsque votre personnage est impliqué dans ces scénarios, vous déterminez la manière dont il s'en sort en faisant des tests - en lançant des dés pour déterminer si ils réussissent ou échouent, et dans quelel mesure. 

Vous faites des tests dans Eclipse Phase en lançant un d100 et en comparant le résultat à un seuil. Le seuil est typiquement déterminé par l'une des compétences de votre personnage (voir plus bas) et est comprises entre 1 et 98. Si vous obtenez un résultat inférieur ou égal au seuil, vous réussissez votre test. Si vous dépassez le seuil, vous échouez à votre test. 

Un résultat de 00 est toujours considéré comme un succès. Un résultat de 99 ets toujours considéré comme un échec. 

\begin{quotation} Le personnage de Jacqui doit faire un test de compétence. Sa compétence est de 55. Jacqui prends deux dés à dix faces et obtient un 53 - elle réussit. Si elle avait obtenu un 55, elle rauait également réussit, mais tout résultat plus élevé que ça aurait été un échec. \end{quotation} 

\subsection{Seuil} \label{sec:target-numbers} 

Comme noté précédement, le seuil pour le jet d'un d100 dans Eclipse Phase est généralement la valeur d'une compétence. Cependant et de manière occasionnelle, un nombre différent sera utilisé. Dans certains cas, un score d'aptitude sera utilisé, ce qui rend les test plsu difficile car les aptitudes sont généralement bien en dessous de 50 (voir le passage Aptitudes, \ref{sec:aptitudes}). Dans d'autres test, la cible sera un score d'aptitude x2 ou x3 ou la somme de deux aptitudes. Dans ces cas là, la description du test indiquera quel(s) score(s) utiliser. 

\subsection{Quand faire des tetst} \label{sec:when-make-tests} 

Le maître de jeu décide quand un eprsonnage doit faire un test. Comme règle de base, les tests doivent être tentés lorsqu'il y a une chance qu'un eprsonnage échoue une action ou lorsque la réussite ou l'échec de l'action puisse avoir une effet sur l'histoire en-cours. Les tetst sont aussi nécessaire lorsque deux personnage ou plus agissent en opposition les uns aux autres (par exemple, si ils s'affrontent au bras de fer ou si ils négoceint un prix). D'un autre côté, les utilisatiosn routinières d'une compétences qui possèdent au moins un score de 30 dans cette compétence peuvent être considérées comme des réussites sansa faire de test. 

Il n'est pas nécessaire de faire de jets de dés pour les actiosn de la vie quotidienne tels que s'habiller ou vérifier ses mails (particulièrement dans Eclipse Phase, où tant de choses sont gérées automatiquement par les machines autour de vous). Même une activité telle que la conduite automobile ne nécessite pas de jets de dés tant que vous avez un minimum dans la compétence. Un test peut cependant être nécessaire si vous conduisez pendant que vous vous videz de votre sang ou que vous poursuivez un gang de charognards à moto à travers les ruines d'une cité dévastée. 

Savoir quand faire appel à des jets et quand laisser l'interprétation et le roleplay se dérouler sans interruption est une compétence que chaque maître de jeu doit acquérir. Parfois, il peut-être de simplement décider arbitrairement sans lancer de dés afin de maintenir le rythme de la partie. De la même manière, dans certaines circonstances décidées, le maître de jeu peut décider de faire des tests pour un personnage en secret, sans que le joueur ne le remarque. Si un ennemi essaye de s'infiltrer sous la vigilance d'un personnage, par example, le maître de jeu signalera au joueur que quelquechose va de travers si il demande au joueur de faire un test de perception. Cela signifie que le maïtre de jeu devrait garder en permanence une copie de chaque fiche de personnage à portée de main. 

\subsection{Difficulté et Modificateurs} \label{sec:difficulty-modifiers} 

L'évaluation de la difficulté d'un test est reflétées par ses modificateurs. Les modificateurs sont des ajustements apportés au seuil du test (pas au jet en lui-même), soit en l'augmentant, soit en le diminuant. Un test de difficulté moyenne n'aura pas de modificateur, alors que les actions plus simple auront des modificateurs posiftifs (augmentant le seuil de réussite et donc les chances de succès) et que les actions plus difficiles auront des modificateurs négatifs (abaissant le seuil de réussite et donc les chances de succès). C'est le boulot du maître de jeu de déterminer si un test particulier est plus difficile ou plus simple que la norme et dans quelle mesure (tel qu'illustré dans la table des Difficultés des Tests) et d'éppliquer ensuite les modificateurs appropriés. 

D'autres facteurs peuvent aussi jouer un rôle dans un test, appliquant des modificateurs additionels au delà du niveau de difficulté général du test. Ces facteurs incluent l'environnement, l'équipement (ou l'absence d'équipement) ainsi que la santé du personnage parmi d'autres. Le personnage peut utiliser des outils de qualité supérieure, travailler dans des conditions lamentables ou être blessé, et chacun de ces facteurs devrait être pris en compte, appliquant des modificateurs supplémentaires au seuil et modifier la probabilité de réussir ou de rater le jet. 

Dans un but de simplification, les modificateurs sont appliqués par multiples de 10 et viennent en trois niveau d'intensité: Mineur (+/-10), Modéré (+/-20) et Majeur (+/-30). Tant que le maître de jeu le pense approprié, n'importe quel nombre de modificateurs peuvent être appliqués, mais la valeur cumulée de ces modificateurs ne peut pas excéder + ou - 60. 

\begin{quotation} Jacqui tente d'aller d'une porte à une autre à travers une grande pièce en gravité zéro. Elle est pressée. Si elle rate la prote, elle perdra beaucoup de temps important et donc le maître de jeu demande un Test de Compétence en Chute Libre. La compétence Chute Libre de Jaqui est de 46. Malheureusement, la pièce est emplie de débris flottant qui pourrait la géner dans ses déplacements. Le maître de jeu détermine qu'il s'agît d'un modificateur Modéré, réduisant le seuil de 20. Jacqui doit obtenir 26 ou moins pour réussir son jet. \end{quotation} 

\subsection{Critiques: obtenir des doubles} \label{sec:crit-roll-doubl} 

A chaque fois que les dés tombent sur le même résultat - 00, 11, 22, 33, 44, etc - vous obtenez un succès critique ou un échec critique, selon que vous ayez battus ou non le seuil de difficulté. 00 est toujour sun succès critique alors que 99 est toujours un échec critique. Obtenir des doubles signifie qu'un petit extra se produit en plus de l'issue du test, qu'il soit positif ou négatif. Les critiques ont une applications trés spécifiques sur les test de Combat (voir \ref{sec:combat}), mais dans tous les autres cas le maître de jeu décide ce qui s'est mal ou bien passé dans une situation particulière. Les critiques peuvent être utilisés pour amplifier un succès ou un échec: vous pouvez terminer avec un bonus ou rater de manière tellement spectaclaire que vous serez la cible des moqueries pendant les semaines à venir. Ils peuvent aussi amener une sorte d'effet secondaire inattendus: vous réparez l'appareil et améliorez ses performances; ou vous pouvez échouer à toucher votre ennemi et blesser un passant innocent à la place. De manière alternative, un critique peut-être utilisé pour donner un boost (ou une gène) sur une action à suivre. Par exemple, vous ne faites pas que remarquer un indice, vous supectez immédiatement qu'il s'agît d'une fausse piste; ou vous ne faites pas que ratez votre cible, mais votre arme casse vous laissant sans défense. Les maîtres de jeu sont encouragés à être inventifs dans leur utilisation des critiques et de choisir les résultats qui créent des situations comiques, dramatiques ou de tendues. 

\begin{quotation} Audrey tente d'intimider une petite-frappe des triades afin d'en obtenir des informations. Malheureusement, elle obtient un 99 - un échec critique. En plus de ne pas parvenir à effrayer le mec, elle laisse filtrer une information importante qu'elle ne voulait pas que les triades obtiennent. Si elle avait obtenu un 00 - un succès critique - elle aurait tellement initimder le caïd qu'il lui aurait balancé des informations supplémentaires importante juste pour qu'elle le laisse tranquille par la suite. \end{quotation} 

\subsection{Défausser: utilisation de compétence non-entraînée} \label{sec:defa-untr-skill} 

Certain tests peuvent demander à un personnage d'utiliser une compétence qu'il ne possède pas - ce qu'on appelle défausser. Dans ce cas, le personnage utilise le score de l'aptitude (voir p. 123) liée à la compétence en question comme seuil de réussite. 

Toutes les compétences ne peuvent aps être défaussée; certaines sont tellement complexe ou nécessite un entrainement tel qu'un personnage ne la possédant pas n'a aucune chance de réussir. Les compétences qui ne peuvent pas être défaussée sont notées dans la Liste de Compétence (p. 176) et dans la description de celles-ci. 

dans de rares cas, le maître de jeu peu autoriser un personnage à se défausser sur une compétence qui pourrait également être liée au test (voir p. 173). Lorsque c'est autorisé, défausser sur une autre compétence inclut un modificateur de -30. 

\begin{quotation} Toljek essayes de s'inflitrer dans un bâtiment d'une hypercop lorsqu'il percute accidentellement une employée hypercoproratiste. La femme qu'il croise n'a pas forcément de raison d'être suspicieuse mvis à vis de la présence de Toljek, mais le maître de jeu demande à Toljek de réussir un Test de Protocole pour se faire passer pour quelqu'un qui est à sa place ici. Malheureusement, Toljek n'as pas cette compétence, il doit donc défausser sur l'aptitude liée, Bon Sens, à la place. Son score de Bon Sens est seulementd e 18, donc Toljek espère avoir de la chance. \end{quotation} 

\subsection{Simplifier les modificateurs} \label{sec:simpl-modif} 

Au lieu de chercher et d'accumuler une longue liste de modificateurs pour chaque action et de calculer le modificateur total, le maître de jeu peu choisir de simplement "évaluer" la situation et d'appliquer le modificateur qui résume le mieux l'effet voulu. Cette méthode est plus rapide et permet des résolutions de tests plus rapide Une manière d'évaluer la situation est de choisir le modificateur le plus sévère qui affecte la situation. 

\begin{quotation} Tyska essaye d'échapper à quelque chose qui la poursuit à travers un habitat abandonné. Le maître de jeu demande un Test de Parkour, mais il y a beaucoup de modificateurs situationnel: il fait sombre, il court avec une lampe de poche et il y a des débris un peu partout. Tyska, cependant, a une carte entoptique du meilleur trajet possible pour le sortir de là. Le maître de jeu estime la situation et décide que l'eefet global se résoud par un test difficile, et un modificateur de -20 est appliqué. \end{quotation} 

\subsubsection{Modificateurs narratifs} \label{sec:narrative-modifiers} 

Si vous voulez développer une ambiance plus cinématique à vos parties, ou si vous voulez simplement encourager vos joueurs à s'investir plus dans les détails et la création narrative, vous pouvez donner des "modificateurs narratifs" aux tests des personnages lorsqu'un joueur décrit les actions de son personnage de manière exceptionnellement colorée, inventive ou dramatique. Meilleurs sont les détails ajoutés, meilleurs sera le modificateur. 

\begin{quotation} Cole ne veux pas que son personnage saute simplement apr-dessus la table, il veut créer un minimum d'impact narratif. Cole annonce au maître de jeu que son personnage balance un coup de pied dans une chaise qui va valser plus loin, roule sur son épaule à travers la table à manger, attrapes une fourchette dans le mouvement, s'assure d'envoyer au sol toute la porcelaine puis atterit dans une posture martiale défensive, fourchette prête à frapper. Le maître de jeu décide que cette escription supplémentaire vaut bien un modificateur de +10 à son jet de Parkour. \end{quotation} 

\subsection{Travail d'équipe} \label{sec:teamwork} 

Si deux personnage ou plus se regroupent pour s'attaquer à un test ensemble, l'un des personnage doit être désigné comme l'acteur primaire. Ce personnage sera habituellement (mais pas forcément automatiquement) celui quei a le plus haut score dans une compétence applicable. Le personnage principal est celui qui lancera les dés, bien qu'il reçoive un modificateur de +10 pour chque personnage supplémentaire qu'i l'assite, jusqu'à un maximum de +30. Notez que les personnage assistants n'ont pas nécessairement besoin de connaître la compétence utilisé si le maître de jeu décide qu'ils peuvent suivre et comprendre les ordres du personnage dirigeant l'action. 

\begin{quotation} La jambe robotisée sur la synthmorph d'Eva a été salement abîmée, au point qu'elle ait besoin de la réparer. Max et Vic se posent et lui donnent un coup de main, lui donnant un modificateur de +20 (+10 pour chaque assistant) à son Test de Matériel: Robotique. \end{quotation} 

\section{Type de tests.} \label{sec:types-tests} 

Il y a deux types de test dans Eclipse Phase: Les tests de Résussite  et les Tests Opposés. 

\subsection{Tests de Réussite} \label{sec:success-tests} 

On demande un Test de Réussite lorsqu'un personnage agît sans opposition directe. Ce sont les tests standard utilisés pour déterminer à comment un personnage utilises une compétence ou une aptitude particulière. 

Les Tests de Réussite sont gérés exactement tels que décrit dans la section Réaliser des Tests, p. 115. Le joueur lance un d100 contre un seuil égal à la compétence du personnage +/- les modificateurs. Si il obtient un résultat inférieur ou égal au seuil, le tests est une réussite et l'action se déroule comme désiré. Si il obtient un résultat supérieur au seuil, le test échoue. 

\subsubsection{Essayer encore} \label{sec:trying-again} 

Si vous ratez un test, vous pouvez le tenter votre chance une fois de plus. Chaque tentative de réussir une action après un échec subira cependant un modificateur cumulatif de -10. Ce qui signifie que le deuxième essai souffre d'un -10, le troisième d'un -20, le quatrième d'un -30 et ainsi de suite jusqu'au maximum de -60. 

\subsubsection{Prendre le temps} \label{sec:taking-time} 

La plupart des tests concernent les Actions Automatique, Rapide ou Complexes (voir pp. 119–120) et sont donc réalisées en un Tour d'Action (3 secondes, voir p. 119). Les tests faits pour des Actiosn de Tâches (p. 120) prennent plus de temps. 

Les joueurs peuvent choisir de prendre du temps dupplémentaires lorsque leur personnage réalise une action, ce qui signifie qu'ils choisissent d'être particulièrement précautionneux lorsqu'ils accomplissent l'action afin d'améliorer leurs chances de succès. Pour chaque minute de temps supplémentaire, ils augmentent leur seui de +10. Une fois qu'ils ont modifiés leur seuil au-delà de 99, ils sont pratiquement sûr de réussir, le maître de jeu peut donc passer outre le jet de dé et leur accorder un succès automatique. Notez que la règle du modificateur maximal de +10 s'applique toujours, et que, apr conséquent, si leur compétence est inférieure à 40, prendre e temps peut ne pas garantir un succès automatique. Vous pouvez prendr evotre temps même lorsque vous défaussez (voir Défausser p. 116). 

Prendre du temps supplémentaires est une bonne décision lorsque le temps n'est pas un facteur décisif, car il élimine les chances d'échec critique et permet au joueur de passer outre des jtes de dés inutiles. Cependant, cela est inapproprié pour certains tests si le maître de jeu décide qu'aucune quantité de temps supplémentaire n'augmentera els chances de réussite. Dans ce cas, le maître de jeu décide simplement que prendre son temps n'a aucun effet. 

Pour les test d'Actions de Tâches (p. 120), qui prennent déjà du temps pour réussir, la durée de cette tâche peut être augmentée de 50 pourcent pour chaque modificateur de +10 gagné par du temps supplémentaire. 

\begin{quotation} Srit est en train de fouiller un vaisseau abandonné, à la recherche d'un signe de ce qui a pu arrivé à l'équiipage manquant. Le maître de jeu lieu dit que cela prendra vingt minutes pour fouiller tout le vaisseau. Cependant, elle veut être absolument sûre, elle prend donc trente minutes supplémentaires pour fouiller. Cinquantes pourcents de l'intervalle originel correspondent à dix minutes, donc en prendre trente de plus signifie que Srit reçoit un modificateur de +30 à son Test d'Investigation. \end{quotation} 

\subsubsection{Simple success tests} \label{sec:simple-success-tests} 

In some circumstances, the gamemaster may not be concerned that a character might fail a test, but instead simply wants to gauge how well the character performs. In this case, the gamemaster calls for a Simple Success Test, which is handled just like a standard Success Test (p. 117). Rather than determining success or failure, however, the test is assumed to succeed. The roll determines whether the character succeeds strongly (rolls equal to or less than the target number) or succeeds weakly (rolls above the target number). 

\begin{quotation} Dav is taking a short spacewalk between two parked ships. The gamemaster determines that this is a routine operation and calls for Dav to make a Simple Success Test using the Freefall skill. Dav's skill is only 35. He rolls a 76, but the gamemaster merely determines that Dav has some trouble orienting himself and has to take some extra time. If Dav had rolled a 77—a critical failure—his suit's maneuvering jets may have died and he may have accidentally propelled himself into deep space. \end{quotation} 

\subsubsection{Margin of success/failure} \label{sec:marg-succ} 

Sometimes it may be important that a character not only succeeds, but that they kick ass and take names while doing it. This is usually true of situations where the challenge is not only difficult but the action must be pulled off with finesse. Tests of this sort may call for a certain Margin of Success (MoS)—an amount by which the character must roll under the target number. For example, a character facing a target number of 55 and a MoS of 20 must roll equal to or less than a 35 to succeed at the level the situation calls for. 

\begin{quotation} An enemy has thrown an incendiary device near Stoya. She has only a moment to act and decides to try to kick it away from herself. Even better, she hopes to kick it into the open maintenance hatch a dozen meters away. The gamemaster determines that in order to kick it into the hatch, Stoya needs to succeed with an MoS of 30. Her Unarmed Combat skill is 66, so Stoya needs to roll 66 or less to kick the device away (though she may still be damaged when it explodes), and 36 or less to kick it into the hatch (in which case she will be completely safe when it detonates). She rolls a 44—missing the hatch, but scoring a critical success! Her aim is off, but the gamemaster decides that the device rebounds off some machinery and falls into the hatch anyway. \end{quotation} 

At other times, it may be important to know how badly a character fails, as determined by a Margin of Failure (MoF), which is the amount by which the character rolled over the target number. In some cases, a test may note that a character who fails with a certain MoF may suffer additional consequences for failing so dismally. 

\begin{quotation} Nico is trying to sketch out a picture of someone's face. He has eidetic memory, but his drawing needs to be good enough for someone else to identify the person. He rolls against his Art: Drawing skill of 34, scoring a 97—a MoF of 63. The illustration is so bad that the gamemaster determines that anyone using that picture to identify the person will need to score a MoS of at least 63 on a Perception Test to recognize the person. \end{quotation} 

\subsubsection{Excellent successes/severe failures} \label{sec:excell-succ-fail} 

Excellent Successes and Severe Failures are a method used to benchmark successes and failures with an MoS or MoF of 30+. Excellent Successes are used in situations where an especially good roll may boost the intended effect, such as inflicting more damage with a good hit in combat. Severe Failures denote a roll that is particularly bad and has a worse effect than a simple failure. Neither Excellent Successes or Severe Failures are as good or bad as criticals, however. 

\begin{quotation} Stoya has been caught in a deal gone bad. She moves to kick her opponent using her Unarmed Combat of 65. She rolls a 33 (for an MoS of 32), and her opponent rolls a 21 (also successful, but less than Stoya, so she wins). She has succeeded and beaten her opponent with an MoS of 30+, scoring an Excellent Success, meaning she will inflict extra damage with the kick. \end{quotation} 

\subsection{Opposed tests} \label{sec:opposed-tests} 

An Opposed Test is called for whenever a character's action may be directly opposed by another. Regardless of who initiates the action, both characters must make a test against each other, with the outcome favoring the winner. 

To make an Opposed Test, each character rolls d100 against a target number equal to the relevant skill(s) along with any appropriate modifiers. If only one of the characters succeeds (rolls equal to or less than their target number), that character has won. If both succeed, the character who gets the highest dice roll wins. If both characters fail, or they both succeed and roll the same number, then a deadlock occurs—the characters remain pitted against each other, neither gaining ground, until one of them takes another action and either breaks away or makes another Opposed Test. 

Note that critical successes trump high rolls in an Opposed Test—if both characters succeed and one rolls 54 while the other rolls 44, the critical roll of 44 wins. 

Care must be taken when applying modifiers in an Opposed Test. Some modifiers will affect both participants equally, and should be applied to both tests. If a modifier arises from one character's advantage in relation to the other, however, that modifier should only be applied to benefit the favored character; it should not also be applied as a negative modifier to the disadvantaged character. 

\begin{quotation} Zhou has been hired by the Jovian Republic to infiltrate his old pirate band. Even though he's resleeved in a new skin, he's worried that one of his old buddies, Wen, might recognize his mannerisms, since they lived, whored, and raided together for years. After Zhou has spent some time in Wen's company, the gamemaster makes a secret Opposed Test, pitting Zhou's Impersonation skill of 57 against Wen's Kinesics of 34. The gamemaster decides to give Wen a bonus +20, since he is so familiar with his former buddy and has been on the lookout for him, eager to repay the old grudge that split them apart. Wen's target number is now 54. 

The gamemaster rolls for both. Zhou scores a 45 and Wen a 39. Both succeed, but Zhou rolled higher, so his deception is successful. The gamemaster decides that Wen finds something about Zhou to be familiar, but he can't put his finger on it. \end{quotation} 

\subsubsection{Opposed tests and margin of success/failure} \label{sec:opposed-tests-margin} 

In some cases, it may also be important to note a character's Margin of Success or Failure in an Opposed Test, as with a Success Test above. In this case, the MoS/MoF is still determined by the difference between the character's roll and their target number—it is not calculated by the difference between the character's roll and the opposing character's roll. 

\subsubsection{Variable opposed test} \label{sec:vari-oppos-test} 

In some cases, the rules will call for a Variable Opposed Test, which allows for slightly more outcomes than a standard Opposed Test. If both characters succeed in a Variable Opposed Test, then an outcome is obtained which is different from just one character winning over the other. The exact outcomes are noted with each specific Variable Opposed Test. 

\begin{quotation} Jaqui needs to hack into a local network to retrieve some video footage. The network is ac- tively defended by an AI, so a Variable Opposed Test is called for, pitting Jaqui's Infosec skill of 48 against the AI's Infosec of 25. Jaqui rolls a 48—a success—but the AI rolls a 14—also a success. In this circumstance, Jaqui succeeds in hacking in, but the AI is aware of the infiltration and can take active countermeasures against her. \end{quotation} 

\section{Time and actions} \label{sec:time-actions} 

Though the gamemaster is responsible for managing the speed at which events unfold, there are times when it is important to know exactly who is acting when, especially if some people are acting before or after other people. In these circumstances, gameplay in Eclipse Phase is broken down into intervals called Action Turns. 

\subsection{Action turns} \label{sec:action-turns} 

Each Action Turn is three seconds long, meaning there are twenty Action Turns per minute. The order in which characters act during a turn is determined by an Initiative Test (see Initiative, p. 121). Action Turns are further subdivided into Action Phases. Each character's Speed stat (p. 121) determines the amount of actions they can take in a turn, represented by how many Action Phases they may take. 

\subsection{Types of actions} \label{sec:types-actions} 

The types of actions a character may take in an Action Turn are broken down to: Automatic, Quick, Complex, and Task actions. 

\subsubsection{automatic actions} \label{sec:automatic-actions} 

Automatic actions are “always on” and require no effort from the character, assuming they are conscious. 

Examples: basic perception, certain psi sleights 

\subsubsection{Quick actions} \label{sec:quick-actions} 

Quick actions are simple, so they can be done fast and can be multi-tasked. The gamemaster determines how many Quick actions a character may take in a turn. 

Examples: talking, switching a safety, activating an implant, standing up 

\subsubsection{Complex actions} \label{sec:complex-actions} 

Complex actions require concentration or effort. The number of Complex actions a character may take per turn is determined by their Speed stat (see p. 121). Examples: attacking, shooting, acrobatics, disarming a bomb, detailed examination 

\subsubsection{Task actions} \label{sec:task-actions} 

Task actions are any actions that require longer than one Action Turn to complete. Each Task action has a timeframe, usually listed in the task description or otherwise determined by the gamemaster. The time-frame determines how long the task takes to complete, though this may be reduced by 10 percent for every 10 full points of MoS the character scores on the test (see Margin of Success/Failure, p. 118). If a character fails on a Task action test roll, they work on the task for a minimum period equal to 10 percent of the timeframe for each 10 full points of MoF before realizing it's a failure. For Task actions with timeframes of one day or longer, it is assumed that the character only works eight hours per day. A character that works more hours per day may reduce the time accordingly. Characters working on Task actions may also interrupt their work to do something else and then pick up where they left off, unless the gamemaster rules that the action requires continuous and uninterrupted attention. Similar to taking the time (p. 117), a character may rush the job on a Task action, taking a penalty on the test in order to decrease the timeframe. The character must declare they are rushing the job before they roll the test. For every 10 percent they wish to reduce the timeframe, they incur a –10 modifier on the test (to a maximum reduction of 60 percent with a maximum modifier of –60). 

\section{Defining your character} \label{sec:defin-your-char} 

In order to gauge and quantify what your character is merely good at and what they excel in—or what they are clueless about and suck at—Eclipse Phase uses a number of measurement factors: stats, skills, traits, and morphs. Each of these characteristics is recorded and tracked on your character's record sheet (p. 399). 

\subsection{concept} \label{sec:concept} 

Your character concept defines who you are in the Eclipse Phase universe. You're not just a run-of-the-mill plebeian with a boring and mundane life, you're a participant in a post- apocalyptic transhuman future who gets caught up in intrigue, terrible danger, unspeakable horrors, and scrambling for survival. Much like a character in an adventure, drama, or horror story, you are a person to whom interesting things happen—or if not, you make them happen. This means your character needs a distinct personality and sense of identity. At the very least, you should be able to sum up your character concept in a single sentence, such as “cantankerous neotenic renegade archaeologist with anger management issues” or “thrill-seeking social animal who is dangerously obsessed with conspiracy theories and mysteries.” If it helps, you can always borrow ideas from characters you've seen in movies or books, modifying them to fit your tastes. Your character's concept will likely be influenced by two important factors: background and faction. Your background denotes the circumstances under which your character was raised, while your faction indicates the post-Fall grouping to which you most recently held ties and allegiances. Both of these play a role in character creation (p. 128). 

\subsection{motivations} \label{sec:motivations} 

The clash of ideologies and memes is a core component of Eclipse Phase, and so every character has three motivations—personal memes that dominate the character's interests and pursuits. These memes may be as abstract as ideologies the character adheres to or supports—for example, social anarchism, Islamic jihad, or bioconservatism -— or they may be as concrete as specific outcomes the character desires, such as revealing a certain hypercorp's corruption, obtaining massive personal wealth, or winning victories for uplifted rights. A motivation may also be framed in opposition to something; for example, anti-capitalism or anti-pod-citizenship, or staying out of jail. In essence, these are ideas that motivate the character to do the things they do. Motivation is best noted as a term or short phrase on the character sheet, marked with a + (in favor of) or – (opposed to). Players are encouraged to develop their own distinct motivations for their characters, in cooperation with the gamemaster. Some examples are provided on p. 138. In game terms, motivation is used to help define the character's personality and influence their actions for roleplaying purposes. It also serves to regain Moxie points (p. 122) and earn Rez Points for character advancement (p. 152). 

Motivational goals may be short-term or long-term, and may in fact change for a character over time. Short-term goals are more immediately obtainable objectives or short-lived interests, and these goals are likely to change once achieved. Even so, they should reflect intentions that will take more than one game session to reach, possibly covering weeks or months. These short-term goals may in fact tie directly into the gamemaster's current storyline. Examples include conducting a full analysis of an alien artifact, completing a research project, or living life as an uplifted dog for a while. Long-term goals reflect deeply rooted beliefs or tasks that require major efforts and time (possibly lifelong) to achieve. For example, finding the lost backup of a sibling missing since the Fall, overthrowing an autocratic regime, or making first contact with a new alien species. For purposes of awarding Moxie or Rez Points, long-term goals are best broken down into obtainable chunks. Someone whose goal is to track down the murderer who killed their parents when they were a child, for example, can be considered to achieve that goal every time they discover some evidence that brings them a little closer to solving the puzzle. 

\subsection{Ego vs. morph} \label{sec:ego-vs.-morph-1} 

Eclipse Phase's setting dictates that a distinction must be made between a character's ego (their ingrained self, their personality, and inherent traits that perpetuate in continuity) and their morph (their ephemeral physical—and sometimes virtual—form). A character's morph may die while the character's ego lives on (assuming appropriate backup measures have been taken), transplanted into a new morph. Morphs are expendable, but your character's ego represents the ongoing, continuous life path of your character's mind, personality, memories, knowledge, and so forth. This continuity may be interrupted by an unexpected death (depending on how recent the backup was made), or by forking (see p. 273), but it represents the totality of the character's mental state and experiences. 

Some aspects of your character—particularly skills, along with some stats and traits—belong to your character's ego, which means they stay with them throughout the character's development. D'autres statistiques et traits sont cependant déterminés par une morph, comme noté précédemment, et changeront donc si votre personnage quitte son corps et en prend un autre. Les morphs peuvent aussi affecter d'autres compétences et statistiques, comme détaillée dans la description des morphs. 

It is important that you keep ego- and morph-derived characteristics straight, especially when updating your character's record sheet. 

\subsection{character stats} \label{sec:character-stats} 

Your character's stats measure several characteristics that are important to game play: Initiative, Speed, Durability, Wound Threshold, Lucidity, Trauma Threshold, and Moxie. Some of these stats are inherent to your character's ego, others are influenced or determined by morph. 

\begin{quotation} \textbf{Ego stats} \begin{itemize} \item Initiative \item Lucidity \item Trauma \item Threshold \item Insanity \item Rating \item Moxie \end{itemize} \end{quotation} 

\begin{quotation} \textbf{Morph stats} \begin{itemize} \item Speed \item Durability \item Wound \item Threshold \item Death \item Rating \item Damage \item Bonus \end{itemize} \end{quotation} 

\subsubsection{Initiative (init)} \label{sec:initiative-init} 

Your character's Initiative stat helps determine when they act in relation to other characters during the Action Turn (see Initiative, p. 188). Your Initiative stat is equal to your character's Intuition + Reflexes aptitudes (see Aptitudes, p. 123) multiplied by 2. Certain implants and other factors may modify this score. 

\begin{quotation} Lazaro's Intuition is 15 and his Reflexes score is 20. That means his Initiative is 70 (15 + 20 = 35, 35 x 2 = 70). \end{quotation} 

\subsubsection{Speed (spd)} \label{sec:speed-spd} 

The Speed stat determines how often your character gets to act in an Action Turn (see Initiative, p. 188). All characters start with a Speed stat of 1, meaning they act once per turn. Certain implants and other advantages may boost this up to a maximum of 4. 

\subsubsection{Durability (dur)} \label{sec:durability-dur} 

Durability is your morph's physical health (or structural integrity in the case of synthetic shells, or system integrity in the case of infomorphs). It determines the amount of damage your morph can take before you are incapacitated or killed (see Physical Health, p. 206). 

Durability is unlimited, though the range for baseline (unmodified) humans tends to fall between 20 and 60. Your Durability stat is determined by your morph. 

\subsubsection{Wound threshold (wt)} \label{sec:wound-threshold-wt} 

A Wound Threshold is used to determine if you receive a wound each time you take physical damage (see Physical Health, p. 206). The higher the Wound Threshold, the more resistant to serious injury you are. 

Wound Threshold is calculated by dividing Durability by 5 (rounding up). 

\subsubsection{Death rating (dr)} \label{sec:death-rating-dr} 

Death Rating is the total amount of damage your morph can take before it is killed or destroyed beyond repair. Death Rating is equal to DUR x 1.5 for biomorphs and DUR x 2 for synthmorphs. 

\begin{quotation} Tyska is sleeved in a run-of-the-mill splicer morph with a Durability of 30. That gives him a Wound Threshold of 6 (30 / 5) and a Death Rating of 45 (30 x 1.5). If Tyska acquired an implant that boosted his Durability by +10 to 40, his Wound Threshold would be 8 (40 / 5) and his Death Rating would be 60 (40 x 1.5). \end{quotation} 

\subsubsection{Lucidity (luc)} \label{sec:lucidity-luc} 

Lucidity is similar to Durability, except that it measures mental health and state of mind rather than physical well-being. Your Lucidity determines how much stress (mental damage) you can take before you are incapacitated or driven insane (see Mental Health, p. 209). 

Lucidity is unlimited, but generally ranges from 20 to 60 for baseline unmodified humans. Lucidity is determined by your Willpower aptitude x 2. 

\subsubsection{Trauma threshold (tt)} \label{sec:trauma-threshold-tt} 

The Trauma Threshold determines if you suffer a trauma (mental wound) each time you take stress (see Mental Health, p. 209). A higher Trauma Threshold means that your mental state is more resilient against experiences that might inflict psychiatric disorders or other serious mental instabilities. 

Trauma Threshold is calculated by dividing Lucidity by 5 (rounding up). 

\subsubsection{Insanity rating (ir)} \label{sec:insanity-rating-ir} 

Your Insanity Rating is the total amount of stress your mind can take before you go permanently insane and are lost for good. Insanity Rating equals LUC x 2. 

\begin{quotation} Cole's Willpower is 16. That makes his Lucidity stat 32 (16 x 2), his Trauma Threshold 7 (32 / 5, rounded up), and his Insanity Rating 64 (32 x 2) \end{quotation} 

\subsubsection{Moxie} \label{sec:moxie} 

Moxie represents your character's inherent talent at facing down challenges and overcoming obstacles with spirited fervor. More than just luck, Moxie is your character's ability to run the edge and do what it takes, no matter the odds. Some people consider it the evolutionary trait that spurred humankind to pick up tools, expand our brains, and face the future head on, leaving other mammals in the dust. When the sky is falling, death is imminent, and no one can help you, Moxie is what saves the day. 

The Moxie stat is rated between 1 and 10, as purchased during character creation (and perhaps raised later). In game play, Moxie is used to influence the odds in your favor. Every game session, your character begins with a number of Moxie points equal to their Moxie stat. Moxie points may be spent for any of the following effects: 

\begin{itemize} \item The character may ignore all modifiers that apply to a test. The Moxie point must be spent before dice are rolled. \item The character may flip-flop a d100 roll result. For example, an 83 would become a 38. \item The character may upgrade a success, making it a critical success, as if they rolled doubles. The character must succeed in the test before they spend the Moxie point. \item The character may ignore a critical failure, treating it as a regular failure instead. \item The character may go first in an Action Phase (p. 189). \end{itemize} 

Only 1 point of Moxie may be spent on a single roll. Moxie points will fluctuate during gameplay, as they are spent and sometimes regained. 

Regaining Moxie: At the gamemaster's discretion, Moxie points may be refreshed up to the character's full Moxie stat any time the character rests for a significant period. Moxie points may also be regained if the character achieves a personal goal, as determined by their Motivation (see p. 121). The gamemaster determines how much Moxie is regained in proportion to the goal achieved. 

\begin{quotation} Audrey has a difficult Piloting: Aircraft roll to make. Her skill is 61, but she's facing a lot of modifiers (–30), and if she fails she's in big trouble. She could spend a point of Moxie before the test to ignore the modifiers, but she decides to take her chances against the target number of 31. Unfortunately, she rolls an 82. Luckily, she can spend a Moxie point to flip-flop that roll and make it a 28—a success! \end{quotation} 

\subsubsection{Damage bonus} \label{sec:damage-bonus} 

The Damage Bonus stat quantifies how much extra oomph your character is able to give their melee and thrown weapons attacks. Damage Bonus is determined by dividing your Somatics aptitude (see below) by 10 and rounding down. 

\subsection{Character skills} \label{sec:character-skills} 

Skills represent your character's talents. Skills are broken down into aptitudes (ingrained abilities that everyone has) and learned skills (abilities and knowledge picked up over time). Skills determine the target number used for tests (see Making Tests, p. 115). 

\subsubsection{Aptitudes} \label{sec:aptitudes} 

Aptitudes are the core skills that every character has by default. They are the foundation on which learned skills are built. Aptitudes are purchased during character creation and rate between 1 and 30, with 10 being average for a baseline unmodified human. They represent the ingrained characteristics and talents that your character has developed from birth and stick with you even when you change morphs—though some morphs may modify your aptitude ratings. 

Each learned skill is linked to an aptitude. If a character doesn't have the skill necessary for a test, they may default to the aptitude instead (see Defaulting, p. 116). 

There are 7 aptitudes in Eclipse Phase: 

\begin{itemize} \item \textbf{Cognition (COG)} is your aptitude for problem solving, logical analysis, and understanding. It also includes memory and recall. \item \textbf{Coordination (COO)} is your skill at integrat ing the actions of different parts of your morph to produce smooth, successful movements. It includes manual dexterity, fine motor control, nimbleness, and balance. \item \textbf{Intuition (INT)} is your skill at following your gut instincts and evaluating on the fly. It includes physical awareness, cleverness, and cunning. \item \textbf{Reflexes (REF)} is your skill at acting quickly. This encompasses your reaction time, your gut-level response, and your ability to think fast. \item \textbf{Savvy (SAV)} is your mental adaptability, social in tuition, and proficiency for interacting with others. It includes social awareness and manipulation. \item \textbf{Somatics (SOM)} is your skill at pushing your morph to the best of its physical ability, including the fundamental utilization of the morph's strength, endurance, and sustained positioning and motion. \item \textbf{Willpower (WIL)} is your skill for self-control, your ability to command your own destiny. \end{itemize} 

\subsubsection{Learned skills} \label{sec:learned-skills} 

Learned skills encompass a wide range of specialties and education, from combat training to negotiating to astrophysics (for a complete skill list, see p. 176). Learned skills range in rating from 1 to 99, with an average proficiency being 50. Each learned skill is linked to an aptitude, which represents the underlying competency in which the skill is based. When a learned skill is purchased (either during character generation or advancement), it is bought starting at the rating of the linked aptitude and then raised from there. If the linked aptitude is raised or modified, all skills built off it are modified appropriately as well. 

Depending on your background and faction, you may receive some starting skills for free during character creation. Like aptitudes, learned skills stay with the character even when they change morphs, though certain morphs, implants, and other factors may sometimes modify your skill rating. If you lack a particular skill called for by a test, in most cases you can default to the linked aptitude for the test (see Defaulting, p. 116). 

\subsubsection{Specializations} \label{sec:specializations} 

Specializations represent an area of concentration and focus in a particular learned skill. A character who learns a specialization is one who not only grasps the basic tenets of that skill, but they have trained hard to excel in one particular aspect of that skill's field. Specializations apply a +10 modifier when the character utilizes that skill in the area of specialization. 

Specializations may be purchased during character creation or advancement for any existing skill the character possesses with a rating of 30 or more. Only one specialization may be purchased for each skill. Specific possible specializations are noted under individual the skill descriptions (see Skills, p. 170). 

\begin{quotation} Toljek has Palming skill of 63 with a specialization in Pickpocketing. Whenever he uses Palming to pick someone's pocket or otherwise steal from someone's person, his target number is 73, but for all other uses of Palming the standard 63 applies. \end{quotation} 

\subsection{Character traits} \label{sec:character-traits} 

Traits include a range of inherent qualities and features that help define your character. Some traits are positive, in that they give your character a bonus to certain stats, skills, or tests, or otherwise give them an edge in certain situations. Others are negative, in that they impair your abilities or occasionally create a glitch in your plans. Some traits apply to a character's ego, staying with them from body to body, while others only apply to a character's morph. 

Traits are purchased during character generation. Positive traits cost customization points (CP), while negative traits give you extra CP to spend on other things (see Traits, p. 145). The maximum number of CP you may spend on traits is 50, while the maximum you may gain from negative traits is 50. In rare circumstances—and only with gamemaster approval—traits may be purchased, bought off, or inflicted during gameplay (see p. 153). 

\subsection{Character morph} \label{sec:character-morph} 

In Eclipse Phase, your body is disposable. If your body gets old, sick, or too heavily damaged, you can digitize your consciousness and download it into a new body. The process isn't cheap or easy, but it offers effective immortality—as long as you remember to back yourself up and don't go insane. The term morph is used to describe any type of form your mind inhabits, whether it be a vat-grown clone sleeve, a synthetic robotic shell, a part-bio/part-flesh pod, or even the purely electronic software state of an infomorph. 

You purchase your starting morph during character creation (see p. 128). This is likely the morph you were born with (assuming you were born), though it may simply be another morph you've moved onto. 

Physical looks aside, your morph has a large impact on your characteristics. Your morph determines certain physical stats, such as Durability and Wound Threshold, and it may also influence Initiative and Speed. Morphs may also modify some of your aptitudes and learned skills. Some morphs come pre-loaded with specific traits and implants, representing how it was crafted, and you can always bling yourself out with more implants if you choose (see Implants, p. 126). All of these factors are noted in the individual morph descriptions (see p. 139). 

If you plan on switching your current morph to another during gameplay, you may first want to back yourself up (see Backups and Uploads, p. 268). Backing up regularly is always a smart option in case you suffer an accidental or untimely death. Acquiring a new morph is not always easy, especially if you want it pre-loaded according to certain specifications. The full process is detailed under Resleeving, p. 271. 

\subsubsection{Aptitude Maximums} \label{sec:aptitude-maximums} 

Every morph has an aptitude maximum, sometimes modified by traits. This maximum represents the highest value at which the character may use that aptitude while inhabiting that morph, reflecting an inherent limitation in some morphs. If a character's aptitude exceeds the aptitude maximum of their morph, they must use it at the maximum value for the duration of the time they remain in that morph. This may also affect the skills linked to that aptitude, which must be modified appropriately. 

Some implants, gear, psi, and other factors may modify a character's natural aptitudes. These augmented values may exceed a morph's aptitude maximums, as they represent external factors boosting the morph's ability. No aptitude, however, augmented or not, may ever exceed a value of 40. Innate ability only takes a person so far—after that point, actual skill is what counts. 

\begin{quotation} Eva has a Cognition aptitude of 25. She is unfortunately forced to sleeve into a flat morph with an aptitude maximum of 20. For the duration of the period she inhabits that morph, her Cognition is reduced to 20, which also impacts all of her COG-linked skills, reducing them by 5. \end{quotation} 

\section{Things characters use} \label{sec:things-char-use} 

In the advanced technological setting of Eclipse Phase, characters don't get by on their wits and morphs alone; they take advantage of their credit and reputation to acquire gear and implants and use their social networks to gather information. Some characters also have the capability to use mental powers known as psi. 

\subsection{Identity} \label{sec:identity} 

In an age of ubiquitous computing and omnipresent surveillance, privacy is a thing of the past—who you are and what you do is easily accessed online. Characters in Eclipse Phase, however, are often involved in secretive or less-than-legal activities, so the way to keep the bloggers, news, paparazzi, and law off your back is to make extensive use of fake IDs. While Firewall often provides covers for its sentinel agents, it doesn't hurt to keep a few extra personas in reserve, in case matters ever go out the airlock in a hurry. Thankfully, the patchwork allegiances of city-state habitats and faction stations means that identities aren't too difficult to fake, and the ability to switch morphs makes it even easier. On the other hand, anyone with a copy of your biometrics or geneprint is going to have an edge tracking you down or finding any forensic traces you leave behind (for more on ID, see p. 279). 

\subsection{Social networks} \label{sec:social-networks} 

Social networks represent people the character knows and social groups with which they interact. These contacts, friends, and acquaintances are not just maintained in person, but also through heavy Mesh contact. Social software allows people to stay updated on what the people they know are doing, where they are, and what they are interested in, right up to the minute. Social networks also incorporate the online projects of individual members, whether it's a mesh-site loaded with a band member's songs, a personal archive of stored media, a decade of blog entries reviewing the best places to score cheap electronics, or a depository of research papers and studies someone has worked on or finds interesting. 

In game play, social networks are quite useful to characters. Their friends list is an essential resource—a pool of people you can actively poll for ideas, troll for news, listen to for the latest rumors, buy or sell gear from, hit up for expert advice, and even ask for favors. 

While a character's social networks are nebulous and constantly shifting, the use of them is not. A character takes advantage of their social networks via the Networking (Field) skill (p. 182). The exact use of this skill is covered under Reputation and Social Networks, p. 285. 

\subsection{Cred} \label{sec:cred} 

The Fall devastated the global economies and currencies of the past. In the years of reconsolidation that followed, the hypercorps and governments inaugurated a new system-wide electronic monetary system. Called credit, this currency is backed by all of the large capitalist-oriented factions and is used to trade for goods and services as well for other financial transactions. Credit is mainly transferred electronically, though certified credit chips are also common (and favored for their anonymity). Hardcopy bills are even used in some habitats. 

Depending on your background or faction, your character may be given an amount of credit at the start of the game. During game play, your character must earn credit the old-fashioned way: by earning or stealing it. 

\subsection{Rep} \label{sec:rep} 

Capitalism is no longer the only economy in town. The development of nanofabricators allowed for the existence of post-scarcity economies, a fact eagerly taken advantage of by anarchist factions and others. When anyone can make anything, concepts like property and wealth become irrelevant. The advent of functional gift and communist economies, among other alternative economic models, means that in such systems you can acquire any goods or services you need via free exchange, reciprocity, or barter—presuming you are a contributing member of such a system and respected by your peers. Likewise, art, creativity, innovation, and various forms of cultural expression have a much higher worth than they do in capitalist economies. 

In alternative economies, money is often meaningless, but reputation matters. Your reputation score represents your social capital—how esteemed you are to your peers. Rep can be increased by positively influencing, contributing to, or helping individuals or groups, and it can be decreased through antisocial behavior. In anarchist habitats, your likelihood of obtaining things that you need is entirely based on how you are viewed by others. 

Reputation is easily measured by one of several online social networks. Your actions are rewarded or punished by those with whom you interact, who can ping your Rep score with positive or negative feedback. These networks are used by all of the factions, as reputation can affect your social activities in capitalist economies as well. The primary reputation networks include: 

\begin{itemize} \item \textbf{The @-list:} the Circle-A list for anarchists, Bar- soomians, Extropians, scum, and Titanians, noted as @-rep. \item \textbf{CivicNet:} used by the Jovian Republic, Lunar- Lagrange Alliance, Morningstar Constellation, Planetary Consortium, and many hypercorps, referred to as c-rep. \item \textbf{EcoWave:} used by nano-ecologists, preservation- ists, and reclaimers, referred to as e-rep. \item \textbf{Fame:} the seen-and-be-seen network used by socialites, artists, glitterati, and media, referred to as f-rep. \item \textbf{Guanxi:} used by the triads and numerous crimi- nal entities, referred to as g-rep. \item \textbf{The Eye:} used by Firewall, noted as i-rep. \item \textbf{RNA:} Research Network Affiliation, used by ar- gonauts, technologists, scientists, and researchers, referred to as r-rep. \end{itemize} 

Reputation is rated from 0-99. Depending on your background and faction, you may start with a Rep score in one or more networks. This can be bolstered through spending customization points during character creation. During game play, your Rep scores will depend entirely on your character's actions. For more information, see Reputation and Social Networks, p. 285. 

Note that each Rep score is tied to a particular identity. 

\subsection{Gear} \label{sec:gear} 

Gear is all of the equipment your character owns and keeps on their person, from weapons and armor to clothing and electronics. You buy gear for your character with customization points during character creation (see p. 136) and in the game with Credit or Rep. Certain restricted, illegal, or hard-to-find items may require special efforts to obtain (see Acquiring Gear, p. 298). If you have access to a nanofabricator, you may be able to simply build gear, presuming you have the proper blueprints (see Nanofabrication, p. 284). For a complete listing of equipment options, see the Gear chapter, p. 296. 

Even among the remaining capitalist economies, prices can vary drastically. To represent this, all gear falls into a cost category. Each category defines a range of costs, so the gamemaster can adjust the prices of individual items as appropriate to the situation. Each category also lists an average price for that category, which is used during character generation and any time the gamemaster wants to keep costs simple. See the Gear Costs table on p. 137. 

\subsection{Implants} \label{sec:implants} 

Implants include cybernetic, bionic, genetech, and nanoware enhancements to your character's morph (or mechanical enhancements in the case of a synthetic shell). These implants may give your character special abilities or modify their stats, skills, or traits. Some morphs come pre-equipped with implants, as noted in their descriptions (see p. 139). You may also special- order morphs with specific implants (see Morph Acquisition, p. 277). If you want to upgrade a morph you are already in, you can undergo surgery or other treatments to have an enhancement installed (see Healing Vats, p. 326. For a complete list of available implant/enhancement options, see pp. 300-311, Gear. 

\subsection{Psi} \label{sec:psi} 

Psi is a rare and anomalous set of mental abilities that are acquired due to infection by a strange nanovirus released during the Fall. Psi abilities are not completely understood, but they give characters certain advantages—as well as some disadvantages. A character requires the Psi trait (p. 147) to use psi abilities, which are called sleights. Psi users are called asyncs. A full explanation of psi and details on the various sleights can be found in the Mind Hacks chapter, p. 216. 

\subsection{Game rules summary} \label{sec:game-rules-summary} 

Everything you need to know about the rules—summed up on a single page. 

\subsubsection{Making tests (P. 115)} 

\begin{itemize} \item Roll d100 (two ten-sided dice, read as a percentile amount, from 00 to 99). \item Target number is determined by the appropriate skill (or occasionally an aptitude). \item Difficulty is represented by modifiers. \item 00 is always a success. \item 99 is always a failure. \item Margin of Success of 30+ is an Excellent Success. \item Margin of Failure of 30+ is a Severe Failure. \item A roll of doubles (00, 11, 22, 33, etc.) equals a critical success or failure. \end{itemize} 

\subsubsection{Success test (P. 117)} 

\begin{itemize} \item To succeed, roll d100 and score equal to or less than the skill +/– modifiers. \end{itemize} 

\subsubsection{Opposed test (P. 119)} 

\begin{itemize} \item Each character rolls d100 against their skill +/– modifiers. \item The character who succeeds with the highest roll wins. If both characters fail, or both succeed but tie, dead- lock occurs. \end{itemize} 

\subsubsection{Simple success test (P. 118)} 

\begin{itemize} \item Simple Success Tests automatically succeed. \item Success or failure on the roll simply indicates if the character succeeded strongly or poorly. \end{itemize} 

\subsubsection{Defaulting (P. 116)} 

\begin{itemize} \item If a character does not have the appropriate skill for a test, they may default to the skill’s linked aptitude. \end{itemize} 

\subsubsection{Modifiers (P. 115)} 

\begin{itemize} \item Modifiers always affect the target number (skill), not the roll. \item Modifiers (positive or negative) come in 3 levels of severity: \begin{itemize} \item Minor (+/–10) \item Moderate (+/–20) \item Major (+/–30) \end{itemize} \item The maximum modifiers that can be applied are +/– 60. \end{itemize} 

\subsubsection{teamwork (P. 117)} 

\begin{itemize} \item One character is chosen as the primary actor; they make the test. \item Each helper character adds a +10 modifier (max. +30). \end{itemize} 

\subsubsection{Taking the time (P. 118)} 

\begin{itemize} \item Character may take extra time to complete an action. \item On Complex actions, each minute taken adds +10 to the test. \item On Task actions, every 50 percent extension to the timeframe adds +10 to the test. \end{itemize} 

\subsubsection{Aptitudes (P. 123)} 

\begin{itemize} \item Aptitudes range from 1 to 30 (average 15). \item Aptitudes are: Cognition, Coordination, Intuition, Reflexes, Savvy, Somatics, and Willpower. \end{itemize} 

\subsubsection{Learned skills (P. 123)} 

\begin{itemize} \item Skills range from 1-99 (average 50). \item Each skill is linked to and based on an aptitude. \item Morphs, gear, drugs, etc. may provide skill bonuses or penalties to individual skills. \end{itemize} 

\subsubsection{Specializations (P. 123)} 

\begin{itemize} \item Specializations add +10 when using a skill for that area of concentration. \item Each skill may have only one specialization. \end{itemize} 

\subsubsection{Action turns (P. 120)} 

\begin{itemize} \item Action Turns are 3 seconds in length. \item The order in which characters act is determined by Initiative. \item Automatic actions are always "on." \item Characters may take any number of Quick Actions in a Turn (minimum of 3), limited only by the gamemaster. \item Characters may only take a number of Complex Actions equal to their Speed stat. \end{itemize} 

\subsubsection{Task actions (P. 120)} 

\begin{itemize} \item Task Actions are any action that requires longer than 1 Action Turn to complete. \item Task Actions list a timeframe (anywhere from 2 Turns to 2 years). \item Timeframe reduced by 10\% for each 10 points of MoS. \item If character fails, they work on the task for a minimum period equal to 10\% of the timeframe for each 10 points of MoF before realizing it's a failure. \end{itemize} 



