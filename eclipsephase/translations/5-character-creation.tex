\chapter{Création de personnage} \label{chap:character-creation} 

\section{Génération de personnage} \label{sec:character-creation} 

Chaque personnage joueur est composé de deux parties. La première est l'ensemble des nombres et des attributs qui définissent ce à quoi un personnage est bon ou mauvais (ou même ce qu'il peuvent ou ne peuvent pas faire). Ils sotn un peu plus que de simple statistiques cependant - ces caractéristiques aident à définir les capacités de votre personnage et ses intérêts et, par extension, son passé, son éducation, son entraînement et son éducation. Pendant le processus de création de personnage, vous aurez la possibilité d'assigner, d'ajuster et de jongler avec ces stats comme vous le désirez. Si vous avez une notion pré-conçues de ce qu'est votre personnage, vous pouvez optimiser les stats pour le refléter. Alternativement, vous pouvez bidouiller les stats jusqu'à obtenir quelque chose que vous aimez, pusi baser l'histoire de votre personnage sur ce que vous avez développé. 

La deuxième partie de chaque personnage joueur est sa eprsonnalité. Qu'est-ce qui le/la définit en temps que personne? Qu'est-ce qui le/la motive? Qu'es-ce qui l'énerve? Qu'est-ce qui attire son attention? Quels aspect de sa personnalité le/la rend attractif/ve en tant qu'ami/e, camarade ou amant/e - ou au moins quelqu'un d'intéressant avc qui jouer? Quels défaut de personnalité ou quelle bizarrerie il/elle a? Ces questions importent car elles vont également vous guider lorsque vous choisirez les stats, les compétences et les traits. 

La génération de personnage est un procédé qui se fait étape par étape. Contrairement à d'autres jeu, le procédé pour générer un personnage d'Eclipse Phase n'est pas aléatoire - vous avez un contrôle total sur tous les aspects de la conception de votre personnage. Certaines étapes doivent être réalisée avant de pouvoir passer ausx suivantes. Les étapes du processuss complet sont détaillée dans la barre latrale Guide à la Création de Personnage Par Étapes. 

\subsection{Guide à la Création de Personnage Par Étapes} 

\begin{enumerate} \item Définir le Concept du Personnage (p. 130) \item Choisir un Historique (p. 131) \item Choisir une Faction (p. 132) \item Penser els Po ints Gratuits (p. 134) \begin{itemize} \item 105 points d'aptitudes \item 1 point de Moxie \item 5 000 crédit \item 50 Rep \item Langue Natale\end{itemize} \item Dépenser les POints de Personnalisation (p. 135) \begin{itemize} \item 1 000 PP à dépenser\begin{itemize} \item 15 PP = 1 Moxie \item 10 PP = 1 point d'aptitude \item 5 PP = 1 exploit psi \item 5 PP = 1 spécialisation \item 2 PP = 1 point de compétence (61-80) \item 1 PP = 1 point de compétence (up to 60) \item 1 PP = 1 000 crédit \item 1 CP = 10 rep \end{itemize} \item Minimum de points de a investir en compétence active: 400 skill points \item Minimum de point à investir en compétence de connaissance: 300 skill points \item Choisir une morph de départ (pp. 136 et 139) \item Choisir des Traits (pp. 136 et 145) \end{itemize} \item Acheter du Matériel (p. 136) \item Choisir des Motivations (p. 137) \item Calculer les Stats Restantes (p. 138) \item Détailler le Personnage (p. 138) \end{enumerate} 

\subsection{Concept de Personnage} \label{character-concept} 

Décider ce que/qui vous voulez jouer avant de créer le personnage est habituellement le meilleur chemin. Choisissez un archétype simple qui correspond à votre personnage, et partez de là. Voulez-vous jouer un explorateur? Quelqu'un de sournois, comme un espion ou un voleur? Quelqu'un de cérébral, comme un scientifique? Un criminel endurci ou un ex-flic? Ou préférez-vous être un agitateur démagogique? Vous pouvez aussi démarrer avec une personnalité type et choisir une profession associée. Si vous voulez un papillon social qui excelle dans la manipulation des personnes, vous pouvez jouer une personnalité des médias, un blogueur ou un socialite fétard. Peut-être que vous préférerez un rebut de la société avec des problèmes de drogues, auquel cas un ex-mercenaire ou un ancien hypercapitaliste qui a perdu sa fortune et sa famille pendant la Chute pourrait correspondre. Et que pesnez-vous d'un personnage énergétique, profitant de la vie au maximum et qui doit absolument voir tout ce qu'il y a à voir? Un adepte de la course libre ou un resquilleur professionnel pourrait alors bien être ce que vous cherchez. 

Soyez sûr de vous concerter avec les autres joueurs et d'essayer de faire un pesonnage qui est complémentaire au reste de l'équipe - de manière préférable, un personnage qui fournit un ensemble de compétences dont manque le groupe. POurquoi créer un chercheur en archéologie si quelqu'un d'autre a déjà décider qu'il en ferait un, particulièrement quand l'équipe manque d'un bon spécialsite du combat ou d'un async? D'un autre côté, si votre équipe va s'embarquer dans une expédition d'archéologie étrnagère, avoir plus d'un chercheur (chacun avec des domaines d'expertise différent) pourrait ne pas être une mauvaise chose. 

Une fois que vous avez le concept de base, essayez de l'étoffer de quelques détails, pour en faire un résumé en une phrase. Si vous avez commencé avec le concept de "xéno-sociologue," étendez-le à "linguiste amateur ouvert d'esprit et expert en xéno-sociologie fasciné par les cultures étrangères, la collection d'objet kitsch des Facteurs, possède une haute-tolérance au 'facteur beurk' et dont les meilleurs amis tendent à être des élevés et des IAs." Cella vous donnera un peu plus de détails sur lesquels vous pourrez concentrez les forces et faiblesses de votre personnage. 

\subsection{Choisir un historique} \label{choose-background} 

La première étape de la création de votre personnage est de lui choisir un historique. Votre personnage est-il/elle né/née sur Terre avant la Chute? A-t-il/elle été élevé/élevée sur un habitat communautaire? Ou a-t-il/elle commencé son existence en tant su'IA désincarnée? 

Vous devez choisir l'historique de votre personnage parmi ceux de la liste ci-dessous. Choisissez sagement, car chaque historique peut fournir à votre personnage certaines compétences, traits, limitations ou d'autres caractéristiques avec lesquelles débuter. Gardez en tête que votre historique représente d'où vous venez, non pas ce que vous êtes maintenant. C'est le passé, alors que votre faction représente ceux avec qui vous êtes actuellement alignés. Votre futur, bien entendu, est ce que vous déciderez d'en faire. 

Les options d'historiques présentées ci-dessous coouvrent une large sélection de la transhumanté, mais ils ne couvrent pas toutes les possibilités. Si votre maître de jeu vous y autorise, vous pouvez travailler ensemble au développement d'un historique qui n'est pas inclut sur cette liste, en s'en servant comme base pour maintenir l'équilibre. 

\subsubsection{Dériveurs} \label{sec:drifters} 

Vous avez été élevés avec un groupe social qui est resté en mouvement à travers tout le système Sol. Cela peut-être des mibre marchands, des pirates, des fermiers d'astéroïdes, des récupérateurs ou juste des travailleurs migrants. Vous êtes habitués aux voyages spatiaux erratiques entre les habitats et les stations. 

Avantages: +10 compétence Navigation, +20 compétence Piloter: Vaisseau, +10 compétence Réseau: [Domaine] de votre choix. 

Désavantages: Aucun 

Morphs Communes: toutes, particulièrement les Bounceurs et les Hibernoïdes. 

\subsubsection{Évacué de la Chute} \label{sec:fall-evacuee} 

Vous êtes nés et avez été élevés sur terre et évacué pendant les horreurs de la Chute, abandonnant votre ancienne vie (et probablement vos amis, votre famille et ceux que vous aimiez). Vous avez été suffisament chanceux pour survivre en conservant votre corps et à vous en sortir seul dans le système. 

Avantages: +10 compétence Piloter: véhicules terrestre, +10 compétence Réseau: [Domaine] de votre choix, +1 Moxie. 

Désavantages: Seulement 2 500 crédit de départ (il reste possible d'acheter des crédits avec des PP) 

Morphs Communes: Plates, Spliceurs 

\subsubsection{Hyperélite} \label{sec:hyperelite} 

Vous avez eu le privlège d'être élevé en tant que membre de la haute société immortelle qui dirige la plupart des habitats du système intérieur et des hypercorps. Vous avez été dorloté par une fortune et une influence dont la plupart des gens ne peuvent que rêver. 

Avantages: +10 compétence Protocole, +10 000 Crédit, +20 compétence Réseau: Hypercorp 

Désavantages: Ne peut démarrer ni avec une morph flat, spliceur ou dans un pod, ni avec des morphs élevée ou syntéhtique. 

Morphs Communes: Exaltés, Sylphs. 

\subsubsection{Infolife} \label{sec:infolife} 

Vous avez démarrer votre existence en tant que conscience numérique - une intelligence artificelle généraliste (IAG). Votre existence même est illégalle dans certains habitats (un héritage de ceux qui accusent les IAs d'être responsables de la Chute). Contrairement aux IAs germe responsable de leur Chute, votre capacité d'auto-amélioration est limité, bien que vous soyez complètement autonome. 

Avantages: +30 compétences Interfaçage, compétences Informatique (Infosec, Interfaçage, Programmation, Recherche) achetées avec des Points de Personnalisation a moitié prix. 

Désavantages: trait Naïveté du Monde Réel, trait Stigmatisation Sociale (IAG), ne peux pas acheter le trait Psi, les compétences Sociales achetées avec les Points de Personnalisations sont au double du prix normal. 

Morphs COmmunes: Informorphs, morphs synthétiques. 

\subsubsection{Isolé} \label{sec:isolite} 

Vous avez été élevés en tant que membre d'un groupe d'exilés volontaires aux limites du système. Que vous ayez été élevés en tant que membre d'un groupe religieux, d'un culte, d'une expérimentation sociale, d'une cellulle luddiste ou d'un groupe qui voulait simplement être isolé, vous avez passé la plupart , si ce n'est l'intégralité, de votre éducvation isolé des autres factions. 

Avantages: +20 à deux compétences de votre choix 

Désavantages: -10 à la rep de départ 

Morphs Communes: Toutes 

\subsubsection{Égaré} \label{sec:lost} 

Vous êtes l'éhritage de l'une des débacles les plus infamantes depusi la Chute. En tant que mebre de la "Génération Égarée," vous avez subit une enfance en croissance accélérée, survivant d'une manière ou d'une autre là où les autres de votre espèces sont morts, sont devenus fous ou ont été persécutés (voir Les Égarés, p. 233). Votre passé est une stigmatisation sociale, mais elle vous fournit certains avantages ... et fardeaux. 

Avantages: +20 à deux compétences de Connaissance de votre choix, trait Psi. 

Désavantages: trait Désordre Mental (choisissez en deux), trait STigmatisation Sociale (Égaré), doit démarrer avec une morph Futura. 

Morphs Communes: Futura 

\subsubsection{Colon Lunaire} \label{sec:lunar-colonist} 

Vous avez vécu votre enfance dans l'une des étroites cités dômes ou stations sous-terraine de la Lune, la lune de la Terre. Vous étiez aux premières loges pour voir la Chute de la Terre. 

Avantages: +10 compétence Piloter: véhicules terrestre, +10 à une compétence Technique, Académique: [domaine] ou Profession: [domaine] de votre choix, +20 à la compétence Réseau: Hypercorp. 

Désavantages: Aucun 

Morphs Communes: Plates, Spliceurs 





\subsubsection{Martien} \label{sec:martian} 

Vous avez été élevé dans l'une des stations sur ou au-dessus de Mars, devenue la planète la plus peuplée du système. Votre ville d'origine peut avoir survécu, ou non, à la Chute. 

Avantages: +10 compétence Piloter: véhicules terrestre, +10 à une compétence Technique, Académique: [domaine] ou Profession: [domaine] de votre choix, +20 à la compétence Réseau: Hypercorp. 

Désavantages: Aucun 

Morphs Communes: Plates, Spliceurs et Rusteurs. 

\subsubsection{Colon Spatial Originel} \label{sec:original-space-colonist} 

Vous, ou vos parents, faisiez parti de la première "génération" de colons/travailleurs envoyé hors de la Terre pour revendiquer une part de l'espace, vous êtes donc familier avec les confins étroits du vol spatial et avec la vie à bord des plus vieilles stations et habitats. En tant que "zéro-un G" (Zéro gravité, première généréation), vous n'avez jamais fait partie de l'élite. Les personnes ayant votre passé onttypiquement une sorte d'entrainement technologique spécialisé en tant que taravilleur du vide ou de technicien d'habitats. 

Avantages: +10 compétence Piloter: vaisseaux ou Chute Libre, +10 à une compétence Technique, Académique: [domaine] ou Profession: [domaine] de votre choix, +20 compétence Réseau: [domaine] de votre choix. 

Désavantages: Aucun 

Morphs Communes: Toutes L'utilisation de morph exotique est courant. 

\subsubsection{Ré-instantié} \label{sec:re-instantiated} 

Vous êtes nés sur Terre et y avez grandi, mais vous n'avez pas survécu à la Chute. Tout ce que vous savez c'est que votre corps est mort là-bas, mais que votre suavegarde a été transmise hors-monde, et vous êtes l'un des rares chanceux à être ré-instantié avec une nouvelle morph. Vous avez pu passer des années en mémoire morte, en simulspace ou en tant qu'esclave infomorph. 

Avantages: +10 compétence Piloter: véhicules terrestre, +10 compétence Réseau: [Domaine] de votre choix, +2 Moxie. 

Désavantages: trait Mémoire Éditée, 0 crédit de dpéart( il reste possible d'acheter des crédits avec des PP) 

Morphs Communes: Boîtiers, Infomorphs, Synths. 

\subsubsection{Racaille-né} \label{sec:scumborn} 

Vous avez été élevés dans le style de vie nomadique et chaotique commun dans les barges racailles. Advantages: +10 compétence Persuasion ou Supercherie, +10 compétence Escamotage, +20 compétence Réseau: Autonomistes.  Désavantages: Aucun. 

Morphs Communes: toutes, particulièrement les Bounceurs. 

\subsubsection{Élevé} \label{sec:uplift} 

Vous n'êtes mêmem pas humain. Vous êtes nés en temps qu'animal élevé: chimpanzée, gorille, orang-outans, perroquets, corneille, corbeau ou poulpe. 

Avantages: +10 compétence Esquive, +10 compétence Perception, +20 à deux compétences de Connaissances de votre choix. 

Désavantages: Vous devez choisir une morph élevé au départ. 

Morphs Communes: Néo-Aviens, Néo-Hominidés, Octomorph. 

\subsection{Choisir une Faction} \label{sec:choose-faction} 

Après avoir déterminé votre historique, vous devez maintenant choisir la faction principale à laquelle appartient votre personnage. Cette faction représente à priori le groupe qui contrôle actuellement l'habitat ou la station où vit votre personnage, et à laquelle votre personnage à préter allégeance, mais ce n'est pas forcément le cas. Vous pouvez être un membre dissident de votre faction, vivant parmi eux mais s'opposant à certains (ou tous) leurs mêmes centraux et probablement jouant aux agitateurs. Quelque soit le cas, votre faction définit comment votre personnage se représente dans la lutte entre sue se livrent les idéologies post-Chute. 

Vous devez choisir l'une des factions de la liste ci-dessous. Comme l'historique de votre personnage, la faction donenra à votre personnage certaines compétences, traits, limitations ou d'autres caractéristiques. 

Les factions présentées ici décrivent les factions les plus nombreuses et influentes de la traanshumanité, mais d'autres peuvent également exister. À la discrétion de votre maître de jeu, vous pouvez développer ensemble une autre faction de départ non incluse dans cette liste. 

\subsubsection{Anarchiste} \label{sec:anarchist} 

Vous êtes opposés à la hiérarchie, favorisant les oragnisation sociales horizontale et la prise de décision en démocratie directe. Vous coyez que le pouvoir corrompt systématiquement et que tout le monde devrait avoir son mot à dire dans les décisions qui affectent leur vie. D'après les politiques primitives et restrictives du système intérieur et de la Junte Jovienne, cela fait de vous un truand irresponsable au mieux et un terroriste au pire. de votre point de vue, ce n'est que de la comédie ventant de gouvernements qui maintiennent leur population en ordre grâce à l'oppression économique et des menaces de violences. 

Avantages: +20 à une compétence de votrre choix, +30 compétence Réseau: Autonomistes. 

Désavantages: Aucun 

Morphs Communes: Toutes 

\subsubsection{Argonaute} \label{sec:argonaut} 

Vous faites parti d'un mouvement scientifique et techno-progressiste qui cherche à résoudre les injustices et les inégalités de la transhumanité par la technologie. Vous défendez l'accès universel à la technologie et aux soins, les modèles de production open source, la liberté morphologique et la démocratisation. Vous essayez d'éviter les politiques factionaliste qui entraîne la division, considérant la sépration de la transhumanité comme une gêne à sa survie. 

Avantages: +10 à deux compétences Technique, Académique: [domaine] ou Profession: [domaine]; +20 compétence Réseau: Scientifiques 

Désavantages: Aucun 

Morphs Communes: Toutes 

\subsubsection{Barsoomien} \label{sec:barsoomian} 

Votre foyer est l'arrière pays sauvage Martien. Vous êtes un "redneck," un membr de la classe sociale Martienne basse des zones rurales qui se trouvent régulièrement en conflit avec les politiques et les objectifs des dômes hypercorps et avec la Ligue Tharsis. 

Avantages: +10 Parkour, +10 à une compétence de votre choix, +20 compétence Réseau: Autonomistes. 

Désavantages: Aucun 

Morphs Communes: Boîtiers, Plates, Rusteurs, Spliceurs, Synths. 

\subsubsection{Bordés} \label{sec:brinker} 

Vous, ou votre faction, rechigne à traiter avec le reste de la transhumanité et à s'impliquer dans les différentes affaires en cours dans le reste du système. Votre groupe particulier pourrait avoir cherché à s'imposer l'isolation afin de poursuivre leurs propres intérêts, ou avoir été exilés en raisons de leurs croyances impopulaires. A moins que vous ne soignez simplement un solitaire qui préfère les vastes étendues vide de l'espace a socialiser avec les autres. Vous pourriez être un religieux dans un culte primitiviste, un utopiste ou quelque chose dont la transhumanité ne veut pas. 

Avantages: +10 compétence Piloter: Vaisseau, +10 à une compétence de votre choix, +20 à une compétence Réseau: [doamine] de votre choix. 

Désavantages: Aucun 

Morphs Communes: Toutes 

\subsubsection{Criminel} \label{sec:criminal} 

Vous êtes impliqué dans l'univers criminel du monde sous-terrain. Vous pouvez travailler avec l'une des factions majeures du système Sol - les traides, le night Cartel, l'ID Crew, les Nine Lives, la Familiae - ou avec l'un des opérateurs locaux se concentrant sur un habitat particulier. Vous pouvez être un membre à vie, une recrue réticente ou juste un indépendant attendant le prochain job. 

Avantages: +10 compétence Intimidation, +30 compétence Réseau: Criminel 

Désavantages: Aucun 

Morphs Communes: Toutes 

\subsubsection{Extropien} \label{sec:extropian} 

Vous êtes un supporters anarchiste du marché libre et de la propriété privée. Vous vous opposez au gouvernement et favoriser un système où la sécurité et les affaires légales sont gérés par des compétiteurs privés. Que vous vous considériez comme un anarcho-capitaliste ou comme un mutualiste (une différence que seuls les autres Extropiens peuvent faire), vous occupez une zone intermédiaire entre les hypercorps et les autonomistes, traitant avec les deux mais considérés par aucun. 

Avantages: +10 compétence Persusasion, +20 compétence Réseau: Autonomistes, +10 compétence Réseau: Hypercorps. 

Désavantages: Aucun 

Morphs Communes: Toutes 

\subsubsection{Hypercorp} \label{sec:hypercorp} 

Vous êtes originaire d'un habitat contrôllé par les hypercorporations. You pouvez être un entrepreneur hypercapitaliste, un socialite hédoniste or un travailleur du vide, mais vous acceptez le fait que certaines libertés doivent être sacrifiées pour la sécurité. 

Avantages: +10 compétence Protocole, +20 compétence Réseau: Hypercorps, +10 à n'imprte quelle compétence Réseau: [domaine]. 

Désavantages: Aucun 

Morphs Communes: Exaltés, Olympiens, Spliceurs, Sylphs. 

\subsubsection{Joviaen} \label{sec:jovian} 

Votre faction est connue pour son régime autoritaire, ses idéologies bio-conservatives et ses tendances militaires. De là où vous venez, personne ne fait confiance à la technologie et les humains ont besoin d'être protégés d'eux-mêmes. Pour assurer sa survie, l'humanité doit être capable de se défendre elle-même, et toute croissance sans entrave doit être vérifiée. 

Avantages: +10 à deux compétences d'armes de votre choix, +10 Esquive, +20 compétence Réseau: Hypercorps. 

Désavantages: Doit commencer avec un morph Flat ou Spliceurs, ne peut pas commencer avec du nanoware ou de la nanotechnologie avancée. 

Morphs Communes: Plates, Spliceurs 

\subsubsection{Lunaire} \label{sec:lunar} 

Vous venez de la Lune, le premier monde colonisé hors-Terre. Maintenant surpeuplée et en déclin, la Lune est l'une des rares endroits où les gens peuvent toujours se cramponner aux anciennes identitées ethnique est nationnale de la Terre. Votre maison est également en vue de la Terre, un souvenir permanentqui encourage beaucoup de "Lunistes" à être des Réclamationniste, déplorant l'interdiction hypercorporatiste et arguant que vous devriez avoir le droit de retourner sur la Terre, la terraformer et s'y ré-établir en temps que monde mère. 

Avantages: +10 à une compétence Langue: [domaine] de votre choix, +20 compétence Réseau: Hypercorp, +10 compétence Réseau: Écologiste. 

Désavantages: Aucun 

Morphs Communes: Boîtiers, Plates, Exaltés, Spliceurs, Synths. 

\subsubsection{Mercurien} \label{sec:mercurial} 

Votre faction ne s'inétresse pas à renier leur vrai nature pour devenir plus "humain." Que vous soyez une IAG qui ne mélange pas forcément sa destinée avec celle de la transhumanité ou un élevé qui cherche à préserver et à protéger la vie non-humaine (ou au moins sa propre espèce). Vous pouvez même être une infomorph ou un posthumain qui s'est tellement éloigné des intérêts et des valeurs transhumains que vous vous considérez maintenant comem forgeant une nouvelle forme de vie unique. 

Avantages: +10 à deux compétences de votre chois, +20 à une compétence Réseau: [domaine] de votre choix. 

Désavantages: Aucun 

Morphs Communes: Informorphs, Synths, morphs élevés. 

\subsubsection{Racaille} \label{sec:background-scum} 

C'est le futur que nous attendions tous, et vous en profitez au maximum. Un changement de apradigme s'est produit, et pendant que tous les autres s'en remettent, votre faction l'a embrasser et s'y est révélée. Il n'y a plus d'envie, plus de morts, plus de limite sur ce que vous pouvez être. La racaille s'est immergée dans un nouveau mode de vie, se changeant comme ils le veulent, essayant de nouvelles expériences et repoussant les limites partout où ils le peuvent ... et emmerdent tout ce qui ne peuvent pas le supporter. 

Avantages: +10 compétence Chute Libre, +10 à une compétence de votre choix, +20 compétence Réseau: Autonomistes. 

Désavantages: Aucun 

Morphs Communes: Toutes 

\subsubsection{Socialite} \label{sec:socialite} Vous faites parti des célébrités du système intérieur, de la clique social saturée de média qui définissent les modes, qui propagent les mêmes et qui font et défont des vies d'un murmure, d'une allusion ou d'un arrangement de l'ombre. Vous êtes à la fois une icône et un suivant dévoué. La culture n'est pas seulement votre vie, c'est votre arme de prédilection. 

Avantages: +10 compétence Persusasion, +10 compétence Protocole, +20 compétence Réseau: Média. 

Désavantages: Ne peut démarrer ni avec une morph flat, spliceur ou dans un pod, ni avec des morphs élevée ou syntéhtique. 

Morphs Communes: Exaltés, Olympiens, Sylphs. 

\subsubsection{Titanien} \label{sec:titanian} 

Vous faires parti de la cyberdémocracie socialiste du Commonwealth Titanien. Contrairement à d'autres projets autonomiste, l'effort collecitf Titanien a assemblé quelques projets d'infrastructure impressionant tels qu'approuvés par la Pluralité Titanienne et menés à bien par des microcorporations publique. 

Avantages: +20 à deux compétence Technique ou Académique de votre choix, +20 compétence Réseau: Autonomistes. 

Désavantages: Aucun 

Morphs Communes: Toutes 

\subsubsection{Ultimes} \label{sec:ultimate} 

Votre faction perçoit pleinement le potentiel du futur de la transhumanité et considère le reste de la transhumnité comme faible et hédoniste. La transhumanité est en place pour franchir la prochaine étapes d'évolutions et il est temps pour les transhumains d'être redessiné pour le meilleur de nos capacités. 

Avantages: +10 à deux compétences de votre choix, +20 à une compétence Réseau: [domaine] de votre choix. 

Désavantages: ne peut pas démarrer avec une morph Plates, Spliceurs, Élevées ou dans un pod. 

Morphs Communes: Exaltés, Refaits. 

\subsubsection{Vénusien} \label{sec:venusian} 

Vous êtes un défenseur de la Confédération Morningstar des aérostats vénusiens, pleine de ressentiment vis à vis de l'influence grandissante du Consortium Planétaire et des autres puissance retranchée e conservative du système intérieur. Vous percevez l'ascenssion de votre faction comme une chance de réformer la vieille guarde des politique du système intérieur. 

Avantages: +10 Piloter:Engins aériens, +10 à une compétence au choix, +20 compétence Réseau: Hypercorp. 

Désavantages: Aucun 

Morphs Communes: Boîtiers, Mentons, Exaltés, Spliceurs, Synths, Sylphs. 

\subsection{Dépenser les Points Gratuits} \label{sec:spend-free-points} 

Chaque personnage reçoit au départ un nombre équivalent de points gratuits pour des choses comme la rep et les aptitudes. Ces points gratuits sont le point de départ pour assembler votre personnage, ne vous inquiétez donc pas si vous ne parvenez pas à obtenir des scores aussi élevés que ce que vous aimeriez. Dans la prochaine étape de la création de personnages, vous gagnerez des points supplémentaires avec lesquels vous pourrez personnaliser votre personnage (voir la section Dépenser les Points de Personnalisation, p. 135). 





\begin{quotation} Exemple 

Tai est en train de créer un personnage. Elle décide de créer un charognard/récupérateur de débris qui aurait commencé en tant que Colon Lunaire mais qui serait maintenant un Bordé. À eux deux, sa faction et son historique donennt à Tai +20 à la compétence Réseau: Autonomistes, +20 à la compétence Réseau: Hypercorp, +10 à la compétence Piloter: Vaisseau et +10 à la compétence Piloter: véhicule terrestre. Elle a également +10 dans deux autres compétences (dont l'une étant une Acadméique, une Profession ou un domaine Technique) qu'elle pourra choisir plus tard. Tai démarre avec 105 points pour les aptitudes, qui permettent de mettre 15 points à chaque aptitude. Elle veut que son personnage soit impulsif et antisocial, elle réduit donc immédiatement son AST et sa VOL à 10. Elle veut également ête intelligente et rapide, elle prends donc les 10 points que cela lui a donné pour augmeneter sa COG et ses REF à 20. Ses aptitudes sont donc: 

\begin{center} \begin{tabular}{ccccccc} COG &COO &INT &REF &AST &SOM &VOL \\ 

20 &15 &15 &20 &10 &15 &10 \\ 

\end{tabular} \end{center} 

Elle note son Moxie de 1 et choisit sa langue natale (Chinois) à 85, tout les deux offerts. En notant ses 5 000 crédits, Tai divisent son score de Rep équitablement entre @-rep et c-rep, leur donnant 25 points à chacun. 

Elle à maintenant 1 000 points pour personnaliser son personnage. Elle veut être chanceuse, elle commence donc par dépenser 60 (4 X 15) PP pour augmenter son Moxie de 1 à 5. Elle décide également qu'elle veut que son personnage soit meilleur pour repérer les choses, elle augmente donc son INT de 15 à 10, au coût de 50 PP (5 x 10). Jusqu'à présent, elle a dépenser 110 CP. Elel doit acheter au moins 400 points de compétences Active, elle s'y attaque donc. Elel sait que les compétences sont liées aux aptitudes et qu'elle deviennent plus cher au-delà de 60, elle décide donc que le maximum qu'elle dépensera sur une seule compétence sera 40 (puisque son aptitude la plus élevée est de 20). Elle choisist ses compétences, assigne les points et ajoute les compétences aux aptitudes liées. Voici ce avec quoi elle commence, en notant les points dépensé sur chacune et la valeur totale (incluant l'aptitude) noté entre parenthèses. Armes à Rayons (COO) 30 (45), Escalade (SOM) 30 (45), Démolition (COG) 40 (60), Esquive (REF) 30 (50), Chute Libre (REF) 40 (60), Parkour (SOM) 30 (45), Matériel: Aérospatialle (COG) 40 (60), Infiltration (COO) 30 (45), Interface (COG) 20 (40), Navigation (INT) 40 (60), Perception (INT) 40 (60), PErsuasion (AST) 20 (30), Recherche (COG) 20 (40) et Escamotage (INT) 40 (60). Cela lui coûte 450 PP, elle a donc dépensé un total de 560 PP jusqu'ici. Elle dépense maintenant 300 points de compétences de Connaissances: Académique: Astrophysique (COG) 40 (60), Académique: Ingénierie (COG) 40 (60), Académique: Histoire de la Chute (COG) 40 (60), Art: Sculpture (INT) 40 (60), Intérêt: Statiosn Bordées (COG) 40 (60), Intérêts: Conspirtaion (COG) 30 (50), Langue: Anglais (INT) 40 (60), Profession: Estimation (COG) 40 (60), Profession: Commerce de Récupérateur (COG) 40 (60). 

Cela lui coûte encore 350 PP, amenant le total de PP dépensés à 910. 

En ajoutant ses compétences d'historique et de faction, elle a également Réseau: AUtonomistes (AST) 30 (40), Réseau: Hypercorp (AST) 30 (40), Piloter: Vaisseau (REF) 30 (50), Piloter: Engins terrestres (REF) 30 (50). Elle prend le bonus de +10 et l'attribue à Esquive (l'amenant à 60) et applique l'autre +10 à Académique: Économie (COG) 30. 

Avec les 90 PP restants, Tai s'attaque à la Rep. Tai veut avoir beaucoup de bonnes connexions, elle augmente donc ses deux score de rep de 30 points chacun, pour un coût de 6 PP. Elle décide également qu'elle a besoin d'un peu de crédibilité avec les criminels, elle achète donc la g-rep à 40 pour 4 PP de plus. Il lui reste donc 80 PP. 

Le personnage de Tai a besoind 'un corps, et elle décide que le bounceurs est le plus adapté au style de vie nomade et spatial de son bordé. Cela lui co ûte encore 40 PP, la liassant avec 50 PP a dépenser. 

En reconsidérant ses compétences, elle décide de développer sa compétence Piloter: Vaisseau de 50 à 70. Cela lui coute 10 PP pour l'amener à 60, puis 20 de plus pour la pousser jusqu'à 70, pour un coût total de 30 PP. Elel veut également développer sa compétence Escamotage de 60 à 70, pour un coût de 20 PP. Cela utilise bien ses derneirs PP. 

En regardant les Traits, Tai décide que Conscience Spatiale pourrait être un bon choix pour son charognard. A un coût de 10 PP, ele aura besoin de prendre un autre trait négatif pour compenser. Elle choisit Séquelle Neuronale (Synaesthesie) - une maladie qu'elle a récupéré d'un nanovirus déchaîné pendant la Chute. 

Le spoints de Tai sont maintenant tous dépensés. \end{quotation} 

\subsubsection{Aptitudes de Départ} \label{sec:starting-aptitudes} 

Votre personnage reçoit 105 points gratuits à distribuer parmi les ses 7 aptitudes: Cognition, Coordination, Intuition, Réflexes, Astuce, Somatique et Volonté (voir Aptitudes p. 123). (Cela se divise à une moyenne de 15 points par aptitude, il peut donc être plus simple de donner 15 points à chacune et ensuite d'ajuster en fonction, en augmenter certaines, en en réduisant d'autres.) Chaque aptitude doit recevoir au moins 5 points (à moins que vous n'ayez choisi le trait Frêle, voir p. 149), et aucune aptitude ne peut dépasser 30 points (sauf si vous avez choisit le trait Aptitude Esxceptionnelle p. 146). Notez que certaines morphs (plates et spliceurs par exemple) peuvent également mettre plafonner le maximum de vos aptitudes (voir Maximum d'Aptitudes, p. 124). 

Pour simplifier le tout, il est recommandé que les scores d'aptitudes soient gérées par multiples de 5, mais ce n'est pas une nécessité. 

\subsubsection{Langue Natale} \label{sec:native-tongue} 

Chaque personnage reçoit la compétence compétence de leur Langue naturelle a un niveau de 70 + INT gratuitement. Cette compétence peut-être améliorée avec des PP (voir plus bas). 

\subsubsection{Moxie de Départ} \label{sec:starting-moxie} 

Chaque personnage commenace abec une stat Moxie de 1 (voir Moxie, p. 122). 

\subsubsection{Crédit} \label{sec:starting-credit} 

Tout les personnages reçoivent 5 000 crédits avec lesquels acheter du matériel pendant la création de personnage, à moins que vous n'ayez l'historique Évacué de la CHute ou Ré-Instantié (auquel cas vous démarrez avec 2 500 ou 0 crédits, respectivement). Voir Acheter du Matériel, p. 136, pour de plus amples détails. 

\subsubsection{Rep} \label{sec:starting-rep} Votre peronnage n'est pas un débutant complet. Vous recevez 50 points de rep à répartir entre les différentes réseaux de réputation de votre choix (voir Réputation et Réseaux Sociaux, p. 285). 

\subsection{Dépenser les Points de Personnalisations} \label{sec:spend-customization-points} Maintenantq ue vous avez les bases de votre eprsonnage définie, vous pouvez dépenser des Points de Personnages (PP) pour détailler finement votre personnage. Chaque personnage reçoit 1 000 PP qui peuvent être utilisé pour augmenter les aptitudes, acheter des compétences, acquérir plus de Moxie, acheter plus de crédit, élever votre rep ou acheter des traits positifs. Vous pouvez également prendre des traits négatifs pour gagner encore plus de PP avec lesquels personnaliser votre personnage. Ce process de personnalisation devrait être utilisé pour bidouiller votre personnage et le spécialiser de la manière dont vous le désirez. 

Si un maître de jeu désire un autre niveau de jeu, le total de PP peut être ajusté. Pour un scénario dans lequel les personnages sont plus jeunes ou moins expérimenté, le nombre de PP peutêtre réduit à 800 ou même 700. D'un autre côté, si vous voulez créer des personnages qui démaare comme des vétérans endurcis, le nombre de PP peut être augmenter à 1 100 ou 1 200. 

Toutes les personnalisations ne sont pas éagles - les aptitudes, par exemple, sont bien plus importante que les compétences individuelles. POur refléter ça, les PP doivent être dépenser à un taux spécifique en fonction des améliorations voulues. 

\begin{quotation} Point de Personnalisation \begin{itemize} \item 15 PP = 1 point de Moxie \item 10 PP = 1 point d'aptitude \item 5 PP = 1 exploit psi \item 5 PP = 1 spécialisation \item 2 PP = 1 point de compétence (61-80) \item 1 PP = 1 point de compétence (jusqu'à 60) \item 1 PP = 1 000 crédit \item 1 PP = 10 rep \end{itemize} \textit{ Le coût des traits et des morphs varient comme spécifié à chaque fois.} \end{quotation} 

\subsubsection{Personnaliser les APtitudes} \label{sec:customizing-aptitudes} 

Augmenter votre score d'Aptitude est relativement cher et coûte 10 PP par points d'aptitudes. Comme noté au-dessus, aucune aptitude ne peut être augemntée au delà de 30. Gardez en tête que votre morph peut également vous fournir certains bonus d'aptitude. 

\subsubsection{Augmenter le Moxie} \label{sec:increasing-moxie} 

Le Moxie peut être élevée au coût de 15 PP par points de Moxie. Le niveau maximum auquel le Moxie peut-être élevé est de 10. 

\subsubsection{Compétences Apprises} \label{sec:buying-learned-skills} 

Chaque eprsonnage doit acheter un minimum de 400 points de compétences en compétences Active et 300 en compétences de Connsaissance (voir Compétences, p. 170). Les compétences sont achetées au coût d'1 PP par point. Gardez en tête que les compétences apprises commencent au nivau de l'aptitude liée. Par exemple, si vous voulez augmenter une compétences à 30 et que l'aptitude liée est à 10, vous devrez dépenser 20 PP. Le sbonus aux compétences venant de l'historique ou de la faction doivent également être appliquée à la compétence avant de commencer à l'augmenter. Dans un but de simplification, il est recommandé que les compétences soient achetées par multiple de 5, mais ce n'est pas une nécessité. Augmenter une compétence au-delà de 60 ets cher. Chaque point au delà de 60 coûte double. Augmenter une compétence ayant une aptitude liée de 20 jusqu'à 70 coûte 60 PP. 40 points pour aller de 20 à 60, et 20 de plus pour aller de 60 à 70. Aucune compétence appris ene peut dépasser les 80 pendant la création de eprsonnage (sauf si vous possédez le trait Expert, p. 146). Même si les compétences de Connaissances sont regroupées en 5 compétences, chacune d'entre elle est une compétence à domaine (p. 172) ce qui implique qu'elles peuvent être choisie plusieurs fois pour différents domaines. Une liste complète de compétences peut être trouvée à la p. 176. 

\subsubsection{Spécialisations} \label{sec:buying-specializations} 

Des spécialisations (P. 173) peuvent également être achetée au coût de 5 PP par spécialisation. Vous pouvez acheter des spécialisatsion à la fois pour les compétences Active ou de Connaissances. Une seule spécialisation ne peut être achetée, et elles ne peuvent l'être que pour les compétences qui ont un niveau de 30+. 

\subsubsection{Acheter Plus de Crédit} \label{sec:buying-credit} 

Si vous voulez plus de crédit à dépenser sur votre matériel, chaque PP vous rapportera 1 000 crédits. Voir Obtenir du Matériel, p. 136, pour les détails sur l'achat de matériel. Vous ne pouvez dépenser plsu de 100 PP pour obtenir des crédits supplémentaires. 

\subsubsection{Augmenter la Rep} \label{sec:increasing-rep} 

Si vous voulez que votre personnage commence le jeu avec plein de capital social, vous pouvez augmenter votre/vos score(s) de rep au coût d'1 PP pour 10 popints additionels. Aucun score individuel de Rep ne peut dépasser les 80? et le montant maximum de PP a dépenser sur la Rep est de 35 points. 

\subsubsection{Morph de Départ} \label{sec:starting-morph} 

Probablement l'utilisation la plus importante des PP est d'acheter la morph avec laquelle votre personnage commencera le jeu. Cela peut-être la forme corporelle originale dans laquelle vous avez commencé votre vie, ou simplement l'incarnation que vous habitez actuellement. Les morphs disponibles sont listées  partir de la p. 139. Notez que toutes les bonus aux compétences ou aux aptitudes fournis par la morph sont appliqués après que tous les PP soient dépensés. En d'autres mots, ces bonus n'affectent pas le coût d'achat des aptitudes et des points de compétences pendant la génération de personnage. Aucune aptitude ne peut-être modifiée au-delà de 40. 

\subsubsection{Acheter des Traits} \label{sec:purchasing-traits} 

Les traits représentent des qualités spécifiques de votre personnage qui peuvent l'aider ou le freiner. Les traits positifs fournissent des bonus dans certaines situations, et chacun d'eux à un coût en PP associé. Vous ne pouvez pas dépenser plus de 50 PP sur les traits positifs. Les traits négatifs infligent des désavantages à votre personnage, mais vous rapporte des PP supplémentaire que vous pouvez dépenser pourpersonnaliser votre personnage. Vous ne pouvez pas obtenir plus de 50 PP  de traits négatifs, et pas plus de 25 d'entre eux ne epuevent être des traits négatifs de morphs. Les traits positifs sont listés à la p. 145, les traits ngatifs le sont p. 148. Notez que les traits que vous recevez de votre historique ou de votre faction ne vous coûtent ni ne vous rapportent de PP. les traits listés en tant que traits de morphs s'appliquent à la morph, pas à l'ego. Si un personnage change de morphs, ces traits sont perdus (et de nouveaux traits de morph peuvent être obtenus). Les traits de morph doivent être acheter comme tous les autrs traits pendant la génération de personnage. 

\subsubsection{Exploits Psi} \label{sec:purchasing-psi-sleights} 

Les personnages qui achètent le trait Psi (p. 147) peuvent dépenser des PP pour acheter des exploits (voir Exploit, p. 223). Ils représentent des capacités psi particulière que le personnage a apprise. Le coût d'achat d'un exploit est de 5 PP. Pas plus de 5 exploits psi-chi et 5 exploits psi-gamma ne peuvent être achetés pendant la création de personnage. Notez que chaque bonus de compétence ou d'aptitude obtenus avec des exploits sont traités comme des modificateurs; ils s'appliquent après que tous les PP aient été dépensés et n'affectent pas le coût d'achat des compétences ou des aptitudes pendant la création de personnage. 

\subsection{Obtenir du Matériel} \label{sec:purchase-gear} 

Peu importe de quelel faction vous venez, vous utilisez les Crédits pour acheter du matériel pendant la création de personnage. Une liste complète du matériel et des coûts peuvent être trouvé dans le chapitre Équipement, p. 294. Le coût moyen de chaque catégorie de prix doit être utilisé lorsqu'il s'agît de calculer le prix du matériel. Chaque personnage démarre avec une piède d'équipement gratuitement: une muse standard (p. 332). C'est un compagnon IA nuémrique que le eprsonnage possède depuis qu'il est un enfant. Additionnellement, chaque personnage démarre avec 1 mois d'assurance sauvegarde (p. 330) sans coût supplémentaire. 

\begin{center} \begin{tabular}{|c|c|c|} \hline

\multicolumn{3}{|c|}{Coût du Matériel} \\ \hline

Catégorie &Étendue (Crédits) &M oyenne(Crédits)\\ \hline

Trivial &1-99 &50\\ \hline

Bas &100-499 &250\\ \hline

Modéré &500-1,499 &1,000\\ \hline

Élevé &1,500-9,999 &5,000\\ \hline

Cher &10,000+ &20,000\\ \hline \end{tabular} \end{center} 

Il n'y a pas de limitations autre que celles fixées par le maître de jeu sur le matériel accessible aux personnage à la création de personnage. Les joueurs et le maître de jeu devraient garder en tête l'historique et la faction du personnage. Comme certaines pièces d'équipement sont extrêmement restreintes dans certains habitats voire franchement illégale, il peut être nécessaire d'avoir une explication plausible sur la façon qu'un personnage d'un tel endroit peut obtenir un équipement de ce type. si il n'y a pas d'explicatiosn plausible, le maître de jeu peut choisir de ne pas autoriser cet équipement. Le pointd ed épart de la partie devrait également être pris en considération. Un personnage de la République Jovienne restrictive pourrait avoir des difficultés à expliquer comment il/elle a obtenu une machine d'abondance illégale dans la République, mais si le jeu démarre à bord d'une barge racaille où tout est disponible et où tout est autorisé, alros une telle explication devient bien plsu simple. 

La seule exception à l'achat de matériel avec des crédits est l'acaht de morph supplémentaires. Les personnages peuvent acheter des morphs supplémentaires lors de la création de personnage, mais elles doivent être achetées avec des PP. Le joueur choisit une morph dans laquelle les personange est incarné. Les morpshs upplémenatires éncessitent aussi de payer les services d'une banque de corps (p. 331). 

Notez que tout bonus de compétences ou d'aptitude hérité de l'équipement est traité comme une modification; ils ne sont appliqués qu'après que tous els PP aient été dépensés et n'affectent pas le coût d'achat des compétences ou des aptitudes pendant la création de personnage. 

\subsection{Choisir les Motivations} \label{sec:choose-motivations} 

L'étape suivante est de déterminer 3 motivations personnelles à votre personnage (voir Motivations, p. 121). Ce sont des mêmes, sous la forme d'idéologies ou d'objectifs, que votre personnage cherche à atteindre. Ils peuvent être aussi spécifique que "battre le chef des triades local" ou aussi large que "promouvoir l'hypercapitalisme," et ils peuvent être à court ou long terme. Quelques motivations types sont fournies sur la table d'Exemple de Motivations (p. 138). Vous devriez travaillez avec votre maître de jeu au moment où vous déterminez vos motivations, car elles peuvent être utilisée pout propulser l'histoire et des scénarios spécifiques peuvent être construits autour des buts de votre personnage. Certaines motivations de votre personnage peuvent changer par la suite (voir Changer de Motivation, p. 152). les motivations aiderons votre personnage à récupérer des points de Moxie (p. 122) et à gagner des points de Rez supplémentaires pendant le jeu (p. 384). 

Le smotivations doivent être listée sur votre fiche de personne comme de simples mots ou des phrases trés courtes, accompagné d'un symbole + ou - selon que vous supportiez ou que vous vous opposiez à l'idée. Par exemple "+Notoriété" indiquera que votre personnage cherche à devenir une personnalité médiatique célèbre, alors que "-Récupérer la Terre" signifie que votre personnage s'oppose aux buts des réclamationnistes. 

\subsubsection{Exemple de Motivations} \label{sec:example-motivations} 

\begin{tabular}{lll} Contact Étranger &Anarchisme &Expression Artistique\\ Bioconservatisme &Éducation &Exploration \\ Notoriété &Fascisme &Hédonisme \\ Hypercapitalisme &Immortalité &Libertarianisme \\ Libération Martienne &Liberté Morphologique &Nano-écologie \\ Open Source &Carrière Personnelle &Développement Personnel\\ Philanthropie &Préervationisme &Récupérer la Terre\\ Religion &Recherche &Droits des (IA/Infomorph/Pod/Élevés)\\ Esclavage des (IA/Infomorph/Pod/Élevé) &Socialisme &Techno-Progressivisme \\ Vengeance &Souveraineté Venusienne &Prospérité\\ \end{tabular} 

\subsection{Touches Finale} \label{sec:final-touches} 

maintenant que tout est en place, il reste quelques étapes finales. 

\subsubsection{Stats Restantes} \label{sec:remaining-stats} 

Quelques stats doivent maintenant être calculée et additionnée à votre fiche de personnage: 

\begin{itemize} \item Lucidité (p. 122) est égale à la VOL x 2 de votre personnage. \item Seuil de Trauma (p. 122) est égal à votre LUC divisée par 5 (arrondir au supérieur). \item Seuil de Folie (p. 122) est égal à LUC x 2. \item Initiative (p. 121) est égal à (REF + INT) x 2 de votre personnage. \item Bonus de dommage (p. 123) pour la mélée est égal à SOM $\div$ 10 (arrondir à l'inférieur). \item Seuild e Mort (p. 122) est égal à DUR x 1,5 (biomorphs, arrondir au supérieur) ou à DUR x 2 (synthmorpsh). \item Vitesse (p. 121) est égale à  1 (3 pour les infomorphs), modifié de manière appropriée par les implants. \end{itemize} 

\subsubsection{Détailler le Personnage} \label{sec:detailing-the-character} 

L'étape finale de la création de personnage est de remplir les détails et déterminer comment votre personnage se comporte et ce qu'il/elle pense. L'historique de votre personnage est un bon point de départ car il définit d'où il/elle vient, mais il peut être développé. Que pesne-t-il/elle de sopn enfance? Y A-t-il/elle encore des attache? Comment il/elle a évolué de son roigine pour arriver dans la Faction dont il/elle fait parti? Est-il/elle un fervent défenseur des objectifs de sa Faction ou est-il/elle en opposition? Comment le personnage perçoit-il/elle les autres Factions? 

Ensuite, jettez un œil aux compétence et autres points structurants - ils peuvent aussi raconter une histoire. Comment a-t-il/elle acqui ces compétences? Pourquoi? Comment a-t-il/elle développé son score de rep (ou son absence)? Comment a-t-il/elle été connecté avec les groupes détaillés dans leurs compétences Réseau? Que disent les traits du personnage à son sujet? Comment a-t-il/elle obtenu sa morph actuelle? Est-ce que c'est leur morph d'origine? Sinon, qu'est-il arrivé à son premier corps? Prenez aussi en considération les facteurs majeurs des Motivations, toutes ces questiosn pourront vous aider à construire une image définissante de votre personnage. Il n'est aps nécessair de définir intégralement votre personnage bien entendu - des blancs peuvent toujours être laissés pour être remplis plus tard. Assembler les points que vous avez définit jusqu'à présent vous aidera à présenter votre personnage comme un tout, un individu unique plutôt qu'un modèle vide. En guise d'étape finale, prenez quelques minutes pour déterminer quelques caractéristiques identitaires et quelques traits de eprsonnalité qui vous aiderons à définir votre personnage aux autres. Cela pourrait être une manière de parler, un sale caractère, une punchline qu'ils utilisent fréquemment, un style unique, un comportement répétitif, un maniérisme génant ou n'importe quoi d'autre de similaire sur lequel il est facile de s'accorcher. De telles idiosyncratie donnent un support aux autre joueurs afin de développer des opportunités d'interprétation. 

\section{Morphs de Départ} 

Chaque morph est associée à un coût en PP. Elel fournie aussi les stats de \textbf{Durabilité} et de \textbf{Seuil de Blessure} du personnage, et beaucoup modifient également l'Initiative, la Vitesse et certaines aptitudes et compétences apprise. Un coût en crédit est également fourni, mais il fait référence au coût d'achat de ces morphs en cours de jeu. 

Bonus d'Aptitude Souple: Certaines morphs ont des bonus d'aptitudes qui peuvent être appliqu"e à une aptitude au choix du joueur. Cela reflète le fait que toutes les morphs ne sont pas équivalentes. En assignant ces bonus d'aptitude universels, chaque amélioration doit être appliquée à une aptitude différente; vous ne pouvez augmenter une aptitude qui est déjà améliorée par cette morph. Une fois que les bonus aux aptitude d'une morph particulière ont été assignés, ils deviennet permanent pour cette morph (p.ex. si un eutre personnage se réincarne dans cette morph, les bonus restent les mêmes). 

\subsection{Biomorphs} \label{sec:starting-biomorphs} 

Les biomorphs sont des incarnations complètement bioloique (habituellement équipée d'implants), mise au monde naturellement ou par un exo-utérus et amené à l'aâge adulte soit naturellement soit à un rythme senseiblement accéléré. 

\subsubsection{Plates} \label{sec:starting-flats} 

Les plates sont l'humain de base non modifié, né avec tout les défauts naturels, des maladies héréditaires et d'autres mutations génétiques que l'évolution applique avec amour. Les plates sont extrêmement rare - la plupart sont mortes avec le reste de l'humanité pendant la Chute. La pluaprt des nouveaux enfants sont des spliceurs - analysée et réparé génétiquement au minimum - excepté dans les habitats où les plates sont considrés comme des citoyens de seconde zone et des travailleurs contractés. 

\begin{description*} \item[Implants] Aucun \item[maximum d'Aptitude] 20 \item[Solidité] 30 \item[Seuil de Blessure] 6 \item[Désavantages] Aucun (le trait Défaut génétique est commun) \item[Coût en PP] 0 \item[Coût en Crédit] Élevé \end{description*} 

\subsubsection{Spliceurs} \label{sec:starting-splicers} 

Les spliceurs sont les humains génétiquement réparés. Leur génome a été nettoyé des maladies héréditaires et optimisé pour la santé et l'apparence, mais il n'a pas été amélioré outre-mesure. Les spliceurs constituent la majorité de la transhumanité. 

\begin{description*} \item[Implants] Biomods de Base, Inserts Mesh Basiques, Pile Corticale\item[Maximum d'Aptitude] 25 \item[Durabilité] 30 \item[Seuil de Blessure] 6 \item[Avantages] +5 à une aptitude au choix du joueur\item[Coût en PP] 10 \item[Coût en Crédit] Élevé \end{description*} 

\subsubsection{Exaltés} \label{sec:starting-exalts} 

Les morphs Éxaltés sont des humains génétiquement améliorés, conçu pour mettre en évidence certains traits. Leur code génétique a été arrangé pour les rendre plus sains, plsu intelligents et plus attractifs. Leur métabolisme est modifié pour les prédisposer à rester athéltique et en forme pour la durée de leur espérance de vie étendue. 

\begin{description*} \item[Implants] Biomods de Base, Inserts Mesh Basiques, Pile Corticale\item[Maximum d'Aptitude] 30 \item[Solidité] 35 \item[Seuil de Blessure] 7 \item[Avantages] +5 en COG, +5 à trois autres aptitudes au choix du joueur\item[Coût en PP] 30 \item[Coût en Crédit] Cher \end{description*} 

\subsubsection{Mentons} \label{sec:starting-mentons} 

Les Mentons sont génétiquement modifié pour augmenter les capacité cognitives, en particulier l'apprentissage, la créativité, l'attention et la mémoire. Les rumeurs font état de mentons suepraméliorés avec des mods d'intelligence pbien plus extrême, mais la bidouille cérébrale est extrêmement difficile, et de nombreuse tentative pour reconcevoir les facultés mentales résultent en un fonctionnement affaibli, à de l'instabilité ou à la folie. 

\begin{description*} \item[Implants] Biomods de Base, Inserts Mesh Basiques, Pile Corticale, Mémoir Éidétique, Hyepr-Linguiste, Stimulateur Mathématique\item[Maximum d'Aptitude] 30 \item[Solidité] 35 \item[Seuil de Blessure] 7 \item[Avantages] +10 en COG, +5 en INT, +5 en VOL, +5 à une aptitude au choix du joueur\item[Coût en PP] 40 \item[Coût en Crédit] Cher \end{description*} 

\subsubsection{Olympiens} \label{sec:starting-olympians} 

Les Olympiens sont des humains avec des capacités athlétiques améliorées comme l'endurance, la coordination œil-main et les fonctions cardio-vasculaire. Les Olympiens sont fréquemment rencontrés chezz les athlètes, les danseurs, les adeptes du parkour et les soldats. 

\begin{description*} \item[Implants] Biomods de Base, Inserts Mesh Basiques, Pile Corticale\item[Maximum d'Aptitude] 30 \item[Solidité] 40 \item[Seuil de Blessure] 8 \item[Avantages] +5 en CO0, +5 en REF, +10 en SOM, +5 à une autre aptitude au choix du joueur\item[Coût en PP] 40 \item[Coût en Crédit] Cher \end{description*} 

\subsubsection{Sylphs} \label{sec:starting-sylphs} 

les morphs Sylphs sont faites sur-mesure pour les icônes médiatique, l'élite socialites, les stars de l'XP, les mannequins et les narcissiques. Les séquences génétiques de Sylphs sont spécifiquement conçue pour paraître belle. Des caractéristiques étéhrées et féériques sont communes, avec des corps fins et souples. Leur métabolisme a également été assainit pour éliminer les ordeurs corporelles déplaisante et leurs phéromones ont été ajustés pour favoriser une attirance universelle. 

\begin{description*} \item[Implants] Biomods de Base, Inserts Mesh Basiques, Pile Corticale, Métabolisme Propre, Phéromones Améliorés\item[Maximum d'Aptitude] 30 \item[Solidité] 35 \item[Seuil de Blessure] 7 \item[Avantages] trait Beauté Ahurissante (Niveau 1), +5 en COO, +10 en AST +5 à une autre aptitude au choix du joueur\item[Coût en PP] 40 \item[Coût en Crédit] Cher \end{description*} 

\subsubsection{Bounceurs} \label{sec:starting-bouncers} 

Les Bounceurs sont des humains génétiquement adaptés aux environnement en zéro-G ou en microgravité. Leurs jambes sont plus souples, et leurs pieds peuvent saisir aussi bien que leurs mains. 

\begin{description*} \item[Implants] Biomods de Base, Inserts Mesh Basiques, Pile Corticale, Patins Antidérapants, Réserve d'Oxygène, Pieds Préhensibles\item[Maximum d'Aptitude] 30 \item[Solidité] 35 \item[Seuil de Blessure] 7 \item[Avantages] Souple (Niveau 1), +5 en COO, +5 en SOM, +5 à une aptitude au choix du joueur\item[Coût en PP] 40 \item[Coût en Crédit] Cher \end{description*} 

\subsubsection{Furys} \label{sec:starting-furies} 

Les Furys sont des morphs de combat. Ces humains transgéniques ont des améliorations génétiques dimensionnées pour l'endurance, la force et les réflexes, et ont des modifications comportementales pour développer l'agressivité et la ruse. Pour limiter les tendances à l'indiscipline et aux comportements machiste, les furys ont des séquences génétiques favorisant les mentalités de meutes et la coopération, et ont tendance à être des femelles biologiques. 

\begin{description*} \item[Implants] Biomods de Base, Inserts Mesh Basiques, Pile Corticale, Armure Biotissées (Légère), Vision Améliorée, Drogues Neurales (Niveau 1), Filtres à Toxines\item[Maximum d'Aptitude] 30 \item[Modificateur de Speed] +1 (Drogues neurales) \item [Solidité] 50 \item[Seuil de Blessure] 10 \item[Avantages] +5 en COO, +5 en REF, +10 en SOM, +5 en VOL, +5 à une autre aptitude au choix du joueur\item[Coût en PP] 75 \item[Coût en Crédit] Cher (minimum 40 000 ¢) \end{description*} 

\subsubsection{Futuras} \label{sec:starting-futuras} 

Une variante d'éxalté, les morphs futura ont été spéciallement fabriquées pour la "Génréation Égarée." Conçus spécifiquement pour la croissance accélérée et adaptée pour la confiance en soi, l'auto-suffisance et l'adaptabilité, les futuras ont été conçues pour permettre à la transhumanité de reprendre ses anciennes bases. Ce programme s'est avéré être un désastre et la conception a été abandonée, mais certains modèles sont toujours actifs, perçus par certains avec dégoût et par d'autres comme des collecotrs ou des bizarrerries exotiques. 

\begin{description*} \item[Implants] Biomods de Base, Inserts Mesh Basiques, Pile Corticale, Mémoire Éidétique, Atténuateurs Émotionnels\item[Maximum d'Aptitude] 30 \item[Solidité] 35 \item[Seuil de Blessure] 7 \item[Avantages] +5 en COG, +5 en AST, +10 en VOL, +5 à une aptitude au choix du joueur\item[Coût en PP] 40 \item[Coût en Crédit] Cher (Exceptionnelelment rare; 50 000+ ¢). \end{description*} 

\subsubsection{Fantômes} \label{sec:starting-ghosts} 

Les Fantômes sont parttiellement conçu pour le combat, mais leur concentration prrincipale ets la furtivité et l'infiltration. Leur profil génétique encourage la vitesse, l'agilité et les réflexes, et leurs esprits sont modifiés pour favoriser la patience et la résolution de problème. 

\begin{description*} \item[Implants] Biomods de Base, Inserts Mesh Basiques, Pile Corticale, Peau Caméléon, Stimulateurs Surrénaux, Vision Améliorée, Patins Antidérapants\item[Maximum d'Aptitude] 30 \item [Solidité] 45 \item[Seuil de Blessure] 9 \item[Avantages] +10 en COO, +5 en REF, +5 en SOM, +5 en VOL, +5 à une autre aptitude au choix du joueur\item[Coût en PP] 70 \item[Coût en Crédit] Cher (minimum 40 000 ¢) \end{description*} 

\subsubsection{Hibernoïdes} \label{sec:starting-hibernoids} 

Les Hibernoïdes sont des humains transgénique ayant leur métabolisme et leur cycles circadien trés lourdement modifiés. Les Hibernoïdes ont un besoind e sommeil réduit, ne nécessitant qune heure ou deux de sommeil par jour en moyenne. Ils ont également la possibilité de déclencher une forme d'hibernation volontaire, arrétant leur métabolisme et leur besoin en oxygène. Les Hibernoïdes font d'excellent voyageurs spatiaux longue-distance et techiciens d'habitats, mais ces morphs sont égelement adoptées par les aissatants personnels et les hypercapitalistes ayant des rythmes de vie non-stop. 

\begin{description*} \item[Implants] Biomods de Base, Inserts Mesh Basiques, Pile Corticale, Régulation Circadienne, Hibernation\item[Maximum d'Aptitude] 25 \item[Solidité] 35 \item[Seuil de Blessure] 7 \item[Avantages] +5 en INT, +5 à une aptitude au choix du joueur\item[Coût en PP] 25 \item[Coût en Crédit] Cher \end{description*} 

\subsubsection{Néoteniques} \label{sec:starting-neonetics} 

Les néoténique sosnt des transhumains modifier pour conserver leur forme enfantine. Ils sont plus petits, plus agiles, plus curieux et consomment moisn de ressources, les rendant idéals pour la vie en habitat ou en vaisseau. Certaines personnes treouvent les incarnations Néoténique comme repoussante, spécifiquement lorsqu'elles sont utilisées dans certains médias ou comme travailleurs sexuels. 

\begin{description*} \item[Implants] Biomods de Base, Inserts Mesh Basiques, Pile Corticale\item[Maximum d'Aptitude] 20 (SOM), 30 (le reste) \item[Solidité] 30 \item[Seuil de Blessure] 6 \item[Avantages] +5 en COO, +5 en REF, +5 en INT, +5 à une autre aptitude au choix du joueur; les néoténiques comptent comme des petites cibles (modificateur de -10 pour toucher en combat)\item[Désavantages] trait Stigmatisation Sociale (Néoténique) \item[Coût en PP] 25 \item[Coût en Crédit] Cher \end{description*} 

\subsubsection{Refait} \label{sec:starting-remade} 

Les refait sont des humains complètements redessinés: l'humain 2.0. Leur système cardiovasculaire est plus robuste, leur système digestif a été assainit et restructuré pour éliminer les défauts et ils ont globalement été optimisés pour être en bonne santé, intelligents et vivre vieux avec de nombreux modificateurs transgéniques. Les refait sont populaire parmi les ultimes. Les remades ressemblent à l'humain, mais sont différents de manières remarquable et parfois effrayante: plus grand, chauves et sans poils, avec un crâne légèrement plsu grand, des nez plus fins et des doigts allongés. 

\begin{description*} \item[Implants] Biomods de Base, Inserts Mesh Basiques, Pile Corticale, Mémoire Éidétique, Métabolisme Propre, Régulation Circadienne, Respiration Améliorée, Tolérance Thermique, Filtres à Toxines\item[Maximum d'Aptitude] 40 \item[Solidité] 40 \item[Seuil de Blessure] 8 \item[Avantages] +10 en COG, +5 en AST, +10 en SOM, +5 à deux aptitudes au choix du joueur\item[Désavantage] trait Apparence Étrange \item[Coût en PP] 60 \item[Coût en Crédit] Cher (minimum 40 000+ ¢). \end{description*} 

\subsubsection{Rusteurs} \label{sec:starting-rusters} 

Adaptées pour la survie avec un équipement minimal dans l'environnement Martien pas encore complètement terraformé, ces morphs transgéniques possèdent une peau isolées pour une thermorégulation plus efficace et un système respiratoire amélioré qui nécessite moins d'oxygène et qui filtre mieux le dioxyde de carbone, entre autres modifications. 

\begin{description*} \item[Implants] Biomods de Base, Inserts Mesh Basiques, Pile Corticale, Respiration Améliorée, Tolérance Thermique\item[Maximum d'Aptitude] 25 \item[Solidité] 35 \item[Seuil de Blessure] 7 \item[Avantages] +5 SOM, +5 à une aptitude au choix du joueur\item[Coût en PP] 25 \item[Coût en Crédit] Élevé \end{description*} 

\subsubsection{Néo-Aviens} \label{sec:starting-neo-avians} 

Les néo-aviens incluent les corbeaux, les corneilles et les perroquets gris élevés à un niveau d'intelligence humain. Ils ont une taille bien plus importante que leur cousins non-élevés (jusqu'à la taille d'un enfant humain), avec des têtes plus grandes en raison de la taille augmentée de leur cerveau. De nombreuses modification transgéniques ont été faites à leurs ailes, leur permettant de conserver des capacités de vol réduite en environnement 1 g, mais leur donnant une physiologie plus proche des chauves-souris afin quelles puissent se plier et se ranger plus facilement, et en y ajoutant des doigts pirmitifs pour la manipulation d'outils basique. Leurs orteils sont également plus articulés et sont maintenant accompagnés d'un pouce opposable. Les néo-aviens se sont bien adaptés aux environnements en microgravité, et sont choisit pour leur petite taille et leur utilisation de ressources réduite. 

\begin{description*} \item[Implants] Biomods de Base, Inserts Mesh Basiques, Pile Corticale\item[Maximum d'Aptitude] 25 (20 en SOM) \item[Solidité] 20 \item[Seuil de Blessure] 4 \item[Avantages] Attaque de Bec/de Griffe (1d10 VD, utiliser la compétence Combat Désarmé), Vol, +5 INT, +10 REF, +5 à une autre aptitude au choix du joueur\item[Coût en PP] 25 \item[Coût en Crédit] Cher \end{description*} 

\subsubsection{Néo-Hominidés} \label{sec:starting-neo-hominids} 

Les néo-hominidés sont les chimpanzés, les gorilles et les orang-outans élevés. Toutes les morphs ont des intelligences améliorés et sont bipèdes. 

\begin{description*} \item[Implants] Biomods de Base, Inserts Mesh Basiques, Pile Corticale\item[Maximum d'Aptitude] 25 \item[Solidité] 30 \item[Seuil de Blessure] 6 \item[Avantages] +5 en COO, +5 en INT, +5 en SOM, +5 à une autre aptitude au choix du joueur, +10 à la compétence Escalade\item[Coût en PP] 25 \item[Coût en Crédit] Cher \end{description*} 

\subsubsection{Octomorphs} \label{sec:starting-octomorphs} 

Ces incarnations de pieuvres élevés se sont révélées être particulièrement efficace dans les environnements en gravité-zéro. Elles ont conservés leurs huits bras, leur capacité caméléonique à changer de couleur de peau, des sacs d'encre et un bec acéré. Elles ont également bénéficiées d'une augmentation de la capacité crânienne et d'une durée de vie étendue, elles peuvent respirer à la fois l'air et l'eau et n'ont pas de structures squelettiques et peuvent donc se compresser à travers des endroits étroits. Typiquement, les octomorphs rampent en gravité-zéro utilisant les ventouses de leurs bras et expulsant de l'air pour se propulser. Ils peuvent même marcher sur deux de leurs bras en faible gravité. Leurs yeux ont été amélioré avec la vision couleur, fournissant un champ de vision à 360° et s'adaptent rotativement pour garder la pupille en forme de fente aligné vers le "haut." Un système vocal transgéniqueleur permet de parler. 

\begin{description*} \item[Implants] Biomods de Base, Inserts Mesh Basiques, Pile Corticale, Peau Caméléon\item[Maximum d'Aptitude] 30 \item[Solidité] 30 \item[Seuil de Blessure] 6 \item[Avantages] 8 bras, Attaque de Bec (1d10 VD, utiliser la compétence Combat Désarmé), Jet d'Encre (attaque aveuglante, utiliser la compétence Armes à Distance Exotique: Jet d'Encre), trait Souple (Niveau 2), Vision à 360-Degré, °30 à la compétence Nage, +10 à la compétence Escalade, +5 COO, +5 INT, +5 à une autre aptitude au choix du joueur\item[Coût en PP] 50 \item[Coût en Crédit] Cher (minimum 30 000+¢) \end{description*} 

\subsection{Pods} \label{sec:starting-pods} 

Les pods (de "pod people") sont des corps biologies développé en cuve avec des cerveaux extrêmement peu développé et qui sont augmentés avec un ordinateur et un système cybernétique implanté. Étant généralement pilotés par des IA, les pods sont socialement défavorisé dans certaines stations, utilisés comme esclaves dans d'autre et sont même illégaux dans certaines zones. En raison de la croissance accélérée pendant la phase de création des pods, et qu'ils sont essentiellement développés en parties sépraées puis assemblés, leur conception biologique inclut de nombreux raccourcis et limite, contrabalancé par des implants et une maintenance régulière. Ils ne possèdent pas de fonction de reproduction. Dans beaucoup d'habitat, le statut légal des pods est le sujet de débat brûlant. Sans mention contraire, les pods sont également considérés comme des biomorphs du point de vue des règles. 

\subsubsection{Pods de Plaisir} \label{sec:starting-pleasure-pods} 

Les Pods de Plaisirs sont exactement ce qu'ils semblent être - de faux humains conçus purement dans un but de distraction intime. les Pods de plaisir ont des grappes de nerfs supplémentaire dans leurs zones érogènes, un contrôle moteur trés précis de certains groupes de muscles, des phéromones améliorés, un métabolisme assainit et les gènes pour ronoronner. Ils sont bien entendus fabriquer pour être attirant, charismatique et amélioré dans d'autres domaines. Les Pods de plaisirs peuvent changer de sexe à volonté pour mâle, femelle, hermaphrodite ou neutre. 

\begin{description*} \item[Implants] Biomods de Base, Inserts Mesh Basiques, Pile Corticale, Métabolisme Propre, Cybercerveau, Phéromones Améliorés, Augmentation Mnémonique, Marionnette, Changement de Sexe\item[Maximum d'Aptitude] 30 \item[Solidité] 30 \item[Seuil de Blessure] 6 \item[Avantages] +5 INT, +5 en AST, +5 à une aptitude au choix du joueur\item[Désavantages] trait Stigmatisation Sociale (Pod de Plaisir) \item [Coût en PP] 20 \item[Coût en Crédit] Élevé \end{description*} 

\subsubsection{Pods Ouvrier} \label{sec:starting-worker-pods} 

A moitié humain exalté, à moitié machine, ces pods basiques sont virtuellement non distinguable des humains. Les pods ouvriers sont souvent utilisés dans les travaux subalternes nécessitant une interaction humaine. 

\begin{description*} \item[Implants] Biomods de Base, Inserts Mesh Basiques, Pile Corticale, Cybercerveau, Augmentation Mnémonique, Marionnette\item[Maximum d'Aptitude] 30 \item[Solidité] 35 \item[Seuil de Blessure] 7 \item[Avantages] +10 en SOM, +5 à une aptitude au choix du joueur\item[Désavantages] trait Stigmatisation Sociale (Pod) \item [Coût en PP] 20 \item[Coût en Crédit] Élevé \end{description*} 

\subsubsection{Novacrabe} \label{sec:starting-novacrab} 

Les novacrabe sont des pod conçu par bio-ingénierie à partir de crabe de cocotier et d'araignée de mer et amené à taille humaine. Les novacrabes sont idéaux pour le travail dans les zones dangereuses ou en tant que travailleurs du vide, policier ou garde du corps étant donné leurs pattes de deux mètre de long, leurs pinces massive et leur armure chitineuse. Ils peuvent grimper et s'en sortir en micro-gravité et ils peuvent supporter une large gamme de pressions atmosphérique (ainsi que els changements de pression soudains) allant du vide aux profondeurs des mers. Les novacrabes possèdent des yeux composés (avec une résolution d'image équivalente à celle de l'œil humain), des branchies, des doigts permettant l'utilisation d'outil sur leur cinquième paire de membre et des cordes vocales transgénique. 

\begin{description*} \item[Implants] Biomods de Base, Inserts Mesh Basiques, Pile Corticale, Respiration Améliorée, Armure Carapace, Cybercerveau, Branchies, Augmentation Mnémonique, Réserve d'Oxygène, Marionette, Tolérance Thermique, Étanchéité au vide \item[Maximum d'Aptitude] 30 \item[Solidité] 40 \item[Seuil de Blessure] 8 \item[Avantages] 10 jambes, Armure Carapace (11/11), Attaque de Pince (2d10 VD), +10 SOM, +5 à deux autres aptitudes au choix du joueur\item[Coût en PP] 60 \item[Coût en Crédit] Cher (minimum 30 000+¢) \end{description*} 

\subsection{Morphs Synthétique} \label{sec:starting-syntheticmorphs} 

Les morphs syntéhtiques sont entièrement artificielle/robotique. Elles sont habituellement manœuvrée par des IA ou par contrôle distant, mais le manque de biomorphs disponible après la Chute a fait que beaucoup d'infugiés se sont résignés à se réincarner dans des coques robotiques, qui étaient également moins chère et plsu rapide à fabriquer et globalement plus disponible. Les synthmorpsh sont cependant toujours perçue avec dédain dans beaucoup d'habitat et considérée comme n'étant une option que pour les plus pauvres et le splus désespérés qui acceptent de s'y réincarner. Les morphs syntéhtiques ne sotn aps sans avantages cependant et sont communément utilisées pour les tâches subalternes, les gros travaux, la construction d'hébaitat et les servics de sécurité. 

Toutes les synthmoprhs bénéficient des avantages suivants: 

\begin{itemize} \item Absence de Fonctions Biologiques. Les synthmorpsh ne s'encombrent pas de trivialités telles que respirer, manger, déféquer, vieillir, dormir ou tout aspect mineur mais crucial de la vie biologique. \item Filtre de Douleur. Les synthmorphs peuvent filtrer leur récepteurs de douleurs, afin qu'elles ne soient pas génées par les blessures ou les dégâts physiques. Cela leur permet d'ignorer le modificateur de -10 pour une blessure (voir Effets des Blessures, p. 207), mais elles souffrent d'un modificateur de -30 à tous les Tests de Perception basés sur le toucher et ne remarquerons pas qu'ils ont étés abîmés sauf si ils réussissent un Test de Perception modifié. \item Immunité aux Armes à Impulsion. Les synthmorphs n'ont aps de système nerveux à déchirer, et leurs électroniques optiques sont soigneusement protégées des interférences. Les attaques à impulsions peuevnt temporairement perturber leurs communications radios sans-fil pendant la durée de l'attaque. \item Robustesse Environnementale. Les synthmorphs sont conçues pour supporter une large gamme d'environnement, de la poussière de Mars aux océans d'Europe en passant par le vide spatial. Ils ne sont affectés que par les températures et pressions atmosphériques les plus extrêmes. Considérez-le comme les traits Tolérance Thermique (p. 305) et Étanchéité au Vide (p. 305). \item Robustesse. Les coques syntéhtiques sont faites pour durer - un fait reflétter par leur meilleure Solidité et leurs seuils d'Armure interne. Leur composition les rends leur attaques physiques plus violentes: appliquez un modificateur de +2 à la VD des attaques à mains nues pour les coques de taille shumaines ou plus grande. \end{itemize} 

\subsubsection{Boîte} \label{sec:starting-case} 

Les boîtes sont des coques robotiques extrêmement bon marché et produites massivement dans le but de fournir une option de remorph abordable pour les infugiées créés par la Chute. Bien qu'il existe beaucoup de variation de boîte, elels sont uniformément considérées comme de mauvaises qualité et inférieures. La plupart des morphs boîtes sont vaguement anthropomorphique, avec une fine structure corporelle, étant juste un peu plsu petite que l'humain moyen et souffrant de disfonctionnement fréquents. 

\begin{description*} \item[Implants] Prise d'Accès, Inserts Mesh Basiques, Pile Corticale, Cybercerveau, Augmentation Mnémonique\item[Mode de déplacement](Allure de déplacement) Marcheur (4/16)\item[Maximum d'Aptitude] 20 \item[Solidité] 20 \item[Seuil de Blessure] 4 \item[Avantages] Armure (4/4)\item[Désavantages] -5 à une aptitude au choix, trait Imbécile, trait Stigmatisation Sociale (Masse Cliquetante) \item [Coût en PP] 5 \item[Coût en Crédit] Modéré \end{description*} 

\subsubsection{Synth} \label{sec:starting-synths} 

Les synths sont des coques robotiques anthropomorphiques (androïdes et gynéoïdes). Elles sont typiquement utilisés poru les travaux sub-alternes pour lesquels les pods ne sont pas une bonne option. Moins chère que beaucoup d'autres morphs, elles sont souvent utilisées par les personens qui ont besoin d'une morph rapidement et pour pas trop cher ou simplement lors d'un transit entre deux incarnations. Bien qu'elles aient l'air huamnoîdes, les synths sont facilement reconnaissable comme non-biologique à moins qu'elles n'aient l'option masque synthétique (p. 311). 

\begin{description*} \item[Implants] Prise d'Accès, Inserts Mesh Basiques, Pile Corticale, Cybercerveau, Augmentation Mnémonique\item[Mode de déplacement](Allure de déplacement) Marcheur (4/20)\item[Maximum d'Aptitude] 30 \item[Solidité] 40 \item[Seuil de Blessure] 8 \item[Avantages] +5 en SOM, +5 à une aptitude au choix du joueur, Armure (6/6)\item[Désavantages] trait Apparence Étrange, trait Stigmatisation Sociale (Masse Cliquetante) \item [Coût en PP] 30 \item[Coût en Crédit] Élevé \end{description*} 

\subsubsection{Arachnoïdes} \label{sec:starting-arachnoids} 

Les coques robotiques arachnoïdes font 1 mètre de long, sont divisées en deux partie, avec une tête plsu petite comme els araignées ou les termites. Elles possèdent quatre paires de membres rétracaples d'1,5m de long, capables de tourner autour de l'axe corporels et équipés de vérins hydraulique pour propulser le bot avec de petits sauts. Les griffes de manipulations sur chaque membre peuvent être échangée par des mini-roues pour un déplacement en patinage à haute-vitesse. Une pair de bras manipulateur plsu petite et proche de la tête permets une utilisation plsu précise d'outils. En environnement en zéro-G, les arachnoïdes peuvent rétracter leurs membres et manœuvrer grâce à des turbines à poussée vectorielles. 

\begin{description*} \item[Implants] Prise d'Accès, Inserts Mesh Basiques, Pile Corticale, Cybercerveau, Augmentation Mnémonique, Vision Améliorée, membres Supplémenatires (6 membres), Lidar, Membres pneumatiques, Radar\item[Mode de déplacement](Allure de déplacement) Marcheur (4/16), Poussée Vectorielle (8/40)\item[Maximum d'Aptitude] 30 \item[Solidité] 40 \item[Seuil de Blessure] 8 \item[Avantages] Armure (8/8), +5 en COO, +10 en SOM \item[Coût en PP] 45 \item[Coût en Crédit] Élevé (minimum 40 000+¢) \end{description*} 

\subsubsection{Libellulle} \label{sec:starting-dragonfly} 

La morph robotique libellulle prend la forme d'une coque flexible d'un mètre de long possédant des ailes multiples et des bras mnipulateurs. Capable de voler quasi-silencieusement grâce à ses turboréacteurs à double flux et avec une gravité Terreste, les bots libellulles sont encore meilleurs en microgravité. 

\begin{description*} \item[Implants] Prise d'Accès, Inserts Mesh Basiques, Pile Corticale, Cybercerveau, Augmentation Mnémonique \item[Mode de déplacement](Allure de déplacement) Ailé (8/32) \item[Maximum d'Aptitude] 30 (20 en SOM) \item[Solidité] 25 \item[Seuil de Blessure] 5 \item[Avantages] Vol, Armure (2/2), +5 en REF \item[Coût en PP] 20 \item[Coût en Crédit] Élevé \end{description*} 

\subsubsection{Transformers} \label{sec:starting-flexbots} 

Conçus pour remplir des fonctions multi-tâches, les transformers peuvent transformer leur coque pour s'adapter à une gamme de tâches et de situations. Leur structure principale consiste en une demi-douzaine de module interconnectés et de forme adaptable capable de s'auto-transformé en une variété de forme: marcheurs à plusieurs jambes ou tentacules, aéroglisseur et bien d'autre. Chaque module dispode de sa propre unité sensorielle et des doigts fractals ramifiés (cahque capable de se diviser en doigts plus petits, jusqu'à une échelle micrométrique permettant des manipulations ultra-fine). l'ordinateur de contrôle du transformers est également répartit entre les différents modules. Chaque transformers n'est pas plus gros qu'un gros chien, mais plusieurs transformers peuvent s'assembler pour opérer à une échelle de amsse différente, pouvant même s'occuper des tâches telles que la démolition, l'excavation, la fabrication ou l'assemblage robotisé. 

\begin{description*} \item[Implants] Prise d'Accès, Inserts Mesh Basiques, Pile Corticale, Cybercerveau, Doigts Fractals, Augmentation Mnémonique, Conception modulaire, Forme Ajustable \item[Mode de déplacement](Allure de déplacement) Marcheur (4/16), Glisseur (8/40) \item[Maximum d'Aptitude] 30 \item[Solidité] 25 \item[Seuil de Blessure] 5 \item[Avantages] Armure (4/4) \item[Coût en PP] 20 \item[Coût en Crédit] Cher (minimum 30 000+¢) \end{description*} 

\subsubsection{Reapeur} \label{sec:starting-reaper} 

Le reapeur est un bot de combat courant, utilisé à la place des soldats biomorphs et typiquement dirigés par téléopération ou par des IA autonomes. Le cœur du reapeur est un disque renforcé, il peut donc tourner et présenter un profil fin à l'ennemi. Il utilise des buses à poussée vectorielle pour manœuvrer en micro-gravité, et prend également avantages d'un moteur ionique pour le déplacement rapide sur de longue distance. Quatre jambes/bras manipulateurs et quatre montures d'armes sont repliées dans sa structure. La coque du reapeur est faites de matériaux intelligents, permettants à ses membres et à ses montures d'armes de sortir dans n'importe quelle direction et même de changer de forme et de taille. Dans les environnemsnt soumis à la gravité, les reapeurs marchent ou sautillent sur deux de leurs membres. Les reapeurs sont ignobles en raison de nombreuses XP de guerre, et en amener un dans un habitat fera indubitablement froncer les sourcils ou vous faire arréter. 

\begin{description*} \item[Implants] Prise d'Accès, Inserts Mesh Basiques, Pile Corticale, Cybercerveau, Vision à 360-Degrés, Anti Éblouissement, Cyber Griffes, Membres Supplémentaires (4), Armure de Combat Lourde, Système Magnétique, Membres Pneumatiques, Marionnette, Radar, Accélérateur de Réflexes, Forme Ajustable, Améliorations Structurelles, Emitter T-ray, Montures d'Armes (Articulées, 4) \item[Mode de déplacement](Allure de déplacement) Marcheur (4/20), Sautilleur (4/20), Ionique (12/40), Poussée Vectorielle (4/20)\item[Maximum d'Aptitude] 40 \item[Modificateur de Vitesse] +1 (Accélérateur de Réflexes) \item [Solidité] 50 (60 avec l'Amélioration Structurelle) \item[Seuil de Blessure] 10 (12 avec l'Amélioration Structurelle) \item[Avantages] 4 Membres, Armure (16/16), +5 en COO, +10 en REF (+20 avec l'1ccélérateur de Réflexes), +10 en SOM \item[Coût en PP] 100 \item[Coût en Crédit] Cher (minimum 50 000+¢) \end{description*} 

\subsubsection{Slitheroïdes} \label{sec:starting-slitheroids} 

Les bots slithéroïdes sont des coques synthétiques prenant la forme d'un serpent métallique et ségmenté de deux mètres de long et possédant deux bras rétractiles pour la manipulation d'outil. Les bots serpents peuvent se lover, se tordre et rouler leur corps en une balle ou onduler, se déplaçant soit en ondulant, en roulant ou en rampant et en s'aidant de leurs bras. Les systèmes sensoriels et l'ordinateur de contrôle sont hébergé dans la tête. 

\begin{description*} \item[Implants] Prise d'Accès, Inserts Mesh Basiques, Pile Corticale, Cybercerveau, Vision Améliorée, Augmentation Mnémonique\item[Mode de déplacement](Allure de déplacement) Serpent (4/16; 8/32 en roulant)\item[Maximum d'Aptitude] 30 \item[Solidité] 45 \item[Seuil de Blessure] 9 \item[Avantages] +5 en COO, +5 en SOM, +5 à une aptitude au choix du joueur, Armure (8/8) \item[Coût en PP] 40 \item[Coût en Crédit] Cher \end{description*} 

\subsubsection{Swarmanoïde} \label{sec:starting-swarmanoid} 

Le swarmanoïde n'est pas une simple coque en soi, masi plutôt une nuiée de centaine de microdrone robotiques de la taille d'un insecte. Chaque "insecte" est capable de ramper, de rouler, de sauter sur plusieurs mètres ou d'utilsier des pales de nanocoptères pour un déplacment aérien. L'ordinateur de contrôlle et le système sensoriel sont distribués sur toute la nuée. Même si la nuée peut se "fusionner" en une grossière forme de la taille d'un enfant, la nuée est incapable d'accomplir des tâches physiques en tant qu'unité telles que attraper, tirer ou tenir. Chaque insecte est relativement capable de s'interfacer avec des systèmes électroniques. 

\begin{description*} \item[Implants] Prise d'Accès, Inserts Mesh Basiques, Pile Corticale, Cybercerveau, Augmentation Mnémonique, Essaim\item[Mode de déplacement](Allure de déplacement) Marcheur (2/8), Sauteur (4/20), Rotor (4/32) \item[Maximum d'Aptitude] 30 \item[Solidité] 30 \item[Seuil de Blessure] 6 \item[Avantages] voir Nuée (p. 311) \item[Désavantages] voir Nuée (p. 311) \item[Coût en PP] 25 \item[Coût en Crédit] Cher \end{description*} 

\subsection{Infomorphs} \label{sec:starting-infomorphs} 

Les infomorphs n'ont qu'une forme numérique - elles ne possèdent pas de corps physique. Les informophs sont parfois portée par d'autres personnage à la place de (ou en addition de) une muse dans un module ghostrider (p. 307). Les règles complètes concernant les infomorphs peuvent être trouvées à la p. 264. 

\begin{description*} \item[Implants] Améliorations Mnémonique\item[Maximum d'Aptitude] 40 \item[Modificateur de Vitesse] +2 \item [Désavantages] Pas de forme physique \item[Coût en PP] 0 \item[Coût en Crédit] 0 \end{description*} 

\section{Traits} les traits listés sont ds traits d'egos, sauf mention contraire. 

\section{Traits Positifs} \label{sec:positive-traits} Les traits positifs fournissent des bonus au personnage dans certaines situations. 

\subsection{Adaptabilité} \label{sec:traits-adaptability} 

\textbf{Coût:} 10 (Niveau 1) ou 20 (Niveau2) PP 

Se réincarner est est un jeu d'enfant pour se personnage. Ils s'adaptent aux nouvelles morphs plus rapidement que la plupart des autres personnes. Appliquez un modificateur de +10 par niveau aux test d'Intégration et aux Tests d'Aliénation (p. 272). 

\subsection{Allies} \label{sec:traits-allies} 

\textbf{Cost:} 30 CP 

The character is part of or has a relationship with some influential group that they can occasionally call on for support. For example, this could be their old gatecrashing crew, former research lab co-workers, a criminal cartel they are part of, or an elite social clique. The gamemaster and player should work out what the character’s relationship is with this group, and why the character can call on them for aid. Gamemaster’s should take care that these allies are not abused, such as calling on them more than once per game session. The character’s ties to this group are also a two-way street—they will be expected to perform duties for the group on occasion as well (a potential plot seed for scenarios). 

\subsection{Ambidextrous} \label{sec:traits-ambidextrous} 

\textbf{Cost:} 10 CP 

The character can use and manipulate objects equally well with both hands (they do not suffer the off-hand modifier, as noted on p. 193). If the character has other prehensile limbs (feet, tail, tentacles, etc), this trait may be applied to a limb other than the hand. This trait may be taken multiple times for multiple limbs. 

\subsection{Animal Empathy} \label{sec:traits-animal-empathy} 

\textbf{Cost:} 5 CP 

The character has an instinctive feel for handling and working with non-sapient animals of all kinds. Apply a +10 modifier to Animal Handling skill tests or whenever the character makes a test to influence or interact with an animal. 

\subsection{Brave} \label{sec:traits-brave} 

\textbf{Cost:} 10 CP 

This character does not scare easily, and will face threats, intimidation, and certain bodily harm without flinching. As a side effect, the character is not always the best at gauging risks, especially when it comes to factoring in danger to others. The character receives a +10 modifier on all tests to resist fear or intimidation. 

\subsection{Common Sense} \label{sec:traits-common-sense} 

\textbf{Cost:} 10 CP 

The character has an innate sense of judgment that cuts through other distractions and factors that might cloud a decision. Once per game session, the player may ask the gamemaster what choice they should make or what course of action they should take, and the gamemaster should give them solid advice based on what the character knows. Alternately, if the character is about to make a disastrous decision, the gamemaster can use the character’s free hint and warn the player they are making a mistake. 

\subsection{Danger Sense} \label{sec:traits-danger-sense} 

\textbf{Cost:} 10 CP 

The character has an intuitive sixth sense that warns them of imminent threats. They receive a +10 modifier on Surprise Tests (p. 204). 

\subsection{Direction Sense} \label{sec:traits-direction-sense} 

\textbf{Cost:} 5 CP 

Somehow the character always knows which way is up, north, etc., even when blinded. The character receives a +10 modifier for figuring out complex di- rections, reading maps, and remembering or retracing a path they have taken. 





\subsubsection{Eidetic Memory (Ego Or Morph Trait)} \label{sec:traits-eidetic-memory} \textbf{Cost:} 10 CP 

Much like a computer, the character has perfect memory recall. They can remember anything they have sensed, often even from a single glance. This works the same as the eidetic memory implant (p. 301). 

\subsection{Exceptional Aptitude} \label{sec:traits-exceptional-aptitude} 

\textbf{Cost:} 20 CP 

The character may raise one of their maximum aptitude up to 10 points over the normal aptitude cap (30 for flats, 35 for splicers, 40 for all others). Note that this trait just raises the maximum, it does not give the character more 10 aptitude points. This trait may be taken only once. 

\subsection{Expert} \label{sec:traits-expert} 

\textbf{Cost:} 10 CP 

The character is a legend in the use of one particular skill. The character may raise one learned skill over 80, to a maximum of 90, during character creation. This trait does not actually increase the skill, it just raises the maximum. This trait may only be taken once. 

\subsection{Fast Learner} \label{sec:traits-fast-learnier} 

\textbf{Cost:} 10 CP 

The character improves skills and learns new ones in half the time it normally takes (see Improving Skills, p. 152). 

\subsection{First Impression} \label{sec:traits-first-impression} 

\textbf{Cost:} 10 CP 

The character has a way of charming or otherwise making a good impression the first time they interact with someone. This innate social lubricant allows them to more readily deal with new contacts and slip right into new social environments. Apply a +10 modifier on social skill tests when the character is interacting with another character for the first time only. 

\subsection{Hyper Linguist} \label{sec:traits-hyper-linguist} 

\textbf{Cost:} 10 CP 

The character has an intuitive understanding of linguistic structures that facilitates learning new languages easily. The character requires one-third the normal amount of time and experience to learn any language (see Improving Skills, p. 152). The character can also learn any human language in one day simply by constant immersive exposure to it. Additionally, the character receives a +10 modifier when attempting to interpret languages they don’t know. 

\subsection{Improved Immune System (Morph Trait)} \label{sec:traits-improved-immune-system} 

\textbf{Coût:} 10 (Niveau 1) ou 20 (Niveau2) PP 

The morph’s immune system is robust and more resistant to diseases, drugs, and toxins—even more than basic bio-mods. At Level 1, apply a +10 modifier whenever making a test to resist infection or the effects of a toxin or drug. At Level 2, increase this modifier to +20. This trait is only available to biomorphs. 

\subsubsection{Innocuous (Morph Trait)} \label{sec:traits-innocuous} 

\textbf{Cost:} 10 CP 

In an age when exotic appearances and good looks are commonplace, the morph’s look is surprisingly bland and undistinguished, in that cookie cutter sort of way. The character’s physical looks are so mundane that others have a hard time picking them out of a crowd, describing their appearance, or otherwise remembering physical details. Apply a -10 modifier to all tests made to spot, describe, or remember the character. This modifier does not apply to psi or mesh searches. 

\subsection{Limber (Morph Trait)} \label{sec:traits-limber} 

\textbf{Coût:} 10 (Niveau 1) ou 20 (Niveau2) PP 

The morph is especially flexible and supple, capable of graceful contortions and interesting positions. At Level 1, the character can smoke with their toes, do the splits, and squeeze into small, cramped spaces. At Level 2, they are double-jointed escape artists. Each level provides a +10 modifier to escaping from bonds, fitting into narrow confines, and other acts relying on contortion or flexibility. This trait is only available to biomorphs. 

\subsection{Math Wiz} \label{sec:traits-mathwiz} \textbf{Cost:} 10 CP 

The character can perform any feat of calculation, including the most complex and advanced mathematics, instantly and with great precision, with the same ease an unmodified human can add 2 + 3. The character can calculate odds with great precision, find correlations in numerical data, and perform similar tasks with great ease. Apply a +30 modifier on tests involving math calculations. 

\subsection{Natural Immunity (Morph Trait)} \label{sec:traits-natural-immunity} 

\textbf{Cost:} 10 CP 

The morph has a natural immunity to a specific drug, disease, or toxin. When afflicted with that specific chemical, poison, or pathogen, the character re- mains unaffected. At the gamemaster’s discretion, this immunity may not apply to certain agents. It may not be applied to nanodrugs or nanotoxins. This trait is only available to biomorphs. 

\subsection{Pain Tolerance (Ego Or Morph Trait)} \label{sec:traits-pain-tolerance} 

\textbf{Coût:} 10 (Niveau 1) ou 20 (Niveau2) PP 

The character has a high threshold for pain tolerance and is better at ignoring the effects of pain on their abilities and concentration. Level 1 allows them to ignore the -10 modifier from 1 wound. Level 2 allows them to ignore the -10 modifiers from 2 wounds. This trait is only available for biomorphs. 

\subsection{Patron} \label{sec:traits-patron} 

\textbf{Cost:} 30 CP 

The character has an influential person in their life who can be relied on for occasional support. This could be a wealthy hyperelite family member, a high-ranking triad boss, or an anarchist networker with an unbeatable reputation. When called upon, this patron can pull strings on the character’s behalf, supply resources, introduce them to people they need to know, and bail them out of trouble. The player and gamemaster should work together to define exactly who this NPC is and what their relationship with the player character is. Specifically, the question of why this patron is supporting the character should be answered (familial obligation? childhood buddies? the character saved their life once?). Gamemasters should be careful that this trait does not get abused. The patron should be an occasional help (probably no more than once per game session at most) but is not always at the character’s beck-and-call. If the character asks for too much, too often, they should find the patron’s support drying up. Additionally, the character may need to take action to maintain the relationship, such as undertaking a mission on the patron’s behalf. In fact, the character may only have their patronage because they are on-call or of use to the NPC in some way. 

\subsection{Psi} \label{sec:traits-psi} \textbf{Cost:} 20 CP (Level 1), 25 CP (Level 2) 

The character has been infected with the MacLeod strain of the Exsurgent virus, which altered their brain structure and opened the potential for their mind to enhance their cognitive abilities and read and manipulate the biological minds of others (see Psi, p. 220). The character may purchase and learn psi sleights (p. 223). At Level 1, the character may only use psi-chi sleights. At Level 2, the character may use both psi-chi and psi-gamma sleights. 

Though this trait is not very expensive, gamemasters should not allow it to be abused. There are a number of negative side effects to Watts-MacLeod infection, noted under Psi Drawbacks, p. 220. 

\subsection{Psi Chameleon (Ego Or Morph Trait)} \label{sec:traits-psi-chameleon} 

\textbf{Cost:} 10 CP 

The character’s mental state is naturally resistant to psi sensing. Apply a -10 modifier to any attempts to locate or detect the character via psi sleights. 

\subsection{Psi Defense (Ego Or Morph Trait)} \label{sec:traits-psi-defense} 

\textbf{Coût:} 10 (Niveau 1) ou 20 (Niveau2) PP 

The character’s mind is inherently resistant to mental attacks. At Level 1, apply a +10 modifier to all defense tests made against psi attacks. At Level 2, apply a +20 modifier. 

\subsection{Rapid Healer (Morph Trait)} \label{sec:traits-rapid-healer} 

\textbf{Cost:} 10 CP 

The morph recovers from damage more quickly. Reduce the timeframes for healing by half, as noted on the Healing table, p. 208. This trait is only available to biomorphs. 

\subsection{Right At Home} \label{sec:traits-right-at-home} 

\textbf{Cost:} 10 CP 

The character chooses one type of morph (splicer, neo-hominid, case, etc.). The character always feels right at home in morphs of this type. When resleeving into this type of morph, the character automatically adjusts to the new body, no Integration or Alienation Test needed, suffering no penalties and no mental stress. 

\subsection{Second Skin} \label{sec:traits-secondskin} 

\textbf{Cost:} 15 CP 

If your character background or faction enforces a restriction on your starting morph (for example, uplifts must start with an uplift morph), this trait allows you to ignore that restriction and purchase a starting morph of your choice. 





The character is very good at maintaining continuous partial awareness of the goings-on in their immediate environment. In game terms, they do not suffer the Distracted modifier on Perception Tests to notice things even when their attention is focused elsewhere, or when making Quick Perception Tests during combat. 

\subsection{Striking Looks (Morph Trait)} \label{sec:traits-striking-looks} 

\textbf{Coût:} 10 (Niveau 1) ou 20 (Niveau2) PP 

In an age where biosculpting is easy, good looks are both cheap and commonplace. This morph, however, possesses a physical look that can only be described as striking and unusual, but also somehow alluring and fascinating—even the gorgeous and chiseled glitterati take notice. On social skill tests where the character’s beauty may affect the outcome, they receive a +10 (for Level 1) or +20 (for Level 2) modifier. This modifier is ineffective against xenomorphs or those with the infolife or uplift backgrounds. This trait is only available to biomorphs. 

This modifier may be purchased for uplift morphs, but at half the cost, and it is only effective against characters with that specific uplift background (i.e., neo-avians, neo-hominids, etc.). 

The one drawback to this trait is that the character is more easily noticed and remembered. 

\subsection{Tough (Morph Trait)} \label{sec:traits-tough} 

\textbf{Cost:} 10 (Level 1), 20 (Level 2), or 30 (Level 3) CP 

This morph is resilient than others of its type and can take more physical abuse. Increase their Durability by +5 per level (+5 at Level 1, +10 at Level 2, and +15 at Level 3). This also increases Wound Threshold by +1, +2, and +3 respectively. 

\subsection{Zoosemiotics} \label{sec:traits-zoosemiotics} 

\textbf{Cost:} 5 

A character with this trait and the Psi trait does not suffer a modifier when using psi sleights on nonsentient or partly-sentient animal species. 

\section{Negative Traits} \label{sec:negative-traits} 

Negative traits generally hinder the character and apply negative modifiers in certain circumstances. 

\subsection{Addiction (Ego Or Morph Trait)} \label{sec:traits-addiction} 

\textbf{Bonus:} 5 CP (Minor), 10 CP (Moderate), or 20 CP 

\textbf{Major:} Addiction comes in two forms: mental (affecting the ego) and physical (affecting the biomorph). The character or morph is addicted to a drug (p. 317), stimulus (XP), or activity (mesh use) to a degree that impacts the character’s physical or mental health. Players and gamemasters should work together to agree on addictions that are appropriate for their game. Addiction comes in three levels of severity: minor, moderate, or major: 

\textbf{Minor:} A minor addiction is largely kept under control—it does not ruin the character’s life, though it may create some difficulties. The character may not even recognize or admit they have a problem. The character must indulge the addiction at least once a week, though they can go for longer without too much difficulty. If they fail to get their weekly dose, they suffer a -10 modifier on all actions until they get their fix. 

\textbf{Moderate:} A moderate addiction is in full swing. The character obviously has a problem, and must satisfy the addiction at least once a day. If they fail to do so, they may suffer mood swings, compulsive behavior, physical sickness, or other side effects until they indulge their craving. Apply a -20 modifier to all of the character’s actions until they get their fix. Additionally, a character with this level of addiction suffers a -5 DUR penalty. 

\textbf{Major:} A character with a major addiction is on the rapid road to ruin. They face cravings every 6 hours, and suffer a -10 DUR penalty as their health is affected. If they fail to get their regular dosage, they suffer a -30 modifier on all actions until they do. If their life hasn’t already been ruined by their obsession, it soon will be. 

\subsection{Aged (Morph Trait)} \textbf{Bonus:} 10 CP 

The morph is physically aged, and has not been rejuvenated. Old morphs are increasingly uncommon, though some people adopt them hoping to gain an air of seniority and respectability. Reduce the character’s aptitude maximums by 5, and apply a -10 modifier on all physical actions. 

This trait may only be applied to flat and splicer morphs. 

\subsection{Bad Luck} \label{sec:traits-bad-luck} 

\textbf{Bonus:} 30 CP 

Due to some inexplicable cosmic coincidence, things seem to go wrong around the character. The gamemaster is given a pool of Moxie points equal to the character’s Moxie stat, which also refreshes at the same rate as the character’s Moxie. Only the gamemaster may utilize this Moxie, however, and the purpose is to use it against the character. In other words, the gamemaster can use this bad Moxie to cause the character to automatically fail, flip-flop a roll, and so on. To reflect the black cloud that follows the character, the gamemaster can even use this bad Moxie against the character’s friends and allies, when they are doing something with or related to the character, though this should be used sparingly. Gamemasters who might be reluctant to sabotage the character should remember that the player asked for it by purchasing this trait. 





The character has managed to get themselves blacklisted in certain circles, whether they actually did something to deserve it or not. In game terms, the character is barred from having a Rep score higher than 0 in one particular reputation network. People within that network will refuse to help the character out of fear of reprisals and ruining their own reputation. The bonus for this trait is 20 CP if chosen for the rep network pertaining to the character’s own starting faction, and 5 CP if chosen for any other. 

\subsection{Black Mark} \label{sec:traits-black-mark} 

\textbf{Bonus:} 10 (Level 1), 20 (Level 2), or 30 (Level 3) CP 

At some point in the character’s past, they managed to do something that earned a black mark on their reputation. For some reason, no matter what they do, this black mark cannot be shaken off and continues to haunt their interactions. In game terms, the character picks one faction. Every time they interact with this faction (such as a Networking Test) or with an NPC from this faction (Social Skill Tests) who knows who the character is, they suffer a -10 modifier per level. 

\subsection{Combat Paralysis} \label{sec:traits-combat-paralysis} 

\textbf{Bonus:} 20 CP 

The character has an unfortunate habit of freezing in combat or stressful situations, like a deer caught in headlights. Anytime violence breaks out around the character, or they are surprised, the character must make a Willpower Test in order to act or respond in any way. If they fail the test, they lose their action and simply stand there, remaining incapable of reacting to the situation. 

\subsection{Edited Memories} \label{sec:traits-edited-memories} 

\textbf{Bonus:} 10 CP 

At some point in the character’s past, the character had certain memories strategically removed or otherwise lost to them. This may have been done to intentionally forget an unpleasant or shameful experience or to make a break with the past. The memory may also have been lost by an unexpected death (with no recent backup), or it may have been erased against the character’s will. Whatever the case, the memory should bear some importance, and there should exist either evidence of what happened or NPCs who know the full story. This is a tool the gamemaster can use to haunt the character at some future point with ghosts from their past. 

\subsection{Enemy} \label{sec:traits-enemy} 

\textbf{Bonus:} 10 CP 

At some point in their past, the character made an enemy for life who continues to haunt them. The gamemaster and player should work out the details on this enmity, and the gamemaster should use the enemy as an occasional threat, surprise, and hindrance. 

\subsection{Feeble} \label{sec:traits-feeble} 

\textbf{Bonus:} 20 CP 

The character is particularly weak with one aptitude. That aptitude must be purchased at a rating lower than 5, and may never be upgraded during character advancement. The aptitude maximum is 10, no matter what morph the character is wearing. 

\subsection{Frail (Morph Trait)} \label{sec:traits-frail} 

\textbf{Bonus:} 10 (Level 1) or 20 (Level 2) CP 

This morph is not as resilient as others of its type. Its Durability is reduced by 5 per level. This also reduces Wound Threshold by 1 or 2, respectively. 

\subsection{Genetic Defect (Morph Trait)} \label{sec:traits-genetic-defect} 

\textbf{Bonus:} 10 CP or 20 CP 

The morph is not genefixed, and in fact suffers from a genetic disorder or other impairing mutation. The player and gamemaster should agree on a defect appropriate to their game. Some possibilities include: heart disease, diabetes, cystic fibrosis, sickle-cell disease, hypertension, hemophilia, or color blindness. A genetic disorder that creates minor complications and/or occasional health problems would be worth 10 CP, a defect that significantly impairs the character’s regular functioning or that inflicts chronic health problems is worth 20 CP. The gamemaster must determine the exact effects of the disorder on gameplay, as appropriate. 

This trait is only available for flats. 

\subsection{Identity Crisis} \label{sec:traits-identity-crisis} 

\textbf{Bonus:} 10 CP 

The character’s ego has trouble adapting itself to the changed look of a new morph—they are stuck with the mental image of their original body, and simply do not grow accustomed to their new face(s). As a result, the character has difficulty identifying themselves in the mirror, photos, surveillance feeds, etc. They frequently forget the look and shape of their current morph, acting inappropriately, describing themselves by their original body, forgetting to duck when walking through doorways, etc. This is primarily a roleplaying trait, but the gamemaster may apply appropriate modifiers (usually -10) to tests affected by this inability to adapt. 

\subsection{Illiterate} \label{sec:traits-illiterate} 

\textbf{Bonus:} 10 CP 

The character knows how to speak, but has difficulty reading or writing. Due to the entoptic-saturated and icon-driven nature of transhuman society, they are able to get by quite comfortably with this handicap. Reduce the character’s Language skills by half (round down) whenever reading or writing. 

\subsection{Immortality Blues} \label{sec:traits-immortality-blues} 

\textbf{Bonus:} 10 CP 

The character has lived so long - over 100 years - they’re bored with life and now have difficulty motivating themselves. They were old when longevity treatments first became available, survived the Fall, and continue to soldier onward—though they find it increasingly harder to care, take interest in things around them, or fear final death. The character only receives half the Moxie and Rez Points award for completing motivational goals. 

This trait may not be purchased by characters with the infolife or uplift backgrounds. 

\subsection{Implant Rejection (Morph Trait)} \label{sec:traits-implant-rejection} 

\textbf{Bonus:} 5 (Level 1) or 15 (Level 2) CP 

This morph does not accept implants well. At Level 1, any implants acquired are more expensive as they required specialized anti-rejection treatments. Increase the Cost category of the implant by one. At Level 2, the morph cannot accept implants of any kind. 

\subsection{Incompetent} \label{sec:traits-incompetent} 

\textbf{Bonus:} 10 CP 

The character is completely incapable of performing a particular chosen active skill, no matter any training they may receive. They may not buy this skill during character creation or later advancement, and the modifier for defaulting to the linked aptitude of this particular skill is -10. This may not be used for exotic weapon skills, and should be used for a skill that could be of use to the character. 

\subsection{Lemon (Morph Trait)} \label{sec:traits-lemon} 

\textbf{Bonus:} 10 CP 

This trait is only available for synthetic morphs. This particular morph has some unfixable flaws. Once per game session (preferably at a time that will maximize drama or hilarity), the gamemaster can call for the character to make a MOX x 10 Test (using their current Moxie score). If the character fails, the morph immediately suffers 1 wound resulting from some mechanical failure, electrical glitch, or other breakdown. This wound may be repaired as normal. 

\subsection{Low Pain Tolerance (Ego Or Morph Trait)} \label{sec:traits-low-pain-tolerance} 

\textbf{Bonus:} 20 CP 

Pain is the character’s enemy. The character has a very low threshold for pain tolerance and is more severely impaired when suffering. Increase the modifier for each wound take by an additional -10 (so the character suffers -20 with one wound, -40 with another, and -60 with a third). Additionally, the character suffers a -30 modifier on any test involving pain resistance. This morph version of this trait is only available for biomorphs. 

\subsection{Mental Disorder} \label{sec:traits-mental-disorder} 

\textbf{Bonus:} 10 CP 

You have a psychological disorder from a previous traumatic experience in your life. Choose one of the disorders listed on p. 211. 

\subsection{Mild Allergy (Morph Trait)} \label{sec:traits-mild-allergy} 

\textbf{Bonus:} 5 CP 

The morph is allergic to a specific chosen allergen (dust, dander, plant pollen, certain chemicals) and suffers mild discomfort when exposed to it (eye irritation, sneezing, difficult breathing). Apply a -10 modifier to all tests while the character remains exposed. This trait is only available for biomorphs. 

\subsection{Modified Behavior} \label{sec:traits-modified-behaviour} 

\textbf{Bonus:} 5 (Level 1), 10 (Level 2), or 20 (Level 3) CP 

The character has been conditioned via time-accelerated behavioral control psychosurgery. This is common among ex-felons, who have been conditioned to respond to a specific idea or activity with vehement horror and disgust, but may have occurred for some other reason or even been self-inflicted. At Level 1, the chosen behavior is either limited or boosted, at Level 2 it is either blocked or encouraged, and at Level 3 it is expunged or enforced (see p. 231 for details). This trait should only be allowed for behaviors that are either limited or, if encouraged, impact the character in a negative way. 

\subsection{Morphing Disorder} \label{sec:traits-morphing-disorder} 

\textbf{Cost:} 10 (Level 1), 20 (Level 2), or 30 (Level 3) CP 

Adapting to new morphs is particularly challenging for this character. The character suffers a -10 modifier per level on Integration Tests and Alien- ation Tests (p. 272). 

\subsection{Neural Damage} \label{sec:traits-neural-damage} 

\textbf{Bonus:} 10 CP 

The character has suffered some type of neurological damage that simply cannot be cured. The affliction is now part of the character’s ego and remains with them even when remorphing. This damage may have been inherited, it may have resulted from a poorly designed morph or implant, or it may have been inflicted by one of the TITAN nanovirii that targeted neural systems during the Fall (p. 384). The gamemaster and player should agree on a specific disorder appropriate to their game. Some possibilities are: 

\begin{itemize} \item Partial aphasia (difficulty communicating or using words) \item Color blindness \item Amusica (inability to make or understand music) \item Synaesthesia \item Logorrhoea (excessive use of words) \item Loss of face recognition \item Loss of depth perception (double range modifiers) \item Repetitive behavior \item Mood swings \item The inability to shift attention quickly \end{itemize} 

The gamemaster may decide to inflict modifiers resulting from this affliction as appropriate. 





The morph lacks the cortical stack that is common to morphs of its type. This means the character cannot be resleeved from the cortical stack if the character dies, they can only be resleeved from a standard backup. This trait is not available for flats. 

\subsection{Oblivious} \label{sec:traits-oblivious} 

\textbf{Bonus:} 10 CP 

The character is particularly oblivious to events around them or anything other than what their attention is focused on. They suffer a -10 modifier to Surprise Tests and their modifier for being Distracted is -30 rather than the usual -20 (see Basic Perception, p. 190). 

\subsection{On The Run} \label{sec:traits-on-the-run} 

\textbf{Bonus:} 10 CP 

The character is wanted by the authorities of a particular habitat/station or faction, who continue to actively search for the character. They either commit- ted a crime or somehow displeased someone in power. The character deals with that faction in question at their own risk, and may occasionally be forced to deal with bounty hunters. 

\subsection{Psi Vulnerability (Ego Or Morph Trait)} \label{sec:traits-psi-vulnerability} 

\textbf{Bonus:} 10 CP 

Something about the character’s mind makes them particularly vulnerable to psi attack. They suffer a -10 modifier when resisting such attacks. The morph version of this trait may only be taken by biomorphs. 

\subsection{Real World Naiveté} \label{sec:traits-real-world-naivite} 

\textbf{Bonus:} 10 CP 

Due to their background, the character has very limited personal experience with the real (physical) world—or they have spent so much time in simulspace that their functioning in real life is impaired. They lack an understanding of many physical properties, social cues, and other factors that people with standard human upbringings take for granted. This lack of common sense may lead the character to misunderstand how a device works or to misinterpret someone’s body language. 

Once per game session, the gamemaster may intentionally mislead the character when giving them a description about some thing or some social in- teraction. This falsehood represents the character’s misunderstanding of the situation, and should be roleplayed appropriately, even if the player realizes the character’s mistake. 

This trait should only be available to characters with the infolife or reinstantiated backgrounds, though the gamemaster may allow it for characters who have extensive virtual reality/XP use in their personal histories. 

\subsection{Severe Allergy (Morph Trait)} \label{sec:traits-severe-allergy} 

\textbf{Bonus:} 10 (uncommon) or 20 (common) CP 

The morph’s biochemistry suffers a severe allergic reaction (anaphylaxis) when it comes into contact (touched, inhaled, or ingested) with a specific allergen. The allergen may be common (dust, dander, plant pollen, certain foods, latex) or uncommon (certain drugs, insect stings). The player and gamemaster should agree on an allergen that fits the game. If exposed to the allergen, the character breaks into hives, has difficulty to breathing (-30 modifier to all actions), and must make a DUR Test or go into anaphylactic shock (dying of respiratory failure in 2d10 minutes unless medical care is applied). This trait is only available to biomorphs. 

\subsection{Slow Learner} \label{sec:traits-slow-learner} 

\textbf{Bonus:} 10 CP 

New skills are not easy for this character to pick up. The character takes twice as long as normal to improve skills or learn new ones (p. 152). 

\subsection{Social Stigma (Ego Or Morph Trait)} \label{sec:traits-social-stigma} 

\textbf{Bonus:} 10 CP 

An unfortunate aspect of the character’s background means that they suffer from a stigma in certain social situations. They may be sleeved in a morph viewed with repugnance, be a survivor of the infamous Lost generation, or may be an AGI in a post-Fall society plagued by fear of artificial intelli- gence. In social situations where the character’s nature is known to someone who view that nature with distaste, fear, or repugnance, they suffer a -10 to -30 modifier (gamemaster’s discretion) to social skill tests. 

\subsection{Timid} \label{sec:traits-timid} 

\textbf{Bonus:} 10 CP 

This character frightens easily. They suffer a -10 modifier when resisting fear or intimidation. 

\subsection{Unattractive (Morph Trait)} \label{sec:traits-unattractive} 

\textbf{Bonus:} 10 CP (Level 1), 20 CP (Level 2), 30 CP (Level 3) 

In a time when good looks are easily purchased, this morph is conspicuously ugly. As unattractiveness is increasingly associated with being poor, backward, or genetically defective, responses to a lack of good looks range from distaste to horror. The character suffers a -10 modifier on social tests for Level 1, -20 for Level 2, and -30 for Level 3. 

Only biomorphs may take this trait. This modifier does not apply to interactions with xenomorphs or those with the infolife or uplift backgrounds. This modifier may be purchased for uplift morphs, but at half the bonus, and it is only effective against characters with that specific uplift background (i.e., neo-avians, neo-hominids, etc.). 

\subsection{Uncanny Valley (Morph Trait)} \label{sec:traits-uncanny-valley} 

\textbf{Bonus:} 10 CP 

There is a point where synthetic human looks become uncannily realistic and human-seeming, but they remain just different enough that their looks seem creepy or even repulsive—a phenomenon called the “uncanny valley.” Morphs whose looks fall into this range suffer a -10 modifier on social skill tests when dealing with humans. This modifier does not apply to interactions with xenomorphs or those with the infolife or uplift backgrounds. 

\subsection{Unfit (Morph Trait)} \label{sec:traits-unfit} 

\textbf{Bonus:} 10 CP (Level 1), 20 CP (Level 2) 

The morph is either not optimized for health and/or just in bad shape. Reduce the aptitude maximums for Coordination, Reflexes, and Somatics by 5 (Level 1 ) or 10 (Level 2). 

\subsection{VR Vertigo} \label{sec:traits-vr-vertigo} 

\textbf{Bonus:} 10 CP 

The character experiences intense vertigo and nausea when interfacing with any type of virtual reality, XP, or simulspace. Augmented reality has no effect, but VR inflicts a -30 modifier to all of the character’s actions. Prolonged use of VR (gamemaster’s discretion) may actually incapacitate the character should they fail a WIL x 2 Test. 

\subsection{Weak Immune System (Morph Trait)} \label{sec:traits-weak-immune-system} 

\textbf{Bonus:} 10 (Level 1) or 20 (Level 2) CP 

The morph’s immune system is susceptible to diseases, drugs, and toxins. At Level 1, apply a -10 modifier whenever making a test to resist infection or the effects of a toxin or drug. At Level 2, increase this modifier to -20. This trait is only available to biomorphs. 

\subsection{Zero-G Nausea (Morph Trait)} \label{sec:traits-zero-g-nausea} 

\textbf{Bonus:} 10 CP 

This morph suffers from space sickness and does not fair well in zero-gravity. The character suffers a -10 modifier in any microgravity climate. Additionally, whenever the character is first getting acclimated or anytime they must endure excessive movement in microgravity, they must make a WIL Test or spend 1 hour incapacitated by nausea per 10 points of MoF. 

\section{Character Advancement}: \label{sec:character-advancement} 

As characters accomplish goals and gather experience during gameplay, they accumulate Rez Points (see Awarding Rez Points, p. 384). Rez Points may be used to improve the character’s skills, aptitudes, and other characteristics per the following rules. The costs for spending Rez Points for advancement are the same as the costs for spending Customization Points. 

\begin{quotation} Spending Rez Points 

\begin{itemize} \item 15 RP = 1 Moxie point \item 10 RP = 1 aptitude point \item 5 RP = 1 psi sleight \item 5 RP = 1 specialization \item 2 RP = 1 skill point (61-99) \item 1 RP = 1 skill point (up to 60) \item 1 RP = 10 Rep \item 1 RP = 1,000 Credits \end{itemize} \end{quotation} 

\subsection{Changing Motivation} \label{sec:changing-motivation} 

It is only natural that over time a character’s driving goals and interests will change. The character may reach a turning point where they feel certain personal agendas have been fulfi lled and it is time to move on, or they have failed and need to be discarded. New urgencies or philosophies may have entered the character’s life, or the character may have become disenchanted with particular memes and ideas they previously took to heart. 

Changing a character’s motivation does not cost Rez Points, but it is something that should only happen in accordance with roleplaying and with life-altering events. Players should not be allowed to simply switch their motivations at whim, there should be a driving reason or explanation for doing so. For this reason, changing a motivation should only happen when the player and gamemaster discuss the matter and both agree that the swap is appropriate to the character’s development and circumstances. If these conditions are met, the character simply drops a previously held motivation and takes on a new one. Only one motivation should be switched out at a time. 

\subsection{Switching Morphs} \label{sec:switching-morphs} 

Resleeving—switching from one morph to another— is handled as an in-character interaction, not with Rez Points. See Resleeving, p. 271. 

\subsection{Improving Aptitudes} \label{sec:improving-aptitudes} 

Aptitudes may be raised with Rez Points at the cost of 10 RP per aptitude point. This represents the character’s improvement in their core characteristics, gained from exercise, learning, and experience. Aptitudes may not be raised above 30 (bonuses from morphs, implants, traits, or other sources do not count towards this total). Raising the value of an aptitude also raises the value of all linked skills by an equivalent amount. If this raises any linked skills over 60, an additional 1 RP must be spent per linked skill over 60. 

\subsection{Improving Skills} \label{sec:improving-skills} 

Characters may also spend Rez Points to increase existing skills or learn new ones. To improve an existing skill, the character must have successfully used that skill in the recent past or must actively practice it in order to get better, perhaps with the aid of an instructor. In the case of Knowledge skills, this means actively studying. As a rough timeframe, this should require around 1 week of learning per skill point. A number of educational resources are freely available via the mesh, though some areas of interest may be restricted or hard to fi nd. This can be handled via roleplaying or designated as something the character is doing during downtime between sessions. If the gamemaster decides that a character has not put enough effort into improving a skill, they may call for more practice/study. The cost to increase a skill is 1 RP per skill point, and no skill may be increased over 99. No skill may be raised by more than 5 points per month. When a character’s skill reaches the level of expertise (skill of 60+), however, they tend to reach a plateau where improvement progresses more slowly and even consistent practice and study have diminished returns. In this case, the Rez Point cost per skill point doubles (i.e., 2 RP = +1 skill point). When a skill reaches 80, improvement slows down even further—a skill of 80+ may not be increased by more than 1 point per month. 

\subsection{Learning New Skills} \label{sec:learning-new-skills} 

Similarly, to learn a new skill, the character must actively study/practice and/or seek instruction. No test to learn is required, unless the period of study was hampered or in some way defi cient, in which case the gamemaster may call for a COG x 3 Test to pick up the new skill. Otherwise, once a character has spent approximately a week learning a new skill, they may purchase their fi rst skill point at the usual cost (1 RP). The skill is bought up from the aptitude rating, per normal. Once a new skill is acquired, it is raised according to the standard rules above. 

\subsection{Specializations} \label{sec:new-specializations} Specializations may be purchased for existing skills, as long as that skill is at least rating 30. Specializations require a total of 1 month of training. The cost to learn a specialization is 5 RP. Only 1 specialization may be purchased per skill. 

\subsection{Improving Moxie} \label{sec:improving-moxie} 

Moxie may be raised at the cost of 15 RP per Moxie point. Le niveau maximum auquel le Moxie peut-être élevé est de 10. 

\subsection{Gaining/Losing Traits} \label{sec:gaining-losing-traits} 

At the gamemaster’s discretion, both positive and negative traits may be acquired or lost during gameplay, though such changes should be rare and only made in accordance with the storyline and unfolding events in the game. Both positive and negative traits may be picked up by a character during gameplay as a consequence of something that did or something that happened to them. In the case of a positive trait, the character must immediately spend Rez Points equal to the trait’s CP cost for the privilege (whether they wanted the new trait or not). If the character has no unspent RP available, they must pay out immediately from any future RP they earn until the debt is paid off. In the case of a negative trait, however, the character is simply saddled with the new fl aw—they do not acquire any extra RP for gaining the negative trait. Getting rid of traits is somewhat more diffi cult. Positive traits may be lost due to unfortunate effects on the character, as the gamemaster sees fi t. Such lost positive traits are simply gone—the character does not receive any Rez Point reimbursement. Negative traits are occasionally eliminated in the same way, but more typically they can only be worked off through the hard work and diligence of a character that seeks to overcome their handicap. Such endeavors should require weeks if not months of effort on the character’s part, with appropriate roleplaying and possibly some diffi cult tests. In fact, overcoming such traits could be the source of an adventure. Once a gamemaster feels that the character has made a strong-enough effort, the character may pay a number of Rez Points equal to the trait’s original CP bonus to negate it. Note, however, that some negative traits may simply not be discarded, no matter what the character does. 

\subsection{Improving Rep} \label{sec:improving-rep} 

Reputation is something that can be increased with appropriate roleplaying and actions during gameplay (see Reputation Gain and Loss, p. 384). Characters that prefer to handle their Rep-boosting activities “off-screen,” however, can simply spend Rez Points to boost their score(s). Each RP spent boosts the character’s Rep by +10 in a single network. Only one such boost may be made to a single rep network per month. 

\subsection{Making Credit} \label{sec:making-credit} 

Rez Points may be spent on Credit at a ratio of 1 RP for 1,000 Credits. This represents income the character has earned “off-screen” or during downtime, such as from odd jobs, selling off possessions, and so on. 

\subsection{Improving Psi} \label{sec:improving-psi} 

Characters who have the Psi trait (p. 147) may purchase new sleights (see Sleights, p. 223) at the cost of 5 RP per sleight. Sleights must be learned through study, training, and practice, requiring approximately 1 month per sleight. No more than one sleight may be learned per month. 







