\chapter{Création de personnage} \label{chap:character-creation} 

\section{Génération de personnage} \label{sec:character-creation} 

Chaque personnage joueur est composé de deux parties. La première est l'ensemble des nombres et des attributs qui définissent ce à quoi un personnage est bon ou mauvais (ou même ce qu'il peuvent ou ne peuvent pas faire). Ils sotn un peu plus que de simple statistiques cependant - ces caractéristiques aident à définir les capacités de votre personnage et ses intérêts et, par extension, son passé, son éducation, son entraînement et son éducation. Pendant le processus de création de personnage, vous aurez la possibilité d'assigner, d'ajuster et de jongler avec ces stats comme vous le désirez. Si vous avez une notion pré-conçues de ce qu'est votre personnage, vous pouvez optimiser les stats pour le refléter. Alternativement, vous pouvez bidouiller les stats jusqu'à obtenir quelque chose que vous aimez, pusi baser l'histoire de votre personnage sur ce que vous avez développé. 

La deuxième partie de chaque personnage joueur est sa eprsonnalité. Qu'est-ce qui le/la définit en temps que personne? Qu'est-ce qui le/la motive? Qu'es-ce qui l'énerve? Qu'est-ce qui attire son attention? Quels aspect de sa personnalité le/la rend attractif/ve en tant qu'ami/e, camarade ou amant/e - ou au moins quelqu'un d'intéressant avc qui jouer? Quels défaut de personnalité ou quelle bizarrerie il/elle a? Ces questions importent car elles vont également vous guider lorsque vous choisirez les stats, les compétences et les traits. 

La génération de personnage est un procédé qui se fait étape par étape. Contrairement à d'autres jeu, le procédé pour générer un personnage d'Eclipse Phase n'est pas aléatoire - vous avez un contrôle total sur tous les aspects de la conception de votre personnage. Certaines étapes doivent être réalisée avant de pouvoir passer ausx suivantes. Les étapes du processuss complet sont détaillée dans la barre latrale Guide à la Création de Personnage Par Étapes. 

\subsection{Guide à la Création de Personnage Par Étapes} 

\begin{enumerate} \item Définir le Concept du Personnage (p. 130) \item Choisir un Historique (p. 131) \item Choisir une Faction (p. 132) \item Penser els Po ints Gratuits (p. 134) \begin{itemize} \item 105 points d'aptitudes \item 1 point de Moxie \item 5 000 crédit \item 50 Rep \item Langue Natale\end{itemize} \item Dépenser les POints de Personnalisation (p. 135) \begin{itemize} \item 1 000 PP à dépenser\begin{itemize} \item 15 PP = 1 Moxie \item 10 PP = 1 point d'aptitude \item 5 PP = 1 exploit psi \item 5 PP = 1 spécialisation \item 2 PP = 1 point de compétence (61-80) \item 1 PP = 1 point de compétence (up to 60) \item 1 PP = 1 000 crédit \item 1 CP = 10 rep \end{itemize} \item Minimum de points de a investir en compétence active: 400 skill points \item Minimum de point à investir en compétence de connaissance: 300 skill points \item Choisir une morph de départ (pp. 136 et 139) \item Choisir des Traits (pp. 136 et 145) \end{itemize} \item Acheter du Matériel (p. 136) \item Choisir des Motivations (p. 137) \item Calculer les Stats Restantes (p. 138) \item Détailler le Personnage (p. 138) \end{enumerate} 

\subsection{Concept de Personnage} \label{character-concept} 

Décider ce que/qui vous voulez jouer avant de créer le personnage est habituellement le meilleur chemin. Choisissez un archétype simple qui correspond à votre personnage, et partez de là. Voulez-vous jouer un explorateur? Quelqu'un de sournois, comme un espion ou un voleur? Quelqu'un de cérébral, comme un scientifique? Un criminel endurci ou un ex-flic? Ou préférez-vous être un agitateur démagogique? Vous pouvez aussi démarrer avec une personnalité type et choisir une profession associée. Si vous voulez un papillon social qui excelle dans la manipulation des personnes, vous pouvez jouer une personnalité des médias, un blogueur ou un socialite fétard. Peut-être que vous préférerez un rebut de la société avec des problèmes de drogues, auquel cas un ex-mercenaire ou un ancien hypercapitaliste qui a perdu sa fortune et sa famille pendant la Chute pourrait correspondre. Et que pesnez-vous d'un personnage énergétique, profitant de la vie au maximum et qui doit absolument voir tout ce qu'il y a à voir? Un adepte de la course libre ou un resquilleur professionnel pourrait alors bien être ce que vous cherchez. 

Soyez sûr de vous concerter avec les autres joueurs et d'essayer de faire un pesonnage qui est complémentaire au reste de l'équipe - de manière préférable, un personnage qui fournit un ensemble de compétences dont manque le groupe. POurquoi créer un chercheur en archéologie si quelqu'un d'autre a déjà décider qu'il en ferait un, particulièrement quand l'équipe manque d'un bon spécialsite du combat ou d'un async? D'un autre côté, si votre équipe va s'embarquer dans une expédition d'archéologie étrnagère, avoir plus d'un chercheur (chacun avec des domaines d'expertise différent) pourrait ne pas être une mauvaise chose. 

Une fois que vous avez le concept de base, essayez de l'étoffer de quelques détails, pour en faire un résumé en une phrase. Si vous avez commencé avec le concept de "xéno-sociologue," étendez-le à "linguiste amateur ouvert d'esprit et expert en xéno-sociologie fasciné par les cultures étrangères, la collection d'objet kitsch des Facteurs, possède une haute-tolérance au 'facteur beurk' et dont les meilleurs amis tendent à être des élevés et des IAs." Cella vous donnera un peu plus de détails sur lesquels vous pourrez concentrez les forces et faiblesses de votre personnage. 

\subsection{Choisir un historique} \label{choose-background} 

La première étape de la création de votre personnage est de lui choisir un historique. Votre personnage est-il/elle né/née sur Terre avant la Chute? A-t-il/elle été élevé/élevée sur un habitat communautaire? Ou a-t-il/elle commencé son existence en tant su'IA désincarnée? 

Vous devez choisir l'historique de votre personnage parmi ceux de la liste ci-dessous. Choisissez sagement, car chaque historique peut fournir à votre personnage certaines compétences, traits, limitations ou d'autres caractéristiques avec lesquelles débuter. Gardez en tête que votre historique représente d'où vous venez, non pas ce que vous êtes maintenant. C'est le passé, alors que votre faction représente ceux avec qui vous êtes actuellement alignés. Votre futur, bien entendu, est ce que vous déciderez d'en faire. 

Les options d'historiques présentées ci-dessous coouvrent une large sélection de la transhumanté, mais ils ne couvrent pas toutes les possibilités. Si votre maître de jeu vous y autorise, vous pouvez travailler ensemble au développement d'un historique qui n'est pas inclut sur cette liste, en s'en servant comme base pour maintenir l'équilibre. 

\subsubsection{Dériveurs} \label{sec:drifters} 

Vous avez été élevés avec un groupe social qui est resté en mouvement à travers tout le système Sol. Cela peut-être des mibre marchands, des pirates, des fermiers d'astéroïdes, des récupérateurs ou juste des travailleurs migrants. Vous êtes habitués aux voyages spatiaux erratiques entre les habitats et les stations. 

Avantages: +10 compétence Navigation, +20 compétence Piloter: Vaisseau, +10 compétence Réseau: [Domaine] de votre choix. 

Désavantages: Aucun 

Morphs Communes: toutes, particulièrement les Bounceurs et les Hibernoïdes. 

\subsubsection{Évacué de la Chute} \label{sec:fall-evacuee} 

Vous êtes nés et avez été élevés sur terre et évacué pendant les horreurs de la Chute, abandonnant votre ancienne vie (et probablement vos amis, votre famille et ceux que vous aimiez). Vous avez été suffisament chanceux pour survivre en conservant votre corps et à vous en sortir seul dans le système. 

Avantages: +10 compétence Piloter: véhicules terrestre, +10 compétence Réseau: [Domaine] de votre choix, +1 Moxie. 

Désavantages: Seulement 2 500 crédit de départ (il reste possible d'acheter des crédits avec des PP) 

Morphs Communes: Plates, Spliceurs 

\subsubsection{Hyperélite} \label{sec:hyperelite} 

Vous avez eu le privlège d'être élevé en tant que membre de la haute société immortelle qui dirige la plupart des habitats du système intérieur et des hypercorps. Vous avez été dorloté par une fortune et une influence dont la plupart des gens ne peuvent que rêver. 

Avantages: +10 compétence Protocole, +10 000 Crédit, +20 compétence Réseau: Hypercorp 

Désavantages: Ne peut démarrer ni avec une morph flat, spliceur ou dans un pod, ni avec des morphs élevée ou syntéhtique. 

Morphs Communes: Exaltés, Sylphs. 

\subsubsection{Infolife} \label{sec:infolife} 

Vous avez démarrer votre existence en tant que conscience numérique - une intelligence artificelle généraliste (IAG). Votre existence même est illégalle dans certains habitats (un héritage de ceux qui accusent les IAs d'être responsables de la Chute). Contrairement aux IAs germe responsable de leur Chute, votre capacité d'auto-amélioration est limité, bien que vous soyez complètement autonome. 

Avantages: +30 compétences Interfaçage, compétences Informatique (Infosec, Interfaçage, Programmation, Recherche) achetées avec des Points de Personnalisation a moitié prix. 

Désavantages: trait Naïveté du Monde Réel, trait Stigmatisation Sociale (IAG), ne peux pas acheter le trait Psi, les compétences Sociales achetées avec les Points de Personnalisations sont au double du prix normal. 

Morphs COmmunes: Informorphs, morphs synthétiques. 

\subsubsection{Isolé} \label{sec:isolite} 

Vous avez été élevés en tant que membre d'un groupe d'exilés volontaires aux limites du système. Que vous ayez été élevés en tant que membre d'un groupe religieux, d'un culte, d'une expérimentation sociale, d'une cellulle luddiste ou d'un groupe qui voulait simplement être isolé, vous avez passé la plupart , si ce n'est l'intégralité, de votre éducvation isolé des autres factions. 

Avantages: +20 à deux compétences de votre choix 

Désavantages: -10 à la rep de départ 

Morphs Communes: Toutes 

\subsubsection{Égaré} \label{sec:lost} 

Vous êtes l'éhritage de l'une des débacles les plus infamantes depusi la Chute. En tant que mebre de la "Génération Égarée," vous avez subit une enfance en croissance accélérée, survivant d'une manière ou d'une autre là où les autres de votre espèces sont morts, sont devenus fous ou ont été persécutés (voir Les Égarés, p. 233). Votre passé est une stigmatisation sociale, mais elle vous fournit certains avantages ... et fardeaux. 

Avantages: +20 à deux compétences de Connaissance de votre choix, trait Psi. 

Désavantages: trait Désordre Mental (choisissez en deux), trait STigmatisation Sociale (Égaré), doit démarrer avec une morph Futura. 

Morphs Communes: Futura 

\subsubsection{Colon Lunaire} \label{sec:lunar-colonist} 

Vous avez vécu votre enfance dans l'une des étroites cités dômes ou stations sous-terraine de la Lune, la lune de la Terre. Vous étiez aux premières loges pour voir la Chute de la Terre. 

Avantages: +10 compétence Piloter: véhicules terrestre, +10 à une compétence Technique, Académique: [domaine] ou Profession: [domaine] de votre choix, +20 à la compétence Réseau: Hypercorp. 

Désavantages: Aucun 

Morphs Communes: Plates, Spliceurs 





\subsubsection{Martien} \label{sec:martian} 

Vous avez été élevé dans l'une des stations sur ou au-dessus de Mars, devenue la planète la plus peuplée du système. Votre ville d'origine peut avoir survécu, ou non, à la Chute. 

Avantages: +10 compétence Piloter: véhicules terrestre, +10 à une compétence Technique, Académique: [domaine] ou Profession: [domaine] de votre choix, +20 à la compétence Réseau: Hypercorp. 

Désavantages: Aucun 

Morphs Communes: Plates, Spliceurs et Rusteurs. 

\subsubsection{Colon Spatial Originel} \label{sec:original-space-colonist} 

Vous, ou vos parents, faisiez parti de la première "génération" de colons/travailleurs envoyé hors de la Terre pour revendiquer une part de l'espace, vous êtes donc familier avec les confins étroits du vol spatial et avec la vie à bord des plus vieilles stations et habitats. En tant que "zéro-un G" (Zéro gravité, première généréation), vous n'avez jamais fait partie de l'élite. Les personnes ayant votre passé onttypiquement une sorte d'entrainement technologique spécialisé en tant que taravilleur du vide ou de technicien d'habitats. 

Avantages: +10 compétence Piloter: vaisseaux ou Chute Libre, +10 à une compétence Technique, Académique: [domaine] ou Profession: [domaine] de votre choix, +20 compétence Réseau: [domaine] de votre choix. 

Désavantages: Aucun 

Morphs Communes: Toutes L'utilisation de morph exotique est courant. 

\subsubsection{Ré-instantié} \label{sec:re-instantiated} 

Vous êtes nés sur Terre et y avez grandi, mais vous n'avez pas survécu à la Chute. Tout ce que vous savez c'est que votre corps est mort là-bas, mais que votre suavegarde a été transmise hors-monde, et vous êtes l'un des rares chanceux à être ré-instantié avec une nouvelle morph. Vous avez pu passer des années en mémoire morte, en simulspace ou en tant qu'esclave infomorph. 

Avantages: +10 compétence Piloter: véhicules terrestre, +10 compétence Réseau: [Domaine] de votre choix, +2 Moxie. 

Désavantages: trait Mémoire Éditée, 0 crédit de dpéart( il reste possible d'acheter des crédits avec des PP) 

Morphs Communes: Boîtiers, Infomorphs, Synths. 

\subsubsection{Racaille-né} \label{sec:scumborn} 

Vous avez été élevés dans le style de vie nomadique et chaotique commun dans les barges racailles. Advantages: +10 compétence Persuasion ou Supercherie, +10 compétence Escamotage, +20 compétence Réseau: Autonomistes.  Désavantages: Aucun. 

Morphs Communes: toutes, particulièrement les Bounceurs. 

\subsubsection{Élevé} \label{sec:uplift} 

Vous n'êtes mêmem pas humain. Vous êtes nés en temps qu'animal élevé: chimpanzée, gorille, orang-outans, perroquets, corneille, corbeau ou poulpe. 

Avantages: +10 compétence Esquive, +10 compétence Perception, +20 à deux compétences de Connaissances de votre choix. 

Désavantages: Vous devez choisir une morph élevé au départ. 

Morphs Communes: Néo-Aviens, Néo-Hominidés, Octomorph. 

\subsection{Choisir une Faction} \label{sec:choose-faction} 

Après avoir déterminé votre historique, vous devez maintenant choisir la faction principale à laquelle appartient votre personnage. Cette faction représente à priori le groupe qui contrôle actuellement l'habitat ou la station où vit votre personnage, et à laquelle votre personnage à préter allégeance, mais ce n'est pas forcément le cas. Vous pouvez être un membre dissident de votre faction, vivant parmi eux mais s'opposant à certains (ou tous) leurs mêmes centraux et probablement jouant aux agitateurs. Quelque soit le cas, votre faction définit comment votre personnage se représente dans la lutte entre sue se livrent les idéologies post-Chute. 

Vous devez choisir l'une des factions de la liste ci-dessous. Comme l'historique de votre personnage, la faction donenra à votre personnage certaines compétences, traits, limitations ou d'autres caractéristiques. 

Les factions présentées ici décrivent les factions les plus nombreuses et influentes de la traanshumanité, mais d'autres peuvent également exister. À la discrétion de votre maître de jeu, vous pouvez développer ensemble une autre faction de départ non incluse dans cette liste. 

\subsubsection{Anarchiste} \label{sec:anarchist} 

Vous êtes opposés à la hiérarchie, favorisant les oragnisation sociales horizontale et la prise de décision en démocratie directe. Vous coyez que le pouvoir corrompt systématiquement et que tout le monde devrait avoir son mot à dire dans les décisions qui affectent leur vie. D'après les politiques primitives et restrictives du système intérieur et de la Junte Jovienne, cela fait de vous un truand irresponsable au mieux et un terroriste au pire. de votre point de vue, ce n'est que de la comédie ventant de gouvernements qui maintiennent leur population en ordre grâce à l'oppression économique et des menaces de violences. 

Avantages: +20 à une compétence de votrre choix, +30 compétence Réseau: Autonomistes. 

Désavantages: Aucun 

Morphs Communes: Toutes 

\subsubsection{Argonaute} \label{sec:argonaut} 

Vous faites parti d'un mouvement scientifique et techno-progressiste qui cherche à résoudre les injustices et les inégalités de la transhumanité par la technologie. Vous défendez l'accès universel à la technologie et aux soins, les modèles de production open source, la liberté morphologique et la démocratisation. Vous essayez d'éviter les politiques factionaliste qui entraîne la division, considérant la sépration de la transhumanité comme une gêne à sa survie. 

Avantages: +10 à deux compétences Technique, Académique: [domaine] ou Profession: [domaine]; +20 compétence Réseau: Scientifiques 

Désavantages: Aucun 

Morphs Communes: Toutes 

\subsubsection{Barsoomien} \label{sec:barsoomian} 

Votre foyer est l'arrière pays sauvage Martien. Vous êtes un "redneck," un membr de la classe sociale Martienne basse des zones rurales qui se trouvent régulièrement en conflit avec les politiques et les objectifs des dômes hypercorps et avec la Ligue Tharsis. 

Avantages: +10 Parkour, +10 à une compétence de votre choix, +20 compétence Réseau: Autonomistes. 

Désavantages: Aucun 

Morphs Communes: Boîtiers, Plates, Rusteurs, Spliceurs, Synths. 

\subsubsection{Bordés} \label{sec:brinker} 

Vous, ou votre faction, rechigne à traiter avec le reste de la transhumanité et à s'impliquer dans les différentes affaires en cours dans le reste du système. Votre groupe particulier pourrait avoir cherché à s'imposer l'isolation afin de poursuivre leurs propres intérêts, ou avoir été exilés en raisons de leurs croyances impopulaires. A moins que vous ne soignez simplement un solitaire qui préfère les vastes étendues vide de l'espace a socialiser avec les autres. Vous pourriez être un religieux dans un culte primitiviste, un utopiste ou quelque chose dont la transhumanité ne veut pas. 

Avantages: +10 compétence Piloter: Vaisseau, +10 à une compétence de votre choix, +20 à une compétence Réseau: [doamine] de votre choix. 

Désavantages: Aucun 

Morphs Communes: Toutes 

\subsubsection{Criminel} \label{sec:criminal} 

Vous êtes impliqué dans l'univers criminel du monde sous-terrain. Vous pouvez travailler avec l'une des factions majeures du système Sol - les traides, le night Cartel, l'ID Crew, les Nine Lives, la Familiae - ou avec l'un des opérateurs locaux se concentrant sur un habitat particulier. Vous pouvez être un membre à vie, une recrue réticente ou juste un indépendant attendant le prochain job. 

Avantages: +10 compétence Intimidation, +30 compétence Réseau: Criminel 

Désavantages: Aucun 

Morphs Communes: Toutes 

\subsubsection{Extropien} \label{sec:extropian} 

Vous êtes un supporters anarchiste du marché libre et de la propriété privée. Vous vous opposez au gouvernement et favoriser un système où la sécurité et les affaires légales sont gérés par des compétiteurs privés. Que vous vous considériez comme un anarcho-capitaliste ou comme un mutualiste (une différence que seuls les autres Extropiens peuvent faire), vous occupez une zone intermédiaire entre les hypercorps et les autonomistes, traitant avec les deux mais considérés par aucun. 

Avantages: +10 compétence Persusasion, +20 compétence Réseau: Autonomistes, +10 compétence Réseau: Hypercorps. 

Désavantages: Aucun 

Morphs Communes: Toutes 

\subsubsection{Hypercorp} \label{sec:hypercorp} 

Vous êtes originaire d'un habitat contrôllé par les hypercorporations. You pouvez être un entrepreneur hypercapitaliste, un socialite hédoniste or un travailleur du vide, mais vous acceptez le fait que certaines libertés doivent être sacrifiées pour la sécurité. 

Avantages: +10 compétence Protocole, +20 compétence Réseau: Hypercorps, +10 à n'imprte quelle compétence Réseau: [domaine]. 

Désavantages: Aucun 

Morphs Communes: Exaltés, Olympiens, Spliceurs, Sylphs. 

\subsubsection{Joviaen} \label{sec:jovian} 

Votre faction est connue pour son régime autoritaire, ses idéologies bio-conservatives et ses tendances militaires. De là où vous venez, personne ne fait confiance à la technologie et les humains ont besoin d'être protégés d'eux-mêmes. Pour assurer sa survie, l'humanité doit être capable de se défendre elle-même, et toute croissance sans entrave doit être vérifiée. 

Avantages: +10 à deux compétences d'armes de votre choix, +10 Esquive, +20 compétence Réseau: Hypercorps. 

Désavantages: Doit commencer avec un morph Flat ou Spliceurs, ne peut pas commencer avec du nanoware ou de la nanotechnologie avancée. 

Morphs Communes: Plates, Spliceurs 

\subsubsection{Lunaire} \label{sec:lunar} 

Vous venez de la Lune, le premier monde colonisé hors-Terre. Maintenant surpeuplée et en déclin, la Lune est l'une des rares endroits où les gens peuvent toujours se cramponner aux anciennes identitées ethnique est nationnale de la Terre. Votre maison est également en vue de la Terre, un souvenir permanentqui encourage beaucoup de "Lunistes" à être des Réclamationniste, déplorant l'interdiction hypercorporatiste et arguant que vous devriez avoir le droit de retourner sur la Terre, la terraformer et s'y ré-établir en temps que monde mère. 

Avantages: +10 à une compétence Langue: [domaine] de votre choix, +20 compétence Réseau: Hypercorp, +10 compétence Réseau: Écologiste. 

Désavantages: Aucun 

Morphs Communes: Boîtiers, Plates, Exaltés, Spliceurs, Synths. 

\subsubsection{Mercurien} \label{sec:mercurial} 

Votre faction ne s'inétresse pas à renier leur vrai nature pour devenir plus "humain." Que vous soyez une IAG qui ne mélange pas forcément sa destinée avec celle de la transhumanité ou un élevé qui cherche à préserver et à protéger la vie non-humaine (ou au moins sa propre espèce). Vous pouvez même être une infomorph ou un posthumain qui s'est tellement éloigné des intérêts et des valeurs transhumains que vous vous considérez maintenant comem forgeant une nouvelle forme de vie unique. 

Avantages: +10 à deux compétences de votre chois, +20 à une compétence Réseau: [domaine] de votre choix. 

Désavantages: Aucun 

Morphs Communes: Informorphs, Synths, morphs élevés. 

\subsubsection{Racaille} \label{sec:background-scum} 

C'est le futur que nous attendions tous, et vous en profitez au maximum. Un changement de apradigme s'est produit, et pendant que tous les autres s'en remettent, votre faction l'a embrasser et s'y est révélée. Il n'y a plus d'envie, plus de morts, plus de limite sur ce que vous pouvez être. La racaille s'est immergée dans un nouveau mode de vie, se changeant comme ils le veulent, essayant de nouvelles expériences et repoussant les limites partout où ils le peuvent ... et emmerdent tout ce qui ne peuvent pas le supporter. 

Avantages: +10 compétence Chute Libre, +10 à une compétence de votre choix, +20 compétence Réseau: Autonomistes. 

Désavantages: Aucun 

Morphs Communes: Toutes 

\subsubsection{Socialite} \label{sec:socialite} Vous faites parti des célébrités du système intérieur, de la clique social saturée de média qui définissent les modes, qui propagent les mêmes et qui font et défont des vies d'un murmure, d'une allusion ou d'un arrangement de l'ombre. Vous êtes à la fois une icône et un suivant dévoué. La culture n'est pas seulement votre vie, c'est votre arme de prédilection. 

Avantages: +10 compétence Persusasion, +10 compétence Protocole, +20 compétence Réseau: Média. 

Désavantages: Ne peut démarrer ni avec une morph flat, spliceur ou dans un pod, ni avec des morphs élevée ou syntéhtique. 

Morphs Communes: Exaltés, Olympiens, Sylphs. 

\subsubsection{Titanien} \label{sec:titanian} 

Vous faires parti de la cyberdémocracie socialiste du Commonwealth Titanien. Contrairement à d'autres projets autonomiste, l'effort collecitf Titanien a assemblé quelques projets d'infrastructure impressionant tels qu'approuvés par la Pluralité Titanienne et menés à bien par des microcorporations publique. 

Avantages: +20 à deux compétence Technique ou Académique de votre choix, +20 compétence Réseau: Autonomistes. 

Désavantages: Aucun 

Morphs Communes: Toutes 

\subsubsection{Ultimes} \label{sec:ultimate} 

Votre faction perçoit pleinement le potentiel du futur de la transhumanité et considère le reste de la transhumnité comme faible et hédoniste. La transhumanité est en place pour franchir la prochaine étapes d'évolutions et il est temps pour les transhumains d'être redessiné pour le meilleur de nos capacités. 

Avantages: +10 à deux compétences de votre choix, +20 à une compétence Réseau: [domaine] de votre choix. 

Désavantages: ne peut pas démarrer avec une morph Plates, Spliceurs, Élevées ou dans un pod. 

Morphs Communes: Exaltés, Refaits. 

\subsubsection{Vénusien} \label{sec:venusian} 

Vous êtes un défenseur de la Confédération Morningstar des aérostats vénusiens, pleine de ressentiment vis à vis de l'influence grandissante du Consortium Planétaire et des autres puissance retranchée e conservative du système intérieur. Vous percevez l'ascenssion de votre faction comme une chance de réformer la vieille guarde des politique du système intérieur. 

Avantages: +10 Piloter:Engins aériens, +10 à une compétence au choix, +20 compétence Réseau: Hypercorp. 

Désavantages: Aucun 

Morphs Communes: Boîtiers, Mentons, Exaltés, Spliceurs, Synths, Sylphs. 

\subsection{Dépenser les Points Gratuits} \label{sec:spend-free-points} 

Chaque personnage reçoit au départ un nombre équivalent de points gratuits pour des choses comme la rep et les aptitudes. Ces points gratuits sont le point de départ pour assembler votre personnage, ne vous inquiétez donc pas si vous ne parvenez pas à obtenir des scores aussi élevés que ce que vous aimeriez. Dans la prochaine étape de la création de personnages, vous gagnerez des points supplémentaires avec lesquels vous pourrez personnaliser votre personnage (voir la section Dépenser les Points de Personnalisation, p. 135). 





\begin{quotation} Exemple 

Tai est en train de créer un personnage. Elle décide de créer un charognard/récupérateur de débris qui aurait commencé en tant que Colon Lunaire mais qui serait maintenant un Bordé. À eux deux, sa faction et son historique donennt à Tai +20 à la compétence Réseau: Autonomistes, +20 à la compétence Réseau: Hypercorp, +10 à la compétence Piloter: Vaisseau et +10 à la compétence Piloter: véhicule terrestre. Elle a également +10 dans deux autres compétences (dont l'une étant une Acadméique, une Profession ou un domaine Technique) qu'elle pourra choisir plus tard. Tai démarre avec 105 points pour les aptitudes, qui permettent de mettre 15 points à chaque aptitude. Elle veut que son personnage soit impulsif et antisocial, elle réduit donc immédiatement son AST et sa VOL à 10. Elle veut également ête intelligente et rapide, elle prends donc les 10 points que cela lui a donné pour augmeneter sa COG et ses REF à 20. Ses aptitudes sont donc: 

\begin{center} \begin{tabular}{ccccccc} COG &COO &INT &REF &AST &SOM &VOL \\ 

20 &15 &15 &20 &10 &15 &10 \\ 

\end{tabular} \end{center} 

Elle note son Moxie de 1 et choisit sa langue natale (Chinois) à 85, tout les deux offerts. En notant ses 5 000 crédits, Tai divisent son score de Rep équitablement entre @-rep et c-rep, leur donnant 25 points à chacun. 

Elle à maintenant 1 000 points pour personnaliser son personnage. Elle veut être chanceuse, elle commence donc par dépenser 60 (4 X 15) PP pour augmenter son Moxie de 1 à 5. Elle décide également qu'elle veut que son personnage soit meilleur pour repérer les choses, elle augmente donc son INT de 15 à 10, au coût de 50 PP (5 x 10). Jusqu'à présent, elle a dépenser 110 CP. Elel doit acheter au moins 400 points de compétences Active, elle s'y attaque donc. Elel sait que les compétences sont liées aux aptitudes et qu'elle deviennent plus cher au-delà de 60, elle décide donc que le maximum qu'elle dépensera sur une seule compétence sera 40 (puisque son aptitude la plus élevée est de 20). Elle choisist ses compétences, assigne les points et ajoute les compétences aux aptitudes liées. Voici ce avec quoi elle commence, en notant les points dépensé sur chacune et la valeur totale (incluant l'aptitude) noté entre parenthèses. Armes à Rayons (COO) 30 (45), Escalade (SOM) 30 (45), Démolition (COG) 40 (60), Esquive (REF) 30 (50), Chute Libre (REF) 40 (60), Parkour (SOM) 30 (45), Matériel: Aérospatialle (COG) 40 (60), Infiltration (COO) 30 (45), Interface (COG) 20 (40), Navigation (INT) 40 (60), Perception (INT) 40 (60), PErsuasion (AST) 20 (30), Recherche (COG) 20 (40) et Escamotage (INT) 40 (60). Cela lui coûte 450 PP, elle a donc dépensé un total de 560 PP jusqu'ici. Elle dépense maintenant 300 points de compétences de Connaissances: Académique: Astrophysique (COG) 40 (60), Académique: Ingénierie (COG) 40 (60), Académique: Histoire de la Chute (COG) 40 (60), Art: Sculpture (INT) 40 (60), Intérêt: Statiosn Bordées (COG) 40 (60), Intérêts: Conspirtaion (COG) 30 (50), Langue: Anglais (INT) 40 (60), Profession: Estimation (COG) 40 (60), Profession: Commerce de Récupérateur (COG) 40 (60). 

Cela lui coûte encore 350 PP, amenant le total de PP dépensés à 910. 

En ajoutant ses compétences d'historique et de faction, elle a également Réseau: AUtonomistes (AST) 30 (40), Réseau: Hypercorp (AST) 30 (40), Piloter: Vaisseau (REF) 30 (50), Piloter: Engins terrestres (REF) 30 (50). Elle prend le bonus de +10 et l'attribue à Esquive (l'amenant à 60) et applique l'autre +10 à Académique: Économie (COG) 30. 

Avec les 90 PP restants, Tai s'attaque à la Rep. Tai veut avoir beaucoup de bonnes connexions, elle augmente donc ses deux score de rep de 30 points chacun, pour un coût de 6 PP. Elle décide également qu'elle a besoin d'un peu de crédibilité avec les criminels, elle achète donc la g-rep à 40 pour 4 PP de plus. Il lui reste donc 80 PP. 

Le personnage de Tai a besoind 'un corps, et elle décide que le bounceurs est le plus adapté au style de vie nomade et spatial de son bordé. Cela lui co ûte encore 40 PP, la liassant avec 50 PP a dépenser. 

En reconsidérant ses compétences, elle décide de développer sa compétence Piloter: Vaisseau de 50 à 70. Cela lui coute 10 PP pour l'amener à 60, puis 20 de plus pour la pousser jusqu'à 70, pour un coût total de 30 PP. Elel veut également développer sa compétence Escamotage de 60 à 70, pour un coût de 20 PP. Cela utilise bien ses derneirs PP. 

En regardant les Traits, Tai décide que Conscience Spatiale pourrait être un bon choix pour son charognard. A un coût de 10 PP, ele aura besoin de prendre un autre trait négatif pour compenser. Elle choisit Séquelle Neuronale (Synaesthesie) - une maladie qu'elle a récupéré d'un nanovirus déchaîné pendant la Chute. 

Le spoints de Tai sont maintenant tous dépensés. \end{quotation} 

\subsubsection{Aptitudes de Départ} \label{sec:starting-aptitudes} 

Votre personnage reçoit 105 points gratuits à distribuer parmi les ses 7 aptitudes: Cognition, Coordination, Intuition, Réflexes, Astuce, Somatique et Volonté (voir Aptitudes p. 123). (Cela se divise à une moyenne de 15 points par aptitude, il peut donc être plus simple de donner 15 points à chacune et ensuite d'ajuster en fonction, en augmenter certaines, en en réduisant d'autres.) Chaque aptitude doit recevoir au moins 5 points (à moins que vous n'ayez choisi le trait Frêle, voir p. 149), et aucune aptitude ne peut dépasser 30 points (sauf si vous avez choisit le trait Aptitude Esxceptionnelle p. 146). Notez que certaines morphs (plates et spliceurs par exemple) peuvent également mettre plafonner le maximum de vos aptitudes (voir Maximum d'Aptitudes, p. 124). 

Pour simplifier le tout, il est recommandé que les scores d'aptitudes soient gérées par multiples de 5, mais ce n'est pas une nécessité. 

\subsubsection{Langue Natale} \label{sec:native-tongue} 

Chaque personnage reçoit la compétence compétence de leur Langue naturelle a un niveau de 70 + INT gratuitement. Cette compétence peut-être améliorée avec des PP (voir plus bas). 

\subsubsection{Moxie de Départ} \label{sec:starting-moxie} 

Chaque personnage commenace abec une stat Moxie de 1 (voir Moxie, p. 122). 

\subsubsection{Crédit} \label{sec:starting-credit} 

Tout les personnages reçoivent 5 000 crédits avec lesquels acheter du matériel pendant la création de personnage, à moins que vous n'ayez l'historique Évacué de la CHute ou Ré-Instantié (auquel cas vous démarrez avec 2 500 ou 0 crédits, respectivement). Voir Acheter du Matériel, p. 136, pour de plus amples détails. 

\subsubsection{Rep} \label{sec:starting-rep} Votre peronnage n'est pas un débutant complet. Vous recevez 50 points de rep à répartir entre les différentes réseaux de réputation de votre choix (voir Réputation et Réseaux Sociaux, p. 285). 

\subsection{Dépenser les Points de Personnalisations} \label{sec:spend-customization-points} Maintenantq ue vous avez les bases de votre eprsonnage définie, vous pouvez dépenser des Points de Personnages (PP) pour détailler finement votre personnage. Chaque personnage reçoit 1 000 PP qui peuvent être utilisé pour augmenter les aptitudes, acheter des compétences, acquérir plus de Moxie, acheter plus de crédit, élever votre rep ou acheter des traits positifs. Vous pouvez également prendre des traits négatifs pour gagner encore plus de PP avec lesquels personnaliser votre personnage. Ce process de personnalisation devrait être utilisé pour bidouiller votre personnage et le spécialiser de la manière dont vous le désirez. 

Si un maître de jeu désire un autre niveau de jeu, le total de PP peut être ajusté. Pour un scénario dans lequel les personnages sont plus jeunes ou moins expérimenté, le nombre de PP peutêtre réduit à 800 ou même 700. D'un autre côté, si vous voulez créer des personnages qui démaare comme des vétérans endurcis, le nombre de PP peut être augmenter à 1 100 ou 1 200. 

Toutes les personnalisations ne sont pas éagles - les aptitudes, par exemple, sont bien plus importante que les compétences individuelles. POur refléter ça, les PP doivent être dépenser à un taux spécifique en fonction des améliorations voulues. 

\begin{quotation} Point de Personnalisation \begin{itemize} \item 15 PP = 1 point de Moxie \item 10 PP = 1 point d'aptitude \item 5 PP = 1 exploit psi \item 5 PP = 1 spécialisation \item 2 PP = 1 point de compétence (61-80) \item 1 PP = 1 point de compétence (jusqu'à 60) \item 1 PP = 1 000 crédit \item 1 PP = 10 rep \end{itemize} \textit{ Le coût des traits et des morphs varient comme spécifié à chaque fois.} \end{quotation} 

\subsubsection{Personnaliser les APtitudes} \label{sec:customizing-aptitudes} 

Augmenter votre score d'Aptitude est relativement cher et coûte 10 PP par points d'aptitudes. Comme noté au-dessus, aucune aptitude ne peut être augemntée au delà de 30. Gardez en tête que votre morph peut également vous fournir certains bonus d'aptitude. 

\subsubsection{Augmenter le Moxie} \label{sec:increasing-moxie} 

Le Moxie peut être élevée au coût de 15 PP par points de Moxie. Le niveau maximum auquel le Moxie peut-être élevé est de 10. 

\subsubsection{Compétences Apprises} \label{sec:buying-learned-skills} 

Chaque eprsonnage doit acheter un minimum de 400 points de compétences en compétences Active et 300 en compétences de Connsaissance (voir Compétences, p. 170). Les compétences sont achetées au coût d'1 PP par point. Gardez en tête que les compétences apprises commencent au nivau de l'aptitude liée. Par exemple, si vous voulez augmenter une compétences à 30 et que l'aptitude liée est à 10, vous devrez dépenser 20 PP. Le sbonus aux compétences venant de l'historique ou de la faction doivent également être appliquée à la compétence avant de commencer à l'augmenter. Dans un but de simplification, il est recommandé que les compétences soient achetées par multiple de 5, mais ce n'est pas une nécessité. Augmenter une compétence au-delà de 60 ets cher. Chaque point au delà de 60 coûte double. Augmenter une compétence ayant une aptitude liée de 20 jusqu'à 70 coûte 60 PP. 40 points pour aller de 20 à 60, et 20 de plus pour aller de 60 à 70. Aucune compétence appris ene peut dépasser les 80 pendant la création de eprsonnage (sauf si vous possédez le trait Expert, p. 146). Même si les compétences de Connaissances sont regroupées en 5 compétences, chacune d'entre elle est une compétence à domaine (p. 172) ce qui implique qu'elles peuvent être choisie plusieurs fois pour différents domaines. Une liste complète de compétences peut être trouvée à la p. 176. 

\subsubsection{Spécialisations} \label{sec:buying-specializations} 

Des spécialisations (P. 173) peuvent également être achetée au coût de 5 PP par spécialisation. Vous pouvez acheter des spécialisatsion à la fois pour les compétences Active ou de Connaissances. Une seule spécialisation ne peut être achetée, et elles ne peuvent l'être que pour les compétences qui ont un niveau de 30+. 

\subsubsection{Acheter Plus de Crédit} \label{sec:buying-credit} 

Si vous voulez plus de crédit à dépenser sur votre matériel, chaque PP vous rapportera 1 000 crédits. Voir Obtenir du Matériel, p. 136, pour les détails sur l'achat de matériel. Vous ne pouvez dépenser plsu de 100 PP pour obtenir des crédits supplémentaires. 

\subsubsection{Augmenter la Rep} \label{sec:increasing-rep} 

Si vous voulez que votre personnage commence le jeu avec plein de capital social, vous pouvez augmenter votre/vos score(s) de rep au coût d'1 PP pour 10 popints additionels. Aucun score individuel de Rep ne peut dépasser les 80? et le montant maximum de PP a dépenser sur la Rep est de 35 points. 

\subsubsection{Morph de Départ} \label{sec:starting-morph} 

Probablement l'utilisation la plus importante des PP est d'acheter la morph avec laquelle votre personnage commencera le jeu. Cela peut-être la forme corporelle originale dans laquelle vous avez commencé votre vie, ou simplement l'incarnation que vous habitez actuellement. Les morphs disponibles sont listées  partir de la p. 139. Notez que toutes les bonus aux compétences ou aux aptitudes fournis par la morph sont appliqués après que tous les PP soient dépensés. En d'autres mots, ces bonus n'affectent pas le coût d'achat des aptitudes et des points de compétences pendant la génération de personnage. Aucune aptitude ne peut-être modifiée au-delà de 40. 

\subsubsection{Acheter des Traits} \label{sec:purchasing-traits} 

Les traits représentent des qualités spécifiques de votre personnage qui peuvent l'aider ou le freiner. Les traits positifs fournissent des bonus dans certaines situations, et chacun d'eux à un coût en PP associé. Vous ne pouvez pas dépenser plus de 50 PP sur les traits positifs. Les traits négatifs infligent des désavantages à votre personnage, mais vous rapporte des PP supplémentaire que vous pouvez dépenser pourpersonnaliser votre personnage. Vous ne pouvez pas obtenir plus de 50 PP  de traits négatifs, et pas plus de 25 d'entre eux ne epuevent être des traits négatifs de morphs. Les traits positifs sont listés à la p. 145, les traits ngatifs le sont p. 148. Notez que les traits que vous recevez de votre historique ou de votre faction ne vous coûtent ni ne vous rapportent de PP. les traits listés en tant que traits de morphs s'appliquent à la morph, pas à l'ego. Si un personnage change de morphs, ces traits sont perdus (et de nouveaux traits de morph peuvent être obtenus). Les traits de morph doivent être acheter comme tous les autrs traits pendant la génération de personnage. 

\subsubsection{Exploits Psi} \label{sec:purchasing-psi-sleights} 

Les personnages qui achètent le trait Psi (p. 147) peuvent dépenser des PP pour acheter des exploits (voir Exploit, p. 223). Ils représentent des capacités psi particulière que le personnage a apprise. Le coût d'achat d'un exploit est de 5 PP. Pas plus de 5 exploits psi-chi et 5 exploits psi-gamma ne peuvent être achetés pendant la création de personnage. Notez que chaque bonus de compétence ou d'aptitude obtenus avec des exploits sont traités comme des modificateurs; ils s'appliquent après que tous les PP aient été dépensés et n'affectent pas le coût d'achat des compétences ou des aptitudes pendant la création de personnage. 

\subsection{Obtenir du Matériel} \label{sec:purchase-gear} 

Peu importe de quelel faction vous venez, vous utilisez les Crédits pour acheter du matériel pendant la création de personnage. Une liste complète du matériel et des coûts peuvent être trouvé dans le chapitre Équipement, p. 294. Le coût moyen de chaque catégorie de prix doit être utilisé lorsqu'il s'agît de calculer le prix du matériel. Chaque personnage démarre avec une piède d'équipement gratuitement: une muse standard (p. 332). C'est un compagnon IA nuémrique que le eprsonnage possède depuis qu'il est un enfant. Additionnellement, chaque personnage démarre avec 1 mois d'assurance sauvegarde (p. 330) sans coût supplémentaire. 

\begin{center} \begin{tabular}{|c|c|c|} \hline

\multicolumn{3}{|c|}{Coût du Matériel} \\ \hline

Catégorie &Étendue (Crédits) &M oyenne(Crédits)\\ \hline

Trivial &1-99 &50\\ \hline

Bas &100-499 &250\\ \hline

Modéré &500-1,499 &1,000\\ \hline

Élevé &1,500-9,999 &5,000\\ \hline

Cher &10,000+ &20,000\\ \hline \end{tabular} \end{center} 

Il n'y a pas de limitations autre que celles fixées par le maître de jeu sur le matériel accessible aux personnage à la création de personnage. Les joueurs et le maître de jeu devraient garder en tête l'historique et la faction du personnage. Comme certaines pièces d'équipement sont extrêmement restreintes dans certains habitats voire franchement illégale, il peut être nécessaire d'avoir une explication plausible sur la façon qu'un personnage d'un tel endroit peut obtenir un équipement de ce type. si il n'y a pas d'explicatiosn plausible, le maître de jeu peut choisir de ne pas autoriser cet équipement. Le pointd ed épart de la partie devrait également être pris en considération. Un personnage de la République Jovienne restrictive pourrait avoir des difficultés à expliquer comment il/elle a obtenu une machine d'abondance illégale dans la République, mais si le jeu démarre à bord d'une barge racaille où tout est disponible et où tout est autorisé, alros une telle explication devient bien plsu simple. 

La seule exception à l'achat de matériel avec des crédits est l'acaht de morph supplémentaires. Les personnages peuvent acheter des morphs supplémentaires lors de la création de personnage, mais elles doivent être achetées avec des PP. Le joueur choisit une morph dans laquelle les personange est incarné. Les morpshs upplémenatires éncessitent aussi de payer les services d'une banque de corps (p. 331). 

Notez que tout bonus de compétences ou d'aptitude hérité de l'équipement est traité comme une modification; ils ne sont appliqués qu'après que tous els PP aient été dépensés et n'affectent pas le coût d'achat des compétences ou des aptitudes pendant la création de personnage. 

\subsection{Choisir les Motivations} \label{sec:choose-motivations} 

L'étape suivante est de déterminer 3 motivations personnelles à votre personnage (voir Motivations, p. 121). Ce sont des mêmes, sous la forme d'idéologies ou d'objectifs, que votre personnage cherche à atteindre. Ils peuvent être aussi spécifique que "battre le chef des triades local" ou aussi large que "promouvoir l'hypercapitalisme," et ils peuvent être à court ou long terme. Quelques motivations types sont fournies sur la table d'Exemple de Motivations (p. 138). Vous devriez travaillez avec votre maître de jeu au moment où vous déterminez vos motivations, car elles peuvent être utilisée pout propulser l'histoire et des scénarios spécifiques peuvent être construits autour des buts de votre personnage. Certaines motivations de votre personnage peuvent changer par la suite (voir Changer de Motivation, p. 152). les motivations aiderons votre personnage à récupérer des points de Moxie (p. 122) et à gagner des points de Rez supplémentaires pendant le jeu (p. 384). 

Le smotivations doivent être listée sur votre fiche de personne comme de simples mots ou des phrases trés courtes, accompagné d'un symbole + ou - selon que vous supportiez ou que vous vous opposiez à l'idée. Par exemple "+Notoriété" indiquera que votre personnage cherche à devenir une personnalité médiatique célèbre, alors que "-Récupérer la Terre" signifie que votre personnage s'oppose aux buts des réclamationnistes. 

\subsubsection{Exemple de Motivations} \label{sec:example-motivations} 

\begin{tabular}{lll} Contact Étranger &Anarchisme &Expression Artistique\\ Bioconservatisme &Éducation &Exploration \\ Notoriété &Fascisme &Hédonisme \\ Hypercapitalisme &Immortalité &Libertarianisme \\ Libération Martienne &Liberté Morphologique &Nano-écologie \\ Open Source &Carrière Personnelle &Développement Personnel\\ Philanthropie &Préervationisme &Récupérer la Terre\\ Religion &Recherche &Droits des (IA/Infomorph/Pod/Élevés)\\ Esclavage des (IA/Infomorph/Pod/Élevé) &Socialisme &Techno-Progressivisme \\ Vengeance &Souveraineté Venusienne &Prospérité\\ \end{tabular} 

\subsection{Touches Finale} \label{sec:final-touches} 

maintenant que tout est en place, il reste quelques étapes finales. 

\subsubsection{Stats Restantes} \label{sec:remaining-stats} 

Quelques stats doivent maintenant être calculée et additionnée à votre fiche de personnage: 

\begin{itemize} \item Lucidité (p. 122) est égale à la VOL x 2 de votre personnage. \item Seuil de Trauma (p. 122) est égal à votre LUC divisée par 5 (arrondir au supérieur). \item Seuil de Folie (p. 122) est égal à LUC x 2. \item Initiative (p. 121) est égal à (REF + INT) x 2 de votre personnage. \item Bonus de dommage (p. 123) pour la mélée est égal à SOM $\div$ 10 (arrondir à l'inférieur). \item Seuild e Mort (p. 122) est égal à DUR x 1,5 (biomorphs, arrondir au supérieur) ou à DUR x 2 (synthmorpsh). \item Vitesse (p. 121) est égale à  1 (3 pour les infomorphs), modifié de manière appropriée par les implants. \end{itemize} 

\subsubsection{Détailler le Personnage} \label{sec:detailing-the-character} 

L'étape finale de la création de personnage est de remplir les détails et déterminer comment votre personnage se comporte et ce qu'il/elle pense. L'historique de votre personnage est un bon point de départ car il définit d'où il/elle vient, mais il peut être développé. Que pesne-t-il/elle de sopn enfance? Y A-t-il/elle encore des attache? Comment il/elle a évolué de son roigine pour arriver dans la Faction dont il/elle fait parti? Est-il/elle un fervent défenseur des objectifs de sa Faction ou est-il/elle en opposition? Comment le personnage perçoit-il/elle les autres Factions? 

Ensuite, jettez un œil aux compétence et autres points structurants - ils peuvent aussi raconter une histoire. Comment a-t-il/elle acqui ces compétences? Pourquoi? Comment a-t-il/elle développé son score de rep (ou son absence)? Comment a-t-il/elle été connecté avec les groupes détaillés dans leurs compétences Réseau? Que disent les traits du personnage à son sujet? Comment a-t-il/elle obtenu sa morph actuelle? Est-ce que c'est leur morph d'origine? Sinon, qu'est-il arrivé à son premier corps? Prenez aussi en considération les facteurs majeurs des Motivations, toutes ces questiosn pourront vous aider à construire une image définissante de votre personnage. Il n'est aps nécessair de définir intégralement votre personnage bien entendu - des blancs peuvent toujours être laissés pour être remplis plus tard. Assembler les points que vous avez définit jusqu'à présent vous aidera à présenter votre personnage comme un tout, un individu unique plutôt qu'un modèle vide. En guise d'étape finale, prenez quelques minutes pour déterminer quelques caractéristiques identitaires et quelques traits de eprsonnalité qui vous aiderons à définir votre personnage aux autres. Cela pourrait être une manière de parler, un sale caractère, une punchline qu'ils utilisent fréquemment, un style unique, un comportement répétitif, un maniérisme génant ou n'importe quoi d'autre de similaire sur lequel il est facile de s'accorcher. De telles idiosyncratie donnent un support aux autre joueurs afin de développer des opportunités d'interprétation. 

\section{Morphs de Départ} 

Chaque morph est associée à un coût en PP. Elel fournie aussi les stats de \textbf{Durabilité} et de \textbf{Seuil de Blessure} du personnage, et beaucoup modifient également l'Initiative, la Vitesse et certaines aptitudes et compétences apprise. Un coût en crédit est également fourni, mais il fait référence au coût d'achat de ces morphs en cours de jeu. 

Bonus d'Aptitude Souple: Certaines morphs ont des bonus d'aptitudes qui peuvent être appliqu"e à une aptitude au choix du joueur. Cela reflète le fait que toutes les morphs ne sont pas équivalentes. En assignant ces bonus d'aptitude universels, chaque amélioration doit être appliquée à une aptitude différente; vous ne pouvez augmenter une aptitude qui est déjà améliorée par cette morph. Une fois que les bonus aux aptitude d'une morph particulière ont été assignés, ils deviennet permanent pour cette morph (p.ex. si un eutre personnage se réincarne dans cette morph, les bonus restent les mêmes). 

\subsection{Biomorphs} \label{sec:starting-biomorphs} 

Les biomorphs sont des incarnations complètement bioloique (habituellement équipée d'implants), mise au monde naturellement ou par un exo-utérus et amené à l'aâge adulte soit naturellement soit à un rythme senseiblement accéléré. 

\subsubsection{Plates} \label{sec:starting-flats} 

Les plates sont l'humain de base non modifié, né avec tout les défauts naturels, des maladies héréditaires et d'autres mutations génétiques que l'évolution applique avec amour. Les plates sont extrêmement rare - la plupart sont mortes avec le reste de l'humanité pendant la Chute. La pluaprt des nouveaux enfants sont des spliceurs - analysée et réparé génétiquement au minimum - excepté dans les habitats où les plates sont considrés comme des citoyens de seconde zone et des travailleurs contractés. 

\begin{description*} \item[Implants] Aucun \item[maximum d'Aptitude] 20 \item[Solidité] 30 \item[Seuil de Blessure] 6 \item[Désavantages] Aucun (le trait Défaut génétique est commun) \item[Coût en PP] 0 \item[Coût en Crédit] Élevé \end{description*} 

\subsubsection{Spliceurs} \label{sec:starting-splicers} 

Les spliceurs sont les humains génétiquement réparés. Leur génome a été nettoyé des maladies héréditaires et optimisé pour la santé et l'apparence, mais il n'a pas été amélioré outre-mesure. Les spliceurs constituent la majorité de la transhumanité. 

\begin{description*} \item[Implants] Biomods de Base, Inserts Mesh Basiques, Pile Corticale\item[Maximum d'Aptitude] 25 \item[Durabilité] 30 \item[Seuil de Blessure] 6 \item[Avantages] +5 à une aptitude au choix du joueur\item[Coût en PP] 10 \item[Coût en Crédit] Élevé \end{description*} 

\subsubsection{Exaltés} \label{sec:starting-exalts} 

Les morphs Éxaltés sont des humains génétiquement améliorés, conçu pour mettre en évidence certains traits. Leur code génétique a été arrangé pour les rendre plus sains, plsu intelligents et plus attractifs. Leur métabolisme est modifié pour les prédisposer à rester athéltique et en forme pour la durée de leur espérance de vie étendue. 

\begin{description*} \item[Implants] Biomods de Base, Inserts Mesh Basiques, Pile Corticale\item[Maximum d'Aptitude] 30 \item[Solidité] 35 \item[Seuil de Blessure] 7 \item[Avantages] +5 en COG, +5 à trois autres aptitudes au choix du joueur\item[Coût en PP] 30 \item[Coût en Crédit] Cher \end{description*} 

\subsubsection{Mentons} \label{sec:starting-mentons} 

Les Mentons sont génétiquement modifié pour augmenter les capacité cognitives, en particulier l'apprentissage, la créativité, l'attention et la mémoire. Les rumeurs font état de mentons suepraméliorés avec des mods d'intelligence pbien plus extrême, mais la bidouille cérébrale est extrêmement difficile, et de nombreuse tentative pour reconcevoir les facultés mentales résultent en un fonctionnement affaibli, à de l'instabilité ou à la folie. 

\begin{description*} \item[Implants] Biomods de Base, Inserts Mesh Basiques, Pile Corticale, Mémoir Éidétique, Hyepr-Linguiste, Stimulateur Mathématique\item[Maximum d'Aptitude] 30 \item[Solidité] 35 \item[Seuil de Blessure] 7 \item[Avantages] +10 en COG, +5 en INT, +5 en VOL, +5 à une aptitude au choix du joueur\item[Coût en PP] 40 \item[Coût en Crédit] Cher \end{description*} 

\subsubsection{Olympiens} \label{sec:starting-olympians} 

Les Olympiens sont des humains avec des capacités athlétiques améliorées comme l'endurance, la coordination œil-main et les fonctions cardio-vasculaire. Les Olympiens sont fréquemment rencontrés chezz les athlètes, les danseurs, les adeptes du parkour et les soldats. 

\begin{description*} \item[Implants] Biomods de Base, Inserts Mesh Basiques, Pile Corticale\item[Maximum d'Aptitude] 30 \item[Solidité] 40 \item[Seuil de Blessure] 8 \item[Avantages] +5 en CO0, +5 en REF, +10 en SOM, +5 à une autre aptitude au choix du joueur\item[Coût en PP] 40 \item[Coût en Crédit] Cher \end{description*} 

\subsubsection{Sylphs} \label{sec:starting-sylphs} 

les morphs Sylphs sont faites sur-mesure pour les icônes médiatique, l'élite socialites, les stars de l'XP, les mannequins et les narcissiques. Les séquences génétiques de Sylphs sont spécifiquement conçue pour paraître belle. Des caractéristiques étéhrées et féériques sont communes, avec des corps fins et souples. Leur métabolisme a également été assainit pour éliminer les ordeurs corporelles déplaisante et leurs phéromones ont été ajustés pour favoriser une attirance universelle. 

\begin{description*} \item[Implants] Biomods de Base, Inserts Mesh Basiques, Pile Corticale, Métabolisme Propre, Phéromones Améliorés\item[Maximum d'Aptitude] 30 \item[Solidité] 35 \item[Seuil de Blessure] 7 \item[Avantages] trait Beauté Ahurissante (Niveau 1), +5 en COO, +10 en AST +5 à une autre aptitude au choix du joueur\item[Coût en PP] 40 \item[Coût en Crédit] Cher \end{description*} 

\subsubsection{Bounceurs} \label{sec:starting-bouncers} 

Les Bounceurs sont des humains génétiquement adaptés aux environnement en zéro-G ou en microgravité. Leurs jambes sont plus souples, et leurs pieds peuvent saisir aussi bien que leurs mains. 

\begin{description*} \item[Implants] Biomods de Base, Inserts Mesh Basiques, Pile Corticale, Patins Antidérapants, Réserve d'Oxygène, Pieds Préhensibles\item[Maximum d'Aptitude] 30 \item[Solidité] 35 \item[Seuil de Blessure] 7 \item[Avantages] Souple (Niveau 1), +5 en COO, +5 en SOM, +5 à une aptitude au choix du joueur\item[Coût en PP] 40 \item[Coût en Crédit] Cher \end{description*} 

\subsubsection{Furys} \label{sec:starting-furies} 

Les Furys sont des morphs de combat. Ces humains transgéniques ont des améliorations génétiques dimensionnées pour l'endurance, la force et les réflexes, et ont des modifications comportementales pour développer l'agressivité et la ruse. Pour limiter les tendances à l'indiscipline et aux comportements machiste, les furys ont des séquences génétiques favorisant les mentalités de meutes et la coopération, et ont tendance à être des femelles biologiques. 

\begin{description*} \item[Implants] Biomods de Base, Inserts Mesh Basiques, Pile Corticale, Armure Biotissées (Légère), Vision Améliorée, Drogues Neurales (Niveau 1), Filtres à Toxines\item[Maximum d'Aptitude] 30 \item[Modificateur de Speed] +1 (Drogues neurales) \item [Solidité] 50 \item[Seuil de Blessure] 10 \item[Avantages] +5 en COO, +5 en REF, +10 en SOM, +5 en VOL, +5 à une autre aptitude au choix du joueur\item[Coût en PP] 75 \item[Coût en Crédit] Cher (minimum 40 000 ¢) \end{description*} 

\subsubsection{Futuras} \label{sec:starting-futuras} 

Une variante d'éxalté, les morphs futura ont été spéciallement fabriquées pour la "Génréation Égarée." Conçus spécifiquement pour la croissance accélérée et adaptée pour la confiance en soi, l'auto-suffisance et l'adaptabilité, les futuras ont été conçues pour permettre à la transhumanité de reprendre ses anciennes bases. Ce programme s'est avéré être un désastre et la conception a été abandonée, mais certains modèles sont toujours actifs, perçus par certains avec dégoût et par d'autres comme des collecotrs ou des bizarrerries exotiques. 

\begin{description*} \item[Implants] Biomods de Base, Inserts Mesh Basiques, Pile Corticale, Mémoire Éidétique, Atténuateurs Émotionnels\item[Maximum d'Aptitude] 30 \item[Solidité] 35 \item[Seuil de Blessure] 7 \item[Avantages] +5 en COG, +5 en AST, +10 en VOL, +5 à une aptitude au choix du joueur\item[Coût en PP] 40 \item[Coût en Crédit] Cher (Exceptionnelelment rare; 50 000+ ¢). \end{description*} 

\subsubsection{Fantômes} \label{sec:starting-ghosts} 

Les Fantômes sont parttiellement conçu pour le combat, mais leur concentration prrincipale ets la furtivité et l'infiltration. Leur profil génétique encourage la vitesse, l'agilité et les réflexes, et leurs esprits sont modifiés pour favoriser la patience et la résolution de problème. 

\begin{description*} \item[Implants] Biomods de Base, Inserts Mesh Basiques, Pile Corticale, Peau Caméléon, Stimulateurs Surrénaux, Vision Améliorée, Patins Antidérapants\item[Maximum d'Aptitude] 30 \item [Solidité] 45 \item[Seuil de Blessure] 9 \item[Avantages] +10 en COO, +5 en REF, +5 en SOM, +5 en VOL, +5 à une autre aptitude au choix du joueur\item[Coût en PP] 70 \item[Coût en Crédit] Cher (minimum 40 000 ¢) \end{description*} 

\subsubsection{Hibernoïdes} \label{sec:starting-hibernoids} 

Les Hibernoïdes sont des humains transgénique ayant leur métabolisme et leur cycles circadien trés lourdement modifiés. Les Hibernoïdes ont un besoind e sommeil réduit, ne nécessitant qune heure ou deux de sommeil par jour en moyenne. Ils ont également la possibilité de déclencher une forme d'hibernation volontaire, arrétant leur métabolisme et leur besoin en oxygène. Les Hibernoïdes font d'excellent voyageurs spatiaux longue-distance et techiciens d'habitats, mais ces morphs sont égelement adoptées par les aissatants personnels et les hypercapitalistes ayant des rythmes de vie non-stop. 

\begin{description*} \item[Implants] Biomods de Base, Inserts Mesh Basiques, Pile Corticale, Régulation Circadienne, Hibernation\item[Maximum d'Aptitude] 25 \item[Solidité] 35 \item[Seuil de Blessure] 7 \item[Avantages] +5 en INT, +5 à une aptitude au choix du joueur\item[Coût en PP] 25 \item[Coût en Crédit] Cher \end{description*} 

\subsubsection{Néoteniques} \label{sec:starting-neonetics} 

Les néoténique sosnt des transhumains modifier pour conserver leur forme enfantine. Ils sont plus petits, plus agiles, plus curieux et consomment moisn de ressources, les rendant idéals pour la vie en habitat ou en vaisseau. Certaines personnes treouvent les incarnations Néoténique comme repoussante, spécifiquement lorsqu'elles sont utilisées dans certains médias ou comme travailleurs sexuels. 

\begin{description*} \item[Implants] Biomods de Base, Inserts Mesh Basiques, Pile Corticale\item[Maximum d'Aptitude] 20 (SOM), 30 (le reste) \item[Solidité] 30 \item[Seuil de Blessure] 6 \item[Avantages] +5 en COO, +5 en REF, +5 en INT, +5 à une autre aptitude au choix du joueur; les néoténiques comptent comme des petites cibles (modificateur de -10 pour toucher en combat)\item[Désavantages] trait Stigmatisation Sociale (Néoténique) \item[Coût en PP] 25 \item[Coût en Crédit] Cher \end{description*} 

\subsubsection{Refait} \label{sec:starting-remade} 

Les refait sont des humains complètements redessinés: l'humain 2.0. Leur système cardiovasculaire est plus robuste, leur système digestif a été assainit et restructuré pour éliminer les défauts et ils ont globalement été optimisés pour être en bonne santé, intelligents et vivre vieux avec de nombreux modificateurs transgéniques. Les refait sont populaire parmi les ultimes. Les remades ressemblent à l'humain, mais sont différents de manières remarquable et parfois effrayante: plus grand, chauves et sans poils, avec un crâne légèrement plsu grand, des nez plus fins et des doigts allongés. 

\begin{description*} \item[Implants] Biomods de Base, Inserts Mesh Basiques, Pile Corticale, Mémoire Éidétique, Métabolisme Propre, Régulation Circadienne, Respiration Améliorée, Tolérance Thermique, Filtres à Toxines\item[Maximum d'Aptitude] 40 \item[Solidité] 40 \item[Seuil de Blessure] 8 \item[Avantages] +10 en COG, +5 en AST, +10 en SOM, +5 à deux aptitudes au choix du joueur\item[Désavantage] trait Apparence Étrange \item[Coût en PP] 60 \item[Coût en Crédit] Cher (minimum 40 000+ ¢). \end{description*} 

\subsubsection{Rusteurs} \label{sec:starting-rusters} 

Adaptées pour la survie avec un équipement minimal dans l'environnement Martien pas encore complètement terraformé, ces morphs transgéniques possèdent une peau isolées pour une thermorégulation plus efficace et un système respiratoire amélioré qui nécessite moins d'oxygène et qui filtre mieux le dioxyde de carbone, entre autres modifications. 

\begin{description*} \item[Implants] Biomods de Base, Inserts Mesh Basiques, Pile Corticale, Respiration Améliorée, Tolérance Thermique\item[Maximum d'Aptitude] 25 \item[Solidité] 35 \item[Seuil de Blessure] 7 \item[Avantages] +5 SOM, +5 à une aptitude au choix du joueur\item[Coût en PP] 25 \item[Coût en Crédit] Élevé \end{description*} 

\subsubsection{Néo-Aviens} \label{sec:starting-neo-avians} 

Les néo-aviens incluent les corbeaux, les corneilles et les perroquets gris élevés à un niveau d'intelligence humain. Ils ont une taille bien plus importante que leur cousins non-élevés (jusqu'à la taille d'un enfant humain), avec des têtes plus grandes en raison de la taille augmentée de leur cerveau. De nombreuses modification transgéniques ont été faites à leurs ailes, leur permettant de conserver des capacités de vol réduite en environnement 1 g, mais leur donnant une physiologie plus proche des chauves-souris afin quelles puissent se plier et se ranger plus facilement, et en y ajoutant des doigts pirmitifs pour la manipulation d'outils basique. Leurs orteils sont également plus articulés et sont maintenant accompagnés d'un pouce opposable. Les néo-aviens se sont bien adaptés aux environnements en microgravité, et sont choisit pour leur petite taille et leur utilisation de ressources réduite. 

\begin{description*} \item[Implants] Biomods de Base, Inserts Mesh Basiques, Pile Corticale\item[Maximum d'Aptitude] 25 (20 en SOM) \item[Solidité] 20 \item[Seuil de Blessure] 4 \item[Avantages] Attaque de Bec/de Griffe (1d10 VD, utiliser la compétence Combat Désarmé), Vol, +5 INT, +10 REF, +5 à une autre aptitude au choix du joueur\item[Coût en PP] 25 \item[Coût en Crédit] Cher \end{description*} 

\subsubsection{Néo-Hominidés} \label{sec:starting-neo-hominids} 

Les néo-hominidés sont les chimpanzés, les gorilles et les orang-outans élevés. Toutes les morphs ont des intelligences améliorés et sont bipèdes. 

\begin{description*} \item[Implants] Biomods de Base, Inserts Mesh Basiques, Pile Corticale\item[Maximum d'Aptitude] 25 \item[Solidité] 30 \item[Seuil de Blessure] 6 \item[Avantages] +5 en COO, +5 en INT, +5 en SOM, +5 à une autre aptitude au choix du joueur, +10 à la compétence Escalade\item[Coût en PP] 25 \item[Coût en Crédit] Cher \end{description*} 

\subsubsection{Octomorphs} \label{sec:starting-octomorphs} 

Ces incarnations de pieuvres élevés se sont révélées être particulièrement efficace dans les environnements en gravité-zéro. Elles ont conservés leurs huits bras, leur capacité caméléonique à changer de couleur de peau, des sacs d'encre et un bec acéré. Elles ont également bénéficiées d'une augmentation de la capacité crânienne et d'une durée de vie étendue, elles peuvent respirer à la fois l'air et l'eau et n'ont pas de structures squelettiques et peuvent donc se compresser à travers des endroits étroits. Typiquement, les octomorphs rampent en gravité-zéro utilisant les ventouses de leurs bras et expulsant de l'air pour se propulser. Ils peuvent même marcher sur deux de leurs bras en faible gravité. Leurs yeux ont été amélioré avec la vision couleur, fournissant un champ de vision à 360° et s'adaptent rotativement pour garder la pupille en forme de fente aligné vers le "haut." Un système vocal transgéniqueleur permet de parler. 

\begin{description*} \item[Implants] Biomods de Base, Inserts Mesh Basiques, Pile Corticale, Peau Caméléon\item[Maximum d'Aptitude] 30 \item[Solidité] 30 \item[Seuil de Blessure] 6 \item[Avantages] 8 bras, Attaque de Bec (1d10 VD, utiliser la compétence Combat Désarmé), Jet d'Encre (attaque aveuglante, utiliser la compétence Armes à Distance Exotique: Jet d'Encre), trait Souple (Niveau 2), Vision à 360-Degré, °30 à la compétence Nage, +10 à la compétence Escalade, +5 COO, +5 INT, +5 à une autre aptitude au choix du joueur\item[Coût en PP] 50 \item[Coût en Crédit] Cher (minimum 30 000+¢) \end{description*} 

\subsection{Pods} \label{sec:starting-pods} 

Les pods (de "pod people") sont des corps biologies développé en cuve avec des cerveaux extrêmement peu développé et qui sont augmentés avec un ordinateur et un système cybernétique implanté. Étant généralement pilotés par des IA, les pods sont socialement défavorisé dans certaines stations, utilisés comme esclaves dans d'autre et sont même illégaux dans certaines zones. En raison de la croissance accélérée pendant la phase de création des pods, et qu'ils sont essentiellement développés en parties sépraées puis assemblés, leur conception biologique inclut de nombreux raccourcis et limite, contrabalancé par des implants et une maintenance régulière. Ils ne possèdent pas de fonction de reproduction. Dans beaucoup d'habitat, le statut légal des pods est le sujet de débat brûlant. Sans mention contraire, les pods sont également considérés comme des biomorphs du point de vue des règles. 

\subsubsection{Pods de Plaisir} \label{sec:starting-pleasure-pods} 

Les Pods de Plaisirs sont exactement ce qu'ils semblent être - de faux humains conçus purement dans un but de distraction intime. les Pods de plaisir ont des grappes de nerfs supplémentaire dans leurs zones érogènes, un contrôle moteur trés précis de certains groupes de muscles, des phéromones améliorés, un métabolisme assainit et les gènes pour ronoronner. Ils sont bien entendus fabriquer pour être attirant, charismatique et amélioré dans d'autres domaines. Les Pods de plaisirs peuvent changer de sexe à volonté pour mâle, femelle, hermaphrodite ou neutre. 

\begin{description*} \item[Implants] Biomods de Base, Inserts Mesh Basiques, Pile Corticale, Métabolisme Propre, Cybercerveau, Phéromones Améliorés, Augmentation Mnémonique, Marionnette, Changement de Sexe\item[Maximum d'Aptitude] 30 \item[Solidité] 30 \item[Seuil de Blessure] 6 \item[Avantages] +5 INT, +5 en AST, +5 à une aptitude au choix du joueur\item[Désavantages] trait Stigmatisation Sociale (Pod de Plaisir) \item [Coût en PP] 20 \item[Coût en Crédit] Élevé \end{description*} 

\subsubsection{Pods Ouvrier} \label{sec:starting-worker-pods} 

A moitié humain exalté, à moitié machine, ces pods basiques sont virtuellement non distinguable des humains. Les pods ouvriers sont souvent utilisés dans les travaux subalternes nécessitant une interaction humaine. 

\begin{description*} \item[Implants] Biomods de Base, Inserts Mesh Basiques, Pile Corticale, Cybercerveau, Augmentation Mnémonique, Marionnette\item[Maximum d'Aptitude] 30 \item[Solidité] 35 \item[Seuil de Blessure] 7 \item[Avantages] +10 en SOM, +5 à une aptitude au choix du joueur\item[Désavantages] trait Stigmatisation Sociale (Pod) \item [Coût en PP] 20 \item[Coût en Crédit] Élevé \end{description*} 

\subsubsection{Novacrabe} \label{sec:starting-novacrab} 

Les novacrabe sont des pod conçu par bio-ingénierie à partir de crabe de cocotier et d'araignée de mer et amené à taille humaine. Les novacrabes sont idéaux pour le travail dans les zones dangereuses ou en tant que travailleurs du vide, policier ou garde du corps étant donné leurs pattes de deux mètre de long, leurs pinces massive et leur armure chitineuse. Ils peuvent grimper et s'en sortir en micro-gravité et ils peuvent supporter une large gamme de pressions atmosphérique (ainsi que els changements de pression soudains) allant du vide aux profondeurs des mers. Les novacrabes possèdent des yeux composés (avec une résolution d'image équivalente à celle de l'œil humain), des branchies, des doigts permettant l'utilisation d'outil sur leur cinquième paire de membre et des cordes vocales transgénique. 

\begin{description*} \item[Implants] Biomods de Base, Inserts Mesh Basiques, Pile Corticale, Respiration Améliorée, Armure Carapace, Cybercerveau, Branchies, Augmentation Mnémonique, Réserve d'Oxygène, Marionette, Tolérance Thermique, Étanchéité au vide \item[Maximum d'Aptitude] 30 \item[Solidité] 40 \item[Seuil de Blessure] 8 \item[Avantages] 10 jambes, Armure Carapace (11/11), Attaque de Pince (2d10 VD), +10 SOM, +5 à deux autres aptitudes au choix du joueur\item[Coût en PP] 60 \item[Coût en Crédit] Cher (minimum 30 000+¢) \end{description*} 

\subsection{Morphs Synthétique} \label{sec:starting-syntheticmorphs} 

Les morphs syntéhtiques sont entièrement artificielle/robotique. Elles sont habituellement manœuvrée par des IA ou par contrôle distant, mais le manque de biomorphs disponible après la Chute a fait que beaucoup d'infugiés se sont résignés à se réincarner dans des coques robotiques, qui étaient également moins chère et plsu rapide à fabriquer et globalement plus disponible. Les synthmorpsh sont cependant toujours perçue avec dédain dans beaucoup d'habitat et considérée comme n'étant une option que pour les plus pauvres et le splus désespérés qui acceptent de s'y réincarner. Les morphs syntéhtiques ne sotn aps sans avantages cependant et sont communément utilisées pour les tâches subalternes, les gros travaux, la construction d'hébaitat et les servics de sécurité. 

Toutes les synthmoprhs bénéficient des avantages suivants: 

\begin{itemize} \item Absence de Fonctions Biologiques. Les synthmorpsh ne s'encombrent pas de trivialités telles que respirer, manger, déféquer, vieillir, dormir ou tout aspect mineur mais crucial de la vie biologique. \item Filtre de Douleur. Les synthmorphs peuvent filtrer leur récepteurs de douleurs, afin qu'elles ne soient pas génées par les blessures ou les dégâts physiques. Cela leur permet d'ignorer le modificateur de -10 pour une blessure (voir Effets des Blessures, p. 207), mais elles souffrent d'un modificateur de -30 à tous les Tests de Perception basés sur le toucher et ne remarquerons pas qu'ils ont étés abîmés sauf si ils réussissent un Test de Perception modifié. \item Immunité aux Armes à Impulsion. Les synthmorphs n'ont aps de système nerveux à déchirer, et leurs électroniques optiques sont soigneusement protégées des interférences. Les attaques à impulsions peuevnt temporairement perturber leurs communications radios sans-fil pendant la durée de l'attaque. \item Robustesse Environnementale. Les synthmorphs sont conçues pour supporter une large gamme d'environnement, de la poussière de Mars aux océans d'Europe en passant par le vide spatial. Ils ne sont affectés que par les températures et pressions atmosphériques les plus extrêmes. Considérez-le comme les traits Tolérance Thermique (p. 305) et Étanchéité au Vide (p. 305). \item Robustesse. Les coques syntéhtiques sont faites pour durer - un fait reflétter par leur meilleure Solidité et leurs seuils d'Armure interne. Leur composition les rends leur attaques physiques plus violentes: appliquez un modificateur de +2 à la VD des attaques à mains nues pour les coques de taille shumaines ou plus grande. \end{itemize} 

\subsubsection{Boîte} \label{sec:starting-case} 

Les boîtes sont des coques robotiques extrêmement bon marché et produites massivement dans le but de fournir une option de remorph abordable pour les infugiées créés par la Chute. Bien qu'il existe beaucoup de variation de boîte, elels sont uniformément considérées comme de mauvaises qualité et inférieures. La plupart des morphs boîtes sont vaguement anthropomorphique, avec une fine structure corporelle, étant juste un peu plsu petite que l'humain moyen et souffrant de disfonctionnement fréquents. 

\begin{description*} \item[Implants] Prise d'Accès, Inserts Mesh Basiques, Pile Corticale, Cybercerveau, Augmentation Mnémonique\item[Mode de déplacement](Allure de déplacement) Marcheur (4/16)\item[Maximum d'Aptitude] 20 \item[Solidité] 20 \item[Seuil de Blessure] 4 \item[Avantages] Armure (4/4)\item[Désavantages] -5 à une aptitude au choix, trait Raté, trait Stigmatisation Sociale (Masse Cliquetante) \item [Coût en PP] 5 \item[Coût en Crédit] Modéré \end{description*} 

\subsubsection{Synth} \label{sec:starting-synths} 

Les synths sont des coques robotiques anthropomorphiques (androïdes et gynéoïdes). Elles sont typiquement utilisés poru les travaux sub-alternes pour lesquels les pods ne sont pas une bonne option. Moins chère que beaucoup d'autres morphs, elles sont souvent utilisées par les personens qui ont besoin d'une morph rapidement et pour pas trop cher ou simplement lors d'un transit entre deux incarnations. Bien qu'elles aient l'air huamnoîdes, les synths sont facilement reconnaissable comme non-biologique à moins qu'elles n'aient l'option masque synthétique (p. 311). 

\begin{description*} \item[Implants] Prise d'Accès, Inserts Mesh Basiques, Pile Corticale, Cybercerveau, Augmentation Mnémonique\item[Mode de déplacement](Allure de déplacement) Marcheur (4/20)\item[Maximum d'Aptitude] 30 \item[Solidité] 40 \item[Seuil de Blessure] 8 \item[Avantages] +5 en SOM, +5 à une aptitude au choix du joueur, Armure (6/6)\item[Désavantages] trait Apparence Étrange, trait Stigmatisation Sociale (Masse Cliquetante) \item [Coût en PP] 30 \item[Coût en Crédit] Élevé \end{description*} 

\subsubsection{Arachnoïdes} \label{sec:starting-arachnoids} 

Les coques robotiques arachnoïdes font 1 mètre de long, sont divisées en deux partie, avec une tête plsu petite comme els araignées ou les termites. Elles possèdent quatre paires de membres rétracaples d'1,5m de long, capables de tourner autour de l'axe corporels et équipés de vérins hydraulique pour propulser le bot avec de petits sauts. Les griffes de manipulations sur chaque membre peuvent être échangée par des mini-roues pour un déplacement en patinage à haute-vitesse. Une pair de bras manipulateur plsu petite et proche de la tête permets une utilisation plsu précise d'outils. En environnement en zéro-G, les arachnoïdes peuvent rétracter leurs membres et manœuvrer grâce à des turbines à poussée vectorielles. 

\begin{description*} \item[Implants] Prise d'Accès, Inserts Mesh Basiques, Pile Corticale, Cybercerveau, Augmentation Mnémonique, Vision Améliorée, membres Supplémenatires (6 membres), Lidar, Membres pneumatiques, Radar\item[Mode de déplacement](Allure de déplacement) Marcheur (4/16), Poussée Vectorielle (8/40)\item[Maximum d'Aptitude] 30 \item[Solidité] 40 \item[Seuil de Blessure] 8 \item[Avantages] Armure (8/8), +5 en COO, +10 en SOM \item[Coût en PP] 45 \item[Coût en Crédit] Élevé (minimum 40 000+¢) \end{description*} 

\subsubsection{Libellulle} \label{sec:starting-dragonfly} 

La morph robotique libellulle prend la forme d'une coque flexible d'un mètre de long possédant des ailes multiples et des bras mnipulateurs. Capable de voler quasi-silencieusement grâce à ses turboréacteurs à double flux et avec une gravité Terreste, les bots libellulles sont encore meilleurs en microgravité. 

\begin{description*} \item[Implants] Prise d'Accès, Inserts Mesh Basiques, Pile Corticale, Cybercerveau, Augmentation Mnémonique \item[Mode de déplacement](Allure de déplacement) Ailé (8/32) \item[Maximum d'Aptitude] 30 (20 en SOM) \item[Solidité] 25 \item[Seuil de Blessure] 5 \item[Avantages] Vol, Armure (2/2), +5 en REF \item[Coût en PP] 20 \item[Coût en Crédit] Élevé \end{description*} 

\subsubsection{Transformers} \label{sec:starting-flexbots} 

Conçus pour remplir des fonctions multi-tâches, les transformers peuvent transformer leur coque pour s'adapter à une gamme de tâches et de situations. Leur structure principale consiste en une demi-douzaine de module interconnectés et de forme adaptable capable de s'auto-transformé en une variété de forme: marcheurs à plusieurs jambes ou tentacules, aéroglisseur et bien d'autre. Chaque module dispode de sa propre unité sensorielle et des doigts fractals ramifiés (cahque capable de se diviser en doigts plus petits, jusqu'à une échelle micrométrique permettant des manipulations ultra-fine). l'ordinateur de contrôle du transformers est également répartit entre les différents modules. Chaque transformers n'est pas plus gros qu'un gros chien, mais plusieurs transformers peuvent s'assembler pour opérer à une échelle de amsse différente, pouvant même s'occuper des tâches telles que la démolition, l'excavation, la fabrication ou l'assemblage robotisé. 

\begin{description*} \item[Implants] Prise d'Accès, Inserts Mesh Basiques, Pile Corticale, Cybercerveau, Doigts Fractals, Augmentation Mnémonique, Conception modulaire, Forme Ajustable \item[Mode de déplacement](Allure de déplacement) Marcheur (4/16), Glisseur (8/40) \item[Maximum d'Aptitude] 30 \item[Solidité] 25 \item[Seuil de Blessure] 5 \item[Avantages] Armure (4/4) \item[Coût en PP] 20 \item[Coût en Crédit] Cher (minimum 30 000+¢) \end{description*} 

\subsubsection{Reapeur} \label{sec:starting-reaper} 

Le reapeur est un bot de combat courant, utilisé à la place des soldats biomorphs et typiquement dirigés par téléopération ou par des IA autonomes. Le cœur du reapeur est un disque renforcé, il peut donc tourner et présenter un profil fin à l'ennemi. Il utilise des buses à poussée vectorielle pour manœuvrer en micro-gravité, et prend également avantages d'un moteur ionique pour le déplacement rapide sur de longue distance. Quatre jambes/bras manipulateurs et quatre montures d'armes sont repliées dans sa structure. La coque du reapeur est faites de matériaux intelligents, permettants à ses membres et à ses montures d'armes de sortir dans n'importe quelle direction et même de changer de forme et de taille. Dans les environnemsnt soumis à la gravité, les reapeurs marchent ou sautillent sur deux de leurs membres. Les reapeurs sont ignobles en raison de nombreuses XP de guerre, et en amener un dans un habitat fera indubitablement froncer les sourcils ou vous faire arréter. 

\begin{description*} \item[Implants] Prise d'Accès, Inserts Mesh Basiques, Pile Corticale, Cybercerveau, Vision à 360-Degrés, Anti Éblouissement, Cyber Griffes, Membres Supplémentaires (4), Armure de Combat Lourde, Système Magnétique, Membres Pneumatiques, Marionnette, Radar, Accélérateur de Réflexes, Forme Ajustable, Améliorations Structurelles, Emitter T-ray, Montures d'Armes (Articulées, 4) \item[Mode de déplacement](Allure de déplacement) Marcheur (4/20), Sautilleur (4/20), Ionique (12/40), Poussée Vectorielle (4/20)\item[Maximum d'Aptitude] 40 \item[Modificateur de Vitesse] +1 (Accélérateur de Réflexes) \item [Solidité] 50 (60 avec l'Amélioration Structurelle) \item[Seuil de Blessure] 10 (12 avec l'Amélioration Structurelle) \item[Avantages] 4 Membres, Armure (16/16), +5 en COO, +10 en REF (+20 avec l'1ccélérateur de Réflexes), +10 en SOM \item[Coût en PP] 100 \item[Coût en Crédit] Cher (minimum 50 000+¢) \end{description*} 

\subsubsection{Slitheroïdes} \label{sec:starting-slitheroids} 

Les bots slithéroïdes sont des coques synthétiques prenant la forme d'un serpent métallique et ségmenté de deux mètres de long et possédant deux bras rétractiles pour la manipulation d'outil. Les bots serpents peuvent se lover, se tordre et rouler leur corps en une balle ou onduler, se déplaçant soit en ondulant, en roulant ou en rampant et en s'aidant de leurs bras. Les systèmes sensoriels et l'ordinateur de contrôle sont hébergé dans la tête. 

\begin{description*} \item[Implants] Prise d'Accès, Inserts Mesh Basiques, Pile Corticale, Cybercerveau, Vision Améliorée, Augmentation Mnémonique\item[Mode de déplacement](Allure de déplacement) Serpent (4/16; 8/32 en roulant)\item[Maximum d'Aptitude] 30 \item[Solidité] 45 \item[Seuil de Blessure] 9 \item[Avantages] +5 en COO, +5 en SOM, +5 à une aptitude au choix du joueur, Armure (8/8) \item[Coût en PP] 40 \item[Coût en Crédit] Cher \end{description*} 

\subsubsection{Swarmanoïde} \label{sec:starting-swarmanoid} 

Le swarmanoïde n'est pas une simple coque en soi, masi plutôt une nuiée de centaine de microdrone robotiques de la taille d'un insecte. Chaque "insecte" est capable de ramper, de rouler, de sauter sur plusieurs mètres ou d'utilsier des pales de nanocoptères pour un déplacment aérien. L'ordinateur de contrôlle et le système sensoriel sont distribués sur toute la nuée. Même si la nuée peut se "fusionner" en une grossière forme de la taille d'un enfant, la nuée est incapable d'accomplir des tâches physiques en tant qu'unité telles que attraper, tirer ou tenir. Chaque insecte est relativement capable de s'interfacer avec des systèmes électroniques. 

\begin{description*} \item[Implants] Prise d'Accès, Inserts Mesh Basiques, Pile Corticale, Cybercerveau, Augmentation Mnémonique, Essaim\item[Mode de déplacement](Allure de déplacement) Marcheur (2/8), Sauteur (4/20), Rotor (4/32) \item[Maximum d'Aptitude] 30 \item[Solidité] 30 \item[Seuil de Blessure] 6 \item[Avantages] voir Nuée (p. 311) \item[Désavantages] voir Nuée (p. 311) \item[Coût en PP] 25 \item[Coût en Crédit] Cher \end{description*} 

\subsection{Infomorphs} \label{sec:starting-infomorphs} 

Les infomorphs n'ont qu'une forme numérique - elles ne possèdent pas de corps physique. Les informophs sont parfois portée par d'autres personnage à la place de (ou en addition de) une muse dans un module ghostrider (p. 307). Les règles complètes concernant les infomorphs peuvent être trouvées à la p. 264. 

\begin{description*} \item[Implants] Améliorations Mnémonique\item[Maximum d'Aptitude] 40 \item[Modificateur de Vitesse] +2 \item [Désavantages] Pas de forme physique \item[Coût en PP] 0 \item[Coût en Crédit] 0 \end{description*} 

\section{Traits} les traits listés sont ds traits d'egos, sauf mention contraire. 

\section{Traits Positifs} \label{sec:positive-traits} Les traits positifs fournissent des bonus au personnage dans certaines situations. 

\subsection{Adaptabilité} \label{sec:traits-adaptability} 

\textbf{Coût:} 10 (Niveau 1) ou 20 (Niveau2) PP 

Se réincarner est est un jeu d'enfant pour se personnage. Ils s'adaptent aux nouvelles morphs plus rapidement que la plupart des autres personnes. Appliquez un modificateur de +10 par niveau aux test d'Intégration et aux Tests d'Aliénation (p. 272). 

\subsection{Alliés} \label{sec:traits-allies} 

\textbf{Coût:} 30 PP 

Le personnage fait partie ou est en relation avec un groupe d'influence qu'ils peuvent contacter occasionnellement pour obtenir de l'aide. Cela pourrait être leur vieille équipe de resquilleurs, d'ancien collègue d'un laboratoire de recherche, un cartel criminel dont ils font partie ou une clique sociéle élitiste par exemple. Le maître de jeu et le joueur doivent définir quelles sont les relations du personnage avce ce groupe, ainsi que les raisons qui font que ce personnage peu demander leur aide. Les maîtres de jeu devraient faire attention que les eprsonnages n'abusent pas de ce trait, en appellant les alliés plus d'une fois par session de jeu par exemple. Les attaches de ce personnage à ce groupe est également un lien à double-sene - leurs alliés attendent d'eux qu'ils effectuent certains tâches pour eux (représentant une base de scénario potentiel). 

\subsection{Ambidextre} \label{sec:traits-ambidextrous} 

\textbf{Coût:} 10 PP 

Le personnage peut utiliser et manipuler des objets aussi bien des deux mains (ils ne subissent pas le modificateur de mauvaise main, tel que noté à la p. 193). Si le personnage possèdent d'autres membres préhenseurs (pied, queue, tentacules, etc) ce trait peut-être appliqué à un membre autre qu'une main. Ce trait peut-être pris de multiples fois pour de multiples membres. 

\subsection{Empathie Animale} \label{sec:traits-animal-empathy} 

\textbf{Coût:} 5 PP 

Le personnage a une perception intuitive de la méthode nécessaire pour interagir et travailler avec des animaux non-sapiens de tout type. Appliquer un modificateur de +10 à la compétence Dressage ou lorsque le personnage fait un test pour influencer ou intergair avec un animal. 

\subsection{Courageux} \label{sec:traits-brave} 

\textbf{Coût:} 10 PP 

Ce personnage n'est pas effrayable facilement, et il fera face aux menaces, à l'intimidation et à un risque de blessure certains sans flancher. En conséquence, le personnage n'est pas toujours le meilleur pour évaluer les risques, particulièrement lorsqu'il s'agît de mettre les autres en situations dangereuses. Le personnage reçoit un modificateur de +10 sur tout les tests pour résister à la peur ou à l'intimidation. 

\subsection{Sens Commun} \label{sec:traits-common-sense} 

\textbf{Coût:} 10 PP 

Le personnage a un sens du jugement inné qui passe au travers des distractiosn et des facteurs qui pourraient obscurcir une décision. Une fois par sessiond e jeu, le joueur peut demaner au maître de jeu quel choix il devrait faire ou quel suite d'action il devrait faire, et le maître de jeu devrait lui donner de bosn conseils basé sur les connaissances du personnage. Alternativement, si le personnage est sur le point de rpendre une décsiion désastreuse, le maître de jeu peut utiliser le conseil gratuit du personnage et aveertir le joueur qu'il est en train de commettre une erreur. 

\subsection{Sens du Danger} \label{sec:traits-danger-sense} 

\textbf{Coût:} 10 PP 

Le personnage a un sixième sens intuitif qui le prévient des menaces imminentes. Ils reçoivent un modificateur de +10 sur les Tests de Surprise (p. 204). 

\subsection{Sens Directionnel} \label{sec:traits-direction-sense} 

\textbf{Coût:} 5 PP 

Du'ne manière ou d'une autre, le personnage sait toujours où est le haut, le nord et autre, même si il est aveuglé. Le personnage reçoit un modificateur de +10 pour comprendre des directiosn complexe, lire une carte et se rappeller ou retracer un chemin qu'ils ont déjà fait. 





\subsubsection{Mamoire Éidétique (Trait d'Ego Ou de Morph)} \label{sec:traits-eidetic-memory} \textbf{Coût:} 10 PP 

De manière similaire à un ordinateur, le personnage à un rappel mémoriel parfait. Ils peuvent se rappeler de tout ce qu'ils ont perçus, souvent d'un simple coup d'œil. Ce trait fonctionne de la même manière que l'implant mémoire éidétique (p. 301). 

\subsection{Aptitude Exceptionnelle} \label{sec:traits-exceptional-aptitude} 

\textbf{Coût:} 20 PP 

Le personnage peut augmenter le maximum d'une de ses aptitude de 10 points au-delà de la limite d'aptitude normale (30 pour les plates, 35 pour les spliceurs et 40 pour les autres). Ce trait ne fait qu'augmenter le maximum, il ne donne pas 10 points d'aptitudes supplémenatire au personnage. Ce trait ne peut-être pris qu'une seule fois. 

\subsection{Expert} \label{sec:traits-expert} 

\textbf{Coût:} 10 PP 

Le personnage est une légende dans l'utilisation de l'une de ses compétences. le personnage peut augmenter uen compétence apprise au-delà de 80, jusqu'à un maximum de 90, pendant la création de personnage. Ce trait n'uagmente pas la compétence, il augmente juste le maximum dans cette compétence. Ce trait ne peut-être pris qu'une seule fois. 

\subsection{Apprenant Rapide} \label{sec:traits-fast-learnier} 

\textbf{Coût:} 10 PP 

le personnage améliore ses compétences et en apprend de nouvelels en moitié moins de temps que ce qui est normalement nécessaire (voir Augmenter des Compétences, p. 152). 

\subsection{Première Impression} \label{sec:traits-first-impression} 

\textbf{Coût:} 10 PP 

Le personnage a une façon de charmer ou de faire bonne impression la première fois qu'ils interagissetn avec quelqu'un. Ce lubrifiant social inné leur permet de traiter plsu facilement avec les nouveaux contacts et de s'intégrer rapidement à de nouveaux environneemnts sociaux. Appliquez un modificateur de +10 sur les tests de compétences socials lorsque le personnage interagît avec un autre personnage pour la première fois (et pour la première fois seulement). 

\subsection{Hyper Linguiste} \label{sec:traits-hyper-linguist} 

\textbf{Coût:} 10 PP 

Le personnage a une compréhension instinctive des structures linguistiques qui facilitent l'apprentissage de nouvelles langues. Le personnage n'a besoin que d'un tiers du montant total de temps et d'expérience pour apprendre de nouvelles langues (voir Améliorer des Compétences, p. 152). Le personnage peut également apprendre toute langue humaine en une journée en s'y immergeant constamment. Le personnage reçoit également un modificateur de +10 lorsqu'il tente d'interpréter des langues qu'il ne connaît pas. 

\subsection{Système Immunitaire Amélioré (Trait de Morph)} \label{sec:traits-improved-immune-system} 

\textbf{Coût:} 10 (Niveau 1) ou 20 (Niveau2) PP 

Le système immunitaire de la morph est robuste et résiste mieux aux maladies, drogues et toxines - mieux que les biomods de bases. Au niveau 1, appliquez un modificateur de +10 lrosque vous faites un test pour résister à l'infection ou aux effets d'une toxine ou d'une drogue. Au niveau 2, augmentez ce modificateur à +20. Ce trait n'est disponible que pour les biomorphs. 

\subsubsection{Inofensif (Trait de Morph)} \label{sec:traits-innocuous} 

\textbf{Coût:} 10 PP 

À une époque où la beauté et les apparences exotiques sont des lieux communs, l'apparence de cette morph est étonnement fade et sans distinction, de cette manière réconfortante qu'on les choses produite à l'identique. L'apparence physique du personnage est tellement banalle que les autres ont du mal à le distinguer dans une foule, à décrire son apparence ou à se rappeler de détails physique. Appliquer un modificateur de -10 à tous les tests faits poru trouver, décrire ou se rappeller du personnage. Ce modificateur ne s'applique pas au psi ou recherche sur le mesh. 

\subsection{Souple (Trait de Morph)} \label{sec:traits-limber} 

\textbf{Coût:} 10 (Niveau 1) ou 20 (Niveau2) PP 

La morph est particulièrement souple et flexible, capable de controsions gracieuse et de prendre des positions intéressantes. Au niveau 1, le personnage peut fumer avec ses orteils, faire le pont et se compresser dans de petits espaces étroits. Au niveau 2, ce sont des artistes de l'évasion désarticulés. Chaque niveau fournit un modificateur de +10 pour s'échapper de liens, s'adapter à des espaces confinés et d'autres actions dépendant de la contorsion ou de la flexibilité. Ce trait n'est disponible que pour les biomorphs. 

\subsection{Génie des Maths} \label{sec:traits-mathwiz} \textbf{Coût:} 10 CP 

Le personnage peut effectuer tout type de calcul, incluant les plus complexe et les mathématiques avancées, instantanément et avec une grande précision, avec la même facilité que d'additioner 2 et 3 pour un humain non-modifié. Le personnage peut calculer des probabilités avec une grande précision, trouver des corrélations dans des données numériques et accomplir des tâches similaires avec une facilité déconcertante. Appliquez un modificateur de +30 sur les tests impliquant des calculs mathématiques. 

\subsection{Immunité Naturelle (Trait de Morph)} \label{sec:traits-natural-immunity} 

\textbf{Coût:} 10 PP 

La morph possède une immunité naturelle à une drogue, une maladie ou une toxine spécifique. Lorsqu'il est affecté par cette substance chimqiue, ce poison ou ce pathogène, le personnage reste non affecté. A la discrétion du maïtre de jeu, cette immunité pourrait ne pas s'appliquer à certains agents. Elle ne peut pas concerner les nanodrogues ou les nanotoxines. Ce trait n'est disponible que pour les biomorphs. 

\subsection{Tolérance à la Douleur (Trait d'Égo ou de Morph)} \label{sec:traits-pain-tolerance} 

\textbf{Coût:} 10 (Niveau 1) ou 20 (Niveau2) PP 

Le personnage possède un seuil de tolérance à la douleur élevé et peut plus facilement ignorer les effets de la douleur sur ces capacités et sa concentration. Le niveau 1 lui permet d'ignorer d'ignorer le modificateur de -10 d'1 blessure. Le niveau 1 lui permet d'ignorer d'ignorer le modificateur de -10 de 2 blessures. Ce trait n'est disponible que pour les biomorphs. 

\subsection{Mentor} \label{sec:traits-patron} 

\textbf{Coût:} 30 PP 

Le personnage connaît une personne influente sur laquelle il eput compter pour de l'aide occasionnelle. Cela pourraît être un membre d'une famille hyperléite prospère, un chef des triades bien placés ou un networker anarchistes avec une réputation imbattable. Lorsqu'il est appelé, le mentor peut tirer quelques ficelles au bénéfice du personnage, lui fournissant des ressource, le présentant aux personnes qu'ils ont besoin de connaître et les tirer d'affaire. Le jouer et le maître de jeu doivent travailler ensemble pour définir exactement qui est ce PNJ et quelle est sa relation avec le personnage joueur. La question spécifique du pourquoi ce mentor soutient le personnage, doit avoir une réponse (obligation familliale? amis d'enfance? le personange lui a sauve la vie une fois?). Le maître de jeu devrait faire attention à ce que ce trait ne soit pas utilisé de manière abusive. Le mentor devrait être une aide occasionnelle (probablement pas plus d'une fois par session de jeu au plus) mais il n'est pas toujours dispo à la demande du personnage. Si le personnage demande trop ettrop souvent, le soutien du mentor devrait s'assécher. Additionellement, les personnages peuvent avoir besoin d'accomplir certaines actions pour préserver cette relation, telles que s'occuper d'une mission pour le compte du mentor. En fait, le personnage pourrait n'avoir leur soutien que parcequ'ils sont disponible à la demande du PNJ d'une certaine manière. 

\subsection{Psi} \label{sec:traits-psi} \textbf{Coût:} 20 PP (Niveau 1), 25 PP (Niveau 2) 

Le personnaga a été infecté avec la souche Watts Mac Leod du virus Exsurgent, virus qui a altéré leur structure cérébrale et à ouvert le potentiel de leur esprit afin d'augmenter leur capacités cognitive ainsi que al capacité à lire et manipuler les esprits biologiques des autres (voir Psi, p. 220). Le personnage peut acheter et apprendre des exploits psis (p. 223). Au niveau 1, le personnage ne peut utiliser que les exploits psi-chi. Au niveau 2, le personnage peu utiliser les exploits psi-chi et psi-gamma. 

Bien que ce trait ne soit pas trés cher, les maître de jeu ne devrait pas permettre q'il soit trop utilisé. Il y a un nombre d'effet secondaire négatifs à l'infection par la souche Watts MacLeod, noté au pragraphe Inconvénients du Psi (p. 220. 

\subsection{Caméléon Psi (Trait d'Ego ou de Morph)} \label{sec:traits-psi-chameleon} 

\textbf{Coût:} 10 PP 

L'état mental du personnage est natruellement resistant aux exploits psi. Appliquez un modificateur de -10 à toute tentative pour localiser ou détecter le personange en utilisant des exploits psi. 

\subsection{Défense Psi (Trai d'Égo Ou de Morph)} \label{sec:traits-psi-defense} 

\textbf{Coût:} 10 (Niveau 1) ou 20 (Niveau2) PP 

L'esprit du personnage est intrinséquement résistant aux attaques mentales. Au niveau 1, appliquez un modificateur de +10 à tous les tests de défense fait contre les attaques psi. Au niveau 2, augmentez ce modificateur à +20. 

\subsection{Guérison Rapide (Trait de Morph)} \label{sec:traits-rapid-healer} 

\textbf{Coût:} 10 PP 

La morph récupère plus rapidement des dommages subit. Réduisez l'intervalle de temps pour les soins de motié, tels que noté sur la table de Soins p. 208. Ce trait n'est disponible que pour les biomorphs. 

\subsection{Bien Dans Ses Bottes} \label{sec:traits-right-at-home} 

\textbf{Coût:} 10 PP 

Le personnage choisit un tympe de morph (splicerus, néo-hominidés, boîte, etc). Le personange se sent toujours bien dans les morphs de ce type. Lorsqu'il se réincarne dans ce type de morph, le personnage s'ajuste automatiquement à son nouveau corps, aucune test d'Intégration ou d'Aliénation ne sont nécessaires, ne souffrant d'aucune pénalité et d'aucun stress mental. 

\subsection{Seconde Peau} \label{sec:traits-secondskin} 

\textbf{Coût:} 15 PP 

Si le passé la faction de votre personnage vous restreint dans votre morph de départ (par exemple, les élevés doivent démarer dans une morph élevée), ce trait vous permet d'ignorer ces restrictions et d'acheter la morph de départ de votre choix. 





Ce eprsonnage est extrêmement doué pour conserver une conscience continue des choses qui se passent dans leur environnement immédiat. En terme de jeu, ils ne souffrent pas du modificateur Distrait sur les Tests de Perception pour remarquer quelque chose même si leur attention est concentrée ailleurs, ou lrosqu'il font des Tests de Perception Rapide pendat les combats. 

\subsection{Beauté Ahurissante (Trait de Morph)} \label{sec:traits-striking-looks} 

\textbf{Coût:} 10 (Niveau 1) ou 20 (Niveau2) PP 

A une époque ou la biosclupture est facile, être beau est à la fois bon marché et courant. Cette morph possède cependant une apparence physique qui ne peut être décrite que comme ahurissante et inhabituelle, mais également comme quelque peu attirante et fascinante - même les célébréités splendides et ciselés y font attention. Sur les tests de compétences sociale où la beauté du personnage peut influer sur le résultat, ils reçoivent un modificateur de +10 (au niveau 1) ou de +20 (au niveau 2). Ce modificateur est inefficace sur les xénomorphs ou ceux qui ont un historique infolife ou élevé. Ce trait n'est disponible que pour les biomorphs. 

Ce modificateur peut être acheté pour les morphs élevées, mais à moitié prix, et il n'est efficace que contre les personnages ayant le même type de morph élevé (càd néo-aviens, néo-homnide, etc). 

L'inconvénient de ce trait est que l'on se souveint plus facilement du personnage et il est plus facilement remarqué. 

\subsection{Solide (Trait de Morph)} \label{sec:traits-tough} 

\textbf{Coût:} 10 (Niveau 1), 20 (Niveau 2) ou 30 (Niveau 3) PP 

Est morph est plus résistante que les autres de même type et peut subir plsu d'abus physique. Augmentez sa Solidité par +5 par niveau (+5 au Niveau 1, +10 au Niveau 2 et +15 au Niveau 3). Cela augmente également le Seuil de Blessure de +1, +2 et +3 respectivement. 

\subsection{Zoosémiotique} \label{sec:traits-zoosemiotics} 

\textbf{Coût:} 5 PP 

Un personnage avec ce trait et le trait Psi ne souffre pas d'un modificateur lorsqu'il utilise des exploits psi contre des espéèces animales non-consciente ou partiellement-conscientes. 

\section{Traits Négatifs} \label{sec:negative-traits} 

Les traits négatifs ont tendance à handicaper le joueur et à applqieur des modificateurs négatifs dans certaines circonstances. 

\subsection{Addiction (Trait d'Ego OU de Morph)} \label{sec:traits-addiction} 

\textbf{Bonus:} 5 PP (Mineure), 10 PP (Modérée), ou 20 PP (Majeure) 

\textbf{} Les addictions existent en deux formes: mentale (affectant l'ego) et physique (affetant la biomorph). Le personnage ou la morph est accroc à une drogue (p. 317), un stimulus (XP) ou une activité (utilisation du mesh) a un degré qui impacte la snaté physique ou mentale du personnage. Les joueurs et le maître de jeu devraient déterminer ensemble une addictions appropriées à leur parties. L'addiction existe en trois niveaux de sévérité: mineure, modérée et majeure: 

\textbf{Mineure:} Une addiction mineure est largement gardée sous contrôle - elle ne ruine pas la vie du personnage, bien qu'elle puisse créer certaines difficultés. Le personnage pourrait même ne pas reconnaître ou admettre qu'il a un problème. Le personnage doit se soumettre à son addiction au moins une fois par semaine, bien qu'ils puisssent supporter de plus longue période sans difficultés. Si ils ne perviennet pas à obtenir leur dose hebdomadaire, ils souffrent d'u modificateur de -10 à toutes leurs actiosn jusqu'à ce qu'ils puissetn avoir leur dose. 

\textbf{Modérée:} Une addiction modérée est en plein essor. Le personnage a manifestement un problème, et il doit satisfaire son addiction au moins une fois par heure. Si il ne peut le faire, il peut souffrir d saute d'humeur, de comportement compulsif, de malaide physique ou d'autres effets secondaires jusq'uà ce qu'il puisse combler leur manque. Appliquez un modificateur de -20 à toutes les actions du personnage jusqu'à ce qu'il obtienne sa dose. Additionellement, un personnage à ce niveau d'addiction souffre d'une pénalité de -5 en SOL. 

\textbf{Majeure:} Un personnage avec uen addiction majeure est sur une pente savonneuse vers la ruine. Ils font face au manque toutes les 6 heures, et souffrent d'une pénalité de -10 en SOL car leur santé est affectée. Si ils ne parviennent pas à obtenir leur dose régulièrement, ils souffrent d'un modificateur de -30 à toutes leurs actions jusqu'à ce qu'ils trouvent une dose. Si leur vie n'as pas déjà été ruinée par leurs obsessions, elle le sera rapidement. 

\subsection{Vieux (Trait de Morph)} \textbf{Bonus:} 10 PP 

La morph est agée physiquement sans avoir bénéficié de réjuvénation. Les vieiles morphs sont incroyablement rare, bein que certaines personens les adoptent en espérant gagner de l'ancienneté et de la respectabilité. Réduisez le maximum d'aptitude du personnage de 5, et appliquez un modificateur de -10 à toutes les actions physiques. 

Ce trait ne peux être appliqué qu'uax morphs plates et spliceurs. 

\subsection{Malchanceux} \label{sec:traits-bad-luck} 

\textbf{Bonus:} 30 PP 

En raison d'une coïncidence cosmique inexplicable, les choses semblent mal se passer autur du personnage. Le maître de jeu reçoit une réserve de points de Moxie égale à la stat Moxie du personnage, qui se reconstitue au même rythme que le Moxie du personnage. Seul le maître de jeu peu utiliser ce Moxie et et il doit le faire uniquement poru agir contre le personnage. En d'autres mots, le maître de jeu peut utilsier ce Mauvais Moxie pour causer un échec automatique au personnage, lui faire un permutter un jet et ainsi de suite. Pour refléter ce mauvais augure qui plane autour du personnage, le maître de jeu peu mrme utiliser ce Mauvais Moxie contre les amis et les alliés du personnage, lorsqu'ils font quelque chose avec ou en lien avec le personnage, bien que cela devrait être utilisé avec parcimonie. Les maître de jeu qui pourraient rechigner à saboter le personnage devraient se souvenri que ce joueur l'a demandé en prenant ce trait. 





Le personnage a réussit à se faire mettre sur liste noire dans certains cercles, qu'isl aient réellement fait quelque chose pour le mériter ou pas. En terme de jeu, le personnage ne pourra jamais avoir un score de Rep supérieur à 0 dans un réseau de réputation particulier. Les personnes faisant parti de ce réseau refuseront d'aider le personnage apr peur de représaille et de ruiner leur propre réputation. Le bonus pour ce trait est de 20 PP si il est choisi pour le réseau réputationnel de la factoin d départ du personnage, et de 5 PP sinon. 

\subsection{Marque Noire} \label{sec:traits-black-mark} 

\textbf{Bonus:} 10 (Niveau 1), 20 (Niveau 2) ou 30 (Niveau 3) PP 

À un certain moment dans le passé du personnage, il s'est débrouillé pour faire quelque chose qui lui a valu de recevoir une marque noire sur sa réputation. Pour diverses raisons, peu importe ce qu'il fait, cette marque noire ne peut être retirée et continue de hanter ses interactions. En terme de jeu, le personnage choisit une faction. À chaque fois qu'il interagît avec cette faction (avec un Test de Réseau par exemple) ou avec un PNJ de cette faction (test de Compétence Sociale) qui connaît le personnage, il subit un modificateur de -10 par niveau. 

\subsection{Paralysise en Combat} \label{sec:traits-combat-paralysis} 

\textbf{Bonus:} 20 PP 

Le personnage a la malheureuse habite de se figer dans les situations de combat ou stressante, comme un faon prit dans les phares d'une voiture. À chaque fois que la violence explose autour du personnage, ou si il est surpris, le personnage doit faire un tst de Volonté afin d'agir ou de répondre de quelque manière que ce soit. Si il rate ce test, il perd ses action et reste simplement planté là, incapable de réagir à la situation. 

\subsection{Souvenirs Édités} \label{sec:traits-edited-memories} 

\textbf{Bonus:} 10 PP 

À un certain moment dans le passé du personnage, certains de ses souvenirs ont étés stratégiquement supprimés ou ont été perdus d'une manière ou d'une autre. Cela peut avoir été fait intentionnellement pour oublier une expérience déplaisante ou honteuse ou alors pour rompre avec le passé. Les souvenirs ont également pu êþre perdus lors d'une mort inattendue (et sans sauvegarde récente), ou il ont pu être effacés contre la volonté du personnage. Quelque soit le cas, les souvenirs perdus devraient revétir une importance quelconque, et il devrait exister soit des preuves de ce qu'il s'est passé, soit des PNJ qui connaissent toute l'histoire. C'est un outil que le maître de jeu peu utiliser pour hanter le personnage à un moment futur avec les fantômes de son passé. 

\subsection{Ennemi} \label{sec:traits-enemy} 

\textbf{Bonus:} 10 PP 

À un certain moment dans le passé du personnage, il s'sst fait un ennemi à vie qui continue à le harceler. Le maître de jeu et le joueur devrait déterminer les détails de ce t innimitié, et le maître de jeu devrait utilsier cet ennemi en tant que menace, surprise et gène occasionnelle. 

\subsection{Faible} \label{sec:traits-feeble} 

\textbf{Bonus:} 20 PP 

le personnage est particulièrement faible sur l'une de ses aptitudes. Cette aptitude doit être achetées à un niveau inférieur à 5, et ne pourra jamais être améliorée lors de l'avancement du personnage. Le maximum de cette aptitude est de 10, quelle que soit la morph que porte le personnage. 

\subsection{Frêle (Trait de Morph)} \label{sec:traits-frail} 

\textbf{Bonus:} 10 (Niveau 1) ou 20 (Niveau 2) PP 

Cette morph n'est pas aussi solide que les autres morphs de son type. Sa Solidité est réduite de 5 par niveau. Cela réduit également le Seuil de Blessure de 1 ou 2 respectivement. 

\subsection{Défaut Génétique (Trait de Morph)} \label{sec:traits-genetic-defect} 

\textbf{Bonus:} 10 PP ou 20 PP 

Cette morph n'est pas génétiquement réparée, et souffre en fait d'un trouble génétique ou d'une mutation handicapante. Le joueur et le maître de jeu doivent se mettre d'accord sur un défaut approprié à leur jeu. Certaines possibilités incluent: maladie cardiaque, diabète, fibrose kystique, drépanocytose, hypertension, hémophilie ou daltonisme. Un trouble génétique qui crée des complications mineures et/ou des problèmes de santé occasionnelle devrait valoir 10 PP, un défaut qui handicape significativement le fonctionnement normal du personnage ou qui lui inflige des problème de santé chronique devrait valoir 20 PP. Le maître de jeu doit déterminer l'effet exact du trouble sur le jeu. 

Ce trait n'est disponible que pour les plates. 

\subsection{Crise Identitaire} \label{sec:traits-identity-crisis} 

\textbf{Bonus:} 10 PP 

L'ego du personnage a un trouble à s'adapter à l'apparence modifiée d'une nouvelle morph - ils sont coincés avec l'image mentale de leur corps original, et ne s'habituent tout simpelment pas à leur(s) nouveau(x) visage(s). En conséquence, le personnage a des difficultés à se reconnaître dans le mirrori, sur les photos ou les flux de surveilalnces, etc. Ils oublient régulièrement l'apparence de leur morph actuelle, agissant de manière inappropriée, se décrivant par leur corps original, oubliant de se baisser lorsqu'il franchit une porte, etc. Il s'agît essentiellement d'un trait d'interprétation, amis el maître de jeu peut appliquer des modificateurs (habituellement un -10) aux tests affectés pas cette incapacité à s'adapter. 

\subsection{Illétré} \label{sec:traits-illiterate} 

\textbf{Bonus:} 10 PP 

Le personnage sait comment s'exprimer à l'oral, mais à des difficultés pour lire ou pour écrire. En raison de la saturation d'entoptique et la nature iconique de la société transhumaine, ils sont capable de s'en sortir de manière relativement confortable avec ce handicap. Réduisez les compétences de Langue du personnage de moitié (arrondissez à l'inférieur) lorsqu'il s'agît de lire ou d'écrire. 

\subsection{Blues de l'Immortalité} \label{sec:traits-immortality-blues} 

\textbf{Bonus:} 10 PP 

Le personnage a vécu tellement longtemps - plus de 100 ans - qu'il s'ennuie de la vie et a maintenant des difficultés à se motiver. Il était déjà vieux quand les traitements de longévité devinrent disponible, il a survécu à la Chute et a continué à vadrouiller depuis - bien qu'il trouvent de plus en plus difficile de faire attention, de s'intéresser au choses autour d'eux, ou de craindre la mort finale. Le personnage ne reçoit que la moitié des Points de Moxie et de Rez en récompense pour la complétion des objectifs réputationnels. 

Ce trait ne peut être acheté par des personnages avec un historique infolife ou élevé. 

\subsection{Rejets des Implants (Trait de Morph)} \label{sec:traits-implant-rejection} 

\textbf{Bonus:} 5 (Niveau 1) ou 15 (Niveau 2) PP 

Cette morph ne supporte pas très bien les implants. Au niveau 1, tout les implants acquis sont plus chers car ils nécessitent des traitements anti-rejets spéciaux. Augmentez la cartéogorie de prix de l'implant d'un rang. Au niveau 2, la morph ne peut accepter d'implants d'aucune sorte. 

\subsection{Incompétent} \label{sec:traits-incompetent} 

\textbf{Bonus:} 10 PP 

Le personnage est complètement incapable d'accomplir une action en lien avec une compétence active déterminéen, peu importe l'entraînement qu'il pourrait recevoir. Ils ne peuvent pas acheter cette compétence durant la création de personnage ou lors de l'avancement du personnage, et le modificateur pour défausser sur l'aptitude liée de cette compétence particulière est de -10. Ce trait ne peut être choisit pour les compétences d'armes exotiques, et devrait être choisit pour une compétence qui pourrait avoir une utilité pour le personnage. 

\subsection{Raté (Trait de Morph)} \label{sec:traits-lemon} 

\textbf{Bonus:} 10 PP 

Ce trait n'est disponible que pour les morphs synthétiques. Cette morph particulière a un défaut non-réparable. Une fois par session de jeu (de manière préférable, à un moment qui maximisera le drame ou l'hilarité), le maître de jeu peut demander au personnage de fare un Test de MOX x 10 (en utilisant leur valeur de Moxie actuelle). Si le personnage rate, la morph souffre immédiatement d'une blessure réultant d'un problème mécanique, d'un défaut électrique ou d'une autre panne. Cette blessure peut être réparée de manière normale. 

\subsection{fauble Tolérance à la Douleur (Trait d'Égo ou de Morph)} \label{sec:traits-low-pain-tolerance} 

\textbf{Bonus:} 20 PP 

La douleur est l'ennemi du personnage. Le personnage possède un seuil de tolérance à la douleur très bas et est plus facilement handicapé lorsqu'il souffre. Augmentez le modificateur de chaque blessure prise par un modificateur additionel de -10 (le eprsonnage subit donc un -20 avec une blessure, -40 avec une autre et -60 avec une troisième). Le personnage souffre également d'un modificateur de -30 sur tous les test mettant en jeu la résistance à la douleur. La version morph de ce trait n'est disponible que pour les biomorphs. 

\subsection{Trouble Mental} \label{sec:traits-mental-disorder} 

\textbf{Bonus:} 10 PP 

Vous avez un désordre psychologique hérité d'une précédente expérience traumatisante. Choisissez l'un des troubles listés à la p. 211. 

\subsection{Allergie Légère (Trait de Morph)} \label{sec:traits-mild-allergy} 

\textbf{Bonus:} 5 PP 

La morph est allergique a un allergène spécifique (poussière, squames, pollens, certains produits chimiques) et souffrent d'un inconfort léger lorsqu'ils y sont exposés (irritations des yeux, nez qui coule, gène respiratoire). Appliquez un modificateur de -10 à tous les tests tant que le personnage est exposé à l'allergène. Ce trait n'est disponible que pour les biomorphs. 

\subsection{Comportement Modifié} \label{sec:traits-modified-behaviour} 

\textbf{Bonus:} 5 (Niveau 1), 10 (Niveau 2) ou 20 (Niveau 3) PP 

Le personnage a été conditionné par de la psychochirurgie comportementale en temps compressé. C'est quelque chose de commun chez les anciens-criminels, qui ont été conditionnés pour répondre à une idée spécifique ou à une activité avec une horreur véhémente et dégoût, mais cela a pu arriver pour d'autre raison ou avoir été auto-infligé. Au niveau 1, le comportement choisit est soit limité, soit développé, au Niveau 2 il est soit bloqué oit encourage, et au niveau 3 il est effacé ou forcé (voir p. 231 pour les détails). Ce trait ne devrait être autorisé que pour les comportements qui soit sont limités ou soit , si ils sont encourgaés, affectent le personnage de manière négative. 

\subsection{Trouble Morphique} \label{sec:traits-morphing-disorder} 

\textbf{Coût:} 10 (Niveau 1), 20 (Niveau 2) ou 30 (Niveau 3) PP 

S'adapter à de nouvelles morphs est un défi particulier pour ce personnage. Il souffre d'un modificateur de -10 par niveau à tous les Tests d'Intégration et Test d'Aliénation (p. 272). 

\subsection{Dommage Neuraux} \label{sec:traits-neural-damage} 

\textbf{Bonus:} 10 PP 

Le personnage souffre d'un type de dommage neurologique qui ne peuvent simplement pas être soigné. L'affliction fait maintenant parti de l'égo du personnage et persiste même lorsqu'il change de morph. Ces dégats peuvent avoir été hérités, ils peuvent résulter d'un implant ou d'une morph mal conçu, ou ils peuvent avoir été infligés par l'un des nanovirus des TITANs qui ont ciblés le système neuronal pendant la Chute (p. 384). Le maître de jeu et le joueur doievnt s'accorder sur un trouble spécifique approprié au jeu. Quelques possibilités: 

\begin{itemize} \item Aphasie partielle (diificulté à communiquer ou à utiliser des mots) \item Daltonisme \item Amusica (incapacité à faire ou à comprendre la musique) \item Synesthésie \item Logorrhée (usage excessif de mots) \item Perte de la reconnaissance faciale \item Perte de la perception du relief (doublez les modificateurs de portée) \item Comportement répétitif \item Sautes d'humeurs \item Incapacité à basculer rapidement son attention d'une tâche à l'autre. \end{itemize} 

Le maître de jeu peu décider d'infliger des modificateurs résultants de cette affliction. 





La morph ne possède pas la pile corticale qui est commune aux morphs de son type. Cela signifie que le personnage ne peut être réincarné depuis la pile corticale si le personnage meurt, il ne peut être réincarné que depuis une sauvegarde standard. Ce trait n'est pas disponible pour les plates. 

\subsection{Oublieux} \label{sec:traits-oblivious} 

\textbf{Bonus:} 10 PP 

le personnage est particulièrement oublieux des évènements qui l'entoure ou de tout ce sur quoi n'est pas concentrer leur attention. Ils souffrent d'un modificateur de -10 pour les Tests de Surprise et leur modificateur pour être Distrait est de -30 au lieu de l'habituel -20 (voir Perception Basique p. 190). 

\subsection{En Fuite} \label{sec:traits-on-the-run} 

\textbf{Bonus:} 10 PP 

Le personnage est recherché par les autorités d'un habitat/d'une station particulier ou par une faction, qui continue de rechercher activement le personnage. Il a soit commis un crime ou il a déplut d'uen manière ou d'une autre à quelqu'un de puissant. le personnage traite avec cette faction à ses propres risques, et peut occasionnellement être forcé de faire face à des chasseurs de primes. 

\subsection{Vulnérabilité Psi (Trait d'Ego Ou de Morph)} \label{sec:traits-psi-vulnerability} 

\textbf{Bonus:} 10 PP 

Quelque chose dans l'esprit du personnage le rend particulièrement vulnérable aux attaques psi. Ils souffrent d'un modificateur de -10 lorsq'uils doivent résister à de telles attaques. La version morph de ce trait n'est disponible que pour les biomorphs. 

\subsection{Naiveté du Monde Réel} \label{sec:traits-real-world-naivite} 

\textbf{Bonus:} 10 PP 

En raison de son passé, le personnage a une expérience personnelle trés limitée avec le monde réel (physique) - à moins qu'il n'ait passé tellement de temps en simulspace que son évolution dans le monde réel est handicapée. Il manque de la compréhension de beaucoup de propriétés physiques, de codes sociaux, et d'autres facteurs que les gens ayant une éducation humain standard considèrent comme acquis. Ce manque de sens commun peut mener les personnages à ne pas comprendre le fonctionnement d'un appareil ou à mal interpréter le language corporel de quelqu'un. 

Une fois par session de jeu, le maître de jeu peut intentionnellemtn induire en erreur le personnage lorsqu'il lui décris quelque chose ou lors d'une interaction sociale. Ce mensonge représente l'incompréhension de la situation par le personnage, et devrait être interprété de manière appropriée, même si le joueur réalise l'erreur du personnage. 

Ce trait ne devrait être disponible que pour les personnage avec l'historique infolife ou réinstantié, bien que le maître de jeu peu l'autoriser pour les personnages qui ont dans leur histoires personnelles un usage extensif de la réalité virtuelle/de l'XP. 

\subsection{Allergie Grave (Trait de Morph)} \label{sec:traits-severe-allergy} 

\textbf{Bonus:} 10 (rare) ou 20 (commune) PP 

la biochimie de la morph souffre de réaction allergique grave (anaphylaxie) lorsqu'il entre en conatct (touché, inhalé, ingéré) ave un allergène spécifique. L'allergène peut-être commun (pouissières, squames, pollens, certains aliments, latex) ou rare (certaines drogues, les piqures d'insectes). Le joueru et le maître de jeu devrait se mettre d'accord sur un allergène correspondant au jeu. Si il est exposé à l'allergène, le personnage entre en crise, a des difficulté à respirer (modificateur de -30 à toutes les actions), et doit faire un Test de SOL ou subir un choc anaphylactique (mort par arrêt respiratoire en 2d10 minutes à moins que des soins madicaux ne soietn faits). Ce trait n'est disponible que pour les biomorphs. 

\subsection{Apprenant Lent} \label{sec:traits-slow-learner} 

\textbf{Bonus:} 10 PP 

Le personnage a du mal à saiair les nouvelles compétences. Le personnage mets deux fois plus de temps que la normale pour améliorer les compétences ou en apprendre de nouvelles (p. 152). 

\subsection{Stigmatisation Sociale (Trait d'Égo ou de Morph)} \label{sec:traits-social-stigma} 

\textbf{Bonus:} 10 PP 

Un aspect malheureux de l'historique du personnage fait qu'il est stigmatisé dans certaines situtations sociales. Il peut avoir été récincarné dans une morph perçues aevc répugnance, être un survivant de l'infâme génération Égarée ou être une IAG dans une société post-Chute en proie à la peur de l'intelligence artificielle. Dans les situations sociales où la nature du personnage est connue qui perçoit cette néture comme avec dégoût, peur ou répugannce, il souffre d'un modificateur allant de -10 à -30 (à la discrétion deu maître de jeu) aux tests de compétences sociale. 

\subsection{Peureux} \label{sec:traits-timid} 

\textbf{Bonus:} 10 PP 

Ce personnage est facilement effrayé. Il souffre d'un modificateur de -10 lorsqu'il résiste à la peur ou à l'intimidation. 

\subsection{Repoussant (Trait de Morph)} \label{sec:traits-unattractive} 

\textbf{Bonus:} 10 (Niveau 1), 20 (Niveau 2) ou 30 (Niveau 3) PP 

À une époque ou être beau est facilement acquis, cette morrph est visiblement moche. Comme la répulsion est de plus en plus associée à la pauvreté, aux rétrogrades ou à un défuat génétique, les réponses au manque de beauté vont du dégoût à l'horreur. Le personnage reçoit un modificateur de -10 sur tous les tests sociaux au niveau 1, de -20 au niveau 2 et de -30 au niveau 3. 

Seules les biomorphs peuvent prendre ce trait. Ce modificateur est inefficace sur les xénomorphs ou ceux qui ont un historique infolife ou élevé. Ce modificateur peut être acheté pour les morphs élevées, mais à moitié prix, et il n'est efficace que contre les personnages ayant le même type de morph élevé (càd néo-aviens, néo-homnide, etc). 

\subsection{Apparence Étrange (Trait de Morph)} \label{sec:traits-uncanny-valley} 

\textbf{Bonus:} 10 PP 

Il y a un point où l'apparence humaine synthétique est devenue étrngement réaliste et ressemblant aux humains, tyout en restant juste suffisament différente pour que leur apparence semble effraaynte ou repoussante - un phénomène appelé "apparence étrange." Les morphs dont l'apparence tombe dans ces extrêmes subissent un modificateur de -10 sur leurs test de compétence sociale lorsqu'elles traitent avec des humains. Ce modificateur est inefficace sur les xénomorphs ou ceux qui ont un historique infolife ou élevé. 

\subsection{Inadapté (Trait de Morph)} \label{sec:traits-unfit} 

\textbf{Bonus:} 10 PP (Niveau 1), 20 PP (Niveau 2) 

Cette morph est soit non optimisé pour la forme et/ou simplement en mauvais étât. Réduisez le maximum des aptitudes Coordination, Réflexes et Somatique de 5 (niveau 1) ou 10 (niveau 2). 

\subsection{Vertige RV} \label{sec:traits-vr-vertigo} 

\textbf{Bonus:} 10 PP 

Le personnage souffre d'intense vertiges et nde nausée lorsqu'il s'interface à n'importe quel type de rélaité virtuelle, d'XP ou de simulspace. La réalité augmentée n'a aucun effet, mais la RV inflige un modificateur de -30 aux actions du personnage. Une utilisation prolongée de la RV (à la discrétion du maître de jeu) peut réellement incapaciter le personnage si il rate un Test de VOL x 2. 

\subsection{Système Immunitaire Faible (Trait de Morph)} \label{sec:traits-weak-immune-system} 

\textbf{Bonus:} 10 (Niveau 1) ou 20 (Niveau 2) PP 

Le système immunitaire de la morph est affaiblie et résiste moisn bien aux maladies, drogues et toxines. Au niveau 1, appliquez un modificateur de -10 lorsque vous faites un test pour résister à l'infection ou aux effets d'une toxine ou d'une drogue. Au niveau 2, augmentez ce modificateur à -20. Ce trait n'est disponible que pour les biomorphs. 

\subsection{Nausée de Zéro-G (Traot de Morph)} \label{sec:traits-zero-g-nausea} 

\textbf{Bonus:} 10 PP 

Cette morph souffre du mal de l'espace et n'est pas à l'aise dans les environnement en gravité zéro. Le personnage souffre d'un modificateur de -10 en environnement en micro-gravité. lrosque le personnage est acclimaté pour la première fois ou à chaque fois qu'ils doivent encurer des déplacements excessifs en microgravité, le personnage doit faire un Test de VOL ou passer 1 heure incapacité par des nausée par 10 points de MdE. 

\section{Progression du Personnage}: \label{sec:character-advancement} 

Alors que les personnages accomplissent des objectifs et accumulent de l'expérience pendant le jeu, ils accumulent des Points de Rez (voir Accorder les Points de Rez, p. 384). Les Points de Rez peuvent être utilisés pour améliorer les compétences du personnage, ses aptitudes et d'autres caractéristiques selon els règles suivantes. Les coûts de l'avancement en Points de Rez sont les mêmes qu'en Points de Personnalisation. 

\begin{quotation} Dépenser les Points de Rez 

\begin{itemize} \item 15 PR = 1 point de Moxie \item 10 PR = 1 point d'aptitude \item 5 PR = 1 exploit psi \item 5 PR = 1 spécialisation \item 2 PR = 1 point de compétence (61-99) \item 1 PR = 1 point de compétence (jusqu'à 60) \item 1 PR = 1 000 crédit \item 1 PR = 10 rep \end{itemize} \end{quotation} 

\subsection{Changer de Motivation} \label{sec:changing-motivation} 

Il est aprfaitement naturel que les buts et els intérêts d'un personnages évoluent au fil du temps. Le personnage peut avoir atteint un tournant où ils pensent que certains projets ont étés complétés et qu'il est temps de passer à la suite, ou il a raté son coup et doit abandonner l'idée. De nouvelles urgences ou philosophies peuvent entrer dans la vie du personnage, à moins qu'il ai perdu ses illusions vis à vis d'un mème et d'idées particulières qu'ils prenaient à cœur précédemment. 

Changer une motivation d'un personnage ne coûte aucun Points de Rez, mais c'est quelque chose qui ne devrait arriver qu'en accord avec l'interprétation et les évènements vécus. Les joueurs ne devraient pas être autorisés à simplement basculer leur motivations à volonté, il doit y avoir une raion ou une explication motivant ce changement. Pour cette raison, changer de motivation ne devrait arriver que lorsque le joueur et le maîtree de jeu discutent du sujet et soient tous les deux d'accord que l'échange est approprié au développement du personnage et aux circonstances. Si ces conditions sont satisfaites, le personnage abandonne une motivation précédente et s'engage sur la nouvelle. Une seule motivation ne peut être changée à al fois. 

\subsection{Changer de Morph} \label{sec:switching-morphs} 

Se réincarner - basculer d'une morph à une autre - est géré comme une interaction en jeu et n'est pas gérée par des Points de Rez. Voir Réincarnation, p. 271. 

\subsection{Améliorer des Aptitudes} \label{sec:improving-aptitudes} 

Les aptitudes peuvent être améliorée avec des points de Rez au coût de 10 PR par point d'aptitude. Cela représente l'amélioration du personnage dans ses caractéristiques principale, profitant de l'exercice, de l'apprentissage et de l'expérience. Les aptitude ne peuvent être améliorée au-dessus de 30 (les bonus des morphs, des implants, des traits ou d'ailleurs ne comptent pas dans ce total). Augmenter la valeur d'une aptitude augmente également la valeur de toutes les compétences liés du même montant. Si cela augmente une compétence liée au-delà de 60, 1 PR supplémentaire doit être dépassé pour chaque compétence liée au-dessus de 60. 

\subsection{Améliorer des Compétences} \label{sec:improving-skills} 

Les personnages peuvent également dépenser des Points de Rez pour augmenter leurs compétences ou en apprendre de nouvelles. Pour améliorer une compétence existante, le personnage doit avoir utilisé cette compétence avec succès dans un apssé récent ou s'entraîner activement afin de devenir meilleur, prut-être avec l'assistance d'un instructeur. Dans le cas des compétences de Connaissances, cela implique de l'étude active. Comme intervalle de temps grossier, cela dervait nécessiter 1 semaine d'apprentissage par point de compétence. Beaucoup de ressources éducatives sont librement disponible via le mesh, bien que certains domaines pourrait être restreint ou dur à trouver. Cela peut-être géré par l'interprétation ou désigné comme quelque chose sur lequel travail le personnage entre deux sessions de jeu. Si le maîþre de jeu décide qu'un personnage ne s'est pas suffisament investi dans l'amélioration d'une compétence, il peut demander plus d'entraînement/d'étude. Le coût d'amélioration d'une compétence est de 1 PR par point de compétence, et aucune ne peut être améliorée au -delà de 99. Aucune compétence ne peut être améliorée de plus de 5 points par mois. Lorsque la compétence d'un personnage atteint le niveau d'expertise (à 60+), le personnage a cependant tendance à atteindre un palier dans la vitesse de progression et où même la pratique régulière et les études ont des gains réduits. Dans ce cas, le coût en Rez Point par point de compétence est doublé (passant de 2PR pour 1 point de compétence). Arrivé à 80, l'amélioration d'une compétence ralentit encore plus - une compétence au niveau 80+ ne peut être améliorée par plus d'un point par mois. 

\subsection{Apprendre de Nouvelels Compétences} \label{sec:learning-new-skills} 

De manière similaire, pour apprendre une nouvelle compétence, le personnage doit s'entraîner/étudeier activement et/ou chercher un instructeur. Aucun test d'apprentissage n'est nécessaire, sauf si l'apprentissage a été entravé ou a été déficient pour une raion ou pour une autre, auquel cas le maître de jeu peu demander un Test de COG x 3 pour apprendre la nouvelel compétence. Sinon, une fois que le personnage a apssé approximativement une semaine à apprendre une nouvelle compétence, il peut acheter son premier point au coût habituel (1 PR). La compétence est acheté au niveau du rang d'aptitude, comme d'habitude. Une fois qu'une nouvelle compétence est acquise, elle est augmentée selon les règles standard ci-dessus. 

\subsection{Spécialisations} \label{sec:new-specializations} Des spécialisations peuvent être achetées pour les compétences existantes, tant que leur niveau est au moins à 30. Les spécialisations nécessitent un moins d'entraïnement. Le coût d'apprentissage d'une spécialisation est de 5 PR. Une compétence ne peut avoir que 1 spécialisation. 

\subsection{Augmenter le Moxie} \label{sec:improving-moxie} 

le Moxie peut-être augmenté au coût de 15 PR par point de Moxie. Le niveau maximum auquel le Moxie peut-être élevé est de 10. 

\subsection{Gagner/Perdre des Traits} \label{sec:gaining-losing-traits} 

À la discrétion du maître de jeu, des traits positifs et négatifs peuvent être acquis ou perdus pendant le jeu, bien que de tels changements devraient être rare et accomplis uniquement en accord avec la narration et les évènements en cours lors de la aprtie. À la fois des traits positifs et négatifs peuvent être attribués à un personnage pendant le jeu comme conséquence de ce qu'ils ont fait ou de ce qu'il leur est arrivé. Dans le cas de traits positifs, le personnage doit immédiatement dépenser un nombre de Point de Rez équivalent au coût en PP du trait (qu'ils aient voulus ou non ce nouveau trait). Si le personnage n'as pas de PR disponible, il doit le payer immédiatement d tout PR futur qu'il pourrait acquérir jusqu'à ce que la dette soit payée. Dans le cas d'un trait négatif cependant, le personnage récupère juste le nouveau défaut - ils n'acquièrent pas de PR supplémentaire pour avoir hrité de ce trait négatif. Se débarasser de trait est quelque peu plus délicat. Les traits positifs peuvent être perdus suite à un effet malheureux que subit le personnage, comme le maître de jeu le souhaite. De tels traits positifs perdus sont simplement partis - le personnage ne reçoit aucun Point de Rez en remboursement. Les traits Négatifs sont éliminés occasionnellement de la même manière, mais en général ils ne peuvent être annulés que par un travail difficile et la volonté d'un personnage à chercher à se débarasser de son handicap. De tels comportements devraient nécessiter des semaines voire des mois d'effort de la part du personnage, avec une interpétation appropriée et probablement quelques tests difficile. En fait, dépasser de tels traits pourrait être à l'origine d'aventures complètes. Une fois que le maître de jeu pense que le personnage a accompli un effort suffisament important, le personnage doit payer un nombre de Point de Rez égal au bonus en PP du trait pour l'annuler. Notez cependant que certains traits négatifs ne peuvent pas être annulés, peu importe ce que fait le personnage. 

\subsection{Augmenter la Rep} \label{sec:improving-rep} 

La réputation est quelque chose qui peu être augmentée par une interprétation appropriée et des actiosn pendant le jeu (voir Gain et Perte de Réputation, p. 384). Les personnages qui préfèrent gérer leur activités d'augmentation de Rep "hors champ" peuvent cependant dépenser des Points rez pour augmenter leur score(s). Chaque point de PR augmente la Rep du personnage de +10 dans un seul réseau. Un seul boost de ce type peut être effectué pour un seul réseau de rep par mois. 

\subsection{Faire des Crédits} \label{sec:making-credit} 

Les Points de Rez peuvent être convertis en Crédit au taux de 1 PR pour 1000 ¢. Cela représente les revenus que le eprsonnage gagne "horsh-champ" ou pendant les temps de pauses, tels que des petits boulots, la vente d'objets et autres. 

\subsection{Améliorer le Psi} \label{sec:improving-psi} 

Les personnages possédant le trait Psi (p. 147) peuvent acheter de nouveaux exploits (voir Exploits, p. 223) au coût de 5 PR par exploit. Les exploits peuvent être appris par l'apprentissage, l'entraînement et la pratique, nécessitant approximativement 1 mois par exploit. On ne peut pas apprendre plus d'un exploit par mois. 







