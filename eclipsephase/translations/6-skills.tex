\chapter{Compétences} \label{chap:skills} 

Dans un cadre où l'apparence et les possibilités physiques sont changés d'un claquement de doigt, qui vous êtes et ce que vous savez est bien plus important que vos capacités innées. Les compétences représentent la connaissance qu'à votre personnage, le cumul des expériences vécues, de l'éducation reçue et du savori faire inhérent possédé par chaque transhumain conscient dans Eclipse Phase. Elles sont ce qui vous permet de vous faufiler dans une station hypercorporatiste, de désactiver le système de sécurité, de hacker le concentrateur mesh puis de vous faire passer pour le personnel de sécurité pour vous échapper. Vos compétences représentent ce que vous avez, peu importe votre apparence ou l'endroit où vous vous trouvez. lorsque vos personnages explorent leur possibilités, leur compétence, ou l'absence de compétence, sont souvent la différence entre la réussite ou l'échec. Avoir un ensemble cohérent de compétence est vital à la survie et à la réussite dans Eclipse Phase. Les compétences ci-dessous regroupent une large sélection de talents, suffisament pour que chaque personnage puisse être unique dans ses capacités et connaissances. 

\section{Aperçu des compétences.} Les compétences sont divisées enre les aptitudes et les compétences apprises (voir Compétences des Personnages, p. 123). La plupart (mais pas toutes) des compétences apprises sont basées et liées à une aptitude. Si un personnage manque d'une compétence spécifique nécessaire à une situtation, il peut défausser sur l'aptitude liée. Vous pouvez choisir de vous spécialiser dans certaines compétences (voir Spécialisations, p. 123), reflétant une connaissance améliorée d'un aspect particulier d'une compétence particulière. 

\subsection{COMPÉTENCES CENTRALES: APTITUDES} Les aptitudes représentent les compétences innées et les capacités acquise à la naissance ou pendant la croissance. Les aptitudes sont parfois utilisées pour les test, mais leur utilisation principale est de déterminer le point de départ duquel les compétences apprises sont développées. Les aptitudes déterminent la valeur de départ des compétences liées. Par exemple, un personnage avec une aptitude Somatique de 10 qui souhaite acheter des points dans la compétence Parkour (qui est liée à Somatique) commencera avec un rang de Parkour de 10 et achètera ensuite les points additionnel dans cette compétence. Les aptitudes sont également utilisées lorsqu'un personnage ne possèdent pas une compétence nécessaire (voir Se Défausser, p. 116). Les aptitudes représentent la connaissance basique qu'un personnage a acquis vis à vis de l'utilisation rudimentaire de cette compétence. Il peut ne pas avoir reçu d'entraînement formel dans la compétence, mais il peut toujours tenter de s'en servir. Les aptitudes vont de 1 à 30, 10 étant la moyenne pour un humain non augmenté et 15 la moyenne des humains modifiés génétiquement. Puisque les aptitudes représentent une capacité non entraînées, elles sont limitées au niveau 30. Il y a sept aptitudes différentes que tous les joueurs possèdent. Ces aptitudes sont achetées pendant la création de personnage (p. 128), mais en fonction de la morph actuellement habitée par le personnage, leurs capacité peuvent être limitées par la qualité de la moprh (voir p. 124). 

\subsubsection{COMPÉTENCES APPRISES} Les compétences apprises d'un joueur sont la part la plus importante de leur personnage, représentant les connaissances acquises qui les suivent de morph à morph, les savoirs qui jouent un rôle fondamental dans la définition de l'ego du personnage. Les compétences apprises regroupent presque toutes les compétences dont vous pourriez avoir besoin dans Eclipse Phase, et elles vont de 0 à 99. Toutes les compétences apprises ont une compétence liées qui est utilisée pour calculer leur valeur initiale, ainsi que lorsque le joueur ne possède pas cette compétence et doit se défausser. 



\subsubsection{CATÉGORIES DE COMPÉTENCE} Chaque compétences apprise est soit une compétence Active, soit une compétence de Connaissance. Les compétences Active représentent les compétences qui nécessitent en général une action physique et qui sont utilisées lors des scènes d'action pendant le jeu. Les ocmpétences de Connaissances sont plsu basées sur le savoir et l'intellect, représéentant les idées et les faits. Les compétences de Connaissances peuvent avoir un rôle moins dramatique dans certain moment de jeu plus orienté vers l'action, mais elles donnent corps aux historiques et intérêts du personnage et sont au cœur des interactions interprétées. Les compétences Actives et de Connaissances sont achetées séparément pendant la création de personnage. 

Les compétences Actives sont également réparties en compétences de Combat, Mentales, Physiques, Psi, Sociales, Techniques et de Véhicule. Certains traits et capacités peuvent s'appliquer à des catégories spécifiques. 



\subsubsection{COMPÉTENCE DE DOMAINE} certaines compétences apprises sont des compétences de domaine, signifiant que lorsque cette compétence est choisie, un domaine ou une emphase aprticulière doit également être choisie. Par exemple, la compétence Académique nécessite que le eprsonnage spécifie une discipline académique dans lequel il est bien informé, telle que Bilogie, Chimie ou Xénosociologie. les compétences de domaine sont notés de la manière suivante "[compétence]: [domaine];" par exemple "Art: Peinture." Les compétences de domaine peuvent être choisie plusieurs fois, en choisissant un champ de concentration à chaque fois, reflétant des compétences dans plsuierus domaines, chaque domaine est une compétence distincte. Plusieurs domaines suggérés sont listés pour chaque compétence de domaine, mais les maître de jeu et les joueurs peuvent également coopérer pour en créer d'autres qui correspondent à leur jeu. 

Les compétence de domaine peuvent également avoir des spécialisations; par exemple, Profession: Comptabilité (Blanchiement d'Argent). 



\subsubsection{COMPÉTENCES PSI} Le psi fait référence à la capacité de percevoir et manipuler les esprits biologiques grâces aux ondes psi et/ou d'autres phénomènes inexpliqués. En raison du caractère unique de cette capacité, les personnages souhaitant utiliser le psi doivent acquérir le trait 



Psi (p. 147). Le psi nécessite également un nombre de compétences spécialisées (Contrôle, Assaut Psi et Divination) qui reflètent un entraînement spécifique que le personnage acquiert pour puiser dans ses pouvoirs psi. Les compétences Psi ne peuvent pas être défausser; la seule façon d'utiliser une compétence psi est de posséder le trait ainsi que l'entraînement dans cette compétence. Pour plus de détails, voir Psi, p. 220. 



\subsubsection{SPÉCIALISATIONS} Tout personange peut chosiir de se spécialiser dans une compétence particulière (voir Sépcilisations, p. 123). Cette spécialisations reflète une connaissance accrue dans un aspect particulier de la compétence. La plupart des compétences proposées ci-dessous incluent des exemples de spécialisations. Les maître de jeu et les joueurs sont encouragés à développer d'autres idées de spécialisations ensemble pour leur campagne. 

Les spécialisations fournissent un modificateur de +10 lorsqu'il utilise cette compétence dans une situation appropriée à la spécialisation. 







\section{UTILISER LES COMPÉTENCES} Lorsqu'un personnage veut faire quelquechose en utilisant une compétence, il doit réussir un test de compétence (voir Faire des Tests, p. 115). La difficulté de l'action est déterminée par un modificateur, ainsi que toute autre cironstances pouvant affecter le test (voir Difficulté et Modificateurs, p. 115). Comme avec tous les autres types de tests, tous les tests de compétences réussissent si le personnage obtient un résultat inférieur ou égal au seil du test après que tous les modificateurs aient été appliqués. Dans le cas desdes tests de compétences, le seuil est le niveau de compétence du personnage dans cette compétence particulière. Des modificateurs représentant la difficulté et d'autres facteurs sont appliqués directement au seuil (voir Diificulté et Modificateurs, p. 115). Un résultat de 00 est toujours un succès, peu importe les modificateurs et un résultat de 99 est toujours un échec, toujours indépendament des modificateurs qui pourraient augmenter le seuil du personnage au-delà de 100. Les règles standard de succès et d'échec critiques s'appliquent aux tests de compétences (voir Crirtiques: Obtenir des Doubles, p. 116), donc à chaque fois qu'un personnage obtiens un double (càd 00, 11, 22, 33, etc) il obtiens un succès ou un échec critique. 



\subsubsection{SE DÉFAUSSER} Parfois vous ne possédez pas la compétence nécessaire à une certaine situation. Dans ces cas là, les personnages peuvent défausser leurs test de compétence sur l'aptitude liée. Cela relfète le fait que la plupart des comptéencs apprises sont développées depuis une sorte de capacité physique de base. Même si vous pouvez ne pas savoir commetn faire quelque chose, vous avez probablement vu comment on peut le faire ou avez une idée de la manière de procéder, ou vous pouvez toujours tenter votre chance. Naturellement, vous n'êtes pas aussi bon que quelqu'un qui a l'entraînement adéquat, mais cela vous permet quand même de tenter votre chance. 

Toutes les compétences ne peuvent être défausser. Certaines compétences sont simplement trop complexe ou obscure, ou demandent une connaissance ou une capacité spécifique, pour que quelqu'un tente de s'en servir sans l'entraînement nécessaire. Par exemple, la chirurgie du cerveau ou la plupart des compétences psi sotn tout simplement au-delà de quiconque ne possédant pas cette capacité ou la connaissance de ce qu'il tentent. 

\subsubsection{DÉFAUSSER UNE COMPÉTENCE DE DOMAINE} Dans certains cas, un personnage peut ne pas posséder le domaine particulier nécessaire pour un test, mais il peut être compétent dans un domaine proche. Par exemple, un test pour diriger l'autopsie d'un alien fait appel à un jet de Académique: Xénobiologie, mais un personnage qui n'a ps cette compétence peut être autoriser à se défausser sur Académique: Biologie. Le maître de jeu décide si et quand autoriser ceci, peut-être en appliquant un modificateur au test basé sur la différence entre les domaines. 



\subsubsection{DÉFAUSSER SUR UNE COMPÉTENCE LIÉE} Si le maître de jeu le permet, les personnages peuvent se défausser sur une compétence ayant un lien avec le test à gérer. Par exemple, un personnage compétent en Armes Cinétique pourrait ne pas être entraîné pour utiliser un laser, mais il en sait suffisament pour pointer la cible et appuyer sur la détente. De manière similaire, un personnage pourrait ne pas être compétent en Investigation, mais le maître de jeu pourrait toujorus autoriser l'utilisation de la compétence Perception afin de réaliser que el corps a été déplacé depuis l'endroit où il a été abattu. Dans des situations comme celles-ci, lorsque le maître de jeu autorise la défausse sur un compétence liée, un modificateur de -30 devrait être appliqué au test. 

\begin{quotation} \textbf{Exemple} \\ Srit se balade dans un souk du marché noir sur Mars, essayant de trouver une piède d'équipement sensoriel particulière. Le maître de jeu demande un Test de Fouille, mais Srit ne possède pas cett compétence. Elle peut défausser sur son INT de 22, mais elle préférer demander au maître de jeu si elle peut défausser sur la compétence liée Perception, qu'elle possède à 82. Le maître de jeu lui donne son accord, elle effectue donc un jet avec un seuil de 52 (82 - 30). \end{quotation} 







\subsubsection{COMPÉTENCES COMPLÉMENTAIRES} Parfois, plus d'une compétence peuvent être utilisée pour un test particulier, ou les connaissances dans un domaine peuvent aider votre compétence dans un autre. Dans ce cas, le maîþre de jeu peu appliquer un modificateur au test de compétence basé sur la force de la compétence complémentaire, telle que noté sur la table des Bonus de Compétences Complémentaire. 



\begin{quotation} \textbf{Exemple} \\ Dav essayent de persuader un pilote bordé de l'amener à un astéroïde isolé qui n'acceuille pas les visiteurs. Pour impressioner le pilote qui est un ami de ces isolés particulier, il fait appel à sa connaissance de leur pratique culturelle particulières (compétence Intérêts: Culte Religieux à 45). Le maîþre de jeu l'autorise et applique un modificateur de +20 sur le Test de Persuasion de Dav. \end{quotation} 

\begin{quotation} \textbf{BONUS DE COMPÉTENCES COMPLÉMENTAIRE} \\ \\ \begin{tabular}{cc} NIVEAU DE COMPÉTENCE &MODIFICATEUR\\ 01-30 &+10 \\ 31-60 &+20 \\ 61+ &+30 \\ \end{tabular} \end{quotation} 



\begin{quotation} \textbf{NIVEAU D'APTITUDE} \\ \\ \begin{tabular}{lllll} 

NIVEAU &ASSERTION &SOMATIQUE &COORDINATION &RÉFLEXES \\ 1–5 &enfant moyen &inapte &maladroit &lent\\ 6–10 &adulte moyen &faible &capable &rythmé \\ 11–15 &moyenne transhumaine &en forme &coordonné &prompt\\ 16–20 &amélioré &amélioré &agile &rapide \\ 21–25 &surhumain &doué &adroit &fulgurant \\ 26-30 &posthumain &élite &infaillible &synaptique \\ 





\\ NIVEAU &COGNITION &INTUITION &ASTUCE &VOLONTÉ \\ 1–5 &limité &conscient &embarassant &distrait\\ 6–10 &intelligent &perceptif &charmant &contrôlé \\ 11–15 &brillant &affuté &charismatique &concentré \\ 16–20 &appris &mystérieux &éblouissant &résolu \\ 21–25 &brillant &prescient &fascinant &décidé \\ 26-30 &génie &presqu'omniscient &hypnotique &inébranlable \\ \end{tabular} \end{quotation} 







\subsection{NIVEAU DE COMPÉTENCES} Quelle est la différence entre être un débutant empoté qui dérive en gravité zéro et être un vétéran glissant sans effort à travers l'espace aussi facilement que si vous dansiez? La réponse est l'entraînement et la compétence. Plsu votree compétence est élevée, plus vous aurez de chance de non seulement réussir à faire ce que vous voulez, mais à le faire bien. 

Les aptitudes dans Eclipse Phase vont de 1 à 30 alors que les compétences apprises vont de 0 à 99. Ces nombres sont une abstraction de la gamme de capacités et de traits transhumains. La table des Niveaux d'Aptittude fournit un récapitulatif des différents niveaux d'aptitudes et de la manière dont ils font référence les uns aux autres. De manière similaire, la table des Niveaux de Compétences fournit une interprétation des capacités à chaque niveau de compétence. 



\begin{quotation} \textbf{NIVEAUX DE COMPÉTENCES APPRISES} \\ \\ \begin{tabular}{cl} COMPÉTENCE &EQUIVALENCE \\ 00 &Aucune exposition ou familiriaté, complètement incompétent\\ 10 &Savoir extrêmement rudimentaire\\ 20 &Opération de base de la compétence (permis de conduire, permis de port d'arme, bac) \\ 30 &Expérience pratique, un peu d'entraînement professionnel \\ 40 &Certification professionelle basique (formation de conuite de la police, certificat d'utilisation d'armes militaire, diplôme universitaire) \\ 50 &Expérience d'un travail professionnel, un peu d'entraînement avancé \\ 60 &Expert (pilote de compétition, armurier, doctorat) \\ 70 &Expérience d'un travail d'expertise, a eu des idées ou des innovations uniques \\ 80 &Mérite d'être une autorité dans la matière, reconnu dans tout le système\\ 90 &Prix Nobel/Champio Olympique/Maître de conférence\\ 99 &Le sommet de la compréehnsion et de l'innovation actuel\\ \end{tabular} \end{quotation} 





\section{APTITUDES} Il y a 7 Aptitudes à Eclipse Phase, décrite à la p. 123. Chaque personnage possède ces aptitudes au niveau minimum de 1. 

\subsection{TEST D'APTITUDES PURES} Dans de rare cas, un test utilisant juste une aptitude peut être nécessaire. Cela ne devrait arriver que lrosqu'aucune compétence apprise ne correspond au test, et ces circonstances sont habituellement notés dans les règles. Les test d'aptitude pures doivent être gérés précautionneusement, car l'amplitude d'une aptitude (1-30) est typiquement bien plus faible que le niveau des compétences apprixes (1-99). Pour cette raison, la plupart des tests d'aptitudes devrait utiliser un seuil égal à l'aptitude x 3. Dans de rare cas et lorsque le test est plus difficile, le maîþre de jeu peu n'utiliser qu'un modificateur de x 2, ou juste la valeur de l'aptitude. Dans quelques cas, plus d'une aptitude peut être pertinentes pour le test, et elle peuvent être ajoutée l'une à l'autre pour en dériver le seuil. Ce qui suit sont quelques exemples où un test d'aptitude pure peut être approprié. Les maîtres de jeu peuvent demander des tests similaires dans d'autres situations, mais les compétences apprises devraient être utilisée partout lorsque c'est possible. 





\begin{quotation} \textbf{COMPARAISON D'APTITUDES: PLATES CONTRE SPLICEURS ET EXALTÉS} \\ Comparé aux humains du début du 21° siècle, le transhumain moyen dans le monde d'Eclipse Phase est plus rapide, plus intelligent, plus fort et en meilleure santé que leurs rpédécesseurs non-augmentés. Les humains normaux non-augmentés, appelés plats (p. 139), sont approximativement  le type de personne le plus proche de nos contemporains. La mejrue parie des personnes habite cependant des corps appelés spliceurs (p. 139) ou exaltés (p. 139) (du moins, ceux qui ont un corps biologique). Les spliceurs sont génétiquement réparés pour éviter les défauts génétiques et sont optimisés pour favoriser certaines caractéristique, alors que les exaltés sont améliorés pour les rendre supérieurs à tous les niveaux: ils sont plsu attractifs, plus athéltique, ont de meilleurs capacités cognitive et sont plus conscient du monde qui les entoure que leurs parents non augmentés. \end{quotation} 



\section{LISTE DE COMPÉTENCES COMPLÈTE} 





Cette section détaille toutes les capacités apprises disponible dans Eclipse Phase. Les maîtres de jeu et les joueurs peuvent, bien entendu, s'accorder pour ajouter des compétences additionnelles a cette liste de manière appropriée à leur campagne. 

\begin{quotation} \textbf{COMPÉTENCES NÉCESSAIRE} 

Alors que les personnages auront besoin d'un mélange de compétence pour réussir les diverses tâches qu'ils recontrerons dans Eclipse Phase, certaines compétences sont cruciale pour tous les personnages. Si un personnage ne possède pas ces compétences, ils auront des difficultés à s'en sortir, il est donc improtant que les joueur et les maîtree de jeu sachent quelles sont ces compétences particulières. 

\textbf{Esquive}: Esquive est la compétence principale que vous utiliserez pour éviter d'être touché en combat. Même si vous envisagez d'éviter el combat, être capable de pouvoir se jetter hors du passage lorsque c'est nécessaire est une compétence de survie pratique à posséder. 

\textbf{Réseau}: Sauf si vous vivez en isolation totale, vous aurez besoin d'une compétence Réseau - et il est préférable d'en avoir plusieurs. Réseau représente la manière dont vous interagissez avec les personnes dans un cercle social particulier pour obtenir des informations, propager des rumeurs, demander des faveurs et ainsi de suite. 

\textbf{Perception}: Les Tests de Perception sont relativemenbt fréquent, si vous voulez que votre personnage sache ce qu'il se passe autour de lui, soyez sûr de prendre cette compétence. Investigation et Fouille sont également de bonnes compétence, mais Perception est de loin la meilleure. \end{quotation} 



\begin{quotation} \begin{tabular}{lll} &\textbf{LISTE DE COMPÉTENCE} &\\ \\ COMPÉTENCE &APTITUDE LIÉE &CATÉGORIE\\ Académique: [Domaine] &COG &Connaissance\\ Dressage Animal &AST &Active, Sociale\\ Art: [Domaine] &INT &Connaissance\\ Armes à Rayons &COO &Active, Combat\\ Lames &SOM &Active, Combat\\ Escalade &SOM &Active, Physique\\ Massues &SOM &Active, Combat\\ Contrôle &VOL (pas de défausse) &Active, Mentale, Psi\\ Supercherie &AST &Active, Sociale\\ Démolitions &COG (pas de défausse) &Active, Technique\\ Déguisement &INT &Active, Physique\\ Arme de Mélée Exotique: [Domaine] &SOM &Active, Combat\\ Arme à Distance Exotique: [Domaine] &COO &Active, Combat\\ Vol &SOM &Active, Physique\\ Esquive &REF &Active, Combat\\ Chute Libre &REF &Active, Physique\\ Parkour &SOM &Active, Physique\\ Artillerie &INT &Active, Combat\\ Matériel: [Domaine] &COG &Active, Technique\\ Imposture &AST &Active, Sociale\\ Infiltration &COO &Active, Physique\\ Infosec &COG (pas de défausse) &Active, Technique\\ Intérêt: [Domaine] &COG &Connaissance\\ Interfaçage &COG &Active, Technique\\ Intimidation &AST &Active, Sociale\\ Investigation &INT &Active, Mentale\\ Kinésique &AST &Active, Sociale\\ Armes Kinétique &COO &Active, Combat\\ Langue: [Domaine] &INT &Connaissance\\ Médecine: [Domaine] &COG &Active, Technique\\ Navigation &INT &Active, Mentale\\ Réseau: [Domaine] &AST &Active, Sociale\\ Manipulation &COO &Active, Physique\\ Perception &INT &Active, Mentale\\ Persuasion &AST &Active, Sociale\\ Pilotage: [Domaine] &REF &Active, Véhicule\\ Profession: [Domaine] &COG &Connaissance\\ Programmation &COG (pas de défausse) &Active, Technique\\ Protocole &AST &Active, Sociale\\ Assaut Psi &VOL (pas de défausse) &Active, Mentale, Psi\\ Psychochirurgie &INT &Active, Technique\\ Recherche &COG &Active, Technique\\ Fouille &INT &Active, Mentale\\ Armes à Tête Chercheuse &COO &Active, Combat\\ Divination &INT (no defaulting) &Active, Mentale, Psi\\ Armes en Spray &COO &Active, Combat\\ Natation &SOM &Active, Physique\\ Armes de Lancer &COO &Active, Combat\\ Combat à Mains Nues &SOM &Active, Combat\\ \end{tabular} \end{quotation} 

\subsubsection{ACADÉMIQUE: [DOMAINE]} \textbf{Type:} Domaine, Connaissance\\ \textbf{Aptitude Liée:} COG \\ \textbf{Qu'est-ce:} Académique couvre tout type de connaissance spécialisée et non-appliquée que vous ne pouvez avoir que par une éducation intensive. La plupart des sciences théoriques et appliquées, des sciences sociales, des sciences transhumaines, etc. sont couverte par cette compétence. La plupart des autrs compétences listées dans ce chapitre peuvent aussi être prise en tant que domaine Académique, reflétant une connaissance théorique de la compétence - par exemple, Académique: Armurieur ou Académique: Interrogation. \\ \textbf{Quand l'utiliser:} Académique est utilisée lorsqu'un personnage souhaite faire appel à un corps de savoir particulier. Par exemple, Académique: Chimie peut être utilisée pour idnetifier une substance particulière, comprendre une réaction chimique inhabituelle ou déterminer quels éléments sont nécessaire pour nanofabriquer quelque chose nécessitant des matériau exotiques. À la discrétion du maître de jeu, certains test en rapport avec Académique en dervait pas être défaussable, étant donén que seul quelqu'un qiu a été suffisament éduqué sur le sujet a une chance de comprendre. \\ \textbf{Exemple de Domaines:} Arcéhologie, Astrobilogie, AStronomie, Astrophysique, Astrosocilogie, Chimie Organique, Biologie, Botanique, Théorie de l'Information, Cryptographie, Économie, Ingénierie, Génétique, Géologie, Linguistique, Mathématique, Mémétique, Nanotechnologie, Histoire de la Vieille Terre, Physique, Science Politique, Psychologie, Sociologie, Xéno-archéologie, Xénolinguistique, Zoologie\\ \textbf{Spécialisations:}Appropriée au domaine 

\subsubsection{DRESSAGE ANIMAL} \textbf{Type:} Active, Sociale \\ \textbf{Aptitude Liée:} AST \\ \textbf{Qu'est-ce:} Les dresseurs d'animaux compétents sont capables d'entraîner et de contrôller une grande variété d'animaux naturels et transgéniques incluant les élevés partiels. Bien que beaucoup d'espèce animales se sont éteintes pendant la Chute, quelques "arches" et habitats zoos ont conservés certaine espèces en vie, et beaucoup d'autres ont étés resscuscitée à partir d'échantillons génétiques. Les animaux exotiques sont considérés comme un signe de prestige parmi les élites hypercorporatistes, et des guardes animaux sont parfois utilisés pour protéger des installations de haute sécurité. De manière similaire, beaucoup d'habitats et d'abris emploient de petites armées d'élevés partiels, de créatures génétiquement modifées et au comprotement modifié pour l'assainissement ou dans d'autres buts. beaucoup de nouveaux animaus et de portées étranges sont créés quotidiennement pour servir une variété de rôle. \\ \textbf{Quand l'utiliser:} Dressage Animal est utilise lorsque vous essayez de manipuler un animal, que vous essayiez de le calmer, de l'empêcher d'attaquer, de l'intimider, d'obtenir sa confiance ou de le faire attaquer. Votre Marge de Réussite détermine votre eefficacité pour convaincre la créature. À la discrétion du maître de je, des modificateurs peuvent s'appliquer au test. De manière similaire, apprivoiser un animal peu prendre du temps, et devrait donc être considérer comme une Action de Tâche avec un intervalle de cinq minutes ou plus. \\ \textbf{Spécialisations:} Par espèce animale (chiens, chevaux, rats intelligents, etc.) 

\begin{quotation} \textbf{ENTRAÎNER DES ANIMAUX} \\ Entraîner des animaux est une tâche nécessitant du temps, des efforts répétés et des récompenses pour renforcer le comportement entraîné. Traitez cette action comme une Action de Tâches avec un intervalle allant de un jour à un mois, dépendant de la complexité de l'action. Appliquez les modificateurs à ce test en se basant sur l'intelligence relative de l'animal entraîné, sur son niveau de dmoestication et sur la complexité de la tâche. Une fois que l'animal a été entraîné, lui donner un ordre est considéré comme un Test de Succès Simple (p. 118) excepté lors de situation inhabituelle ou stressante, auquel cas, l'entraîneur reçoit un modificateur de +30 sur son Test de Dressage Animal lorsqu'il essaye de convaincre l'animal d'accomplir l'action apprise. \end{quotation} 

\subsubsection{ART: [DOMAINE]} \textbf{Type:} Domaine, Connaissance\\ \textbf{Aptitude Liée:} INT \\ \textbf{Qu'est-ce:} Art confère la possibilité de créer et d'éavluer les efforts artistiques. C'est une compétence particulièrement utile dans Eclipse Phase, et plus particulièrement dans les économies post-pénuries où la créativité et les visions peuvent être une composente clef de la réputation d'un personnage. \\ \textbf{Quand l'utiliser:} La compétence Art peut-être utilisée soit pour créer une nouvelle œuvre d'art ou pour en dupliquer une existante en espérant la faire passer pour une création originale. Cette compétence détermine également la valeur approximative d'une œuvre d'art soit sur le marché ouvert, dans les systèmes d'échanges monétaires, ou en terme de réputation pour l'artiste. \\ \textbf{Exemple de Domaine:} Architecture, Critique, Danse, Dramaturgie, Dessin, Peinture, Représentation, Sculpture, Conception de Simulspace, Chant, Dialectique, Écriture \\ \textbf{Spécialisations:}Appropriée au domaine 

\subsubsection{ARMES À RAYONS} \textbf{Type:} Active, Combat \\ \textbf{Aptitude Liée:} COO \\ \textbf{Qu'est-ce:} La compétence Armes à Rayons couvre l'utilisation et la maintenance des armes a énergie rayonnante cohérente standard telles que les lasers, les armes à particules, les fusils à plasma et les armes à micro-onde (p. 338). \\ \textbf{Quand l'utiliser:} Un joueur utilise la compétence Armes à Rayon lorsqu'il attaque avec une arme à rayopns en combat (p. 191). Armes à Rayons peut également être utilisée pour les tests incluant la maintenance de l'arme, mais pas pour réparer ou modifer l'arme (utilisez la compétence Matériel: Armurerie). \\ \textbf{Spécialisations:} Lasers, Armes à Micro-ondes, Armes à Particules, Fusils à Plasma. 

\subsubsection{LAMES} \textbf{Type:} Active, Combat \\ \textbf{Aptitude Liée:} SOM \\ \textbf{Qu'est-ce:} La compétence Lames couvre l'utilisation et la maintenance des armes à lames standard (p. 334). \\ \textbf{Quand l'utiliser:} Un joueur utilise la compétence Lames lorsqu'il attaque avec une arme à lames en combat rapproché (p. 191). Lames peut également être utilisée pour les tests incluant la maintenance de l'arme, mais pas pour réparer ou modifer l'arme (utilisez la compétence Matériel: Armurerie). Cette compétence est utilisée pour les lames implantées dans le corps à l'extrémité d'un appendice (mains, avant-bras, bras d'octomorph, pieds, etc.), mais la compétence Arme de Mélée Exotique est utilisées pour les lames implantées dans d'autres endroits du corps. \\ \textbf{Spécialisations:} Hâches, Lames Implantées, Couteaux, Épées. 

\subsubsection{ESCALADE} \textbf{Type:} Active, Physique\\ \textbf{Aptitude Liée:} SOM \\ \textbf{Qu'est-ce:} Escalade est la compétence pour monter et descendre sur des surfaces abrupte avec ou sans l'aide de matériel spécialisé. \\ \textbf{Quand l'utiliser:} Cette compétence est utilisée lorsqu'un personnage souhaite grimper sur une surface escaladable. Pour les altitudes nécessitant un peu plus que de la narration, l'escalade est géré comme une Action de Tâches avec un intervalle équivalent à un mètre par Phase d'Action. Pour descendre en rappel, l'intervalle est de 50 mètres par Tour d'Action. Le matériel d'escalade (p. 332-333) fournit des bonus appropriés. \\ \textbf{Spécialisations:} Assistée, Libre, Rappel 



\subsubsection{MASSUES} \textbf{Type:} Active, Combat \\ \textbf{Aptitude Liée:} SOM \\ \textbf{Qu'est-ce:} La compétence Massue couvre l'utilisation et la maintenance des armes à de mélée contondantes standards telles que les bâtons ou les matraques (p. 334). \\ \textbf{Quand l'utiliser:} Un joueur utilise la compétence Massue lorsqu'il attaque avec une arme contondantes en combat rapproché (p. 191). La compétence Massue peut également être utilisée pour les tests incluant la maintenance de l'arme, mais pas pour réparer ou modifer l'arme (utilisez la compétence Matériel: Armurerie). \\ \textbf{Spécialisations:} Bâtons, Marteaux, Matraques 

\subsubsection{CONTRÔLE} \textbf{Type:} Active, Mentale, Psi \\ \textbf{Aptitude Liée:} VOL \\ \textbf{Qu'est-ce:} Contrôle est l'utilisation du psi pour manipuler les individus ou pénétrer activement leurs processuss mentaux. Cette compétence n'est disponible que pour les personnages avec le trait Psi (p. 147). \\ \textbf{Quand l'utiliser:} Utilisez Contrôle lorsque vous effectuez une visite psionique dans un égo étranger - mettre le bazar inclut. Voir Psi, p. 220. \\ \textbf{Spécialisations:} Une par exploit 

\subsubsection{SUPERCHERIE} \textbf{Type:} Active, Sociale \\ \textbf{Aptitude Liée:} AST \\ \textbf{Qu'est-ce:} Supercherie est votre capacité à jouer un rôle, à bluffer, à arnaquer, à baratiner, à mentir, à dénaturer les faits et à prétendre. Les utilisateurs accomplis de supercherie sont capable de convaincre n'importe qui d'à peu près n'importe quoi. Cette compétence n'inclut pas l'utilisation d'un déguisement physique pour paraître être une autre personne (la compétence Imposture couvre ce domaine). \\ \textbf{Quand l'utiliser:} Utilisez cette compétence quand vous voulez tromper quelqu'un avec des mots et des gestes. Un test de compétence réussit signifie que vous avez réussit à faire gober votre mensonge de manière convaincante. À la discrétion du maître de jeu, quelqu'un qui est activement en laerte de signe de supercherie peut faire un Test Opposé en utilisant la compétence Kinésique. \\ \textbf{Spécialisations:} Endosser un Rôle, Bluffer, Barratin. 

\subsubsection{DÉMOLITIONS} \textbf{Type:} Active, Technique\\ \textbf{Aptitude Liée:} COG (pas de défausse) \\ \textbf{Qu'est-ce:} Démolitions couvre l'utilisation conrôlée des explosifs. \\ \textbf{Quand l'utiliser:} Utilisez Démolitions lorsque vous fabriquez, placez et désarmez des explosifs et des engins explosifs. Voir Démolitions, p. 197. \\ \textbf{Spécialisations:} Explosifs Commerciaux, Désarmement, Explosifs Improvisés. 

\subsubsection{DÉGUISEMENT} \textbf{Type:} Active, Physique \\ \textbf{Aptitude Liée:} INT \\ \textbf{Qu'est-ce:} Déguismeent est l'art d'altérer physiquement son apaprence pour ressembler à quelqu'un d'autre. Cela inclut l'utilisation d'accessoire (perruques, lentilles de contact, pigments de peau) et l'altération subtile de caractéristiques physique (démarche, posture, maintien). \\ \textbf{Quand l'utiliser:} Utilisez Déguisement pour faire croire à quelqu'un que vous êtes quelqu'un d'autre. Cela peut être utiliser pour cacher son identité ou pour se faire passer pour quelqu'un en aprticulier. Lorsqu'utilisé contre quelqu'un qui connaît votre véritable apparence ou l'apparence de celui que vous prétendez être, la situation est gérée par un Test en Opposition contre Perception ou Investigation. \\ \textbf{Spécialisations:} Cosmétique, Théatral 

\subsubsection{ARME DE MÉLÉE EXOTIQUE: [DOMAINE]} \textbf{Type:} Domaine, Active, Combat \\ \textbf{Aptitude Liée:} SOM \\ \textbf{Qu'est-ce:} La compétence Armes de Mélée Exotique couvre l'utilisation et la maintenance de toutes les armes de mélée non couvertes par les compétences Massues et Lames (p. 334). \\ \textbf{Quand l'utiliser:} Un joueur utilise la compétence Armes de Mélée Exotique lorsqu'il attaque avec une arme exotique en combat rapproché (p. 191). \\ \textbf{Exemple de Domaines:} Morgenstern, Lance, Fouet \\ \textbf{Spécialisations:} Non applicable 

\subsubsection{ARME À DISTANCE EXOTIQUE: [DOMAINE]} \textbf{Type:} Domaine, Active, Combat \\ \textbf{Aptitude Liée:} COO \\ \textbf{Qu'est-ce:} La compétence Armes à Distance Exotique couvre l'utilisation et la maintenance de toutes les armes à distance non couvertes par les compétences Armes à Rayons, Armes Cinétique, Armes Sonique, Armes de Jets ou Fléchettes (p. \\ \textbf{Quand l'utiliser:} Un joueur utilise la compétence Armes à Distance Exotique lorsqu'il attaque avec une arme exotique en combat à distance (p. 191). \\ \textbf{Exemple de Domaines:} Sarbacane, Arbalète, Lance-flamme, Fronde \\ \textbf{Spécialisations:} Non applicable 

\subsubsection{VOL} \textbf{Type:} Active, Physique\\ \textbf{Aptitude Liée:} SOM \\ \textbf{Qu'est-ce:} Vol est la compétence de l'utilisationd e votre corps pour voler. Cette compétence est utilisée lorsque vous êtes incarnés dans ou que vous interceptez ou morph ailée ou capable de voler (le vol manuel et téléguidé sont gérés par la compétence Piloter). \\ \textbf{Quand l'utiliser:} Utilisez cette compétence quand vous avez besoin de faire des manœuvres aérienne, atterrir dans des conditions difficiles, maintenir un cap au milieu de perturbations ou éviter de s'écraser ou de tomber d'une manière ou d'une autre. \\ \textbf{Spécialisations:} Plonger, Atterrir, Décoller, manœuvres spécifiques. 

\subsubsection{ESQUIVE} \textbf{Type:} Active, Combat \\ \textbf{Aptitude Liée:} REF \\ \textbf{Qu'est-ce:} Esquive est votre capacité a ne pas rester en travers des attaques, débris ou passants génants. Les personnages qui ont un score éléveé en Esquive sont capable de régair plus rapiement que les autres lrosqu'il s'agît d'esquiver ou de manœuvrer. \\ \textbf{Quand l'utiliser: } Lorsqu'un personnage est attaqué physiquement par un adversaire en combat au corps - à corps, faites un test sous Esquive pour éviter d'être toucher (voir p. 191). Esquive peut également être utilisée 

pour éviter d'autres évènements qui pourraient blesser le personnage, tels qu'éviter un véhicule ou dégager du chemin d'une pile de caisse s'effondrant. \\ \textbf{Spécialisations:} Lames, Massues, Défense Totale, Désarmé. 

\subsubsection{CHUTE LIBRE} \textbf{Type:} Active, Physique\\ \textbf{Aptitude Liée:} REF \\ \textbf{Qu'est-ce:} Chute Libre concerne tout ce qui a trait aux environnement en micro-gravité et à la chute-libre. \\ \textbf{Quand l'utiliser:} Utilisez cette compétence lorsque vous avez besoin de manœuvrer dans des situations en zéro-g, comem se rpopulser à travers un grand espace vide ou s'assurer de ne pas s'envoyer en vrille dans l'espace accidentellement. Chute Libre est également utilisée lorsque vous vous déplacez avec des propulseurs d'exocombinaison et lors de sauts en parachutes. \\ \textbf{Spécialisations:} Microgravité, Parachute, Exocombinaisons 

\subsubsection{PARKOUR} \textbf{Type:} Active, Physique\\ \textbf{Aptitude Liée:} SOM \\ \textbf{Qu'est-ce:} Parkour est en aprtie de la course, en partie de la gymnastique. Il s'agît de se déplacer rapidement, de manœuvrer au-dessus/au-dessous/à-côté/à travers des obstacles et de placer votre corps là où il doit être. Le Parkour est populaire dans les vieux habitats où l'espace ouvert est limité. \\ \textbf{Quand l'utiliser:} Utilisez Parkour lorsque vous avez besoin de franchir un obstacle grâce au mouvement, comme sauter par-dessus une balustrade, rouler sur le capot d'une voiture, sauter au-dessus d'une fosse ou tourner autour d'un poteau. Parkour est également utiliser pour sprinter (p. 191) et en défense totale contre les attaque (p. 198). \\ \textbf{Spécialisations:} Équilibre, Gymanstique, Saut, Course 

\subsubsection{ARTILLERIE} \textbf{Type:} Active, Combat \\ \textbf{Aptitude Liée:} INT \\ \textbf{Qu'est-ce:} La compétence Artillerie couvre l'utilisation et la maintenance des gros système d'armement, montées sur véhicule ou non-portable. Utiliser ces armes ressemble plus à jouer à un jeu vidéo qu'utiliser une arme. \\ \textbf{Quand l'utiliser:} Un joueur utilise la compétenceArtillerie lorsqu'il attaque avec une arme montée sur un véhicule ou une plate-forme d'arme lors d'un combat à distance (p. 191). \\ \textbf{Spécialisations:} Canons, Missiles 

\subsubsection{MATÉRIEL: [DOMAINE]} \textbf{Type:} Domaine, Active, Technique \\ \textbf{Aptitude Liée:} COG \\ \textbf{Qu'est-ce:} Cette compétence regroupe la capacité à construire, réparer, bidouiller et améliorer un équioement d'un type spécifique. \\ \textbf{Quand l'utiliser:} Matériel est utilisé principalement pour réparer des apapreils, des véhicules, des habitats ou des morphs synthétiques. Voir Construire, Réparer et Modifier ci-dessous. \\ \textbf{Exemple de Domaines:}Aérospatiale (tous les véhicules aériens et spatiaux), Armurerie (armure et armes), Électronique (tout appareils informatisé), Engisn terrestres, Implants, Industrie (habitat, usine et systèmes de survie), Nautique (sous-marins et bateaux),Robotique (morphs synthétique) \\ \textbf{Spécialisations:}Appropriée au domaine 



\begin{quotation} \textbf{Construire} \\ Créer un objet en partant de rien est géré comme une Action de Tâche avec un intervalle déterminé par le maître de jeu. L'intervalle de temps devrait être définie en fonction de la complexité de l'objet et devrait aller d'une heure (construire un jeu d'étagère) à plusieurs jours (assembler un robot à partir de pièces détachées) voire plusieurs mois (construire une maison). De nombreux facteurs peuvent appliquer des modificateurs au test, tels que l'utilisation de plans/manuels entoptiques (+20) ou de mauvaises conditions de travail (-10 à -30). Les outils sont également un facteur, peut-être facilitant le travail (outils supérieurs +10 à +30), le compliquant (outils de mauvaises qualité ou inadaptés, -10 à -30) ou le rendant impossible (absence des outils nécessaire). \end{quotation} 

\begin{quotation} \textbf{Réparer} \\ Les objets abîmés peuvent être réparés de manière similaire. Coir les règles pour les Synthmorphs et la Réparation d'Objet p. 209. \end{quotation} 



\begin{quotation} \textbf{Modifier} \\ Altérer la conception d'un objet et son fonctionnement respecte les mêmes règles que celels de construction et réparation au-dessus. L'intervalle de temps est déterminé par le maître de jeu de manière approprié à la modification. \end{quotation} 









\subsubsection{IMPOSTURE} \textbf{Type:} Active, Sociale \\ \textbf{Aptitude Liée:} AST \\ \textbf{Qu'est-ce:} Imposture est la compétence pour vous faire passer pour quelqu'un d'autre dans les situations sociales, incluant les situations virtuelles. Cela inclut de copier le manièrisme et les tournures linguistique ainsi que l'utilisation des informations accumulées pour convaincre les autres que vous êtes qui vous prétendez être. Dans un univers où l'apparence est extrêmement variable, la question de l'identité est essentiellement une question à la fois de confiance et de reconnaissance de bizarreries comportementale et de repères verbaux  nécessaire à reconnaître une personne. \\ \textbf{Quand l'utiliser:} Parfois il peut-être rigolo de prétendre que vous êtes quelqu'un d'autre, et pafois cela est profitable ou pourra vous sauver la vie. Utiliser cette compétence lorsque vous tentez de convaincre quelqu'un que vous êtes en fait quelqu'un d'autre grâce à une certaine interaction sociale ou en-ligne. Les forks utilisent cette compétences quand ils essayen t de se faire apsser pour leur égo alpha. Imposture est géré par un Test en Opposition contre la compétence Kinésique. \\ \textbf{Spécialisations:} Avatar, Face-à-Face, Verbale 

\subsubsection{INFILTRATION} \textbf{Type:} Active, Physique\\ \textbf{Aptitude Liée:} COO \\ \textbf{Qu'est-ce:} Infiltration est l'art d'échapper à la détection. \\ \textbf{Quand l'utiliser:} Utilisez Infiltration quand vous avez besoin de vous cacher physiquement ou de vous déplacer discrètement pour éviter que quelqu'un ne vous perçoive, que vous vous cachiez derière un arbre, que vous vous faufiliez derrière un guarde ou que vous vous mélangiez dans une foule. Infiltration peut également être utilisé pour suivre les personnes (filature) sans qu'ils ne vous détectent. Infiltration est géré par un Test en opposition contre la Perception de tout ce dont vous essayer de vous cacher. Le maître de jeu peut vouloir lancer de tels tests en secret pour que les joueurs ne savent pas si ils ont réussit ou raté. \\ \textbf{Spécialisations:} Se Fondre Dans al Foule, Se cacher, Filature, S'infiltrer. 

\subsubsection{INFOSEC} \textbf{Type:} Active, Technique\\ \textbf{Aptitude Liée:} COG (pas de défausse) \\ \textbf{Qu'est-ce:} Infosec est une abréviation pour "information security (sécurité de l'information)." Elle regroupe l'entraînement en intrusion électronique et les techniques de contre intrusion, ainsi que le chiffrement et le déchiffrement. \\ \textbf{Quand l'utiliser:} Académique est utilisée à la fois pour pénétrer un appareil électronique et les réseaux meshés et pour les protéger. Voir le chapitre sur Le Mesh, p. 234, pour plus de détails. \\ \textbf{Spécialisations:} Attaque en Force-Brute, Déchiffrement, Analyse, Sécurité, Interception de traffic, Spoofing 

\subsubsection{INTÉRÊT: [DOMAINE]} \textbf{Type:} Domaine, Connaissance\\ \textbf{Aptitude Liée:} COG \\ \textbf{Qu'est-ce:} Intérêt inclut tout ce qui concerne un sujet qui capte votre attention et qui n'est pas couvert par une autre compétence. Elle inclut les hobbys, les obsessions, les causes, les passe-temps et d'autres objetcifs récréatifs. \\ \textbf{Quand l'utiliser:} Utilisez la coméptence Intérêt lorsque vous devez faire appel à ou utiliser une connaissance liée à l'intérêt particulier en question. Exemple de Domaines: Ancients Sports,  Actualité People , Conspiration, Information sur les Facteurs, Politique Hypercorporatiste, Habitats Lunaire, Bières Martiennes, Nation-État de la Vieille Terre, Blog Réclamationnistes, Science Fiction, Racailles Trafiquants de Drogue, Modèles de Vaisseaux, Économie des Triades, XP Clandestine\\ \textbf{Spécialisations:}Appropriée au domaine 

\subsubsection{INTERFAÇAGE} \textbf{Type:} Active, Technique\\ \textbf{Aptitude Liée:} COG \\ \textbf{Qu'est-ce:} Interfaçage concerne l'utilisation de matériel électronique informatisé et des logiciels. \\ \textbf{Quand l'utiliser:} Utilisez Interfaçage poru comprendre le fonctionnement d'un appareil électronique avec lequel vous n'êtes pas familier, utiliser un programme d'arpès ses paramètres normaux d'utilisation, manipuler des fichiers électroniques de différents type (incluant les images, la vidéo, l'XP et l'audio), chercher des apapreils sans-fil et interagir d'une manière ou d'une autre avec votre ecto, votre muse ou d'autre appareils informatisés. Certaines actions d'Interfaçage peuvent être des Actions de Tâches, avec un intervalle de temps déterminé par le maître de jeu. Pour plus de détail, voir le chapitre sur le Mesh p. 234. \\ \textbf{Spécialisations:} Falsification, Balayage, Stéganographie, par programme. 

\subsubsection{INTIMIDATION} \textbf{Type:} Active, Sociale \\ \textbf{Aptitude Liée:} AST \\ \textbf{Qu'est-ce:} Intimidation c'est convaincre quelqu'un de faire ce que vous voulez grâce à des menaces directes (insinuées ou réelle) ou par la seule force de sa personnalité. \\ \textbf{Quand l'utiliser:} Utilisez Intimidation pour effrayer quelqu'un et le soumette, le forcer à faire ce que vous voulez, lui ordonner de suivre vos ordres ou l'admonester de vous donner des informations. Intimidation est géré par un Test en Opposition contre lae total VOL + VOL + AST de la cible. \\ \textbf{Spécialisations:} Interrogation, Physique, Verbale 

\subsubsection{INVESTIGATION} \textbf{Type:} Active, Mentale, Psi \\ \textbf{Aptitude Liée:} INT \\ \textbf{Qu'est-ce:} Investigation est l'art d'analyser des preuves, d'assembler des indices, de résoudre els mystères et de faire des dédcutions logiques à aprtir d'un ensemble de fait. Investigation diffère de Perception car il s'agît d'une recherche minutieuse d'indice ou de pièce du puzzle. \\ \textbf{Quand l'utiliser:} Utilisez cette compétence pour tirer des conclusions depuis un ensemble de détails. Par exemple, Investigation peut être utilisé pour déterminer la séquence probable des évènements sur une scène du crime, déterminer une connexion sociale possible entre deux personnes ou déduire comment un ennemi à pu s'échapper. Investigation ets un bon moyen 

de fournir des indices aux jouerus, particulièrement lorsque le sujet concerné est quelque chose de familier ua personnage masi pas pour le joueur. \\ \textbf{Spécialisations:} Analyse de Preuve, Déductions Logique, Investigation physique, Filature Physique. 

\subsubsection{KINÉSIQUE} \textbf{Type:} Active, Sociale \\ \textbf{Aptitude Liée:} AST \\ \textbf{Qu'est-ce:} Kinésique est l'art de l'empathie et de la communication non-verbale. \\ \textbf{Quand l'utiliser:} Utilisez Kinésique pour lire le language corporel, les tics, les indices sociaux et autres indicateurs subconscient. Elle peut aussi être utilisée pour retranscrire ses émotions plus efficacement. Kinésique est utilisée de manière défensive lorsque quelqu'un essaye de vous mentir; faite sun Test en Opposition contre la compétence Supercherie ou Imposture de cette personne. Même si les morphs synthétique sont également conçue pour exprimer des émotions, les lire n'est pas aussi simple. Appliquez un modificateur de -30 lorsque vous évaluez une morph synthétique habitée par un personnage ou une IAG. De manière similaire, les IAs standard sont également difficile à lire; appliquez un modificateur de -60 lorsque vous essayez de jauger une morph synthétique ou un pod contrôlé par une IA. \\ \textbf{Spécialisations:} Jauger les Intentions, Communication Non-Verbale. 

\begin{quotation} \textbf{JAUGER DES ÉMOTIONS ET DES INTENTIONS} \\ Les transhumaisn sont des êtres empathique, et vous pouvez donc tenter de deviner les motivations et/ou les intentions de quelqu'un avec qui vous faites affaire en tentant un Test de Kinésique. Cette tentative de lire quelqu'un est cpednant loin d'être fiable et est facilement mal interprétée. le maître de jeu devrait faire ce test en secret et n'autoriser un indice que si il est réussi - il n'est pas possible de lire quelqu'un avec une certitude absolue. Si la personne ciblée essaye intentionnellement de mentir au personnage, cela devrait se résoudre en un Test en Opposition contre sa compétence Supercherie. \end{quotation} 

\begin{quotation} \textbf{COMMUNICATION NON VERBALE} \\ Les experts en Kinésique peuvent communiquer efficacement avec les autres en utilisant simplement des postures, des attitudes, des gestes, des comportement et des regards. Une telle communication est nécessairement limitée dans la quantité d'information qu'elle eput relayer, mais les sentiments, les attitudes, la'ffirmation/la négation et des concepts simples peuvent être transmis. Pour communiquer effectivement des concepts complexe, le maître de jeu peut nécessiter des Test de Kinésiques réussit des deux parties, appliquant des modificateurs appropriés. \end{quotation} 

\subsubsection{ARMES CINÉTIQUE} \textbf{Type:} Active, Combat \\ \textbf{Aptitude Liée:} COO \\ \textbf{Qu'est-ce:} La compétence Armes Cinétique couvre l'utilisation et la maintenance des armes à projectiles cinétiques standard telle que les armes à feu et les armes à rail (p. 335). \\ \textbf{Quand l'utiliser:} Un joueur utilise la compétence Armes Cinétique lorsqu'il attaque avec une arme cinétique en combat à distance (p. 191). \\ \textbf{Spécialisations:} Fusils d'Assaut, Mitrailleuses, Pistolets, Fusils de Sniper, Armes Automatiques 



\subsubsection{LANGUE: [DOMAINE]} \textbf{Type:} Domaine, Connaissance\\ \textbf{Aptitude Liée:} INT \\ \textbf{Qu'est-ce:} Langue couvre la capacité à parler et lire une langue autre que la langue natale du personnage. Parler la langue de manière fluide correspond à un niveau de 50; tout ce qui est au-dessus indique un développement du vocabulaire technique, des accents et la connaissance des dialectes. \\ \textbf{Quand l'utiliser:} Utilisez la compétence Lnague lorsque vous devez parler, comprendre ou lire quelque chose dans une langue pour laquelle vous êtes compétent. la plupart des test de compréhension écrite et orale peuvent être considéré comem des Tests de Succès Simple si votre compétence dépasse 50, à moins que le maître de jeu décide que le sujet de la discussion est suffisament complexe pour que les pratiquants non-natifs puissent avoir du mal à comprendre. \\ \textbf{Exemple de Domaines:} Arabe, Cantonais, Anglais, Français, Hindoux, Mandarin, Portuguais, Russe, Espagnol \\ \textbf{Spécialisations:} Appropriée au domaine, représentant les dialectes, le jargon technique et l'argot sous-culturel 

\begin{quotation} \textbf{LES LANGUES DANS ECLIPSE PHASE} \\ Avec la Chute de la Terre, les langues qui sont encore dominantes dans le système solaire sont celles qui ont été largement emmenée dans l'espace par les pays et les hypercorps ayant des programmes spatiaux agressifs ou par les grosses populations de travailleurs pauvre et d'infugiés qui ont suivis. Aucune langue ne dominait le royaume de l'expansion spatiale, et le multilingusiem était commun. Beaucoup d'habitats et de groupes (sous) culturel s'en tiennent à une lnague spécifique comme méthode de conserver une identité culturelle. En dépit de la disponibilité de traduction instanténée via le mesh, beaucoup de personne reste compétent dans deux langues ou plus. Les dix langues les plus communes dans le système solaire en taille de population les parlants sotn: Arabe, Cantonais, Anglais, Français, Hindoux, Japonais, Mandarin, Portuguais, Russe et Espagnol. D'autres langues encore trés parlées incluent le Bengali, le Flamand, le Farsi, l'Allemand, l'Italien, le Javanais, le Coréen, le Polonais, le Punjabi, le Suédois, le Tamoul, le Turc, l'urdu, le Viétnamien et le Wu.  Certaines langue ont étés effectivement perdue pendant la Chute, particulièrement celles des régions non développés, alros que la population qui parlait cette langue n'a pu migrer dans l'espace pré-Chute et n'étaient pas suffisament priviligiés pour survivre en assez grnad nombre comme infugiés. \end{quotation} 





\subsubsection{MÉDECINE: [DOMAINE]} \textbf{Type:} Domaine, Active, Technique \\ \textbf{Aptitude Liée:} COG \\ \textbf{Qu'est-ce:} Médecine regroupe les soins et la maintenance des êtres et formes de vie biologiques. \\ \textbf{Quand l'utiliser:} Utilisez la coméptence Médecine lrosque vous devez administrer des soins au-delà de l'aide immédiate apportée par les premiers secours. Cela inclut la conduite d'examens physique, le diagnostique des maux, le traitement des problèmes et des maladies, la chirurgie, l'utiulisation d'ouril médicaux biotechnologique et nanotechnologique et les soins à long-terme. Voir Soins et Réparations, p. 208. \\ \textbf{Exemple de Domaine:} Biosculpture, Biomorphs Exotique, Thérapie Génique, Pratique générale, Chirurgie Implantatoire, Nanomédecine, Mercuriens (par type), Paramédecine, Pods, Psychiatry, Chirurgie à Distance, Chirurgie Traumatique, Vétérinaire \\ \textbf{Spécialisations:}Appropriée au domaine 

\subsubsection{NAVIGATION} \textbf{Type:} Active, Mentale, Psi \\ \textbf{Aptitude Liée:} INT \\ \textbf{Qu'est-ce:} Investigation est l'art de trouver son chemin, que vous utilisiez des cartes AR, un compas, les étoiles ou une IA d'astrogation. \\ \textbf{Quand l'utiliser:} Utilisez cette compétence quand vous avez besoin de calculer une route, déterminer une direction ou de garder une trace quelconque pour éviter de se perdre. \\ \textbf{Spécialisations:} Astrogation, Cartographie, Topographie 

\subsubsection{RÉSEAU: [DOMAINE]} \textbf{Type:} Active, Sociale \\ \textbf{Aptitude Liée:} AST \\ \textbf{Qu'est-ce:} Réseau est la compétence qui sert à travailler vos contacts, échanger des faveurs et garder le pouls d'une faction ou d'un groupe culturel particulier. \\ \textbf{Quand l'utiliser:} Utiliser Réseau pour rassembler des information ou demander des services en utilisant votre Réputation (voir Réputation et Réseaux Sociaux, p. 285). \\ \textbf{Exemple de Domaine:} Autonomistes (@-rep), Criminels (g-rep), Écologistes (e-rep), Firewall (i-rep), Hypercorporations (c-rep), Média (f-rep), Scientifiques (r-rep). À la discrétion du maître de jeu, cette liste peut être étendue à d'autre groupe (sous) culturel. \\ \textbf{Spécialisations:}Appropriée au domaine 

\subsubsection{MANIPULATION} \textbf{Type:} Active, Physique\\ \textbf{Aptitude Liée:} COO \\ \textbf{Qu'est-ce:} Manipulation est la compétence pour manipuler des objets rapidement et lestement sans que les autres ne le remarque. Manipulation n'est pas qu'à propos de la manipulation habile d'objet mais repose aussi sur la déconcertation, le rythme et la déosrientation. \\ \textbf{Quand l'utiliser:} Utilisez Manipulation à chaque fois que vous essayez de cacher un objet sur votre personne, de voler à l'étalage, de jouer au pick pocket, de subreciptement jetter quelque chose ou d'accomplir un tour de prestidigitation. Manipulation est un Test en Opposition contre la Perception des spectateurs. Le maître de jeu peut désirer faire ce jet en secret. \\ \textbf{Spécialisations:} Pick Pocket, Vol à l'Étalage, Tour 

\subsubsection{PERCEPTION} \textbf{Type:} Active, Mentale, Psi \\ \textbf{Aptitude Liée:} INT \\ \textbf{Qu'est-ce:} Perception est l'utilisation de vos sens physiques (incluant la cybernétique) et la conscience du monde physique autour de vous. Investigation diffère de Perception car il s'agît de remarquer quelque chose par hasard, plutôt que de le rechercher activement. \\ \textbf{Quand l'utiliser:} Utilisez cette compétence quand vous avez besoin d'observer votre environnement en détail. \\ \textbf{Quand l'utiliser:} Utilisez Persuasion à chaque fois que vous essayez de troquer avec, de convaincre ou de manipuler quelqu'un. Cela peut inclure motiver vos subordonnées ou pairs à agir, séduire un compagnon, gagner un débat politique ou négocier un contrat entre autres choses. La Persuasion est gérée comme un Test en Opposition contre la somme de VOL + VOL + AST de la cible lorsqu'une personne essaye simplement d'avoir raison par rapport à l'autre. Si les deux parties essayent de se convaincre mutuellement, faites un Test en Opposition entre les compétences Persuasion. \\ \textbf{Spécializations:} Diplomacie, Remonter le Moral, Négocier, Séduire 

\subsubsection{PILOTER: [DOMAINE]} \textbf{Type:} Domaine, Active, Véhicule\\ \textbf{Aptitude Liée:} REF \\ \textbf{Qu'est-ce:} Piloter est votre compétence pour conduire/faire voler un véhicule d'un type particulier. \\ \textbf{Quand l'utiliser:} Utilisez la compétence Piloter lorsque vous devez manœuvrer, contrôller ou éviter de s'écraser, que vous soyez dans le siège du pilote, que vous contrôliez à distance un robot ou en court-circuitant directement le véhicule en RV. Chaque véhicule a un modificateur de Maniabilité qui s'applique à ce test, ainsi que d'autres modificateurs situationnels (voir Bots, Synthmorphs et Véhicules, p. 195). \\ \textbf{Exemples de Domaine:} Avions, Antropomorphe (marcheurs), Véhicule Exotique, Engins Terrestre (à roue ou à chenille), Vaisseaux Spatiaux, Bateaux \\ \textbf{Spécialisations:}Appropriée au domaine 

\subsubsection{PROFESSION: [DOMAINE]} \textbf{Type:} Domaine, Connaissance\\ \textbf{Aptitude Liée:} COG \\ \textbf{Qu'est-ce:} Les compétences Profession indiquent un entraînement dans l'une des profession pratiquée dans Eclipse Phase. Cela peu indiquer soit un entraînement formel ou informel, une expérience faites sur le tas, et inclut à la fois les marchés légaux et paralégaux. \\ \textbf{Quand l'utiliser:} Utilisez la compétence Profession pour effectuer des tâches en lien avec un travail pour un marché spécifique (telles que miner, équilibrer des comptes, concevoir un système de sécurité, etc.) ou pour référencer une connaissance spécialisé que quelqu'un d'entraîné dans cette profession pourrait avoir . \\ \textbf{Exemples de Domaine:} Comptabilité, Estimation, Prospection d'Astéroïde, Planificateur d'Arnaque, Clkassification, Criminalistique, Techynicien de Laboratoire, Extratcion Minière, Procédures Policières, Psychothérapie, Opération de Sécurité, Route de Contrebande, Ingénierie Sociale, Tactiques d'Escouade, Marketing Viral, Production XP\\ \textbf{Spécialisations:}Appropriée au domaine 

\subsubsection{PROGRAMMATION} \textbf{Type:} Active, Technique\\ \textbf{Aptitude Liée:} COG (pas de défausse) \\ \textbf{Qu'est-ce:} Programmation est votre talent pour écrire et modifier du code logiciel. \\ \textbf{Quand l'utiliser:} Utilisez Programmation pour écrire de nouveaux programmes, modifier ou patcher un logiciel existant, casser un système de protection des données ou introduire des failels exploitable, écrire des virus ou des vers, concevoir des environnements virtuels, et ainsi de suite. Voir le chapitre sur Le Mesh, p. 234. Programmation est également utilisant lors de l'utilisation d'appareils de nanofabrication. \\ \textbf{Spécialisations:} Code d'IA, Malware, Nanofabrication, Piratage, Code de Simulspace 

\subsubsection{NANOFABRICATION} La nanofabrication est l'utilisation de la compétence Programmation pour créer des objets en utilisant une machine d'abondance, un fabeur ou un faiseur (p. 327). si vous avez les plans et les matériau bruts appropriés, la plupart des utilisation d'un nanofabricateur peut être traité comme un Test de Succès Simple (p. 118). Si vous souhaitez créer un objet pour lequel vous n'avez pas de schémas ou les matériaux bruts adéquats ou que vous souhaitez altérer la conception d'un objet, un test de nanofabrication est cependant nécessaire. Voir Nanofabrication, p. 284. \\ \textbf{Spécialisations:} Art, Vêtements, Électroniques, Nourriture, Falsification, Armes 

\subsubsection{PROTOCOLE} \textbf{Type:} Active, Sociale \\ \textbf{Aptitude Liée:} AST \\ \textbf{Qu'est-ce:} Protocole est l'art de faire bonne impression dans un cadre social. Cela inclus le fait de se tenir au courant des derniers mèmes, tendances, rumeurs, intérêts et habitude de divers groupes (sous)culturel. \\ \textbf{Quand l'utiliser:} Utilisez cette compétence quand vous avez besoin de choisir vos mots avec attention, déterminer qui est l'interlocuteur idéal, impressioner quelqu'un avec votre compréhension des coutumes ou de vous adaptez d'une manière ou d'une autre à un groupe social/culturel. A la fois étiquette et connaissance de la rue, Protocole vous permet de naviguer parmi les requins et de mettre les gens à l'aise. Si le personnage traite avec une audience suspicieuse ou hostile, faites un Test en Opposition contre VOL + VOL + AST de la cible. \\ \textbf{Spécialisations:} Anarchiste, Bordés, Criminels, Facteurs, Hypercorporation, Infomorph, Mercurien, Réclamationnistes, Préservationnistes, Racaille, Ultimes. 

\subsubsection{ANNULER DES GAFFES SOCIALES} parfois, un joueur fera une erreur que leur personnage n'aurait jamais faites, que ce soit de ne pas rreconnaître la royauté d'un hypercorporatistes, de confondre un chef de gang avec un troufion ou d'insulter accidentellement l'héritage de quelqu'un. Dans des cas comme ceux-là, le joueur peut faire un Test de Protocole pour le domaine approprié afin d'annuler la gaffe. Si le test est réussi, le personnage n'a en fait jamais merdé, ou s'en est sorti pour au moins couvrir ses traces sans voler dans les plumes de quelqu'un. 



\subsubsection{ASSAUT PSI} \textbf{Type:} Active, Mentale, Psi \\ \textbf{Aptitude Liée:} VOL \\ \textbf{Qu'est-ce:} Assaut Psi est la compétence pour endommager l'esprit d'un autre ego. Cette compétence n'est disponible que pour les personnages avec le trait Psi (p. 147). Quand l'utiliser: Utilisez Assaut Psi lorsque vous attaquer l'esprit d'un autre ego en combat psi. \\ \textbf{Spécialisations:} Une par exploit 

\subsubsection{PSYCHOCHIRURGIE} \textbf{Type:} Active, Technique\\ \textbf{Aptitude Liée:} INT \\ \textbf{Qu'est-ce:} Psychochirurgie est l'utilisation de techniques psychologique assistée par ordinateur pour réparer, endommager ou manipuler la psyché. \\ \textbf{Quand l'utiliser:} Utiliser Psychochirurgie lorsque vous tenter le processuss hasardeux de l'édition e l'esprit de quelqueun (voir Psychochirurgie, p. 229). Psychochirurgie peut-être utilisée au bénéfice de patient qui se rappellent de leur mort, se sentent déconnectés après un remorph ou ont vécu d'autres types de traumatisme mentaux. Cette compétence peut également interroger, torturer ou d'une manière ou d'une autre trafiquer des esprits captifs d'environnements RV. \\ \textbf{Spécialisations:} Manipulation de Souvenir, Édition de PErsonnalité, Psychothérapie. 

\subsubsection{RECHERCHE} \textbf{Type:} Active, Technique\\ \textbf{Aptitude Liée:} COG \\ \textbf{Qu'est-ce:} Recherche est la compétence utilisée pour trouver de l'information sur le Mesh: recherche; analyse, exploitation et interprétation. Cela inclut la connaissance des lieux ou chercher, quels liens suivre, et comment optimiser vos requètes. \\ \textbf{Quand l'utiliser:} Utiliser la compétence Recherche lorsque vous avez besoin de trouver la réponse à une question, trouver des bases de données, fouiller dans des archives ou suivre quelque chose en ligne. Recherche est généralement une Action de Tâche avec un intervalle et un modificateur de difficulté déterminé par le maître de jeu. Voir Recherche En Ligne, p. 249. \\ \textbf{Spécialisations:} Suivre, par type d'information 

\subsubsection{FOUILLE} \textbf{Type:} Active, Mentale \\ \textbf{Aptitude Liée:} INT \\ \textbf{Qu'est-ce:} Fouille est votre capacité à trouver des choses, en particulier les objets cachés, enterrés ou difficile à trouver  pouvant être utiles ou ayant une certaine valeur. Cela inclut la connaissance des endroits à chercher et ce qu'il faut chercher. Fouille diffère de Perception et d'Investigation car elle s'occupe de trouver des objets cachés aprmi d'autre, et dans la plupart des cas il s'agit de trouver quelque chose de particulier (nourriture, objets de valeurs, etc). \\ \textbf{Quand l'utiliser:} Utilisez Fouille pour trouver de quoi manger dans une poubelle, fouiller des ruines à la recherche de relique, trouver quelque chose d'intéressant dans un bazar, ramasser des baies dans une forêt, localiser une exocombinaison dans un vaisseau abandonné, etc. Fouille est généralement une Action de Tâche avec un intervalle et un modificateur de difficulté déterminé par le maître de jeu. \\ \textbf{Spécialisations:} Bazars, Forêts, Habitats, Ruines 

\subsubsection{ARMES A TÊTE CHERCHEUSE} \textbf{Type:} Active, Combat \\ \textbf{Aptitude Liée:} COO \\ \textbf{Qu'est-ce:} La compétence Armes à Tête Chercheuse couvre l'utilisation et la maintenance des lanceurs à têtes chercheuses  (p. 339) et des missiles à tête chercheuse (p. 340). \\ \textbf{Quand l'utiliser:} Un joueur utilise cett compétence lorsqu'il attaque avec une armeà tête chercheuse en combat à distance (p. 191). \\ \textbf{Spécialisations:} Lanceurs de Poignets, Pistolets, Fusils, Lancuer Sous Canon 

\subsubsection{DIVINATION} \textbf{Type:} Active, Mentale, Psi \\ \textbf{Aptitude Liée:} INT \\ \textbf{Qu'est-ce:} Divination est l'utilisation du psi pour analyser les egos. Cette compétence n'est disponible que pour les personnages avec le trait Psi (p. 147). Pour plus de détails, voir Psi, p. 220. \\ \textbf{Spécialisations:} Une par exploit 

\subsubsection{ARMES À SPRAY} \textbf{Type:} Active, Combat \\ \textbf{Aptitude Liée:} COO \\ \textbf{Qu'est-ce:} La compétence Armes à Spray couvre l'utilisation et la maintenance des armes à distance ayant une zone d'effet en cône (p. 340). \\ \textbf{Quand l'utiliser:} Un joueur utilise la compétence Armes à SPray lorsqu'il attaque avec une arme à spray en combat à distance (p. 191). \\ \textbf{Spécialisations:} Buzzer, Freezer, Aiguilles, Éclats, Torche 

\subsubsection{NATATION} \textbf{Type:} Active, Physique\\ \textbf{Aptitude Liée:} SOM \\ \textbf{Qu'est-ce:} Natation est l'art de se déplacer et de en pas se noyer en milieu liquide. Cela inclus la flottaison, la natation de surface, la plongée en apnée, le plongeon et l'utilisation du matériel lié. \\ \textbf{Quand l'utiliser:} Utilisez cette compétence quand vous avez besoin de vous déplacer et de survivre dans l'eau ou tout autre environnement liquide. Nager dans un environnement non dangereux peut être géré par un Test de Succès Simple. Nager sur une longue distance peut être géré par une Action de Tâche. Plonger d'une falaise dans un lac, éviter d'être emporter par le courant rageur d'une rivière ou s'assurer que vous avez le bon mélange de gaz pour une plongée en haute-mer, entre auters chsoes, nécessite un Test de Succès. \\ \textbf{Spécialisations:} Plonger, Natation Synchronisée, PLongée Sous-Marine 

\subsubsection{ARMES DE JET} \textbf{Type:} Active, Combat \\ \textbf{Aptitude Liée:} COO \\ \textbf{Qu'est-ce:} La compétence Armes de Jet couvre l'utilisation et la maintenance des armes de jet standard, comme les grenades (p. 340). \\ \textbf{Quand l'utiliser:} Un joueur utilise la compétence Armes de Jet lorsqu'il attaque avec une arme de jet en combat à distance (p. 191). \\ \textbf{Spécialisations:} Hâches, Lames Implantées, Couteaux, Épées. 







\subsubsection{COMBAT À MAINS NUES} \textbf{Type:} Active, Combat \\ \textbf{Aptitude Liée:} SOM \\ \textbf{Qu'est-ce:} Combat à Mains Nues est votre capacité à attaquer et à vous défendre sans utilsier d'arms. \\ \textbf{Quand l'utiliser:} utilisez Combat à Mains Nues lorsque vous attaquez quelqu'un avec vos poings, pieds, coudes, genoux ou d'autres parties de votre corps en combat au contact (p. 191). \\ \textbf{Spécialisations:} Armes Implantées, Coup de Pied, Coup de Poing, Contrôle 

\begin{quotation} \textbf{UTILISER LES COMPÉTENCES DE CONNAISSANCES } \\ Au premier abrod, il pourrait sembler que les Compétences de Connaissances ont moins d'applications en jeu que les Compétences Actives. C'est el cas dans une certaine mesure. L'importance des Compétences de Connaissances, ne devrait cependant pas être sous-estimé. Alors qu'elles jouent un rôle dans l'analyse des indices et la résolution de  mystères, la véritable valeur des Compétences de Connaissances et dans l'aide qu'elles fournissent aux personnages - et aux joueurs - pour comprendre le monde d'Eclipse Phase. Ces compétences peuvent être utilisée en aprticulier pour faire des plans, analyser une situation, identifier les forces et les faiblesses, évaluser la valeur, faire des comparaisons, anticiper les issues probables ou comprendre la science appliquée, les facteurs socio-économiques, ou les contexte culturels ou historiques. Par exemple, un groupe de personnage cherchant àpénétrer une installation pourrait utiliser Profession: Procédures de Sécurité pour évaluer les défenses, Académique: Architecture pour identifier les entrées dissimulées, Intérêts: Sports pour planifier leur infiltration au moment où les gardes seron tprobablement distrait, Intérêts: Triades pour identifier un groupe criminel local qui pourrait leur vendre du matériel de cambriolage et Art: Sculpture au moment de prendre une œuvre d'art de valeur avec laquelle acheter du personnel à l'intérieur. Lorsqu'elles sont utilisées de manière appropriée, ces compétences peuvent être aussi bénéfique que les Compétences Actives utilisées pour s'infiltrer, si ce n'est plus car le plan à plus de chances de réussir en raison de cette préparation. C'est essentiellement au maître de jeu de renforcer le rôle de ces Compétences de Connaissances durant leurs parties. La méthode la plus simple pour augmenter leur pertinence est de panéliser les personnages qui ne s'en servent pas. Par exemple, les personnages qui n'utilisent pas leur compétence Profession: Security Procedures dans l'exemple ci-dessus pourrait se retrouver surpris en traversant un système de sécurité qu'ils n'avaient pas anticipés, les forçant à improviser ou même à abandonner leurs plans. \end{quotation} 

\section{COMPÉTENCES SPÉCIALES} Alros que la liste précédente représente les ocmpétences les plus communément utilisées dans Eclipse Phase, il peut y avoir certaines compétences nécessaire à une campagne et qui ne figurent pas dans ce livre. Dans ce cas, le maître de jeu peu travailler avec le joueur à définir une nouvelle compétence pour remplir le vide. Cette option ne devrait être utilisée que rarement afin d'éviter une surcharge de compétence, et toutes les compétences doivent être approuvée par le maitre de jeu. Si vous choisissez de créer une nouvelle compétence, gardez en têe qu'elle doit être liée à une aptitude existante et qu'elle devrait être une compétence disponible pour tous les personnages et non pas réservée à un seul d'entre eux. 







































































































































