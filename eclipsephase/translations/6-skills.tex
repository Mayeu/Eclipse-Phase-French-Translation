\chapter{Skills} \label{chap:skills} 

In a setting where physical looks and capabilities are easily changed at the push of a button, who you are and what you know is more important than any inborn ability. Skills represent the knowledge your character has, the accumulated set of experience, education, and inherent know-how possessed by each and every sentient transhuman in Eclipse Phase. They are what allow you to sneak into a hypercorp station, disable the security systems, hack the mesh hub, and then impersonate security personnel to make your escape. Your skills represent the one thing you have no matter what you look like or where you find yourself. When your characters explore what they can do, their skills, or lack thereof, often determine the margin between success and failure. Having a well-rounded set of skills is vital to survival and success in Eclipse Phase. The skills below encompass a wide selection of talents, enough so that each character can be unique in their abilities and knowledge. 

\section{Skill overview} Skills are divided into aptitudes and learned skills (see Character Skills, p. 123). Most (but not all) learned skills are built on and linked to an aptitude. If a character lacks the specific skill needed in a situation, they may default to the linked aptitude. You may also choose to specialize in certain skills (see Specializations, p. 123), reflecting an enhanced knowledge of a particular aspect of a certain skill. 

\subsection{CORE SKILLS: APTITUDES} Aptitudes represent inherent skills and abilities acquired at birth or during the course of growing up. Aptitudes are sometimes used for tests, but their primary use is determining the starting point at which learned skills are developed. Aptitudes determine the starting value of their linked skills. For example, a character with Somatics aptitude 10 who wishes to purchase points in the Freerunning skill (which is linked to Somatics) would start with a Freerunning rank of 10 and then buy additionally points in that skill. Aptitudes are also used when a character doesn’t posses knowledge of a needed skill (see Defaulting, p. 116). Aptitudes represent the basic knowledge that a character has acquired regarding rudimentary use of that skill. They may not have ever received any formal training with the skill, but they can still attempt to use it. Aptitudes range in value from 1 to 30, with 10 being the unaugmented human average and 15 representing the average of most genetically modified transhumans. Since aptitudes represent untrained ability, they are capped at a maximum rating of 30. There are seven different aptitudes that all players possess. These aptitudes are purchased during character creation (p. 128), but depending on the morph the character is currently inhabiting, they may find their aptitudes capped by the quality of the morph (see p. 124). 

\subsubsection{LEARNED SKILLS} A player’s learned skills are the most important part of their character, representing the acquired knowledge they carry with them from morph to morph, knowledge that plays a fundamental role in helping define the person’s ego. Learned skills encompass nearly any skill that you might need to use in Eclipse Phase, and they range in value from 0 to 99. All learned skills have a linked aptitude that is used to calculate their initial value, and which is also defaulted to if the player does not have that particular skill. 



\subsubsection{SKILL CATEGORIES} Each learned skill is classified as either an Active skill or a Knowledge skill. Active skills represent skills that typically require physical actions and are used in action scenes within game play. Knowledge skills are more knowledge-based and intellectual, representing ideas and facts. Knowledge skills may play a less dramatic role in certain action-oriented game play moments, but they flesh out the character’s background and interests and are integral to roleplaying interactions. Active and Knowledge skills are purchased separately during character creation. 

Active skills are further divided into Combat, Mental, Physical, Psi, Social, Technical, and Vehicle skills. Certain traits and abilities may apply to specific categories. 



\subsubsection{FIELD SKILLS} Some learned skills are field skills, meaning that when this skill is chosen a particular field of emphasis must also be selected. For example, the skill of Academics requires the character to specify a specific academic discipline in which they are knowledgeable, such as Biology, Chemistry, or Xenosociology. Field skills are written as “[skill]: [field];” for example: “Art: Painting.” Field skills can be taken multiple times, choosing a different area of emphasis each time, reflecting skills in different fields; that is to say, each field is a separate skill. Several suggested fields are listed for each field skill, but gamemasters and players may also cooperate to create others that fit their games. 

Field skills may also have specializations; for example, Professional: Accounting (Money Laundering). 



\subsubsection{PSI SKILLS} Psi refers to the ability to perceive and manipulate biological minds via psi waves and/or other inexplicable phenomena. Due to the uniqueness of this ability, characters that wish to wield psi must acquire the Psi 



trait (p. 147). Psi use also requires a number of specialized skills (Control, Psi Assault, and Sense) that reflect special training characters acquire to tap into their psi powers. Psi skills may not be defaulted on; the only way to use a psi skill is to possess the trait along with training in that skill. For more details, see Psi, p. 220. 



\subsubsection{SPECIALIZATIONS} Any character may opt to specialize in a given skill (see Specializations, p. 123). This specialization reflects increased knowledge in one particular aspect of the skill. Many of the skills offered below include sample specializations. Gamemasters and players are encouraged to develop other specialization ideas together for their campaigns. 

Specialization provides a +10 modifier when using that skill in a situation appropriate to that specialization. 







\section{USING SKILLS} Whenever a character wants to do something using a skill, they must succeed at a skill test (see Making Tests, p. 115). The difficulty of the action is applied as a modifier, as are any other extenuating circumstances that may affect the test (see Difficulty and Modifiers, p. 115). As with other types of tests, all skill tests are successful when the character rolls less than or equal to the test’s target number after any modifiers have been applied. In the case of skill tests, the target number is the character’s skill rating with that particular skill. Modifiers representing difficulty and other factors are applied directly to the target number (see Difficulty and Modifiers, p. 115). A roll of a 00 is always a success, regardless of modifiers, and a result of 99 is always a failure, again despite any modifiers that may increase a character’s target number over 100. Standard critical success and failure rules apply to skill tests (see Criticals: Rolling Doubles, p. 116), so any time a character rolls a double (i.e. 00, 11, 22, 33, etc.) they score a critical success or failure. 



\subsubsection{DEFAULTING} Sometimes you lack the skill needed in a certain situation. In these instances, characters may default their skill test to the linked aptitude. This reflects the fact that most learned skills are developed from some sort of baseline physical ability. Even though you may not know how to do something, you’ve likely seen how it’s done at some point or have some idea of how to do it, or can at least take a shot at it. Naturally, you’re not as good as someone who has training in that skill, but it still allows you to make an attempt. 

Not all skills can be defaulted. Some skills are simply too complex or obscure, or demand special knowledge or ability, for someone to attempt their use untrained. For example, brain surgery or most psi skills are simply beyond anyone who doesn’t have that ability or the knowledge of what they’re attempting. 

\subsubsection{DEFAULTING TO FIELD SKILLS} In some cases, a character may not possess the particular field skill that a test calls for, but they may be skilled in another related field. For example, a test to conduct an alien autopsy might call for an Academics: Xenobiology roll, but a character who doesn’t have that skill may be allowed to default to Academics: Biology instead. The gamemaster decides if and when to allow this, perhaps applying a modifier to the test based on the difference between fields. 



\subsubsection{DEFAULTING TO RELATED SKILLS} If the gamemaster allows it, characters may default to a related skill that also has some relevance to the test at hand. For example, a character skilled in Kinetic Weapons might not be trained in the use of a laser, but they know enough to point at the target and pull the trigger. Likewise, a character might not be skilled in Investigation, but the gamemaster could still allow them to use their Perception skill instead in order to realize that a body had been moved from the place where it had been shot. In situations like this, when the gamemaster allows defaulting to a related skill, a –30 modifier should be applied to the test. 

\begin{quotation} \textbf{Exemple} \\ Srit is wandering through a black market souk on Mars, trying to find a particular piece of sensory equipment. The gamemaster calls for a Scrounging Test, but Srit does not have that skill. She could default her INT of 22, but instead she asks the gamemaster if she can default to the related skill of Perception, which she has at 82. The gamemaster agrees, and so Srit rolls against a target number of 52 (82 – 30). \end{quotation} 







\subsubsection{COMPLEMENTARY SKILLS} Sometimes more than one skill may apply to a particular test, or knowledge in one area can aid your skill in another. In this case, the gamemaster may apply a modifier to the skill test based on the strength of the complementing skill, as noted on the Complementary Skill Bonus table. 



\begin{quotation} \textbf{Exemple} \\ Dav is hoping to persuade a brinker pilot to take him to an isolated habitat that doesn’t welcome visitors. To impress upon the pilot that he is a friend of these particular isolates, he calls on his knowledge of their particular cultural practices (Interests: Religious Cults skill at 45). The gamemaster allows this and applies a +20 modifier to Dav’s Persuasion Test. \end{quotation} 

\begin{quotation} \textbf{COMPLEMENTARY SKILL BONUS} \\ \\ \begin{tabular}{cc} SKILL RATING &MODIFIER \\ 01-30 &+10 \\ 31-60 &+20 \\ 61+ &+30 \\ \end{tabular} \end{quotation} 



\begin{quotation} \textbf{APTITUDE RANGE} \\ \\ \begin{tabular}{lllll} 

RATING &ASSESSMENT &SOMATICS &COORDINATION &REFLEXES \\ 1–5 &child average &inept &clumsy &slow \\ 6–10 &adult average &weak &able &paced \\ 11–15 &transhuman average &fit &coordinated &swift \\ 16–20 &enhanced &enhanced &agile &fast \\ 21–25 &superhuman &gifted &nimble &lightning \\ 26-30 &posthuman &elite &unerring &synaptic \\ 





\\ RATING &COGNITION &INTUITION &SAVVY &WILLPOWER \\ 1–5 &limited &aware &awkward &distracted \\ 6–10 &intelligent &perceptive &personable &controlled \\ 11–15 &bright &sharp &charismatic &focused \\ 16–20 &learned &uncanny &dazzling &resolute \\ 21–25 &brilliant &prescient &mesmerizing &unwavering \\ 26-30 &genius &near omniscient &hypnotic &unshakable \\ \end{tabular} \end{quotation} 







\subsection{SKILL RANGES} What is the difference between being a clumsy neophyte wobbling in zero gravity and being a veteran gliding effortlessly through space as though you were dancing? The answer is training and skill. The greater your skill, the more likely you are to not only succeed at what you want to do, but succeed well. 

Aptitudes in Eclipse Phase range from 1 to 30, while learned skills range from 0 to 99. These numbers are an abstraction of the range of transhuman abilities and traits. The Aptitude Range table provides a breakdown of different aptitude levels and how they relate to each other. Likewise, the Learned Skill Range table provides an interpretation for the capabilities at different skill levels. 



\begin{quotation} \textbf{LEARNED SKILL RANGES} \\ \\ \begin{tabular}{cl} SKILL &EQUIVALENCE \\ 00 &No exposure or familiarity, completely unskilled \\ 10 &Very rudimentary knowledge \\ 20 &Basic operator’s proficiency (driver’s license, gun permit, high school diploma) \\ 30 &Hands-on experience, some professional training \\ 40 &Basic professional certification (police driving, army rifle certified, college diploma) \\ 50 &Experience from professional-level work, some advanced training \\ 60 &Expert competence (competitive driver, marksman, PhD) \\ 70 &Experience from expert-level work, has had unique innovations or insights \\ 80 &Worthy of being a system-renowned authority on the subject \\ 90 &Nobel/Olympic/grandmaster \\ 99 &Pinnacle of current understanding and innovation \\ \end{tabular} \end{quotation} 





\section{APTITUDES} There are 7 aptitudes in Eclipse Phase, described on p. 123. Each character has these aptitudes at a minimum rating of 1. 

\subsection{APTITUDE-ONLY TESTS} In rare cases, a test may call for using an aptitude only, rather than a learned skill. This should only occur when no learned skills are appropriate to the test, and these circumstances are usually noted in the rules. Aptitude-only tests must be handled carefully, as the range of aptitude ratings (1–30) is typically much smaller than the rating of learned skills (0–99). For this reason, most aptitude tests should use a target number equal to the aptitude x 3. In rare cases where the test is more difficult, the gamemaster may simply use an aptitude x 2, or just the straight aptitude rating. In some cases, more than one aptitude may be relevant to the test, and so they may be added together to derive the target number. What follows are a few examples where an aptitudeonly test might be appropriate. Gamemasters may call for similar tests in other situations, but learned skills should be used whenever possible. 





\begin{quotation} \textbf{APTITUDE COMPARISON: FLATS VS. SPLICERS AND EXALTS} \\ Compared to humans in the early 21st Century, the average transhuman in the world of Eclipse Phase is faster, smarter, stronger, and healthier than their unaugmented predecessors. Normal unaugmented humans, called flats (p. 139), most closely approximate the type of person that was born in our time. The majority of people, however, inhabit bodies that are known as splicers (p. 139) or exalts (p. 139) (well, those with biological bodies anyway). Splicers are genefixed to avoid genetic defects and optimized for certain characteristics, while exalts are tweaked to make them superior across the board: they are more attractive, more athletic, have greater cognitive capacity, and are more attuned to the world around them than their unaugmented kin. \end{quotation} 



\section{COMPLETE SKILL LIST} 





This section details all of the learned skills available in Eclipse Phase. Gamemasters and players may, of course, agree to add additional skills to this list as appropriate to their campaign. 

\begin{quotation} \textbf{NECESSARY SKILLS} 

While characters will need a mix of skills to succeed in the varied tasks they encounter in Eclipse Phase, some skills are crucial for any character. If a character lacks these, they will have a difficult time getting by, so it is important for players and gamemasters to know these particular skills. 

\textbf{Fray}: Fray is the primary skill you use to avoid getting hit in combat. Even if you plan to avoid combat, being able to get out of the way when necessary is a handy survival skill to have. 

\textbf{Networking}: Unless you live in total isolation, you need a Networking skill—preferably several. Networking is how you interact with people in a particular social circle to obtain information, spread rumors, call in favors, and so on. 

\textbf{Perception}: Perception Tests get called for quite often, so if you want your character to know what’s going on around them, make sure to get this skill. Investigation and Scrounging are also good, but Perception is king. \end{quotation} 



\begin{quotation} \begin{tabular}{lll} &\textbf{SKILL LIST} &\\ \\ SKILL &LINKED APTITUDE &CATEGORY\\ Academics: [Field] &COG &Knowledge\\ Animal Handling &SAV &Active, Social\\ Art: [Field] &INT &Knowledge\\ Beam Weapons &COO &Active, Combat\\ Blades &SOM &Active, Combat\\ Climbing &SOM &Active, Physical\\ Clubs &SOM &Active, Combat\\ Control &WIL (no defaulting) &Active, Mental, Psi\\ Deception &SAV &Active, Social\\ Demolitions &COG (no defaulting) &Active, Technical\\ Disguise &INT &Active, Physical\\ Exotic Melee Weapon: [Field] &SOM &Active, Combat\\ Exotic Ranged Weapon: [Field] &COO &Active, Combat\\ Flight &SOM &Active, Physical\\ Fray &REF &Active, Combat\\ Free Fall &REF &Active, Physical\\ Freerunning &SOM &Active, Physical\\ Gunnery &INT &Active, Combat\\ Hardware: [Field] &COG &Active, Technical\\ Impersonation &SAV &Active, Social\\ Infiltration &COO &Active, Physical\\ Infosec &COG (no defaulting) &Active, Technical\\ Interest: [Field] &COG &Knowledge\\ Interfacing &COG &Active, Technical\\ Intimidation &SAV &Active, Social\\ Investigation &INT &Active, Mental\\ Kinesics &SAV &Active, Social\\ Kinetic Weapons &COO &Active, Combat\\ Language: [Field] &INT &Knowledge\\ Medicine: [Field] &COG &Active, Technical\\ Navigation &INT &Active, Mental\\ Networking: [Field] &SAV &Active, Social\\ Palming &COO &Active, Physical\\ Perception &INT &Active, Mental\\ Persuasion &SAV &Active, Social\\ Pilot: [Field] &REF &Active, Vehicle\\ Profession: [Field] &COG &Knowledge\\ Programming &COG (no defaulting) &Active, Technical\\ Protocol &SAV &Active, Social\\ Psi Assault &WIL (no defaulting) &Active, Mental, Psi\\ Psychochirurgie &INT &Active, Technical\\ Research &COG &Active, Technical\\ Scrounging &INT &Active, Mental\\ Seeker Weapons &COO &Active, Combat\\ Sense &INT (no defaulting) &Active, Mental, Psi\\ Spray Weapons &COO &Active, Combat\\ Swimming &SOM &Active, Physical\\ Throwing Weapons &COO &Active, Combat\\ Unarmed Combat &SOM &Active, Combat\\ \end{tabular} \end{quotation} 

\subsubsection{ACADEMICS: [FIELD]} \textbf{Type:} Field, Knowledge \\ \textbf{Linked Aptitude:} COG \\ \textbf{What it is:} Academics covers any sort of specialized non-applied knowledge you can only get through intensive education. Most theoretical and applied sciences, social sciences, transhumanities, etc. are covered by this skill. Most of the other skills listed in this chapter could also be taken as an Academics field, reflecting a working theoretical knowledge of the skill—for example, Academics: Armorer or Academics: Interrogation. \\ \textbf{When you use it:} Academics is used when a character wishes to call upon a specific body of knowledge. For example, Academics: Chemistry could be used to identify a particular substance, understand an unusual chemical reaction, or determine what elements are needed to nanofabricate something that requires exotic materials. At the gamemaster’s discretion, some Academics-related tests might not be defaultable, given that only someone who has been educated in that subject is likely to be able to tackle it. \\ \textbf{Sample Fields:} Archeology, Astrobiology, Astronomy, Astrophysics, Astrosociology, Biochemistry, Biology, Botany, Computer Science, Cryptography, Economics, Engineering, Genetics, Geology, Linguistics, Mathematics, Memetics, Nanotechnology, Old Earth History, Physics, Political Science, Psychology, Sociology, Xeno-archeology, Xenolinguistics, Zoology \\ \textbf{Specializations:} As appropriate to the field 

\subsubsection{ANIMAL HANDLING} \textbf{Type:} Active, Social \\ \textbf{Linked Aptitude:} SAV \\ \textbf{What it is:} Skilled animal handlers are able to train and control a wide variety of natural and transgenic animals, including partial uplifts. Though many animal species went extinct during the Fall, a few “ark” and zoo habitats keep some species alive, and many others can be resurrected from genetic samples. Exotic animals are considered a sign of prestige among the hypercorp elites, and guard animals are occasionally used to protect high-security installations. Likewise, many habitats and settlements employ small armies of partially uplifted, genetically modified, and behavior-controlled creatures for sanitation or other purposes. Many new and strange breeds of animal are created daily to serve a variety of roles. \\ \textbf{When you use it:} Animal Handling is used whenever you are trying to manipulate an animal, whether your intent is to calm it down, keep it from attacking, intimidate it, acquire its trust, or goad it into attacking. Your Margin of Success determines how effective you are at convincing the creature. At the gamemaster’s discretion, modifiers may be applied to the test. Likewise, winning an animal over may sometimes take time, and so could be handled as a Task Action with a timeframe of five minutes or more. \\ \textbf{Specializations:} Per animal species (dogs, horses, smart rats, etc.) 

\begin{quotation} \textbf{TRAINING ANIMALS} \\ Training animals is a time-consuming task requiring repeated efforts and rewards to reinforce the trained behavior. Treat this as a Task Action with a timeframe of one day to one month, depending on the complexity of the action. Apply modifiers to this test based on the relative intelligence of the animal being trained, how domestic it is, and the complexity of the task. Once an animal has been trained, commanding it is treated as a Simple Success Test (p. 118) except for unusual or stressful situations, in which case the trainer receives a +30 modifier on their Animal Handling Tests when convincing the animal to complete the trained action. \end{quotation} 

\subsubsection{ART: [FIELD]} \textbf{Type:} Field, Knowledge \\ \textbf{Linked Aptitude:} INT \\ \textbf{What it is:} Art confers the ability to create and evaluate artistic endeavors. This is a particularly useful skill in Eclipse Phase, especially in the post-scarcity economies where creativity and vision can be a key component to a character’s reputation. \\ \textbf{When you use it:} The Art skill can be used to either create a new work of art or to duplicate an existing piece of art in the hopes of passing it off as your own. The skill can also determine the approximate value of a piece of art either on the open market, for monetary exchange systems, or in terms of reputation for the artist. \\ \textbf{Sample Fields:} Architecture, Criticism, Dance, Drama, Drawing, Painting, Performance, Sculpture, Simulspace Design, Singing, Speech, Writing \\ \textbf{Specializations:} As appropriate to the field 

\subsubsection{BEAM WEAPONS} \textbf{Type:} Active, Combat \\ \textbf{Linked Aptitude:} COO \\ \textbf{What it is:} The Beam Weapons skill covers the usage and maintenance of standard coherent beam energy weapons such as lasers, particle beam weapons, plasma rifles, and microwave weapons (p. 338). \\ \textbf{When you use it:} A player uses their Beam Weapons skill whenever attacking with a beam weapon in combat (p. 191). Beam Weapons may also be used for tests involving maintenance of the weapon, but not for repairing or modifying the weapon (that would be Hardware: Armorer skill). \\ \textbf{Specializations:} Lasers, Microwave Weapons, Particle Beam Weapons, Plasma Rifles 

\subsubsection{BLADES} \textbf{Type:} Active, Combat \\ \textbf{Linked Aptitude:} SOM \\ \textbf{What it is:} The Blades skill covers the usage and maintenance of standard bladed weapons (p. 334). \\ \textbf{When you use it:} A player uses their Blades skill whenever attacking with a blade weapon in melee combat (p. 191). Blades may also be used for tests involving maintenance of the weapon, but not for repairing or modifying the weapon (that would be Hardware: Armorer skill). This skill is used for blade weapons implanted in the body at the end of an appendage (hands, forearms, feet, octomorph arms, etc.), but the Exotic Melee Weapon skill is used for blades implanted in other parts of the body. \\ \textbf{Specializations:} Axes, Implant Blades, Knives, Swords 

\subsubsection{CLIMBING} \textbf{Type:} Active, Physical \\ \textbf{Linked Aptitude:} SOM \\ \textbf{What it is:} Climbing is the skill of ascending and descending sheer surfaces with or without the aid of specialized equipment. \\ \textbf{When you use it:} This skill is used whenever a character wishes to scale a climbable surface. For heights greater than one story, climbing is handled as a Task Action with a timeframe equivalent to one meter per Action Phase. For rappelling, the timeframe for descent is 50 meters per Action Turn. Climbing gear (p. 332-333) provides appropriate modifiers. \\ \textbf{Specializations:} Assisted, Freehand, Rappelling 



\subsubsection{CLUBS} \textbf{Type:} Active, Combat \\ \textbf{Linked Aptitude:} SOM \\ \textbf{What it is:} The Clubs skill covers the usage and maintenance of standard blunt melee weapons such as batons or sticks (see p. 334). \\ \textbf{When you use it:} Players use their Clubs skill whenever they want to attack with a blunt weapon in melee combat (p. 191). The Clubs skill may also be used for tests involving maintenance of the weapon, but not for repairing or modifying the weapon (that would be Hardware: Armorer skill). \\ \textbf{Specializations:} Batons, Hammers, Staffs 

\subsubsection{CONTROL} \textbf{Type:} Active, Mental, Psi \\ \textbf{Linked Aptitude:} WIL \\ \textbf{What it is:} Control is the use of psi to manipulate individuals or actively penetrate their mental processes. This skill is only available to characters with the Psi trait (p. 147). \\ \textbf{When you use it:} Use Control when taking a psionic tour through a foreign ego—messing around included. See Psi, p. 220. \\ \textbf{Specializations:} By sleight 

\subsubsection{DECEPTION} \textbf{Type:} Active, Social \\ \textbf{Linked Aptitude:} SAV \\ \textbf{What it is:} Deception is your ability to act, bluff, con, fast talk, lie, misrepresent, and pretend. Accomplished users of deception are able to convince anyone of nearly anything. This skill does not include using a physical disguise to appear to be another person (the Impersonate skill covers that area). \\ \textbf{When you use it:} Use this skill whenever you want to deceive someone with words or gestures. A successful skill test means that you have passed off your deception convincingly. At the gamemaster’s discretion, someone who is actively alert for signs of deception may make an Opposed Test using the Kinesics skill. \\ \textbf{Specializations:} Acting, Bluffing, Fast Talk 

\subsubsection{DEMOLITIONS} \textbf{Type:} Active, Technical \\ \textbf{Linked Aptitude:} COG (no defaulting) \\ \textbf{What it is:} Demolitions covers the use of controlled explosives. \\ \textbf{When you use it:} Use it when making, placing, and disarming explosives and explosive devices. See Demolitions, p. 197. \\ \textbf{Specializations:} Commercial Explosives, Disarming, Improvised Explosives 

\subsubsection{DISGUISE} \textbf{Type:} Active, Physical \\ \textbf{Linked Aptitude:} INT \\ \textbf{What it is:} Disguise is the art of physically altering your appearance so that you look like someone else. This includes both the use of props (wigs, contacts, skin pigments) and the altering of subtle physical characteristics (gait, posture, poise). \\ \textbf{When you use it:} Use Disguise to fool someone into thinking you’re someone you’re not. This can be used to hide your identity or to make yourself look like someone in particular. When used against someone who knows your true look or the appearance of the person you are imitating, this is handled as an Opposed Test against Perception or Investigation. \\ \textbf{Specializations:} Cosmetic, Theatrical 

\subsubsection{EXOTIC MELEE WEAPON: [FIELD]} \textbf{Type:} Field, Active, Combat \\ \textbf{Linked Aptitude:} SOM \\ \textbf{What it is:} The Exotic Melee Weapon skill covers the use and maintenance of all melee weapons not covered by the Clubs or Blades skills (see p. 334). \\ \textbf{When you use it:} Use the Exotic Melee Weapon skill when attacking someone with an exotic melee weapon in melee combat (p. 191). \\ \textbf{Sample Fields:} Morning Star, Spear, Whip \\ \textbf{Specializations:} N/A 

\subsubsection{EXOTIC RANGED WEAPON: [FIELD]} \textbf{Type:} Field, Active, Combat \\ \textbf{Linked Aptitude:} COO \\ \textbf{What it is:} Exotic Ranged Weapon skill includes the use and maintenance of all ranged weapons not covered by the Beam, Flechette, Kinetic, Sonic, or Throwing Weapons skills. \\ \textbf{When you use it:} Use this skill whenever attacking with an exotic ranged weapon in ranged combat (p. 191). \\ \textbf{Sample Fields:} Blowgun, Crossbow, Flamethrower, Slingshot \\ \textbf{Specializations:} N/A 

\subsubsection{FLIGHT} \textbf{Type:} Active, Physical \\ \textbf{Linked Aptitude:} SOM \\ \textbf{What it is:} Flight is the skill of using your body to fly. This skill is used when sleeved in or jamming a winged or otherwise flight-capable morph (manual and remote-control flight are handled using Pilot skill). \\ \textbf{When you use it:} Use this skill whenever you need to make an aerial maneuver, land in difficult conditions, maintain your course in steep winds, or otherwise keep from crashing or falling. \\ \textbf{Specializations:} Diving, Landing, Takeoff, specific maneuvers 

\subsubsection{FRAY} \textbf{Type:} Active, Combat \\ \textbf{Linked Aptitude:} REF \\ \textbf{What it is:} Fray is the ability to get out of the way of incoming attacks, debris, or inconvenient passers-by. Characters that have a high Fray score are able to react quicker than others when dodging or maneuvering. \\ \textbf{When you use it:} Whenever a character is physically attacked by an opponent in melee combat, roll Fray to avoid getting hit (see p. 191). Fray may also be used 

to dodge other events that may harm the character, such as avoiding a charging vehicle or jumping out of the way of a collapsing stack of crates. \\ \textbf{Specializations:} Blades, Clubs, Full Defense, Unarmed 

\subsubsection{FREE FALL} \textbf{Type:} Active, Physical \\ \textbf{Linked Aptitude:} REF \\ \textbf{What it is:} Free Fall is about moving in free-fall and microgravity environments. \\ \textbf{When you use it:} Use whenever you need to maneuver in a zero-g situation, such as propelling yourself across a large open space or making sure you don’t accidentally send yourself spinning off into space. Free Fall is also used when moving with spacesuit maneuvering jets and when parachuting. \\ \textbf{Specializations:} Microgravity, Parachuting, Vacsuits 

\subsubsection{FREERUNNING} \textbf{Type:} Active, Physical \\ \textbf{Linked Aptitude:} SOM \\ \textbf{What it is:} Freerunning is part running, part gymnastics. It is about moving fast, maneuvering over/under/ around/through obstacles, and placing your body where it needs to go. Freerunning/parkour is a popular pastime in habitats where open space is limited. \\ \textbf{When you use it:} Use Freerunning whenever you need to overcome an obstacle via movement, such as hurdling a railing, rolling across the hood of a car, jumping across a pit, or swinging around a pole. Freerunning is also used for sprinting (p. 191) and full defense against attacks (p. 198). \\ \textbf{Specializations:} Balance, Gymnastics, Jumping, Running 

\subsubsection{GUNNERY} \textbf{Type:} Active, Combat \\ \textbf{Linked Aptitude:} INT \\ \textbf{What it is:} Gunnery skill covers the use and maintenance of large, vehicular, or non-portable weapons systems. Firing these weapons is more like playing a video game than firing a gun. \\ \textbf{When you use it:} Use Gunnery when attacking with a vehicle-mounted weapon or weapon emplacement in ranged combat (p. 191). \\ \textbf{Specializations:} Artillery, Missiles 

\subsubsection{HARDWARE: [FIELD]} \textbf{Type:} Field, Active, Technical \\ \textbf{Linked Aptitude:} COG \\ \textbf{What it is:} This skill encompasses the ability to build, repair, physically hack, and upgrade equipment of a specific type. \\ \textbf{When you use it:} Hardware is primarily used to repair devices, vehicles, habitat systems, or synthetic morphs. See Building, Repairing, and Modifying below. \\ \textbf{Sample Fields:} Aerospace (all air and space vehicles), Armorer (armor and weapons), Electronics (all computerized devices), Groundcraft, Implants, Industrial (habitat, factory, and life support systems), Nautical (watercraft and submarines), Robotics (synthetic morphs) \\ \textbf{Specializations:} As appropriate to the field 



\begin{quotation} \textbf{Building} \\ Creating an item from scratch is handled as a Task Action with a timeframe determined by the gamemaster. The timeframe should be set according to the complexity of the object and could range from an hour (constructing a set of shelves) to days (assembling a robot from spare parts) to even months (building a house). Numerous factors may apply modifiers to the test, such as the use of entoptic blueprints/help manuals (+20) or poor working conditions (–10 to –30). Tools are also a factor, perhaps making the job easier (superior tools +10 to +30), more difficult (poor or inadequate tools, –10 to –30), or even impossible (lack of required tools). \end{quotation} 

\begin{quotation} \textbf{Repairing} \\ Damaged items may be repaired in a similar manner. See the rules for Synthmorph and Object Repair, p. 209. \end{quotation} 



\begin{quotation} \textbf{Modifying} \\ Altering an object’s design and function follows the same basic rules as build and repair, above. The time-frame is determined by the gamemaster as appropriate to the modification. \end{quotation} 









\subsubsection{IMPERSONATION} \textbf{Type:} Active, Social \\ \textbf{Linked Aptitude:} SAV \\ \textbf{What it is:} Impersonation is the skill of trying to pass yourself off as someone else in social situations, including virtual ones. This includes copying mannerisms and speech patterns and using accumulated information to convince others that you are that person. In a universe where appearance is highly variable, the question of identity is largely one of both trust and picking up on behavioral quirks and verbal cues to recognize a given individual. \\ \textbf{When you use it:} Sometimes it’s fun to pretend you’re someone else, and sometimes it’s profitable or lifesaving. Use this skill whenever you attempt to convince someone that you are actually someone else through some sort of social or online interaction. Forks use this skill when passing themselves off as their alpha ego. Impersonate is handled as an Opposed Test against the Kinesics skill. \\ \textbf{Specializations:} Avatar, Face-to-Face, Verbal 

\subsubsection{INFILTRATION} \textbf{Type:} Active, Physical \\ \textbf{Linked Aptitude:} COO \\ \textbf{What it is:} Infiltration is the art of escaping detection. \\ \textbf{When you use it:} Use Infiltration whenever you need to physically hide or move with stealth to avoid someone sensing you, whether you are hiding behind a tree, sneaking past a guard, or blending into a crowd. Infiltration can also be used to follow people (shadowing) without them detecting you. Infiltration is an Opposed Test against the Perception of whomever you are hiding from. The gamemaster may wish to roll such tests in secret so the player does not know whether they have succeeded or failed. \\ \textbf{Specializations:} Blending In, Hiding, Shadowing, Sneaking 

\subsubsection{INFOSEC} \textbf{Type:} Active, Technical \\ \textbf{Linked Aptitude:} COG (no defaulting) \\ \textbf{What it is:} Infosec is short for “information security.” It encompasses training in electronic intrusion and counterintrusion techniques, as well as encryption and decryption. \\ \textbf{When you use it:} Infosec is used both for hacking into electronic devices and mesh networks and for protecting them. See the Mesh chapter, p. 234, for more details. \\ \textbf{Specializations:} Brute-Force Hacking, Decryption, Probing, Security, Sniffing, Spoofing 

\subsubsection{INTEREST: [FIELD]} \textbf{Type:} Field, Knowledge \\ \textbf{Linked Aptitude:} COG \\ \textbf{What it is:} Interest includes just about any topic that captures your attention that isn’t covered by another skill. This includes hobbies, obsessions, causes, pastimes, and other recreational pursuits. \\ \textbf{When you use it:} Use the Interest skill whenever you need to recall or use knowledge related to the particular interest in question. Field Examples: Ancient Sports, Celebrity Gossip, Conspiracies, Factor Trivia, Gambling, Hypercorp Politics, Lunar Habitats, Martian Beers, Old Earth Nation-States, Reclaimer Blogs, Science Fiction, Scum Drug Dealers, Spaceship Models, Triad Economics, Underground XP \\ \textbf{Specializations:} As appropriate to the field 

\subsubsection{INTERFACING} \textbf{Type:} Active, Technical \\ \textbf{Linked Aptitude:} COG \\ \textbf{What it is:} Interfacing is about using computerized electronic devices and software. \\ \textbf{When you use it:} Use Interfacing to understand an electronic device you are not familiar with, use a program according to its normal operating parameters, manipulate electronic files of various types (including images, video, XP, and audio files), scan for wireless devices, and otherwise interact with and command your ecto, muse, and other computerized devices. Some Interfacing actions may be Task Actions, with a timeframe determined by the gamemaster. For more detail, see the Mesh chapter, p. 234. \\ \textbf{Specializations:} Forgery, Scanning, Stealthing, by program 

\subsubsection{INTIMIDATION} \textbf{Type:} Active, Social \\ \textbf{Linked Aptitude:} SAV \\ \textbf{What it is:} Intimidation is convincing someone to do what you want based on direct threats (implied or actual) or sheer force of personality. \\ \textbf{When you use it:} Use Intimidation to scare someone into submission, browbeat them into getting your way, command them to follow your orders, or berate them into giving up information. Influence is handled as an Opposed Test, pitted against the target’s WIL + WIL + SAV. \\ \textbf{Specializations:} Interrogation, Physical, Verbal 

\subsubsection{INVESTIGATION} \textbf{Type:} Active, Mental \\ \textbf{Linked Aptitude:} INT \\ \textbf{What it is:} Investigation is the art of analyzing evidence, piecing together clues, solving mysteries, and making logical deductions from groups of facts. Investigation differs from Perception in that it is the conscious search for clues or pieces of a puzzle. \\ \textbf{When you use it:} Use Investigation to draw conclusions from assorted details. For example, Investigation could be used to determine the likely sequence of events at a crime scene, determine a possible social connection between two people, or deduce how an enemy made their escape. Investigation is a great way 

to provide clues to players, especially when the subject matter is something their character might know well but the player does not. \\ \textbf{Specializations:} Evidence Analysis, Logical Deductions, Physical Investigation, Physical Tracking 

\subsubsection{KINESICS} \textbf{Type:} Active, Social \\ \textbf{Linked Aptitude:} SAV \\ \textbf{What it is:} Kinesics is the art of empathy and non-vocal communication. \\ \textbf{When you use it:} Use Kinesics to read body language, tells, social cues, and other subconscious indicators. It can also be used to emote more effectively. Kinesics is used defensively whenever someone is trying to deceive you; make an Opposed Test against that person’s Deception or Impersonation skill. Though synthetic morphs are also designed to emote, reading them is not as easy. Apply a –30 modifier when judging a synthetic morph inhabited by a character or AGI. Likewise, standard AIs are also difficult to read; apply a –60 modifier when judging a synthetic morph or pod operated by an AI. \\ \textbf{Specializations:} Judge Intent, Nonvocal Communication 

\begin{quotation} \textbf{JUDGING EMOTIONS AND INTENTIONS} \\ Transhumans are empathic beings, and so you can attempt to gauge the demeanor and/or intent of someone you are dealing with by rolling a Kinesics Test. This attempt to read someone is far from exact, however, and it is easy to misjudge. The gamemaster should make this test in secret and only allow a hint if successful—it is not possible to read someone with absolute certainty. If the person being judged is intentionally trying to deceive the character, this should be an Opposed Test against their Deception skill. \end{quotation} 

\begin{quotation} \textbf{NONVOCAL COMMUNICATION} \\ Experts in Kinesics can effectively communicate with each other simply by posture, stances, gestures, demeanors, and looks. Such communication is necessarily limited in the amount of information it can convey, but feelings, attitudes, affirmation/negation, and simple concepts may be passed. To effectively communicate complex concepts, the gamemaster may require successful Kinesics Tests from both parties, applying modifiers as appropriate. \end{quotation} 

\subsubsection{KINETIC WEAPONS} \textbf{Type:} Active, Combat \\ \textbf{Linked Aptitude:} COO \\ \textbf{What it is:} Kinetic Weapons covers the use and maintenance of standard kinetic projectile weapons like firearms and railguns (p. 335). \\ \textbf{When you use it:} Use this skill whenever attacking with a kinetic weapon in ranged combat (p. 191). \\ \textbf{Specializations:} Assault Rifles, Machine Guns, Pistols, Sniper Rifles, Submachine Guns 



\subsubsection{LANGUAGE: [FIELD]} \textbf{Type:} Field, Knowledge \\ \textbf{Linked Aptitude:} INT \\ \textbf{What it is:} Language covers the speaking and reading of languages other than the player’s native tongue. A speaker is considered fluent at a skill level of 50; anything above this indicates further refinement in technical vocabulary, accents, and knowledge of dialects. \\ \textbf{When you use it:} Use the Language skill whenever you want to speak, understand, or read something in a language at which you are skilled. Most speaking and reading comprehension tests can be considered Simple Success Tests if your skill is over 50, unless the gamemaster rules the subject is sufficiently complex that a non-native speaker would have trouble understanding it. \\ \textbf{Sample Fields:} Arabic, Cantonese, English, French, Hindi, Japanese, Mandarin, Portuguese, Russian, Spanish \\ \textbf{Specializations:} As appropriate to the field, representing dialects, technical jargon, and subcultural slang 

\begin{quotation} \textbf{LANGUAGES IN ECLIPSE PHASE} \\ With the Fall of Earth, the languages that remain most prominent in the solar system are those that were extensively carried into space by countries and hypercorps with aggressive space programs or by the large populations of poor laborers and infomorph refugees that followed. No single language dominated the realm of space expansion, and multilingualism was common. Many habitats and (sub) cultural groupings cling to specific languages as a method of retaining cultural identity. Despite the availability of instant translation via the mesh, many people remain versed in two or more languages. The ten most common languages in the solar system by speaking populations are: Arabic, Cantonese, English, French, Hindi, Japanese, Mandarin, Portuguese, Russian, and Spanish. Other languages that remain strong include Bengali, Dutch, Farsi, German, Italian, Javanese, Korean, Polish, Punjabi, Swedish, Tamil, Turkish, Urdu, Vietnamese, and Wu. Some languages were effectively lost during the Fall, especially those in some undeveloped regions, as their speaking populations did not migrate into space pre-Fall and were not privileged enough to survive in large numbers as infomorph refugees. \end{quotation} 





\subsubsection{MEDICINE: [FIELD]} \textbf{Type:} Field, Active, Technical \\ \textbf{Linked Aptitude:} COG \\ \textbf{What it is:} Medicine is the applied care and maintenance of biological beings and life. \\ \textbf{When you use it:} Use Medicine whenever you need to apply medical care beyond the immediate help provided by first responders. This includes conducting physical exams, diagnosing ailments, treating problems and illnesses, surgery, using biotech and nanotech medical tools, and long-term care. See Healing and Repair, p. 208. \\ \textbf{Sample Fields:} Biosculpting, Exotic Biomorphs, Gene Therapy, General Practice, Implant Surgery, Nanomedicine, Mercurials (by type), Paramedic, Pods, Psychiatry, Remote Surgery, Trauma Surgery, Veterinary \\ \textbf{Specializations:} As appropriate to the field 

\subsubsection{NAVIGATION} \textbf{Type:} Active, Mental \\ \textbf{Linked Aptitude:} INT \\ \textbf{What it is:} Navigation is the art of finding your way, whether using AR maps, a compass, the stars, or an astrogation AI. \\ \textbf{When you use it:} Use Navigation whenever you need to plot out a course, determine a direction, or otherwise keep from getting lost. \\ \textbf{Specializations:} Astrogation, Map Making, Map Reading 

\subsubsection{NETWORKING: [FIELD]} \textbf{Type:} Active, Social \\ \textbf{Linked Aptitude:} SAV \\ \textbf{What it is:} Networking is your skill at working your contacts, trading favors, and keeping your finger on the pulse of a particular faction or cultural grouping. \\ \textbf{When you use it:} Use Networking to gather information or call on services using your Reputation (see Reputation and Social Networks, p. 285). \\ \textbf{Sample Fields:} Autonomists (@-rep), Criminals (g-rep), Ecologists (e-rep), Firewall (i-rep), Hypercorps (c-rep), Media (f-rep), Scientists (r-rep). At the game-master’s discretion, this list can be expanded to other (sub)cultural groupings. \\ \textbf{Specializations:} As appropriate to each field 

\subsubsection{PALMING} \textbf{Type:} Active, Physical \\ \textbf{Linked Aptitude:} COO \\ \textbf{What it is:} Palming is the skill of handling items quickly and nimbly without others noticing. Palming is not only about dexterous manipulation of objects but also relies heavily on obfuscation, timing, and misdirection. \\ \textbf{When you use it:} Use Palming any time you are trying to conceal an item on your person, shoplift, pick a pocket, surreptitiously discard something, or perform a magic trick. Palming is an Opposed Test against the Perception of any onlookers. The game-master may wish to make this roll secretly. \\ \textbf{Specializations:} Pickpocketing, Shoplifting, Tricks 

\subsubsection{PERCEPTION} \textbf{Type:} Active, Mental \\ \textbf{Linked Aptitude:} INT \\ \textbf{What it is:} Perception is the use of your physical senses (including cybernetic) and awareness of the physical world around you. Perception differs from Investigation in that it is noticing things by chance, rather than actively searching for something. \\ \textbf{When you use it:} Use Perception whenever you wanted to take a detailed account of your surroundtion) or by lying (covered by Deception). \\ \textbf{When you use it:} Use Persuasion any time you are trying to bargain with, convince, or manipulate someone. This can include motivating your subordinates or peers to take action, seducing a companion, winning a political debate, or negotiating a contract, among other things. Persuasion is handled as an Opposed Test against the target’s WIL + WIL + SAV when one person is simply trying to win over another. If both parties are trying to convince each other, make it an Opposed Test between Persuasion skills. \\ \textbf{Specializations:} Diplomacy, Morale Boosting, Negotiating, Seduction 

\subsubsection{PILOT: [FIELD]} \textbf{Type:} Field, Active, Vehicle \\ \textbf{Linked Aptitude:} REF \\ \textbf{What it is:} Pilot is your skill at driving/flying a vehicle of a particular type. \\ \textbf{When you use it:} You use Pilot skill whenever you need to maneuver, control, or avoid crashing a vehicle, whether you are in the pilot’s seat, remote controlling a robot, or directly jamming a vehicle with VR. Each vehicle has a Handling modifier that applies to this test, along with other situational modifiers (see Bots, Synthmorphs, and Vehicles, p. 195). \\ \textbf{Sample Fields:} Aircraft, Anthroform (walkers), Exotic Vehicle, Groundcraft (wheeled or tracked), Spacecraft, Watercraft \\ \textbf{Specializations:} As appropriate to the field 

\subsubsection{PROFESSION: [FIELD]} \textbf{Type:} Field, Knowledge \\ \textbf{Linked Aptitude:} COG \\ \textbf{What it is:} Profession skills indicate training in a profession practiced in Eclipse Phase. This can indicate either formal training or informal, on-the-job type training, and includes both legal and extralegal trades. \\ \textbf{When you use it:} Use Profession to perform work-related tasks for a specific trade (i.e. mining, balancing accounts, designing a security system, etc.) or to reference specialized knowledge that someone trained in that profession might have. \\ \textbf{Sample Fields:} Accounting, Appraisal, Asteroid Prospecting, Banking, Cool Hunting, Con Schemes, Distribution, Forensics, Lab Technician, Mining, Police Procedures, Psychotherapy, Security Ops, Smuggling Tricks, Social Engineering, Squad Tactics, Viral Marketing, XP Production \\ \textbf{Specializations:} As appropriate to the field 

\subsubsection{PROGRAMMING} \textbf{Type:} Active, Technical \\ \textbf{Linked Aptitude:} COG (no defaulting) \\ \textbf{What it is:} Programming is your talent at writing and modifying software code. \\ \textbf{When you use it:} Use Programming to write new programs, modify or patch existing software, break copy protection, find or introduce exploitable flaws, write virii or worms, design virtual settings, and so on. See the Mesh chapter, p. 234. Programming is also applied when using nanofabrication devices. \\ \textbf{Specializations:} AI Code, Malware, Nanofabrication, Piracy, Simulspace Code 

\subsubsection{NANOFABRICATION} Nanofabrication is use of Programming skill to create objects using a cornucopia machine, fabber, or maker (p. 327). If you have appropriate blueprints and raw materials, most uses of a nanofabricator can be treated as a Simple Success Test (p. 118). If you wish to create an item for which you do not have blueprints or the proper raw materials, however, or you wish to alter an item’s design, then a Nanofabrication Test is called for. See Nanofabrication, p. 284. \\ \textbf{Specializations:} Art, Clothing, Electronics, Food, Forgery, Weapons 

\subsubsection{PROTOCOL} \textbf{Type:} Active, Social \\ \textbf{Linked Aptitude:} SAV \\ \textbf{What it is:} Protocol is the art of making a good impression in social settings. This includes keeping up with the latest memes, trends, gossip, interests and habits of various (sub)cultural group. \\ \textbf{When you use it:} Use Protocol whenever you need to choose your words carefully, determine who is the appropriate person to speak to, impress someone with your grasp of customs, or otherwise fit into a specific social/cultural grouping. Part etiquette, part streetwise, Protocol allows you to navigate treacherous social waters and put people at ease. If the character is dealing with a suspicious or hostile audience, make this an Opposed Test against the target’s WIL + WIL + SAV. \\ \textbf{Specializations:} Anarchist, Brinker, Criminal, Factor, Hypercorp, Infomorph, Mercurial, Reclaimer, Preservationist, Scum, Ultimate 

\subsubsection{NEGATING SOCIAL GAFFES} Sometimes a player will make a mistake that their character never would, whether that’s failing to stand in the presence of hypercorp royalty, confusing a gang leader for a peon, or accidentally insulting someone’s heritage. In cases like this, the player may make a Protocol Test for the appropriate field in order to negate the gaffe. If successful, the character never actually screwed up, or at least managed to cover their tracks without ruffling any feathers. 



\subsubsection{PSI ASSAULT} \textbf{Type:} Active, Mental, Psi \\ \textbf{Linked Aptitude:} WIL \\ \textbf{What it is:} Psi Assault is the skill of damaging another ego’s mind. It can only be purchased by characters with the Psi trait (p. 147). What it does: Use Psi Assault when attacking another ego’s mind in psi combat. \\ \textbf{Specializations:} By sleight 

\subsubsection{PSYCHOSURGERY} \textbf{Type:} Active, Technical \\ \textbf{Linked Aptitude:} INT \\ \textbf{What it is:} Psychosurgery is the use of machine-aided psychological techniques to repair, damage, or manipulate the psyche. \\ \textbf{When you use it:} Use Psychosurgery to attempt the tricky process of editing someone’s mind (see Psychosurgery, p. 229). Psychosurgery can be used beneficially to help patients who remember their deaths, feel disconnected after remorphing, or have experienced other sorts of mental traumas. This skill may also be used to interrogate, torture, or otherwise mess with captive minds in a VR environment. \\ \textbf{Specializations:} Memory Manipulation, Personality Editing, Psychotherapy 

\subsubsection{RESEARCH} \textbf{Type:} Active, Technical \\ \textbf{Linked Aptitude:} COG \\ \textbf{What it is:} Research is the skill for looking up information on the Mesh: searching, sifting, mining, and interpreting data. This includes knowing where to look, what links to follow, and how to optimize your queries. \\ \textbf{When you use it:} Use the Research skill whenever you need to look up the answer to a question, find databases, search archives, or track down anything online. Research is typically a Task Action with the timeframe and difficulty modifier determined by the gamemaster. See the Online Research, p. 249. \\ \textbf{Specializations:} Tracking, by information type 

\subsubsection{SCROUNGING} \textbf{Type:} Active, Mental \\ \textbf{Linked Aptitude:} INT \\ \textbf{What it is:} Scrounging is your ability to find things, particularly things of use or value that are concealed, buried, or hard to find. This includes knowing where to look and what to look for. Scrounging differs from both Perception and Investigation in that it is about finding items hidden among others, and in most cases about finding something in particular (food, valuables, etc.). \\ \textbf{When you use it:} Use Scrounging to dumpster-dive a meal, search ruins for relics, find bargains at a bazaar, forage berries in the forest, locate a spacesuit in an abandoned ship, etc. Scrounging is typically handled as a Task Action with a timeframe and difficulty modifier determined by the gamemaster. \\ \textbf{Specializations:} Bazaars, Forests, Habitats, Ruins 

\subsubsection{SEEKER WEAPONS} \textbf{Type:} Active, Combat \\ \textbf{Linked Aptitude:} COO \\ \textbf{What it is:} Seeker Weapons covers the use and maintenance of seeker launchers (p. 339) and seeker missiles (p. 340). \\ \textbf{When you use it:} Use this skill when attacking with a seeker in ranged combat (p. 191). \\ \textbf{Specializations:} Armband, Pistol, Rifle, Underbarrel 

\subsubsection{SENSE} \textbf{Type:} Active, Mental, Psi \\ \textbf{Linked Aptitude:} INT \\ \textbf{What it is:} Sense is the use of psi to scan egos. Only characters with the Psi trait (p. 147) may purchase this skill. What it does: See Psi, p. 220. \\ \textbf{Specializations:} By sleight 

\subsubsection{SPRAY WEAPONS} \textbf{Type:} Active, Combat \\ \textbf{Linked Aptitude:} COO \\ \textbf{What it is:} The Spray Weapons skill covers the use and maintenance of cone-effect ranged weapons (see Spray Weapons, p. 340). \\ \textbf{When you use it:} A player uses their Sonic Weapons skill whenever they are attacking with a spray weapon in ranged combat (p. 191). \\ \textbf{Specializations:} Buzzer, Freezer, Shard, Shredder, Torch 

\subsubsection{SWIMMING} \textbf{Type:} Active, Physical \\ \textbf{Linked Aptitude:} SOM \\ \textbf{What it is:} Swimming is the art of moving and not drowning within fluids. It includes floating, surface swimming, snorkeling, diving, and related equipment use. \\ \textbf{When you use it:} Use Swimming whenever you need to move and survive in water or another liquid environment. Swimming in a non-threatening environment can be handled as a Simple Success Test. Swimming over a long distance could be handled as a Task Action. Diving off a cliff into a lake, preventing yourself from being swept away in a raging river current, or making sure you’ve set a proper gas mix for a deep-sea dive, among other things, requires a Success Test. \\ \textbf{Specializations:} Diving, Freestyle, Underwater Diving 

\subsubsection{THROWING WEAPONS} \textbf{Type:} Active, Combat \\ \textbf{Linked Aptitude:} COO \\ \textbf{What it is:} Throwing Weapons skill covers the use and maintenance of standard throwing weapons, like grenades (p. 340). \\ \textbf{When you use it:} Use Throwing Weapons skill whenever you are attacking with a throwing weapon in ranged combat (p. 191). \\ \textbf{Specializations:} Grenades, Knives, Rocks 







\subsubsection{UNARMED COMBAT} \textbf{Type:} Active, Combat \\ \textbf{Linked Aptitude:} SOM \\ \textbf{What it is:} Unarmed Combat is your ability to attack and defend without using weapons. \\ \textbf{When you use it:} Use Unarmed Combat whenever you want to attack someone with your fists, feet, elbows, knees, or other body parts in melee combat (p. 191). \\ \textbf{Specializations:} Implant Weaponry, Kick, Punch, Subdual 

\begin{quotation} \textbf{USING KNOWLEDGE SKILLS} \\ At first glance, it may seem that Knowledge skills have fewer in-game applications than Active skills. To some degree this is the case. The importance of Knowledge skills, however, should not be underestimated. While they play a role in analyzing clues and solving mysteries, the real value of Knowledge skills is in helping the characters—and the players—understand the world of Eclipse Phase. In particular these skills can be used to make plans, assess a situation, identify strengths and weaknesses, evaluate worth, make comparisons, forecast probable outcomes, or understand the applicable science, socio-economic factors, or cultural or historical context. For example, a group of characters looking to break into a facility could use Profession: Security Procedures to evaluate the defenses, Academic: Architecture to identify covert points of entry, Interests: Sports to plan their infiltration at a time when the guards are likely to be distracted, Interests: Triads to identify a local crime group that can sell them breaking and entering gear, and Art: Sculpture when picking a valuable art piece with which to bribe an insider. When used appropriately, these skills can be just as beneficial as the Active skills used to break inside, if not more so because the plan is more likely to succeed as a result of this preparation. It is largely up to the gamemaster to enforce how useful Knowledge skills are in their game. The easiest way to reinforce their relevance is to penalize characters who don’t take advantage of them. For example, characters who didn’t use their Profession: Security Procedures in the example above might end up being surprised when they run across a security system they are not prepared to deal with, forcing them to improvise or even abandon their plans. \end{quotation} 

\section{SPECIAL SKILLS} While the preceding list represents the skills most commonly used in Eclipse Phase, there may be certain skills called for in a campaign that are not found in this book. In this case, the gamemaster may work with the players to create a new skill to fill this void. This option should be used sparingly to prevent skill bloat, and all skills are subject to approval by the gamemaster If you choose to create a new skill, keep in mind that it needs to be linked to an existing aptitude and should be a skill that is available to all characters, not just specific to one character. 







































































































































