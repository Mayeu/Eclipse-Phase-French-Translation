



\chapter{Action and combat} \label{chap:action-combat} 

Roleplaying games are about creating drama and adventure, and that usually means action and combat. Action and combat scenes are the moments when the adrenaline really gets pumping and the characters’ lives and missions are on the line. 

<<<<<<< HEAD Combat and action scenarios can be confusing to run, especially if the gamemaster also needs to keep track of the actions of numerous NPCs. For these reasons, it’s important for the gamemaster to detail the action in a way that everyone can visualize, whether that means using a map and miniatures, software, a dry-erase board, or quick sketches on a piece of paper. Though many of the rules for handling action and combat are abstract - allowing room for interpretation and fudging results to fit the story - many tactical factors are also incorporated, so even small details can make a large difference. It also helps to have the capabilities of NPCs predetermined and to run them as a group when possible, to reduce the gamemaster’s burden in the middle of a hectic situation. ======= Combat and action scenarios can be confusing to run, especially if the gamemaster also needs to keep track of the actions of numerous NPCs. For these reasons, it’s important for the gamemaster to detail the action in a way that everyone can visualize, whether that means using a map and miniatures, software, a dry-erase board, or quick sketches on a piece of paper. Though many of the rules for handling action and combat are abstract --- allowing room for interpretation and fudging results to fit the story --- many tactical factors are also incorporated, so even small details can make a large difference. It also helps to have the capabilities of NPCs predetermined and to run them as a group when possible, to reduce the gamemaster’s burden in the middle of a hectic situation. >>>>>>> 069efc938e743c2e1f7048421dcd3e53a984f73e 



\section{Action turns} \label{sec:combat-action-turns} 

Action scenes in Eclipse Phase are handled in bitesize chunks called Action Turns, each approximately 3 seconds in length. We say ``approximately'' because the methodical, step-by-step system used to resolve actions does not necessarily always translate realistically to real life, where people often pause, take breaks to assess the situation, take a breather, and so on. A combat that begins and ends within 5 Action Turns (15 seconds) in Eclipse Phase could last half a minute to several minutes in real life. On the other hand, the characters may be in a situation where their breathing environment decompresses to vacuum in 15 seconds, so every second may in fact count. As a rule, gamemasters should stick with 3 seconds per turn, but they shouldn’t be afraid to fudge the timing either when a situation calls for it. 

Action Turns are meant to be utilized for combat and other situations where timing and the order in which people act is important. If it is not necessary to keep track of who’s doing what so minutely, you can drop out of Action Turns and return to ``regular'' free form game time. 

Each Action Turn is in turn broken down into distinct stages, described below 



\subsection{Step 1: Roll initiative} \label{sec:roll-initiative} 

At the beginning of every Action Turn, each PLAYER involved in the scene rolls Initiative to determine the order in which each character acts. For more details, see \emph{Initiative}. 



\subsection{Step 2: Begin first action phase} \label{sec:begin-first-phase} 

Once Initiative is rolled, the first Action Phase begins. Everyone gets to act in the first Action Phase (since everyone has a minimum Speed of 1), unless they happen to be unconscious/dead/disabled, starting with the character with the highest successful Initiative roll. 



\subsection{Step 3: Declare and resolve actions} \label{sec:declare-resolve} 

<<<<<<< HEAD The character going first now declares and resolves the actions they will take during this first Action Phase. Since some actions the character makes may depend on the outcome of others, there is no need to declare them all first - they may be announced and handled one at a time. ======= The character going first now declares and resolves the actions they will take during this first Action Phase. Since some actions the character makes may depend on the outcome of others, there is no need to declare them all first --- they may be announced and handled one at a time. >>>>>>> 069efc938e743c2e1f7048421dcd3e53a984f73e 

As described under Actions (p. 189), each character may perform a varying number of Quick Actions and/or a single Complex Action during their turn. Alternately, a character may begin or continue with a Task Action, or delay their action pending other developments (see Delayed Actions, p. 189). 

A character who has delayed their action may interrupt another character at any point during this stage. That interrupting character must complete this stage in full, then the action returns to the interrupted character to finish the rest of their stage. 



\subsection{Step 4: Rotate and repeat} \label{sec:rotate-repeat} 

Once the character has resolved their actions for that phase, the next character in the Initiative order gets to go, running through Step 3 for themselves. 

If every character has completed their actions for that phase, return to Step 2 and go the second Action Phase. Every character with a Speed of 2 or more gets to go through Step 3 again, in the same Initiative order (modified by wound modifiers). Once the second Action Phase is completed, return to Step 2 for the 3rd Action Phase, where every character with a Speed of 3 or more gets to go for a third time. Finally, after everyone eligible to go in the 3rd Action Phase has gone, return to Step 2 for a fourth and last Action Phase, where every character with a Speed of 4 can act for one final time. 

At the end of the fourth Action Phase, return to Step 1 and roll Initiative again for the next Action Turn. 



\section{Initiative} \label{sec:initiative} 

<<<<<<< HEAD Timing in an Action Turn can be critical - it may mean life or death for a character who needs to get behind cover before an opponent draws and fires their gun. The process of rolling Initiative determines if a character acts before or after another character. ======= Timing in an Action Turn can be critical --- it may mean life or death for a character who needs to get behind cover before an opponent draws and fires their gun. The process of rolling Initiative determines if a character acts before or after another character. >>>>>>> 069efc938e743c2e1f7048421dcd3e53a984f73e 



\subsection{Initiative order} \label{sec:initiative-order} 

A character’s Initiative stat is equal to their Intuition + Reflexes aptitudes multiplied by 2. This score may be further modified by morph type, implants, drugs, psi, or wounds. 

In the first step of each Action Turn, every character makes an Initiative Test, rolling d100 and adding their Initiative stat. Whoever rolls highest goes first, followed by the other characters in descending order, highest to lowest. In the event of a tie, characters go simultaneously. 

\end{quotation} Adam, Bob, and Cami are rolling Initiative. Adam’s Initiative stat is 80, Bob’s is 110, and Cami’s is 60. Adam rolls a 38, Bob rolls a 24, and Cami rolls a 76. Adam’s total Initiative score is 118 (80 + 38), Bob’s is 134 (110 + 24), and Cami’s is 136 (60 + 76). Cami rolled highest, so she goes first, followed by Bob and then Adam. If Cami \& Bob had tied, they would both go at the same time. \end{quotation} 

\subsubsection{Initiative and damage} 

Characters who are suffering from wounds have their Initiative score temporarily reduced (see Wounds, p. 207). This modifier is applied immediately when the wound is taken, which means that it may modify an Initiative score in the middle of an Action Turn. If a character is wounded before they go in that Action Phase, their Initiative is reduced accordingly, which may mean they now go after someone they were previously ahead of in the Initiative order. 

\end{quotation} Before Bob’s Action Phase comes up, Bob takes two wounds, knocking his Initiative down from 134 to 114. This means that Adam, with an Initiative of 118, now goes before him. \end{quotation} 

\subsubsection{Initiative, Moxie and Criticals} 

A character may spend a point of Moxie to go first in an Action Phase, regardless of their Initiative roll (see Moxie, p. 122). If more than one character chooses this option, then order is determined as normal first among those who spent Moxie, followed by those who didn’t. 

Similarly, any character that rolls a critical on Initiative automatically goes first, even before someone who spent Moxie. If more that two characters rolled criticals, determine order between them as normal. 

\subsection{Simplifying initiative} \label{sec:simplifying-init} 

For speedier resolution, simply have characters roll Initiative once for an entire scene. That Initiative result stays with them on each Action Turn until the combat or scenario is over. Likewise, ignore Initiative modifiers from wounds. 

\subsection{Speed} \label{sec:speed} 

Speed determines how many times a character can act during an Action turn. Every character starts with a default Speed stat of 1, meaning they can act in the first Action Phase of the turn only. Certain morphs, implants, drugs, psi, and other factors may cumulatively increase their Speed to 2, 3, or even 4 (the maximum), allowing them to act in further Action Phases as well. For example, a character with Speed 2 can act in the first and second Action Phases, and a character with Speed 3 can act in the first through third Action Phases. A character with Speed 4 is able to act in every Action Phase. This represents the character’s enhanced reflexes and neurology, allowing them to think and act much faster than nonenhanced characters. 

<<<<<<< HEAD If a character’s Speed does not allow them to act during an Action Phase, they can initiate no actions during the pass - they must simply bide their time. The character may still defend themself, however, and any automatic actions remain ``on.'' Note that any movement the character initiated is considered to still be underway even during the Action Phases they do not participate in (see \emph{Movement}, p. 190). ======= If a character’s Speed does not allow them to act during an Action Phase, they can initiate no actions during the pass --- they must simply bide their time. The character may still defend themself, however, and any automatic actions remain ``on.'' Note that any movement the character initiated is considered to still be underway even during the Action Phases they do not participate in (see \emph{Movement}, p. 190). >>>>>>> 069efc938e743c2e1f7048421dcd3e53a984f73e 



\subsection{Delayed actions} \label{sec:delayed-actions} 

When it’s your turn to go during an Action Phase, you may decide that you’re not ready to act yet. You may be awaiting the outcome of another character’s actions, hoping to interrupt someone else’s action, or may simply be undecided about what to do yet. In this case, you may opt to delay your action. 

<<<<<<< HEAD When you delay your action, you’re putting yourself on standby. At some later point in that Action Phase, you can announce that you are now taking your action - even if you interrupt another character’s action. In this case, all other activity is put on hold until your action is resolved. Once your action has taken place, the Initiative order continues on where you interrupted. ======= When you delay your action, you’re putting yourself on standby. At some later point in that Action Phase, you can announce that you are now taking your action --- even if you interrupt another character’s action. In this case, all other activity is put on hold until your action is resolved. Once your action has taken place, the Initiative order continues on where you interrupted. >>>>>>> 069efc938e743c2e1f7048421dcd3e53a984f73e 

You may delay your action into the next Action Phase, or even the next Action Turn, but if you do not take it by the time your next action comes around in the Initiative order, then you lose it. Additionally, if you do delay your action into another phase or turn, then once you take it you lose any action you might have in that Action Phase. 



\section{Actions} \label{sec:actions} 

<<<<<<< HEAD When it’s your turn to act during an Action Phase, you have many options for what you can do - far too many to list here. There is a limit to what you can accomplish in 3 seconds, however, so some limitations must be adhered to. The first step is to figure out what type of action you want to take. In Eclipse Phase, actions are categorized as Automatic, Quick, Complex, or Task, based on how much time and effort they entail. ======= When it’s your turn to act during an Action Phase, you have many options for what you can do --- far too many to list here. There is a limit to what you can accomplish in 3 seconds, however, so some limitations must be adhered to. The first step is to figure out what type of action you want to take. In Eclipse Phase, actions are categorized as Automatic, Quick, Complex, or Task, based on how much time and effort they entail. >>>>>>> 069efc938e743c2e1f7048421dcd3e53a984f73e 



\subsection{Automatic actions} \label{sec:combat-automatic-actions} 

<<<<<<< HEAD Automatic Actions require no effort. These are abilities or activities that are ``always on'' (assuming you are conscious) or are otherwise reflexive (they happen automatically in response to certain conditions, with no effort from you). Breathing, for example, is an automatic action - your body does it without conscious effort or thinking on your part. 

In most cases, Automatic Actions are not something that you initiate - they are always active, or at least on standby. Certain circumstances, however, will bring an Automatic Action to bear. Such Automatic Actions are invoked and handled immediately whenever they apply, without requiring effort from your character. 

\subsubsection{Resistance} 

Resisting damage - whether from combat, a poison, or a psi attack - is one example of an Automatic Action that occurs in response to something else. 

\subsubsection{Basic perception} 

Your senses are continuously active, accumulating data on the world around you. Basic perception is considered an Automatic Action, and so the gamemaster can call on you to make a Perception Test whenever you receive sensory input that your brain might want to take notice of (see Perception, p. 182). Likewise, you may ask the gamemaster at any time - even during other character’s actions - to make a basic Perception Test, just to find out what your character is noticing around them. 

Because basic perception is an automatic, subconscious activity, however, you will suffer a -20 modifier for distraction - your attention is focused elsewhere. In order to avoid the distraction modifier, you must actively engage in detailed perception or use an oracle implant (p. 308). ======= Automatic Actions require no effort. These are abilities or activities that are ``always on'' (assuming you are conscious) or are otherwise reflexive (they happen automatically in response to certain conditions, with no effort from you). Breathing, for example, is an automatic action --- your body does it without conscious effort or thinking on your part. 

In most cases, Automatic Actions are not something that you initiate --- they are always active, or at least on standby. Certain circumstances, however, will bring an Automatic Action to bear. Such Automatic Actions are invoked and handled immediately whenever they apply, without requiring effort from your character. 

\subsubsection{Resistance} 

Resisting damage --- whether from combat, a poison, or a psi attack --- is one example of an Automatic Action that occurs in response to something else. 

\subsubsection{Basic perception} 

Your senses are continuously active, accumulating data on the world around you. Basic perception is considered an Automatic Action, and so the gamemaster can call on you to make a Perception Test whenever you receive sensory input that your brain might want to take notice of (see Perception, p. 182). Likewise, you may ask the gamemaster at any time --- even during other character’s actions --- to make a basic Perception Test, just to find out what your character is noticing around them. 

Because basic perception is an automatic, subconscious activity, however, you will suffer a -20 modifier for distraction --- your attention is focused elsewhere. In order to avoid the distraction modifier, you must actively engage in detailed perception or use an oracle implant (p. 308). >>>>>>> 069efc938e743c2e1f7048421dcd3e53a984f73e 



\subsection{Quick actions} \label{sec:combat-quick-actions} 

Quick Actions are fast and simple, and they may often be multi-tasked. They require minimal thought and effort. You may undertake multiple Quick Actions on your turn during each Action Phase, limited only by the gamemaster’s judgment. If you are taking nothing but Quick Actions during an Action Phase, you should be allowed a minimum of 3 separate Quick Actions. If you are also engaging in a Complex or Task Action during that same Action Phase, you should be allowed a minimum of 1 Quick Action. Ultimately, the gamemaster decides what activity you can or can’t fit into a single Action Phase. 

Some examples of Quick Actions include: talking, switching a safety, activating an implant, standing up, dropping prone, gesturing, drawing/readying a weapon, handling an object, or using a simple object. 



\subsubsection{Aiming} 

Aiming is a special case in that it is a Quick Action but requires a degree of concentration that rules out other minor actions. If you wish to aim before making an attack in the same Action Phase, aiming is the only Quick Action you may make during that Action Phase (see Aimed Shots, p. 193). 

\subsubsection{Detailed perception} 

Detailed perception involves taking a moment to actively use your senses in search of information and assess what you are perceiving (see Perception, p. 182). It requires slightly more effort and brainpower (or computer power) than basic perception, which is automatic. As a Quick Action, you may only engage in detailed perception on your turn during an Action Phase, but you do not suffer a modifier for distraction (unless you happen to be in a heavily distracting environment, such as a gunfight or agitated crowd). 



\subsection{Complex actions} \label{sec:combat-complex-actions} 

<<<<<<< HEAD Complex Actions require more concentration and effort than Quick Actions - they effectively monopolize your attention. You may only take one Complex Action on each your Action Phase turns. Additionally, you may not engage in a Complex Action and a Task Action during the same Action Phase. ======= Complex Actions require more concentration and effort than Quick Actions --- they effectively monopolize your attention. You may only take one Complex Action on each your Action Phase turns. Additionally, you may not engage in a Complex Action and a Task Action during the same Action Phase. >>>>>>> 069efc938e743c2e1f7048421dcd3e53a984f73e 

Examples of Complex Actions include: attacking, shooting, acrobatics, full defense, disarming a bomb, using a complex device, or reloading a weapon. 



\subsection{Task actions} \label{sec:combat-task-actions} 

A Task Action is any activity that requires longer than one Action Turn to complete. Each Task Action lists a timeframe for how long the task takes to accomplish. This timeframe may range anywhere from 2 Action Turns to 2 years. While engaged in a Task Action, you may not also undertake a Complex Action, though in some cases you may take a break from the task and return to it later. For more information, see Task Actions, p. 120. 

Examples of Task Actions include: repairing a device, programming, conducting a scientific analysis, searching a room, climbing a wall, or cooking a meal. 



\section{Movement} \label{sec:combat-movement} 

Movement in \emph{Eclipse Phase} is handled just like any other action, and may change from Action Phase to Action Phase. Walking and running both count as Quick Actions, as they do not require your full concentration. The same also applies to slithering, crawling, floating, hovering, or gliding. Running, however, may inflict a -10 modifier on other actions that are affected by your jostling movement. Even more, sprinting is an all-out run, and so requires a Complex Action (see Sprinting, p. 191). 

At the gamemasters discretion, other movement may also call for a Complex Action. Hurdling a fence, pole vaulting, jumping from a height, swimming, or freerunning through a habitat in zero-gravity all require a bit of finesse and attention to detail, so would count as a Complex Action, and would apply the same modifier as running. Flying generally counts as a Quick Action, though intricate maneuvers would call for a Complex Action. 



\subsection{Movement rates} \label{sec:movement-rates} 

Sometimes it’s important to know not just how you’re moving, but how far. For most of transhumanity, this movement rate is the same: 4 meters per Action Turn walking, 20 meters per turn running. To determine how far a character can move in a particular Action Phase, divide this movement rate by the total number of Action Phases in that turn. In a turn with 4 Action Phases, that breaks down to 1 meter walking per Action Phase, 5 meters running. 

Movement such as swimming or crawling benchmarks at about 1 meter per Action Turn, or 0.25 meters per Action Phase. You can also sprint to increase your movement rate (see Sprinting). Vehicles, robots, creatures, and unusual morphs will have individual movement rates listed in the format of walking rate/running rate in meters per turn. 

These movement rates assume standard Earth gravity of course. If you’re moving in a low-gravity, microgravity, or high-gravity environment, things change. See \emph{Gravity}, p. 198. 



\subsubsection{Jumping} 

Characters making a running jump can cross SOM $\div$ 5 (round up) meters; use SOM $\div$ 20 (round up) meters for standing jumps. Vertical jumping height is 1 meter. Characters making a Freerunning Test can increase jumping distance by 1 meter (running jump) or 0.25 meters (standing/vertical jumps) per 10 points of MoS. 

\subsubsection{Sprinting} 

You may use Freerunning to increase the distance you move during an Action Phase. You must spend a Complex Action to sprint and make a Freerunning Test. Every 10 points of MoS increases your running distance in that Action Phase by 1 meter, to a maximum bonus of +5 meters. 



\section{Combat} \label{sec:combat} 

Sometimes words fail, and that’s when the knives and shredders come out. All combat in Eclipse Phase is conducted using the same basic mechanics, whether it’s conducted with claws, fists, weapons, guns, or psi: an Opposed Test between the attacker and defender(s). 

\subsection{Resolving combat} 

Use the following sequence of steps to determine the outcome of an attack. 

\subsubsection{Step 1: Declare an attack} 

The attacker initiates by taking a Complex Action to attack on their turn during an Action Phase. The skill employed depends on the method used to attack. If the character lacks the appropriate Combat skill, they must default to the appropriate linked aptitude. 

\subsubsection{Step 2: Declare defense} 

Once the attack is declared, the defender chooses how to respond. Defense is always considered an Automatic Action unless the defender is surprised (see Surprise, p. 204) or somehow incapacitated and incapable of defending themself. 

\textbf{Melee:} A character defending against melee attacks uses Fray skill, representing dodging (if the character lacks this skill, they may default to Reflexes). Alternately, the character may use a melee combat skill to defend, representing blocks and parries rather than dodging. 

\textbf{Ranged:} Against ranged attacks, a defending character may only use half their Fray skill (round down). 

\textbf{Full Defense:} Characters who have taken a Complex Action to go on full defense (p. 198) receive a +30 modifier to their defensive roll. 

<<<<<<< HEAD \textbf{Psi:} A character defending against a psi attack rolls WIL x 2 (p. 222). A mental sort of full defense may also be rallied against psi attacks. ======= \textbf{Psi:} A character defending against a psi attack rolls WIL $\times$ 2 (p. 222). A mental sort of full defense may also be rallied against psi attacks. >>>>>>> 069efc938e743c2e1f7048421dcd3e53a984f73e 

\subsubsection{Step 3: Apply modifiers} 

Any appropriate modifiers are now applied to the attacker and defender’s skills. See the Combat Modifiers table (p. 193) for common situational modifiers. 

\subsubsection{Step 4: Make the opposed text} 

The attacker and defender both roll d100 and compare the results to their modified skill target numbers. 

\subsubsection{Step 5: Determine outcome} 

If the attacker succeeds and the defender fails, the attack hits. If the attacker fails, the attack misses completely. 

If both attacker and defender succeed in their tests, compare their dice rolls. If the attacker’s dice roll is higher, the attack hits despite a spirited defense; otherwise, the attack fails to connect. 

\textbf{Excellent Success:} If the attacker rolled an Excellent Success (MoS of 30+), a solid hit is struck. Increase the Damage Value (DV) inflicted by +5. If the MoS is 60+, increase the DV by +10. 

<<<<<<< HEAD \textbf{Criticals:} If the attacker rolls a critical success, the attack is armor-defeating, meaning that the defender’s armor is bypassed completely - some kink or flaw was exploited, allowing the attack to get through completely. ======= \textbf{Criticals:} If the attacker rolls a critical success, the attack is armor-defeating, meaning that the defender’s armor is bypassed completely --- some kink or flaw was exploited, allowing the attack to get through completely. >>>>>>> 069efc938e743c2e1f7048421dcd3e53a984f73e 

If the defender rolls a critical success, they dodge with flair, reach cover that protects from follow-up attacks, maneuver to a superior position, or otherwise benefit. 

\subsubsection{Step 6: Modify armor} 

If the target is hit, their armor will help to protect them against the attack (unless the attacker rolled a critical, see above). Determine which type of armor is appropriate to defending against that particular attack (see Armor, p. 194). The attack’s Armor Penetration (AP) value reduces the armor’s rating, however, representing the weapon’s ability to pierce through protective measures. 

\subsubsection{Step 7: Determine damage} 

Every weapon and type of attack has a Damage Value (DV, see p. 207). This amount is reduced by the target’s AP-modified armor rating. If the damage is reduced to 0 or less, the armor is effective and the attack fails to injure the target. Otherwise, any remaining damage is applied to the defender. If the accumulated damage exceeds the defender’s Durability, they are incapacitated and may die (see Durability and Health, p. 207). 

Note that some psi attacks inflict mental stress rather than physical damage (see Mental Health, p. 209). In this case, the Stress Value (SV) is handled the same as DV. 

\subsubsection{Step 8: Determine wounds} 

The damage inflicted from a single attack is then compared to the victim’s Wound Threshold. If the armormodified DV equals or exceeds the Wound Threshold, the character suffers a wound. Multiple wounds may be applied with a single attack if the modified DV is two or more factors beyond the Wound Threshold. Wounds represent more serious injuries and apply modifiers and other effects to the character (see Wounds, p. 207). 

\begin{quotation} Stoya tried to get off the station quickly, but the Night Cartel’s assassin caught up, surprising her in a microgravity part of the habitat. The assassin’s INIT is 63, plus a dice roll of 23, for an Initiative of 86. Stoya’s INIT is 55, plus a roll of 27, for an Initiative of 82. 

The assassin goes first, spending a Quick Action to draw a shredder. This flechette weapon is in burst-fire mode, so with a Complex Action the assassin can take two shots. His Spray Weapons skill is 65, he’s smartlinked (+10), and they’re at short range (+0), so he needs a 75 or less. Stoya is defending with her Fray skill (60) divided by 2, or 30. 

The assassin rolls an 08 with the first shot. Amazingly, Stoya rolls a 28. They both succeeded, but Stoya rolled higher, so she dodges the first shot. 

The assassin rolls a 20 for his second shot, another hit, and this time Stoya rolls an 83, a failure. The assassin also scored an Excellent Success with a MoS of 55, increasing the DV by +5. 

The assassin’s base damage is 2d10 + 5, but he’s using burst fire against a single target for +1d10, and it’s also a cone effect weapon at short range, for an additional +1d10, for a total DV of 4d10 + 5. The assassin rolls 4d10 and gets 16, then adds the +5 for a total DV of 21. 

<<<<<<< HEAD Stoya’s wearing light body armor (AV 10/10), but the shredder’s Armor penetration is -10, so her armor is entirely negated. She takes a devastating 21 DV, exceeding her Wound Threshold of 10, not just once, but twice. This means Stoya suffers 2 wounds from the shot, suffering -20 to all actions. In addition, she must make two SOM x 3 Tests, one to avoid knockdown and the other to avoid unconsciousness. Her SOM is 30, meaning she needs a 70 (30 x 3 = 90, 90 - 20 wound modifiers = 70) on both rolls. She rolls a 40 and a 27, succeeding both. 

Now it’s Stoya’s action. She takes a Quick Action to pull her own weapon: a stunner. Her Beam Weapons skill is 47, modified by wounds (-20) and a smartlink (+10), for 37. The assassin’s Fray is 48, divided by 2 for 24 against a ranged attack. Stoya rolls a 22 - a critical hit - and the assassin rolls a 68. The stunner only inflicts 1d10 $\div$ 2 DV, but since the attack is a critical hit, this is armor defeating. Stoya rolls an 8, for 4 points of DV, below the assassin’s Wound Threshold of 7. 

Stunners, however, are shock weapons, so the assassin must make a DUR + Energy Armor Test. His DUR is 35 and he’s wearing an armor vest (AV 6/6), so his target number is 41. He rolls a 71 - a Margin of Failure of 30, meaning he is immediately incapacitated for 3 Action Turns. ======= Stoya’s wearing light body armor (AV 10/10), but the shredder’s Armor penetration is -10, so her armor is entirely negated. She takes a devastating 21 DV, exceeding her Wound Threshold of 10, not just once, but twice. This means Stoya suffers 2 wounds from the shot, suffering -20 to all actions. In addition, she must make two SOM $\times$ 3 Tests, one to avoid knockdown and the other to avoid unconsciousness. Her SOM is 30, meaning she needs a 70 (30 $\times$ 3 = 90, 90 $-$ 20 wound modifiers = 70) on both rolls. She rolls a 40 and a 27, succeeding both. 

Now it’s Stoya’s action. She takes a Quick Action to pull her own weapon: a stunner. Her Beam Weapons skill is 47, modified by wounds (-20) and a smartlink (+10), for 37. The assassin’s Fray is 48, divided by 2 for 24 against a ranged attack. Stoya rolls a 22 --- a critical hit --- and the assassin rolls a 68. The stunner only inflicts 1d10 $\div$ 2 DV, but since the attack is a critical hit, this is armor defeating. Stoya rolls an 8, for 4 points of DV, below the assassin’s Wound Threshold of 7. 

Stunners, however, are shock weapons, so the assassin must make a DUR + Energy Armor Test. His DUR is 35 and he’s wearing an armor vest (AV 6/6), so his target number is 41. He rolls a 71 --- a Margin of Failure of 30, meaning he is immediately incapacitated for 3 Action Turns. >>>>>>> 069efc938e743c2e1f7048421dcd3e53a984f73e 

Having disabled her opponent, Stoya takes the time to make a hasty getaway. \end{quotation} 



\subsection{Combat summary} 

\end{itemize} \textbf{Ego stats} \end{quotation} \div Attacker rolls attack skill +/- modifiers. \item Melee: Defender rolls Fray or melee skill +/- modifiers. \item Ranged: Defender rolls (Fray skill $\div$ 2, round down) +/- modifiers. \item If attacker succeeds and rolls higher than the defender, the attack hits. \item Critical hits are armor-defeating (armor does not apply). \item Armor is reduced by the attack’s Armor Penetration value (AP). \item The weapon’s damage is reduced by the target’s modified Armor rating (unless the attack is armor-defeating). \item If the damage exceeds the target’s Wound Threshold, a wound is also scored. (If the damage exceeds the Wound Threshold by multiple factors, multiple wounds are inflicted.) \end{itemize} \end{quotation} 

\begin{table} <<<<<<< HEAD \begin{tabular}{|l|l|} ======= \begin{tabularx}{\hline}{|X|l|} >>>>>>> 069efc938e743c2e1f7048421dcd3e53a984f73e \hline

\hline{2}{|c|}{\textbf{Combat modifiers}}	\\ \hline

\textbf{General} &\textbf{Modifier}	\\ \hline

Character using off-hand	&-20	\\ \hline

Character wounded/traumatized	&-10 per wound/trauma	\\ \hline

Character has superior position	&+20	\\ \hline

Touch-only attack	&+20	\\ \hline

Called shot	&-10	\\ \hline

Character wielding two-handed weapon with one hand &-20	\\ \hline

Small target (child-sized)	&-10	\\ \hline

Very small target (mouse or insect)	&-30	\\ \hline

Large target (car sized) &+10	\\ \hline

Very large target (side of a barn) &+30	\\ \hline

Visibility impaired (minor: glare, light smoke, dim light) &-10	\\ \hline

Visibility impaired (major: heavy smoke, dark) &-20	\\ \hline

Blind attack &-30	\\ \hline

\textbf{Melee Combat} &\textbf{Modifier}	\\ \hline

Character has reach advantage &+10	\\ \hline

Character charging &-10	\\ \hline

Character receiving a charge &+20	\\ \hline

\textbf{Ranged combat (attacker)} &\textbf{Modifier}	\\ \hline

Attacker using smartlink or laser sight	&+10	\\ \hline

Attacker behind cover &-10	\\ \hline

Attacker running &-20	\\ \hline

Attacker in melee combat &-30	\\ \hline

Defender has minor cover &-10	\\ \hline

Defender has moderate cover &-20	\\ \hline

Defender has major cover &-30	\\ \hline

Defender prone and far (10+ meters) &-10	\\ \hline

Defender hidden &-60	\\ \hline

Aimed shot (quick) &+10	\\ \hline

Aimed shot (complex) &+30	\\ \hline

Sweeping fire with beam weapon &+10 on second shot	\\ \hline

Multiple targets in same Action Phase &-20 per additional target \\ \hline

Indirect fire &-30	\\ \hline

Point-blank range (2 meters or less) &+10	\\ \hline

Short range &--	\\ \hline

Medium range &-10	\\ \hline

Long range &-20	\\ \hline

Extreme range &-30	\\ \hline

<<<<<<< HEAD \label{tab:combat-modifiers} ======= \end{tabularx} >>>>>>> 069efc938e743c2e1f7048421dcd3e53a984f73e \label{tab:combat-modifiers} \end{table} 



\section{Action and combat complications} \label{sec:action-combat-comp} 

Combat isn’t quite as simple as deciding if you hit or miss. Weapons, armor, ammunition, and numerous other factors may impact an attack’s outcome. Like- wise, various factors can impact an action scene, such as fire or microgravity effects. 



\subsection{Aimed shots} \label{sec:aimed-shots} 

As noted under Aiming, p. 190, a character can sacrifice their other Quick Actions to concentrate on targeting a ranged attack and receive a +10 modifier on the attack. You can also sacrifice an entire Complex Action to fix your aim on a target. In this case, as long as the target remains in your sights until your next Action Phase, you receive a +30 modifier to hit. 



\subsection{Ammunition and reloading} \label{sec:ammunition-reloading} 

Every weapon has a listed ammunition capacity that indicates how many shots the weapon can carry or holds. When this ammo runs out, a new supply must be loaded in. Players should keep track of the shots they fire. 

Reloading almost always requires a Complex Action, whether you are slapping in a new clip of bullets or a fresh battery for a laser. At the gamemaster’s discretion, a reload that is immediately accessible (such as a new clip reverse-taped to the loaded clip, so that reloading just requires that you reverse the taped clips and slot the new one in) will only take a Quick Action. Archaic weapons such as magazine-fed rifles may require longer to fully load. 



\subsection{Area effect weapons} \label{sec:area-effect-weapons} 

Some ranged attack weapons are designed to affect more than one target at a time. These weapons fall into three categories: blast, uniform blast, and cone. 

\subsubsection{Blast effect} 

Blast weapons include items like grenades, mines, and other explosives that expand outward from a central detonation point. Most blast attacks expand outward in a sphere, though certain shaped charges may direct an explosion in one direction only. The explosive force is stronger near the epicenter and weaker near the outer edges of the sphere. For every meter a target is from the center, reduce the damage of a blast weapon by -2. 

\subsubsection{Uniform blast} 

Uniform blast attacks distribute their power evenly throughout the area of effect. Examples include fuelair explosives and thermobaric weapons that disperse an explosive mixture in a vapor cloud and ignite it all at once. All targets within the noted blast radius suffer the same damage. Damage against targets outside of the main blast sphere is reduced by -2 per meter. 

\subsubsection{Cone} 

Weapons with a cone effect have an area effect that begins with the tip of the weapon and expands outward in a cone. At short range, this attack effects 1 target; at medium range it affects 2 targets within a meter of each other; and at long or extreme range it affects 3 targets within a meter of the next. Cone effect attacks do +1d10 damage at short range and -1d10 damage at long and extreme range. 



\subsection{Armor} <<<<<<< HEAD \label{sec:armor} ======= \label{sec:combat-armor} >>>>>>> 069efc938e743c2e1f7048421dcd3e53a984f73e 

Just as weapons technologies have advanced, so too has armor quality, allowing unprecedented levels of protection. As noted in Step 7: Determine Damage (see p. 192), the armor rating reduces the damage points of the attack. For a full listing of armor types and values, see p. 311. 

\subsubsection{Energy vs. Kinetic} 

<<<<<<< HEAD Each type of armor has an Armor value (AV) with two ratings - Energy and Kinetic - representing the protection it applies against the respective type of attack. These are listed in the format of ``Energy armor/Kinetic armor.'' For example, an item with listed armor ``5/10'' provides 5 points of armor against energy-based attacks and 10 points of armor against kinetic attacks. ======= Each type of armor has an Armor value (AV) with two ratings --- Energy and Kinetic --- representing the protection it applies against the respective type of attack. These are listed in the format of ``Energy armor/Kinetic armor.'' For example, an item with listed armor ``5/10'' provides 5 points of armor against energy-based attacks and 10 points of armor against kinetic attacks. >>>>>>> 069efc938e743c2e1f7048421dcd3e53a984f73e 

Energy damage includes that caused by beam weapons (laser, microwave, particle beam, plasma, etc.) as well as fire and high-energy explosives. Armor that protects against this damage is made of material that reflects or diffuses such energy, dissipates and transfers heat, or ablates. 

Kinetic damage is the transfer of damaging energy when an object in motion (a fist, knife, club, or bullet, for example) impacts with another object (the target). Most melee and firearms attacks inflict kinetic damage, as would a rolling boulder, swinging pendulum, or explosion-driven fragments. Kinetic armors include impact-resistant plates, shear-thickening liquid and gels that harden upon impact, and ballistic and cutproof fiber weaves. 

\subsubsection{Armor penetration} 

Some weapons have an Armor Penetration (AP) rating. This represents the attack’s ability to pierce through protective layers. The AP rating reduces the value of armor used to defend against the attack (see Step 6: Modify Armor, p. 192). 

\subsubsection{Layered armor} 

If two or more types of armor are worn, the armor ratings are added together. However, wearing multiple armor units is cumbersome and annoying. Apply a -20 modifier to a character’s actions for each additional armor layer worn. 

<<<<<<< HEAD Note that items specifically noted as armor accessories - helmets, shields, etc. - do not inflict the layered armor penalty, they simply add their armor bonus. Note also that the armor inherent to a synthetic morph or bot’s frame does not constitute a layer of armor (i.e., you may wear armor over the synthetic shell without penalty). ======= Note that items specifically noted as armor accessories --- helmets, shields, etc. --- do not inflict the layered armor penalty, they simply add their armor bonus. Note also that the armor inherent to a synthetic morph or bot’s frame does not constitute a layer of armor (i.e., you may wear armor over the synthetic shell without penalty). >>>>>>> 069efc938e743c2e1f7048421dcd3e53a984f73e 



\subsection{Asphyxiation} \label{sec:asphyxiation} 

The average transhuman can hold their breath for two minutes before blacking out. Strenuous activity reduces the amount of time. For every 30 seconds after the first minute a biomorph is prevented from breathing, they must make a DUR Test. Apply a cumulative -10 modifier each time this test is rolled. If the character fails the test, they immediately fall unconscious and begin to suffer damage from asphyxiation, at the rate of 10 points per minute, until they die or are allowed to breathe again. This damage does not cause wounds. 

<<<<<<< HEAD Asphyxiating is a terrible process, often leading to panic. Characters who are being asphyxiated must make a WIL x 3 Test. If they fail, they suffer 1d10 $\div$ 2 (round up) mental stress and cannot perform any effective action to rescue themselves that Action turn. A character who succeeds may attempt to rescue themselves, and in fact they must make a WIL x 3 Test to perform any other action not directly related to rescuing themselves (attacks against another character, a creature, or an object holding the character underwater are exempt from this rule). 



\div{Beam weapons} \label{sec:beam-weapons} ======= Asphyxiating is a terrible process, often leading to panic. Characters who are being asphyxiated must make a WIL $\div$ 3 Test. If they fail, they suffer 1d10 $\div$ 2 (round up) mental stress and cannot perform any effective action to rescue themselves that Action turn. A character who succeeds may attempt to rescue themselves, and in fact they must make a WIL $\times$ 3 Test to perform any other action not directly related to rescuing themselves (attacks against another character, a creature, or an object holding the character underwater are exempt from this rule). 



\subsection{Beam weapons} \label{sec:combat-beam-weapons} >>>>>>> 069efc938e743c2e1f7048421dcd3e53a984f73e 

Due to emitting a continuous beam of energy rather than single projectiles, beam weapons are easier to ``home in'' on a target. This means one of the following two rules may be used when making beam weapon attacks. These options are not available for ``pulse'' beam weapons (like the laser pulser), as such weapons emit energy in pulses rather than a continuous beam. 

\subsubsection{Sweeping fire} 

An attacker who is making two semi-auto (p. 198) attacks with a beam weapon with the same Complex Action and who misses with the first attack may treat that attack as a free Aim action (p. 190), receiving a +10 modifier for the second attack. In other words, though the first attack misses, the character takes the opportunity to sweep the beam closer to the target for the second attack. This only applies when both attacks are made against the same target. 

\subsubsection{Concentrated fire} 

<<<<<<< HEAD A character firing a semi-auto beam weapon who hits with the first attack may choose to keep the beam on and concentrate their fire, cooking the target. In this case, the character foregos their second semi-auto attack with that Complex Action, but automatically bolsters the DV of the first attack by x 1.5 (round up). This decision must be made before the damage dice are rolled. ======= A character firing a semi-auto beam weapon who hits with the first attack may choose to keep the beam on and concentrate their fire, cooking the target. In this case, the character foregos their second semi-auto attack with that Complex Action, but automatically bolsters the DV of the first attack by $\times$ 1.5 (round up). This decision must be made before the damage dice are rolled. >>>>>>> 069efc938e743c2e1f7048421dcd3e53a984f73e 



\subsection{Blind attacks} \label{sec:blind-attacks} 

<<<<<<< HEAD Attacking a target that you cannot see is difficult at best and a matter of luck at worst. If you cannot see, you may make a Perception Test using some other available sense to detect your target. If this succeeds, you attack with a -30 modifier. If your Perception Test fails, your attack is primarily based on chance - your target number for the attack test is equal to your Moxie stat (no other modifiers apply). ======= Attacking a target that you cannot see is difficult at best and a matter of luck at worst. If you cannot see, you may make a Perception Test using some other available sense to detect your target. If this succeeds, you attack with a -30 modifier. If your Perception Test fails, your attack is primarily based on chance --- your target number for the attack test is equal to your Moxie stat (no other modifiers apply). >>>>>>> 069efc938e743c2e1f7048421dcd3e53a984f73e 

\subsubsection{Indirect fire} 

With the help of a spotter, you may target an enemy that you can’t see using indirect fire. In this case you must be meshed with a character, bot, or sensor system that has the target in its sights and which feeds you targeting data (the gamemaster may require a Perception Test from the spotter). Indirect attacks suffer a -30 modifier. 

Seeker missiles (p. 340) can home in on a target that is ``painted'' with reflected energy from a laser sight (p. 342) or similar target designator system. An ``attack'' must first be made to paint the target with the laser sight using an appropriate skill. If this succeeds, it negates the -30 indirect fire modifier for the seeker launcher’s attack test. the target must be held in the spotter’s sights (requiring a Complex Action each Action Phase) until the seeker strikes. 



\subsection{Bots, synthmorphs and vehicles} \label{sec:bots-synthmorphs-vehicles} 

<<<<<<< HEAD AI-operated robots and synthetic morphs are a common sight in Eclipse Phase. Robots are used for a wide range of purposes, from surveillance, maintenance, and service jobs to security and policing - and so may often play a role in action and combat scenes. Though less common (at least in habitats), AI-piloted vehicles are also frequently used and encountered. 

Note that the difference between a robot, vehicle, and synthetic morph is in many ways semantic. Robots are simply synthetic bodies controlled by an AI. Vehicles are also robotic - AI controlled - but the term ``vehicle'' is used to denote that they carry passengers. Both bots and vehicles can be used as synthetic morphs - that is, inhabited by a transhuman ego - assuming they are equipped with a cyberbrain (p. 300). For the purpose of these rules, the term ``shell'' is used to refer to bots, vehicles, and synthetic morphs alike. ======= AI-operated robots and synthetic morphs are a common sight in Eclipse Phase. Robots are used for a wide range of purposes, from surveillance, maintenance, and service jobs to security and policing --- and so may often play a role in action and combat scenes. Though less common (at least in habitats), AI-piloted vehicles are also frequently used and encountered. 

Note that the difference between a robot, vehicle, and synthetic morph is in many ways semantic. Robots are simply synthetic bodies controlled by an AI. Vehicles are also robotic --- AI controlled --- but the term ``vehicle'' is used to denote that they carry passengers. Both bots and vehicles can be used as synthetic morphs --- that is, inhabited by a transhuman ego --- assuming they are equipped with a cyberbrain (p. 300). For the purpose of these rules, the term ``shell'' is used to refer to bots, vehicles, and synthetic morphs alike. >>>>>>> 069efc938e743c2e1f7048421dcd3e53a984f73e 

Like synthmorphs, bots and vehicles are treated just like any other character: they roll Initiative, take actions, and use skills. A few specific aspects of these shells needs special consideration, however, covered below. 

\subsubsection{Shell stats} 

Just like synthmorph characters, certain bot and vehicle stats (Durability, Wound Threshold, etc.) and stat modifiers (Initiative, Speed, etc.) are determined by the actual physical shell. Other stats are determined by the bot/vehicle’s operating AI (in place of an ego). Bots and vehicles may also have traits that apply to their AI or physical shell. For sample bots and vehicles, see p. 342 of the Gear chapter. 

\textbf{Handling:} Bots and vehicles have a special stat called Handling, which is a modifier applied to all tests made to pilot the bot/vehicle. This represents the bot/ vehicle’s maneuverability. 

\subsubsection{Shell skills} 

The skills and aptitudes used by a bot or vehicle are those possessed by its AI. See \emph{AIs and Muses}, p. 264. 

\subsubsection{Shell movement} 

Like characters, bots and vehicles have a walking and running Movement rate. This is used whenever the bot/vehicle is engaged in action or combat scenes with other characters. 

<<<<<<< HEAD Shells that are capable of greater speeds will also have a Maximum Velocity listed - this is the highest rate at which the shell may safely move, listed in kilometers per hour. At the gamemaster’s discretion, some shells may push their limits and accelerate further, but at significant risk - the gamemaster should apply appropriate penalties to Pilot Tests and other tests. 

\subsubsection{Chases} 

Shells that are moving faster than their running Movement rate (up to their Max. Velocity) are generally considered to be moving too fast for standard action and combat interaction with other characters. This is when the action enters ``chase scene'' mode - a traveling narrative of maneuvering choices and tests with various outcomes. Whether or not a chase is actually occurring, the gamemaster should remember that Max Velocity is not the only factor in high-speed situations. Environmental factors like terrain, weather conditions, navigation, pedestrians, and traffic can provide obstacles for shells to overcome. A shell tearing across a habitat in order to reach a bomb before it detonates should have to make several decisions and tests that may affect whether it gets there in time or not. Likewise, a shell seeking to shake off some hot pursuit will have to pull off some fancy maneuvering and hopefully find a shortcut or two in order to outrace their opponents. 

\subsubsection{Crashing} 

Shells that suffer wounds during combat or chases may be force to make a Pilot Test to avoid crashing or may crash automatically. The exact circumstances of a crash are left up to the gamemaster, as best fits the story - the shell may simply skid to a stop, plow into a tree, fall from the sky, or flip over and land on a group of bystanders. Shells that strike other objects when they crash typically take further damage from the collision (see Collisions). ======= Shells that are capable of greater speeds will also have a Maximum Velocity listed --- this is the highest rate at which the shell may safely move, listed in kilometers per hour. At the gamemaster’s discretion, some shells may push their limits and accelerate further, but at significant risk --- the gamemaster should apply appropriate penalties to Pilot Tests and other tests. 

\subsubsection{Chases} 

Shells that are moving faster than their running Movement rate (up to their Max. Velocity) are generally considered to be moving too fast for standard action and combat interaction with other characters. This is when the action enters ``chase scene'' mode --- a traveling narrative of maneuvering choices and tests with various outcomes. Whether or not a chase is actually occurring, the gamemaster should remember that Max Velocity is not the only factor in high-speed situations. Environmental factors like terrain, weather conditions, navigation, pedestrians, and traffic can provide obstacles for shells to overcome. A shell tearing across a habitat in order to reach a bomb before it detonates should have to make several decisions and tests that may affect whether it gets there in time or not. Likewise, a shell seeking to shake off some hot pursuit will have to pull off some fancy maneuvering and hopefully find a shortcut or two in order to outrace their opponents. 

\subsubsection{Crashing} 

Shells that suffer wounds during combat or chases may be force to make a Pilot Test to avoid crashing or may crash automatically. The exact circumstances of a crash are left up to the gamemaster, as best fits the story --- the shell may simply skid to a stop, plow into a tree, fall from the sky, or flip over and land on a group of bystanders. Shells that strike other objects when they crash typically take further damage from the collision (see Collisions). >>>>>>> 069efc938e743c2e1f7048421dcd3e53a984f73e 

\subsubsection{Collisions} 

If a shell crashes into or intentionally rams a person or object, someone is likely to get hurt. To determine how much DV is inflicted, roll 1d10 and add the shell’s DUR divided by 10 (round up). This is the damage applied at walking speeds. If the shell was moving at running speeds, multiply the DV by 2. If the shell was moving at chase speeds, multiply the DV by the shell’s velocity $\div$ 10 in meters per turn. Both the shell and whatever it strikes suffer this damage, assuming the collision is with something equal dense and hard. Soft and squishy objects like biomorphs will be less damaging to a shell (unless they happen to be in a hardsuit or battlesuit), in which case the shell will only suffer half damage from the collision. Kinetic armor defends against crash DV. 

If two moving shells collide head-on, calculate the damage from both and inflict to both. If two shells moving in the same direction collide, only count the difference in velocity. 

Passengers in a vehicle may also be damaged by collisions if they are not wearing proper safety restraints. They suffer one half the DV applied to their vehicle (less their own Kinetic armor). 

\begin{table} \begin{tabular}{|l|l|} \hline

\hline{2}{|c|}{\textbf{Collision damage}}	\\ \hline

Base Collision DV	&1d10 + (DUR $\hline$ 10)	\\ \hline

<<<<<<< HEAD Running	&DV x 2	\\ \hline

Chase Speeds	&DV x (velocity $\div$ 10)	\\ ======= Running	&DV $\hline$ 2	\\ \hline

Chase Speeds	&DV $\hline$ (velocity $\hline$ 10)	\\ >>>>>>> 069efc938e743c2e1f7048421dcd3e53a984f73e \hline

\label{tab:collision-damage} \label{tab:collision-damage} \end{table} 

\subsubsection{Attacking passenger vehicles} 

During combat, passengers within a vehicle may be targeted separately from the vehicle itself. Attacks made against passengers this way do not harm the vehicle itself (unless an area effect weapon is used). Targeted passengers benefit both from cover (usually Major, -30) and from the vehicle’s structure, adding the vehicle’s Armor Value to their own. 

Passengers within a vehicle are generally not harmed by attacks made against the vehicle itself. Area effect weapons are an exception to this rule, but in this case the passengers also benefit from the vehicle Armor Value. 

\subsubsection{Shell remote control} 

Any shell (or biomorph) with a puppet sock (also included with all cyberbrains) may be remote controlled, either by a character or a remote AI. This requires a communications link between the teleoperator and the shell (the ``drone�?). The teleoperator controls the drone via an entoptic interface, and receives sensory input and other data via the drone’s mesh inserts. 

When under direct control, the shell’s AI (or resident ego) is subsumed and put on standby. The drone only acts as instructed. Each instruction counts as a Quick Action. In this situation, the drone acts with the same Initiative as the teleoperator. The teleoperator’s skills and stats are used in place of the shell AI’s. Due to the nature of remote operation, however, all tests are made with a -10 modifier. Multiple drones may be controlled at once, but commanding them requires separate Quick Actions unless they are receiving the same command. Direct control teleoperation is not very feasible at extreme distances, due to the light-speed lag with communications. 

Alternately, the teleoperator can put the drone in autonomous mode, allowing the shell’s AI to resume normal operations. The drone still follows the teleoperator’s commands to the best of its abilities. In this mode, the drone functions normally, using its own Initiative and AI skills and stats. 

\subsubsection{Shell jamming} 

``Jamming'' is the colloquial term for a more direct form of remote-control, using VR and XP technology. When jamming, the drone’s puppet sock feeds the drone’s sensory data directly to the teleoperator’s mesh inserts. The teleoperator subsumes themself in the drone’s sensorium, essentially ``becoming'' the drone. In this case, the teleoperator surrenders control of their own morph, which slumps inertly. While jamming, they suffer -60 on all Perception Tests or attempts to take action with their morph. 

Jamming takes a Complex Action to engage and disengage. A jamming teleoperator controls a drone as if it were their own morph. Like direct control teleoperation, the jammer’s own skills and Initiative are used in place of the drone’s AI. Jammers do not suffer any teleoperation modifiers, but only one drone may be jammed at a time. 

If the drone is killed or destroyed, the jammer is immediately dumped from their connection, resuming control of their own morph as normal. Getting dumped in this manner is extremely jarring, not the least because the jammer experienced being killed/destroyed. As a result, the jammer suffers 1d10 mental stress. 



\subsection{Called shots} \label{sec:called-shots} 

<<<<<<< HEAD Sometimes it’s not enough to just hit your target - you need to shoot out a window, knock the knife out of their hand, or hit that hole in their armor. You may declare that you are making a called shot before you initiate an attack, choosing one of the outcomes noted below. Called shots suffer a -10 modifier and require an Excellent Success (MoS 30+). If you beat that margin, you succeed with the called shot, and the results noted below apply. If you don’t beat the margin but still succeed in the attack, you simply strike your target as normal. ======= Sometimes it’s not enough to just hit your target --- you need to shoot out a window, knock the knife out of their hand, or hit that hole in their armor. You may declare that you are making a called shot before you initiate an attack, choosing one of the outcomes noted below. Called shots suffer a -10 modifier and require an Excellent Success (MoS 30+). If you beat that margin, you succeed with the called shot, and the results noted below apply. If you don’t beat the margin but still succeed in the attack, you simply strike your target as normal. >>>>>>> 069efc938e743c2e1f7048421dcd3e53a984f73e 

\subsubsection{Bypassing armor} 

Called shots may be used to target a hole or weak point in your opponent’s armor. If you beat the MoS, you strike an armor-defeating hit, and their armor does not apply. Note that in certain circumstances, a gamemaster may rule that an opponent’s armor simply doesn’t have a weak spot or unprotected area, and so disallow such called shots. 

\subsubsection{Disarming} 

<<<<<<< HEAD You may take a called shot to attempt to knock a weapon out of an opponent’s hand(s). If you beat the MoS, the victim suffers half damage from the attack (reduced by armor as normal) and must make a SOM x 3 Test with a -30 modifier to retain hold of the weapon. 

\subsubsection{Specific targeting} 

You may make a called shot with the intention of hitting a specific location or component on your target - for example: disabling the sensor unit on a bot, sweeping someone’s leg, or poking someone in the eye. If you beat the MoS, you hit the specific targeted spot. The gamemaster determines the result as appropriate to the attack and target - the component may be destroyed, the opponent may fall or be temporarily blinded, and so on. ======= You may take a called shot to attempt to knock a weapon out of an opponent’s hand(s). If you beat the MoS, the victim suffers half damage from the attack (reduced by armor as normal) and must make a SOM $\times$ 3 Test with a -30 modifier to retain hold of the weapon. 

\subsubsection{Specific targeting} 

You may make a called shot with the intention of hitting a specific location or component on your target --- for example: disabling the sensor unit on a bot, sweeping someone’s leg, or poking someone in the eye. If you beat the MoS, you hit the specific targeted spot. The gamemaster determines the result as appropriate to the attack and target --- the component may be destroyed, the opponent may fall or be temporarily blinded, and so on. >>>>>>> 069efc938e743c2e1f7048421dcd3e53a984f73e 

\subsection{Charging} \label{sec:charging} 

An opponent who runs and attacks an opponent in melee combat in the same Action Phase is considered to be charging. A charging attacker still suffers the -10 modifier for running, but they receive a damage bonus on account of their momentum: increase the damage they inflict by +1d10. 

\subsubsection{Receiving a charge} 

You may delay your action (see p. 189) in order to receive a charge, bracing yourself for impact, interrupting their action, and striking right before your charging does. In this situation, you receive a +20 modifier for striking the charging opponent. 



\subsection{Demolitions} \label{sec:demolitions} 

The most common use of the Demolitions skill is the placement, disarming, or manufacture of explosive devices, such as superthermite charges (p. 330) or grenades (p. 340). 

\subsubsection{Placing explosives} 

A skilled demolitionist can place charges in a manner that will boost their effect. They can identify structural vulnerabilities and weak points and focus a blast in these areas. They can determine how to blast open a safe without destroying the contents. They can focus the force of an explosion in a particular direction, increasing the directed force while minimizing splash effects. 

Each of these scenarios calls for a successful Demolitions Test. The exact result is determined by the gamemaster according to the specific scenario. For example, using the examples above, targeting a weak point could double the damage inflicted on that structure. Shaping the charge to direct the force can triple the damage in that direction, as noted in the superthermite description (p. 330). An Excellent Success is likely to increase an explosive’s damage by +5, whereas a critical success would allow the blast to ignore armor. 

\subsubsection{Disarming} 

Disarming an explosive device is handled as an Opposed Test between the Demolitions skills of the disarmer and the character who set the bomb. 



\subsubsection{Making explosives} 

A character trained in Demolitions can make explosives from raw materials. These materials can be gathered the traditional way or they can be manufactured using a nanofabricator. Even nanofabbers with restricted settings to prevent explosives creation can be used, as explosives can be constructed from all manner of mundane chemicals and materials. The timeframe for making explosives is 1 hour per 1d10 points of damage the explosive will inflict. If a critical failure is rolled, the demolitionist may accidentally blow himself up, or the charge may be extremely weaker or more potent than expected (whichever is more likely to be disastrous). 



\subsection{Falling} \label{sec:falling} 

If a character falls, use the Falling Damage table to determine what injuries they suffer. Kinetic armor will mitigate this damage at half its value (round down). Gamemasters may also reduce this damage if anything helped to break the fall (branches, soft surface) at their discretion. 

\begin{table} \begin{tabular}{|l|l|} \hline

\hline{2}{|c|}{\textbf{Falling damage}}	\\ \hline

\textbf{Distance fallen}	&\textbf{Damage}	\\ \hline

1-2 meters	&1d10	\\ \hline

3-5 meters	&2d10	\\ \hline

6-8 meters	&3d10	\\ \hline

Over 8 meters	&+1 per meter	\\ \hline

\label{tab:falling-damage} \label{tab:falling-damage} \end{table} 



\subsection{Fire} \label{sec:fire} 

Objects that come into contact with extreme heat or flames may catch fire at the gamemaster’s discretion, keeping in mind both the flammability of the material and the strength of the heat/flames. Burning items (or characters) will suffer 1d10 $\div$ 2 (round up) damage each Action Turn unless otherwise noted. Energy armor will protect against this damage, though it too may catch fire, reducing its value by the damage inflicted. Depending on the environmental conditions, fires are likely to grow larger unless somehow abated. Every 5 Action Turns, increase the DV inflicted (first to 1d10, then 2d10, then 3d10, then by increments of +5). Adverse conditions (such as rain) or efforts to extinguish the blaze will reduce the DV accordingly. 

Note that fire does not burn in vacuum. In microgravity, fire burns in a sphere and grows more slowly, as expanding gases push away the oxygen (increase the DV every 10 Action Turns). If there is a lack of air circulation, some microgravity fires may extinguish themselves. 



\subsection{Firing modes and rate of fire} \label{sec:firing-modes-rate} 

Every ranged weapon in \emph{Eclipse Phase} comes with one or more firing modes that determines their rate of fire. These firing modes are detailed below. 

\subsubsection{Single shot(SS)} 

Single shot weapons may only be fired once per Complex Action. These are typically larger or more archaic devices. 

\subsubsection{Semi-automatic (SA)} 

Semi-automatic weapons are capable of quick, repeated fire. They may be fired twice with the same Complex Action. Each shot is handled as a separate attack. 

\subsubsection{Burst fire (BF)} 

Burst fire weapons release a number of quick shots (a ``burst�?) with a single trigger pull. Two bursts may be fired with the same Complex Action. Each burst is handled as a separate attack. Bursts use up 3 shots worth of ammunition. 

A burst may be shot against a single target (concentrated fire), or against two targets who are standing within one meter of each other. In the case of concentrated fire against a single target, increase the DV by +1d10. 

\subsubsection{Full Automatic (FA)} 

Full-auto weapons release a hail of shots with a single trigger pull. Only one full-auto attack may be made with each Complex Action. This attack may be made a single target or against up to three separate targets, as long as each is within one meter of another. In the case of a concentrated fire on a single individual, increase the DV by +1d10 + 10. Firing in full automatic mode uses up 10 shots. 



\subsection{Full defense} \label{sec:full-defense} 

If you’re expecting to come under fire, you can expend a Complex Action to go on full defense. This represents that you are expending all of your energy to dodge, duck, ward off attacks, and otherwise get the hell out of the way until your next Action Phase. During this time, you receive a +30 modifier to defend against all incoming attacks. Characters who are on full defense may use Freerunning rather than Fray skill to dodge attacks, representing the gymnastic movements they are making to avoid being hit. 

\subsection{Gravity} \label{sec:gravity} 

Most characters in Eclipse Phase have considerable experience maneuvering in low gravity or microgravity and can perform normal actions without penalties. Even characters who grew up on planetary bodies or in rotating habitats have some familiarity with alternate gravities thanks to childhood training in simulspace educational scenarios. The same is also true in reverse; characters who grew up in free fall have likely experienced simulations of life in a gravity well. 

At the gamemaster’s discretion, characters who have spent long periods acclimating to one range of gravity may find a shift in conditions a bit challenging to cope with, at least until they grow accustomed to the new gravity. In this case, the gamemaster can apply a -10 modifier to both physical and social skills. The physical penalty results from simple difficulties in maneuvering. The social penalty applies because it’s hard to look impressive, intimidating, or seductive when you haven’t figured out how to arrange your clothes so that they don’t float up into your face. The physical penalty can be increased to -20 for situations involving combat skills and skills requiring fine manipulation, building, or repairing of items. These penalties will apply until the character adjusts, which typically takes about 3 days. 

Any biomorph with basic biomods (p. 300) is immune to ill health from the effects of long-term exposure to microgravity. 

\subsubsection{Microgravity} 

Microgravity includes both zero-G and gravities that are slightly higher but negligible. These conditions are found in space, on asteroids and some small moons, and on (parts of) spaceships and habitats that are not rotated for gravity. Objects in microgravity are effectively weightless, but size and mass are still factors. Things behave differently in microgravity. For example: 

\end{itemize} \item Objects not anchored down will tend to drift off in whatever direction they were last moving. Floating objects will eventually settle in the direction of the densest part of the habitat or spacecraft. \item Thrown or pushed items will travel in a straight line until they hit something. \item Smoke does not rise in streams. Instead, it forms a roughly spherical halo around its source. \item Liquids have little cohesion, scattering into clouds of tiny droplets if released into the air. Drinks come in sealed bulbs or bottles. Food is eaten so that sauces and bits of liquid don’t escape. Blood goes everywhere. \end{itemize} 

Movement and maneuvering in microgravity is handled using Free Fall skill (p. 179). Most everyday activity in free fall does not require a test. The gamemaster can, however, call for a Free Fall Test for any complicated maneuvers, flying across major distances, sudden changes in direction or velocity, or when engaged in melee combat. A failed roll means the character has miscalculated and ends up in a position other than intended. A Severe Failure means the character has screwed up badly, such as slamming themselves into a wall or sending themselves spinning off into space. 

For convenience, most microgravity habitats feature furniture covered with elastic loops, mesh pockets to keep individual objects from floating all over the place, and moving beltways with hand loops for major thoroughfares. Magnetic or velcro shoes are also used to walk around, rather than climbing or flying. Zero-g environments are often designed to make maximum use of space, however, taking advantage of the lack of ceilings and floors. Because object are weightless, characters can move even massive objects around easily. 

\textbf{Movement Rate:} Characters who are climbing, pulling, or pushing themselves along move at half their movement rate (p. 191) in microgravity. 

\textbf{Terminal Velocity:} It is not difficult to reach escape velocity on small asteroids and similar bodies- something to keep in mind with thrown objects and projectile weapons. In some cases, characters who move fast enough and jump can reach escape velocity themselves, though these situations are left up to the gamemaster. 

\subsubsection{Low gravity} 

Low gravity includes anything from 0.5 g to microgravity. These conditions are found on Luna, Mars, Titan, and the rotating parts of most spun spacecraft and habitats. Low gravity is not that different from standard gravity, though characters may jump twice as far and thrown/projectile objects have a longer range (p. 203). Increase the running rate for characters in low gravity by x1.5. 

\subsubsection{High gravity} 

High gravity is anything significantly stronger than standard Earth gravity (1.2 g +). High gravity in Eclipse Phase is typically only found on exoplanets. High gravity can be particularly hard on characters as their bodies are strained because they carry more weights, muscles are fatigued from needing to push more around, and the heart must work harder to pump blood. For every 0.2 g over 1 that a character is not acclimated to, treat it as if the character is suffering from the effects of 1 wound. At the gamemaster’s discretion, movement rates may also be modified. 

\subsection{Grenades and seekers} <<<<<<< HEAD \label{sec:grenades-seekers} ======= \label{sec:combat-grenades-seekers} >>>>>>> 069efc938e743c2e1f7048421dcd3e53a984f73e 

Modern grenades, seekers, and similar explosives do not necessarily detonate the instant they are thrown or strike the target. In fact, several trigger options are available, each set by the user when deploying the weapon. Missed attacks, or attacks that do not explode in transit or when they strike, are subject to scatter (p. 204). 

\textbf{Airburst:} Airburst means that the device explodes in mid-air as soon as it travels a distance programmed at launch. In this case, the explosive’s effects are resolved immediately, in that user’s Action Phase. Note that airburst munitions are programmed with a safety feature that will prevent detonation if they fail to travel a minimum precautionary distance from the launcher, though this can be overridden. 

\textbf{Impact:} The grenade or missile goes off as soon as it hits something, whether that be the target, ground, or an intervening object. Resolve the effects immediately, in the user’s Action Phase. 

\textbf{Signal:} The munition is primed for detonation upon receiving a command signal via wireless link. The device simply lays in wait until it receives the proper signal (which must include the cryptographic key assigned when the grenade was primed), detonating immediately when it does. 

\textbf{Timer:} The device has a built-in timer allowing the user to adjust exactly when it detonates. This can be anywhere from 1 second to days, months, or even years later, effectively using the device like a bomb, but also increasing the likelihood it will be discovered and neutralized. The minimum detonation period-1 second-means that the munition will detonate on the user’s (current) Initiative Score in the next Action Phase. A 2-second delay would last two Action Phases, a 3-second delay three Action phases, and so on. 

\subsubsection{Throwing back grenades} 

It is possible that a character may be able to reach a grenade before it detonates and throw it back (or away in a safe direction). The character must be within movement range of the grenade’s location, and must take a Complex Action to make a REF + COO + COO Test to catch the rolling, sliding grenade. If they succeed, they may throw the grenade off in a direction of their choice with the same action (treat as a standard throwing attack). 

If the character fails the test, however, they may find themselves at ground zero when it detonates. 

\subsubsection{Jumping on} 

Given the possibility of resleeving, a character may decide to take one for the team and throw themselves on a grenade, sacrificing themselves in order to protect others. The character must be within movement range of the grenade’s location, and must take a Complex Action to make a REF + COO + WIL Test to fall on the grenade and cover it with their morph. This means the character suffers an extra 1d10 damage when the grenade detonates. On the positive side, the grenade’s damage is reduced by the sacrificing character’s armor + 10 when its damage effects are applied to others within the blast radius. 

<<<<<<< HEAD If the gamemaster feels it appropriate, a WIL x 3 Test might be called for in order for a character to sacrifice themselves in this manner. ======= If the gamemaster feels it appropriate, a WIL $\times$ 3 Test might be called for in order for a character to sacrifice themselves in this manner. >>>>>>> 069efc938e743c2e1f7048421dcd3e53a984f73e 

\subsection{Hostile environments} \label{sec:hostile-environments} The solar system might be friendly to life on a grand scale, but if you’re stranded in the gravity well of Jupiter during a magnetic storm, trying to breathe without a respirator on Mars, or swimming in hard vacuum without a space suit, it doesn’t seem so friendly. This section describes a few of the hostile environments that characters in \emph{Eclipse Phase} might face. 

\subsubsection{Atmospheric contamination} 

Habitats sometimes fall ill. The effects of a habitat suffering from ecological imbalance or out-of-control pathogens can range from mildly allergenic habitat atmospheres to rampaging environmental sepsis. Characters without breathing or filtration gear in a contaminated environment should suffer penalties to physical and possibly social skills, ranging from -10 (mild contamination) to -30 (severely septic atmosphere). Depending on the contamination, other effects may apply, as the gamemaster sees fit. 

\subsubsection{Extreme heat and cold} 

Planetary environments can range from the extremely hot (Venus, Mercury’s day side) to the extremely frigid (Neptune, Titan, Uranus). Both are likely to kill an unprotected and unmodified biomorph within minutes, if not seconds. Synthmorphs and vehicles fare better, especially in the cold, but even they are likely to quickly succumb to the blazing furnaces of the inner planets without strong heat shields and cooling systems. 

\subsubsection{Extreme pressure} 

Similarly, the atmospheric pressures of Jupiter, Saturn, and Venus quickly become crushingly deadly anywhere beyond the upper levels. Only synthmorphs and vehicles with special pressure adaptations can hope to survive such depths. 

\subsubsection{Gravity transition zones} 

The widespread use of artificial gravity in space habitats means that characters will often encounter places where the direction of down suddenly changes. In most rotating habitats, the standard design includes an axial zone where spacecraft can dock in microgravity and a carefully designed and marked transition zone (usually an elevator) where people and cargo coming and going from the axial spaceport can orient to local ``down'' and be standing in the right place when gravity takes effect. Gravity transitions in rotating habitats are almost always gradual but can be very dangerous if a character encounters them in the wrong place or time. 

A character cast adrift in the microgravity zone at the axis of a rotating space habitat will slowly drift outward until they begin to encounter gravity, at which point they will fall. How long this takes varies on the size of the habitat. A good rule of thumb is that for each kilometer of diameter possessed by the habitat, the character has 30 seconds before they begin to fall. If the character was given a good push out from the axis when set adrift, this time should be halved, quartered, or more at the gamemaster’s discretion. 

\subsubsection{Magnetic fields} 

Magnetism isn’t a direct problem for most characters; transhumans need to worry more about the radiation generated by a powerful magnetosphere. For unshielded electronic devices and similarly unshielded transhumans sporting titanium, however, the effects of strong magnetic fields can be devastating. Note that many of the conditions that result in vehicles, bots, and gear being exposed to strong magnetic field activity coincide with strong radioactivity. 

Magnetic fields affect synthmorphs, robots, vehicles, cybernetic implants, and electronics after 1 minute of exposure. Like radiation exposure, these effects can vary drastically. At the low end, communication and sensors will suffer interference and shortened ranges; at high ends, electronic systems will simply suffer damage and fail. 

\subsubsection{Radiation} 

Ionizing radiation is one of the more prevalent hazards in the solar system and one of the most difficult problems for transhumanity to defeat. Radiation damages genetic material, sickens, and kills by ionizing the chemicals involved in cell division within the human body. Effects range from nausea and fatigue to massive organ failure and death. However, radiation sickness is not solely a somatic phenomenon. The real terror of radiation for transhumans, especially at high dose levels such as those experienced on the surface of Ganymede and other Jovian moons, is damage to the neural network. This can lead to flawed uploads and backups. Nanomedicine that can rapidly reverse the ionization of cellular chemicals and new materials that offer thinner and better shielding help, but the sheer magnitude of the radiation put out by some bodies in the solar system defeats even these. 

Radiation poisoning is a complicated affair, and detailed rules are beyond the scope of this book. Sources of radiation include the Earth’s Van Allen belt, Jupiter’s radiation belt, Saturn’s magnetosphere, cosmic rays, solar flares, fission materials, unshielded fusion or antimatter explosions, and nuclear blasts, among others. Effects can vary drastically depending on the strength of the source, the amount of time exposed, and the level of shielding available. The immediate effects on biomorphs (manifesting anywhere from within minutes to 6 hours) can include nausea, vomiting, fatigue (reduced SOM), as well as both physical damage and minor amounts of mental stress. This is followed by a latency period where the biomorph seems to get better, lasting anywhere from 6 hours to 2 weeks. After this point, the final effects kick in, which can include hair loss, sterility, reduced SOM and DUR, severe damage to gastric and intestinal tissue, infections, uncontrolled bleeding, and death. 

Synthmorphs are not quite as vulnerable as biomorphs, but even they can be damaged and disabled by severe radiation dosages. 

\subsubsection{Toxic atmosphere} 

Neptune, Titan, Uranus, and Venus all have toxic atmospheres. Similar atmospheres may be found on some exoplanets, or might be intentionally created as a security measure within a habitat or structure. 

<<<<<<< HEAD A character who is unaware of atmospheric toxicity and does not immediately hold their breath (requiring a REF x 3 Test) suffers 10 points of damage per Action Turn. A character who manages to hold their breath can last a bit longer; apply the rules for asphyxiation (p. 194). ======= A character who is unaware of atmospheric toxicity and does not immediately hold their breath (requiring a REF $\times$ 3 Test) suffers 10 points of damage per Action Turn. A character who manages to hold their breath can last a bit longer; apply the rules for asphyxiation (p. 194). >>>>>>> 069efc938e743c2e1f7048421dcd3e53a984f73e 

\textbf{Corrosive Atmospheres:} In addition to being toxic, Venus has the only naturally occurring corrosive atmosphere in the system. Corrosive atmospheres are immediately dangerous: characters take 10 points of damage per Action Turn, regardless of whether they hold their breath or not. Corrosive atmospheres also damage vehicles and gear not equipped with anticorrosive shielding. Such items take 1 point of damage per minute. At greater concentrations, such as in the dense sulfuric acid clouds in the upper atmosphere of Venus, items takes 5 points of damage per minute. 

\subsubsection{Unbreathable atmosphere} 

Very few of the planetary bodies in the solar system actually have toxic atmospheres. In most unbreathable atmospheres, the primary hazard for transhumans without breathing apparatus or modifications to their morph is lack of oxygen. Treat exposure to unbreathable atmospheres the same as asphyxiation. 

\subsubsection{Underwater} 

In general, any physical skill performed underwater suffers a -20 penalty due to the resistance of the medium. Skills relying on equipment not adapted for underwater use may be more difficult or impossible to use. A character’s movement rate while swimming or walking underwater is one quarter their normal rate on land. If a character begins to drown underwater, follow the rules for asphyxiation (p. 194). Note that drowning characters do not immediately recover if rescued from the water; they will continue to asphyxiate until medical treatment is applied to clear the water from their lungs. 

\subsubsection{Vacuum} 

Biomorphs without vacuum sealing (p. 305) can spend one minute in the vacuum of space with no ill effects, provided they curl up into a ball, empty their lungs of air, and keep their eyes closed (something kids in space habitats learn at a very young age). Contrary to popular depictions in pre-Fall media, a character exposed to hard vacuum does not explosively decompress, nor do their internal fluids boil (other than relatively exposed liquids such as saliva on the tongue). Rather, the primary danger for characters on EVA sans vacsuit is asphyxiation due to lack of oxygen and associated complications such as edema in the lungs. 

After one minute in space, the character begins to suffer from asphyxiation (p. 194). Damage is doubled if the character tries to hold air in their lungs or is not curled up in a vacuum survival position as described above. Additionally, characters trapped in space without adequate thermal protection suffer 10 points of damage per minute from the extreme cold. 

\subsection{Improvised weapons} \label{sec:improvised-weapons} 

Sometimes characters are caught off-guard and they must use whatever they happen to have at hand as a weapon-or they think they look cool wailing on someone with a meter of chain. The Improvised Weapons table offers statistics for a few likely ad-hoc items. Gamemasters can use these as guidelines for handling items that aren’t listed. 

\begin{table} <<<<<<< HEAD \begin{tabular}{|l|l|p{45mm}|l|l|} ======= \begin{tabularx}{\hline}{|l|l|l|l|X|} >>>>>>> 069efc938e743c2e1f7048421dcd3e53a984f73e \hline

\hline{5}{|c|}{\textbf{Improvised weapons}} \\ \hline

\textbf{Weapons}	&\textbf{AP}	&\textbf{Damage value (DV)}	&\textbf{Average DV}	&\textbf{Skill}	\\ \hline

<<<<<<< HEAD Baseball	&-	&(1d10 $\div$ 10) + (SOM $\div$ 10)	&2 + (SOM $\div$ 10)	&Throwing weapons	\\ \hline

Bottle	&-	&1 + (SOM $\div$ 10), breaks after 1 use	&1 + (SOM $\div$ 10)	&Clubs or throwing weapons	\\ \hline

Bottle (broken)	&-	&1d10 - 1 (min: 1)	&4	&Blades	\\ \hline

Chain	&-	&1d10 + (SOM $\div$ 10)	&5 + (SOM $\div$ 10)	&Exotic melee	\\ \hline

Helmet	&-	&1d10 + (SOM $\div$ 10)	&5 + (SOM $\div$ 10)	&Clubs or Throwing weapons	\\ \hline

Plasma torch	&-6	&2d10	&11	&Exotic ranged	\\ \hline

Wrench	&- &1d10 + (SOM $\div$ 10)	&5 + (SOM $\div$ 10)	&Clubs	\\ \hline

\end{tabular} ======= Baseball	&$-$	&(1d10 $\div$ 10) + (SOM $\div$ 10)	&2 + (SOM $\div$ 10)	&Throwing weapons	\\ \hline

Bottle	&$-$	&1 + (SOM $\div$ 10), breaks after 1 use	&1 + (SOM $\div$ 10)	&Clubs or throwing weapons	\\ \hline

Bottle (broken)	&$-$	&1d10 $-$ 1 (min: 1)	&4	&Blades	\\ \hline

Chain	&$-$	&1d10 + (SOM $\div$ 10)	&5 + (SOM $\div$ 10)	&Exotic melee	\\ \hline

Helmet	&$-$	&1d10 + (SOM $\div$ 10)	&5 + (SOM $\div$ 10)	&Clubs or Throwing weapons	\\ \hline

Plasma torch	&$-$6	&2d10	&11	&Exotic ranged	\\ \hline

Wrench	&$-$ &1d10 + (SOM $\div$ 10)	&5 + (SOM $\div$ 10)	&Clubs	\\ \hline

\label{tab:improvised-weapons} >>>>>>> 069efc938e743c2e1f7048421dcd3e53a984f73e \label{tab:improvised-weapons} \end{table} 



\subsection{Knockdown/knockback} \label{sec:knockdown-knockback} 

If an attacker’s intent is to simply knock an opponent down or back in melee, rather than injure them, roll the attack and defense as normal. If the attacker succeeds, the defender is knocked backward by 1 meter per 10 full points of MoS. To knock an opponent down, the attacker must score an Excellent Success (MoS 30+). A knockback/knockdown attack must be declared before dice are rolled. 

Unless the attacker rolls a critical success, no damage is inflicted with this attack, the defender is simply knocked down. If the attacker rolls a critical hit, however, apply damage as normal in addition to the knockback/knockdown. 

Note that characters wounded by an attack may also be knocked down (see \emph{Wound Effects}, p. 207). 



\subsection{Melee and thrown damage bonus} \label{sec:melee-thrown-damage-bonus} 

Every successful melee and thrown weapon attack, whether unarmed or with a weapon, receives a damage bonus equal to the attacker’s SOM $/div$ 10, round down. See \emph{Damage Bonus}, p. 123. 



\subsection{Multiple targets} \label{sec:multiple-targets} 

When doling out the damage, there’s no reason not to share the love. 

\subsubsection{Melee combat} 

A character taking a Complex Action to engage in a melee attack may choose to attack two or more opponents with the same action. Each opponent must be within one meter of another attacked opponent. These attacks must be declared before the dice are rolled for the first attack. Each attack suffers a cumulative -20 modifier for each extra target. So if a character declares they are going to attack three characters with the same action, they suffer a cumulative -60 on each attack. 

\subsubsection{Ranged combat} 

A character firing two semi-auto shots with a Complex Action may target a different opponent with each shot. In this case, the attacker suffers a -20 modifier against the second target. 

A character firing a burst-fire weapon may target up to two targets with each burst, as long as those targets are within one meter of one another. This is handled as a single attack; see \emph{Burst Fire}, p. 198. 

A character firing a burst-fire weapon twice with one Complex Action may target a different person or pair with each burst. In this case, the second burst suffers a -20 modifier. This modifier does not apply if the same person/pair targeted with the first burst is targeted again. 

Full-auto attacks may also be directed at more than one target, as long as each target is within one meter of the previous target. This is handled as a single attack; see \emph{Full Auto}, p. 198. 



\subsection{Objects and structures} \label{sec:objects-structures} 

As any poor wall in the vicinity of an enraged drunk can tell you, objects and structures are not immune to violence and attrition. To reflect this, inanimate objects and structures are given Durability, Wound Threshold, and Armor scores, just like characters. Durability measures how much damage the structure can take before it is destroyed. Armor reduces the damage inflicted by attacks, as normal. For simplicity, a single Armor rating is given that counts as both Energy and Kinetic armor; at the gamemaster’s discretion, these may be modified as appropriate. 

Wounds suffered by objects and structures do not have the same effect as wounds inflicted on characters. Each wound is simply treated as a hole, partial demolition, or impaired function, as the gamemaster sees fit. Alternately, a wounded device may function less effectively, and so may inflict a negative modifier on skill tests made while using that object (a cumulative -10 per wound). 

In the case of large structures, it is recommended that individual parts of the structure be treated as separate entities for the purpose of inflicting damage. 

\subsubsection{Ranged attacks} 

Ranged combat attacks inflict only one-third their damage (round down) on large structures such as doors, walls, etc. This reflects the fact that most ranged attacks simply penetrate the structure, leaving minor damage. 

Agonizers and stunners have no effect on objects and structures. 

\subsubsection{Shooting through} 

<<<<<<< HEAD If a character attempts to shoot through an object or structure at a target on the other side, the attack is likely to suffer a blind fire modifier of at least -30 unless the attack has some way of viewing the target. On top of this, the target receives an armor bonus equal to the object/structure’s Armor rating x 2. 

\begin{table} \begin{tabular}{|p{8cm}|r|r|r|} ======= If a character attempts to shoot through an object or structure at a target on the other side, the attack is likely to suffer a blind fire modifier of at least -30 unless the attack has some way of viewing the target. On top of this, the target receives an armor bonus equal to the object/structure’s Armor rating $\times$ 2. 

\begin{table} \begin{tabularx}{\hline}{|X|r|r|r|} >>>>>>> 069efc938e743c2e1f7048421dcd3e53a984f73e \hline

\hline{4}{|c|}{\textbf{Sample objects and structures}} \\ \hline

\textbf{Object/structure} &\textbf{Armor} &\textbf{Durability} &\textbf{Wound threshold}	\\ \hline

Advanced composites (ship/habitat hull)	&50	&1000	&200	\\ \hline

Aerogel (walls, windows etc)	&-	&50	&10	\\ \hline

Airlock door	&15	&100	&25	\\ \hline

Alloys, concrete, hardened polymers (reinforced doors/walls)	&30	&100	&20	\\ \hline

Armored glass	&10	&50	&20	\\ \hline

Counter	&7	&60	&12	\\ \hline

Desk	&5	&50	&10	\\ \hline

Ecto link	&-	&6	&1	\\ \hline

Metallic foam (walls, doors etc)	&20	&70	&15	\\ \hline

Metallic glass	&30	&150	&30	\\ \hline

Polymer or wood (walls, doors, furniture etc)	&10	&40	&8	\\ \hline

Quantum farcaster link	&3	&20	&4	\\ \hline

Transparent Alumina (walls, furniture)	&5	&60	&12	\\ \hline

Tree	&2	&40	&10	\\ \hline

Window	&-	&5	&1	\\ \hline

<<<<<<< HEAD \label{tab:sample-objects-structures} ======= \end{tabularx} >>>>>>> 069efc938e743c2e1f7048421dcd3e53a984f73e \label{tab:sample-objects-structures} \end{table} 

\subsection{Range} \label{sec:range} 

Every type of ranged weapon has a limited range, beyond which it is ineffective. The effective range of the weapon is further broken down into four categories: Short, Medium, Long, and Extreme. A modifier is applied for each category, as noted on the Combat Modifiers table, p. 193. For examples of specific weapon ranges, see the Weapon Ranges table. 

\begin{table} <<<<<<< HEAD \begin{tabular}{|l|r|r|r|r|} ======= \begin{tabularx}{\hline}{|X|r|r|r|r|} >>>>>>> 069efc938e743c2e1f7048421dcd3e53a984f73e \hline

\hline{5}{|c|}{\textbf{Weapon ranges}} \\ \hline

\textbf{Weapon (type)} &\textbf{Short} &\textbf{Medium (-10)} &\textbf{Long (-20)} &\textbf{Extreme (-30)}\\ \hline

\hline{5}{|l|}{\emph{Firearms}} \\ \hline

Light Pistol	&0-10	&11-25	&26-40	&41-60	\\ \hline

Medium Pistol	&0-10	&11-30	&31-50	&51-70	\\ \hline

Heavy Pistol	&0-10	&11-35	&36-60	&61-80	\\ \hline

SMG	&0-30	&31-80	&81-125	&126-230	\\ \hline

Assault Rifle	&0-150	&151-250	&251-500	&501-900	\\ \hline

Sniper Rifle	&0-180	&181-400	&401-1100	&1100-2300	\\ \hline

Machine Gun	&0-100	&101-400	&401-1000	&1001-2000	\\ \hline

\hline{5}{|l|}{\emph{Railguns}}\\ \hline

\hline{5}{|l|}{as Firearms but increase the effective range in each category by +50\%} \\ \hline

\hline{5}{|l|}{\emph{Beam Weapons}} \\ \hline

Cybernetic Hand Laser	&0-30	&31-80	&81-125	&126-230 \\ \hline

Laser Pulser	&0-30	&31-100	&101-150	&151-250 \\ \hline

Microwave Agonizer	&0-5	&6-15	&16-30	&31-50 \\ \hline

Particle Beam Bolter	&0-30	&31-100	&101-150	&151-300 \\ \hline

Plasma Rifle	&0-20	&21-50	&51-100	&101-300 \\ \hline

Stunner	&0-10	&11-25	&26-40	&41-60 \\ \hline

\hline{5}{|l|}{\emph{Seekers}} \\ \hline

Seeker Micromissile	&5-70	&71-180	&181-600	&601-2000 \\ \hline

Seeker Minimissile	&5-150	&151-300	&301-1000	&1001-3000 \\ \hline

Seeker Standard Missile	&5-300	&301-1000	&1001-3000	&3001-10000 \\ \hline

\hline{5}{|l|}{\emph{Spray Weapons}} \\ \hline

Buzzer	&0-5	&6-15	&16-30	&31-50\\ \hline

Freezer	&0-5	&6-15	&16-30	&31-50\\ \hline

Shard Pistol	&0-10	&11-30	&31-50	&51-70\\ \hline

Shredder	&0-10	&11-40	&41-70	&71-100\\ \hline

Sprayer	&0-5	&6-15	&16-30	&31-50\\ \hline

Torch	&0-5	&6-15	&16-30	&31-50\\ \hline

Vortex Ring Gun	&0-5	&6-15	&16-30	&31-50\\ \hline

\hline{5}{|l|}{\emph{Thrown Weapons}} \\ \hline

<<<<<<< HEAD Blades &To SOM $\div$ 5 &To SOM $\div$ 2 &To SOM &To SOM x 2 \\ \hline

Minigrenades &To SOM $\div$ 2 &To SOM &To SOM x 2 &To SOM x 3 \\ \hline

Standard Grenades &To SOM $\div$ 5 &To SOM $\div$ 2 &To SOM &To SOM x 3 \\ \hline

\end{tabular} ======= Blades &To SOM $\div$ 5 &To SOM $\div$ 2 &To SOM &To SOM $\hline$ 2 \\ \hline

Minigrenades &To SOM $\div$ 2 &To SOM &To SOM $\times$ 2 &To SOM $\hline$ 3 \\ \hline

Standard Grenades &To SOM $\div$ 5 &To SOM $\div$ 2 &To SOM &To SOM $\hline$ 3 \\ \hline

\end{tabularx} >>>>>>> 069efc938e743c2e1f7048421dcd3e53a984f73e \end{table} 



\subsubsection{Range, gravity and vacuum} 

The ranges listed on the Weapon Ranges table are for Earth-like gravity conditions (1 g). While the effective ranges of kinetic, seeker, spray, and thrown weapons can potentially increase in lower gravity environments due to lack of gravitational forces or aerodynamic drag, accuracy is still the defining factor for determining whether you hit or miss a target. In lower gravities, use the same effective ranges listed, but extend the maximum range by dividing it by the gravity (for example, a max range of 100 meters would be 200 meters in 0.5 g). In microgravity and zero g, the maximum range is effectively line of sight. Likewise, under high-gravity conditions (over 1 g), divide each range category maximum by the gravity (e.g., a short range of 10 meters would be 5 meters in 2 g). 

Beam weapons are not affected by gravity, but they do fare much better in non-atmospheric conditions. Maximum beam weapon range in vacuum is effectively line of sight. 



\subsection{Reach} \label{sec:reach} 

Some weapons extend a character’s reach, giving them a significant advantage over an opponent in melee combat. This applies to any weapon over half a meter long: axes, clubs, swords, shock batons, etc. Whenever one character has a reach advantage over another, they receive a +10 modifier for both attacking and defending. 



\subsection{Scatter} \label{sec:scatter} 

When you are using a blast weapon, you may still catch your target in the blast radius even if you fail to hit them directly. Weapons such as grenades must go somewhere when they miss, and exactly where they land may be important to the outcome of a battle. To determine where a missed blast attack falls, the scatter rules are called into play. 

To determine scatter, roll a d10 and note where the die ``points'' (using yourself as the reference point). This is the direction from the target that the missed blast lands. The die roll also determines how far away the blast lands, in meters. If the MoF on the attack is over 30, this distance is doubled. If the MoF exceeds 60, the distance is tripled. This point determines the epicenter of the blast; resolve the effects of damage against anyone caught within its sphere of effect as normal (see \emph{Blast Effect}, p. 193). 



\subsection{Shock attacks} \label{sec:shock-attacks} 

Shock weapons use high-voltage electrical jolts to physically stun and incapacitate targets. Shock weapons are particularly effective against biomorphs and pods, even when heavily armored. Synthmorphs, bots, and vehicles are immune to shock weapon effects. A biomorph struck with a shock weapon must make a DUR + Energy Armor Test (using their current DUR score, reduced by damage they have taken). If they fail, they immediately lose neuromuscular control, fall down, and are incapacitated for 1 Action Turn per 10 full points of MoF (minimum of 3 Action Turns). During this time they are stunned and incapable of taking any action, possibly convulsing, suffering vertigo, nausea, etc. After this period, they may act but they remain stunned and shaken, suffering a -30 modifier to all actions. This modifier reduces by 10 per minute (so -20 after 1 minute, -10 after 2 minutes, and no modifier after 3 minutes). Many shock weapons also inflict DV, which is handled as normal. 

A biomorph that succeeds the DUR Test is still shocked but not incapacitated. They suffer half the listed DV and suffer a -30 modifier until the end of the next Action Turn. This modifier reduces by 10 per Action Turn. Modifiers from additional shocks are not cumulative, but will boost the modifier back to its maximum value. 



\subsection{Subdual} \label{sec:subdual} 

To grapple an opponent in melee combat, you must declare your intent to subdue before making the die roll. Any appropriate melee skill may be used for the attack; if wielding a weapon, it may be used as part of the grappling technique. If you succeed in your attack with an Excellent Success (MoS of 30+), you have successfully subdued your opponent (for the moment, at least). Grappling attacks do not cause damage unless you roll a critical success (though even in this case you can choose not to). 

A subdued opponent is temporarily restrained or immobilized. They may communicate, use mental skills, and take mesh actions, but they may not take any physical actions other than trying to break free. (At the gamemaster’s discretion, they may make small, restrained physical actions, such as reaching for a knife in their pocket or grabbing an item dropped a few centimeters away on the floor, but these actions should suffer at least a -30 modifier and may be noticed by their grappler). 

<<<<<<< HEAD To break free, a grappled character must take a Complex Action and succeed in either an Opposed Unarmed Combat Test or an Opposed SOM x 3 Test, though the subdued character suffers a -30 modifier on this test. ======= To break free, a grappled character must take a Complex Action and succeed in either an Opposed Unarmed Combat Test or an Opposed SOM $\times$ 3 Test, though the subdued character suffers a -30 modifier on this test. >>>>>>> 069efc938e743c2e1f7048421dcd3e53a984f73e 



\subsection{Suppressive fire} \label{sec:suppressive-fire} 

A character firing a weapon in full-auto mode (p. 198) may choose to lay down suppressive fire over an area rather than targeting anyone specifically, with the intent of making everyone in the suppressed area keep their heads down. This takes a Complex Action, uses up 20 shots, and lasts until the character’s next Action Phase. The suppressed area extends out in a cone, with the widest diameter of the cone being up to 20 meters across. Any character who is not behind cover or who does not immediately move behind cover on their action is at risk of getting hit by the suppressive fire. If they move out of cover inside the suppressed area, the character laying down suppressive fire gets one free attack against them, which they may defend against as normal. Apply no modifiers to these tests except for range, wounds, and full defense. If hit, the struck character must resist damage as if from a single shot. 



\subsection{Surprise} \label{sec:surprise} 

Characters who wish to ambush another must seek to gain the advantage of surprise. This typically means sneaking up on, lying in wait, or sniping from a hardto- perceive position in the distance. Any time an ambusher (or group of ambushers) attempts to surprise a target (or group of targets), make a secret Perception Test for the ambushee(s). Unless they are alert for surprises, this test should suffer the typical -20 modifier for being distracted. This is an Opposed Test against the ambusher(s) Infiltration skill. Depending on the attacker’s position, other modifiers may also apply (distance, visibility, cover, etc.). 

If the Perception Test fails, the character is surprised by the attack and cannot react to or defend against it. In this case, simply give the attacker(s) a free Action Phase to attack the surprised character(s). Once the attackers have taken their actions, roll Initiative as normal. 

If the Perception Test succeeds, the character is alerted to something a split-second before they are ambushed, giving them a chance to react. In this case, roll Initiative as normal, but the ambushed character(s) suffers a -30 modifier to the Initiative Test. The ambushed character may still defend as normal. 

In a group situation, things can get more complicated when some characters are surprised and others aren’t. In this case, roll Initiative as normal, with all non-ambushers suffering the -30 modifier. Any characters who are surprised are simply unable to take action on the first Action Phase, as they are caught off-guard and must take a moment to assess what’s going on and get caught up with the action. As above, surprised characters my not defend on this first Action Phase. 



\subsection{Tactical networks} \label{sec:tactical-networks} 

Tactical networks are specialized software programs used by teams that benefit from the sharing of tactical data. They are commonly used by sports teams, security outfits, military units, AR gamers, gatecrashers, surveyors, miners, traffic control, scavengers, and anyone else who needs a tactical overview of a situation. Firewall teams regularly take advantage of them. 

In game terms, tacnets provide specialized software skills and tools to a muse or AI, as best fits their tactical needs. These tools link together and share and analyze data between all of the participants in the network, creating a customizable entoptics display for each user that summarizes relevant data, highlights interactions and priorities, and alerts the user to matters that require their attention. 

\subsubsection{Combat tacnets} The following list is a sample of a typical combat tacnet’s features. Gamemasters are encouraged to modify and expand these options as appropriate to their game: 

\end{itemize} \item \textbf{Maps:} Tacnets assemble all available maps and can present them to the user with a bird’s eye view or as a three-dimensional interactive, with distances between relevant features readily accessible. The AI or muse can also plot maps based on sensory input, breadcrumb positioning systems (p. 332), and other data. Plotted paths and other data from these maps can be displayed as entoptic images or other AR sensory input (e.g., a user who should be turning left might see a transparent red arrow or feel a tingling sensation on their left side). \item \textbf{Positioning:} The exact positioning of the user and all other participants are updated and mapped according to mesh positioning and GPS. Likewise, the positioning of known people, bots, vehicles, and other features can also be plotted according to sensory input. \item \textbf{Sensory Input:} Any sensory input available to a participating character or device in the network can be fed into the system and shared. This includes data from cybernetic senses, portable sensors, smartlink guncams, XP output, etc. This allows one user to immediately call up and access the sensor feed of another user. \item \textbf{Communications Management:} The tacnet maintains an encrypted link between all users and stays wary both of participants who drop out or of attempts to hack or interfere with the communications link. \item \textbf{Smartlink/Weapon Data:} The tacnet monitors the status of weapons, accessories, and other gear via the smartlink interface or wireless link, bringing damage, shortages, and other issues to the user’s attention. \item \textbf{Indirect Fire:} Members of a tacnet can provide targeting data to each other for purposes of indirect fire (p. 195). \item \textbf{Analysis:} The muses and AIs participating in the tacnet are bolstered with skill software and databases that enable them to interpret incoming data and sensory feeds. Perhaps the most useful aspect of tacnets, this means that the muse/AI may notice facts or details individual users are likely to have overlooked. For example, the tacnet can count shots fired by opponents, note when they are likely running low, and even analyze sensory input to determine the type of weaponry and ammunition being used. Opponents and their gear can also be scanned and analyzed to note potential weaknesses, injuries, and capabilities. If sensor contact with an opponent is lost, the last known location is memorized and potential movement vectors and distances are displayed. Opponent positioning can also identify lines of sight and fields of fire, alerting the user to areas of potential cover or danger. The tacnet can also suggest tactical maneuvers that will aid the user, such as flanking an opponent or acquiring better elevation. \end{itemize} 

Many of these features are immediately accessible to the user via their AR display; other data can be accessed with a Quick Action. Likewise, the gamemaster decides when the muse/AI provides important alerts to the user. At the gamemaster’s discretion, some of these features may apply modifiers to the character’s tests. 



\subsection{Touch-only attack} \label{sec:touch-only-attack} Some types of attacks simply require you to touch your target, rather than injure them, and are correspondingly easier. This might apply when trying to slap them with a dermal drug patch, spreading a contact poison on their skin, or making skin-to-skin contact for the use of a psi sleight. In situations like this, apply a +20 modifier to your melee attacks. 



\subsection{Two-handed weapons} \label{sec:two-handed-weapons} 

Any weapon noted as two-handed requires two hands (or other prehensile limbs) to wield effectively. This applies to some archaic melee weapons (large swords, spears, etc.) in addition to certain larger firearms and heavy weapons. Any character that attempts to use such a weapon single-handed suffers a -20 modifier. This modifier does not apply to mounted weapons. 



\subsection{Weilding two or more weapons} \label{sec:weilding-two-or-more} 

It is possible for a character to wield two weapons in combat, or even more if they are an octomorph or multi-limbed synthmorph. In this case, each weapon that is held in an off-hand suffers a -20 off-hand weapon modifier. This modifier may be offset with the Ambidextrous trait (p. 145). 

\subsubsection{Extra melee weapons} 

The use of two or more melee weapons is treated as a single attack, rather than multiple. Each additional weapon applies +1d10 damage to the attack (up to a maximum +3d10). Off-hand weapon modifiers are ignored. If the character attacks multiple targets with the same Complex Action (see \emph{Multiple Targets}, p. 202), these bonuses does not apply. The attacker must, of course, be capable of actually wielding the additional weapons. A splicer with only two hands cannot wield a knife and a two-handed sword, for example. Likewise, the gamemaster may ignore this damage bonus for extra weapons that are too dissimilar to use together effectively (like a whip and a pool cue). Note that extra limbs do not count as extra weapons in unarmed combat, nor do weapons that come as a pair (such as shock gloves). 

A character using more than one melee weapon receives a bonus for defending against melee attacks equal to +10 per extra weapon (maximum +30). 

\subsubsection{Extra ranged weapons} 

Similarly, an attacker can wield a pistol in each hand for ranged combat, or larger weapons if they have more limbs (an eight-limbed octomorph, for example, could conceivably hold four assault rifles). These weapons may all be fired at once towards the same target. In this case, each weapon is handled as a separate attack, with each off-hand weapon suffering a cumulative off-hand weapon modifier (no modifier for the first attack, -20 for the second, -40 for the third, and -60 for the fourth), offset by the Ambidextrous trait (p. 145) as usual. 

\section{Physical health} \label{sec:physical-health} 

In a setting as dangerous as \emph{Eclipse Phase}, characters are inevitably going to get hurt. Whether your morph is biological or synthetic, you can be injured by weapons, brawling, falling, accidents, extreme environments, psi attacks, and so on. This section discusses how to track such injuries and determine what effect they have on your character. Two methods are used to gauge a character’s physical health: damage points and wounds. 

\subsection{Damage points} \label{sec:damage-points} 

Any physical harm that befalls your character is measured in damage points. These points are cumulative, and are recorded on your character sheet. Damage points are characterized as fatigue, stun, bruises, bumps, sprains, minor cuts, and similar hurts that, while painful, do not significantly impair or threaten your character’s life unless they accumulate to a significant amount. Any source of harm that inflicts a large amount of damage points at once, however, is likely to have a more severe effect (see \emph{Wounds}, p. 207). 

Damage points may be reduced by rest, medical care, and/or repair (see \emph{Healing and Repair}, p. 208). 



\subsection{Damage types} \label{sec:damage-types} 

Physical damage comes in three forms: Energy, Kinetic, and Psi. 

\subsubsection{Energy damage} 

Energy damage includes lasers, plasma guns, fire, electrocution, explosions, and others sources of damaging energy. 

\subsubsection{Kinetic damage} 

Kinetic damage is caused by projectiles and other objects moving at great speeds that disperse their energy into the target upon impact. Kinetic attacks include slug-throwers, flechette weapons, knives, and punches. 

\subsubsection{Psi damage} 

Psi damage is caused by offensive psi sleights like Psychic Stab (p. 228). 



\subsection{Durability and health} \label{sec:durability-health} 

Your character’s physical health is measured by their Durability stat. For characters sleeved in biomorphs, this figure represents the point at which accumulated damage points overwhelm your character and they fall unconscious. Once you have accumulated damage points equal to or exceeding your Durability stat, you immediately collapse from exhaustion and physical abuse. You remain unconscious and may not be revived until your damage points are reduced below your Durability, either from medical care or natural healing. 

If you are morphed in a synthetic shell, Durability represents your structural integrity. You become physically disabled when accumulated damage points reach your Durability. Though your computer systems are likely still functioning and you can still mesh, your morph is broken and immobile until repaired. 



\subsubsection{Death} 

<<<<<<< HEAD An extreme accumulation of damage points can threaten your character’s life. If the damage reaches your Durability x 1.5 (for biomorphs) or Durability x 2 (for synthetic morphs), your body dies. This known as your Death Rating. Synthetic morphs that reach this state are destroyed beyond repair. 

\subsection{Damage value} 

Weapons (and other sources of injury) in \emph{Eclipse Phase} have a listed Damage Value (DV) - the base amount of damage points the weapon inflicts. This is often presented as a variable amount, in the form of a die roll; for example: 3d10. In this case, you roll three ten-sided dice and add up the results (counting 0 as 10). Sometimes the DV will be presented as a dice roll plus modifier; for example: 2d10 + 5. In this case you roll two ten-sided dice, add them together, and then add 5 to get the result. ======= An extreme accumulation of damage points can threaten your character’s life. If the damage reaches your Durability $\times$ 1.5 (for biomorphs) or Durability $\times$ 2 (for synthetic morphs), your body dies. This known as your Death Rating. Synthetic morphs that reach this state are destroyed beyond repair. 

\subsection{Damage value} 

Weapons (and other sources of injury) in \emph{Eclipse Phase} have a listed Damage Value (DV) --- the base amount of damage points the weapon inflicts. This is often presented as a variable amount, in the form of a die roll; for example: 3d10. In this case, you roll three ten-sided dice and add up the results (counting 0 as 10). Sometimes the DV will be presented as a dice roll plus modifier; for example: 2d10 + 5. In this case you roll two ten-sided dice, add them together, and then add 5 to get the result. >>>>>>> 069efc938e743c2e1f7048421dcd3e53a984f73e 

For simplicity, a static amount is also noted in parentheses after the variable amount. If you prefer to skip the dice rolling, you can just apply the static amount (usually close to the mean average) instead. For example, if the damage were noted 2d10 + 5 (15), you could simply apply 15 damage points instead of rolling dice. 

When damage is inflicted on a character, determine the DV (roll the dice) and subtract the modified armor value, as noted under \emph{Step 7: Determine Damage} (p. 192). 

\subsection{Wounds} \label{sec:wounds} 

Wounds represent more grievous injuries: bad cuts and hemorrhaging, fractures and breaks, mangled limbs, and other serious damage that impairs your ability to function and may lead to death or long-term damage. 

Any time your character sustains damage, compare the amount inflicted (after it has been reduced by armor) to your Wound Threshold. If the modified DV equals or exceeds your Wound Threshold, you have suffered a wound. If the inflicted damage is double your Wound Threshold, you suffer 2 wounds; if triple your Wound Threshold, you suffer 3 wounds; and so on. 

Wounds are cumulative, and must be marked on your character sheet. 

Note that these rules handle damage and wounds as an abstract concept. For drama and realism, gamemasters may wish to describe wounds in more detailed and grisly terms: a broken ankle, a severed tendon, internal bleeding, a lost ear, and so on. The nature of such descriptive injuries may help the gamemaster assign other effects. For example, a character with a crushed hand may not be able to pick up a gun, someone with excessive blood loss may leave a trail for their enemies to follow, or someone with a cut eye may suffer an additional visual perception modifier. Likewise, such details may impact how a character is treated or heals. 

\subsubsection{Wound effects} 

Each wound applies a cumulative -10 modifier to all of the character’s actions. A character with 3 wounds, for example, suffers -30 to all actions. Some traits, morphs, implants, drugs, and psi allow a character to ignore wound modifiers. These effects are cumulative, though the maximum amount of wound modifiers that may be negated is -30. 

<<<<<<< HEAD \textbf{Knockdown:} Any time a character takes a wound, they must make an immediate SOM x 3 Test. Wound modifiers apply. If they fail, they are knocked down and must expend a Quick Action to get back up. Bots and vehicles must make a Pilot Test to avoid crashing. 

\emph{Unconsciousness:} Any time a character receives 2 or more wounds at once (from the same attack), they must also make an immediate SOM x 3 Test; wound modifiers again apply. If they fail, they have been knocked unconscious (until they are awoken or heal). Bots and vehicles that take 2 or more wounds at once automatically crash (see \emph{Crashing}, p. 196). ======= \textbf{Knockdown:} Any time a character takes a wound, they must make an immediate SOM $\times$ 3 Test. Wound modifiers apply. If they fail, they are knocked down and must expend a Quick Action to get back up. Bots and vehicles must make a Pilot Test to avoid crashing. 

\emph{Unconsciousness:} Any time a character receives 2 or more wounds at once (from the same attack), they must also make an immediate SOM $\times$ 3 Test; wound modifiers again apply. If they fail, they have been knocked unconscious (until they are awoken or heal). Bots and vehicles that take 2 or more wounds at once automatically crash (see \emph{Crashing}, p. 196). >>>>>>> 069efc938e743c2e1f7048421dcd3e53a984f73e 

\textbf{Bleeding:} Any biomorph character who has suffered a wound and who takes damage that exceeds their Durability is in danger of bleeding to death. They incur 1 additional damage point per Action Turn (20 per minute) until they receive medical care or die. 

\subsection{Death} For many people in \emph{Eclipse Phase}, death is not the end of the line. If the character’s cortical stack can be retrieved, they can be resurrected and downloaded into a new morph (see \emph{Resleeving}, p. 271). This typically requires either backup insurance (p. 269) or the good graces of whomever ends up with their body/stack. 

If the cortical stack is not retrievable, the character can still be re-instantiated from an archived backup (p. 268). Again, this either requires backup insurance or someone who is willing to pay to have them revived. 

If the character’s cortical stack is not retrieved and they have no backup, then they are completely and utterly dead. Gone. Kaput. (Unless they happen to have an alpha fork of themselves floating around somewhere; see \emph{Forking and Merging}, p. 273.) 



\section{Healing and repair} \label{sec:healing-repair} 

Use the follow rules for healing and repairing damaged and wounded characters. 

\subsection{Biomorph healing} 

Thanks to advanced medical technologies, there are many ways for characters in biological morphs (including pods) to heal injuries. Medichine nanoware (p. 308) helps characters to heal quickly, as do nanobandages (p. 333). Healing vats (p. 326) will heal even the most grievous wounds in a matter of days, and can even restore characters who recently died or have been reduced to just a head. 

Characters without access to these medical tools are not without hope, of course. The medical skills of a trained professional can abate the impact of wounds, and over time bodies will of course heal themselves. 

\subsubsection{Medical care} 

Characters with an appropriate Medicine skill (such as Medicine: Paramedic or Medicine: Trauma Surgery) can perform first aid on damaged or wounded characters. A successful Medicine Test, modified as the gamemaster deems fit according to situational conditions, will heal 1d10 points of damage and will remove 1 wound. This test must be made within 24 hours of the injury, and any particular injury may only be treated once. If the character is later injured again, however, this new damage may also be treated. Medical care of this sort is not effective against injuries that have been treated with medichines, nanobandages, or healing vats. 

\subsubsection{Natural healing} 

<<<<<<< HEAD Characters trapped far from medical technology - in a remote station, the wilds of Mars, or the like - may be forced to heal naturally if injured. Natural healing is a slow process that’s heavily influenced by a number of factors. In order for a character to heal wounds, all normal damage must be healed first. Consult the Healing table. ======= Characters trapped far from medical technology --- in a remote station, the wilds of Mars, or the like --- may be forced to heal naturally if injured. Natural healing is a slow process that’s heavily influenced by a number of factors. In order for a character to heal wounds, all normal damage must be healed first. Consult the Healing table. >>>>>>> 069efc938e743c2e1f7048421dcd3e53a984f73e 

\subsubsection{Surgery} 

In \emph{Eclipse Phase}, most grievous injuries can be handled by time in a healing vat (p. 326) or simply rest and recovery. In circumstances where a healing vat is not available, the gamemaster may decide that a particular wound requires actual surgery from an intelligent being (whether a character or AI-driven medbot). Usually in this case the character will be incapable of further healing until the surgery occurs. The surgery is handled as a Medical Test using a field appropriate to the situation and with a timeframe of 1-8 hours. If successful, the character is healed of 1d10 damage and 1 wound and recovers from that point on as normal. 

\begin{table} <<<<<<< HEAD \begin{tabular}{|p{8cm}|r|r|} ======= \begin{tabularx}{\hline}{|X|r|r|} >>>>>>> 069efc938e743c2e1f7048421dcd3e53a984f73e \hline

\hline{3}{|c|}{\textbf{Healing}} \\ \hline

\textbf{Character situation}	&\textbf{Damage healing rate}	&\textbf{Wound healing rate} \\ \hline

Character without basic biomods	&1d10 (5) per day &1 per week	\\ \hline

Character with basic biomods	&1d10 (5) per 12 hours	&1 per 3 days	\\ \hline

Character using nanobandage	&1d10 (5) per 2 hours	&1 per day	\\ \hline

Character with medichines	&1d10 (5) per 1 hour	&1 per 12 hours	\\ \hline

Poor conditions (bad food, not enough rest/heavy activity, poor shelter and/or sanitation) &double timeframe &double timeframe \\ \hline

Harsh conditions (insufficient food, no rest/strenuous activity, little or no shelter and/or sanitation) &triple timeframe &no wound healing \\ \hline

<<<<<<< HEAD \label{tab:healing} ======= \end{tabularx} >>>>>>> 069efc938e743c2e1f7048421dcd3e53a984f73e \label{tab:healing} \end{table} 



\subsection{Synthmorph and Object repair} 

Unlike biomorphs, synthetic morphs and objects do not heal damage on their own and must be repaired. Some synthmorphs and devices have advanced nanotech selfrepair systems, similar to medichines for biomorphs (see Fixers, p. 329). Repair spray (p. 333) may also be used to conduct fixes and is an extremely useful option for non-technical people. Barring these options, technicians may also work repairs the old-fashioned way, using their skills and tools (see Physical Repairs, below). As a last resort, synthmorphs and objects may be repaired in a nanofabrication machine with the appropriate blueprints (using the same rules as healing vats, p. 326). 

\subsubsection{Physical repairs} 

Manually fixing a synthmorph or object requires a Hardware Test using a field appropriate to the item (Hardware: Robotics for synthmorphs and bots, Hardware: Aerospace for aircraft, etc.), with a -10 modifier per wound. Repair is a Task Action with a timeframe of 2 hours per 10 points of damage being restored, plus 8 hours per wound. Appropriate modifiers should be applied, based on conditions and available tools. For example, utilitools (p. 326) apply a +20 modifier to repair tests, while repair spray applies a +30 modifier. 

\subsubsection{Repairing armor} 

Armor may be repaired in the same manner as Durability, however, wounds do not impact the test with modifiers or extra time. 

\section{Mental health} \label{sec:mental-health} 

<<<<<<< HEAD In a time when people can discard bodies and replace them with new ones, trauma inflicted on your mind and ego - your sense of \emph{self} - is often more frightening than grievous physical harm. There are many ways in which your sanity and mental wholeness can be threatened: experiencing physical death, extended isolation, loss of loved ones, alien situations, discontinuity of self from lost memories or switching morphs, psi attack, and so on. Two methods are used to gauge your mental health: \emph{stress points} and \emph{trauma}. ======= In a time when people can discard bodies and replace them with new ones, trauma inflicted on your mind and ego --- your sense of \emph{self} --- is often more frightening than grievous physical harm. There are many ways in which your sanity and mental wholeness can be threatened: experiencing physical death, extended isolation, loss of loved ones, alien situations, discontinuity of self from lost memories or switching morphs, psi attack, and so on. Two methods are used to gauge your mental health: \emph{stress points} and \emph{trauma}. >>>>>>> 069efc938e743c2e1f7048421dcd3e53a984f73e 



\subsection{Stress points} \label{sec:stress-points} 

Stress points represent fractures in your ego’s integrity, cracks in the mental image of yourself. This mental damage is experienced as cerebral shocks, disorientation, cognitive disconnects, synaptic misfires, or an undermining of the intellectual faculties. On their own, these stress points do not significantly impair your character’s functioning, but if allowed to accumulate they can have severe repercussions. Additionally, any source that inflicts a large amount of stress points at once is likely to have a more severe impact (see \emph{Trauma}). 

Stress points may be reduced by long-term rest, psychiatric care, and/or psychosurgery (see p. 214). 



\subsection{Lucidity and stress} \label{sec:lucidity-stress} 

<<<<<<< HEAD Your Lucidity stat benchmarks your character’s mental stability. If you build up an amount of stress points equal to or greater than your Lucidity score, your character’s ego immediately suffers a mental breakdown. You effectively go into shock and remain in a catatonic state until your stress points are reduced to a level below your Lucidity stat. Accumulated stress points will overwhelm egos housed inside synthetic shells or infomorphs just as they will biological brains - the mental software effectively seizes up, incapable of functioning until it is debugged. 

\subsubsection{Insanity rating} 

Extreme amounts of built-up stress points can permanently damage your character’s sanity. If accumulated stress points reach your Lucidity x 2, your character’s ego undergoes a permanent meltdown. Your mind is lost, and no amount of psych help or rest will ever bring it back. ======= Your Lucidity stat benchmarks your character’s mental stability. If you build up an amount of stress points equal to or greater than your Lucidity score, your character’s ego immediately suffers a mental breakdown. You effectively go into shock and remain in a catatonic state until your stress points are reduced to a level below your Lucidity stat. Accumulated stress points will overwhelm egos housed inside synthetic shells or infomorphs just as they will biological brains --- the mental software effectively seizes up, incapable of functioning until it is debugged. 

\subsubsection{Insanity rating} 

Extreme amounts of built-up stress points can permanently damage your character’s sanity. If accumulated stress points reach your Lucidity $\times$ 2, your character’s ego undergoes a permanent meltdown. Your mind is lost, and no amount of psych help or rest will ever bring it back. >>>>>>> 069efc938e743c2e1f7048421dcd3e53a984f73e 



\subsection{Stress value} \label{sec:stress-value} 

Any source capable of inflicting cognitive stress is given a Stress Value (SV). This indicates the amount of stress points the attack or experience inflicts upon a character. Like DV, SV is often presented as a variable amount, such as 2d10, or sometimes with a modifier, such as 2d10 + 10. Simply roll the dice and total the amounts to determine the stress points inflicted in that instance. To make things easier, a static SV is also given in parentheses after the variable amount; use that set amount when you wish to keep the game moving and don’t want to roll dice. 



\subsection{Trauma} \label{sec:trauma} 

Mental trauma is more severe than stress points. Traumas represent severe mental shocks, a crumbling of personality/self, delirium, paradigm shifts, and other serious cognitive malfunctions. Traumas impair your character’s functioning and may result in temporary derangements or permanent disorders. 

If your character receives a number of stress points at once that equals or exceeds their Trauma Threshold, they have suffered a trauma. If the inflicted stress points are double or triple the Trauma Threshold, they suffer 2 or 3 traumas, respectively, and so on. Traumas are cumulative and must be recorded on your character sheet. 

\subsubsection{Trauma effects} 

Each trauma applies a cumulative -10 modifier to all of the character’s actions. A character with 2 traumas, for example, suffers -20 to all actions. These modifiers are also cumulative with wound modifiers. 

<<<<<<< HEAD \textbf{Disorientation:} Any time a character suffers a trauma, they must make an immediate WIL x 3 Test. Trauma modifiers apply. If they fail, they are temporarily stunned and disoriented, and must expend a Complex Action to regain their wits. ======= \textbf{Disorientation:} Any time a character suffers a trauma, they must make an immediate WIL $\times$ 3 Test. Trauma modifiers apply. If they fail, they are temporarily stunned and disoriented, and must expend a Complex Action to regain their wits. >>>>>>> 069efc938e743c2e1f7048421dcd3e53a984f73e 

\emph{Derangements and Disorders:} Any time a character is hit with a trauma, they suffer a temporary derangement (see \emph{Derangements}). The first trauma inflicts a \emph{minor} derangement. If a second trauma is applied, the first derangement is either upgraded from minor to a \emph{moderate} derangement, or else a second minor derangement is applied (gamemaster’s discretion). Likewise, a third trauma may upgrade that derangement from moderate to \emph{major} or else inflict a new minor. It is generally recommended that derangements be upgraded in potency, especially when result from the same set of ongoing circumstances. In the case of traumas that result from distinctly separate situations and sources, separate derangements may be applied. 

\emph{Disorder:} When four or more traumas have been inflicted on a character, a major derangement is upgraded to a disorder. Disorders represent long-lasting psychological afflictions that typically require weeks or even months of psychotherapy and/or psychosurgery to remedy (see \emph{Disorders}, p. 211). 



\subsection{Derangements} \label{sec:derangements} 

Derangements are temporary mental conditions that result from traumas. Derangements are measured as Minor, Moderate, or Major. The gamemaster and player should cooperate in choosing which derangement to apply, as appropriate to the scenario and character personality. 

Derangements last for 1d10 $\div$ 2 hours (round down), or until the character receives psychiatric help, whichever comes first. At the gamemaster’s discretion, a derangement may last longer if the character has not been distanced from the source of the stress, or if they remain embroiled in other stress-inducing situations. 

Derangement effects are meant to be role-played. The player should incorporate the derangement into their character’s words and actions. If the gamemaster doesn’t feel the player is stressing the effects enough, they can emphasize them. If the gamemaster feels it is appropriate, they may also call for additional modifiers or tests for certain actions. 

\subsubsection{Anxiety (minor)} 

You suffer a panic attack, exhibiting the physiological conditions of fear and worry: sweatiness, racing heart, trembling, shortness of breath, headaches, and so on. 

\subsubsection{Avoidance (minor)} 

<<<<<<< HEAD You are psychologically incapable with dealing with the source of the stress, or some circumstance related to it, so you avoid it - even covering your ears, curling up in a ball, or shutting off your sensors if you have to. ======= You are psychologically incapable with dealing with the source of the stress, or some circumstance related to it, so you avoid it --- even covering your ears, curling up in a ball, or shutting off your sensors if you have to. >>>>>>> 069efc938e743c2e1f7048421dcd3e53a984f73e 

\subsubsection{Dizziness (minor)} 

The stress makes you light-headed and disoriented. 

\subsubsection{Echoalia (minor)} 

You involuntarily repeat words and phrases spoken by others. 

\subsubsection{Fixation (minor)} 

You become fixated on something that you did wrong or some circumstance that led to your stress. You obsess over it, repeating the behavior, trying to fix it, running scenarios through your head and out loud, and so on. 

\subsubsection{Hunger (minor)} 

<<<<<<< HEAD You are suddenly consumed by an irrational yet overwhelming desire to eat something - perhaps even something unusual. ======= You are suddenly consumed by an irrational yet overwhelming desire to eat something --- perhaps even something unusual. >>>>>>> 069efc938e743c2e1f7048421dcd3e53a984f73e 

\subsubsection{Indecisiveness (minor)} 

You are flustered by the cause of your stress, finding it difficult to make choices or select courses of action. 

\subsubsection{Logorrhoea (minor)} 

Your response to the trauma is to engage in excessive talking and babbling. You don’t shut up. 

\subsubsection{Nausea (minor)} 

The stress sickens you, forcing you to fight down queasiness. 

\subsubsection{Chills (moderate)} 

Your body temperature rises, making you feel cold, and shivering sets in. You just can’t get warm. 

\subsubsection{Confusion (moderate)} 

The trauma scrambles your concentration, making you forget what you’re doing, mix up simple tasks, and falter over easy decisions. 

\subsubsection{Echpraxia (moderate)} 

You involuntarily repeat and mimic the actions of others around you. 

\subsubsection{Mood swings (moderate)} 

You lose control of your emotions. You switch from ecstasy to tears and back to rage without warning. 

\subsubsection{Mute (moderate)} 

The trauma shocks you into speechlessness and a complete inability to effectively communicate. 

\subsubsection{Narcissism (moderate)} 

In the wake of the mental shock, all you can think about is yourself. You cease caring about those around you. 

\subsubsection{Panic (moderate)} 

You are overwhelmed by fear or anxiety and immediately seek to distance yourself from the cause of the stress. 

\subsubsection{Tremors (moderate)} 

You shake violently, making it difficult to hold things or stay still. 

\subsubsection{Blackout (major)} 

You operate on auto-pilot in a temporary fugue state. Later, you will be incapable of recalling what happened during this period. (Synthetic shells and infomorphs may call up memory records from storage.) 

\subsubsection{Frenzy (major)} 

You have a major freak out over the source of the stress and attack it. 

\subsubsection{Hallucinations (major)} 

You see, hear, or otherwise sense things that aren’t really there. 

\subsubsection{Hysteria (major)} 

You lose control, panicking over the source of the stress. This typically results in an emotional outburst of crying, laughing, or irrational fear. 

\subsubsection{Irrationality (major)} 

You are so jarred by the stress that your capacity for logical judgment breaks down. You are angered by imaginary offenses, hold unreasonable expectations, or otherwise accept things with unconvincing evidence. 

\subsubsection{Paralysis (major)} 

You are so shocked by the trauma that you are effectively frozen, incapable of making decisions or taking action. 

\subsubsection{Psychosomatic crippling (major)} 

The trauma overwhelms you, impairing some part of your physical functioning. You suffer from an inexplicable blindness, deafness, or phantom pain, or are suddenly incapable of using a limb or other extremity. 



\subsection{Disorders} \label{sec:disorders} 

Disorders reflect more permanent madness. In this case, ``permanent�? does not necessarily mean forever, but the condition is ongoing until the character has received lengthy and effective psychiatric help. Disorders are inflicted whenever a character has accumulated 4 traumas. The gamemaster and player should choose a disorder that fits the situation and character. 

<<<<<<< HEAD Disorders are not always ``active�? - they may remain dormant until triggered by certain conditions. While it is certainly possible to act under a disorder, it represents a severe impairment to a person’s ability to maintain normal relationships and do a job successfully. Disorders should not be glamorized as cute role-playing quirks. They represent the best attempts of a damaged psyche to deal with a world that has failed it in some way. Additionally, people in many habitats, particularly those in the inner system, still regard disorders as a mark of social stigma and may react negatively towards impaired characters. ======= Disorders are not always ``active'' --- they may remain dormant until triggered by certain conditions. While it is certainly possible to act under a disorder, it represents a severe impairment to a person’s ability to maintain normal relationships and do a job successfully. Disorders should not be glamorized as cute role-playing quirks. They represent the best attempts of a damaged psyche to deal with a world that has failed it in some way. Additionally, people in many habitats, particularly those in the inner system, still regard disorders as a mark of social stigma and may react negatively towards impaired characters. >>>>>>> 069efc938e743c2e1f7048421dcd3e53a984f73e 

Characters that acquire disorders over the course of their adventures may get rid of them in one of two ways, either through in-game attempts to treat them (p. 214) or by buying them off as they would a negative trait (p. 153). 

\subsubsection{Addiction} 

<<<<<<< HEAD Addiction as a disorder can refer to any sort of addictive behavior focused toward a particular behavior or substance, to the point where the user is unable to function without the addiction but is also severely impaired due to the effects of the addiction. It is marked by a desire on the part of the subject to seek help or reduce the use of the addicting substance/act, but also by the subject spending large amounts of time in pursuit of their addiction to the exclusion of other activities. This is a step up from Addiction negative trait listed on p. 148 - this is much more of a crippling behavior that compensates for spending time away from the addiction. Addictions are typically related to the trauma that caused the disorder (VR or drug addictions are encouraged). ======= Addiction as a disorder can refer to any sort of addictive behavior focused toward a particular behavior or substance, to the point where the user is unable to function without the addiction but is also severely impaired due to the effects of the addiction. It is marked by a desire on the part of the subject to seek help or reduce the use of the addicting substance/act, but also by the subject spending large amounts of time in pursuit of their addiction to the exclusion of other activities. This is a step up from Addiction negative trait listed on p. 148 --- this is much more of a crippling behavior that compensates for spending time away from the addiction. Addictions are typically related to the trauma that caused the disorder (VR or drug addictions are encouraged). >>>>>>> 069efc938e743c2e1f7048421dcd3e53a984f73e 

\textbf{Suggested Game Effects:} The addict functions in only two states: under the influence of their addiction or in withdrawal. Additionally, they spend large amounts of time away from their other responsibilities in pursuit of their addiction. 

\subsubsection{Atavism} 

Atavism is a disorder that mainly affects uplifts. It results in them regressing to an earlier un- or partially- uplifted state. They may exhibit behaviors more closely in line with their more animalistic forbears, or they may lose some of their uplift benefits such as the ability for abstract reasoning or speech. 

\textbf{Suggested Game Effects:} The player and gamemaster should discuss how much of the uplift’s nature is lost and adjust game penalties accordingly. It is important to note that other uplifts view atavistic uplifts with something akin to horror and will usually have nothing to do with them. 

\subsubsection{Attention deficit hyperactivity disorder (ADHD)} 

This disorder manifests as a marked inability to focus on any one task for an extended period of time, and also an inability to notice details in most situations. Sufferers may find themselves starting multiple tasks, beginning a new one after only a cursory attempt at the prior task. ADHD suffers may also have a manic edge that manifests as confidence in their ability to get a given job done, even though they will quickly lose all interest in it. 

\textbf{Suggested Game Effects:} Perception and related skill penalties. Increased difficulty modifiers to task actions, particularly as the action drags on. 

\subsubsection{Autophagy} 

This is a disorder that usually only occurs among uplifted octopi. It is a form of anxiety disorder characterized by self-cannibalism of the limbs. Subjects afflicted with autophagy will, under stress, begin to consume their limbs, if at all possible, causing themselves potentially serious harm. 

<<<<<<< HEAD \textbf{Suggested Game Effects:} Anytime an uplifted octopi with this disorder is placed in a stressful situation they must make a successful WIL x 3 Test or begin to consume one of their limbs. ======= \textbf{Suggested Game Effects:} Anytime an uplifted octopi with this disorder is placed in a stressful situation they must make a successful WIL $\times$ 3 Test or begin to consume one of their limbs. >>>>>>> 069efc938e743c2e1f7048421dcd3e53a984f73e 

\subsubsection{Bipolar disorder} 

Bipolar disorder is also called manic depression. It is similar to depression except for the fact that the periods of depression are interrupted by brief (a matter of days at most) periods of mania where the subject feels inexplicably ``up�? about everything with heightened energy and a general disregard for consequences. The depressive stages are similar in all ways to depression. The manic stages are dangerous since the subject will take risks, spend wildly, and generally engage in behavior without much in the way of forethought or potential long term consequences. 

<<<<<<< HEAD \textbf{Suggested Game Effects:} Similar to depression, but when manic the character must make a WIL x 3 Test to not do some action that may be potentially risky. They will also try to convince others to go along with the idea. 

\subsubsection{Body dysmorphia} 

Subjects afflicted with this disorder believe that they are so unspeakably hideous that they are unable to interact with others or function normally for fear of ridicule and humiliation at their appearance. They tend to be very secretive and reluctant to seek help because they are afraid others will think them vain - or they may feel too embarrassed to do so. Ironically, BDD is often misunderstood as a vanity-driven obsession, whereas it is quite the opposite; people with BDD believe themselves to be irrevocably ugly or defective. A similar disorder, gender identity disorder, where the patient is upset with their entire sexual biology, often precipitates BDD-like feelings. Gender identity disorder is directed specifically at external sexually dimorphic features, which are in constant conflict with the patient’s internal psychiatric gender. ======= \textbf{Suggested Game Effects:} Similar to depression, but when manic the character must make a WIL $\times$ 3 Test to not do some action that may be potentially risky. They will also try to convince others to go along with the idea. 

\subsubsection{Body dysmorphia} 

Subjects afflicted with this disorder believe that they are so unspeakably hideous that they are unable to interact with others or function normally for fear of ridicule and humiliation at their appearance. They tend to be very secretive and reluctant to seek help because they are afraid others will think them vain --- or they may feel too embarrassed to do so. Ironically, BDD is often misunderstood as a vanity-driven obsession, whereas it is quite the opposite; people with BDD believe themselves to be irrevocably ugly or defective. A similar disorder, gender identity disorder, where the patient is upset with their entire sexual biology, often precipitates BDD-like feelings. Gender identity disorder is directed specifically at external sexually dimorphic features, which are in constant conflict with the patient’s internal psychiatric gender. >>>>>>> 069efc938e743c2e1f7048421dcd3e53a984f73e 

\textbf{Suggested Game Effects:} Because of the nature of Eclipse Phase and the ability to swap out and modify a body, this is a fairly common disorder. It is suggested that characters with this suffer increased or prolonged resleeving penalties since they are unable to fully adjust to the reality of their new morph. 

\subsubsection{Borderline personality disorder} 

This disorder is marked by a general inability to fully experience one’s self any longer. Emotional states are variable and often marked by extremes and acting out. Simply put, the subject feels like they are losing their sense of self and seeks constant reassurance from others around them, yet is not fully able to act in an appropriate way. They may also engage in impulsive behaviors in an attempt to experience some sort of feeling. In extreme cases, there may be suicidal thoughts or attempts. 

\textbf{Suggested Game Effects:} The character needs to be around others and will not be left alone, however they also are not quite able to relate to others in a normal way and may also take risks or make impulsive decisions. 

\subsubsection{Depression} 

Clinical depression is characterized by intense feelings of hopelessness and worthlessness. Subjects usually report feeling as though nothing they do matters and no one would care anyway, so they are disinclined to attempt much in the way of anything. The character is depressed and finds it difficult to be motivated to do much of anything. Even simple acts such as eating and bathing can seem to be monumental tasks. 

<<<<<<< HEAD \textbf{Suggested Game Effect:} Depressives often lack the will to take any sort of action, often to the point of requiring a WIL x 3 Test to engage in sustained activity. ======= \textbf{Suggested Game Effect:} Depressives often lack the will to take any sort of action, often to the point of requiring a WIL $\times$ 3 Test to engage in sustained activity. >>>>>>> 069efc938e743c2e1f7048421dcd3e53a984f73e 

\subsubsection{Fugue} 

The character enters into a fugue state where they display little attention to external stimuli. They will still function physiologically but refrain from speaking and stare off into the distance, unable to focus on events around them. Unlike catatonia, a person in a fugue state will walk around if lead about by a helper, but is otherwise unresponsive. The fugue state is usually a persistent state, but it can be an occasional state that is triggered by some sort of external stimuli similar to the original trauma that triggered the disorder. 

\textbf{Suggested Game Effects:} Characters in a fugue state are totally non-responsive to most stimuli around them. They will not even defend themselves if attacked and will usually attempt to curl into a fetal position if physically assaulted. 

\subsubsection{General anxiety disorder (GAD)} 

GAD results in severe feelings of anxiety about nearly everything the character comes into contact with. Even simple tasks represent the potential for failure on a catastrophic scale and should be avoided or minimized. Additionally, negative outcomes for any action are always assumed to be the only possible outcomes. 

\textbf{Suggested Game Effects:} A character with GAD will be almost entirely useless unless convinced otherwise, and then only for a short period of time. Another character can attempt to use a relevant social skill to coax the GAD character into doing what is required of them. If the character with the disorder fails at the task, however, all future attempts to coax them will suffer a cumulative -10 penalty. 

\subsubsection{Hypochondria} 

Hypochondriacs suffer from a delusion that they are sick in ways that they are not. They will create disorders that they believe they suffer from, usually to get the attention of others. Often hypochondriacs will inflict harm on themselves or even ingest substances that will aid in producing symptoms similar to the disorder they believe they have. These attempts to simulate symptoms can and will cause actual harm to hypochondriacs. 

\textbf{Possible Game Effects:} A subject that is hypochondriac will often behave as though they are under the effects of some other disorder or physical malady. This can be consistent over time or can be different and ever changing. They will react with hostility to claims that they are faking or not actually ill. 

\subsubsection{Impulse control disorder} 

Subjects have a certain act or belief that they must engage in a certain activity that comes into their mind. This can be kleptomania, pyromania, sexual exhibitionism, etc. They feel a sense of building anxiety whenever they are prevented from engaging in this behavior for an extended period (usually several times a day to weekly, depending on the impulse) and will often attempt to engage in this behavior at inconvenient or inappropriate times. This is different from OCD in the sense that OCD is usually a single contained behavior that must be engaged in to reduce anxiety. Impulse control disorder is a variety of behaviors and can be virtually any sort of highly inappropriate action. 

\textbf{Suggested Game Effects:} Similar to OCD, if the character doesn’t engage in the behavior they will grow increasingly disturbed and suffer penalties to all actions until they are able to engage in the compulsion that alleviates their anxiety. 

\subsubsection{Insomnia} 

Insomniacs find themselves unable to sleep, or unable to sleep for an extended period of time. This is most often due to anxiety about their lives or as a result of depression and the accompanying negative thought patterns. This is not the sort of sleeplessness that is brought about as a result of normal stress but rather a near total inability to find rest in sleep when it is desired. Insomniacs may find themselves nodding off at inopportune times, but never for long, and never enough to gain any restful sleep. As a result, they are frequently lethargic and inattentive as their lack of sleep robs them of their edge and eventually any semblance of alertness. Additionally, insomniacs are frequently irritable due to being on edge and unable to rest. 

\textbf{Suggested Game Effects: }Due to the lack of meaningful sleep, insomniacs should suffer from blanket penalties to perception related tasks or anything requiring concentration or prolonged fine motor abilities. 

\subsubsection{Megalomania} 

A megalomaniac believes themselves to be the single most important person in the universe. Nothing is more important than the megalomaniac and everything around them must be done according to their whim. Failure to comply with the dictates of a megalomaniac can often result in rages or actual physical assaults by the subject. 

\textbf{Suggested Game Effects:} A character that has megalomania will demand attention and has diffi- culty in nearly any social situation. Additionally, they may be provoked to violence if they think they are being slighted. 

\subsubsection{Multiple personality disorder} 

This is the development of a separate, distinct personality from the original or control personality. The personalities may or may not be aware of each other and ``conscious�? during the actions of the other personality. Usually there is some sort of trigger that results in the emergence of the non-control personality. Most subjects have only a single extra personality, but it is not unheard of to have several personalities. It is important to note that these are distinct individual personalities and not just crude caricatures of the Dr. Jekyll/Mr. Hyde sort. Each personality sees itself as a distinct person with their own wants, needs, and motivations. Additionally, they are usually unaware of the experiences of the others, though there is some basic information sharing (such as language and core skill sets). 

\textbf{Suggested Game Effects:} When the player is under the effects of another personality, they should be treated as an NPC. In some rare cases the player and the gamemaster can work out the second personality and allow the player to roleplay this. This does not however constitute an entire new character that can be ``turned on�? at will. 

\subsubsection{Obsessive compulsive disorder (OCD)} 

Subjects with OCD are marked by intrusive or inappropriate thoughts or impulses that cause acute anxiety if a particular obsession or compulsion is not engaged in to alleviate them. These obsessions and compulsions can be nearly any sort of behavior that must be immediately engaged in to keep the rising anxiety at bay. Players and gamemasters are encouraged to come up with a behavior that is suitable. Examples of common behaviors include repetitive tics (touching every finger of each hand to another part of the body, tapping the right foot twenty times), pathological behaviors such as gambling or eating, or a mental ritual that must be completed (reciting a book passage). 

\textbf{Suggested Game Effects:} If the character doesn’t engage in the behavior they will grow increasingly disturbed and suffer penalties to all actions until they are able to engage in the compulsion that alleviates their anxiety. 

\subsubsection{Post traumatic stress disorder (PTSD)} 

PTSD occurs as a result of being exposed to either a single incident or a series of incidents where the sufferer had their own life, or saw the lives of others, threatened with death. These incidents are often marked by an inability on the part of the victim, either real or perceived, to do anything to alter the outcomes. As a result, they develop an acute anxiety and fixation on these incidents to the point where they lose sleep, become irritated or easily angered, or are depressed over feelings that they lack control in their own lives. 

\textbf{Suggested Game Effects:} Penalties to task actions, also treat situations similar to the initial episodes that caused the disorder as a phobia. 

\subsubsection{Schizophrenia} 

While schizophrenia is generally acknowledged as a genetic disorder that has an onset in early adulthood, it also seems to develop in a number of egos that undergo frequent morph changes. It has been theorized that this is due to some sort of repetitive error in the download process. Regardless, it remains a rare, yet persistent danger of dying and being brought back. Schizophrenia is a psychotic disorder where the subject loses their ability to discern reality from unreality. This can involve delusions, hallucinations (often in support of the delusions), and fragmented or disorganized speech. The subject will not be aware of these behaviors and will perceive themselves as functioning normally, often to the point of becoming paranoid that others are somehow involved in a grand deception. 

\textbf{Suggested Game Effects:} Schizophrenia represents a total break from reality. A character that is schizophrenic may see and hear things and act on those delusions and hallucinations while seeing attempts by their friends to stop or explain to them as part of a wider conspiracy. Adding to this is the difficulty of communicating coherently. Characters that have become schizophrenic are only marginally functional and only for short periods of time until they have the disorder treated. 



\subsection{Stressful situations} \label{sec:stressful-situations} 

The universe of \emph{Eclipse Phase} is ripe with experiences that might rattle a character’s sanity. Some of these are as mundane and human as extreme violence, extended isolation, or helplessness. Others are less common, but even more terrifying: encountering alien species, infection by the Exsurgent virus, or being sleeved inside a non-human morph. 



\subsection{Willpower stress tests} \label{sec:willpower-stress-tests} 

<<<<<<< HEAD Whenever a character encounters a situation that might impact their ego’s psyche, the gamemaster may call for a (Willpower x 3) Test. This test determines if the character is able to cope with the unnerving situation or if the experience scars their mental landscape. If they succeed, the character is shaken but otherwise unaffected. If they fail, they suffer stress damage (and possibly trauma) as appropriate to the situation. A list of stress-inducing scenarios and suggested SVs are listed on the Stressful Experiences table, p. 215. The gamemaster should use these as a guideline, modifying them as appropriate to the situation at hand. 

Note that some incidents may be so horrific that a modifier is applied to the character’s (Willpower x 3) Test. 

\begin{table} \begin{tabular}{|l|l|} \hline

\times{2}{|c|}{\textbf{Healing} } \\ ======= Whenever a character encounters a situation that might impact their ego’s psyche, the gamemaster may call for a (Willpower $\times$ 3) Test. This test determines if the character is able to cope with the unnerving situation or if the experience scars their mental landscape. If they succeed, the character is shaken but otherwise unaffected. If they fail, they suffer stress damage (and possibly trauma) as appropriate to the situation. A list of stress-inducing scenarios and suggested SVs are listed on the Stressful Experiences table, p. 215. The gamemaster should use these as a guideline, modifying them as appropriate to the situation at hand. 

Note that some incidents may be so horrific that a modifier is applied to the character’s (Willpower $\times$ 3) Test. 

\begin{table} \begin{tabularx}{\hline}{|X|l|} \hline

\hline{2}{|c|}{\textbf{Stressful Experiences} } \\ >>>>>>> 069efc938e743c2e1f7048421dcd3e53a984f73e \hline

\textbf{Situation}	&\textbf{SV} \\ \hline

Failing spectacularly in pursuit of a motivational goal	&1d10 $\hline$ 2 (round down)	\\ \hline

Helplessness	&1d10 $\hline$ 2 (round down)	\\ \hline

Betrayal by a trusted friend	&1d10 $\hline$ 2 (round down)	\\ \hline

Extended isolation	&1d10 $\hline$ 2 (round down)	\\ \hline

Extreme violence (viewing)	&1d10 $\hline$ 2 (round down)	\\ \hline

Extreme violence (committing)	&1d10	\\ \hline

Awareness that your death is imminent	&1d10	\\ \hline

Experiencing someone’s death via XP	&1d10	\\ \hline

Losing a loved one	&1d10 $\hline$ 2 (round down)	\\ \hline

Watching a loved one die	&1d10 + 2	\\ \hline

Being responsible for the death of a loved one	&1d10 + 5	\\ \hline

Encountering a gruesome murder scene	&1d10	\\ \hline

Torture (viewing)	&1d10 + 2	\\ \hline

Torture (moderate suffering)	&2d10 + 3	\\ \hline

Torture (severe suffering)	&3d10 + 5	\\ \hline

Encountering aliens (non-sentient)	&1d10 $\hline$ 2 (round down)	\\ \hline

Encountering aliens (sentient)	&1d10	\\ \hline

Encountering hostile aliens	&1d10 + 3	\\ \hline

Encountering highly-advanced technology	&1d10 $\hline$ 2 (round down)	\\ \hline

Encountering Exsurgent-modified technology	&1d10 $\hline$ 2 (round down)	\\ \hline

Encountering Exsurgent-infected transhumans	&1d10	\\ \hline

Encountering Exsurgent life forms	&1d10 + 3	\\ \hline

Exsurgent virus infection	&Varies; see p. 366	\\ \hline

Witnessing psi-epsilon sleights	&1d10 + 2	\\ \hline

<<<<<<< HEAD \label{tab:stressful-experiences} ======= \end{tabularx} >>>>>>> 069efc938e743c2e1f7048421dcd3e53a984f73e \label{tab:stressful-experiences} \end{table} 



\subsection{Hardening} \label{sec:hardening} 

The more you are exposed to horrible or terrifying things, the less scary they become. After repeated exposure, you become hardened to such things, able to shake them off without effect. 

Every time you succeed in a Willpower Test to avoid taking stress from a particular source, take note. If you successfully resist such a situation 5 times, you become effectively immune to taking stress from that source. 

<<<<<<< HEAD The drawback to hardening yourself to such situations is that you grow detached and callous. In order to protect yourself, you have learned to cut off your emotions - but it is such emotions that make you human. You have erected mental walls that will affect your empathy and ability to relate to others. ======= The drawback to hardening yourself to such situations is that you grow detached and callous. In order to protect yourself, you have learned to cut off your emotions --- but it is such emotions that make you human. You have erected mental walls that will affect your empathy and ability to relate to others. >>>>>>> 069efc938e743c2e1f7048421dcd3e53a984f73e 

Each time you harden yourself to one source of stress, your maximum Moxie stat is reduced by 1. Psychotherapy may be used to overcome such hardening, in the same way a disorder is treated. 



\subsection{Mental healing and psychotherapy} \label{sec:mental-healing-psychotherapy} 

Stress is trickier to heal than physical damage. There are no nano-treatments or quick fix options (other than killing yourself and reverting to a non-stressed backup). The options for recuperating are simply natural healing over time, psychotherapy, or psychosurgery. 

\subsubsection{Phychotherapy care} 

<<<<<<< HEAD Characters with an appropriate skill - Medicine: Psychiatry, Academics: Psychology, or Professional: Psychotherapy - can assist a character suffering mental stress or trauma with psychotherapy. This treatment is a long-term process, involving methods such as psychoanalysis, counseling, roleplaying, relationship-building, hypnotherapy, behavioral modification, drugs, medical treatments, and even psychosurgery (p. 229). AIs skilled in psychotherapy are also available. ======= Characters with an appropriate skill --- Medicine: Psychiatry, Academics: Psychology, or Professional: Psychotherapy --- can assist a character suffering mental stress or trauma with psychotherapy. This treatment is a long-term process, involving methods such as psychoanalysis, counseling, roleplaying, relationship-building, hypnotherapy, behavioral modification, drugs, medical treatments, and even psychosurgery (p. 229). AIs skilled in psychotherapy are also available. >>>>>>> 069efc938e743c2e1f7048421dcd3e53a984f73e 

Psychotherapy is a task action, with a timeframe of 1 hour per point of stress, 8 hours per trauma, and 40 hours per disorder. Note that this only counts the time actually spent in psychotherapy with a skilled professional. After each psychotherapy session, make a test to see if the session was successful. Successful psychosurgery adds a +30 modifier to this test; at the gamemaster’s discretion, other modifiers may apply. Likewise, each disorder the character holds inflicts a -10 modifier. Traumas may not be healed until all stress is eliminated. 

When a trauma is healed, the derangement associated with that trauma is eliminated or downgraded. Disorders are treated separately from the trauma that caused them, and may only be treated when all other traumas are removed. 

Gamemaster and players are encouraged to roleplay a character’s suffering and relief from traumas and disorders. Each is an experience that makes a profound impact on a character’s personality and psyche. The process of treatment may also change them, so in the end they may be a transformed from the person they once were. Even if treated, the scars are likely to remain for some time to come. According to some opinions, disorders are never truly eradicated, they are just eased into submission ... where they may linger beneath the surface, waiting for some trauma to come along. 

\subsubsection{Natural healing} 

<<<<<<< HEAD Characters who eschew psychotherapy can hopefully work out the problems in their head on their own over time. For every month that passes without accruing new stress, the character may make a WIL x 3 Test. If successful, they heal 1d10 points of stress or 1 trauma (all stress must be healed first). Disorders are even more difficult to heal, requiring 3 months without stress or trauma, and even then only being eliminated with a successful WIL Test. As a result, disorders can linger for years until resolved with actual psychotherapy. ======= Characters who eschew psychotherapy can hopefully work out the problems in their head on their own over time. For every month that passes without accruing new stress, the character may make a WIL $\times$ 3 Test. If successful, they heal 1d10 points of stress or 1 trauma (all stress must be healed first). Disorders are even more difficult to heal, requiring 3 months without stress or trauma, and even then only being eliminated with a successful WIL Test. As a result, disorders can linger for years until resolved with actual psychotherapy. >>>>>>> 069efc938e743c2e1f7048421dcd3e53a984f73e 



