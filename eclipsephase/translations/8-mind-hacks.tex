\chapter{Piratages Cognitifs} \label{cha:mind-hacks} 













\section{PSI} 

\begin{quotation} 

$\triangleright $ Desdemona: Contente de te revoir. J'espère que tu as eu un agréable farcast depuis Pelion et que tu ne ressent pas trop de manque. Pendant que tu étais en ballade, un message d'Aeneas accompagné d'un précis sur les psi, extrait du backup de l'infomorph du psygénéticien Daborva(Stellint, Station de Recherche Dipôle sur ganymède), a été rerouté pour diffusion sur ton nœud Firewall. \end{quotation} 

Inventé par le biologiste Bertold P. Wlesner, "psi" était originellement un furre-tout utilisé pour décrire un grand nombre d'aptitude "pyschique" et d'autres phénomènes apranormaux supposés tels que la télépathie et la perception extra-sensorielle. Alors que le terme était largement utilisé dans le domain de la parapsychologie et dans la pop culture au vingtième siècle et au début du vingt et unième siècle, l'étude de psi était largement considéré comme une pseudoscience utilisant une méthodologie vciée et ayant graduellement perdu ses supports et ses financements. pendant la Chute, cependant, des rumeurs répétées et nombre de phénomènes inexpliqués furent rapportés aux scientifiques, aux chefs militaires et aux adeptes de la singularité. De nombreux nanovirus ont avaient étés lachés sur la transhumanité, se propageant à travers les populations et se transformant au fur et à mesure de leur propagation. Certains n'infligèrent que des changements biologiques ou psychologiques mineurs et des déficiences légères, mais un grand nombre d'entre eux étaient virulents et mortels. La variantes qui inspira le plus de crainte étaient celles que Firewall allait appelé le Virus Exsurgent - une nano-peste transformatrice qui fait muter ses victime et les asservit à sa volonté. Il a également été observé que le Virus Exsurgent modifiait radicallement les motifs neuronaux et l'état mental du sujet, affectant l'ordonnancement synaptique et allant jusqu'à moduler les courants électriques qui parcourent les synapses. Ces changements altèrent et amplifie la cognition de la victime et semble les doter d'une capacité à percevoir et même affecter les pensées des autres à une faible distance - une capacité appellée "psi" car les facteurs de causalités continuent de nous mystifier. L'existence et la nature de ce phénomène restent prudemment dissimulés et tenue à l'abri dans les habitats contrôllés, afin de ne pas déclencher de panique générale. Aprmi les anarchistes et d'autres communautées ouvertes, la connaissance du psi est plus répandue, mais les détails restent vagues et les rapports sont généralement acceuillis avec scepticisme. Le virus Exsurgent est cependant extraordinairement mutable et adaptable, et deux chercheurs argonautes qui étaient au courant et qui étudiaent le phénomène psi ont rapidement fait une découverte intéressante. Une souche particulière du virus qui dotait e sujet de capacités mental exceptionnelles s'évarait également ne pas engager le sujet dans le processus de transformation des autres souches. Bien que l'infection ait toujours d'autres inconvénients, Firewall et d'autres agents en sont venus à considérés cette souche comme "sûre" dans le sens où le sujet ne se métamorphose pas en quelque chose d'autres et que leur personnalité principale reste intacte. Intrigué par le fait que cette piste puisse mener à invalider les effets d'autres souches Exsurgente, Firewall et d'autres continuèrent d'expérimenter avec la souche et avec la coopération de sujets de test volontaires (ou, d'après certains rapport, de victimes involontaires dans le cas de certaines autoritées et hypercops). 

\subsection{La Nature du Psii} Notée la souche Watts-MacLeod d'après le nome des chercheurs qui l'ont sisolés, d'autres études ont acquis une connaissance des effetc du virus sur les cerveaux transhumains. Des analyses minutieuse de sujest infectés ont permis de dcouvrir que leurs synapses altérées génèrent une forme d'onde cérébrale modulée qui est extrêmement difficile à détecter. "Ceux qui savent" en sont venus à faire référence à ces ondes cérébrales asynchrones sou sle nom de "ondes psi," en suivant la désignation par lettre Grecque  des autres ondes cérébrales (alpha, béta, delta, gamma, théta). De la même manière, les individus affectés sont appelés "async." L'exploration des facteurs causals explicites à l'origine des ondes psi restent inconnues. Des théories à propos du processus mental extraordinaire qui incluent le changement d'état quantique ont été explorée mais reste non-concluantes. Neuroimaging and mapping have enabled scientists to pinpoint structures within the brain, neural activity, and perturbations in the brain’s bioelectric field that are associated with psi processes, but attempts to duplicate these features in non-infected brains have resulted in failure or worse. Attempts to identify asyncs by psi brainwave patterns are not even assured of success. Numerous dead ends have prompted many researchers to postulate that the mechanics underlying psi are simply too strange and too far beyond transhumanity’s understanding of physical sciences—perhaps reinforcing theories that the Exsurgent virus is in fact of alien origin. One leading speculation is that the changes wrought in the mind by infection actually entangle some of the neural sub-systems, enable some sort of quantum field within the brain, or possibly create Bose-Einstein condensates within the brain, allowing for quantum computation or perhaps hypercomputation. This enhances the async’s mental capabilities to the level provided by modern implants and neuro-mods—and sometimes beyond. This does not explain the capabilities of other asyncs, however, especially those used to read or affect other biological minds. These abilities seem to involve reading brain waves from a short range or affecting another’s mind via direct physical contact with the target’s bio-electric fields. Of course I can only speculate in accordance with what Firewall has uncovered—it is quite possible that certain hypercorps or other factions have made further breakthroughs, but are keeping the information to themselves. The initiation and use of psi talents is generally understood to take place on a subconscious level, meaning that the async is not actively aware of the fundamental processes that fuel the psi-waves. Training in certain skills, however, allows an async to focus on certain tasks and psi abilities. These are called “sleights:” mnemonic or cognitive algorithms of psi use rooted in the async’s ego. The percentage of the transhuman population believed to have contracted the Watts-MacLeod strain remains statistically insignificant—less than .001\% of the population. The vast number of asyncs have been recruited by various agencies, “disappeared” for study, or simply eliminated as a potential threat. Ten years after the Fall, Firewall and other agencies have come to regard Watts-MacLeod infection as comparatively safe, though we remain quite wary of unforeseen side effects or other hidden dangers. Most of us engaged in studying the phenomenon now consider asyncs to be useful as a tool for fighting the Exsurgent virus and other threatsdespite the protests of those who are convinced that asyncs are not in control of their own minds and are not to be trusted. As of yet we have encountered no cases of Watts-MacLeod infection that have inflicted anything other than psi abilities, though there seems to be an increased risk for asyncs to succumb to other Exsurgent strains should they encounter them. There are other risks associated with Watts-MacLeod infection, such as extreme fatigue and even lethal biofeedback resulting from extensive use of psi sleights and a statistically likelihood of developing mental disorders due to the increased mental stress placed on the async’s mind. 

\newpage

\#{Æther Jabber: Asyncs} \# Start Æther Jabber \# \\ \# Active Members: 2 \# 

\textit{\textit{1}} Sorry to bother you, but my muse just alerted me to this excerpt that was sent around to my Firewall team. Is this for real? I’ve heard the talk about psi before—enough to be convinced that there’s something to it, even if we can’t explain it—but this bit about variant Exsurgent infection is too much. Are we seriously going to be working with someone who’s a known carrier? And can you shed any more light on how asyncs do their mojo? I’m worried now. And since you are connected to the Medeans, I thought I’d take the chance and ask. 

\textit{\textit{2}} Well, as to the Medeans ... that’s history. I am back on the freelancing market right now. But no problem, I’ll try and explain. I know it is not easy to grasp. 

\textit{\textit{1}} Shiny. 

\textit{\textit{2}} Yes, Srit was once infected with a strain of the Exsurgent virus, probably on Mars near the end of the Fall. I say “was” because the Watts-MacLeod strain seems to go dormant shortly after it finishes rewiring the victim’s brain; the plague nanobots die off and get flushed out of the system, unlike other Exsurgent strains, which continue to stick around and transform the subject. At least, that’s the dominant theory—I’ve also seen some speculation that async minds might be modified so that they continue to produce bio-nanobots that linger in the brain, though what function these serve remains unclear. However, the prevailing opinion among our best neuroscientists is that people like Srit are safe and non-infectious once the virus has run its course. I’ll even go a bit further and say that prevailing opinion is that they can be trusted, assuming they don’t catch another infection ... which they unfortunately seem to be a bit prone too. Not everyone agrees of course, but we have an abundance of paranoia in our circles. So far, we haven’t seen any evidence that any of our asyncs have been turned by that initial infection, and the utility and usefulness of having psiactives on our side has simply been too important to push aside. Très bien. I can’t say that I’ll trust her, but I’ll try and give her the benefit of the doubt. I’ll be damned if I’m going to trust an async that’s not vouched for by Firewall though—who knows what the hell a hypercorp like Skinthetic might be cooking up in their black labs. 

\textit{\textit{2}} That seems like a wise choice. 

\textit{\textit{1}} Maybe you can put my mind at ease by explaining to me in a bit more detail how Watts-MacLeod infection occurs. 

\textit{\textit{2}} Well, like the other Exsurgent strains you are unfortunately familiar with, the primary transmission vector is a nanovirus, but we speculate that it may also be transmitted as a digital computer virus or possibly even as a basilisk hack. The physical plague form is spread by highly-advanced techno-organic nanobots that infect a biomorph and use bio-mimicry mechanisms to pass as normal cells and penetrate the blood-brain barrier and central nervous system. The nanobots are several steps beyond anything our technology can produce, are very difficult to detect, and can overwhelm most defensive countermeasures. Infected minds are essentially rewired, and these changes will be copied when the ego is uploaded. Synthmorphs and infomorphs remain immune to this nano-infection, but they are theoretically vulnerable to other transmission vectors. 

\textit{\textit{1}} I’ve heard that synthmorphs are effectively invulnerable to psi as well. This true? 

\textit{\textit{2}} Yes. As far as we can tell, async abilities only effect biological minds—either their own or others. And they can only read/affect others from a very short distance, requiring physical contact in most cases. The half-biological minds of pods are also vulnerable, though to a lesser extent. Likewise, asyncs need a biological brain to use their abilities—they can’t use their psi if sleeved in a synthmorph and have difficulty in a pod. 

\textit{\textit{1}} Interesting. So, I have to ask again— you’re sure she’s safe? I’ve heard that some of these asyncs can be real nut-jobs. 

\textit{\textit{2}} I’ve heard from several of these asyncs directly. The fact is, infection rewrites their brain, and some of them came out the other side feeling fundamentally altered. Either they felt like a different person, or they felt like there was something new that was part of them—something that they didn’t necessarily like. One described it as presence, another as a black void that whispered at them. Yet another described it as giving a personality to their unconscious mind, which only made the gulf between unconscious and conscious mind all the more intimidating. Some of them preferred to suicide and revert to a pre-infection backup. While they may be more prone to cracking up as a result, I haven’t ever heard one talk about their abilities as something they couldn’t control. 

1 Well, that’s fucking cheery. There’s nothing else we have on how this psi stuff actually works? 

\textit{\textit{2}} Unfortunately, we don’t. Even the Prometheans haven’t been much help. There are theories, of course, but nothing that we’ve been able to verify with rigorous experimentation. It doesn’t help that the factions that are aware of psi’s existence don’t exactly compare notes—they’re all too busy looking into ways to weaponize it and use it against each other, instead of figuring out how to use it for the benefit of transhumanity. 

\textit{\textit{1}} Of course. The TITANs didn’t get us, but we can still get ourselves. It worries me that the best we’ve come up with is nothing. 

\textit{\textit{2}} It’s important to keep perspective. Transhumanity has come quite a distance and made some impressive accomplishments, but our understanding of the universe is still in its infancy. What we may be facing here is something concocted by an intelligence so far beyond our own that we are but insignificant insects in comparison. It likely has a grasp on the universe that is simply beyond our ability to understand. We shouldn’t be cocky and think that we can decipher any mystery thrown at us ... we should instead be very, very afraid. 

\newpage



Though neuroscience has ascended to impressive pinnacles, allowing minds to be thoroughly scanned, mapped, and emulated as software, the transhuman brain remains a place that is complicated, not fully understood, and thoroughly messy. Despite a prevalence of neural modifications, meddling with the seat of consciousness remains a tricky and hazardous procedure. Nevertheless, psychosurgery—editing the mind as software—remains common and widespread, sometimes with unexpected results. Likewise, even as the knowledge of neuroscientists grows on an exponential basis, some are discovering that minds are far more mysterious than they had ever imagined. During the Fall, scattered reports of “anomalous activity” by individuals infected by one of the numerous circulating nanoplagues were discounted as fear and paranoia, but subsequent investigations by black budget labs has proven otherwise. Now, toplevel confidential networks whisper that this infection inflicts intricate changes in the victim’s neural network that imbue them with strange and inexplicable abilities. The exact mechanism and nature of these abilities remains unexplained and outside the grasp of modern transhuman science. Given the evidence of a new brainwave type and the paranormal nature of this phenomenon, it is loosely referred to as “psi.” 

\section{Psi} In Eclipse Phase, psi is considered a special cognitive condition resulting from infection by the mutant—and hopefully otherwise benign—Watts-Macleod strain of the Exsurgent virus (p. 367). This plague modifies the victim’s mind, conferring special abilities. These abilities are inherent to the brain’s architecture and are copied when the mind is uploaded, allowing the character to retain their psi abilities when changing from morph to morph. 

\subsubsection{Prerequisites} 

To wield psi, a character must acquire the Psi trait (p. 147) during character creation. It is theoretically also possible to acquire the use of psi in game via infection by the Watts-MacLeod strain; see The Exsurgent Virus, p. 362. Psi ability is considered an innate ability of the ego and not a biological or genetic predisposition of the morph. While psi researchers do not understand how it is possible to transfer this ability via uploads, backups, and farcasting, it has been speculated that all components of an async’s ego are entangled on a quantum level, or that they possess the ability to entangle themselves or form a unique conformation or alignment as a whole even after they have been copied, up-, or downloaded. This speculated entanglement process is also thought to be the origin of the impairment that asyncs experience when adapting to a new morph (see below). 

\subsubsection{Morphs And Psi} 

Asyncs require a biological brain to draw on their abilities (the brains of uplifted animals count). An async whose ego is downloaded into an infomorph or fully computerized brain (synthmorphs) has no access to their abilities as long they remain in that morph. Asyncs inhabiting a pod morph may use psi, but their abilities are restricted as pod brains are only partly biological. Pod-morphed asyncs suffer a –30 modifier on all tests involving the use of psi sleights and the impact from using sleights would be doubled. 

\subsubsection{Morph Acclimatization} 

Async minds undergo extra difficulty adjusting to new morphs. For 1 day after the character has resleeved, they will suffer the effects of a single derangement (p. 210). The gamemaster and player should choose a derangement appropriate to the character and story. Minor derangements are recommended, but at the gamemaster’s discretion moderate or major derangements may be applied. No trauma is inflicted with this derangement. 

\subsubsection{Morph Fever} 

Asyncs find it irritating and traumatizing to endure life as an infomorph, pod, or synthmorph for long periods of time. This phenomenon, known as morph fever, might cause temporary derangements and trauma to the asyncs’ ego, possibly even to the grade of permanent disorders. If stored or held captive as an active infomorph (i.e. not in virtual stasis), the async might go insane if not psychologically aided by some sort of anodyne program or supporting person during storage. In game terms, asyncs take 1d10 $\div$ 2 (round up) points of mental stress damage per month they stay in a pod, synthmorph or infomorph form without psychological assistance by a psychiatrist, software, or muse. 

\subsubsection{Psi Drawbacks} 

There are several drawbacks to psi ability: 

\end{itemize} \item The variant Exsurgent strain that endows psi ability rewires the character’s brain. An unfortunate side effect to this change is that asyncs acquire a vulnerability to mental stress. Reduce the async’s Trauma Threshold by 1. \item The mental instability that accompanies psi infection also tends to unhinge the character’s mind. Asyncs acquires one Mental Disorder negative trait (p. 150) for each level they have of the Psi trait without receiving any bonus CP. The gamemaster and player should agree on a disorder appropriate to the character. This disorder may be treated over time, according to normal rules (see Mental Healing and Psychotherapy, p. 215). \item Characters with the Psi trait are also vulnerable to infection by other strains of the Exsurgent virus. The character suffers a –20 modifier when resisting Exsurgent infection (p. 362). \item Critical failures when using psi tend to stress the async’s mind. Each time a critical failure is rolled when making a sleight-related test, the async suffers a temporary brain seizure. They suffer a –30 modifi er and are incapable of acting until the end of the next Action Turn. They must also succeed in a WIL + COG Test or fall down. \end{itemize} 

\subsubsection{Psi Skills And Sleights} 

Transhuman psi users can manipulate their egos and otherwise create effects that can often be neither matched nor mimicked by technological means. To use these abilities, they train their mental processes and practice cognitive algorithms called sleights, which they can subconsciously recall and use as necessary. Sleights fall into two categories: psi-chi (cognitive enhancements, p. 223) and psi-gamma (brainwave reading and manipulation, p. 225). Psi-chi sleights are available to anyone with the Psi trait (p. 147), but psigamma sleights are only available to characters with the Psi trait at Level 2. In order to use these sleights, the async must be skilled in the Control (p. 178), Psi Assault (p. 183), and/or Sense skills (p. 184), as appropriate to each sleight. 

\subsubsection{Roleplaying Asyncs} 

Any player who chooses to play an async should keep the origin of their abilities in mind: Watts-MacLeod strain infection. The character may not be aware of this source, but they undoubtedly know that they underwent some sort of transformation and have talents that no one else does. If unaware of the infection, they have likely learned to keep their abilities secret lest they be ridiculed, attacked, or whisked away to some secret testing program. Learning the truth about their nature could even be the starting point of a campaign and/or their introduction to Firewall. If they know the truth, however, the character must live with the fact that they are the victim of a nanoplague likely spread by the TITANs that may or may not lead to complications, side effects, or other unexpected revelations in their future. Gamemasters and players should make an effort to explore the nature of this infection and how the character perceives it. As noted previously, asyncs are often profoundly-changed people. The invasive and alien aspect of their abilities should not be lost on them. For example, an async might conceive of their psi talents as a sort of parasitic entity, living off their sleights, or they might feel that using these powers puts them in touch with some sort of fundamental substrate of the universe that is weird and terrifying. Alternately, they could feel as if their personality was melded with something different, something that doesn’t belong. Each async is likely to view their situation differently, and none of them pleasantly. 

\subsection{Using Psi} 

Using psi—i.e., drawing on a certain sleight to procure some kind of effect—does not always require a test. Each sleight description details how the power is used. 

\subsubsection{Active Psi} 

Active psi sleights must be “activated” to be used. These sleights usually require a skill test. Sleights that target other sentient beings or life forms are always Opposed Tests, while others are handled as Success Tests. The level of concentration required to use these sleights varies, and so may call for a Quick, Complex, or Task Action. Active sleights also cause strain (p. 223) to the async. Most psi-gamma sleights fall into this category. 

\subsubsection{Passive Psi} 

Passive psi sleights encompasses abilities that are considered automatically active and subconscious. They rarely require an action to be activated and require no effort or strain by the psi user. Passive sleights typically add bonuses to various activities or allow access to certain abilities rather than calling for some kind of skill test. Most psi-chi sleights fall into this category. 

\subsubsection{Psi Range} 

Sleights have a Range of either Self, Touch, or Close. Self: These sleights only affect the async. Touch: Sleights with a Touch range may be used against other biological life, but the async must have physical contact with the target. If the target avoids being touched, this requires a successful melee attack, applying the touch-only +20 modifier. This attack does not cause damage, and is considered part of the same action as the psi use. Close: Close sleights involve interaction with other biological life from a short distance. The optimal distance is within 5 meters. For each meter beyond that, apply a –10 modifier to the test. Psi vs. Psi: Due to the nature of psi, sleights are more effective against other psi users. Sleights with a range of Touch may be used from a Close range against another async. Likewise, a sleight with a Close range may be used at twice the normal distance (10 meters) when wielded on another async. 

\subsubsection{Targeting} 

Synthmorphs, bots, and vehicles may not be targeted by psi sleights, as they lack biological brains. Pods - with brains that are half biological and half computer - are less susceptible and receive a +30 modifier when defending against psi use. Note that infomorphs may never be targeted by psi sleights as psi is not effective within the mesh or simulspace. Multiple Targets: An async may target more than one character with a sleight with the same action, as long as each of them can be targeted via the rules above. The psi character only rolls once, with each of the defending characters making their Opposed Tests against that roll. The psi character suffers strain (p. 223) for each target, however, meaning that using psi on multiple targets can be extremely dangerous. Animals and Less Complex Life Forms: Psi works against any living creature with a brain and/or nervous system. Against partially-sentient and partially- uplifted animals, it suffers a –20 modifier and increases strain by +1. Against non-sentient animals, it suffers a –30 modifier and increases strain by +3. It has no effect on or against less complex life forms like plants, algae, bacteria, etc. Factors and Aliens: At the gamemaster’s discretion, psi sleights may not work on alien creatures at all, depending on their physiology and neurology. If it does work, it is likely to suffer at least a –20 modifier and +1 strain. 

\subsubsection{Opposed Tests} 

Psi that is used against another character is resisted with an Opposed Test. Defending characters resist with WIL x 2. Willing characters may choose not to resist. Unconscious or sleeping characters cannot resist. If the psi-wielding character succeeds and the defender fails, the sleight affects the target. If the psi user fails, the defender is unscathed. If both parties succeed in their tests, compare their dice rolls. If the psi user’s roll is higher, the sleight bypasses the defender’s mental block and affects the target; otherwise, the sleight fails to affect the defender’s ego. 

\subsubsection{Target Awareness} 

The target of a psi sleight is aware they are being targeted any time they succeed on their half of the Opposed Test (regardless on whether the async rolls higher or not). Note that awareness does not necessarily mean that the target understands that psi abilities are being used on them, especially as most people in Eclipse Phase are unaware of psi’s existence. Instead, the target is simply likely to understand that some outside influence is at work, or that something strange is happening. They may suspect that they have been drugged or are under the influence of some strange technology. Targets who fail their roll remain unaware. 

\subsubsection{Psi Full Defense} 

Like full defense in physical combat (p. 198), a defender may spend a Complex Action to rally and concentrate their mental defenses, gaining a +30 modifier to their defense test against psi use until their next Action Phase. 

\subsubsection{Criticals} 

If the defender rolls a critical success, the character attempting to wield psi is temporarily locked out of the target’s mind. The psi user may not target that character with sleights until an appropriate “reset” period has passed, determined by the gamemaster. If the async rolls a critical failure, they suffer temporary incapacitation as their mind dysfunctions in some harsh and distressing ways (see Psi Drawbacks, p. 221). If a psi user rolls a critical success against a defender, or the defender rolls a critical failure, double the potency of the sleight’s effect. In the case of psi attacks, the DV can be doubled or mental armor can be bypassed. Alternately, when using Psi Assault (p. 183), the targeted character may be in danger of infection by the Watts-Macleod strain (p. 362). 

\subsubsection{Mental Armor} 

The Psi Shield sleight (p. 228) provides mental armor, a form of neural hardening against psi-based attacks. Like physical armor, this mental armor reduces the amount of damage inflicted by a psi assault. 

\subsubsection{Duration} 

Psi sleights have one of four durations: constant, instant, temporary, or sustained. Constant: Constant sleights are always “on.” Instant: Instant sleights take effect only in the Action Phase in which they are used. Temporary: Temporary sleights last for a limited duration with no extra effort from the async. The temporary duration is determined by the async’s WIL $\div$ 5 (round up) and is measured in either Action Turns or minutes, as noted. Strain for the sleight is applied immediately when used, not at the end of the duration. Sustained: Sustained sleights require active effort to maintain for as long as the async wants to keep it active. Sustaining a sleight requires concentration, and so the async suffers a –10 modifier to all other skill tests while the sleight is sustained. The async must also stay within the range appropriate to the sleight, otherwise the sleight immediately ends. More than one sleight may be sustained at a time, with a cumulative modifier. Strain for the sleight is applied immediately when used, not at the end of the duration. At the gamemaster’s discretion, sleights that are sustained for long periods may incur additional strain. 

\subsubsection{Strain} 

The use of psi is physically (and sometimes psychologically) draining to a psi user. This phenomenon is known as strain, and manifests as fatigue, exhaustion, pain, neural overload, cardiovascular stress, and adynamia (loss of vigor). Though strain has only rarely been known to actually kill an async, the use of too much active psi can be life-threatening in some circumstances. In game terms, every active sleight has a Strain Value of 1d10 $\div$ 2 (round up) DV. Every active sleight lists a Strain Value Modifier that modifies this amount. For example, a sleights with a Strain Value Modifier of –1 inflicts (1d10 $\div$ 2) –1 DV. If the damage points suffered from strain exceed the character’s Wound Threshold, they may inflict a wound just like other damage (see Wounds, p. 207). 

\begin{quotation} \textbf{Example} \\ Matric is investigating a disappearance, so he decides to use his Qualia sleight to boost his Intuition while hunting for clues. That psi-chi sleight takes only a Quick Action to initiate and requires no test. Matric’s WIL is 25, so the duration of this temporary sleight is 5 Action Turns (25 $\div$ 5 = 5). The sleight’s Strain modifier is –1, so he is facing (1d10 $\div$ 2) –1 DV. He rolls a 1, so he takes no strain at all! Later on, Matric finds himself in a life-or-death struggle with a kidnapper. Lucky for Matric, they’re in a melee, so he’s close enough to try and touch his opponent. On his Action Phase, he makes an Unarmed Combat Test with a +20 modifi er (for a touch-only attack) and succeeds. This allows him to try and use his Psychic Stab sleight. He rolls his Psi Assault of 57 against the target’s WIL x 2 (32). His target is in a worker pod morph, however, which is less susceptible to psi, so he receives a +30 modifier (32 + 30 = 62). Matric rolls a 32 and the worker pod a 64—Matric wins! For damage, he rolls 1d10 + (WIL $\div$ 10). His WIL is 25, so that’s 1d10 + 3. He rolls a score a 7 and inflicts 10 (7 + 3) points of damage. The worker pod screams in pain, suffering a wound from the psychic assault. 

\end{quotation} 





\subsection{Psi-Chi Sleights} Psi-chi sleights are async abilities that speed up cognitive informatics (internal information processing) and enhance the user’s perception and cognition. 

\subsubsection{Ambience Sense} \textbf{Psi Type:} Passive \\ \textbf{Action:} Automatic \\ \textbf{Range:} Self \\ \textbf{Duration:} Constant \\ This sleight provides the async with an instinctive sense about an area and any potential threats nearby. The async receives a +10 modifier to all Investigation, Perception, Scrounging, and Surprise Tests. 

\subsubsection{Cognitive Boost} \textbf{Psi Type:} Active \\ \textbf{Action:} Quick \\ \textbf{Range:} Self \\ \textbf{Duration:} Temp (Action Turns) \\ \textbf{Strain Mod:} –1 \\ The async can temporarily elevate their cognitive performance. In game terms, Cognition is raised by 5 for the chosen duration. This boost to Cognition also raises the rating of skills linked to that aptitude. 

\subsubsection{Downtime} \textbf{Psi Type:} Active \\ \textbf{Action:} Task (min. 4 hours) \\ \textbf{Range:} Self \\ \textbf{Duration:} Sustained \\ \textbf{Strain Mod:} 0 \\ This sleight provides the async with the ability to send the mind into a fugue-state regenerative downtime, during which the character’s psyche is repaired. The async must enter the downtime for at least 4 hours; every 4 hours of downtime heals 1 point of stress damage. Traumas, derangements, and disorders are unaffected by this sleight. For all sensory purposes, the async is catatonic during downtime, completely oblivious to the outside world. Only severe disturbances or physical shock (such as being wounded or hit by a shock weapon) will bring the async out of it. 

\subsubsection{Emotion Control} \textbf{Psi Type:} Passive \\ \textbf{Action:} Automatic \\ \textbf{Range:} Self \\ \textbf{Duration:} Constant \\ Emotion Control gives the async tight control over their emotional states. Unwanted emotions can be blocked out and others embraced. This has the benefit of protecting the async from emotional manipulation, such as the Drive Emotion sleight or Intimidation skill tests. The async receives a +30 modifier when defending against such tests. 

\subsubsection{Enhanced Creativity} \textbf{Psi Type:} Passive \\ \textbf{Action:} Automatic \\ \textbf{Range:} Self \\ \textbf{Duration:} Constant \\ An async with Enhanced Creativity is more imaginative and more inclined to think outside the box. Apply a +20 modifier to any tests where creativity plays a major role. This level of ingenuity can sometimes seem strange and different, manifesting in odd or creepy ways, especially with artwork. 

\subsubsection{Filter} \textbf{Psi Type:} Passive \\ \textbf{Action:} Automatic \\ \textbf{Range:} Self \\ \textbf{Duration:} Constant \\ Filter allows the async to filter out out distractions and eliminate negative situational modifiers from distraction, up to the gamemaster’s discretion. 

\subsubsection{Grok} \textbf{Psi Type:} Active \\ \textbf{Action:} Complex \\ \textbf{Range:} Self \\ \textbf{Duration:} Instant \\ \textbf{Strain Mod:} –1 \\ By using the Grok sleight, the async is able to intuitively understand how any unfamiliar object, vehicle, or device is used simply by looking at and handling it. If the character succeeds in a COG x 2 Test, they achieve a basic ability to use the object, vehicle, or device, no matter how alien or bizarre. This sleight does not provide any understanding of the principles or technologies involved—the psi user simply grasps how to make it work. If a test is called for, the psi user receives a +20 modifier to use the device (this bonus only applies to unfamiliar devices, and/or tests the character is defaulting on—it does not apply to devices the character is familiar with). 

\subsubsection{High Pain Threshold} \textbf{Psi Type:} Passive \\ \textbf{Action:} Automatic \\ \textbf{Range:} Self \\ \textbf{Duration:} Constant \\ This sleight allows the async to block out, ignore, or otherwise isolate pain. The async reduces negative modifiers from wounds by 10. 

\subsubsection{Hyperthymesia} \textbf{Psi Type:} Passive \\ \textbf{Action:} Automatic \\ \textbf{Range:} Self \\ \textbf{Duration:} Constant \\ Hyperthymesia grants the async a superior autobiographical memory, allowing them to remember the most trivial of events. A hyperthymestic async can be asked a random date and recall the day of the week it was, the events that occurred that day, what the weather was like, and many seemingly trivial details that most people would not be able to recall. 

\subsubsection{Instinct} \textbf{Psi Type:} Passive \\ \textbf{Action:} Automatic \\ \textbf{Range:} Self \\ \textbf{Duration:} Constant \\ Instinct bolsters the async’s subconscious ability to gauge a situation and make a snap judgment that is just as accurate as a careful, considered decision. For Task Actions that involve analysis or planning alone (typically Mental skill actions), the async may reduce the timeframe by 90\% without suffering a modifier. For Task Actions that involve partial analysis/ planning, they may reduce the timeframe by 30\% without penalty. 

\subsubsection{Multitasking} \textbf{Psi Type:} Passive \\ \textbf{Action:} Automatic \\ \textbf{Range:} Self \\ \textbf{Duration:} Constant \\ The async can handle vast amounts of information without overload and can perform more than one mental task at once. The character receives an extra Complex Action each Action Phase that may only be used for mental or mesh actions. 

\subsubsection{Pattern Recognition} \textbf{Psi Type:} Passive \\ \textbf{Action:} Automatic \\ \textbf{Range:} Self \\ \textbf{Duration:} Constant \\ The character is adept at spotting patterns and correlating the non-random elements of a jumble—related items jump out at them. This is useful for translating languages, breaking codes, or find clues hidden among massive amounts of data. The character must have a sufficiently large sample enough time to study, as determined by the gamemaster. This might range from a few hours of listening to a spoken transhuman language to a few days of investigating inscriptions left by long-dead aliens to a week or more of researching a lengthy cipher. Languages may be comprehended by reading or listening to them being spoken. Apply a +20 modifier to any appropriate Language, Investigation, Research, or cod-breaking Tests (note that this does not apply to Infosec Tests made by software to decrypt a code). The async may also use this ability to more easily learn new languages, reducing the training time by half. 

\subsubsection{Predictive Boost} \textbf{Psi Type:} Passive \\ \textbf{Action:} Automatic \\ \textbf{Range:} Self \\ \textbf{Duration:} Constant \\ The Bayesian probability machine features of the async’s brain are boosted by this sleight, enhancing their ability to estimate and predict outcomes of events around them as they unfold in real-time and update those predictions as information changes. In effect, the character has a more intuitive sense for which outcomes are most likely. This grants the character a +10 bonus on any skill tests that involve predicting the outcome of events. It also bolsters the async’s decision-making in combat situations by making the best course of action more clear, and so provides a +10 bonus to both Initiative and Fray Tests. 

\subsubsection{Qualia} \textbf{Psi Type:} Active \\ \textbf{Action:} Quick \\ \textbf{Range:} Self \\ \textbf{Duration:} Temp (Action Turns) \\ \textbf{Strain Mod:} –1 \\ The async can temporarily increase their intuitive grasp of things. In game terms, Intuition is raised by 5 for the chosen duration. This boost to Intuition also raises the rating of skills linked to that aptitude. 

\subsubsection{Savant Calculation} \textbf{Psi Type:} Passive \\ \textbf{Action:} Automatic \\ \textbf{Range:} Self \\ \textbf{Duration:} Constant \\ The character possesses an incredible facility with intuitive mathematics. They can do everything from calculate the odds exactly when gambling to predicting precisely where a leaf falling from a tree will land by observing the landscape and local wind currents. The character specializes in calculation involving the activity of complex chaotic systems and so can calculate answers that even the fastest computers could not, including things like patterns of rubble distribution from an explosion. However, this mathematic facility is largely intuitive, so the character does not know the equations they are solving, they merely know the solution to the problem. This sleight also provides a +30 modifier to any skill tests involving math (which the character is calculating, not a computer). 

\subsubsection{Sensory Boost} \textbf{Psi Type:} Active \\ \textbf{Action:} Quick \\ \textbf{Range:} Self \\ \textbf{Duration:} Temp (Action Turns) \\ \textbf{Strain Mod:} –2 \\ An async uses this sleight to increase their natural or augmented sensory perception (sight, audio, smell, augmented) by enhanced cerebral processing, granting a +20 bonus modifier on sensory-based Perception Tests. 

\subsubsection{Superior Kinesics} \textbf{Psi Type:} Passive \\ \textbf{Action:} Automatic \\ \textbf{Range:} Self \\ \textbf{Duration:} Constant \\ The async acquires more insight into people’s emotive signals, gestures, facial expressions, and body language when it comes time to predict the person’s emotional state, intent, or reactions. Apply a +10 modifier to Kinesics Skill Tests. 

\subsubsection{Time Sense} \textbf{Psi Type:} Active \\ \textbf{Action:} Automatic \\ \textbf{Range:} Self \\ \textbf{Duration:} Temp (Action Turns) \\ \textbf{Strain Mod:} –1 \\ An async with this ability can slow down his perception of time, making everything appear to move in slow motion or at reduced speed. In game terms, this sleight grants the async a Speed of +1. This extra Action Phase, however, can only be spent on mental and mesh actions. 

\subsubsection{Unconscious Lead} \textbf{Psi Type:} Active \\ \textbf{Action:} Automatic \\ \textbf{Range:} Self \\ \textbf{Duration:} Temp (Action Turns) \\ \textbf{Strain Mod:} +0 \\ This sleight allows the async to override their consciousness and let their unconscious mind take point. While in this state, the async’s conscious mind is only dimly aware of what is transgressing, and any memories of this period will be hazy at best. The advantage is that the unconscious mind acts more quickly, and so the async’s Speed is boosted by +1. The character remains aware and active, but is incapable of complex communication or other mental actions and is motivated by instinct and primitive urges more than conscious thought. Though it is recommended that the player retain control of their character while using Unconscious Lead, the gamemaster should feel free to direct the character’s actions as they see fit. 



\subsection{Psi-Gamma Sleights} Psi-gamma sleights deal with contacting (reading and communicating) and influencing the function of biological minds (egos within a biomorph, but also including animal life). Psi-gamma is only available to characters with Level 2 of the Psi trait. Most psi-gamma use is handled as an Opposed Test between the async and the target (p. 222). 

\subsubsection{Alienation} \textbf{Psi Type:} Active \\ \textbf{Action:} Complex \\ \textbf{Range:} Touch \\ \textbf{Duration:} Temp (Action Turns) \\ \textbf{Strain Mod:} +0 \\ \textbf{Skill:} Psi Assault \\ Alienation is an offensive sleight that create a sense of disconnection between an ego and its morph—similar to that experienced when resleeved into a new body. The ego finds their body cumbersome, strange, and alien, almost like they are a prisoner within it. If the async beats the target in an Opposed Test, treat the test as a failed Integration Test (p. 272) for the target. This effect lasts for the sleight’s duration. 

\subsubsection{Charisma} \textbf{Psi Type:} Active \\ \textbf{Action:} Quick \\ \textbf{Range:} Touch \\ \textbf{Duration:} Temp (Minutes) \\ \textbf{Strain Mod:} –1 \\ \textbf{Skill:} Control\\ The async uses this sleight to influence the target’s mind on a subconscious level, so that the target perceives them to be charming, magnetic, and persuasive. If the async beats the target in an Opposed Test, they gain a +30 modifier on all subsequent Social Skill Tests for the chosen duration. 

\subsubsection{Cloud Memory} \textbf{Psi Type:} Active \\ \textbf{Action:} Complex \\ \textbf{Range:} Touch \\ \textbf{Duration:} Temp (Minutes) \\ \textbf{Strain Mod:} –1 \\ \textbf{Skill:} Control\\ Cloud Memory allows the async to temporarily disrupt the target’s ability to form long-term memories. If the async wins the Opposed Test, the target’s memorysaving ability is negated for the duration. The target will retain short-term memories during this time, but will soon forget anything that occurred while this sleight was in effect. 

\subsubsection{Deep Scan} \textbf{Psi Type:} Active \\ \textbf{Action:} Complex \\ \textbf{Range:} Touch \\ \textbf{Duration:} Sustained \\ \textbf{Strain Mod:} +1 \\ \textbf{Skill:} Sense\\ Deep Scan is a more intrusive version of Thought Browse (p. 228), made to extract information from the targeted individual. If the Opposed Test succeeds, the async telepathically invades the target’s mind and can probe it for information. For every 10 full points of MoS the async achieves on their test, they retrieve one piece of information. Each item takes one full Action Turn to retrieve, during which the sleight must be sustained. The target is aware of this mental probing, though they will not know what information the async acquired. 

\subsubsection{Drive Emotion} \textbf{Psi Type:} Active \\ \textbf{Action:} Complex \\ \textbf{Range:} Touch \\ \textbf{Duration:} Temp (Action Turns) \\ \textbf{Strain Mod:} –1 \\ \textbf{Skill:} Control\\ This sleight allows the async to stimulate cortical areas of the target’s brain related to emotion. This allows the async to induce, amplify, or tone down specific emotions, thereby manipulating the target. If the async beats the target in an Opposed Test, they will act in accordance with the emotion for the duration and under certain circumstances may suffer from certain penalties (up to +/–30), as determined by the gamemaster. For example, an async might receive a +30 Intimidation Test modifier against a target imbued with fear. 

\subsubsection{Ego Sense} \textbf{Psi Type:} Active \\ \textbf{Action:} Complex \\ \textbf{Range:} Close \\ \textbf{Duration:} Temp (Action Turns) \\ \textbf{Strain Mod:} –1 \\ \textbf{Skill:} Sense\\ Ego Sense can be used to detect the presence and location of other sentient and biological life forms (i.e., egos) within the async’s range. To detect these life forms, the async makes a single Sense Test, opposed by each life form within range. The async may suffer a modifier for detecting small animals and insects, similar to the modifier applied for targeting them in ranged combat (see p. 193); likewise, a modifier for detecting larger life forms may also be applied. If successful, the async has detected that the life form is nearby. Every 10 full points of MoS will ascertain another piece of information regarding the detected life: direction from async, approximate size, type of creature, distance from async, etc. The async will know if the target moves, if they do so during the sleight’s duration. 

\subsubsection{Empathic Scan} \textbf{Psi Type:} Active \\ \textbf{Action:} Quick \\ \textbf{Range:} Close \\ \textbf{Duration:} Sustained \\ \textbf{Strain Mod:} –2 \\ \textbf{Skill:} Sense\\ Empathic Scan enables the async to sense the target’s base emotions. If the async wins the Opposed Test, they intuitively feel the target’s emotional current state for as long as the sleight is sustained. At the gamemaster’s discretion, this knowledge may provide a modifier (up to +30) for certain Social skill tests. 

\subsubsection{Implant Memory} \textbf{Psi Type:} Active \\ \textbf{Action:} Complex \\ \textbf{Range:} Touch \\ \textbf{Duration:} Instant \\ \textbf{Strain Mod:} +0 \\ \textbf{Skill:} Control\\ An async using this sleight can implant a memory of up to an hour’s length inside the target’s mind. This memory very obviously does not belong to the target—there is no way they will confuse it for one of their own. The intent is not to fake memories, but to place one of the async’s memories in the target’s mind so that the target can access it just like any other memory. This can be useful for “archiving” important data with an ally, providing a literal alternate perspective, or simply making a “data dump” for the target to peruse. Implant Memory requires an Opposed Test against unwilling participants. At the gamemaster’s discretion, particularly traumatic memories might inflict mental stress on the recipient (p. 215). 

\subsubsection{Implant Skill} \textbf{Psi Type:} Active \\ \textbf{Action:} Complex \\ \textbf{Range:} Touch \\ \textbf{Duration:} Temp (Action Turns) \\ \textbf{Strain Mod:} +0 \\ \textbf{Skill:} Control\\ Similar to Implant Memory, this sleight allows the async to impart some of their expertise and implant it into the target’s mind. For the duration of the sleight, the target benefits when using that skill. If the async’s skill is between 31 and 60, the target receives a +10 bonus. If the async’s skill is 61+, the target receives a +20 bonus. Implant Skill requires an Opposed Test against unwilling participants. In some cases, the target has been known to use the skill with the async’s flair and mannerisms. 

\subsubsection{Mimic} \textbf{Psi Type:} Active \\ \textbf{Action:} Quick \\ \textbf{Range:} Close \\ \textbf{Duration:} Instant \\ \textbf{Strain Mod:} +0 \\ \textbf{Skill:} Sense\\ In a setting where changing your body and face is not unusual, people learn to recognize habits and personality quirks more often. The async must use this sleight on a target and succeed in a Success Test. If successful, the async acquires an “imprint” of the target’s mind that they can take advantage of when impersonating that ego. The async then receives a +30 bonus on Impersonation Tests when mimicking the target’s behavior and social cues. 

\subsubsection{Mindlink} \textbf{Psi Type:} Active \\ \textbf{Action:} Quick \\ \textbf{Range:} Touch \\ \textbf{Duration:} Sustained \\ \textbf{Strain Mod:} +1/target after first \\ \textbf{Skill:} Control\\ Mindlink allows two-way mental communication with a target. This may be used on more than one target simultaneously, in which case the async can act as a telepathic “server,” so that everyone mindlinked with the async may also telepathically communicate with each other (via the async, however, so they overhear). Language is still a factor in mindlinked communications, but this barrier may be overcome by transmitting sounds, images, emotions, and other sensations. Mindlink requires an Opposed Test against unwilling participants. 

\subsubsection{Omni Awareness} \textbf{Psi Type:} Active \\ \textbf{Action:} Quick \\ \textbf{Range:} Close \\ \textbf{Duration:} Temp (Minutes) \\ \textbf{Strain Mod:} –1 \\ \textbf{Skill:} Sense\\ An async with Omni Awareness is hypersensitive to other biological life that is observing them. During this sleight’s duration, the async makes a Sense Test that is opposed by any life that has focused their attention on them within the sleight’s range; if successful, the async knows they are being watched, but not by whom or what. It does, however, apply a +30 Perception bonus to spot the observer. This sleight does not register partial attention or fleeting attention, or simple perception of the async, it only notices targets who are actively observing (even if they are concealing their observation). This sleight is effective in spotting a tail, as well as finding potential mates in a bar. 

\subsubsection{Penetration} \textbf{Psi Type:} Active \\ \textbf{Action:} Quick \\ \textbf{Range:} Touch \\ \textbf{Duration:} Instant \\ \textbf{Strain Mod:} 1 per AP point \\ \textbf{Skill:} Psi Assault \\ Penetration is a sleight that works in conjunction with any offensive sleight that involves the Psi Assault skill. It allows the async to penetrate the Psi Shield of an opponent by concentrating their psi attack. Every point of Armor Penetration applied to a psi attack inflicts 1 point of strain. The maximum AP that may be applied equals the async’s Psi Assault skill divided by 10 (round down). 

\div{Psi Shield} \textbf{Psi Type:} Passive \\ \textbf{Action:} Automatic \\ \textbf{Range:} Self \\ \textbf{Duration:} Constant \\ Psi Shield bolsters the async’s mind to psi attack and manipulation. If the async is hit by a psi attack, they receive WIL $\div$ 5 (round up) points of armor, reducing the amount of damage inflicted. They also receive a +10 modifier when resisting any other sleights. 

\div{Psychic Stab} \textbf{Psi Type:} Active \\ \textbf{Action:} Complex \\ \textbf{Range:} Touch \\ \textbf{Duration:} Instant \\ \textbf{Strain Mod:} +0 \\ \textbf{Skill:} Psi Assault \\ Psychic Stab is an offensive sleight that seeks to inflict physical damage on the target’s brain and nervous system. Each successful attack inflicts 1d10 + (WIL $\div$ 10, round up) damage. Increase the damage by +5 if an Excellent Success is scored. 

\subsubsection{Scramble} \textbf{Psi Type:} Passive \\ \textbf{Action:} Automatic \\ \textbf{Range:} Self \\ \textbf{Duration:} Constant \\ Scramble allows the async using the sleight to hide from another async using the Ego Sense or Omni Awareness sleights. Apply a +30 modifier to the defending async’s Opposed Test. 

\subsubsection{Sense Block} \textbf{Psi Type:} Active \\ \textbf{Action:} Complex \\ \textbf{Range:} Touch \\ \textbf{Duration:} Temp (Action Turns) \\ \textbf{Strain Mod:} –1 \\ \textbf{Skill:} Psi Assault \\ Sense Block disables and short circuits one of the target’s sensory cortices (chosen by the async), interfering with and possibly negating a specific source of sensory input for the chosen duration. If the async beats the target in the Opposed Test, the target suffers a –30 modifier to Perception Tests with that sense equal (doubled to –60 if the async scores an Excellent Success). 



\subsubsection{Spam} \textbf{Psi Type:} Active \\ \textbf{Action:} Complex \\ \textbf{Range:} Touch \\ \textbf{Duration:} Temp (Action Turns) \\ \textbf{Strain Mod:} +0 \\ \textbf{Skill:} Psi Assault \\ The sleight allows the async to overload and flood one of the target’s sensory cortices (chosen by the async), spamming them with confusing and distracting sensory input and thereby impairing them. If the async wins the Opposed Test, the target suffers a –10 modifi er to all tests the duration of the sleight (doubled to –20 if the async scores an Excellent Success). 

\subsubsection{Static} \textbf{Psi Type:} Active \\ \textbf{Action:} Complex \\ \textbf{Range:} Close \\ \textbf{Duration:} Sustained \\ \textbf{Strain Mod:} +0 \\ \textbf{Skill:} None\\ The async generates an anti-psi jamming field, impeding any use of ranged sleights within their range. All such ranged sleights suffer a –30 modifier. This sleight has no effect on self or touch-range sleights. 

\subsubsection{Subliminal} \textbf{Psi Type:} Active \\ \textbf{Action:} Complex \\ \textbf{Range:} Touch \\ \textbf{Duration:} Instant \\ \textbf{Strain Mod:} +2 \\ \textbf{Skill:} Control\\ The Subliminal sleight allows the async to influence the train of thought of another person by implementing a single post-hypnotic suggestion into the mind of the target. If the async wins the Opposed Test, the recipient will carry out this suggestion as if it was their own idea. Implanted suggestions must be short and simple; as a rule of thumb, the gamemaster may only suggestions encompassed by a short sentence (for example: “open the airlock,” or “hand over the weapon”). At the gamemaster’s discretion, the target may receive a bonus for resisting suggestions that are immediately life threatening (“jump off the bridge”) or that violate their motivations or personal strictures. Suggestions do not need to be carried out immediately, they may be implanted with a short trigger condition (“when the alarm goes off, ignore it”). 

\subsubsection{Thought Browse} \textbf{Psi Type:} Active \\ \textbf{Action:} Complex \\ \textbf{Range:} Touch \\ \textbf{Duration:} Sustained \\ \textbf{Strain Mod:} –1 \\ \textbf{Skill:} Sense\\ Thought Browse is a less-intrusive form of mind reading which scans the target’s surface thoughts for certain “keywords” like a particular word, phrase, sound, or image chosen by the async. Rather than digging through the target’s ego as with the Deep Scan sleight, Thought Browse merely verifies whether a target has a particular person, place, event, or thing in mind, which can be used by a savvy investigator to draw conclusions without the need to invade the mind directly. Thought Browse may be sustained, allowing the async to continue scanning the target’s thoughts over time. The async must beat the target in an Opposed Test for each scanned item. 

\section{Psychosurgery} Given the reach of neuroscience in the time of Eclipse Phase, it is easy to think of the mind as programmable software, as something that can be reverse-engineered, re-coded, upgraded, and patched. To a large degree, this is true. Aided by nanotechnology, genetics, and cognitive science, neuroscientists have demolished numerous barriers to understanding the mind’s structure and functions, and even made great leaps in unveiling the true nature of consciousness. Genetic tweaks, neuro-mods, and neural implants offer an assortment of options for improving the brain’s capabilities. The transhuman mind has become a playground—and a battlefield. Nanovirii unleashed during the Fall infected millions, altering their brains in permanent ways, with occasional outbreaks still occurring a decade later. Cognitive virii roam the mesh, plaguing infomorphs and AIs, reprogramming mind states. An “infectious idea” is now a literal term. In truth, mind editing is not an easy, safe, and error-proof process—it is difficult, dangerous, and often flawed. Neuroscience may be light years ahead of where it was a century ago, but there are many aspects of the brain and neural functions that continue to confound and elude even the brightest experts and AIs. Technologies like nanoneural mapping, uploading, digital mind emulation, and artificial intelligence are also comparatively in their infancy, being mere decades old. Though transhumanity has a handle on how to make these processes work, it does not always fully understand the underlying mechanisms. Any neurotech will tell you that mucking around in the mind’s muddy depths is a messy business. Brains are organic devices, molded by millions of years of unplanned evolutionary development. Each is grown haphazardly, loaded with evolutionary leftovers, and randomly modified by an unlimited array of life events and environmental factors. Every mind features numerous mechanisms—cells, connections, receptors—that handle a dizzying array of functions: memory, perception, learning, reasoning, emotion, instinct, consciousness, and more. Its system of organization and storage is holonomic, diffused, and disorganized. Even the geneticallymodifi ed and enhanced brains of transhumans are crowded, chaotic, cross-wired places, with each mind storing its memories, personality, and other defining features in unique ways. What this means is that though the general architecture and topography of neural networks can be scanned and deduced, the devil is in the details. Techniques used to modify, repair, or enhance one person’s mind are not guaranteed equal success when applied to another’s brain. For example, the process by which brains store knowledge, skills, and memories results in a strange chaining process where these memories are linked and associated with other memories, so attempts to alter one memory can have adverse affects on other memories. In the end, minds are slippery and dodgy things, and attempts to reshape them rarely go as planned. 

\subsection{The Process Of Psychosurgery} Psychosurgery is the process of selective, surgical alteration of a transhuman mind. It is a separate field from neural genetic modification (which alters genetic code), neuralware implantation (adding cybernetic or biotech inserts to the brain or nervous system), or brain hacking (software attacks on computer brains, neural inserts, and infomorphs), though they are sometimes combined. Psychosurgery is almost always performed on a digital mind-state, whether that be a real-time emulation, a backup, or a fork. In most cases, the subject’s mind-state is copied via the same technology and process as uploading or forking, and run in a simulspace. The subject need not be willing, and in these cases the subject’s permissions are restricted. Numerous psychosurgery simulspace environments are available, each custom-designed for facilitating specific psychosurgical goals and programmed with a thorough selection of psychotherapy treatment options. The actual process of psychosurgery breaks down into several stages. First is diagnosis, which can involve the use of several neuro-imaging techniques on morphed characters, mapping synaptic connections, and building a neurochemical model. It can also involve complete psychological profiling and psychometric behavioral testing, including personality tests and simulspace scenario simulations. Digital mind-states can be compared to records of people with similar symptoms in order to identify related information clusters. This analysis is used to plan the procedure. The actual implementation of psychosurgical alteration can involve several methods, depending on the desired results. Applying external modules to the mind-state is often the best approach, as it doesn’t meddle with complicated connections and new inputs are readily interpreted and assimilated. For treatments, mental health software patches compiled from databases of healthy minds are matched, customized, and applied. Specialized programs may be run to stimulate certain mental processes for therapeutic purposes. Before an alteration is even applied, it may first be performed on a fork of the subject and run at accelerated speeds to evaluate the outcome. Likewise, multiple treatment choices may be applied to time-accelerated forks this way, allowing the psychosurgeon to test which is likely to work best. Not all psychosurgery is performed for the subject’s benefit, of course. Psychosurgery can be used to interrogate or torture prisoners, erase memories, modify behavior, or inflict crippling impairments. It is also sometimes used for legal punishment purposes, in an attempt to impair criminal activity. Needless to say, such methods are often brute-forced rather than fine-tuned, ignoring safety parameters and sometimes resulting in detrimental side effects. 



\end{quotation} \textbf{Solarchive Search: “The Human Cognome Project” } \\ The Human Cognome Project was an academic research venture to reverse engineer the human brain, paralleling in many ways the Human Genome Project and its success in deciphering the human genome. The HCP was a multidisciplinary undertaking, relevant to biology, neuroscience, psychology, cognitive science, artificial intelligence, and philosophy of mind. Funded and supported by scientific and corporate entrepreneurs and early transhumanist groups, the HCP developed the fundamentals of digitizing an ego and was a major driving force towards the first transhumans with elevated intelligence and brain capacity. The HCP has also been instrumental in cataloging transhuman minds and developing databases of “mind patches” based on the mind-states of healthy individuals for treating mental diseases and damage. Though most HCP data is available to the public, some argonauts claim that certain data is held hostage by some hypercorps, potentially for the development of proprietary mind-altering technologies. After the Fall, the remnants of this project were acquired by the Planetary Consortium. \end{quotation} 

\subsection{Psychosurgery Mechanics} In game terms, psychosurgery is handled as a Task Action requiring an Opposed Test. The psychosurgeon rolls Psychosurgery skill against the target’s WIL x 3. Apply modifiers as appropriate from the Psychosurgery Modifiers table. If the psychosurgeon succeeds and the subject fails, the psychosurgery is effective and permanent. The alteration becomes a permanent part of the subject’s ego, and will be copied when uploaded (and sometimes when forking). If both sides succeed but the psychosurgeon rolls higher, the psychosurgery is effective but temporary. It lasts for 1 week per 10 points of MoS. If the subject rolls higher, or if the psychosurgeon fails their roll, the attempt does not work. The timeframe listed for psychosurgical procedures is according to the patient’s subjective point of view. Since most subjects are treated in a simulspace, time acceleration may drastically reduce the amount of real-time such a procedure requires (see Defying Nature’s Laws, pp. 240–241). 



\subsubsection{Mental Stress} Psychosurgery is a modification to the transhuman mind, and sometimes to the actual person that resides in that mind. It is unsurprising then that psychosurgery places stress on the subject’s mental state and sometimes even inflicts mental traumas. Each psychosurgery option lists a Stress Value (SV) that is inflicted on the subject regardless of the tests’ success or failure. If the psychosurgeon achieves an Excellent Success (MoS 30+), this stress is halved (round down). If the psychosurgeon rolls a Severe Failure (MoF 30+), the stress is doubled. Alternately, a Severe Failure could result in unintended side effects, such as affecting other behaviors, emotions, or memories. If a critical success is rolled, no stress is applied at all. If a critical failure is rolled, however, an automatic trauma is applied in addition to the normal stress. Some psychosurgery conditions may also affect the SV, as noted on the Psychosurgery Modifiers table. 



\subsection{Roleplaying Mind Edits} Many of the changes incurred by psychosurgery are nebulous and difficult to pin down with game mechanics. Alterations to a character’s personality and mind-state are often better handled as roleplaying factors anyway. This means that players should make a real effort to integrate any such mental modifications into their character’s words and actions, and gamemasters should ensure that a character’s portrayal plays true to their mind edits. Some psychosurgical mods can be reflected with ego traits, while others might incur modifiers to certain tests or in certain situations. The gamemaster should carefully weigh a brain alteration’s effects, and apply modifiers as they see appropriate. 



\subsection{Psychosurgery Procedures} The following alterations may be accomplished with psychosurgery. At the gamemaster’s discretion, other mind-editing procedures may be attempted, using these as a guideline. 



\begin{table} \begin{tabular}{|l|r|r|} \hline

\hline{3}{|c|}{\textbf{Psychosurgery Modifiers}} \\ \hline

Situation &Psychosurgery Test Modifier &SV Modifier \\ \hline

Improper Preparatory Diagnosis &–30 &+1 \\ \hline

Safety Protocols Ignored &+20 &x2 \\ \hline

Simulspace Time Acceleration &–20 &+2 \\ \hline

Subject is an AI, AGI, or uplift &–20 &+1 \\ \hline

\label{tab:psychosurgery-modifiers} \label{tab:psychosurgery-modifiers} \end{table} 



\div{Behavioral Control} \textbf{Timeframe:} 1 week \\ \textbf{PM:} Limit/Boost –10; Block/Encourage –20, Expunge/Enforce –30 \\ \textbf{SV:} (1d10 $\div$ 2, round up) \\ Commonly used for criminal rehabilitation, behavioral control attempts to limit, block, or expunge a specific behavior from the subject’s psyche. For example, a murderer may be conditioned against acts of aggression, or a kleptomaniac might be restricted from stealing. Some people seek this adjustment willingly, such as socialite glitterati who restrict their desire to eat, or an addict who cuts out their craving for a fix. Behavioral control can also be applied as an unleashing or reinforcement. A companion may desire to eliminate their sexual inhibitions, for example, or a hypercorp exec may boost his commitment to place work above all else. A character will simply feel compelled to avoid a behavior that is limited (perhaps suffering a –10 modifier), but will find it quite difficult to pursue a behavior that is blocked (requiring a WIL x 3 Test, and suffering a –20 modifier). They will find themselves completely incapable of initiating a behavior that is expunged, and if forced into the behavior will suffer a –30 modifier and (1d10 $\div$ 2, round up) points of mental Stress. Likewise, a character will feel compelled to pursue a behavior that is boosted, and will find it hard to avoid engaging in a behavior that is encouraged (requiring a WIL x 3 Test to avoid). They will have no choice but to engage in enforced behaviors, and will suffer (1d10 $\div$ 2, round up) points of mental Stress if prevented from doing so. 

\div{Behavioral Masking} \textbf{Timeframe:} 1 week \\ \textbf{PM:} –20 \\ \textbf{SV:} 1d10 $\div$ 2, round up \\ Given the ability to switch bodies, many security and law enforcement agencies have resorted to personality and behavioral profiling as a means of identifying people even when they resleeve. Though such systems are far from perfect, someone’s unconscious habits and quirks could potentially give them away. Characters who wish to elude identification in this way may undergo behavioral masking, which seeks to alter and change the character’s unconscious habits and social cues. Apply a +30 modifier when defending against such identification systems and Kinesics Tests. 

\div{Deep Learning} \textbf{Timeframe:} Skill Learning Time $\div$ 2 \\ \textbf{PM:} +20 \\ \textbf{SV:} 1 \\ Using tutorial programs, memory reinforcement protocols, conditioning tasks, and deep brain stimulation, the subject’s learning ability is reinforced, allowing them to learn new skills more quickly. 

\div{Emotional Control} \textbf{Timeframe:} 1 week \\ \textbf{PM:} Limit/Boost –10; Block/Encourage –20, Expunge/Enforce –30 \\ \textbf{SV:} (1d10 $\div$ 2, round up) + 2 \\ Similar to behavioral control, emotional control seeks to modify, enhance, or restrict the subject’s emotional responses. Some choose these modifications willingly, such as limiting sadness in order to be happier, or encouraging aggression in order to be more competitive. Mercenaries and soldiers have been known to expunge fear. Follow the same rules as given for Behavioral Control. 



\subsubsection{Interrogation} \textbf{Timeframe:} Variable (gamemaster discretion; 1 week default) \\ \textbf{PM:} +30 \\ \textbf{SV:} 1d10 \\ Psychosurgery can be used for interrogative purposes via the application of mental torture and manipulation. A successful Psychosurgery Test applies a +30 modifier to the Intimidation Test for interrogation. 

\div{Memory Editing} \textbf{Timeframe:} 1 week (2 weeks adding/replacing) \\ \textbf{PM:} –10 (willing) or –30 (forced) \\ \textbf{SV:} (1d10 $\div$ 2, round up) \\ By monitoring memory recall (forcibly invoked if necessary), psychosurgeons can identify where memories are stored in the brain and target them for removal. Memory storage is complex and diffused, however, and often linked to other memories, so removing one memory may affect others (gamemaster discretion). Adding or replacing memories is a much more complicated operation and requires that such memories be copied from someone who has experienced them or manufactured with XP software. Even when successfully implanted, fake memories may clash with other (real) memories unless those are also erased. 

\div{Personality Editing} \textbf{Timeframe:} 1 week \\ \textbf{PM:} Minor –10; Moderate –20, Major –30 \\ \textbf{SV:} (1d10 $\div$ 2, round up) + 3 \\ Possibly the most drastic psychosurgery procedure, personality editing involves altering the subject’s core personality traits. The personality factors that may be modified is almost unlimited, including traits such as openness, conscientiousness, altruism, extroversion/ introversion, impulsiveness, curiosity, creativity, confidence, sexual orientation, and self-control, among others. These traits may be enhanced or reduced to varying degrees. The effect is largely reflected by roleplaying, but the gamemaster may apply modifiers as they see fit. 

\subsubsection{Psychotorture} \textbf{Timeframe:} Variable \\ \textbf{PM:} +30 \\ \textbf{SV:} 1d10 SV per day \\ Psychotorture is mental manipulation for the simple intention of causing pain and anguish, reflected in game terms as mental stress and traumas. Prolonged torture can lead to serious mental disorders or worse. 

\subsubsection{Psychotherapy} \textbf{TIMEFRAME:} Variable \\ \textbf{PM:} +0 \\ \textbf{SV:} 0 \\ Therapeutic psychosurgery is beneficial for characters suffering from mental stress, traumas, and disorders. A successful Psychosurgery Test applies a +30 modifier to mental healing tests, as noted on p. 215. 

\subsubsection{Skill Imprints} \textbf{Timeframe:} 1 week per +10 \\ \textbf{PM:} +0 \\ \textbf{SV:} 1 per +10 \\ Skill imprinting is the use of psychosurgery to insert skill-set neural patterns in the subject’s brain, temporarily boosting their ability. Skill imprints are artificial boosts, however, degrading at the rate of –10 per day. No skill may be boosted higher than 60. 

\subsubsection{Skill Suppression} \textbf{Timeframe:} 1 day per –10 \\ \textbf{PM:} –10 \\ \textbf{SV:} 1 per +10 \\ Skill suppression attempts to identify where skills are stored in the brain and then block or remove them. The subject’s skill is impaired and may be lost entirely. 

\subsubsection{Tasping} \textbf{Timeframe:} 1 day \\ \textbf{PM:} +10 \\ \textbf{SV:} 1 \\ Tasping is the use of deep brain stimulation techniques to tickle the mind’s pleasure centers. Though this procedure is often used for therapeutic purposes for patients suffering from depression or other mental illnesses, the intent with tasping is to overload the subject into a prolonged state of almost unendurable bliss. Such stimulation is highly addictive, however, so character’s exposed to it for any length of time (over 1 hour, subjective) are likely to pick up the Addiction trait (p. 148). Some criminal organizations have been known to use tasping addiction and rewards as a means of controlling those under their thrall. 



\subsection{The Lost} 

\begin{quotation} $\succ$ begin excerpt $\succ$ \\ PSICLONE Project Quarterly Board Meeting \\ 2nd Quarter 8 AF \\ FUTURA Project Conclusion— \\ Executive Summary Report \\ Prepared by Dr. Amelia Sheppard	\\ Per request, I have compiled a review of the Futura Project and its fallout, 5 years after whistleblowers and intense media attention forced us to end the project and release the remaining subjects (dubbed “the Lost” by popular media). 

Futura was a joint initiative spearheaded by Hanto Genomics and strongly backed by Cognite, with numerous other partners (complete list). The project was initially proposed by my mentor, Dr. Antonio Pascal, whose team had proven the feasibility of Accelerated Life Experience Training (ALET) after a series of pilot studies with two small (N $ \prec $ 1000) samples. While it is true that these early pilot studies used both older subjects and a lesser amount of time dilation, the rationale for the Futura Project’s ambitious program was justified by a remarked decrease in transhumanity’s population due to the Fall, a system-wide stagnant population growth rate (blamed on various factors including increased longevity, available contraception, and rising despair over troubling times), as well as a desire to move aggressively into a new technological sector in the hopes of obtaining a competitive advantage. 

Futura began immediately in the wake of the Fall with an initial seed population of *** test subjects culled from extant genetic material and gestated to between 1 week and 6 months after birth. Of these, less than 10\% were live births from either a surrogate or genetic birth mother who had perished during the Fall. The majority came from our Lunar and Martian labs and were brought to term within an exowomb. 

After the sample was selected, all subjects were sleeved into our fast-growth futura-brand biomorph bodies and inducted into customized simulspace accelerated learning environments. The project made extensive use of emergent technologies and techniques culled from recaptured TITAN facilities, including neogenetic traits for the futura morphs and time distortion applications for captive simulspace populations. Futura ran concurrently on three different research stations with a combined staff of 2,211 researchers and support personnel and 45 AGIs custom-programmed for expert child development. Project goals were to raise each child to a subjective 18 years life experience in 3 years objective time. 

Despite omnipresent observation and real-time adjusting of the simulspace and educational programming for optimal normality, somewhere along the way the project suffered a breakdown in quality assurance and parameter monitoring that resulted in a near total failure at empathy modeling. We first observed this effect 11 months into the project when the subjects had aged to approximately 6 years of age. Incidences of animal cruelty and acting out had spiked, though at that time they remained within acceptable standards. Over the next few months this trend continued and Dr. Pascal authorized the usage of more authoritative “parenting” to attempt to correct for the borderline sociopathic behavior that was being exhibited by 23.19\% of all subjects by the 18- month mark (9 years of age). 

We now know that these changes had the unintended consequence of suppressing overt displays of cruelty and violence and merely taught the majority of subjects how to conceal their psychoses. It was also at this time that the first deaths occurred. The initial waves were thought to be accidents and both the victim and perpetrator were usually backed up to a week or so of subjective time. Post-project analysis now shows that 43.87\% of our subjects had engaged in at least one act of premeditated murder by the 24-month mark (12 years of age) and the counseling protocols were only training them how to lie more effectively. 

It was at this point that myself and Dr. Aaron Bharani advocated pulling the plug on the project and bringing the subjects out to real time and intensive counseling. Dr. Pascal vetoed our concerns without ever taking them to the board. As the project spiraled towards its conclusion, a fork of Dr. Bharani went public at the 34 month mark, inciting a firestorm of controversy. While Dr. Pascal successfully tied up investigators, hoping to see the project through to its conclusion, the incident at our Legacy research station occurred. Initial findings concluded that one or more of the subjects had escaped the program and were in fact responsible for the habitat’s environmental failures and the thousands of subsequent deaths. 

In the face of intense public and private scrutiny, many of the partners involved in the project attempted to pull out and even eliminate all traces of their involvement. In the resulting chaos, an estimated *** subjects were quietly released into the system’s general population. It was only after this occurred that all known subjects were identified as having been infected with the Watts-Macleod strain of the Exsurgent virus, though when and how this occurred remains troubling and unclear. Though later orders resulted in all remaining subjects being euthanized and/or backed up into cold storage, only of *** the released subjects were recaptured. Of the rest, *** pursued sanctuary with sympathetic authorities, *** went public and submitted themselves to extensive psychotherapy, *** were killed in incidents of violence and not resurrected, and the rest presumably went into hiding. 

$\succ$ end excerpt $\succ$ \end{quotation} 