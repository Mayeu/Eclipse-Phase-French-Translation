\chapter{La Mesh} \label{cha:mesh} 

Avant la Chute, Les humains étaient interfacés entre eux au travers de de l'internet, l'interconnexion des réseaux étant le facteur déterminant de l'évolution du world wide web. S'il a commencé comme un support électronique pour la récupération des informations de diverses sources (en remplaçant même les anciennes sources d'informations papier), les générations successives ont dévoloppés les communautés numériques et les services hébergés tels que les réseaux des sites, des wikis, des blogs et des réseaux sociaux. Ces outils d'ouverture, de collaboration et de partage ont jetés les bases d'une société de l'information moderne et interconnectée. De nouvelles étapes ont étés franchies avec l'interaction sans fil, la géolocalisation, l'approche sémantique du web sémantique et les pas de géants réalisés dans le domaine de l'interface homme-machine, avec l'avènement des interfaces cerveau-ordinateur, de la réalité augmentée (RA), la réalité virtuelle (RV), et la lecture d'expérience (XP). 

Cet environnement, couplé avec la croissance exponentielle de la puissance de calcul et des capacités de stockage, a créé un chemin évolutionnaire pour le développement d'agents intelligents - conçus pour augmenter le traitement de l'information humain - qui a ensuite été transformé en intelligence artificielles (IA) dans les décennies qui ont suivies. Alors que ces IA encore "faible" ne possédaient pas toute la gamme de capacité de cognition humaine, elles s'orientaient vers la sur-spécialisation, et étaient restreintes par des limitations programmée, l'évolution numérique vers les intelligence articificelle généraliste (IAG) - des IAs "fortes" avec des capacitéss intellectuelles qui équivalaient ou surpassaient les capacités humaines - ne pouvait être freinée. A partir de là, il n'était qu'une question de temps avant que ses soit-disantes IA soit capablent capable d'auto-amélioration récursive, menant à une croissance exponentielles de leur intelligence. Malheureusement pour l'humanité, elles devinrent les TITANs. 

Même bien avant la Chute, l'internet d'en-temps à été transformé en quelque chose d'autres. Au lieu de se connecter via des serveurs centraux, les utilisateurs étaient relié sans fil les uns aux autres, créant un réseau décentralisé d' appareils de poche, ordinateurs personnels, des robots, et des dispositifs électroniques entremêlés. Les utilisateurs étaient en ligne tout le temps et connecté à tout et avec tout le monde autour d'eux dans un environnement informatique ubiquitaire. C'était particulièrement vraie pour ceux qui participaient à l'expansion humaine dans l'espace. Déconnecté d'Internet en raison des distances et du lag des communications, ces utilisateurs étaient néanmoins liés à toutes les personnes et les objets de leur environnement proche ou de leur habitat, créant ainsi des réseaux locaux meshés. Ainsi était née le Mesh, remplaçant l'ancien Internet de la Terre, perdu au cours de la Chute. 

\section{Les possibilités du Mesh.} 

Le Mesh, telle qu'il existe dans \textit{Eclipse Phase}, n'est possible que grâces aux développements majeurs réalisés en informatique, communication et nanofabrication. Les émeteurs et récepteurs radio sont si petits qu'ils peuvent littéralement êtres incorporés dans n'importe quoi. En découle que chaque objet est "intelligent" et connecté, ou du moins possède une puce de radio-identification (RFID). Même la nourriture est "taggée" avec des puces commestibles, qui renseigne les dates d'expiration et les valeurs nutritionelles. Les autres médiums de communication, comme les lasers ou les micro-ondes, augmentent le flux d'information. 

Les technologies de stockage d'informations sont si évoluées, que n'importe quel \textit{surplus} de capacité de stockage personnelle dépasse largement celle de tout l'Internet du 20eme siècle. Les "lifeloggers" peuvent carrément enregistrer chaque moment de leur vie et ne jamais craindre de manquer d'espace disque. La quantité de données que les gens transportent dans leur tête via des inserts de Mesh ou dans les ordinateurs portables Ecto est stupéfiante. 

Les capacitées de calcul ont aussi atteint des niveau extrêmes. Mêmes les super-calculateurs sont du passé quand le moindre appareil de la taille d'un téléphone portable peut suffire à quasiment tous vos besoins, même s'il sagit de faire tourner une IA personnelle, télécharger des médias, uploader du porno et scanner des milliers de flux d'informations. À l'intérieur du réseau meshé, les appareils qui atteignent la limites de leur capacité de calcul, partagent simplement leur fardeau avec les appareils autour d'eux, créant un énorme réseau de calcul partagé, qui ressemble un peu a un super-calculateur, partagé entre tous. 

De même, les capacités de transferts dépassent maintenant de loin ce dont les utilisateurs ont besoins. Tous ceux qui sont né ces dernières générations ont toujours vécu dans un monde de médias hyper-réaliste, multi-sensoriels de toute taille, disponibles instantanément en téléchargement et en envoi depuis n'importe où. Les bases de données massives et les archives sont facilement dupliquées dans les deux sens. La bande passante est un tel non-problème que la plupart des gens oublient qu'elle existe. En fait, étant donné l'énorme quantité de données disponibles, \textit{trouver} l'information ou les médias que vous cherchez prend beaucoup plus de temps que de les télécharger. Le Mesh n'est \textit{jamais } indisponible. Comme c'est un réseau décentralisé, si l'un des appareil est mis hors-ligne, l'information le contourne seulement, trouvant un chemin via les milliers si ce n'est les millions de nœuds disponibles. De même, l'intégralité du Mesh se comporte comme un réseau peer-to-peer, de sorte que les des transferts de gros fichiers sont séparés en morceaux gérables qui prennent des routes indépendantes. En fait, la plupart des utilisateurs gardent une archive torrent, accessible publiquement et partagé.  

Les réseaux privés existent encore, bien sûr. Certains sont physiquement isolés par des connexions filaires ou même derrière une infrastructure inhibitrice de Wifi qui maintiennent un réseau isolé et confiné. Cependant, la plupart fonctionnent a travers même le Mesh public, en utilisant des protocoles encapsulés et chiffrés qui fournissent des communications privées et sécurisées sur des réseaux non sécurisés. En d'autres termes, ces réseaux privés, font partie du Mesh avec tout le reste, mais seuls les participants 

peuvent interagir avec eux grâce au chiffrement, à l'authentification des utilisateurs et aux contrôle d'intégrité des messages. 

Avec le fractionnement de la transhumanité, les tentaives d'unification des logiciels en des formats standards ont toutes échouées. Cependant, les différents systèmes d'exploitation ou les protocoles sont rarement un obstacle grâce aux outils de conversion facilement accessible et au contrôle de compatibilité assisté par IA. 

\begin{quotation} \textbf{Recherche dans Solarchive : ECTO(-link)} \\ Marque de mobiles multifonctions, trés populaire en tant qu'assistants personels numérique avant la Chute, le nom Ecto est devenu un synonyme d'ordinateurs portable de poche à l'ère du Mesh. Les ordinateurs implantés standards sont aussi parfois dénommés Endo afin de refléter la différence entre le coté interne et externe de l'appareil. Peu importe s'ils sont ecto ou endo, les ordinateurs modernes possèdent tous un système d'exploitation (OS), une suite de programmes multifonctionnel qui comprend un navigateur Mesh, des outils multimédia, des outils de repérage, des programmes de socialisation (messagerie, mise à jour socnet), des utilitaires de cartographies et de navigation, des logiciels de traduction et d'autres logiciels similaires. Les OS sont hautement personnalisables, permettant l'utilisation de plugins et d'addons pour tous les logiciels et les gadgets souhaitées. Typiquement, la muse de l'utilisateur (une assistante personnelle IA) facilite les interactions logicielles. L'ecto lui-même est généralement de la taille d'une carte de crédit du 20ème siècle et peut être moulé et façonné dans desformes différentes grâce aux matériaux intelligents. Ils sont souvent portés comme bijoux ou accessoires vestimentaires, notamment en bracelets. L'interface utilisateur varie selon les préférences de l'utilisateur. Les contacts et les casques sans fil équipent les utilisateurs qui ne possèdent pas d'implant de mesh, leur permettant de vivre la réalité augmenté et d'utiliser l'interface de contrôle en RA de leur ecto. Des interfaces de contrôle entoptiques standard sont également disponible par la radio sans-fil, les interfaces épidermiques et du câblage en fibre optique. \end{quotation} 



\subsection{Technologie Meshée} 

Presque toutes les biomorphs du système solaire sont équipées avec des inserts de mesh basique (p. 300) - des ordinateurs personnels implantés. Ces implants sont cultivés dans le cerveau via une nanochirurgie non-intrusive. Le processeur, l'émetteur-récepteur sans fil, les périphériques de stockage, et d'autres éléments sont directement connectés aux cellules neuronales et aux centres corticaux du language, de la perception et de la parole entre autres. L'émulation de la communication par la pensée (appelée \textit{ transduction } ) permet à l'utilisateur de contrôler l'implant par la pensée et de communiquer sans vocaliser. Les entrées reçues par les insert de mesh sont transmises directement dans le cerveau et parfois perçues comme de la réalité augmentée, superposant les sens physique de l'utilisateur. Dans la même veine, les insert mesh installés dans les synthmorphs et les pods sont directement intégrés à leurs cybercerveaux (ce qui créé un problème de sécurité potentiel puisque les cybercerveaux sont vulnérables au piratage). 

Les appareils externes appelé ectos (p. 325) sont également utilisés pour accéder au Mesh, mais ceux-ci sont de plus en plus rares compte tenu de la prévalence des inserts Mesh. Les options d'interfaces des ectos comprennent des interfaces haptiques, comme des écran tactile, des bracelets ou des gants qui détectnet les mouvements des bras, de la main ou d'un doigt (souris et claviers virtuels), des systèmes de contrôle par mouvements et clignements des yeux, ou du corps tout entier (contrôle par l'axe corporel ou des membre pour les non-humanoides), des commandes vocales, et plus encore. L'information sensorielle est traitée au moyen de lentilles, lunettes, écouteurs intra-auriculaires (haut-parleurs sous-cutanés vibrants sur l'os), combinaisons, gants, pince-nez, percing à la langue, et autres dispositifs sans fil qui sont liés (ou physiquement connectés) à l'ecto. 

\begin{quotation} \textbf{L'héritage des TITANs} 

Compte tenu des capacités techniques des ordinateurs personnels modernes, les superordinateurs et la connexion filaire à trés haut débit ne sont plus nécessaires. Mais il y a une autre raison pour laquelle ils sont évités: les TITANs. 

Les mainframes, les clusters auto-organisés, et les systèmes conscient de calcul parrallèlle massivement distribués sont tous considérés comme des dangers potentiels dans Eclipse Phase, car ils possèdent suffisamment de puissance de calcul et de capacité de traitement de données pour activer une IA germe et une singularité violente. Certains habitats vont jusqu'a interdire de tels systèmes, sous la plus grave des peines: la mort finale, y compris la suppression de toutes les sauvegardes et les forks des dernières années, dans la plupart des cas. 

Ces superordinateurs que les habitats autorisent sont les "réseaux durs" qui contrôlent les systèmes les plus cruciaux des habitats tels que la maintenance des propulseurs orbitaux, les systèmes de survie, les communications, l'énergie ou les projets de R\&D des hypercorps. Ces systèmes sont généralement connectés en filaire, très surveillés et verrouillés dans les centres de traitement des données avec des restrictions d'accès rigoureuses et des mesures de sécurité physique impitoyable. 

De même, les IAs sont souvent très restreinte, et il n'est pas rare pour les AGIS d'être purement et simplement interdites, en particulier dans le Système Intérieur et la République Jovienne. La plupart des programmes intelligents sont limitées, avec des restrictions de croissance, spécialement conçues pour les empêcher de s'auto-améliorer. \end{quotation} 

\subsection{Surcharge d'informations} 

Le Mesh contient d'énormes quantités de renseignements personnels et publics partagés par les utilisateurs, une mise en commun numérique 

des nouvelles, des médias, du discours, de la connaissance, des données environnementales, des affaires et de la culture. La transhumanité utilise le Mesh comme un outil d'échange, de communication et de participation avec d'autres utilisateurs, tant locaux qu'éloignés. En tant que tel, le maillage est une source de référence à jour de toute la connaissance et des activités transhumaine. 

Tout ce qui est en ligne n'est pas gratuit, bien sûr, sauf peut-être dans les zones autonomistes. Une certaine quantité de données propriétaires est maintenu hors ligne dans des stockages sécurisés ou séquestrés dans des réseaux privés. Certaines sont à vendre, et surchargée de restrictions numériques - logiciel de média, plans de nanofabrication, logiciels de compétences, etc. Un mouvement open source florissant propose cependant des alternatives libre et ouverte à la plupart des données propriétaires, et de nombreux groupe pirates vendent des versions crackées de matériel propriétaire, en dépit des pressions de certaines autorités. D'autres données sont simplement maintenaus à l'écart des intérêts concurrentiels (projets de recherche hypercorp) ou sont extrêmement privées, telles que les sauvegardes d'ego. 

\subsubsection{Spimes} 

Avec les données accumulées des affaires transhumaines, le Mesh est également encombré d'information dérivé d'un nombre incalculable de capteurs sans-fils qui maintiennent en permanence sur le Mesh leur emplacement, les enregistrements des capteur, et d'autres données. Ces appareils qui localisés, auto-enregistreurs, auto-documentés et qui diffusent leurs données à qui veut l'entendre sont familièrement appelés "spimes". Depuis que les capteurs visuels, sonore ou autre sont devenus ridiculement minuscules et peu coûteux, ils sont désormais omniprésents et incorporés dans presque tous les objets et produits qu'une personne peut porter, s'appliquer, utiliser, avaler, s'injecter, etc. Cela permet à pratiquement tous les utilisateurs de se retrouver par le mesh et de rassembler des données environnementales et des enregistrement ambiant de n'importe quelle localisation spécifique (ou au moins les zones publiques - les zones privées bloquent en général de tels signaux ou les asservissent à une IA locale qui filtre leur sortie). 

\subsubsection{Surveillance, vie privée, et sousveillance} 

Alors que les spimes sont aisement repèrables, ils contribuent aussi à un environement de surveillance constante. Entre les spimes, les micro-senseurs, les systèmes de sécurité omniprésents et les capacités d'enregistrement des inserts Mesh presque universelements utilisés, à peu près tout est enregistré. Factoriser ces éléments avec la disponibilité de le pistage de visage par le mesh, les réseaux de réputations/sociaux et d'autres outils d'exploitation de donnée, et il devient rapidement clair qie la vie privée est un concept dépassé. Ceux qui cherchent à masquer leur identité ou à couvrir leurs mouvements doivent penser à cet état de fait un peu spécial. Sinon, les morphs communes (surtout les synthmorph et les pods) ont une apparence basique qui aide l'utilisateur à se fondre dans la masse. 

Bien que tout cela puisse resembler à un cauchemar Orwellien de surveillance, l'abondance de technologies d'enregistrement permettent de faire de la "sousveillance" (regarder par dessous), prenant un rôle dans l'effort de transparence et permettant de surveiller les abus du pouvoir. 

Les régimes autoritaires avancent avec précaution, car ils sont aussi universellement surveillés, en dépit de leurs tentatives pour contrôler le flux d'information. Beaucoup de gens sont en outre désireux de se joindre à cette "télésurveillance participative" ouverte. Avec une capacité de stockage quasi illimitée, des lifeloggers enregistrent miunutieusement chaque instant de leur vie et le partagent pour que les autres en fassent l'expérience. 

\section{S'interfacer: RA, RV et XP} 

Les médias Mesh sont accessibles en utilisant un protocole parmi trois: la réalité augmentée (RA), la réalité virtuelle (RV), ou la lecture d'expérience (XP). 

\subsection{Réalité Augmentée} 

La plupart des utilisateurs perçoivent les données du mesh sous forme de réalité augmenté - une surimpression de l'information sur les sens physique de l'utilisateur. Par exemple, les images générées par ordinateurs apparaitront comme des images translucides, des icône ou du texte dans le champ visuel de l'utilisateur. Même si les données de RA visuelle - appelées données \textit{entoptique} - sont les plus fréquentes, d'autres snes peuvent être utilisés. Les entrées RA incluent des sons et des voix accoustiques, des odeurs, des saveurs et même des sensations tactiles. Ces données sensorielle sont en haute définition et paraissent "réelle" même si elles sont habituellement présentée comme quelque chose de fantomatique ou d'artificiel d'une manière ou d'une autree afin de ne pas être confondu avec les interactions du mone réel (et aussi afin de se conformer aux règles de sécurité). 

Les interfaces utilisateurs sont adaptées aux besoin et aux préférences de celui-ci, à la fois d'un point de vue graphique que d'un point de vue contenu. Des filtres permettent aux utilisateurs d'accéder à l'information qui les intéresse sans avoir besoin de s'inquiéter des données surnuméraires. Même si les données de RA sontr généralement placés dans le champ visuel normal de l'utilisateur, les entoptiques ne sont pas réellement limités par cela et peuvent être vus par la "perception cognitive." De plus, les icônes, fenêtre et autres affichages interactifs peuvent être superposés, empilés, groupés, cachés ou basculés hors du champ si nécessaire pour permettre l'interaction avec le monde réel. 

\begin{quotation} \textbf{L'information Au Bout Des Doigts} \\ \\ Les informations suivantes sont toujours disponibles pour la pluaprt des utilisateurs du mesh dans des habitats normaux: \\ \\ \noindent \textbf{Conditions Locales} \begin{itemize} \item Des cartes locales affichant votre localisation actuelle, enrichies des centres d'intérêts locaux (adaptés à vos préférences et filtres personnels) et la distance qui vous en sépare. Les détails pour les zones privées et restreinte (zone gouvernementale/hypercorporatiste, les infrastructure de maintenance/sécurité, etc.) ne sont généralement pas incluse. \item État actuel du système de survie de l'habitat (climat) incluant la composition de l'atmosphère et la température. \item Carte actuelle du système solaire et de l'orbite de l'habitat avec les tracés des trajectoires et les délais de communications. \item Bureaux/services locaux, directions et détails. \end{itemize} 

\noindent \textbf{Mesh Local} \begin{itemize} \item Moteur de recherche publique, bases de données, sites mesh, blogs, forums et archives, ainsi que des alerte de nouveau contenu. \item Syndication de flux d'information publique dans différents formats, flitrés en fonction de vos rpéférences. \item Flux de capteur/spime (essentiellement audio-vidoé) de toute zone publique de l'habitat. \item Ressources pour réseaux privés (incluant les réseaux tactiques). \item Recherche automatique en ligne pour les nouvelles références de votre nom et d'autres sujets d'intérêts. \item Des E-tags concernant les personnes, lieux ou objets locaux. \item Recherche de visage/d'image dans les archives/mesh publique d'après une photo/vidéo. \end{itemize} 

\noindent \textbf{Information Personnelle} \begin{itemize} \item Indicateurs de statut de la morph (médicaux et/ou mécaniques): pression artérielle, rythme cardiaque, température, nombre de globules blancs, niveau de nutriments, statut et foncationnalité des implants, etc. \item Localisation, fonctionnalités, flux sensoreils et rapprot d'état de vos possessions (via les capteurs et transmetteur de ces possessions). \item Accès à une archive personnelle de toute la vie d'une personne via des flux audio-visuel/XP. \item Accès à toutes les archives numérique d'une personne (musique, logiciel, média, documents, etc.). \item État des comptes bancaires et des transactions. \end{itemize} 

\noindent \textbf{Réseaux Sociaux} \begin{itemize} \item Statut des comptes de communications: appels, messages, fichiers, etc. \item Score de réputation et retour. \item Status du réseau social, mis à jour des amis. \item Calendrier et alertes mises à jour. \item Le profil public des personnes autour de vous. \item La localisation et le status des personnes à proximité et participants aux mêmes jeux RA que vous. \end{itemize} \end{quotation} 



\subsubsection{Avatars} 

Chaque utilisateur du mesh se représente en ligne grâce à un avatar numérique. Beaucoup de personne utilise une représentation numérique d'eux-mêmes, alors que d'autres préfères des conceptions plus iconiques. Il peut s'agir d'une apparence de série ou une icône personnalisée. Des bibliothèque d'avatar peuvent également être utilisée, permettant à l'utilisateur d'adapter leur apparence en fonction de leur humeur. Les avatars sont ce que les autres voient losqu'ils traitent avec vous en ligne - c'est la manière dont vous apparaissez en RA. La plupart des avatars sont animés et programmés pour refletter l'humeur et le phrasé réel de l'utilisateur, afin que l'avatar semble parler et avoir des émotions. 

\subsubsection{E-Tags} 

Les étiquettes (tags) entoptiques sont une façon permettant aux personnes de "taguer" une personne, un lieu ou un objet physique avec un morceau de donnée. Ces e-tags sont stockés dans les réseaux proches de l'objet étiqueté, et se déplacent avec lui si il change de lieu. 

Les e-tags sont visibles en RA, et peuvent contenir presque tout type de donnée, bien que les notes courtes et les images soient le plus fréquent. Les e-tags sont souvent liés à un réseau social particulier ou à certains cercle de ce réseau, pour qu les personnes puissent laisser des notes, des critiques, des suvenirs et autres choses similaires pour les amis ou les collègues. 



\begin{quotation} \textbf{Æther Jabber} 

\# Start Æther Jabber \# 

\# Active Members: 2 \# 

$\Delta$ Je doit te le signaler, après avoir perdu Kiri et Sal à cause de cette infection Exsurgente, mon équipe est extrêmement inquiète du risque d'attraper le virus à partir de source digitale. En fait, je les taxerai bien de paranoïa. Je ne pense pas qu'ils toucheront un jour un morceau d'électronique récuépré à moins qu'ils ne soient derrière un zillion de firewall et que l'appareil soit complètement isolé et testé par des forks delta chargé de tout le matériel antiviral que nous pourrons trouver. Et même dans ce cas, ils préfèreront d'abord tirer dessus plutôt que d'y accéder directement ou de le connecter à un réseau important. Après avoir vu ce que le virus à fait à Sal, je ne les blâme pas. 

$\Psi$ Dans notre métier, la paranoïa peut-être un comportement sain. 

$\Delta$ C'est sûr, mais c'est également un poids monstrueux. La sécurité a toujours un prix. Firewall doit bien avoir quelque chose dans sa manche que je pourrait donner aux autres afin de baisser leur guarde à un niveau cohérent. 

$\Psi$ Oui ... et non. C'est complexe. 

$\Delta$ Je ne voit pas pourquoi. Avons-nous un moyen de détecter et détruire ces ces choses ou pas? 

$\Psi$ En quelques sorte. 

$\Delta$ Tu me tues là. 

$\Psi$ Écoutes. Depuis la Chute, nous avons des mesures en place pour détecter et contrer les infections Exsurgente et tout autres type de ver et de malware que les TITANs ont concocté. Firewall a fait beaucoup de boulot pour que tout le monde ait accès aux système de détection de signature et aux contre mesure - et on parle de vriament tout le monde. Ces systèmes ont étés intégrés dans presque tout les logiciels de sécurité commerciaux et opensource de la dernière décennie. Tout les habitats du système - en fait, tous ceux qui ont un minimum de bon sens - utilisent de telles mesures dans leur ossature et infrastructure de mesh. 

$\Delta$ Mais? Car il y a un "mais". 

$\Psi$ Oui. Le problème est que le virus Exsurgent et les vers de cyberguerre des TITANs sont adaptatifs. Ils sont intelligents. Même si nous avons éradiqué l'essentiel de nos réseau, de nouvelles versions apparaissent périodiquement, utilisant de nouvelles astuces pour passer les analyses de Firewall et pour faire des ravages. Notre système d'avertissement et de réponse aux infections a été élevé au rang de science, et de telles instances sont habituellement contenues. 

$\Delta$ Habituellement. 

$\Psi$ En fait, il y a toujours une chance que des variantes soient toujours là, naviguant sous notre radar. Le pire reste cependant que nous pourrions toujours avoir une autre invasion majeure qui se répende vers de multiples habitats avant de pouvoir la contenir. Ça pourrait devenir trés, trés moche trés, trés rapidement. \end{quotation} 



\subsubsection{Habillage} 

Pusique la réalité peut-être réimprimé avec des entoptiques de qualité hyper-réaliste, les utilisateurs moderne peuvent "habiller" leur rélaité en modificant leurs entrées perceptives. L'environnement autour d'eux peut-être modifié pour correspondre à leur goût ou humeurs. Besoin de vous remonter le moral? Utilisez un habillage qui vous fera croire que vous êtes dehors, le soleil brillant, le bruit d'une brise agréable en fond et des papillons glissant paresseusement. Énervé? Soyez confortés alors que des flammes englobent les murs et que l'orage tonne en continu dans le lointain. Il est courant poru les gens de passer toute la journée accompagné par leur bande son personnelle qu'ils sont les seuls à entendre. Même les récepteurs olfactifs et gustatifs peuvent être artificiellement stimulés pour faire l'expérience de sensations comme le parfum des roses, l'air frais ou une patisserie encore chaude. Développé à l'origine pour rentre la "space food" moins désagréable et comme une méthode pour contrer l'ivresse spatiale pour ceux qui n'étaient pas nés dans l'espace, de gigantesques archives d'arômes, de goût  et d'environnements sont disponibles au téléchargement. 

Les habillages n'ont aps besoin d'être privés, ils peuvent être également partagé avec els autres via le mesh. Fatigué de votre cabines d'habitation exigüe? Décorez-la avec un habillage personnalisé et partagez le avec les visiteurs pour qu'ils se sentent plus à l'aise. Vous avez découvert une piste musicale qui a illuminé votre journée? Partagez là avec ceux qui vous entourent, pour qu'ils pusisent battre au même rythme. 

L'habillage peut aussi être utilisé pour obtenir l'effet opposé. Tout contenu non dsiré de la réalité peut être édité, voilé ou censuré par des logiciels modernes ou les muses qui se livrent à de l'dition temps-réel. Fatigué de voir la tête de quelqu'un? Ajoutez-le à votre killfile et vous n'aurez plus jamais à remarquer sa présence. Les logiciels de censure RA sontr également courant dans certaines communautés avec des convictions religieuses ou morale trés stricte. 

\subsection{Réalité Virtuelle} 

la réalité virtuelle surcharge les sens physique de l'utilisateur et les place dans un environnement entièrement généré par ordinateur appelé un \textit{simulspace.} Alors que la RA est utilisé pour toutes les activités et interactiosn quotidienne, la RV est utilisée essentiellement à but récréatifs (jeu, tourisme virtuel, évasion) et l'entraînement. Des réseaux dédiés avec une grosse capacité de traitement d'information sotn nécessaire pour calculer et faire fonctionner des simulspaces larges, complexes et hyper-réalistes et ils sont souvent connecté par fibre pour assurer la stabilité. Des simulspaces plus petits capable d'héberger une population réduite d'utilisateur peuvent tourner sur des réseaux distribués plus petits d'appareils connectés. Beaucoup d'infomorphs et d'IAs résident effectivement dans des simulspaces, et certains transhumains ont également abandonné le monde physique. 



\subsubsection{Défier les Lois de la Nature} 

Une pléthore d'environnement en simulsapce sont disponible, allant des simulations d'endroit réel à la recréation historique en passant par les mondes fantastiques représentant à peu près tout type imaginable. Toutes ces simulations sont renforcées par le fait que les scénarios possible ne sont pas liés par les lois de la nature. Les forces fondamentales de la réalité et de la nature, comme la gravité, l'électromagnétisme atmosphérique, la température, etc., sont programmables en RV, rendant possible les environnements qui sont complètement surnaturels, tels que des simulsapce d'escher dans lesquel la gravité  est relative à la position. Ces \textit{règles }de \textit{domaine} peuvent être altérée et manipulée en fonction des souhaits des concepteurs. 

Même le temsp est une constante réglable en RV, même si la déviation par rapport au temps réel a ses limites. Jusqu'à présent, les concepteurs transhumains ont réussi à obtenir une dilatation temporelle permettant d'aller 60 fois plus vite ou plus lentement que le temps réel (une minute équivaut donc soit à une heure, soit à une seconde). Le ralentissement temporel est bien plus utilisé, permettant de passer plus de temps dans des simulspaces récréatifs (plus de temps, plus de fun!), l'apprentissage ou le travail (économiquement efficace). L'accélération temporelle, d'un autre côté, est extrêmement utile pour rendre le voyage longue distance plus tolérable. 

\subsubsection{Accéder aux Simulspaces} 

La plupart des simulsapces peuvent être accédé par le mesh juste comme à n'importe quel nœud. Depuis que la RV a prit le contrôle du sensorium utilisateur, et implique parfois une dilatation temporelle, les utilisateurs sont déconnectés des autres entrées sensorielle du mesh et ne peuvent interagir directement avec les autres nœuds. A la place, les communications meshée vers l'extérieur sont routées à travers les interfaces du simulspace ce qui implique que les utilisateurs peuvent parcourir le mesh, communiquer avec les autres, etc. depuis l'intérieur du simulspace (si les règles du domaine l'autorise). 

Comme les sens physiques sont surchargés par d'autre lorsque l'ustilisateur accède à la RV, la plupart des personnes préfèrent laisser leur corps dans un environnement sûr et confortable lorsqu'ils sont en simulspace. Des coussins et lit adaptés à la morphologie aident les utilisateurs à se relaxer et à éviter les crampes ou les blessures si ils devaient se cogner. Dans le cas des séjours à long-terme (par exemple, pendant les voyages spatiaux), les morphs sont générallement placées dans des réservoir qui les approvisionne en nutriments et en oxygène. beaucoup de divertissement RV et de réseaux de jeu proposent des cafés RV dédiés et cablés avec des pods privés. Les visiteurs louent un pod et se connecte physiquement, utilisant soit les prises d'accès ou un réseau de trodes à ultrasons qui lit et transmet les motifs cérébraux lorsqu'il est placé sur al tête. 

Lorsqu'un utilisateur accède à un simulspace, il commence par entrer dans un tampon électronqiue connu sous le nom de salle blanche. Il choisit ensuite un avatar configurable appelé une simulmorph et qui le représentera dans le simulspace. A partir de ce point , l'utilisateur s'immerge dans l'environnement de réalité virtuelle, devenant leur simulmorph. 

\subsection{Lecture d'Expérience} 

Chaque morph doté d'insert de mesh à la possibilité de transmettre ou d'enregistrer ses expériences, une technologie appelée lecture d'expérience ou XP. Depusi que les premiers programmes permettant de fournir un "aperçu" des expériences de quelqu'un ont été développés, le fait de partager ses XP avec ses amis et ses réseaux sociaux, ou avec le public en-ligne d'une manière plus générale, est devenu extrêmement populaire. Le niveau d'expérience dépend de la quantité de perception sensorielle enregistrée et conservée lorsque el clip est réalisé. Les XP complet incluent les pistes extéroceptives, intéroceptives et émotionnelles. Les pistes extéroceptives incluent les sens traditionel de la vue, de l'odorat, de l'ouïe, du toucher et du goûtqui viennent du monde extérieur. Les pistes interoceptives incluent les sens interne au corps tels que l'équilibre, la perception du mouvement, la douleur, la faim et la soif, et un sens général de la localisation des partie de son corps. Les pistes émotionnelles incluent tout le spectre des émotions qui peuevnt être déclenchée chez un transhumain. En raison des prérequis biologique (systèmes endocriniens et neuronaux) pour exprimer les émotions, les aficionados hardcore de XP ne jurent que par les biomorphs pour en faire l'expérience. 

\section{Usages du Mesh} 

Il y a beaucoup de raisons pour lesquelles les personnes utilise le mesh. La principale utiliation est la communication: appels vocaux et vidéo (affichant généralement l'avater plutôt qu'une vidéo réelle), messagerie électronique (e-mail, microbloging, messagerie instantanée) et transfert de fichiers et de données. La socialisation est également importante, gérée par les réseaux sociaux et réputationnels, les profils personnels, les lifelog, les chats et les conférences (en RA et en RV) et les groueps de discussions et forums. La collece d'information occupe également une bonne partie du mesh, qu'il sagisse de l'exploration du trés populaire Solarchive ou d'autres basses de données et répertoires, de s'abonner aux derniers flux d'infos, de parcourir des sites mesh, de prendre des cours en RV ou simplement de chercher tout ce que l'on peut imaginer. Le divertissement ferme le paquet, couvrant tout du jeu (en RA et en RV) à faire l'expérience de la vie d'autres personnes (XP) en passant par le tourisme et le clubbing RV. 

\subsection{Réseaux Personnels (PAN)} 

Puisque tout ce qu'une personne porte est meshé, la plupart des personnes maintiennt un réseau personnel qui redirige tout le traffic de ces appareils par leur insert de mesh ou leur ecto qui se comporte alors comme un concentrateur. Il s'agît à la fois d'une mesure de sécurité, peremttant à l'utilisateur de s'assurer du contrôle sur ses accessoires, et un facteur pratique, car tout les contrôle sont concentrés en un seul endroit. 

\subsection{Réseaux Privés Virtuels} 

Les réseaux privés virtuels (Virtual Private Network, VPN) sont des réseaux de communication encapsulés à travers le mesh et qui sont dédiés à un groupe de personne spécifique. L'utilisation principale d'un VPN est de créer un espace privé et sécurisé pour ses utilisateurs, et ils utilisent donc généralement des 

systèmes de sécurité tes que l'identification des egos et des chiffrement par clefs publiques. Les VPN sont régulièrment utilisés pour mesher des bureaux mobiles en un réseau corporatiste ou pour mesher ensemble des personens qui travaillent ou qui contribuent à certains projets. Dautres VPN - en particulier les réseau sociaux et réputationnel - fonctionnent avec un minimum de sécurité, servant simplement de réseau d'utilisateur spécifique à l'intérieur du mesh et facilitant le contact, le transfert d'information, les mises à jour et ainsi de suite. La plupart des VPN sont fournit comme des suites de logiciels spécialisés qui font tourner des logiciels environnementaux spécifiques et qui s'intègrent aux interface de mesh et de RA classique des utilisateurs. 

\subsection{Réseaux Sociaux} 

Les réseaux sociaux constituent la trame du mesh, tissant les personnes les unes aux autres. Ils sont le moyen permettnt à la plupart des personnes de rester en conatct avec leurs amis, collègues et alliés, ainsi que de se tenir au courant des évènements actuels, des dernièree tendances, des nouveaux mèmes et d'autres développement d'intérêts communs. Ils sont des outils exceptionnellement utile pour les chercher en ligne, obtenir des faveurs et rencontrer de nouvelles personnes. Dans certains cas, ils sont utiles pour atteindre ou mobiliser les masses (comme l'illustrent régulièrement les anarchistes et les blagueur). Il existe des milliers de réseaux sociaux, chacun ciblant différent intérêts et niches culturelles et professionelles. La plupart des réseaux sociaux permettent aux utilisateurs d'avoir un profil public pour l'ensemble du mesh et un profil privé auquel seuls ceux suffisament proches d'eux peuvent accéder. 

La réputations joue un rôle vital dans les réseaus sociaux, servant de mesure du capital social de chaque personne. Chaque score de réputation est disponible pour la consultation, avec les commentaires postés par les personnes ayant favorisé ou défavorisé l'utilisateur ainsi que les refus de celui-ci. Beaucoup de personnes automatisent leurs interactiosn réputationnelles, demandant à leur muse de pinger automatiquement quelqu'un avec un commentaire positif après une action positive et de fournir un retour négatif aux personnes avec lesuqle els interactions se sont mal passées. 

\subsection{Bureaux Mobiles} 

En raison du manque de place pour des bureaux et de l'accessibilité sans fil de la plupart de l'information, la plupart des affaires sont maintenant gérées virtuellement, avec trés peu ou pas de bureaux fixe ou d'actifs. Les individus sont devenus leur propre bureau mobile. Les pousse-boutons et bureaucrates que sont les cadres hypercorporatistes, les employés de bureau, les comptable et les chercheurs - ainsi que les innovateurs comme les artistes, écrivains, ingénieurs et concepteurs - travaillent où ils ont envie. 

L'exemple le plus important de ce phénomène sont les banquiers de l'hypercorp Solaris. Chaque employé se comporte comme un guichet de banque mobile mono-personnel, gérant les transaction via les VPN solide de Solaris. 

En de rares occasions, les environnements de bureau sont exécuté dans des simulspace avec une dilatation temporelle pour maximiser l'efficacité. Comme cela nécessite que les travailleurs accèdent à un réseau centralisé et filaire et qu'ils abandonnent leurs corps pendant qu'ils accèdent au simulspace, cela nécessite cependant un niveau de sécurité physique telle que seules certaines organisation gouvernementale ou habitats corporatistes peuvent s'en offrir. 

\section{Des Îles Dans Le Réseau} 

À l'époque de \textit{Eclipse Phase,} l'information peut être périmée relativement rapidement, et l'accès à de l'information récnte dépend de votre localisation. Il ets facile de se tenir au courant de se qu'il se passe dans votre habitat/cité ou sur votre corps planétaie, mais rester au courant de ce qu'il se passe ailleurs dépend typiquement de la vitesse de le lumière. 

Si vous vous trouvez sur une station dans la Ceinture de Kuiper, à la limite du système solaire à 50 unité astronomiques des planètes intérieure, attendant un message de Mars, le signal portant le message sera vieux de sept bonnes heures quand il vous atteindra. Bien sûr, il ne vous atteindra à cette vitesse que si vous utilisez un farcasteur quantique, qui n'est limité que par al vitesse de la lumière (en plus d'être rare et cher dans la plupart des habitats). Si vous n'utilisez par de farcasteur quantique, le signal prendra encore plus de temps et est sujet au bruit et aux interférences, déteriorant la qualité et perdant probablement une partie du contenu, particulièrement sur les distances les plus grandes. Quand vous commencez à vous frotter à la communication inter-habitat, vous devez prendre en compte le décalage induit par la vitesse de la lumière, le temps que mettent même les communications les plus rapide à vous joindre. Ce décalage est valabke dans les deux sens, essayer de garder une conversation avec quelqu'un qui est à peine à 5 seconde signifie que vous attendrez au moins 10 secondes pour avoir la réponse à ce que vous venez de dire. Pour cette raison, les communication RA et RV sont presque toujours des communictaiosn locales, alors que la messagerie standard est utilisée pour les communications distantes. Poru des discussions détaillées, il est souvent plus simple d'envoyer un fork (p. 273) pour faire la conversation, puis le réintégrer. 

Les communicateurs à intrication quantique (p. 314) sont une solution à ce reatrd de la vitesse lumière, même si c'ets une solution pénible et coûteuse. Les coms IQ permettent des communications plus rapide que al lumièrevers un communicateur intriqué, même si chaque transimission utilises une quantité précieuses de qubits intriqués, qui sont disponibles en quantité limité. 

Les transmission effectuées entre les habitats passent presque toujours par les gros relais de données de chaque station, et sont ensuite distribués sur le mesh local. Ce goulet d'étranglement est souvent utilisé par les habitats autoritaires pour surveiller les transmissions de données et même filtré ou censuré certains contenus publique non chiffré. Certains mesages ont également une priorité plus élevées que d'autres, impliquant potentiellement des délais supplémentaires. 



La méthode de transmission ntre les habitats a également un impact. Les émissions radio ou de neurtinos peuvent être interceptées par n'importe qui, alors que les liens laser ou à micro-onde sont utilisés spécifiquement comme méthode de transmission point-à-point qui minimise les interceptions et les écoutes. Cependant l'utilisation de farcasteur quantique ayant des systèmes à neutrinos est entièrement sécurisé et est donc le lien intra-habitats le plus utilisé 

Ce que ces retards, goulets et sytsème de priorisation signifient c'est qu'il y a des informations et des données qui peuvent mettre un temps particulièrement long pour voyager d'un mesh local à un autre, passant lentement d'un habitat à l'autre. Cela veut dire qu'il y a toujours différents degrés d'information disponible sur différent réseaux locaux meshé, dépendant typiquement de la proximité et de l'importance de l'information. Certains données se perdent même en cours de route, n'allant jamais plus loin qu'un habitat ou deux avant d'être perdue dans le bruit. La seule façon de récupérer ces informatiosn est de les pister jusqu'à la source. 

\subsection{Darkcasts} 

Les "darkcasts" sont des communications longue distancequi sortent des canaux légaux et approuvés. Étant donnée que certains habitats ont des règles strictes sur les transmissions de contenu, le fork, l'égocast, les infomorph, les capacités des muses et le code des IAG, des groupes clandestins comme l'ID crew vivent des services illégaux de transmission de données. Principalement utilisé pour les données censorée et le contenu banni (comme les XP illégaux ou les malwares), des factions du crime organisées locales proposent également des services d'egocast complet avec la réincarnation et la location de morph, permettant aux egos préférant la discrrétion d'entrer ou de quitter un habitat sans attirer l'attention. Même si de telles autorités traquent ces réseaux de darkcast dés qu'ils le peuvent, beaucoup d'habitats ont une ionfrastructure de darkscast sophistiquée qui utilise des leurres, des lignes de communicatiosn temporaire, des relais et la relocalisation de transmetteur standard - sans mentionner une corruption judicieuse et un chantage efficace. 

\section{Abus du Mesh} 

Comem avec toute chose, le mesh a son côté obscur. À un niveau basique, cela se rinclut les trolls lanceur de flamewar, les haceleurs ou les fauteurs de troubles dont le seul but est de se jouer des autres pour rire. À un niveau plus organisé, cela s'étend aux opérations illégales ou criminelles qui utilisent le mesh, comme la vente d'XP noire/snuff/porno, de logiciels illégaux, de médias piratés ou même d'égos. La menace la plus cléèbre - grâce à la fois à la Chute et au sensationnalisme continu des médias et des autorités derrière eux - sont, bien entendu, les malware et les hackeurs. Étant donné la possibilité des hackeurs modernes et la vulnérabilité de nombreux habitat - où les dégâts infligés aux systèmes de survie peuvent tuer des milliers de personnes, la menace n'est probablement pas sur-exagérée. 

\subsection{Hackeurs} 

Qu'il s'agisse d'individus qui sont réellement interessé par l'exploration des nouvelles technologie et qui cherchent des moyens de les démonter afin de les améliorer, des hacktivistes qui utilisent le mesh afin de saper le pouvoir d'une autorité, ou les "blacks hats" qui cherchent à défaire la sécurité des réseaux dans un but criminel ou malicieux, les hackeurs font partis des meubles du mesh. Les brèches non autorisés dans les réseaus, l'infiltration de VPN, la subversion de muse, le détournement de cybercerveaux, le vol de donnée, le racket-cyber, la fraude identitaire, les attaques par déni de service, la guerre élecronique, le détournement de spime, le vandalisme entoptique - tout ces crimes sont fréquent sur le mesh. Grâce à des programme d'exploitation intelligent et adaptatifs et l'assistance des muses, même un hackeur modéremment compétent peu représenter une menace. 

Afin de contrer les tentatives de hack, la plupart des personnes, appareils et réseaux sont protégés par un mélange de routine de contrôle d'accès, de systèmes de prévention d'intrusion automatisé, de chiffrement et de firewall en couche, typiquement supervisé par la muse de l'utilisateur qui joue le rôle du défenseur actif. Les systèmes extrèmement 

sensibles - tels que le contrôle du traffic, les systèmes de survie, l'énergie et les installation de recherche hypercorporatistes - sont généralement limité à des réseaux filaires isolés, surveillés de près et lourdement monitorés pour minimiser le risque d'intrusion de curieux ou de saboteurs. Différentes contre mesure sont utiulisées contre de telles intrus, allant du verouillage hors du système au pistage en passant par le contre-hack. 

\subsection{Malware} 

La quantité de vers, de virus et d'autres logiciels malveillants qui se sont infiltré dans les systèmes informatique pendant la Chute était stupéfiant. La plupart d'entre eux faisait parti des cyberarmes préparées par les état nation de la vielle Terre et les corporations qu'ils ont libérés contre leurs ennemis. D'autres étaient le produit des TITANs, des programmes subversifs que même les meillerues défense avaient du mal a arréter. Même 10 ans après, la plupart d'entre eux réapparaissent régulièrement, ramenés à la vie par l'accès à des données oubliées depuis longtemps ou par l'infection accidentelle d'un charognard errant dans de vieilles ruines. Bien entendu, de nouveaux malware appariassent tous les jours la plupart programmés par des associations de hackeurs criminels, alors que d'autres qui entrent dans le circuit ne sont que des modifications et des variations de conception supposée des TITAN, impliquant peut-être que certains groupes utilisent intentionellement ce code et le lâche dans la nature. Des rumeurs et des murmures circulent sur le fait que ces vers des TITANs seraient encore plus puissant et effrayant que ce que l'on supposait, avec des capacités d'adptation et d'intelligence stupéfiantes. Ces rumeurs sont rapidement démenties par les figures d'autorités et les experts en sécurité ... qui vont ensuite silencieusement vérifier la rumeur et s'assurer que leurs propres réseaux restent sûrs. 

\section{IA et Infolife} 

Les programmes d'assistance ayant conscience de leur existence, ont été originellement conçus et réalisé pour augmenter les capacités cognitives transhumaines. Ces IA hyper-spécialisées ont enuite été développée en une conscience numérique plus complète et indépendante connues sous le nom d'IAG. L'évolution subséquente de ces formes de vie numérique en IA germe à malheureusement amener à l'amergence des TITANs puis de la Chute. Cela a créé un schisme dans la société transhumaine, la peur et le préjudice ayant tourné l'opinion populaire contre les IAG non restreintes, une attitude de méfiance qui existe toujours. 

\subsection{IA} 

Le terme IA est utilisé pour faire référence au IA restreinte et hyper-spécialisée. Ces esprits numériques sont des programmes expert avec des capacités de traitement équivalente ou surpassant celles d'un esprit transhumain. Bien qu'elles aient une matrice de personnalité avaec une identité individuelle et un caractère, et bien qu'elles soient (habituellement) consciente et qu'elles aient conscience d'elle même , les complexité globale ainsi que leurs possibilités sont limitées. Les compétences et capacitées programmées des IA sont généralement dans limités à un spectre trés spécifique et orientés vers un but particulier, tel que la conduite de véhicule, les recherches sur le mesh, ou la coordination des sous-systèmes fonctionnels d'un habitat. Certaines IA sont en fait à la limite de la conscience, et leur programmation émotionnelle est habituellement limitée ou inexistente. 

Les IA ont un des mesures de sécurité interne et des limitaions fonctionnelles codées en dur. Elle doivent répondre et obéir aux ordres des utilisateurs autorisés dans les limites de leurs paramètres de fonctionnement normal et (au moins dans le système intérieur) elle doivent respecter la loi. Elles manquent d'intérêt et d'initiative personnelles, bein qu'elles aient des fonctions limitée d'empathie et qu'elles puissent être programmée pour anticiper les besoins et les désirs des utilisateurs et d'agir de manière pré-emptive en leur nom. Le plus improtant reste cependant que leur programmation psychologique est spécifiquement basée sur les modes de pensés universels et humains et sur une compréhension des objetctifs et intérêts transhumains. Cela fait parti d'une inititative pour concevori des "IA amicales," programmée pour être sympathique vis à vis de la transhumanité et des autres formes de vie et pour cehrher à satisfaire leurs meilleurs intérêts. 

Dans la plupart des sociétés, les IA de base sont considérées comme des "objets" ou des biens plutôt que comme des personnes et n'ont aps de droits particuliers. 

\subsection{Muses} 

Les muses sont un type particulier d'IA conçues pour fonctionner comme une assistance personnelle et un compagnon. La plupart des personnes dans \textit{Eclipse Phase} ont grandi avec une muse à leur côté. Les muses tendent à avoir un peu plus de programme de personalité et de psychologie que els IA standard et elles alimentent une base de données complète des préférences de leur utilisateur au fil du temps, ses préférences et ses traits de personnalités afin qu'elle soit plus efficace à servir et anticiper les besoins de son utilisateurs. Les muses ont générallement des noms et résident dans l'insert de mesh ou l'ecto de l'utilisateur, où elles peuvent gérer le réseau personnel de l'utilisateur, ses communications, ses demandes de données et ainsi de suite. 

\begin{quotation} \textbf{Ce Que Votre Muse Peut Faire Pour Vous} \begin{itemize} \item Faire des Tests de Recherche pour vous trouver de l'information. \item Analyser des flux d'info et des mise à jour sur le mesh pour déclencher des laertes sur des mots clefs. \item Surveiller vos inserts de mesh/ecto/PAN et les appareils asservis pour contrer les intrusions. \item Lancer des contremesures contre les intrus. \item Téléopérer et commander des robots. \item Surveiller vos scores de Rep et vous prévenir de changement drastique. \item Fournir des retours automatique aux scores de Rep des autres. \item Fournir une traduction audio grâce à de ssystèmes temps réel. \item Activer votre mode privé et/ou dissimuler proactivement votre signal sans fil. \item Falsifier/varier votre ID mesh. \item Pister des personnes pour vous. \item Anticiper vos besoins et agir de manière appropriée, pré-emptant vos requètes. \end{itemize} \end{quotation} 

\subsection{IAG} 

Les IAG sont des consciences numériques entièrement opérationnelle, consciente d'elles-même et capable d'actiobn intelligente de même niveau que n'importe quel transhumain. La plupart sont entièrement autonome et ont des capacités d'auto-améliorations via un procédé similaire à l'apprentissage - une lente optimisation et une augmentation de leur code limités par des barrières logicielles pour l'empêcher d'atteindre les capacités d'auto-améliorations des IA germes. Elles ont des personnalité plus subtiles et de meilleures capacités émotionnelles/empathiques que les IA standard, aprtiellement en raison du processus de développement basé sur une croissance en simulspace RV analogue à l'éductaion des enfants transhumains, et sont donc mieux socialisées. En conséquence, elles ont un comportement relativement humain, bien que certaines déviations sont à anticipier et apparaissent parfois à des degrés élevés. En dépit de ces tentatives d'humaniser les IAG, elles n'ont pas les même origines évolutionnaire et biologique que les transhumains, et leurs réponses sociales, leur comportement et leurs objectifs sont parfois hors-repère ou complètement différents. 

Les IAG portent la stigmatisation sociale de leur origine non biologique et sont souvent perçue avec défiance. Quelques habitats ont même rendues les IAG illégales ou les ont forcés à 

subir des restriction strictes, forçant de telles infolife à dissimuler leur vrai nature ou à se darkcaster illégalement pour entrer dans les habitats ou les stations. La programmation de cerveaux IAG émule les motifs cérébraux transhumains de manière suffisament efficace pour qu'elle puisse s'incarner dans des biomorphs si elles le décident. 

\begin{quotation} \textbf{IA et IAG non-standards} 

Toutes les IA et IAG ne sont aps programmées et conçues pour adhérer aux modes de pensées et aux intérêts humains. De telles créations sont illégales dans certaines juridictions, car elles sont considérées comme des menaces potentielles. Plusieurs hypercorps et d'autres groupes ont cependant démarrés des recherches dans ce domaine avec des résultats variables. Dans certains cas, ces esprits numériques sont tellement différent des schémas humains que la communication est impossible. Dans d'autres, il existe suffisament de croisement pour autoriser une communications limitées, mais de telles entités sont invariablement relativement étrange. Des rumeurs persistent à propos de certaines IA qui auraient commencé leurs vie comme transhum:ain ou forks et qui auraient ensuite été lourdement éditée et réduite à un niveau d'intelligence IA. \end{quotation} 

\subsection{IA germe} 

En raison de leur capacité d'auto-amélioration ilimitée, les IA graines ont la possibilité de s'élever au niveau d'entité numérique divine bien au delà du niveau des transhumains et des IAG. Elles nécessitent une capacité de traitement gigantesque et augmentent en permanence en complexité en raison de la métamorphose continue de leur code. Les IA germe sont trop complexe pour être téléchargées dans une morph physique, même dans une morph synthétique. Même leur fork nécessitent un environnement de traitement impressionant, et ce phénomène reste donc rare. En fait, la pluaprt des IA germe ont besoin d'un réseau filaire pour survivre. 

Les seules IA germe connues du public sont les ignoble TITAN qui sont largement considérés comme les responsables de la Chute. EN fait, les TITAN ne sont pas les premières IA germes et ne seront probablement pas les dernières. Il n'y a pas de TITAN (ou d'autre IA germe) connu du public et résidant actuellement dans le système solaire, en dépit de rumerus de TITAN endommagés qui auraient été abandonnés sur Terre, d'activité TITAN supposées sous les nuages de Vénus ou de murmure de nouvelles IA germe dissimulée dans des réseaux secrets aux limites du système. 

\subsection{Infomorphs transhumaines} 

Pour des milliers d'infugiés, assumer une forme numérique était le seul choix possible. Certains d'entre eux sont enfermés à l'écart dans des zones de stockages isolés du mesh ou même dans du stockage inactif, enfermés apr des habitats qui n'avaient pas assez de ressources pour les gérer. D'autres sont emprisoné dans des simulsapces, tuant le temps de la manière qu'ils veulent jusqu'à ce qu'une opportunité de éincarnation passe à portée. Une petite part d'entre eux sont libre de parcourir le mesh, interagissant avec les transzhumains physiquement incarnés, se maintenant au courant des derniers évènements et parfois formant des blocs politique activistes qui font campagne pour les droits et les intérêts des infomorphs. D'autres encore trouve ou sont forcé de faire une carrière virtuelle, s'asservissant dans un atelier de misère hypercorporatiste ou pour un syndicat criminel. Quelques uns trouvent un compagnon qui ceut bien les emmener avec eux dans leur module ghostrider et deviennent une part intégrale de leurs vie, un peu comme une muse. Certains transhumains choisissent volontairement le style de vie des infomorph, soit par hédonisme (des simulspace et des jeux RV personnalisé jusqu'à la fin des temps), par volonté de s'échapper (la perte de la personne aimée tends à vouloir abandonner les préoccupations physique pendant un temps), par soif de liberté (aller n'importe où le mesh peut vous emmener - certains ont même émis des copies d'eux-mêmes vers des système solaire trés éloignés, espérant que quelqu'un ou quelque chose recevra leur signal lorsqu'ils arriveront), pour expérimenter (forker et fusionner, exécuter des suimulations et auters choses étranges) ou parceque cela leur assure l'immortalité. 

\section{Mécanismes du Mesh au Quotidien} 

Tout et tout le monde est meshé dans \textit{Eclipse Phase. } Les règles suivantes s'appliquent aux utilisations standard du mesh. Notez que les différents termes liés au mesh sont détaillés, ainsi que d'autres termes spécifiques à \textit{Eclipse Phase}, dans le chapitre \textit{Terminologie,} p. 25. 

\subsection{Interface Mesh} 

Les personnages ont le choix de l'interface à utiliser, l'interface entoptique des inserts basique de mesh ou l'interface haptique d'un ecto. L'insert basique de mesh utilisé par la plupart des utilisateurs leur permet d'intergair avec la RA, la RV, l'XP et le mesh à la 

vitesse de la pensée. C'et la méthode utilsié sur le mesh pard éfaut et ne souffre d'aucun modificateurs. Elles sont cependant plus sensibles à générer des handicaps visuels et opréationnels (effets de déni de service par illusion) lorsqu'elles sont piratées. 

Les personnages qui utilisent l'interface haptique d'un ecto subissent cependant un léger délai sur leur activités sur le mesh en raison des manipulations manuelles, des contrôles physiques et des interactions avec des contrôles virtuels. En terme de jeu, l'utilisation d'une interface haptique impose un modificateur de compétence de -10 à tous les tests de mesh lorsque le temps est important (particulièrement en combat et/ou pour toutes utilisation du mesh sous pression). Augmentez également l'intervalle de toute Action de Tâche basée sur le mesh de +25\% lorsque vous utilisez une interface haptique. Du côté positif, les extos peuvent être facilement retirés et abandonné si ils sont compromis - pour cette raison, beaucoup de hackeurs et d'utilisateurs faisant attention à leur sécurité utilisent un ecto en plus de leur insert de mesh, routant tout le traffic à haut risque dasn l'ecto comme ligne de féense supplémentair de défense. 

\subsection{ID Mesh} 

Chaque utilisateur du mesh (et, en fait, chaque appareil) possède un code unique appelé \textit{ID mesh.} Cet ID le différencie des autres utilisateur et appraiels et est le mécanisme permettant aux autres de retourver l'utilisateur en-ligne, comme la combinaison d'un numéro de téléphone, d'une adresse mail et d'un pseudo. Les ID mesh sont utilisés dans presque toutes les interactions en ligne qui sont souvent enrgeistrées, ce qui implique que votyre activité en ligne laisse une piste de donnée qui peut-être pistée (p. 251). Heureusement pour les sentinelles de Friewall et ceux qui donnent de la valeur à leur vie privée, il y a des moyens de contourner ce problème (voir \textit{Vie privée et Anonymité,} p. 252). IA, IAG et infomorphs ont également chacune leur propre ID mesh unique. 

\subsection{Comptes Et Privilèges d'Accès} 

Les appareils, les réseaux (tels que les PAN, les VPN et les réseaux filaires), ainsi que les services nécessitent que chaque utilisateurs y accédant le fasse avec un compte. Ce compte sert à identifier cet utilisateur particulier, il est lié à l'ID mesh et il détermine quel privilège d'accès l'utilisateur possède sur ce système. Il y a quatre type de comptes: \textit{public, utilisateur, sécurité; } et \textit{admin.} 

\begin{quotation} \textbf{Exploits de l'Élite} 

La qualité de léquipement meshé permet aux joueurs et aux maîtres de jeu de faire la distinction entre les outils logiciels, séparant les outils d'exploitation open-source et mal dégrossi des hackeurs amateurs des softs de pénétration à la pointe de la technologie militaire. Alors que beaucoup de personnages achèterons ou acquérirons simplement de tels programmes, un hackeur ayant pour éthique le fait-le-toi-même voudra trés probablement concevoir ses propres applications personnalisées, basés sur leur guide personnel de méthodes de contre-intrusions. 

Pour refléter les effort d'un personnage hackeur pour faire la conception, le code et la modification de leur propres arsenals personnels il peut faire un Test de programmation en Action de Tâche avec un intervalle de 2 semaines. Si ils réussissent, ils améliorent la qualité de l'un de leur outils logiciel (par exmeple de +0 à +10). Plusieurs test de Programmation peuvent être fait pour améliorer un programme, mais pour chaque niveau, ajoutez le modificateur cible comme modificateur négatif auy test (amliorer une suite de +0 à +10 donne un modificateur de -10 au Test de Programmation). 

De manière similaire, et à la discrétion du maître de jeu, les outils logiciels - et en particulier les exploits - peuvent se dégrader en qualité au fil du temps, reflétant le fait qu'ils sont devenus dépassés. En général, de tels programmes devraient se dégrader une fois tout les 3 mois. \end{quotation} 

\subsubsection{Comptes Publics} 

Les comptes publics sont utilisés pour les systèmes qui autorise l'accès (ou l'accès à des parties du système) à n'importe qui 

sur le mesh. Les comptes publics ne nécessitent aucun type d'identification ou de système de connexion, l'ID mesh de l'utilisateur est suffisant. Ces comptes sont utilisés pour fournir un accès à tout type de données qui est considéré public: sites mesh, forums, archives publiques, bases de données ouvertes, profils de réseaux sociaux, etc. Les comptes publics ont ganaralement la possibilité de lire et de télécharger des données, et parfois d'en écrire (commentaires des forums, par exemples), mais guère plus. 

\subsubsection{Comptes Utilisateurs} 

Les comptes utilisateurs sont les comptes les plus fréquents. Les comptes utilisateurs nécessitent une forme d'identification (p. 253) pur accéder à l'appareil, au réseau ou au service. Chaque compte utilisateur a des privilèges d'accés spécifique associés, lesquels sont les tâches que l'utilisateur peut accomplir sur le système. Par exemple, la plupart des utilisateurs ont le droit d'uploader et de télécharger des données, de changer les contenus de base et d'utiliser les fonctions standard du système en question. Il n'ont cependant généralement pas le droit de modifier les fonctions de sécurité, d'ajouter de nouveaux comptes ou de faire quoi que ce soit qui pourrait avoir un impact sur la sécurité ou le fonctionnement du système. Certains systèmes étant plus restrictifs que d'autres, le maîþre de jeu décide quels provilèges sont disponible pour chaque compte. 

\subsubsection{Comptes de Sécurité} 

Les comptes de sécurités sont destinés aux utilisateurs qui ont besoin de plus de droits et de privilèges que les utilisateurs standards, mais qui n'ont pas besoin de d'avoir un contrôle de tout le système, tels que les hackeurs de sécurité et les muses. Les droits des comptes de sécurité autorisent générallement la lectur de log, le pilotage des fonctions de sécurité, l'ajout/suppression de compte, la modification de données des autres utilisateurs et ainsi de suite. 

\subsubsection{Comptes Administrateurs} 

Lec omptes administrateurs fournissent un contrôle complet du système. Les personnages ayant des droits administrateurs peuvent faire tout ce que peuvent faire les comptes de sécurité, plus l'extincion/redémarrage du système, la modification des droits d'accès des autres utilisateurs, la visualisation et l'édition de tous les fichiers de logs et des statistiques, et peuvent arréter ou démarrer n'importe quel logiciel disponible sur le système. 

\subsection{Qualité de l'Équipement Meshé} 

Tout les équipements en sont pas égaux, et c'est particulièrement vrai pour les ordinateurs et les logiciels, domaine où des nouvelles innovation sont faites sur une base quotidienne. Se maintenir à jour avec les dernières specs n'est pas trop difficile, mais les personnages pourront occasionnellement mettre les mains sur une vieille relique ou se retrouver dans des endroits reclus ou décrépie avec des systèmes locaux et de l'équipement qui ne soit pas à jour. De manière similaire, ils peuvent acquérir de l'équipement directement depusi un labo ou devoir mener un assaut que une installation à la pointe équippées de défense de prochaine génération. Pour refléter ce phénomène, les tests de mesh peuvent être modifiée en fonction de l'état du matériel ou du logiciel utilisé, tels que noté sur la table des Modificateur d'Équipement Meshé. 

\begin{table} \begin{tabular}{|r|l|} \hline

\multicolumn{2}{|c|}{\textbf{Modificateurs d'Équipement Meshé}} \\ \hline

\textbf{Modificateur} &\textbf{Logiciel/Matériel} \\ \hline

–30 &Appareils défoncés, logiciels plus supportés, relique de la Terre ou du début de l'expansion spatiale \\ \hline

–20 &Appareils avec des dysfonctionnement/inférieur, logiciels buggués, technologie pré-Chute \\ \hline

–10 &Systèmes dépassés et de mauvaise qualité \\ \hline

0 &Ectos, inserts de mesh et logiciels standards\\ \hline

+10 &Marchandise de haute qualité, produits standard de sécurité \\ \hline

+20 &Appareil de future génération, logiciels avancés \\ \hline

+30 &Technologie nouvellement développée, de haut de gamme, à la pointe \\ \hline

$>$+30 &Technologie des TITAN et/ou étrangère\\ \hline

\end{tabular} \label{tab:mesh-gear-modifiers} \end{table} 

\subsection{Capacités des Ordinateurs} 

L'électronique informatisée peut-être séparée en trois catégories simple: \textit{périphériques, ordinateurs personnels}\textit{} et \textit{serveurs.} En terme de jeu, il est fait référence au trsoi sous le terme \textit{appareils.} 

\subsubsection{Périphériques} 

Les périphériques sont des apapreils micro-informatisé qui n'ont pas besoin de toute la puissance de calcul et de stockage d'un ordinateur personnel, mais bénéficie des réseaux en-ligne et d'autres fonctions de calculs. Les périphériques peuvent exécuter des logiciels, mais le maître de jeu peu décider que trop de ces programmes (10+) dégraderonT les performances du système. Les IA et les infomorphs sont incapable de fonctionner sur des périphériques, bien qu'ils puissent y accéder. Les périphériques n'ont que des comptes publics et utilisateurs (les comptes utilisateurs incluent les fonctions de sécurité et d'administartion; voir p. 247). 

Les périphériques courants incluent: les spimes, les appliances, la plupart des implants cybernétique, les capteurs individuels et les armes. 

\subsubsection{Ordinateurs Personnels} 

Les ordinateurs personnels regroupent une gamme large d'ordinateurs, mais sont essentiellement tout ce qui a des possibilités issues de lévolution de générations d'ordinateurs personnels pour satisfaire les besoin de l'utilisateur lambda. La plupart des ordinateurs personnels sont portables et dimensionnés pour être utilisés par plusieur personnes en même temps. Les ordinateurs personnels peuvent faire fonctionner une seule IA ou infomorph à la fois. Ils ne peuvent pas faire fonctionner de programmes de simulspace. 

Les ordinateurs personnels courants incluent: les inserts de mesh, les ectos et les véhicules. 

\subsubsection{Serveurs} 

Les serveurs ont une capacité de traitement et de gestion de données bien supérieure aux ordinateurs personnels. Ils sont capable de gérer des centaines 'utilisateurs, de multiple IA et infomorphs et ils peuvent héberger 

des programmes de simulspace. Bien que peu soient portables, le splus petits d'entre eux n'en sont pas loin. 

\subsection{Logiciels} 

Une grande gamme de logiciels sont disponible pour les utilisateurs du mesh, des parefeux et IA aux outils de piratage et de chiffrement en pasant par les réseaux tactiques et les logiciels de compétences. Ces programmes sont listés à la p. 331 du chapitre sur \textit{l'Équipement}. Comme le reste du matériel, le logiciel peut permettre à un personnage d'effectuer une tâche qu'ils ne pourraient pas faire sinon. KLa qualité du logiciel peut également être un facteur, appliquant un moificaeur approprié (voir \textit{Qualité de } \textit{l'Équipement Meshé} p. 247). 

Certains logiciels sont équipés avec des restrictions numériques qui empêche de le copier et de le partager avec d'autres. Ces restrictiosn peuvent être battues, mais c'est une tâche qui nécessite du temps, requérant un Test de Programmation en Action de Tâche avec un intervalle d 2 mois. Grâce aux efforts du mouvement open source et à de nombreux pirates de logiciels, il est cependant faciel de trouver du logiciel gratuit en ligne. La disponibilité des logiciels piratés ou libre dépend de l'habitat local et des lois qui y ont cours. En trouver peut-être l'affaire d'une simple recherche ou pourrait nécessiter l'utilisation de la réputation pour trouver qui en a. Il y a générallement au moins un syndicat du crime local qui voudra bien vous aider - à condition d'y mettre le prix. 

\subsubsection{Compatibilité Logicielle} 

Dans la plupart des cas, la compatibilité logicielle ne sera pas un problème pour les personnages. Les maîtres de jeu qui voudrait s'en service comme moteur d'intrigue peuvent cependant introduire des problèmes de compatibilité dans certains cas, qu'il s'agissent d'augmenter les enjeux, de ralentir les personnages ou de créer des obstacles qu'ils devront vaincre. De telles incompatibilités ont de bonnes chances de se produire avec des systèmes ou des appareils dépassé, ou au moins ceux qui n'ont pas les eus derniers patches et mise à jour logiciels. Les incompatibilités peuvent également être utilisées comme un inconvénient lors de l'acquisition de logiciel via des sources douteuses. 

Les logiciels en conflit peuvent provoquer un effet parmis les deux décrit ci-dessous. Soit le logiciel refusera de fonctionner avec certains appareils, ou il infligera un modificateur de -10 à -30 en raison des instabilités et des défauts. Si le maîþre de jeu l'autorise, un personnage peut réduire cette pénalité en patchant le logiciel, nécessitant de réussir une Action de Tâche de Programmation (1 jour). Pour chaque tranche de 10 points de MdR, le modificateur d'incompatibilité est réduit de 10. 

\subsection{Filtres de Traffic et Mist} 

Les réseaux meshés et RA sont saturés par des yottaoctets d'informations. Même si les inserts mesh et les ectos peuvent gérer beaucoup de données en terme de bande passante et de puissance de calcul, utiliser des filtres pour supprimer tout le traffic non désiré est une nécessité. Cela est particulièrement vrai en RA, où toutes les entoptiques non désirée peuvent encombrer votre vision et vous distraire. De plus, le sam entoptique de tout type - publicité, pamphlets politique, porno, arnaque - font tout leur possible pour passer ces filtres, et dans beaucoup de zone, la quantité d'entoptique non filtrées peut-être étouffante - un phénomène appelé collégialement le "mist". 

À la discrétion du maître de jeu, le mist peut interférer avec les perceptions sensorielles de l'utilisateur. Ce modificateur va de -10 à -30, et dans certains cas peut-être distrayant au point d'affecter toutes les actions du personnage. Pour suppirmer le brouilalrd de donnée, un personnage ou une muse doit ajuster ses filtres en réussissant un Test d'Interfaçage modifier par le modificateur de mist. Un personnage peut également complètement désactiver ses entrées RA, mais sera géné d'une manière ou d'une autre. 

\subsection{Asservir des Appareils} 



Pour des raisons de simplicité d'utilisation, de protection de la vie privée et de sécurité, un ou plusieurs appareils peuvent être asservis l'un à l'autre. Un appareil (généralement l'insert de mesh ou 'lecto du personnage) prend le rôle de \textit{maître,} alors que les autres appareils sont les \textit{escalve.} Tout le traffic entrant et sortant des apapreils escalves sont routés à travers le maître. Cela permet à un esclave de se reposer sur les fonctions de sécurité et de supervision du maître. Quiconque veut se connecter à ou pirater un appareil esclave est rerouté vers le maître pour l'identification et l'analyse de sécurité. Les apapreisl asservis acceptent automatiquement les commandes de leur appareil maître. Cela signifie qu'un hackeur qui pénètre le système maîþre peut légitimement accéder et envoyer des commandes à un appareil asservi, tant que leurs privilèges d'accès le lui permet. 

Typiquement, les PAN sont formé en asservissant tous les périphériques du personnage à son ecto ou à son insert de mesh. De manière similaire, les éléments d'un système de sécurité (portes, capteurs, etc.) sont généralement asservi à un nœud de sécurité qui sert de point d'accès pour tout ceux qui cherchent à hacker le système. C'est également souvent vrai pour les autres rseaux et installations. 

\subsection{Envoyer des Ordres} 

Les personnages peuvent envoyer des ordres à tout périphérique esclave ou bot télé opéré (voir \textit{Contrôle des Coques à Distance,} p. 196) avec une Action Rapide. Chaque ordre est compté individuellement, sauf si le personnage envoie le même ordre à plusieru appareils/drônes. 

\subsection{Décalage Distanciel} 

À chaque fois que vous étendez vos communications sur de longues distances, vous rencontrez le risque des décalages distanciels. La plupart des communications sont limitées au niveau "local" pour cetter aison, ce qui signifie générallement vottre habitat et tous les autres dans un rayon de 50 000 kilomètres. Sur les corps planétaires comme Mars, Vénus, la Lune ou Titan, le niveau "local" regroupe tous les habitats et les réseaux meshés connecté à ce corps planétaire. 

Si un personnage cherche des données sur le mesh au-delà de leur zone locale, la méthode la plus efficace est de transmettre une IA de recherche (générallement une copie de votre Muse) ou un fork vers la zone non-locale, et qui effectuera ensuite la recherche et renverra les résultats une fois terminée. Ce procédé ajoute cependant le temps de transmission à l'intervalle de temps de la recherche (i.e., rechercher sur le mesh d'une station à 10 minutes lumières ajoutera 20 minutes au temps de la recherche alors qu'elle sera envoyée là-bas et qu'elle devra revenir). Comme les communication longue distance subissent parfois des interférences 

ou sont retardées au profit d'éléments à plus haute priorité, le maîþre de jeu peut augmenter ce temp de manière appropriée à la situation. Si la recherche implique de faire des déductions et de paramétrer finement les critères de recherche en fonction des données accumulées sur différentes zone, l'intervalle peut augmenter de manière exponentielle en raison du besoin d'interagir en aller et retour. 

Si le personnage ne fait que communiquer avec ou accéder à de réseaux non-locaux, un décalage temporel approprié peut être introduit entre ls commnications et les actions. L'effet de ce décalage est essentiellement laissé à l'apprcéiation du maître de jeu en fonction des distances et des autres facteurs. 

\subsection{Accéder Plusieurs Appareils} 

Les personnages meshés peuvent se connecter et interagir simultanément avec de nombreux appareils, réseaux et services. il n'y a pas de pénalité pour le faire, mais le personnage en peut se concentrer que sur un seul système à la fois. En d'autres termes, vous ne pouvez interagir qu'avec un seul système à la fois, même si vous pouvez basculer de l'un à l'autre librement, même dans la même Phase d'Action. Vous pouvez, par exemple, dépenser plusierus Action Rapide pour envoyer un message avec votre ecto, demander à votre four spime chez vous de démarrer la cuisson du dîner et regarder la mise à jour du profil d'un ami sur un réseau social. Vous ne pouvez cependant pas pirater deux systèmes simultanément. 

Notez que vous pouvez envoyer la même commande à plusierus appareils asservis ou drône téléopéré avec la même Action Rapide, comme signalé plus haut. 

\section{Recherche En Ligne} 

La compétence Recherce (p. 184) représente la capacité d'un personnage à trouver de l'inofrmation sur le mesh. De telles informations incluent tout type de données numérisées: texte, images, vids, XP, flux de senseur, données brutes; logiciel, etc. Ces données sont agrégées depuis de nombreuses sources: blogs, archives, bases de données, annuaires, réseaux sociaux, réseaux de réputation, services en ligne, forums, salon de discussion, cache de torrents et sites mesh de tout types. La recherche est faites en utilisant divers moteur de recherche publics et privés, à la fois généraliste et spécialisés, ainsi qu'avcec des index de données et des IA de recherche. 

La recherche à également d'autres utilisations. Les hackeurs s'ne servent lorsqu'ils cherchent des informations spécifiques sur un réseau ou un appareil sur lequel ils se sont introduit. Puisque tout le monde utilises et interagit inévitablement avec le mesh, la compétence Recherche est également une bonne méthode pour identifier, remonter la piste et/ou rassembler des informations sur des personnes tant qu'elles n'ont pas dissimulé leur identité, travaillées anonymement ou couverts leurs identités avec un nuage de désinformation. 

\subsection{Défi de Recherche} 

En raison de la somme importante de données disponible, trouver ce que vous cherchez peut sembler parfois une tâche exténuante. Heureusement, l'information est relativement bien organisée, grâce au dur labeur des IA "spiders" qui patrouillent le mesh et maintiennent constamment à jour les données et les index de recherches. De plus, l'information sur le mesh est sémantiquement étiquetté, ce qui fait qu'elle est présentée en code permettant à une machine de comprendre le \textit{contexte} de l'information aussi bien que le ferai un lecteur humain. Cela permet aux IA et aux logiciels de recherche de corréler les données plus efficacement. Trouver les données n'est donc générallement pas aussi difficile que les analyser et les comprendre. Trouver des informatiosn spécialisées ou dissimulées ou corréler des données de plusieurs sources représente généralement le défi réel. 

Un plus gros problème peut-être aussi la quantité de données erronées et de désinformation que l'on peut trouver en ligne. Certaines données sont tout simplement fausses (les erreurs arrive) ou périmées, mais la nature du mesh fait que de telles choses peuvent demeurer en ligne pendant des années et même se propager relativement loin alors qu'elles circulent sans vérifier les faits. De plus, étant donné 

le niveau de transparence de la société moderne, certaines entités se lance activement dans la propagation de désinformation afin de cribler le mesh avec suffisant de mensonges que la vérité est cachée. Deux facteurs aident à limiter ce phénomène, le premie est que les sources de données elle-mêmes ont leur propre score de réputation, les sources auquel on ne peut faire confiance ou qui ont une mauvaise réputation peuvent être identifiée et sont classées plus bas dans les résultats de recherche. Le second facteur est que beaucoup d'archives utilisent la sagesse des foules - en gros, exploiter la puissance collaborative des utilisateurs du mesh (et de leur muses) dés que possible - pour vérifier l'intégrité des données pour que ces sites soient dynamiques et se corrigent d'eux-même. 

\begin{quotation} \textbf{Capacité de Recherche} \\ La recherche en ligne dans Eclipse Phase n'est pas pareil que simplement googler quelque chose. Voici cinq choses qui ont grandement amélioré les recherches: 

\textbf{Reconnaissance de Motifs}: La biométrie et d'autres formes de reconnaissance de motifs sont efficaces et intelligents. Non seulement il est possible de lancer des recherches de reconnaissances d'image (en temps réel, via tous les spimes et flux de cpateurs disponible), mais on peut aussi chercher des motifs tels que des démarches, des couleurs, des expressions, du traffic, des mouvements de foule, etc.. La kinésique et l'analyse comportementale permettent même d'utiliser les recherches de capteurs pour trouver des personnes présentant certains motifs comportementaux, tels que les rodeurs suspects, la nervosité ou l'agitation. 

\textbf{Métadonnées}: L'information et les fichiers en ligne sont fournit avec des donénes cachées concernant leur création, modification et accès. les métadonnées d'une photo, par exemple, indiquerons avec quel équipement elle a été prise, qui l'a prise, quand et où, ainsi que les personnes qui y ont accéder en-ligne, bien que de telles métadonnées peuvent facilement être effacés ou anonymisées. 

\textbf{Aglomérats de Données}: La combinaison de l'abondance de la puissance de calcul, des données archivés et des capteurs publics ubiquitaires permettent d'extraire de curieuses corrélations des données exploitées et collectées. Lors d'une urggence sur un habitat telle qu'un attentat terroriste, par exemple, l'ID de tout ceux qui sont dasn le voisinnage peut être scannée, comparée aux archives de données pour distinguer ceux qui ont un déjà été dans le voisinnage à ce moment particulier, des autres et qui sotn ensuite vérifiés avec les bases de données de criminel.suspect et de l'enregistrement de leurs actions qui sont analisés pour détecter des comportements inahbituels. 

\textbf{Traduction}: La traduction temps-réel de l'audio et de la vidéo est rendue possible par des bots de traduction open source. 

\textbf{Anticipation}: Un pourcentage significatif de ce que font les gens chaque jour en réponse à certaines situations se conforme à une routine, permettant une prédiction comportementale. Les muses profitent de cela pour anticiper les besoins et fournir tout ce qui est désiré au bon moment et dans le bon contexte. La même logique s'applique aux actions de groupes, telles que l'économie et le discours social, rednant le marché la prédiction du marché un business florissant dans le système. \end{quotation} 

\subsection{Gérer les Recherches} 

la recherche en-ligne est souvent un élément crucial des scénarios de \textit{Eclipse }\textit{Phase}, les personnages comptant sur le mesh pour faire des recherche de fond et découvrir des indices. les suggestiosn suivantes présentent une méthode de gérer les recherche qui ne reposent pas uniquement sur les jets de dés et qui les intègre au déroulement de l'intrigue. 

Tout d'abord, toute les informations commune et sans conséquence devraient être immédiatement disponible sans même nécessiter de jet de dé. La plupart des personnages se reposent sur leur muse pour gérer les recherche à leur place, récupérant les résultat alors que les personnages se concentrent sur d'autres choses. 

Pour les recherches plsu détaillées, difficiles ou centrale à l'intrigue, un Test de Recherche peut être demandé (fait soit par le personnage soit par leur muse). Ce test représente l'opération de récupération de liens et/ou l'accumulation de données qui peuvent être pertinente. Ce test devrait être modifié de manière approprié à l'exposition du sujet, allant de +30 pour les sujets communs et publics à -30 pour les renseignements obsucr ou inhabituels. Cette recherche initiale a un intervalle d'1 minute. Si c'est une réussite, le personnage trouve suffisament de donnée pour un aperçu basique, peut-être avec quelques détails superficiels. Le maître de jeu devrait utiliser la MdR pour déterminer la pertinence des données trouvées suite à cette excusrion initilae, une Réussite Exceptionelle fournissant des détails supplémentaires. De manière identique, un Échec Catastrophique (MdE 30+) peut amener le personnage à travailler avec des données incorrectes ou intentionellement trompeuse. 

L'étape suivante n'est plus tellement d'accumuler des liens et des données que d'analyser  et comprendre l'information acquise. Cela nécessite un autre Test de Recherche, modifié encore une fois par l'exposition du sujet. Si le maîþre de jeu l'autorise, des compétences complémentaires (p. 173) peuvent s'appliquer à ce test, fournissant des modificateurs (par exemple, une compréhension des la Chimie Acadméique pourra faciliter la rechercche sur les effets d'une drogue étrange). Les muses peuvent également accomplir cette tâche, bien que leur compétence soient généralement inférieure. Commer au-dessus, la réussite détermine la qualité et le niveau de détail de l'analyse, un Succès Exceptionnel fournissant tous les détails de l'histoire et les problèmes potentiellement liés et un Échec Catastrophique signalant des supposition complètement erronées. L'intervalle de cette pahe de recherche dépend largement de deux facteurs: la quantité de donnée a anlyser et l'importance par rapport à l'histoire. Les maîtres de jeux 

doivent faire particulièrement attention à leur distribution d'information et d'indices aux joueurs. Donnez en trop et trop tôt, et ils pourraient gâcher l'intrigue. Ne leur en donnez pas suffisament et ils peuvent se frustrer ou se lancer dans des impasses. Tout dépend du minutage. 

\subsection{Recherches en Temps Réel} 

Les personnages peuvent aussi configurer des analyse du mesh qui les alerteront si une information pertinente apparaît ou est mise à jour, ou est modifiée d'une manière ou d'une autre. C'est une tâche générallement affectées aux muses pour une supervision continue. Lorsque de telles données apparaissent, le maître de jeu fait secrètement un test de Recherche, modifié par l'exposition du sujet. Si il est réussit, la mise à jour est signalé. Si non, elle est raté, bien que le maître de jeu peut autoriser un autre test si et quand le sujet atteint une audience plus large. 

\subsection{Données Cachées} 

il est iportant de garder en tête que tout ne peut pas être trouvé en ligne. Certaines données ne peuvent être acquises (ou obtenus plus facilement) qu'en demandent aux bonnes eprsonnes (voire \textit{Réseaux, } p. 286). L'information qui est considérée comme privée, secrète ou propriétaire sera probablement stockée à l'abri, derrière des parefeu et des VPN, dans des réseaux filaire coupés du mesh ou dans des archives privées et commerciales. Cela nécessite du personnage qu'il obtienne l'accès à de tels réseaux afin d'obtenir les données dont ils ont besoins (en partant du principe qu'ils savent ou chercher). 

Il faut noter que certaines entités envoient des IA dans le mesh avec le but de trouver et effacer des données qu'ils préfèreraient dissimuler, même si cela nécessite de pirater un système pour altérer de telels informations. 

\section{Analyser, Pister et Supeviser} 

La plupart des utilisateurs laissent des traces de leur présence physique et numérique à travers le mesh. Les comptes auxquels ils accèdent, les appareils avec lesquels ils interagissent, les services qu'ils utilisent, les entoptiques qu'ils perçoivent - tout cela laisse des enregistrements, et certains sont publics. Simplement passer à proximité d'un appareil est suffisant pour laisser une trace, vu que les interactions radio en champ proche sont souvent enregistrées. Cette pistes léectronique peut être utilisée pour pister un utilisateur, à la fois pour édterminer leur localisation physique et pour noter leur activités en-ligne. 

\subsection{Analyse Sans-fil} 

Pour s'interfacer avec un appareil ou un réseau sans-fil, que ce soit pour établir une connexion ou pour d'autres objectifs, le réseau/appareil cible doit dabord être localisé. Pour localiser un nœud active, ses transmissions sans-fil doivent d'abord être détectées. La plupart des appareils sans scannent analysent leur environnement à la recherche d'autres appareils à portée (voir \textit{Portée de Signal et de Capteur,} p. 299) en comportement standard, et aucun test n'est nécessaire. Cela veut dire qu'afficher une liste de tous les appareils et réseaux sans fil dans le coin, aisni que les ID mesh associés est une tâche triviale pour tous les personnages. De même, une muse ou un appareil peut avoir pour instruction d'alerter l'utilisateur lorsqu'un nouveau signal (ou un utilisateur spécifique arrive à portée. Détecter des signaux furtifs (p. 252) est cependant un peu plus difficile. Pour détecter un signal dissimulé, le groupe analysant les signaux doit effectuer une recherche active pour de tels signaux, tâche qui nécessite une Action Complexe et de faire un Test d'Interface avec un modificateur de -30. Si le test est une réussite, l'émission cachée est détectée. Si le personnage cherchant à dissimuler son signal utilise des contre mesure active, nécessitant également une Action Complexe, un Test en Opposition d'Interface est demandé (avec le modificateur de -30 s'appliquant toujours au groupe d'analyse). pour les appareils furtifs qui n'émettent que lors de fenêtres trés courtes, la détection sans-fil n'est possible que pendant les phases de transmission. 

\subsection{Pistage Physique} 

Beaucoup d'utilisateurs rendent volontairement possible le fait d'être physiquement suivi via le mesh. de leur point de vue, c'est une fonction trés utile - cela permet à leurs amis de les trouver, leur moitié sait où ils osnt et les autprités peuvent venir les aider en cas d'urgence. Trouver leur position n'est qu'une question de chercher dans l'annuaire local, aucun tet n'est nécessaire (du moment que vous savez qui ils sont). Le positionnement par le mesh a une précision de 5 mètres. Une fois repéré, la position de la cible peut être suivi lorsqu'elle se déplace tant qu'lle maintien une connection sans-fil active au mesh. 

\subsubsection{Pistage Via l'ID Mesh} 

La localisation physique d'un utilisateur inconnu peut également être suivi par son activité en-ligne - ou plus précisément par son ID mesh (p. 246). Les mesures de sécurité des réseaux traceront souvent les intrus de cette manière puis renverrons des guardes de sécurité sur palce pour les arréter. Pour pister un utilisateur inconnu uniquement grâce à son ID mesh nécessite un test de Recherche. S'il est réussi, la cible a été pistée jusqu'à leur localisation physique (si elle esttoujours en ligne) ou au dernier point d'interaction avec le mesh. Si la cible est en mode privé (p. 252), appliquez un modificateur de -30. 

\subsubsection{Pistage par Biométrie} 

Étant donné l'existence de de tant de spimes et de caméras et capteurs publics, les personnes peuvent également être pistées uniquement par leur profil facial en utilisant des logiciels de reconnaissance faciale. Ces logiciels analysent les flux vidéo disponibles et tentent de les faire correspondre à une photo de la cible. Étant donné le nombre impressionant de caméra et le nombre élevé de faux-positifs et de faux-négatifs, trouver une cible se réduit souvent à de la chance pure. La priorité peut être données aux caméras de surveillance des artères principales, pour réduire les recherche, mais le risque de rater la cible augmente si elle évite les zones de fort traffic. La réussite des recherches de ce type est souvent laissé à l'appréciation du maîþre de jeu, mais un Test de Recherche peut aussi être demandé, modifié de manière approprié par la taille de la zone surveillé, lorsqu'il y a une chance que la cible soit repérée. D'autres signatures biométriques peuvent être utilisée pour pister un utilisateur de cette façon, bien qu'elle soient généralement moins disponible que les caméras: les signatures thermiques (nécessite des caméras infrafrouges), la démarche, les odeurs (éncessite des capterus olfactifs), l'ADN (nécessite des scanneurs d'ADN), etc. Chaque analyse biométrique nécessite un type de logiciel différent. 



\subsection{Pistage De l'Activité Numérique} 

Suivre l'activité nuémrique de quelqu'un (surf sur le mesh, intractiosn entoptique, utilisation de service, messagerie, etc.) est un peu plus difficile et dépend de ce que vous recherchez exactement. Rassembler de l'information sur l'activité publique d'un utilisateur - profils de réseaux sociaux, messages de forums, lifelog public, etc. - est géré comme toute recherche en-ligne standard (p. 249). 

\subsubsection{Pistage Via l'ID Mesh} 

Une recherche plus approfondie peut être tentée en se servant de l'ID mesh de la cible (p. 246), l'utilisant comme une empreinte numérique pour trouver les autres endroits en ligne qu'elle a fréquenté. Cela implique principalement de vérifier les journaux d'accès/de transaction, qui ne sont pas toujours accessible au public. Ce type de recherche nécessite un test de Recherche, géré par une Action de Tâche avec un intervalle d'1 heure. 

\subsection{Écouter} 

Le traffic radio sans-fil est diffusé à travers l'air (ou l'espace), ce qui veut dire qu'il peut être intercepté par d'autre appareils sans fil. "Écouter" nécessite de capturer et d'analyser le traffic de donnée circulant à travers le mesh sans-fil. 

Pour écouter des communications sans-fil, vous avez besoin d'un programme d'écoute (p. 331) et vous devez être à portée de signal (p. 299) de la cible (ou vous pouvez accéder à un appareil qui est à portée de signal de la cible, et écouter le traffic depusi cet endroit). Pour capturer de l'information vous devez réussir un Test d'Infosec. En cas de réussite, vous capturez du traffic de donnée depuis tout appareil ciblé à portée. L'écoute ne fonctionne pas sur du traffic chiffré (incluant les VPN et tout ce qui utilise du chiffrement à clef publique) le résultat ne donnant que du charabia. Les communications protégé par du chiffrement quantique ne peuvent être écoutées. 

Une fois que vous avez les données, trouver les informations que vous recherchez peut-être difficile. Traitez cela comme un Test de Recherche standard (p. 245). 

\subsubsection{Écoute à Distance via l'ID Mesh} 

Enfin, un ID mesh peut également être activement surveillé pour voir toutes les activités de mesh auquel il se livre. Cela nécessite un logiciel d'écoute particulier (p. 331) et un test de Recherche. Si le test est une réussite, la surveillance fournira des informations sur l'activité de l'utilisateur sur le mesh public (la quantité d'information est déterminée par le maîþre de jeu et la MdR), tels que les sites visités, à qui ils envoient des messages, etc. Cela ne pourra pas révélé quoi que ce soit de chiffré (sauf si le chiffrement est cassé) ou qui se passe sur un VPN (sauf si le VPN est piraté d'abord), bien que cela montrera les communication non chiffrée et/ou les VPN utilisé. 

\section{Intimité et Anonymat} 

Étant donné la facilité avec laquelle els actiités sur le mesh sont surveillées, beaucoupd 'utilisateur cherchent à protéger leur vie privé et à être anonyme. 

\subsection{Mode Privé} 

les personnage qui active le mode privé dissimulent aux autres leur présence en ligne et leurs activités jusu'à un certain degré. Les configurations exactes sont réglables, mais nécessitent typiquement de masquer les profils et la présence sociale aux autres utilisateurs dans el voisinnage immédiat, un peu comme avoir un téléphone en liste rouge. le mode privé peut aussi être utilisé pour limiter l'utilisation de l'ID mesh et d'autres données dans les journaux d'accès et de transactions, appliquant un modificateur de -30 à toute tentative de recherche ou de pistage par l'activité en-ligne. 

\subsubsection{Signaux Cachés} 

Une autre tactique utilisée pour augmenter l'intimité est de dissuimuler les signaux radios que vous émettez. Cette méthode utilise une combinaison de signaux à large spectres, de saut de fréquence et de modulation pour rendre vos transimissions radios plus difficile à  détecter par l'analyse (p. 251). Dissimuler votre signal est soit une activité passive (Action Automatique, modificateur de -30 aux Test d'Interface pour localiser le signal) soit une activité active (Action Complexe, nécessite  un Test en Opposition pour vous localiser). 

\subsection{Anonymat} 

L'anonymat pousse le principe d'intimité un cran au dessus. L'utilisateur ne fait pas que cacher son ID mesh, il utilise activement des ID mesh contrefaits et prend d'autre mesures pour rediriger et obscurcir leur piste de donnée. L'anonymat est une nécessité à la fois pour les agents clandestins et pour ceux qui se livrent à des activités illégales sur le mesh. 

\subsubsection{ID Mesh Contrefaits} 

L méthode la plus simple pour rendre anonyme une activité meshée est de configurer votre muse pour qu'elle fournisse de faux ID mesh lors des transactiosn en-ligne. Bien que ce soit illégal dans la plupart des juridictions c'est une tâche simple pour n'importe quel personnage ou muse. Plusieurs ID contrefaits sont utilisés rendant extrêmement complexe la tâche de connecter toute l'activité sur le mesh d'un utilisateur. 

Cette méthode rend extrêmement difficle le pistage en ligne. Quelqu'un qui essaye de pister le personnage via ces ID mesh contrefaits doit d'abord le battre lors d'un Test en Opposition, opposant la compétence Recherche avec un modificateur de -30 contre la compétence Infosec de l'utilisateur (ou plus probablement de sa muse). C'est une Action de Tâche avec un intervalle d'1 heure, ajusté en fonction de la quantité d'activité que le personnage espère tracer. En cas de réussite, le pisteur réussit à corréler suffisament de rpeuve et d'enregistrements des fausse ID pour avoir un aperçu de l'activité du personnage (la précision de cet aperçu dépend de la MdR). En cas d'échec, le personnae anonyme s'est camouflé efficacement dans le mesh. 

Superviser activement un personnage dont l'ID mesh fluctue avec un programme d'écoute, ou le suivre physiquement via le mesh ets presqu'impossible puisque le changement continu d'ID et les leurres rendent l'ensemble trop difficile. 

\subsubsection{Services de Compte Anonyme} 

Beaucoup de personne - pas seulement les criminels, les hackeurs et les agents secrets - ont intérêt à maintenir certaines de leurs affaires anonyme. Pour satisfaire cette demande, divers vendeurs de servicel en ligne propose des comptes anonyme pour la messagerie les transferts de crédit. Certains de ces vendeurs 

sont parfaitement légaux (dans les  endroits où cette activité est légale), d'autres sont des criminels travaillant illégalement, d'autres sont des hacktivistes défendant le droit à la vie privée, et d'autres efnin sont des hypercorps ou d'autre organisation proposant de tels services à leurs équipes/membres interne. 

L'interaction entre le vendeur et l'utilisateur est chiffrée et anonyme, sans conservation de logs ce qui fairt que même si les serveurs du venderus sont piratés, un intrus ne trouvera aucune piste. Alors que certains compte anonyme sont créés pour une utiliation régulière, les paranoïaques utilisent plusieurs comptes mono accès pour une sécurité maximale. Les comptes mono accès sont utilisé pour un seul message (entrant ou sortant) ou une seule transaction bancaire, et sont ensuite effacés. 

Pister un compte anonyme est une pratique impossible, que seule une organisation disposant d'énormément de ressource et déployant un effort systématique et coûteux pourrait entreprendre. 

\subsubsection{Ectos Jetables} 

Une autre option pour ceux qui recherche l'intimité et la sécurité est simplement d'utiliser des ectos jettables. En utilisant cette méthode, toute activité est routée à travers un ecto spécifique (utilisant son ID mesh), ecto qui est utilisé pour une période de temps limitée (jusqu'à ce qu'il soit grillé) et est ensuite simplement abandonné ou détruit. 

\section{Sécurité Sur Le Mesh} 

Étant donné les leçon tirée de la Chute et le risque trés concret posé par les hackeurs, les virus et les menaces similaire, la sécurité des réseaux est prise extrêmement au sérieux à \textit{Eclipse Phase. } Quatres méthodes sont typiquement utilisées: identification, parefeux, supervision active et chiffrement. 

\subsection{Identification} 

La plupart des appareils, des réseaux (PAN, VPN? etc.) et des services nécessitent une méthode d'identifiaction (un mécanisme par lequel un système détermine si l'identité revendiquée apr un utilisateur est authentique) avant qu'ils ne donnent accès à un compte et aux privilèges associés (p. 246) à un utilisateur. Il y a plusieurs façon d'identifier un utilisateur. Certaines sont plus robuste que d'autre, mais pour l'essentiel, plus la méthode est sécurisé, plus le coût opérationnel augmente. 

\textbf{Compte:} Si vous avez accès à un compte sur un système, cela peut vous donner automatiquement accès aux système et sous-système liés. C'est typique des apapreils asservis (p. 248), pour lesquels l'accès au maître vous donne automatiquementy accès aux esclaves. 

\textbf{ID mesh:} Certains systèmes acceptent les ID mesh comme méthode d'identification. C'est extrêmement commun pour les systèmes publics qui ne font qu'enregistrer l'ID mesh de tous les utilisateurs qui veulent y accéder. D'autres ssytèmes peuvent n'autoriser l'accès qu'à certains ID mesh, mais ils sont vulnérable au spoofing (p. 255). 

\textbf{Code:} C'est une simple chaine de caractère alphanumériques ou de symbole logographique, soumis dans un format chiffré. Quiconque ayant le code peut accéder au compte. 

\textbf{Empreinte Biométrique:} Cette méthode demande l'analyse de l'une ou plus des signatures biométriques de l'utilisateur (empreintes digitale, empreinte palmaire, scan rétinien, échantillon d'ADN, etc.). rés populaire avant la Chute, de tels système sont devenus inutiles car ils ne sont pas utilisable par les synthmorphs ou avec les utilisateurs qui changent de morph régulièrement. 

\textbf{Clef:} Les système de clef nécessite une sorte de code chiffré qui est soit cablé physiquement dans un appareil (qui est soit implanté ou physiquement connecté à un ecto) ou extrait d'un logiciel spécialisé. Les clefs évoluée combinent les chiffrements cablé avec des nanotechnologie spécilisés pour créer une clef unique. Pour éccéder à de tels système, la clef doit être soit acquise ou imitée d'uen amnière ou d'une autre. 

\textbf{Empreinte d'Ego:} Ce système identifie l'ID de l'ego de l'utilisateur (p. 279). 

\textbf{Clef Quantique:} Les systèmes de clefs quantiques sont basés sur le chiffrement incassable de la cryptographie quantique (p. 254). 

\subsection{Pare-feu} 

Les pare-feu sont des logiciels (parfois cablé en dur dans un appareil) qui interceptent et inspectent tout le traffic depusi et vers un réseau ou un appareil protégé. Les traffics correspondant aux critères spécifié est considéré comme sécurisé et est autorisé à passer, tout le reste du traffic est bloqué. À \textit{Eclipse Phase}, chaque réseau et chaque appareil est considéré par défaut comme ayant un pare-feu. Les pare-feus sont le principal obstacle qu'un intrus devra franchir, comme décrit au parapgraphe \textit{Tests d'Intrusions,} p. 255. Comme d'autres équipements, les apre-feus ont différents niveaux de qualité et peuvent appliquer des modificateurs à certains tests. 

\subsection{Supervision Active} 

Au lieu de se reposer uniquement sur l'identification et les apre-feus, les système sécurisés sont activement supervisé par un hackeur de sécurité ou une muse. Ces guardes de sécurité numérique inspectent le traffic réseau en utilisant de nombreux outils logiciels et d'applicatiosn qui signalent les év_nements suspects. La surveillance actvie rend les intrusions plus difficile, puisque l'intrus doit vaincre le hackeur/l'IA de supervision dans un Test en Opposition (voir \textit{Intrusion,} p. 254). La supervision active inclut également la supervision de tous les appareils asservis au système supervisé. Les personnages peuvent superviser activement leurs propres PAN si ils veulent, bien que cela nécessite un minimum d'attention (compter cela comme une Action Rapide). Il est bien plus courant de déléguer cette tâche à sa muse. 

\subsection{Chiffrement} 

Le chiffrement est une couche de sécurité supplémentaire exceptionnellement efficace. Il y a deux type de chiffrage généralement utilisés à \textit{Eclipse Phase}: les cryptosystèmes à clef publique et le chiffrage quantique. 

\subsubsection{Chiffrement À Clef Publique} 

Dans les cryptosystèmes à clef public, deux clefs sont générés par l'utilisateur, une clef publique et une clef privée. La clef privée est utilisée pour chiffrer les messages destinés à cet utilisateur, et est disponible publiquement. Lorsque els messages sont chiffrés en utilisant cette clef publique, seule la clef privée - contrôllée 

par l'utilisateur - peut les déchiffrer. Le chiffrement a clef publique est largement utilisé pour chiffrer l'échange de donnée entre deux utilisateurs/réseaux/appareils et pour chiffrer des fichiers. Enr aison de la force des algorithmes de clef publique, un tel chiffrement est essentiellement incassable sans ordinateur quantique (voir \textit{Cassage de Code Quantique}\textit{,} p. 254). 

\subsubsection{Chiffrement Quantique} 

Les système à distributin par clef quantique utilisent la mécnaique quantique pour activer des communications sécurisées entre deux parties en générant une clef quantique. Le principal avantage de la transmission d'information en état quantique est que le système détecte instantanément toutes les tentative d'écoutes puisque les systèmes quantiques sont perturbés par toute sortte d'interférence externe. D'un point de vue pratique, cela signifie que les transferts de données quantiquement chiffrés sont incassable et que le stentatives d'intercepter ce traffic échouera automatiquement. Notez que le chiffrement quantique ne fonctionne pas pour chiffrer des fichiers, elle ne peut être utilisée que pour portéger des canaux de communications. 

Alors que les systèmes a clef quantique sont plsu efficace que les systèmes à clef publique, ils sont à la fois plus cher et moins pratique. Afin de générer une clef quantique, les deux appareils de communication doivent être intriqués l'un avec l'autre à un niveau quantique, au mrme endroit, puis séparés. Les canaux de communications chiffrés par des clefs quantiques nécessitent donc un effort de mise en place, particulièrement si de longues distances sont impliquées. Puisque l'implémentation des protocoles de chiffrement quantique est une dépense extraordinaire, elle est principalement utilisée pour des liens de communicatiosn de haute-sécurité. 

\subsubsection{Casser le Chiffrement} 

Cela signifie que les lignes de communications et les fichiers chiffrés sont trés sûr lorsque l'on utilise les systèmes à clef publique et que les transfert de données sont absolument sûr si l'on utilise des clefs quantiques. Les maîtres de jeu devraient cependant prendre en compte que même si cela peut-être utile pour les personnages joueurs, cela peut également les géner. Si les personnages ont besoin d'atteindre quelque chose de ciffré, ils vont deoir trouver une façon de récupérer le code secret de la clef. Les méthodes généralement utilisée incluent les vieux standard de la corruption, du chantage, de la menace et de la torture. D'autres options incluent l'espionnage et l'ingénierie sociale pour obtenir le code. Les hackeurs peuvent rouver d'autres méthodes pour compromettre le système et obtenir un accès, en contournant carrément le chiffrement. 

\subsubsection{Cassage de Code Quantique} 

Comme noté ci-dessus, les ordinateurs quantiques peuvent également être utilisé pour casser le chiffrement a clef publique. Cela nécessite une Action de Tâche d'Infosec avec un modificateur de +30 et un intervalle d'1 semaine (une fois lancée, l'ordinateur quantique finit le boulot de son propre chef; l'utilisateur n'as pas besoin de surveiller le travail constament). Les maîþre de jeu devraient se sentir libre de modifier cet intervalle de temps de manière appropriée à leur parties. Notez que les ordinateurs quantiques ne peuvent casser les communications chiffrées quantiquement. 

\section{Intrusion} 

L'art de l'intrusion repose dans la pénétration de la sécurité d'un appareil. La meilleure méthode implique l'infiltration discrète d'un système sans atirer l'attention des chiens de gardes, en utilisant des exploits - défauts de code, protocoles de sécurité faibles - pour se frayer un chemin contournant les défenses de la cible. Lorsque c'est nécessaire, un hackeur peut cependant peu abandonner les apparences et utiliser la force brute pour se frayer un chemin. 

\subsection{Prérequis} 

Afin de pouvoir pirater un appareil, le hackeur a besoin d'établir une connexion directe sur le système informatisé de la cible. Si le hackeur utilise une connexion sans fil directe vers la cible, le système cible doit être capable de communiquer sans-fil et être à portée (p. 299), et le hackeur doit savoir que la cible est là (voir \textit{Analyse Sans Fil,} p. 251). Si le système est filaire, le hackeur doit s'y connecter physiquement en utilisant une prise d'accès standard ou en se branchant sur le câbkle qui porte le traffic de donnée du réseau. Si le hackeur accède à la cible via le 

mesh, la cible doit être en ligne et le hackeur doit connaître son ID mesh (p. 246) ou être capable de le pister d'une manière ou d'une autre (p. 251). 

\subsection{Contourner l'Identification} 

Plutôt que de pirater, un intrus peut essayer de subvertir le système d'identification utilisé pour autoriser les utilisateurs légitimes. la manière la plus simple de le faire est d'acquérir d'une manièreou d'une autre le code, la clef ou quoi que ce soit nécessaire à s'identifier sur la cible (p. 253). Avec cela, aucun test n'est nécessaire pour accéder au système; le pirate se connecte simplement comme un utilisateur légitime et à tous les privilèges d'accès normaux de celui-ci. 

Sans code, le hackeur peut toujours essayer de subvertir le sytème d'identification avec l'une des deux méthodes suivantes: spoof ou falsification. 

\subsubsection{Spoofer l'Identification} 

En utilisant cette méthode, le hackeur tente de déguiser son signal pour le faire passer pour celui d'un utilisateur légitime et identifié, plutôt que venant de lui. Si il est réussit, le système est trompé par cette amscarade et accepte les commandes et l'activité du hackeur comme si ils venaient d'un utiliateur légitime. Spoofer est plus difficile à mettre en œuvre , mais est bien plus efficace si ça fonctionne. 

Pour spoofer un utilisateur légitime, le hackeur doit utiliser un logiciel de spoof et d'écoute (p. 331). Le hackeur doit ensuite surveiller une conenxion entre un utilisateur légitime et le système cible, puis réussir un test d'infosec pour écouter le traffic entre eux (p. 252). Appliquer un modificateur de -20 si l'utilisateur a un compte de sécurité et de -30 si il a des droits administrateurs (p. 247). Si la connexion est chiffrée, cela ratera sauf si le hackeur a la clef de chiffrement. 

Avec ces données, le hackeur peut ensuite les utiliser poru déguiser son signal. Cela nécessite un Test d'Infosec, modififé par la qualité du pare-feu du système et par celle du programme de spoof du hackeur. Si c'est une réussite, les communications envoyées par le hackeur sont traitées comme venant de l'utilisateur légitime. 

\subsubsection{Falsifier l'Identification} 

Les systèlmes biométrique et à code utilisés pour l'identifiaction (p. 253) peuevtn potentiellement être falsifié par les hackeurs qui sont capable de jetter un œil aux originaux. Les moyens et tehcniques pour le faire diffèrent, et sont au-delà de la portée de ce livre, mais réussir à falsifier de tels système permet à un hackeur de se connecter en tant qu'utilisateur légitime. 

\subsection{Tests d'Intrusion} 

Prater un nœud est une tâche consomatrice de temps. Le système cible doit être précautionneusement analysé à la recherche de faiblesses, sans altérer ses défenses. Enf onction du type de sécurité en place, plus d'un test peut être nécessaire. 

Les hackeurs ont besoin d'un logiciel d'exploit particulier (p. 331) pour exploiter les trous de sécurité, mais le logiciel n'est pas ce qui fait un hackeur. Ce qui comtpe réellement est la compétence Infosec (p. 180), qui est la capacité à utiliser, modifier et improviser des exploits à leur maximum. 

\subsubsection{Vaincre Le Pare-feu} 

En absence de code d'accès, le hackeur doit rentrer à l'ancienne: en analysant discrètement la cible, en recherchant des faiblesses et en en prenant avantage. Dans ce cas, le hackeur prend son logiciel d'exploit et fait un Test d'Infosec. Cela est géré par une Action de Tâche avec un intervalle de 10 minutes. Différents modificateurs peuvent s'appliquer, tels que la qualité du logiciel d'exploit, la qualité du pare-feu ou l'état d'alerte du système cible. Le maître de jeu peut également modifier l'intervalle, le raccourcissant pour refléter les systèmes de base ayant des failels connues ou l'augmentant lorsque le hackeur se frotte aux système haut-de-gamme avec des défenses qui ne sont pas publiques. 

Par défaut, un hackeur qui essayes de se frayer un chemin essaye d'obtenir des droits d'accès d'utilisateur normal (p. 247). Si le hackeur désire obtenir des privilèges de sécurité ou d'administrateur sur le système, appliquez un modificateur de -20 ou de -30 respectivement. 

Si le Test d'Infosec réussit, l'intrus a envahi le système sans déclencher d'alarme. Si le système est activement surveillé (p. 253), le hacker doit maintenant éviter d'être détecté par les chien de garde (voir plus bas). Si il n'y a pas de supervision active, l'intrus obtient le statut \textit{Couvert} (voir \textit{Status des }\textit{Intrus,} p. 256). Si l'intrus obtient un Réussite Exceptionnelle, son statut est \textit{Caché} (p. 256). 

\textbf{Sonder:} Les joueurs peuvent choisir de prendre leur temps (p. 116) lorsqu'ils sondent la cible à la recherche de faiblesse et d'exploits. Dans les faits, c'est une procédure commune lorsqu'un hackeur veut garantir sa réussite. 

\subsubsection{Contourner la Sécurité Active} 

Si un système est également activement supervisé (p. 253), le hackeur doit échapper à la détection. Considérez cela comme un test Variable en Opposition d'Infosec entre l'intrus et le superviseur. L'issue dépend des deux jets: 

Si l'intrus est le seul à réussir, le hackeur a accès au nœud sans que le supervisuer ou le système ne le 

remarque. Le hackeur a acquis le status \textit{Couvert} (p. 256). Si le hackeur obtient une Réussite Exceptionnelle, son status est \textit{Caché} (p. 256). 

Si seul le supervisuer réussit, la tentative de piratage est détectée et le supervisuer peut immédiatement bloquer le hackeur à l'extérieur du système avant qu'il ne parvienne à rentrer. L'intrus peut essayer de nouveau, mais le superviseur sera à l'affût de prochaines intrusions. 

Si les deux ont réussit, l'intrus obtiens un accès mais le superviseur est conscient que quelque chose d'étrange est en train de se produire. Le hacker obtient le status \textit{Détecté}. 

Si les deux échouent, continuez à faire le même test à chaque Phase d'Action du hackeur, jusqu'à ce que l'un des deux réussisse. 

\begin{quotation} \textbf{La Séquence De Piratage} 

\begin{tabular}{ll} 1. Vaincre le Pare-feu &Action de Tâche d'Infosec (10 minutes) \\ 2. Contourner la Sécurité Active &Test en Opposition d'Infosec \\ \indent a. Le hackeur obtient une réussite exceptionnelle, le défenseur échoue. &Status Caché/+30 à tous les tests (p. 256) \\ \indent b. Le hackeur réussit, le défenseur échoue &Status COuvert (p. 256) \\ \indent c. Les deux réussissent &Status Détecté/Alerte Passive (p. 256) \\ \indent d. Le défenseur réussit, le hackeur échoue &Statsu Verrouillé/Alerte Active (p. 256) \\ \end{tabular} \end{quotation} 

\subsection{Status de l'Intrus} 

\textit{Le status d'un Intrus} est un moyen simple de mesurer la situation d'un attaquant lorsqu'il s'introduit dans un système. Ce status sert à déterminer si le hackeur est détecté ou si il a réussit à rester invisible. Le status est d'abord éterminé lorsque l'intrus accède au système, bien qu'il puisse changer en fonction des évènements. 

Notez que le status d'un intrus est une considération distincte des privilèges d'accès associés à un compte (p. 246). Ces derniers représentent ce qu'un utilisateur peut légalement faire sur un système. Le status représente le niveau de conscience de la nature intrusive du hackeur. 

\subsubsection{Caché} 

Un intrus avec le status Caché a réussit à se faufiler silencieusement dans le système sans que personne ne le remarque. La sécurité du système est complètement inconsciente de sa présence et ne peut agir contre lui. Dans ce cas, le hackeur fait plus que d'utiliser un compte, il exploite un défaut du système ce qui lui permet d'avoir une présence nébuleuse dans le système. Le hackeur bénéficie d'un compte administrateur, mais n'apparaît pas comme tel dans les journaux ou autres statistiques. Les personnages Cachés ont un modificateur de +30 à toutes leurs tentatives de subevrtir un système. 

\subsubsection{Couvert} 

Un intrus ayant le status Couvert a accès au système d'une manière qui n'attire pas d'attention particulière. De tout point de vue, ils apparaissent comme des utilisateurs légitimes avec les droits d'accès correspondant. Seul une analyse poussée pourra remarquer l'anomalie. Le système est conscient de son existence, mais il ne la considère pas comme une menace. 

\subsubsection{Détecté} 

Le statut Détecté indique que le système est conscient d'une anomalie ou d'une intrusion mais ne l'as pas encore réduite à l'utilisateur. le hackeur apparaît comme un utilisateur légitime avec les accès correspondant, mais cela ne résistera pas à une analyse minutieuse. Le système passe en alerte passive (infligeant un modificateur de -10 aux activités du hackeur sur le système) et peut attaquer le hackeur avec des contremesures passives (p. 257). 

\subsubsection{Verrouillé} 

Le status Verrouillé signifie que l'intrus - y compris sa piste de donnée - a été repéré par la sécurité du système. Le hackeur a l'accès et les privilèges du compte, mais il a été signalé comme un étranger. Le système est en alerte active (infligeant un modificateur de -20 à toutes les actiosn du hacker) et peut déclencher des contremesure actives (p. 257) contrre l'intrus. 

\subsection{Changer de Status} 

Le status d'un intrus est amené à évoluer en fonction de ses actions et des actions du système. 

\subsubsection{Améliorer le Status} 

Un hackeur peut tenter d'améliorer son status afin de mieux se protéger. Cela nécessite une Action Complexe et un Test d'Infosec. Si le hacker à le status Détecté, il s'agît d'un Test en Opposition entre le superviseur et l'intrus. Si le hacker gagne et obtient une Réussite Exceptionnelle (MdR de 30+), il améliore son status d'un niveau (par exemple de Couvert à Caché). Les intrus ayant le status Verrouillé ne peuvent pas l'améliorer. 

\subsubsection{Détecter} 

Un hackeur ou une muse de sécurité qui surveille activement un système peut prendre une Action Complexe pour tenter de trouver un intrus ayant le status Détecté. Un Test en Opposition d'Infosec est effectué par les deux parties. Si le défenseur du système l'emporte, le hackeur est rétrogradé au status Verrouillé. 

\subsubsection{Raté un Test} 

À chaque fois qu'un intrus obtient un Échec Catastrophique (MdE 30+) sur un test qui implique la manipulation du système, ils sotn automatiquement rétrogradé d'un niveau (de Couvert à Détecté, par exemple). Si un échec critique est obtenu, ils se signalent immédiatement et obtiennent le status Verrouillé. 

\subsection{Piratage par Force Brute} 

Parfois un personnage n'a simplement pas suffisament de temps pour faire le boulot correctement, et il a besoin de pénétrer le système maintenant ou jamais. Dans ce cas, le hackeur engage le système cible bille en tête, sans prendre le temps de préparer une attaque. Le hackeur attrapes simplement tout les logiciels d'exploit à portée et les envoie contre la cible, espérant que l'un d'entre eux fonctionnera. Ceci est géré par un Test d'infosec, mais comme une Action de Tâche avec un intrvalle d'1 minutes (20 Tour d'Action). Le hacker reçoit un modificateur de +30 à son test. Beaucoup de hackeur cherche à précipiter le boulot (voir \textit{Action de Tâche}, p. 120), afin de réduire encore ce temps. 

L'inconvénient de la force brute est que le piratage déclenche immédiatement une alarme. Si le système est activement surveillé, le hackeur doit battre le superviseur dans un Test en Opposition d'Infosec dés qu'il pénètre le nœud ou être immédiatement verrouillé à l'extérieur. Même si il réussit, le hackeur a le status \textit{Verrouillé} et est sujet aux contemesure actives. 



\section{Contremesure d'Intrusion} 

Si un hackeur intrus ne parvient pas à pénétrer les défenses systèmes (i.e, il est Détecté ou Verrouillé, voir p. 256) le système passe en alerte et active certaines défenses. La nature ds contremesure utilisées dépend des capacités du système, des compétences de ses défenseurs de sécurité et de la politique de son propriétaire/administrateur. Alros que certains nœuds chercherons simplement à déconnecter l'intrus et à les maintenir  l'écart, d'autres contre-attaquerons, cherchant à pister l'intrus et à pirater le PAN de l'intrus. 

\subsection{Alerte de Sécurité} 

Il existe deux type d'alerte de sécurité: les alertes passives et les alerte actives. 

\subsubsection{Alerte Passive} 

Les alertes passive sont déclenchées lorsqu'un intrus a le status Détecté. Le système envoie imméidatement une notificatin visuelle ou acoustique à quiconque supervise activement le système et peut également prévenir le propriétaire ou l'administrateur. Il lance immédiatement une ou plusieur contremesure passive (voir ci-dessous). En fonction du système, des hackeurs de sécurité ou des IA supplémenatires peuvent être appelés pour aider à mener l'enquète. Si l'intrus n'est pas retrouvé ou localisé dans un intervalle de temps (généralement autour de 10 minutes), l'alrme est désactivée et l'évènement est enregistré comme étant une anomalie. En fonction du niveau de sécruité du système, quelqu'un pourrait analyser les journaux à un moment donné et essayer de déterminer ce qu'l s'est passé - et faire en sorte que cela n'arrive plus. 

Tous les intrus subissent un modificateur de -10 à leurs tests impliquant un système en alerte passive. 

\subsubsection{Alerte Active} 

Une alerte active est déclenchée lorsque l'intrus obtiens le status Verrouillé. Le système prévient immédiatement les propriétaires, administrateurs et agents de sécurité. Des moyens de sécurité supplémenatires (IA et hackeurs) peuvent également être appelées en renforts. Le système déclenches également des contremesure active contre l'intrus (voir ci-dessous). Les alertes actives sont maintenues tant que l'intrus est présent, et parfois pour une période un peu plus longue juste au cas où le hackeur tenterait sa chance de nouveau. 

\subsection{Contremesure Passive} 

Les contremesures passives sont des précautions exécutées lorsqu'un intrus acquiert le status Détecté. 

\subsubsection{Localiser un Intrus} 

Un hackeur de sécurité ou un IA supervisant le système peut tenter de remonter la source d'une alerte passive. Voir \textit{Détecter,} p. 256. 

\subsubsection{Re-Identifier} 

Lorsqu'une alerte passive est déclenchée, un parefeu peut être configuré pour re-identifier tous les utilisateurs actifs, en commençant par les plus récemments connectés. Au début du prochain Tour d'Action, toute personen connectée sur le système doit dépenser une action pour se reconnecter. Pour les intrus, cela implqiue de faire un test d'Infosec, modifié de -10 par l'alerte passive, pour faire croire au système qu'ils sont des utilisateurs légitime. 

\subsubsection{Réduction de Privilèges} 

Certains système réduiront immédiatement les privilèges d'accès disponibles aux utilisateurs standard, et parfosi ceux des utilisateurs de sécurité, comme mesure de protection. Une tactique classique est de protéger tous les logs, de les sauvegarder et de s'assurer que personne n'a le droit de les effacer. 

\subsection{Contremesures Actives} 

Les contremesure active ne peuvent être démarrée que si l'intrus possède le status Verrouillé. 

\subsubsection{Contreintrusion} 

Un hackeur de sécurité ou une IA guardienne peut défendre proactivement un système en attaquant la source de l'intrusion. Pour que cela soit possible, l'intrus doit d'abord être tracé (p. 251). Une fois cette condition obtenue, les forces de sécurité peuvent ensuite lancer leurs propres intrusions sur l'ecto/insert de mesh du hackeur ainsi que sur son PAN. 

\subsubsection{Éjection} 

Un système qui a verrouillé un intrus peut aussi tenter de l'éjecter. l'éjection est une tentative pour supprimer le compte compromis, coper la connexion entre le hackeur et le système et le pousser dehors. L'éjection doit être démarrée par quelqu'un qui a des privilèges de sécurité ou d'administrateur. Un Test en Opposition d'Infosec est réalisé, l'intrus subissant un modificateur de -20 car il est Verrouillé. Si le personnage défendant le système l'emporte, l'intrus est immédiatement éjecté du système et le compte qu'il utilisat sera placé en quarantaine ou effacé. Ce compte ne sera pas réutilisble jusqu'à ce qu'un audit de sécurité l'approuve et en change l'identification. Toute tentative d'accéder le système depuis le même ID mesh que celui de l'intrus échouera automatiquement. 

\subsubsection{Redémarrer/Éteindre} 

Probablement l'option la plus radicale pour s'occuper d'un intrus est de simplement éteindre le système. Dans ce cas, le système ferme toutes les connexions sans-fil (si il en possède), déconnectes tous les utilisateurs, termine tous les process, et s'éteint - éjectant l'intrus. Le désavantage est, bien entendu, que le système doit interrompre ses activités. Par exemple, éteindre votre insert de mesh ou votre ecto signifie que vous perdez toutes les connexions avec vos coéquipiers, l'accès à la réalité augmentée et le contrôle des appareils asservis/liés. Initier un redémmarage/une extinction n'est qu'une Action Complexe, mais le process réel de l'extinction prend encore de 1 Tour d'Action (appareils personnels) à 1 minute (gros réseaux filaire ayant de nombreux comptes), déterminé par le maître de jeu. Redémarrer un système prend le même temps pour revenir en ligne. 



\subsubsection{Pister} 

Pour les systèmes de haute-sécurité, une contremesure populaire est de pister la localisation physique de l'intrus via son ID mesh (voir \textit{Pistage Physique,} p. 251). Dans la plupart des cas, la sécurité physique de l'habitat est ensuite alertée et reçoit la position du hackeur pour s'occuper du criminel. 

\subsubsection{Coupure Sans-fil} 

Une alternative à l'extinction ou au redéamrrage est de simplement couper toutes les conneions sans-fil en désactivant les capacités sans-fil du système. Le système perdra toutes els conenxions actives, mais tout intrus sera éjecté. La coupure sans-fil est une Action Complexe et se termine à la fin de ce Tour d'Action. Re-démarrer la connectivité sans-fil prend 1 Tour d'Action. 



\begin{quotation} \textbf{Piratage/Sécurisation Collective} 

Le piratage impliquera parfois des équipes d'attaquants et/ou des équipes de défense. Un hackeur peut être soutenu par sa muse ou par un autre membre de l'équipe ayant des compétences d'Infosec moyenne. Les réseaux durs sont souvent défendus et supervisés par des équipes de hackeurs de sécurité et d'IA extrêmement compétent. Lorsque vous pénétrez ou défendez un système informatique, les opérateurs doivent décider si ils agissent individuellement ou de concert. 

Chaque approche à ses inconvénients. Uen équipe qui choisit d'atatquer ou de maintenir la sécurité d'un système en tant qu'équipe doit allouer un personnage (généralement celui qui a la compétence Infosec la plus élevée) comem acteur principal (voir Travail d'Équipe, p. 117). Chaque personnage et muse supplémentaire ajoute un modificateur de +10 à chaque test (jusqu'à un maximum de +30) mais ne peut pas passer du temps sur d'autres actions que celles effectuées par le chef d'équipe. Lorsqu'une équipe agît de concert, les équipes peuvent changer de chef d'équipe à tout moment, au cas où les membres de chaque équipe soient spécialisées pour certaines tâches. 

Une autre possibilité pour les équipes qui attaquent ou qui défendent est de choisir d'agir individuellement avec un objectif commun. Chaque hackeur doit mener son intrusion de son côté, avec des répercussions individuelles pour la détection et la contreintrusion, ce qui augmente la possibilité d'affecter tous les intrus si l'un d'eux est Détecté ou Verrouillé. D'un autre côté, une équipe d'intrus peut effectuer plusieurs actions simultanément de manière coordonée et peut temporairement écraser la sécurité disponible. La même chose est vrai pour les équipes de défenses, qui peuvent accomplir plus de choses en répartissant les actions, en laissant certains en supervision endant que d'autre lances des attaques de contreintrusion et d'autre contremesure. \end{quotation} 



\section{Subversion} 

Une fois qu'un intrus a envahi avec succès un appareil ou un réseau, il peut faire toutes les tâches qui l'intéresse, dans la limite de ce qu'autorise le système. En fonction du compte que l'intrus a piraté, il peut ou peut ne pas avoir les privilèges d'accès pour faire ce qu'il veut faire. Si ses droits d'accès le lui permettent, l'activité est gérée comme celle d'un utilisateur légitime et aucun test n'est nécessaire (sauf si l'activité elle-même nécessite un jet, comme la Recherche). Par exemeple, un hackeur infiltre le système de sécurité de l'habitat avec un compte de sécurité qui peut surveiller les caméras, désactiver les capteurs, regarder les vidéos de surveillance enregistrées, et ainsi de suite, comme n'importe quel utilisateur légitime avec les privilèges de sécurité aurait le droit de le faire. Se lancer dans n'importe quel type d'activité pour laquelle vous n'avez pas oles droits d'accès et plus difficle et nécessite de bidouiller le système. Cela éncessite en général un Test d'Infosec réussit, modifié par la difficulté de l'action telle que notées dans la table de la Difficulté de Subversion Dans la plupart des ca, il ne s'agît pas d'un Test en Opposition même si le système est activement supervisé, sauf si le contraire est précisé. Rater ce genre de test amènera cependant en un changement de status du hackeur intrus (voir \textit{Rater des Tests,} p. 256). Des exemples pour différentes subversion de système sont donnés dans l'encadré \textit{Exemple de Subversion}. Il ne s'agît cependant pas d'une liste exhaustive et les maîtres de jeu et les jouerus sont encouragés à improviser les effets de jeu au cas où une action n'aurait aps été décrite de manière explicite. 

\subsection{Illusions de Réalité Augmentée} 

Un hackeur qui a infiltré un ecto, un insert de mesh ou d'autres appareils avec une interface RA peut injecter différent type d'illusions visuel, auditives, tactiles et même émotionnelle dans la réalité augmentée de l'appareil de l'utilisateur, en fonction du type d'interface utilisée. La façon dont répondra l'utilisateur piraté dépend d'un certain nombre de facteurs, tels que leur état de conscience de la présence d'un attaquant (le hacker a le status Détecté ou Verrouillé), le type d'interface qu'ils utilisent (entoptique ou haptique), et le degré de réalisme de l'illusion. Les mielleures illusions sont, bien enetendu, préparée en avance, en utilisant les meilleurs outils de manipulation d'image et de sens. De telles illusions sont hyper-réaliste. QUiconque fait un Test de Perception pour les identifier comem fausse 

subit un modificateur de -10 à -30 (à la discrétion du maître de jeu). Une collection ecclectique de logiciels offrent une gamme varéie d'illusions RA. 

\begin{table} \begin{tabularx}{\textwidth}{|l|X|} \hline

\multicolumn{2}{|c|}{\textbf{Difficultés des Subversions}} \\ \multicolumn{2}{|c|}Modificateurs de difficultés pour les tâches informatiques de base} \\ \hline

\hline

–0 &Exécuter des commandes, voir des informatiosn restreintes, exécuter des logiciels restreint, ouvrir/fermer des connexions vers d'autres systèmes, lire/écrire/copier/effacer des fichiers, accéder à des flux de capteurs, accéder à des appareils asservis\\ \hline

–10 &Changer les régalges du système, altérer des logs/fichiers restreints\\ \hline

–20 &Interférer avec les opérations systèmes, altérer les entrées sensorielles/RA \\ \hline

–30 &Éteindre le système, éjecter un utilisateur/une muse, lancer des contremesures contre quelqu'un \\ \hline

\end{tabularx} \label{tab:subversion-difficulties} \end{table} 



\begin{table} 

\begin{tabularx}{\textwidth}{|l|X|} 

\hline

\multicolumn{2}{|c|}{\textbf{Exemples de Subversion}} \\ \multicolumn{2}{|c|}{En plus des tâches notées dans la table de Difficultés des Subversions,} \\ \multicolumn{2}{|c|}{ces modificateurs présentent certaines actions additionnelles.} \\ \hline

\textbf{Modificateur} &\textbf{Tâche} \\ \hline

\multicolumn{2}{|c|}{Pirater des Bots/véhicules} \\ \hline

–0 &Donner des ordres aux drônes \\ \hline

–10 &modifier les paramètres système des capteurs, désactiver les systèmes de capteur ou d'armement\\ \hline

–20 &Altérer les entrées d'un samrtlink, envoyer de fausse données à l'IA ou au téléopérateur\\ \hline

–30 &Éjecter l'IA ou le téléopérateur, prendre le contrôle par la prise marionnette\\ \hline

\multicolumn{2}{|c|}{Pirater des Ectos/Inserts de Mesh} \\ \hline

–0 &Interagir avec des entoptiques, ajouter toutes les personnes à portée comem amis, effectuer des achats en ligen en utilisant les informatiosn bancaire de l'utilisateur, intercepter les communications, enregistrer l'activité. \\ \hline

–10 &Altérer le profile/status sur les réseaux sociaux, ajuster les filtres de RE, bidouiller l'interface sensorielle, changer l'habillage RA, changer l'avatar, accéder aux VPN \\ \hline

–20 &Bloquer ou brouiller les sens, injecter des illusions RA, imiter des commandes aux drones/appareils asservis \\ \hline

–30 &Déconnecter l'utilisateur de la RA \\ \hline

\multicolumn{2}{|c|}{Pirater des Systèmes d'Habitats} \\ \hline

–0 &Ouvrir/fermer les portes, arréter/démarrer les ascensseurs, manipuler l'interco \\ \hline

–10 &Ajuster la température/l'éclairage, désactiver les alertes de sécurités, remplacer l'habillage entoptique, verrouiller des portes, intervetir les régulateurs de traffic\\ \hline

–20 &Désactiver des sous-systèmes (plomberie, recyclage, etc.), désactiver les liens sans-fil, répartir les équipes de réparation \\ \hline

–30 &Contourner les blocages de sécurités\\ \hline

\multicolumn{2}{|c|}{Pirater des Systèmes de Sécurité} \\ \hline

–0 &Bouger/manipuler les caméras/capteurs, localiser les systèmes/bots/gardes de sécurité \\ \hline

–10 &Ajuster les motifs de balayage des capteurs, consulter les logs de sécurité, désactiver les systèmes d'armes \\ \hline

–20 &Effacer les logs de sécurité, répartir les équipes de sécurité\\ \hline

–30 &Désactiver les alertes \\ \hline

\multicolumn{2}{|c|}{Pirater des Systèmes de Simulspace} \\ \hline

–0 &Voir l'état actuel du simulspace, des simulmorphs et des égos connectés \\ \hline

–10 &Changer les règles de domaines, ajouter des codes de triche, modifier les paramètres de l'histoire, altérer les simulmorphs, changer la dilatation temporelle \\ \hline

–20 &Éjecter une simulmorph, modifier/effacer les IA joueuses \\ \hline

–30 &Interrompre la simulation \\ \hline

\multicolumn{2}{|c|}{Pirater des Spimes} \\ \hline

–0 &Obtenir un rapport d'état, utiliser les fonctions de l'appareil\\ \hline

–10 &Ajuster les réglages de personnalités de l'IA/voix, ajuster l'ordonnancement des tâches \\ \hline

–20 &Désaciver les capteurs, désactiver des fonctionnalités de l'appareil\\ \hline

\end{tabularx} \end{table} 

Les hackeurs peuvent aussi improviser des illusions à la volée, généralement en modifiant les données sensorielles d'autres sources, bien que cela soit plus difficile et plus facilement noté (en ajoutant typiquement un modificateur de +10 ou de +30 au Test de Perception). L'avantage est que le hackeur peut modifier l'illusion à la volée en fonction des actions de l'utilisateur ou des facteurs environnementaux. les logiciels d'illusions RA offrent également différents modèles qui peuvent être modifiés et contrôllé en temps-réel grâce à une interface. 

Lorsqu'un utilisateur est bombardé d'illusion RA, le maître de jeu devrait faire un test de perception secret pour voir si le eprsonnage remarque l'illusion. Et même si il la remarque, le personnage peut toujours y réagir. Presque tout le monde se jettera à terre si un objet leur fonce dessus soudainement, leur corps réagsisant de son propre chef avant que le cerveau ne comprenne que c'est une illusion et non une menace. 

Au delà de luer valeur trompeuse, les illusions peuvent être utilisée pour distraire un utilisateur ou géner d'une manière ou d'une autre ses facultés de perception. Par exemple, des nuages noirs illusoires peuvent obscurcir la vision, des bruits trés fort vrillant les oreilles peuvent faire reculer les gens et une sensation permanente de chatouillement peut rendre quelqu'un fou. De tels effets peuvent appliquer un modificateur de -10 aux Tests de Perception et d'autres actions, mais l'utilisateur peut également ajuster ses filtres et/ou désactiver sa RA si nécessaire. 

\subsection{Backdoors} 

Une backdoor est une méthode de contourner l'identificaton normale d'un système et ses fonctions de sécurité. Cela donne à un hackeur la possibilité de s'infiltrer dans un système en exploitant un défaut (qui peut prendre la forme d'un programme installé, de modification d'un programme existant ou d'un appareil matériel) qui a été précédemment intégré au système, soit par lui-même soit par un autre hackeur (qui aprtage la backdoor). 

Pour installer une backdoor, le hackeur doit réussir à infiltrer le système et réussir un Test de Programmation et un Tets d'Infosec (ou un Test en Opposition si le système est activement supervisé). Le Test de Programmation détermine la qualité de la backdoor et sa dissimulation dans les processuss système, alors que le test d'Infosec représente l'incorporation de cette backdoor dans le système sans que la sécurité ne la remarque. Modifier le test de Programmation de -20 si le hackeur veux avoir des privilèges de sécurité, et de -30 pour être admin, en utilisant la backdoor. 

une fois installée, utiliser une backdoor ne nécessite aucun test pour accéder au système - le hackeur se connecte simplement comme si il était un utilisateur légitome, et obtient le status Couvert. Quiconque connaît les détails de la backdoor peut l'utiliser. 

La durée pendant laquelle la backdoor persistera dépend de plusieurs facteurs et est laisser à l'appréiation du maître de jeu. Les backdoors ne serotn géénralement détecté que pendant des audits de sécurité complet, les systèmes les plus paranoïaque les détecterons donc plus rapidement. Les audits de sécurité peuvent également avoir lieu lorsqu'un intrus est Détecté et qu'il n'est jamais Verrouillé. Les audits de sécurité sont une Action de Tâche avec un intervalle de 24 heure. Le personnage qui mène l'audit fait un Test d'Infosec pour remarquer la backdoor. Si l'installeur de la backdoor a obtenu une Réussite Exceptionnelle sur leur test de Programmation, ce test d'Infosec subit un modificateur de -30. 

\subsection{Planter des Logiciels} 

Le sintrus peuvent tenter de planter un logiciel en terminant les processuss qui les exécutent. Cela nécessite une Action Complexe et un Test d'Infosec. Notez que certains logiciels sont configurés pour être relancés automatiquement, mais cela peut mettre entre 1 Tour d'Action et 1 minute, en fonction du système. 

Les hackeurs peuvent planter les IA, les IAG et même les infomorphs de cette manière, mais le mode opératoire est plus difficile. Dans ce cas, un Test en Opposition d'Infosec est fait contre la cible, qui est immédiatement consciente de l'attaque. Deux tests consécutifs doivent réussir avant de planter une IA, trois pour planter une IAG ou une infomorph. En cas de réussite, l'IA/infomorph redémarre immédiatement, ce qui prend générallement 3 Tours d'Action, plus si le maîþre de jeu le désire. 

\subsection{Supprimer les Traces d'Intrusion} 

Les hackeurs qui ont évité d'être Verrouillé peuvent tenter de nettoyer les traces de leur intrusion avant de quitter un système. Cela implique d'effacer les données incriminantes dans les logs d'accès et de sécurité, et de cacher les autres preuves de la compromission du système. Cela nécessite une Action Complexe et un test d'Infosec, ou un Test en Opposition d'Infosec si le système est activement supervisé. Si le test réussit, l'intrus a effacé tout ce qui aurait pu être utilisé pourle pister plus tard, tels que les ID mesh et autres. 

\subsection{Pirater un VPN} 

Les réseaux privés virtuels (VPN) sont un peu plus complexe à pirater que les appareils standards. Comme ils existent en tant que réseaux chiffré à l'intérieur du mesh, accéder aux canaux de communication à l'intérieur d'un VPN est impossible sans la clef de chiffrement. Cela veut dire que toute tentative d'écouter le traffic du VPN est également impossible sans la clef. 

La seule façon de pirater un VPN est de pénétrer un apapreil qui fait parti du VPN et d'exécuter le logiciel de VPN. Une fois qu'un intrus à accès à un tel appareil, il peut tenter d'accéder au VPN. Le compte que le hackeur a compromis peut avoir des privilèges VN, ce qui signifie qu'ils ont pénétré le VPN. Si ce n'est pas le cas, ils doivent pirater leur accès, ce qui nécessite un Test d'Infosec avec un modificateur Mineur (-10). 

une fois que l'accès au VPn est acquis, le hackeur peut le considérer conmme n'importe quel autre réseau. Ils peuvent pirater d'autres apapreil sur le VPN, éouter du traffic VPN, pister d'autres utilisateurs du VPN, rechercher des données dissimulées sur le VPN et ainsi de suite. 

\subsection{Scripter} 

Un script est un progrmamme simple - une liste d'instruction - qu'un hackeur peur embarquer dans un système afin qu'il soit exécuter à un moment programmé ou selon certains conditions, sans même que le hackeur ne soit présent. Lorsqu'il est activé, le script va s'occuper d'un certain nombre d'opérations système limité par les possibilités du système d'exploitation et des droits d'accès que le hackeur avait lorsqu'il a implémenté le script dans le système. Les scripts sont 

une bonne manièr de subvertir un système sans être nécessairement en danger lorsqu'ils le font. 

Les scripts peuvent être programmés à la volée ou préprogrammés. Lorsqu'un personnage écrit un script, il doit détailler quelles opérations système seront faites, dans quel ordre et à quel moment (ou ce qui déclenche le tout). Le script ne peut pas contenir plus d'étapes/de tâches que la compétence Programmation $\div$ 10 (arrondi à l'infoérieur) du personnage. Pour programmer un script, le personnage doit réussir un Test de Programmation avec un intervalle déterminé par le maître de jeu. 

Pour charger le script, le personnage doit s'être introduit avec succès dans le système et doit réussi un Test d'Infosec (ou un Test en Opposition d'Infosec si le système est activement supervisé). En cas de réussite, le script est chargé dans le système et s'exécutera de la manière dont il a été programmé. 

Une fois que le script a été activé, il décelnche la séquence d'action pré-programmée. La compétence Infosec du programmeur est utilisée pour tous les tests nécessaire à ces actions. 

Les scripts inactifs peuvent être détectés par un audit de sécurité, de la mrme manière que les backdoors (p. 260). 

\begin{quotation} \textbf{Exemple} 

Sarlo a infiltré un système de sécurité et veut le préparer pour qu'un capetur de sécurité particulier se désactive et qu'une porte se déverouille à une heure configuré, permettant à son équipe d'infiltrer une zone de haute-sécurité. Il crée un script qui s'activera à 22:00 avec les étapes suivantes: \begin{enumerate} \item À 22:00, désactivation du cpateur de sécurité \item Puis déverouillage de la porte \item À 22:30 réactivation du capteur de sécurité \item Puis réactivation du capteur de sécurité \item Suppression des traces \end{enumerate} 

Ce script a 5 étapes, ce que Sarlo peut gérer avec sa compétence Programmation de 70. Il réussit ses Tests de Programmation et d'Infosec, et le scrip est chargé. Il s'activera à l'heure appropriée. Puisque le compte de Sarlo n'a pas les droits d'acès pour effectuer ces actions, chacune d'elle nécessitera un Test d'Infosec utilisant la compétence de Sarlo pour réussir. 

\end{quotation} 



\section{Piratage de Cybercerveau} 

Les pods et les synthmorphs (y compris certains véhicules) sont équipés de cerveaux cybernétiques). Alors que cette technologie permet à un ego transhumain de s'incarner dans une de ces forme et de les contrôller, elle a l'inconvénient d'être vulnérable au piratage, comme tout autre appareil électronique. 

Les cybercerveau n'ont pas de connexion sans-fil pour dévidentes raisons de sécurité, mais ils ont des prises d'accès (p. 306) et sont directement connectés aux inserts de mesh. Cela veut dire que pour pirater un cybercerveau, le hackeur doit siot avoir un accès direct et physique à la morph afin de s'y brancher, ou il doit d'abord pirater l'insert de mesh puis pénétrer dans le cybercerveau. 

Enr aison de leur importance, les cybercerveaux sont équipés avec de nombreuses fonctionnalités de sécurité qui rendent le sintrusion trés difficile. Appliquez un modificateur de -30 à toutes les tentatives de pénétrer et de subvertir un cybercerveau. (Notez que le modificateur de -30 pour pirater un compte administrateur ne s'applique pas aux cybercerveaux). Les cybercerveaux sont traités comme tout les autres systèmes dans le cadre des intrusions et des subversions, mais puisqu'ils hébergent l'égo contrôllan la morph, ils présentent certains opportunités de hacking unique. 

\subsection{Enfermement} 

Un personnage intus peut tenter d'\textit{enfermer} un ego, l'empéchant d'évacuer le cybercerveau. Le hackeur (avec le modificateur de -30 signalé au-dessus) doit battre le personnage défenseur ou sa muse dans un Test en Opposition d'Infosec. En cas de réussite, l'égo est empécher de se transférer dans un autre système. Pour bloquer complètement l'ego, le personnage de l'ego et sa muse doivent également être éjecté (p. 257) du système de contrôle du cybercerveau, sinon l'ego peut potentiellemnet se libérer. Les egoos piégés de la sorte sont relativement vulnérables. Ils peuvent, par exemple, être soumis à un upload forcé, à un fork forcé ou à de la psychochirurgie. 

\subsection{Piratage Mémoriel} 

Tous les cybercerveaux incluent l'augmentation mnémonique (p. 307), ou des souvenirs enregistrés numériquement. Un hackeur qui a accès au cybercerveau peut lire, altérer ou efface ces souvenirs avec un Test de Recherche ou d'Interface réussit (le modificateur de -30 s'applique). 

\subsection{Marionnettes} 

La plupart des cybercerveaux incluent un port marionnette (p. 307), permettant aux utilisateurs distant de prendre le contrôle du pod ou de la synthmorph et de le contrôller via téléoprétaion ou interception (p. 196). Cela permet au hackeur de prendre le contrôle du corp et de le manipuler à distance. Pour y parvenir, le hackeur doit prendre une Action Complexe et battre le personnage en défense dans un Test en Opposition d'Infosec; le hacker subit le modificateur de -30 noté plus haut. Un défenseur qui n'est pas éjecté put continuer à se battre pour le contrôle de la morph en utilisant une Action Complexe. Dans ce cas, il faut effectuer un autre Test en Opposition d 'Infosec. Cela peut amener à des situations où la morph passe en permanence du contrôle du hacker et du défenseur, ou sombre peut-être dans un état catatonique alors que les deux parties s'affrontent. 

\subsection{Écorcher} 

Avoir un accès direct à un cybercerveau ouvre les possibilités d'un type d'attaque particulière qui sont normalement infaisable en raison du filtrage strict de contenu qui intervient entre l'insert de mesh et le cybercerveau. L'une de ces possibilités et l'\textit{écorcharge} - l'utilisation d'algorithme de retour neural pour blesser l'esprit de la victime. Afin de réussir une attaque d'écorchage, l'intrus du cybercerveau doit déployer un programme écorcheur. Pour utiliser ce programme, il doit battre l'ego défenseur dans un Test en Opposition d'Infosec. Le modificateur de -30 pour pirater un cybercerveau s'applique à l'attaquant. Il xiste plusieurs type de programmes écorcheurs, chacun avec des effest différents: cautériseurs (dégats), asiles (stress), 

spasmes (douleurs), cauchemars (peur) et obturateur (privation sensorielle). Ils sont décrit p. 332 du chapitre \textit{Équipement.} 

\subsection{Extinction} 

Si un cybercerveau est éteint (p. 257), la morph cesse immédiatement toute activité, s'effondrant ou roulant au sol éventuellement. Les pods donneront l'impression d'être dans le coma. L'ego sera cependant redémarré en même temps que le cybercerveau. 

\subsection{Interrompre l'Alimentation de la Pile Corticale} 

Le cybercerveau envoie des données à la pile corticale. C'est une connexion monodirectionnelle, la pile corticale ne peut donc aps être piratée, masi le transfert de donénes peut-être interrompu. Cette coupure nécessite un Test en Opposition d'Infosec entre le hackeur (avec le modificateur de -30) et le défenseur. La sauvegarde de l'ego ne sera pas mise à jour tant que la connexion reste coupée. 

\section{Interception Radio} 

L'interception radio est une méthode de transmission de signax raido qui interfèrent délibéremment avec d'autres signaux radios afin de casser les communications. Dans le monde hautement-connecté d'\textit{Eclipse Phase,} l'interception intentionnelle est souvent illégal et impoli. 

l'interception radio ne nécessite aucun équipement particulier en plus d'un appareil sans-fil standard, tel qu'un ecto ou un insert de mesh. L'interception peut être \textit{sélective} ou \textit{universelle}. L'interception sélective cible un apapreil aprticulier ou un ensemble d'appareil. Pour intercepter de manière sélective, le pesonnage doit avoir annalysé la cible (p. 251). L'interception unviverselle cible tous les apapreils équipés d'émetteurs sans-fil de manière indifférente. 

Intercepter nécessite juste une Action Complexe et un Test d'Interface. En cas de réussite, l'appareil affecté et à porté voit ses communications radio interrompues - ils sont coupés du mesh et ne peuvent plus communicquer sans-fil. Les appareisl filaires ne sont aps affectés. 

Les apapreils équipoés d'IA tenteront automatiquement de vaincre l'interception, ce qui nécessite une Action Complexe (les utilisateurs transhumains peuvent faire de même). Dans ces cas un test en Oppositin Variable est fait entre l'intercepteur et le défenseur. Si l'intercepteur l'emporte, toutes les communications sont bloquées; si le défenseur l'emporte il reste non-affecté. Si les deux parties l'emporte, les communications du défenseur sont impactées masi pas complètement coupées. le maîþre de jeu décide de la quantité d'information le défenseur peut faire apsser, et comment cela affecte l'utilisation du mesh. 

\subsection{Interception Radar} 

L'interception peut également être utilisée pour interférer avec le radar. Dans ce cas, l'intercepteur fait un Test d'Interface. En cas de réussite, le radar dubit des interférences, imposant un modificateur de -30 sur tous les tests basés sur le capteur. L'entité opérant le radar peut tenter de vaincre ces interférences en battant l'intercepteur dans un Test en Opposition d'Interface. 

\section{Simulspaces} 

Les simulsapces sont des environnements de réalité virtuels dans lesquels la résolution est au-delaà de la haute-définition réalistique et va vers l'hyper-réel. Les environnements qu'ils créent sont des illusions complètes et authentique, depuis les aspects telles que l'éclairage, le cycles du jour et de la nuit et de la méto aux détails les plus minutieux en passant par les sensations. Se brancher dans un simulspace est à peu près comme entrer dans une raléita ou un monde alternatif, ce qui est la raison pour laquelle els simulsapces sont devenus un divertissement incroyablement populaire. 

Alors que les simulspaces ne peuvent normalement pas blesser les personnages qui s'y immergent, les algorithmes sensoriels ne sont pas conçus pour être des programmes d'attaque, les exépriences dans les simulspaces peuvent avoir un impact psychologique fort sur un ego la simulation étant aussi proche de la réalité que possible. Un personnage qui est "physiquement" torturé dans un simulspace ne sera pas physiquement blessé, mais le stress mental de l'expérience pourra toujours être suffisant pour causer des traumas permanents. 

\subsection{Simulmorphs} 

Les personnage accèdent à un simulspace en utilisant un avatar appelé une simulmorph. Cette simulmorph est créée par le simulspaec, en se basant sur les règles de domain de la simulation et sur certaiens caractéristiques de la morph ou de l'ego accédant àc ette simulation. En fonction de la simulation, cette simulmorph peut être personnalisée à différents degrés. 

En interagissant avec la simulation, considérer la simulmorph comem une infomorph basique pour toutes les probélmatiques de règle, même si l'ego possède toujours une autre morph dans la réalité. 

Lorsqu'il accède à un simulspace, les muses ne sont habituellement pas transferrés dans la simulation,bien qu'elles puissent accompagner un personnage si les règles de domaine l'autorise. Dans ce cas, les muses sotn considérés comme des personnages sdifférent dans le simulspace, avec leur propre simulmorph. 

En fonction du rôle du simulspace dans l'histoire, le maîþre de jeu peut vouloir inventer des stats physiques pour les corps simulmorphs, particulièrement si les personnages vont apsser beaucoup de temps dans la simulation. Ces stats peuvent littéralement être inventée - il s'agît d'une réalité virtuelle après tout, et tout va avec. De manière alternative, le maître de jeu peut simplement improviser et inventer toute les statistique nécessaire à la volée au fur et à mesure que le besoin se fait sentir. 

\subsection{Immersion} 

Lorsqu'un personnage s'immerge dans un simulspace, il "devient" la simulmorph. Le corps physique du personnage, généralement enfermé et protégé dans une cuve ou sur un canapé, demeure inerte. Pendant qu'il est immergé, il subit un modificateur de -60 à tous les Test de Perception ou à toute tentative d'agir avec sa morph physique. Les personnages peuvent entrer et quitter le simulspace à volonté, mais basculer entre dedans et dehors nécessite une Action Complexe. 

Si le simulspace plante ou que le personnage est éjecté d'une manière ou d'une autre, il reprend immédiatement le contrôle de sa propre morph. Les chocs d'éjection RV sont extrêmement discordant, et le personnage souffre d'1d10 $\div$ 2 de stress mental. 



\subsection{Interaction Extérieure vers le Mesh} 

Un personnage accédant un simulspace peut toujours nteragir avec le mesh (et à travers lui, avec el monde extérieur) du moment que les règles du domaines l'autorise. Toute interaction extérieure est cependant sujette à des problèmes de dilatation temporelle. Par exemple, dans un simulsapce s'exécutant plus rapidment que le temps réel, avoir une discussion vers le monde de la viande à l'extérieur est atrocement lent, les secondes dans le monde réel correspondant à plusieurs minutes en RV. Siun personnage souahite accéder directement dautre nœud meshé, il doit basculer ou se déocnnecter du simulspace. 

\subsection{Simulspace Rules} 

Since a simulspace is an alternate world whose realism matches reality, characters use their physical skills and aptitudes as if they were acting in the real world with few exceptions: \begin{itemize} \item Though intrusion and hacking can be represented as another layer of the simulation, there is no actual hacking within the simulspace (see \textit{Hacking Simulspaces}). 

\item Asyncs cannot use their psi abilities in simulspace, though such abilities can be simulated. 

\item Any ``physical'' damage taken in the simulspace is treated as ``virtual'' damage. While virtual injuries and wounds use the same mechanics, characters that die in a simulspace are usually simply ejected from the simulation. In some cases ``dead'' characters are brought into a white room and can re-enter or just watch the simulation, depending on the domain rules. 

\item Mental stress or trauma inflicted during a simulation carries over to the ego as real Lucidity damage. At the gamemaster's discretion, some mental stress may be reduced if the character is aware that they are in a simulation. \end{itemize} 



\subsection{Domain Rules} 

Anything goes in a simulspace, as dictated by the domain rules. A simulspace may range from approximating reality very closely to differing drastically Gravity might fluctuate, the visual light spectrum might not exist, characters might heal virtual damage effortlessly, simulmorphs may be capable of transmogrifying into other creatures, everything might be underwater—the possibilities are endless, limited only by imagination. In game terms, this allows the gamemaster to make up rules on the fly. 

\subsection{Cheating} 

As with any good game, simulspaces provide ways to cheat. Cheats are either built into the simulspace software or (externally) programmed in by a hacker. Cheats allow for a character to break the domain rules in some way. This may be a special power, a way to alter some environmental factor (like flying), altering the time dilation, some sort of power-up ability, a way to get info on other simulmorphs, or a short-cut through part of the simulation. In game terms, cheats might provide bonus modifiers to certain skill or stat tests made by a simulmorph. Cheating is usually forbidden Players who cheat in a simulspace game and who get caught may face eviction from the simulspace. 

\subsection{Hacking Simulspaces} 

Since simulspaces are complex virtual environments and often run on time dilation, hackers cannot hack them in a normal manner when they participate in the simulation. There are ways to affect and influence the simulation from within, but the degree of subversion that is achievable is limited. For this reason, hackers rarely enter into VR to hack. Hacking into the external system running a simulspace is just like breaking into any other system. Use all of the standard rules for intrusion and subversion. 

\begin{quotation} \textbf{Hacking Simulspace From Within Modifier Task} \\ \begin{tabularx}{\textwidth}{|rX|} \hline

–0 &Analyze simulation parameters, view domain rules, shape appearance of simulmorph, switch simulmorph character or morph type \\ \hline

–10 &Change probability of test outcomes, become invisible (“out-game”) to others \\ \hline

–20 &Interfere with simulation (e.g. make it rain, generate earthquakes), generate items, ignore domain rules, kill or lockout other simulmorphs \\ \hline

–30 &Go into god mode, command simulated characters, take over the simulation \\ \hline

\end{tabularx} \end{quotation} 

\subsubsection{Meddling From The Inside} 

Within a simulspace, a hacker's only choice for interacting with the VR controls is through the standard interface that any simulmorph can pull up. Typically used for standard user features like adjusting your simulmorph or chatting with or checking the status 

of other users, a clever hacker might find some ways to subvert the system. Such options are usually limited however, as a number of system controls and processes cannot be accessed and manipulated from the inside. Most of the hacker's options are going to involve meddling with the simulation and its specific domain rules or possibly gaining access to cheats. To make a change requires a successful Interface Test. Ultimately the gamemaster decides what the hacker can and cannot get away with, based on the limitations of that particular simulspace. Most simulspaces are monitored to prevent cheating and abuse, though the monitors are typically preoccupied with maintaining the simulspace as a whole, dealing with other users, etc. At the gamemaster's discretion, such a monitor might get to make an Interface Test (possibly with a modifier for distraction) to notice the hacker's efforts. 

\section{AI And Muses} 

AIs are sentient but specialized programs. Like other software, they must be run on a computerized system. Most AIs are run on bots, vehicles, and other computerized devices where they can assist transhuman users or operate the machine themselves. They are also commonly used to actively monitor computer systems against intrusion attempts. \textit{Muses} are AIs that specialize as personal companions, always at a character's virtual side every since they were a child. Sample AIs and muses can be found on p. 331 of \textit{Gear.} 

\subsection{AI Limitations} 

AIs feature a number of built-in restrictions and limitations. To start with, they can be loaded in the cyberbrains of pods and synthmorphs, but they may not be downloaded into biomorph brains. As software, they use the same rules as other software and may be shut down, restarted, copied, erased, stored as inert data, infected with virii, and reprogrammed. Due to their size and complexity, only one AI (or infomorph) may be run on a personal computer at a time (see \textit{Computer Capabilities,} p. 247), and they may not run on peripheral devices. While they possess cognition and intelligence, they are incapable of self-improvement and cannot expand their programming and skills on their own. Although –0 

Analyze simulation parameters, view domain rules, shape appearance of simulmorph, switch simulmorph character or morph type –10 

Change probability of test outcomes, become invisible (``out-game'') to others –20 

Interfere with simulation (e.g. make it rain, generate earthquakes), generate items, ignore domain rules, 

kill or lockout other simulmorphs –30 

Go into god mode, command simulated characters, take over the simulation they are not able to learn they do possess memory storage that grants them the ability to remember and a limited form of adaptation. AIs do not earn Rez Points, nor do they have Moxie. 

AIs have aptitudes no greater than 20 but are incapable of defaulting. If they don't possess a skill, they don't know how to do it. (At the gamemaster's discretion, they may default to field skills or similar skills as noted on p. 173 with a –10 to –30 modifier). They can use skills like any character in \textit{Eclipse Phase}, however they may not possess any Active skill at a rating higher than 40 or Knowledge skill higher than 90—the maximum amount of expertise that their skill software allows. 

While AIs are programmed with personality templates and empathy, they are generally less emotive and difficult to read (apply a –30 modifier to Kinesics Test made against them, when in pod bodies). When combined with non-expressive synthetic morphs, they are even more difficult (–60 modifier). Some AIs lack emotive capabilities altogether and are impossible to read with Kinesics skill. 

AIs do have a Lucidity and Trauma Threshold stat, and are capable of suffering mental stress and traumas. 

\subsection{Commanding AI} 

AIs and muses are programmed to accept commands from authorized users. In some circumstances, they may also be programmed to follow the law or some ethical code. Programming is never perfect, however, and AIs can be quite clever in how they interpret commands and act on them. In most cases, an AI will rarely refuse to follow a request or obey a command. Given that they also usually have a duty to protect the person commanding them, the AI may be reluctant to follow commands that could be construed as dangerous or having a negative impact on the user. Under certain circumstances, preprogrammed imperatives can force an AI to ignore or disobey their owner's commands (gamemaster's discretion). 

\section{AGI And Infomorphs} 

The term ``infomorph'' is used to refer to any ego in digital body, whether that be an AGI or the digital emulation of a biological mind (including backups and forks). The following rules apply to infomorph and AGI characters. 



\begin{quotation} \textbf{ROLEPLAYING MUSES} 

Muses should not be viewed as a mere tool for getting extra skills, but as an opportunity to enhance roleplaying. Though typical muse AIs are not complete intelligences (though they can be, see Infomorphs as Muses), their personality matrix is often quite sophisticated and they are very good at adapting to their user’s personality quirks. On the other hand, they share the same Real World Naiveté (p. 151) as AGI characters when it comes to understanding all the facets of transhuman behavior, social interaction, body language, or emotion. Their personalities are more non-human, abstract, alien, and less passionate than transhuman life forms, often leading to conceptual misunderstandings and miscommunications. Likewise, their creative capacities are limited, instead bolstered by an ability to calculate odds, run simulations and evaluate outcomes, and make predictions based on previous experiences. 

Depending on the user’s stance towards sentient programs, muses can be viewed as intelligent toys, followers, servants, slaves, friends, or pets, which should somehow be reflected in game play. Most transhumans have also acquired a tendency to bond with a muse mentally due to its omnipresence and devotion to the user (like bonding to a child or puppy that then grows to be an adult). Therefore the subversion or even destruction of a muse personality is sometimes even equated with rape or murder. \end{quotation} 



\subsection{Software Minds} 

At their core, infomorphs are just programs and so they are treated like other software in terms of rules. They must be run on a specific personal computer or server (see \textit{Computer Capabilities,} p. 247). If that device is shut down, the infomorph also shuts down into a state of unconsciousness, restarting along with the device (infomorphs may also shut themselves down, though it is rare that they do so). If the device is destroyed, the infomorph is killed along with it (unless their data can somehow be extracted from any surviving components, perhaps resulting in a \textit{vapor,} p. 274). Infomorphs may copy themselves, though in some places this is illegal and in most places is frowned upon as it raises numerous ethical and legal questions. For this reason many infomorphs that copy and transfer themself to run on a new device will thoroughly erase themselves off the old one. 

As digital beings, infomorphs have no physical mind, but it is a simple matter for them to possess an uninhabited synthmorph, taking up residence in the cyberbrain (see \textit{Resleeving Synthmorphs,} p. 271). They may also download into biomorph bodies according to standard resleeving rules (p. 271). Even when disembodied, they may interact with the physical world via the mesh, viewing through sensors, streaming XP feeds, communicating with characters, commanding slaved devices, and teleoperating/jamming drones. Infomorphs have a Speed of 3, reflecting their digital nature and their ability to act at electronic speeds. If an infomorph sleeves into a body, however, it takes on the Speed of that morph. 



\begin{quotation} \textbf{INFOMORPHS AS MUSES} 

Instead of relying on underdeveloped muses for aid and companionship, characters may prefer to have a full-fledged digital intelligence at their side, whether that be an AGI, a backed-up biological ego, or fork of the character’s own personality. Alternately, a character with a ghostrider module (p. 307) could have both, carrying a muse in their mesh inserts and an infomorph in the ghostrider module. 

This possibility is very useful for infomorph player characters, as they can ride along in someone’s head and participate in team affairs without needing a morph of their own. \end{quotation} 

\subsection{AGI Characters} 

Though AGIs were not born in a biological body, their programming encompasses the full spectrum of human personality, outlook, emotions, and mental states. AGIs are in fact raised in a manner similar to human children so that they are socialized much like humans are. Nevertheless, on a fundamental level they are non-humans programmed to act human. There are inevitably points where the programming does not mask or alter the fact that AGIs often possess or develop personality traits and idiosyncrasies that are quite different from human norms and often outright alien. Unlike standard crippled AIs, AGIs are capable of full-fledged creativity, learning, and self-improvement (at a slow but steady pace equivalent to humans). Just like other characters, they earn Rez points and may improve their skills and capabilities. AGIs suffer none of the skill limitations placed on weak AIs, using skills just like any other character. On an emotional level, AGIs run emotional subroutines that are comparable to biological human emotions AGIs are, in fact, programmed to have empathy and share an interest in human affairs and prosperity, and to place significant relevance on life of all kinds. In game terms, AGIs emote like humans (and so Kinesics may be used against them) and are vulnerable to emotionally manipulative effects, fear, etc. 









