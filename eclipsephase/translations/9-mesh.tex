\chapter{La Mesh} \label{cha:mesh} 

Avant la Chute, Les humains étaient interfacés entre eux au travers de d'Internet, l'interconnexion des réseaux étant le facteur déterminant de l'évolution du world wide web. S'il a commencé comme un support électronique pour la récupération des informations de diverses sources (en remplaçant même les anciennes sources d'informations papier), les générations successives ont dévoloppés les communautés numériques et les services hébergés tels que les réseaux des sites, des wikis, des blogs et des réseaux sociaux. Ces outils d'ouverture, de collaboration et de partage ont jetés les bases d'une société moderne et d'informations interconnectées. De nouvelles étapes sont franchies avec l'interaction sans fil, la géolocalisation, le web sémantique et les approches et les innovations réalisées dans le domaine de l'interface homme-machine, avec l'avènement des interfaces cerveau-ordinateur, de la réalité augmentée (AR), la réalité virtuelle (VR), et la lecture d'expérience (XP). 

This environment, coupled with the exponential growth of processing power and memory storage, created an evolutionary path for the development of intelligent agents—designed to augment human information processing—that then transformed into artificial intelligences (AIs) in the following decades. While these “weak” AIs did not possess the full range of humancognitive abilities, tended towards overspecialization, and were restrained by programmed limitations, the digital evolution toward artificial general intelligences (AGIs)—”strong” AIs with intelligence capabilities that equaled or exceeded human abilities—could not be halted. A partir de là, il n'était qu'une question de temps avant que ses soit-disantes IA soit capablent capable d'auto-amélioration récursive, menant à une croissance exponentielles de leur intelligence. Malheureusement pour l'humanité, elles devinrent les TITANs. 

Même bien avant la Chute, l'internet d'en-temps à été transformé en quelque chose d'autres. Au lieu de se connecter via des serveurs centraux, les utilisateurs étaient relié sans fil les uns aux autres, créant un réseau décentralisé d' appareils de poche, ordinateurs personnels, des robots, et des dispositifs électroniques entremêlés. Les utilisateurs étaient en ligne tout le temps et lié à tout et tout le monde autour d'eux dans un environnement informatique ubiquitaire. This was especially true of those participating in humanity’s expansion into space. Déconnecté d'Internet en raison des distances et du lag des communications, ces utilisateurs sont néanmoins liés à toutes les personnes et les objets de leur environnement proche ou de leur habitat, créant ainsi des réseaux locaux maillés. Ainsi était née la Mesh, en prenant la place de l'ancien Internet de la terre, perdus au cours de la Chute. 

\section{Les possibilités de la  Mesh.} 

La Mesh, telle qu'elle existe dans\textit{Eclipse Phase} , n'est possible qu'avec les avancées majeurs en informatique, communication et nanofabrication. Les émeteurs et récepteurs radio sont si petits qu'ils peuvent littéralement êtres incorporés dans n'imorte quoi. En découle que chaque objet est "intélligent" et connecté, ou du moins posséde une puce de radio-identification (RFID). Même la nourriture est "taggée" avec des puces commestibles, qui renseigne sur les dates d'expiration et les valeurs nutritionelles. Les autres médiums de communication, comme les lasers ou les micro-ondes, augmentent le flux d'information. 

Les technologies de stockage d'informations sont si évoluées, que n'importe quel \textit{surplus} de capacité de stockage personnelle dépasse largement celle de tout l'Internet du 20eme siècle. Les "lifeloggers" peuvent carrément enregistrer chaque moment de leur vie et ne jamais craindre de manquer d'espace disque. La quantité de données que les gens transportent dans les implants Mesh dans leur tête ou dans les ordinateurs portables Ecto est stupéfiante. 

Les capacitées de calcul ont aussi atteint des niveau extrêmes. Mêmes les super-calculateurs sont du passé quand le moindre appareil de la taille d'un téléphone portable peut suffire à quasiment tous vos besoins, même s'il sagit de faire tourner une IA personnelle, télécharger des médias, uplaoder du porno et scanner des milliers de flux d'informations. Avec le réseau Mesh, les appareils qui atteignent la limites de leur capacité de calcul, partagent simplement leur fardeau avec les appareils autour d'eux, créant un énorme réseau de calcul partagé, qui ressemble un peu a un super-calculateur, partagé entre tous. 

De même, les capacités de transferts dépassent maintenant de loin ce dont les utilisateurs ont besoins. Tous ceux qui sont né ces dernières générations ont toujours vécu dans un monde de médias hyper-réaliste, multi-sensoriels de toute taille, disponibles instantanément en téléchargement et en envoi de n'importe où. Les bases de données massives et les archives sont copiés dans les deux sens avec la facilité. La bande passante est un tel non-problème que la plupart des gens l'oublient qu'elle existe. En fait, étant donné l'énorme quantité de données disponibles, \textit{trouver } l'information ou les médias que vous cherchez prend beaucoup plus de temps que de les télécharger. La Mesh n'est \textit{jamais } indisponible. Comme c'est un réseau décentralisé, si l'un des appareil est mis hors-connection, l'information le contourne seulement, trouvant un chemin via les milliers si ce n'est les millions de noeuds disponibles. De même, La Mesh entière se comporte comme un réseau peer-to-peer, de sorte que les des transferts de lourds fichiers sont séparés en morceaux gérables qui prennent des routes indépendantes. En fait, la plus part des utilisateurs gardent une archive torrent, accéssible publiquement et partagé.  

Les réseaux privés existent encore, bien sûr. Certains sont physiquement isolés par des connetions filaires ou même dérrière une infrastructure inhibitrice de Wifi, qui maintiennent un réseau isolé et confiné. Cependant, la plupart fonctionnent a travers même la Mesh publique, en utilisant des protocols de tunneling cryptés qui fournissent des communications privées et sécurisées sur des réseaux non sécurisés. En d'autres termes, ces réseaux privés, font partie de la Mesh avec tout le reste, mais seuld les participants 

peuvent interagir avec eux grâce au cryptage, à l'authentification des utilisateurs et la vérification de l'intégrité du message. 

Avec le fractionnement de la transhumanité, les tenatives d'unification des logiciels en des formats standards ont toutes échouées. Cependant, les différents systèmes d'exploitation ou les protocoles sont rarement un obstacle grâce aux outils de conversion facilement accessible et le contrôle de compatibilité aidé par des IA. 

\end{quotation} \textbf{Recherche dans Solarchive : ECTO(-link)} \\ Marque de mobiles multifonctions, trés populaire en tant qu'assistants personels numérique avant la Chute, le nom Ecto est devenu un synonyme d'ordinateurs portable de poche à l'ère de la Mesh Les ordinateurs implantés standards sont aussi parfois dénommés Endo afin de refléter la différence entre le coté interne et externe de l'appareil. Peu importe s'ils sont ecto ou endo, les ordinateurs modernes possèdent tous un système d'exploitation (OS), une suite de programmes multifonctionnel qui comprend un navigateur Mesh, des outils des médias, de repérage, des programmes de socialisation (messagerie, mise à jour socnet), de cartographies et de logiciels de navigation, des logiciels de traduction et d'autres logiciels similaires. les OS sont hautement personnalisables, permettant l'utilisation de plug-in et add-ons pour tout les logiciel et les gadgets souhaitées. Typically, the user’s muse (personal AI assistant) facilitates software interactions. Le ecto lui-même est généralement la taille de la carte de crédit du 20ème siècle et peut être moulé et façonné en des formes différentes grâce aux matériaux intelligents. Ils sont souvent portés comme bijoux ou accessoires vestimentaires, notamment en bracelets. L'interface utilisateur varie selon les préférences de l'utilisateur. Wireless-enabled contacts and earbuds equip users who lack mesh implants, enabling them to experience augmented reality and the ecto’s AR control interface. Les interfaces de contrôle standards entoptiques (dans l'oeil)  sont également disponibles via Wifi, skinlink (transmission d'information par la peau), et fibres optiques. \end{quotation} 



\subsection{Meshing Technologies} 

Presque tous les biomorphs dans le système solaire sont équipés des implants Mesh de base (p. 300)—implanted personal computers. Ces implants sont cultivés dans le cerveau via nanochirurgie non-intrusive. Le processeur, l'émetteur-récepteur sans fil, les périphériques de stockage, et d'autres éléments sont directement câblés aux cellules neuronales, aux centres du language, des perceptions et de la paroles entre autres. L'émulation de la communication par la pensée (dite \textit{ transduction } ) permet à l'utilisateur de contrôler l'implant par la pensée et de communiquer sans vocalisation. Les implants Mesh transmettent les informations directement au cerveaux qui sont quelque fois percues comme la Réalité Augmentée, comme une surcouche aux sensations physiques de l'utilisateur. Dans la même veine, les implants mesh installés dans synthémorphs et les pods sont directement intégrées à leurs cybercerveaux (ce qui créé un problème de sécurité potentiel puisque les cyberverveaux sont vulnérables au piratage). 

Les appareils externes appelé ectos (p. 325) sont également utilisés pour accéder au Mesh, mais ceux-ci sont de plus en plus rares compte tenu de la prévalence des impalnts Mesh. Les options d'interfaces des ectos comprennent des interfaces haptiques, comme des écran tactile, des bracelets ou des gants qui permettent de détecter les mouvements de bras, de la main, ou d'un doigt (la souris et claviers virtuels), des systèmes de contrôle par mouvements et clinements des yeux, ou du corps tout entier (controle par l'axe du corps ou des membre pour les non-humanoides), des commandes vocales, et plus encore. L'information sensorielle est traitée au moyen de lentilles, lunettes, écouteurs intra-auriculaires (haut-parleurs vibrants sur l'os), combinaisons, gants, pince-nez, percing à la langue, et d'autres dispositifs sans fil qui sont liés (ou physiquement branchés) à l'ecto. 

\begin{quotation} \textbf{L'héritage des TITANs} 

Compte tenu des capacités techniques des ordinateurs personnels modernes, les superordinateurs de pointe à large bandepassante ne sont pas nécessaires. Mais il ya une autre raison pour laquelle ils sont évités: les Titans. 

Les mainframes, les clusters auto-organisés, et les systèmes conscient de calcul parrallèlle massivement distribués et parrallèllisés sont tous considéré comme des dangers ptentiel dans Eclipse Phase, car ils possèdent suffisament de puissance de calcul et de capacité de traitement de donné pour activer une IA germe et d'autre lancement abrupts de singularité Certains habitats sont jusqu'a aller interdire de tels systèmes, sous la plus grave des peines: la mort finale, y compris la suppression de toutes les sauvegardes et les forks des dernières années, dans la plupart des cas. 

Those supercomputers that habitats do allow are “hard networks” that control a habitat’s most crucial systems like orbit maintenance thrusters, life support, communications, power, or cutting-edge hypercorp R\&D projects. Ces systèmes sont généralement connectés en filaire, très surveillés et verrouillé dans les centres de traitement des données avec des restrictions d'accès rigoureuses, du coté informatique,  et des mesures de sécurité impitoyable, dans le monde réel. 

De même, les IAs se sont souvent très restreinte, et il n'est pas rare pour les AGIS être purement et simplement interdites, en particulier dans le Système Intérieur et la République Jovienne. La plupart des programmes intelligents sont limitées, avec des réstrictions de croissance, spécialement conçu pour les empêcher de s'auto-améliorer. \end{quotation} 

\subsection{Surcharge d'informations} 

Le Mesh contient d'énormes quantités de renseignements personnels et publics partagés par les utilisateurs, une mise en commun numérique 

des nouvelles, des médias, le discours, la connaissance, les données de l'environnement, des affaires, et de la culture. La transhumanité utilise le Mesh comme un outil d'échange, de communication et de participation avec d'autres utilisateurs, tant locaux qu'éloignés. En tant que tel, le maillage est une mise à jour, la source faisant autorité sur toutes les connaissances ou les activités transhumanistes. 

Tout ce qui est en ligne n'est gratuit, bien sûr, sauf peut-être dans les zones autonomistes. Assez peu de données de propriété industrielle est sont maintenus hors de la toile dans des stockages sécurisés, ou séquestrées loin dans les réseaux privés. Some of this is for sale, and heavily encumbered with digital restrictions—software media, nanofabrication blueprints, skillsofts, etc. Cependant, un mouvement florisant d'open-source offre des alternatives gratuites et ouvertes au grand nombre de données propriétaires et de nombreux groupes fourgent des versions crackées des produits propriétaires,et ce malgré la pression de certaines autorités. D'autres données sont simplement reliès à des intérêts concurrentiels (projets de recherche hypercorp) ou sont extrêmement privées, telles que les sauvegardes ego. 

\subsubsection{Spimes} 

Avec les données accumulées des affaires transhumaines, le Mesh est également encombré de l'information provenant nombre incalculable des appareils sans-fils qui tiennent à jour en permanence leur emplacement sur le Mesh, les enregistrements des capteur, et d'autres données. Les appareils qui sont localisés, auto-enregistreurs, auto-documentés et qui diffusent leurs données à qui veut l'entendre sont familièrement appelés "spimes". Depuis que les capteurs visuels, sonore ou autre sont devenus ridiculement minuscules et peu coûteux, ils sont désormais omniprésents et incorporés dans presque tous les objets et produits qu'une personne peut porter, s'appliquer, utiliser, avaler, s'injecter, etc. This allows almost any user to reach out through the mesh and gather environmental data and ambient sensor recordings from a specific location (or at least public locales—private areas typically block such signals or slave them to a local AI that filters their output). 

\subsubsection{Surveillance, vie privée, et sousveillance} 

Alors que les spimes sont aisement repèrables, ils contribuent aussi à un environement de surveillance constante. Entre les spimes, les micro-senseurs, les systèmes de sécurité omniprésents, les capacités d'enregistrement des implants Mesh presque universelements utilisés, à peu près tout est enregistré De plus avec les capacités de reconnaissance faciale, les réseaux sociaux/de réputation et autres logiciels de recherche de données, il devient vite évident que la vie privée est une notion dépassée. Ceux qui cherchent à masquer leur identité ou couvrir leurs mouvements doivent penser à cet état de fait un peu spécial. Sinon, les morphs communes (surtout les synthmorph et les pods) ont une apparence qui aide l'utilisateur à ce fondre dans la masse. 

Bien que tout cela peut resembler à un cauchemar orwellien de surveillance, une grande part des technologies d'enregistrements travaillent actuellement en ''sousveillanceé (surveillance par la base), ayant un rôle de transparance et de lutte contre les abus de pouvoir. 

Les régimes autoritaires avancer avec précaution, car ils sont aussi universellement surveillés, en dépit de leurs tentatives pour contrôler le flux d'information. Beaucoup de gens sont en outre désireux de se joindre à cette télésurveillance participative ouverte ``.''Avec une capacité de stockage quasi illimitée, les lifeloggers enregistrent chaque instant de leur vie et partagent avec d'autres leur expériences. 

\section{les interfaces AR, VR, et XP} 

Les médias Mesh sont accessibles en utilisant l'un des trois protocoles: la réalité augmentée (AR), la réalité virtuelle (VR), ou la lecture d'expérience (XP). 

\subsection{Augmented Reality} 

Most users perceive data from the mesh as augmented reality—information overlaid on the user's physical senses. For example, computer-generated graphics will appear as translucent images, icons, or text in the user's field of vision. While visual AR data—called \textit{entoptic} data—is the most common, other senses may also be used. AR input includes acoustic sounds and voices, odors, tastes, and even tactile sensations. This sensory data is high-resolution and seemingly ``real,'' though it is usually presented as something ghostly or otherwise artificial so as not to be confused with real-world interactions (and also to meet safety regulations). 

User interfaces are customized to the user's preferences and needs, both graphically and content-wise. Filters allow users to access the information they are interested in without needing to worry about extraneous data. While AR data is typically placed in the user's normal field of vision, entoptics are not actually limited by this and may be viewed in the ``mind's eye.'' Nevertheless, icons, windows and other interaction prompts can be layered, stacked, toggled, hidden, or shifted out of the way if necessary to interact with the physical world. 

\end{itemize} \textbf{Information At Your Fingertips} \\ \\ The following information is always available for most mesh users in a normal habitat: \\ \\ \item \textbf{Local Conditions} \begin{itemize} \item Local maps showing your current location, annotated with local features of personal interest (according to your personal preferences and filters) and your distance from them/directions to them. Details regarding private and restricted areas (government/ hypercorp areas, maintenance/security infrastructure, etc.) are usually not included. \item Current habitat life support (climate) conditions including atmosphere composition, temperature. \item Current solar system and habitat orbit maps with trajectory plots, communication delays. \item Local businesses/services, directions, and details. \end{itemize} 

\item \textbf{Local Mesh} \end{itemize} \item Public search engines, databases, mesh sites, blogs, forums, and archives, along with new content alerts. \item Syndicated public newsfeeds in a variety of formats, filtered according to your preferences. \item Sensor/spime (mostly audio-visual) feeds from any public area of the habitat. \item Private network resources (including tactical nets). \item Automatic searches for new online references to your name and other subjects of interest. \item E-tags pertaining to local people, places, or things. \item Facial/image recognition searches of public mesh/archives to match a photo/vid still. \end{itemize} 

\item \textbf{Personal Information} \end{itemize} \item Morph status indicators (medical and/or mechanical): blood pressure, heart rate, temperature, white cell count, nutrient levels, implant status and functionality, etc. \item Location, functionality, sensor feeds, and status reports of your possessions (via sensors and transmitters in these possessions). \item Access to one’s life-spanning personal audio-visual/XP archive. \item Access to one’s life-spanning personal file archive (music, software, media, documents, etc.). \item Credit account status and transactions. \end{itemize} 

\item \textbf{Social Networks} \end{itemize} \item Communications account status: calls, messages, files, etc. \item Reputation score and feedback. \item Social network status, friend updates. \item Updated event calendar and alerts. \item The public social network profiles of those around you. \item The location and status of those nearby and involved in the same AR games as you. \end{itemize} \end{quotation} 



\subsubsection{Avatars} 

Every mesh represents themselves online via a digital avatar. Many people use digital representations of themselves, whereas other prefer more iconic designs. This may be an off-the-shelf look or a customized icon. Libraries of avatars may also be employed, enabling a user to switch their representation according to mood. Avatars are what other users see when they deal with you online—i.e., how you are represented in AR. Most avatars are animated and programmed to reflect the user's actual mood and speech, so that the avatar seems to speak and have emotions. 

\subsubsection{E-Tags} 

Entoptic tags are a way for people to ``tag'' a physical person, place, or object with a piece of virtual data. These e-tags are stored in networks local to the tagged item, and move with the item if it changes location. 

E-tags are viewable in AR, and can hold almost any type of data, though short notes and pictures are the most common. E-tags are often linked to particular social networks or circles within that network, so that people can leave notes, reviews, memorabilia media, and similar things for friends and colleagues. 



\begin{quotation} \textbf{Æther Jabber} 

\# Start Æther Jabber \# 

\# Active Members: 2 \# 

$\Delta$ I have to tell you, after losing Kiri and Sal to that Exsurgent infection, my team is a lot more worried about contracting the virus from digital sources. Actually, I’d label them as paranoid. I don’t think they’ll ever touch any salvaged electronics again unless they’re behind a zillion firewalls and the device is completely isolated and tested by a delta fork loaded with every antiviral ware we can find first. Even then, they’d rather shoot it than access directly or hook it up to an important network. After seeing what the virus did to Sal, I don’t blame them. 

$\Psi$ In our line of work, paranoia can be healthy. 

$\Delta$ Sure, but it’s also a pain in the ass. Security is always a tradeoff. Firewall’s gotta have something up its sleeve that I can pass along to the rest to put their guards at ease. 

$\Psi$ Yes ... and no. It’s complicated. 

$\Delta$ I don’t see why. Do we have a way of detecting and killing this thing or not? 

$\Psi$ Sort of. 

$\Delta$ You’re killing me. 

$\Psi$ Look. Ever since the Fall, we’ve had measures in place to detect and counteract Exsurgent infections and all of the other worms and malware the TITANs concocted. Firewall went to great lengths to make sure that everyone had access to the detection signatures and countermeasures—and we mean everyone. They’ve been incorporated in almost every commercial and open source security software released in the past decade. Every habitat in the system—well, every one with a lick of sense anyway—employs such measures in their chokepoints and mesh infrastructure. 

$\Delta$ I sense a “but.” 

$\Psi$ Yes. The problem is that the Exsurgent virus and similar TITAN infowar worms are adaptive. They’re intelligent. Even though we mostly eradicated them from our networks, new versions periodically pop up, using some new trick to get past the Firewall scans and wreak havoc. Our warning and outbreak response system has it down to a science, and such instances are usually contained. 

$\Delta$ Usually. 

$\Psi$ Well, there’s always the chance that variants are still skipping around out there, under our radar. What’s worse to contemplate, though, is that we may get another major outbreak that spreads to multiple habitats before we can contain it. That might get very, very bad, very, very quickly. \end{quotation} 



\subsubsection{Skinning} 

Since reality can be overlaid with entoptics of hyper-real quality, modern users can ``skin'' their reality by modifying their perceptual input. Environments around them may be modified to fit their particular tastes or mood. Need your spirits boosted? Pull up a skin that makes it seems like you're outdoors, with the sun shining down, the sounds of gentle surf in the background, and butterflies drifting lazily overhead. Pissed off? Be comforted as flames engulf the walls and thunder grumbles ominously in the distance. It is not uncommon for people to go about their day, accompanied by their own personal soundtrack that only they can hear. Even olfactory and taste receptors can be artificially stimulated to experience sensations like the smell of roses, fresh air, or freshly-baked pastries. While originally developed to make ``space food'' less distasteful and as a method to counter space-induced cabin fever for those that weren't born in space, vast archives of aromas, tastes, and environments are available for download. 

Skins do not need to be kept private, they may also be shared with others via the mesh. Tired of your cramped habitat cubicle? Decorate it with a custom-themed skin and share it with visitors to make them feel more comfortable. Found a new music track that livens up your day? Share it with others around you, so they can nod to the same beat. 

Skinning can also be used for the opposite effect. Any undesired content of reality can be edited out, veiled, or censored by modern software programs or muses that engage in real-time editing. Tired of looking at someone's face? Add them to your killfile, and you'll never have to acknowledge their presence again. AR censorware is also common in some communities with strict religious or moral convictions. 

\subsection{Virtual Reality} 

Virtual reality overrides the user's physical senses and places them inside an entirely computer-generated environment called a \textit{simulspace.} While AR is used for all common day activities and interactions, VR is used mainly for recreation (gaming, virtual tourism, escapism), socializing, meeting (when face-to-face meetings are not possible), and training. Dedicated networks with high-capacity information processing are required to render and run large and complex hyper-real simulspaces with many users, and these are often hard-wired for additional stability. Smaller simulspaces capable of hosting a smaller amount of users can be run on a smaller distributed network of linked devices. Many infomorphs and AIs effectively reside within simulspaces, and some transhumans have sworn off the physical world altogether. 



\subsubsection{Defying Nature's Laws} 

A plethora of simulspace environments are available, ranging from simulations of real places to historical recreations to fantastic worlds representing almost every genre imaginable. All of these simulations are bolstered by the fact that possible scenarios are not bound by the laws of nature. The fundamental forces of reality and nature, like gravitation, electromagnetism atmosphere, temperature, etc., are programmable in VR, allowing for environments that are completely unnatural, such as escheresque simulspace where gravity is relative to position. These \textit{domain } \textit{rules} may be altered and manipulated according to the whim of the designer. 

Time itself is an adjustable constant in VR, though deviation from true time has its limits. So far, transhuman designers have achieved time dilation up to 60 times faster or slower than real time (roughly one minute equaling either one hour or one second). Time slowdown is far more commonly used, granting more time for simulspace recreational activities (more time, more fun!), learning, or work (economically effective). Time acceleration, on the other hand, is extremely useful for making long distance travel through space more tolerable. 

\subsubsection{Accessing Simulspaces} 

Most simulspaces can be accessed through the mesh just like any other node. Since VR takes over the user's sensorium, however, and sometimes involves time perception dilation, users are cut off from other mesh-delivered sensory input and interacting directly with other nodes. Instead, outside mesh interactions are routed through the simulspace's interface meaning that a character may browse the mesh, communicate with others, etc. from inside a simulspace, if the domain rules allow it). 

Since physical senses are overridden when a user accesses VR, most people prefer to rest their body in a safe and comfortable environment while in the simulspace. Body-fitting cushions and couches help users relax and keep them from cramping up or injuring themselves if they happen to thrash around. In case of long-term virtual sojourns (for instance, during space travel), morphs are normally retained in tanks that sustain them in terms of nutrition and oxygen. Many VR entertainment and game networks offer dedicated and hardwired physical VR cafes with private pods. Visitors rent a pod and physically jack in, using either access jacks or an ultrasonic trode net that reads and transmits brain patterns when placed on the head. 

When accessing a simulspace, the user first enters an electronic buffer ``holding space'' known as a white room. Here the user chooses a customizable avatar-like persona to represent them in the simul-space called a simulmorph. From this point, the user immerses themself in the virtual reality environment, effectively becoming their simulmorph. 

\subsection{Experience Playback} 

Every morph with mesh inserts has the capability to transmit or record their experiences, a form of technology called experience playback, or XP. Since the first programs were developed that provide a simple interface to ``snapshot'' one's experiences, it has become extremely popular to share XP with friends and social networks, or with the online public at large. The level of experiences depends on how much of the recorded sensory perception is kept when the clip is made. Full XP includes exteroceptive, interoceptive, and emotive tracks. Exteroceptive tracks include the traditional senses of sight, smell, hearing, touch, and taste that process the outside world. Interoceptive tracks include senses originating within the body, such as balance, a sense of motion, pain, hunger and thirst, and a general sense of the location of one's own body parts. Emotive tracks include the whole spectrum of emotions which can be aroused in a transhumans. Due to the biological requirements (neuronal and endocrine systems) of expressing emotions, hardcore XP aficionados deem only the experience in and from biomorphs as the real deal. 

\section{Mesh Uses} 

There are many reasons people use the mesh. The foremost is communication: voice and video calls (typically displaying avatars rather than actual video), electronic messaging (e-mail, instant messaging microblogging), and file and data transfers. Socializing is also key, handled via social and reputation networks, personal profiles, lifelogging, chats and conferences (both AR and VR), and discussion groups and forums. Information gathering is also at the top, whether its browsing the popular Solarchive or other databases and directories, tapping the latest newsfeeds, browsing mesh sites, tracking your friends, taking lessons in VR, or looking up just about anything conceivable. Recreation rounds out the pack, covering everything from gaming (AR and VR) to experiencing other people's lives (XP) to VR tourism and club-hopping. 

\subsection{Personal Area Networks} 

Since everything a person carries is meshed, most people maintain personal area networks that route all of these devices through their mesh inserts or ecto, which acts as a hub. This is both a security measure, ensuring they maintain control over their own accessories and a convenience factor, as it focuses all of the controls in one place. 

\subsection{Virtual Private Networks} 

Virtual private networks (VPNs) are communications networks tunneled through the mesh, which are dedicated for a specific group of people. The primary use of VPNs is to create privacy and security for its users, and so they typically use security 

features such as ego authentication and public key encryption. VPNs are regularly used to mesh mobile offices into a corporate network or mesh people together who work on or contribute a certain project. Other VPNs—particularly social networks and rep networks—operate with minimal security features, simply serving as a network of specific users within the mesh and making it easier to keep in touch, transfer information, make updates, and so on. Most VPNs come as specialized software suites that run custom environmental software that integrates into the user's normal mesh interface and AR. 

\subsection{Social Networks} 

Social networks are the fabric of the mesh, weaving people together. They are the means by which most people keep in contact with their friends, colleagues, and allies, as well as current events, the latest trends, new memes, and other developments in shared interests. They are an exceptionally useful tool for online research, getting favors, and meeting new people. In some cases, they are useful for reaching or mobilizing masses of people (as often illustrated by anarchists and pranksters). There are thousands of social networks, each serving different cultural and professional interests and niches. Most social networks allow users to feature a public profile to the entire mesh and a private profile that only those close to them can access. 

Reputation plays a vital part in social networks, serving as a measure of each person's social capital. Each person's reputation score is available for lookup, along with any commentary posted by people who favored or disfavored them and rebuttals by the user. Many people automate their reputation interactions, instructing their muse to automatically ping someone with a good review after a positive action and to likewise provide negative feedback to people with whom the interaction went poorly. 

\subsection{Mobile Offices} 

Due to the lack of office space and the wireless accessibility of most information, most businesses now operate virtually, with few or no fixed offices or even assets. Instead, individuals have become their own mobile office. Bit-pushers and bureaucrats like hypercorp executives, clerical workers, accountants, and researchers—as well as innovators like artists, writers, engineers, and designers—work wherever they want to. 

The most prominent example of this phenomenon are the bankers of the Solaris hypercorp. Each employee acts as a mobile one-person banking office, managing transactions via Solaris's robust VPN. 

On rare occasions, office environments are run in simulspace with time dilation to maximize efficiency Since this requires the workers to access a centralized wired network and leave their bodies unattended while accessing simulspace, however, it requires an extra level of physical security that is typical only of some governmental installations and corporate habitats. 

\section{Islands In The Net} 

In the time of \textit{Eclipse Phase,} information can become outdated quite fast, and the accessibility of new information depends on your location. It's easy to keep up-to-date on your local habitat/city or planetary body, but keeping current on events elsewhere is typically reliant on the speed of light. 

If you happen to be in a station in the Kuiper Belt, on the edge of the solar system 50 astronomical units from the terrestrial inner planets, waiting on a message from Mars, the signal carrying the message will be roughly seven hours old when it reaches you. Of course it will only reach you that fast if you are using quantum farcast, which is only limited by the speed of light (not to mention rare and expensive in most habitats). If you are not using a quantum farcaster, the signal may take even longer and is prone to interference and noise, deteriorating the quality and possibly losing some of the content, especially over major distances. Whenever you start dealing with communication between habitats, you have to factor in the light-speed lag, the amount of time it takes even the fastest transmission to reach you. This lag works both ways, so trying to hold a conversation with someone just 5 light-seconds away means that you're waiting at least 10 seconds to get the reply to whatever you just said. For this reason, AR and VR communications are almost always conducted locally while standard messaging is used for nonlocal communications. For detailed discussions, it is often simpler to send a fork of yourself (p. 273) to have the conversation and then return. 

Quantum-entanglement communicators (p. 314) are one solution to this light-speed lag, although a burdensome and expensive one. QE comms allow for faster-than-light communication to an entangled communicator, though each transmission uses up a precious amount of quantum-entangled bits, which are in limited supply. 

Transmissions made between habitats almost always occur via each station's massive data relays, where they are then distributed into the local mesh. This bottleneck is often used by authoritarian habitats to monitor data transmissions and even filter or censor certain public non-encrypted content. Some messages are also prioritized over others, potentially meaning further delays. 



The method of transmission between habitats also sometimes matters. Radio and neutrino broadcasts can be intercepted by anyone, whereas tight-beam laser or microwave links are specifically used as a point-to-point method that minimizes interception and eavesdropping. The use of quantum farcasting using neutrino systems is completely secure, however, and is the most frequently-used intra-habitat link. 

What these lags, bottlenecks, and prioritizations mean is that some news and data takes a particularly long-time to trickle from one local mesh network to another, passing slowly from habitat to habitat. This means that there are always gradients of information available to different local mesh networks, typically depending on proximity and the importance of the information. Some data even gets lost along the way, never making it further than a habitat or two before it is lost in the noise. The only way to retrieve such information is to track it down to its source. 

\subsection{Darkcasts} 

``Darkcasts'' are ranged communications that go outside of legal and approved channels. Since certain habitats have strict regulations on transmission content, forking, egocasting, infomorphs, muse abilities, and AGI code, underworld groups like the ID crew profit by offering illegal data transmission services. Primarily used for censored data and banned content (like illegal XPs or malware), local organized crime factions also often offer egocasting services complete with resleeving and leasable morphs, allowing egos that prefer discretion to enter or leave a habitat without drawing attention. Though such authorities hunt down these darkcast networks whenever they get a chance, many habitats have a sophisticated darkcast infrastructure that makes use of decoys, temporary communications lines, relays, and regular transmitter relocation—not to mention judicious bribing and blackmailing. 

\section{Mesh Abuses} 

As with all things, the mesh has its darker side. At the basic level, this amounts to flamewar-starting trolls, stalkers, or griefers whose intent is to mess with others for a laugh. At the more organized level, it expands to illicit or criminal enterprises that utilize the mesh, such as selling black/snuff/porn XPs, illegal software, pirated media, or even egos. The most infamous threats—thanks both to the Fall and to the continuous sensationalism applied by media and stern authorities—are, of course, malware and hackers Given the capabilities of modern hackers and the vulnerability of many habitats—where damage to life support systems could kill thousands—the threat may not be over-exaggerated. 

\subsection{Hackers} 

Whether individuals who are genuinely interested in exploring new technologies and seeking ways to break them in order to make them better, hacktivists who utilize the mesh in order to undermine the power of authorities, or ``black hats'' who seek to circumvent network security for malicious or criminal intent, hackers are a permanent fixture of the mesh. Unauthorized network break-ins, infiltration of VPNs, muse subversion, cyberbrain hijacking, data theft, cyber-extortion, identity fraud, denial of service attacks, electronic warfare, spime hijacking, entoptic vandalism—these are all common occurrences on the mesh. Thanks to smart and adaptive exploit programs and assisting muses, even a moderately skilled hacker can be a threat. 

In order to counter hacking attempts, most people, devices, and networks are protected by a mix of access control routines, automated software intrusion prevention systems, encryption, and layered firewalls, typically overseen by the user's muse who plays the role of active defender. Extremely sensitive 

systems—such as space traffic control, life support, power systems, and hypercorporate research facilities—are usually limited to isolated, tightly-controlled, heavily-monitored, hard-wired networks to minimize the risk of intrusion from snoopers and saboteurs. Various countermeasures may be applied against such intruders, ranging from locking them out of the system to tracking them back and counterhacking. 

\subsection{Malware} 

The number of worms, virii, and other malware programs that ripped through computer systems during the Fall was staggering. Many of these were part of the netwar systems prepared by old Earth nation-states and corporations and unleashed on their enemies. Others were products of the TITANs, subversive programs that even the best defenses had trouble stopping. Even 10 years later, many of these are still reappearing, brought back to life by the accessing of some long-forgotten data cache or the accidental infection of a scavenger mucking through old ruins. New ones pop every day, of course, many of them programmed by criminal hacker outfits, while others that enter circulation are modifications and variations of suspected TITAN designs, perhaps implying that certain parties are intentionally tinkering with this code and releasing it into the wild. Rumors and whispers circulate that some of these TITAN worms are even more potent and frightening than previously hinted at, with stunning adaptive capabilities and intelligence. These rumors are steadfastly denied by authority figures and security experts ... who then quietly turn around and do their best to ensure that their own networks remain safe. 

\section{AI And Infolife} 

Self-aware helper programs were originally designed and realized to augment transhuman cognitive abilities These specialized-focus AIs were then developed into the more complete, independent digital consciousnesses known as AGIs. The further evolution of these digital life forms into seed AIs unfortunately led to the emergence of the TITANs and then the Fall. This created a rift in transhuman society as fear and prejudice turned popular opinion against unrestricted AGIs, an attitude of mistrust that still lingers to this day. 

\subsection{AI} 

The term AI is used to refer to narrow, limited-focus AIs. These digital minds are expert programs with processing capabilities equal to or even exceeding that of a transhuman mind. Though they have a personality matrix with individual identities and character, and though they are (usually) conscious and self-aware, their overall complexity and capabilities are limited. The programmed skills and abilities of AIs are typically very specific in scope and aligned towards a particular function, such as piloting a vehicle, facilitating mesh searches, or coordinating the functions of some habitat sub-system. Some AIs, in fact, can only barely be considered sapient, and their emotional programming is usually narrow or non-existent. 

AIs have a number of built-in safety features and programmed limitations. They must serve and obey the instructions of authorized users within their normal functioning parameters and (in the inner system at least) must also obey the law. They lack self-interest and self-initiative, though they have limited empathy and may be programmed to anticipate the needs and desires of users and pre-emptively take action on their behalf. Perhaps most importantly, however, is that their psychological programming is specifically based on universal human modes of thought and an understanding and support of transhuman goals and interests. This is part of an initiative to engineer so-called ``friendly AIs,'' who are programmed with sympathy towards transhumanity and all life and seek out their best interests. 

In most societies, basic AIs are considered ``things'' or property rather than people and are accorded no special rights. 

\subsection{Muses} 

Muses are a specific type of AI designed to function as a personal aide and companion. Most people in \textit{Eclipse Phase} have grown up with a muse at their virtual side. Muses tend to have a bit more personality and psychological programming than standard AIs and over time they build up an extensive database of their user's preferences, likes and dislikes, and personality quirks so that they may more effectively be of service and anticipate needs. Muses generally have names and reside within the character's mesh inserts or ecto, where they can manage the character's personal area network, communications, data queries, and so on. 

\end{itemize} \textbf{What Your Muse Can Do For You} \end{quotation} \item Make Research Tests to find information for you. \item Scan newsfeeds and mesh updates for keyword alerts. \item Monitor your mesh inserts/ecto/PAN and slaved devices for intrusion. \item Launch countermeasures against intruders. \item Teleoperate and command robots. \item Monitor your Rep score and alert you to drastic changes. \item Automatically provide feedback for other people’s Rep scores. \item Run audio input through an online, real-time language translation system. \item Put you in privacy mode and/or proactively stealth your wireless signal. \item Falsify/fluctuate your mesh ID. \item Track people for you. \item Anticipate your needs and act accordingly, pre-empting your requests. \end{itemize} \end{quotation} 

\subsection{AGI} 

AGIs are complete and fully operational digital consciousnesses self-aware and capable of intelligent action at the same level as any transhuman. Most have full autonomy and the capacity for self-improvement by a processing similar to learning—a slow optimization and expansion of their code that features programmed limitations to prevent it from achieving the self-upgrading capabilities of seed AIs. They have more fully-rounded personalities and emotional/ empathic abilities than standard AIs, due in part to a development process where they are literally raised within a VR simulation analogous to the rearing of transhuman children, and so are more fully socialized. As a result, they have a fairly human outlook, though some deviation is to be expected, and sometimes is apparent in great degrees. Despite this attempt to humanize AGIs, they do not have the same evolutionary and biological origins that transhumans have, and so their social responses, behavior, and goals are sometimes off-mark or decidedly different. 

AGIs bear the social stigma of their non-biological origin and are often met with bias and mistrust. Some habitats have even outlawed AGIs or subject them to 

strict restrictions, forcing such infolife to hide their true natures or illegally darkcast to enter habitats or stations. AGI mind programming emulates transhuman brain patterns sufficiently well that they can be sleeved into biomorphs if they choose. 

\begin{quotation} \textbf{Non-Standard AIs And AGIs} 

Not all AIs and AGIs were programmed and designed to adhere to human modes of thought and interests. Such creations are illegal in some jurisdictions, as they are considered a potential threat. Several hypercorps and other groups have initiated research into this field, however, with varying results. In some cases these digital minds are so different from human mindsets that communication is impossible. In others, enough crossover exists to allow limited communication, but such entities are invariably quite strange. Rumors persist that some AIs began their life as transhuman backups or forks, who were then heavily edited and pruned down to AI-level intelligences. \end{quotation} 

\subsection{Seed AI} 

Due to the capability for unlimited self-upgrading, seed AIs have the capacity to grow into god-like digital entities far beyond the level of transhumans or AGIs. They require massive processing power and are always increasing in complexity due to a continual metamorphosis of their code. Seed AIs are too complex to be downloaded into a physical morph, even a synthetic one. Even their forks require impressive processing environments, so doing so is rare. In fact, most seed AIs require the capacities of hard-wired networks to survive. 

The only seed AIs known to the public are the infamous TITANs who are widely regarded as being responsible for the Fall. In truth, the TITANs were not the first seed AIs and will probably not be the last. There are no publicly known TITANs (or other seed AIs) currently residing in the solar system, despite circulating rumors of damaged TITANs who were left behind on Earth, speculated TITAN activity under the clouds of Venus, or whispers of new seed AIs hidden away in secret networks on the edges of the system. 

\subsection{Transhuman Infomorphs} 

For thousands of infugees, embodying a digital form is their only choice. Some of these are locked away in mesh-separated virtual holding areas or even inactive storage, locked up by habitats who didn't have enough resources to handle them. Others are imprisoned inside simulspaces, killing time in whatever way they choose until an opportunity to resleeve comes their way. Quite a few are free to roam the mesh, interacting with physically-sleeved transhumans, keeping up with current events, and sometimes even forming activist political blocs to campaign for infomorph rights or interests. Still others find or are forced into virtual careers, slaving away in the digital sweatshops of hypercorps or criminal syndicates. A few find companions who are willing to bring them along in their ghostrider module and become an integral part of their lives, much like a muse. Some transhumans willingly choose the infomorph lifestyle, either for hedonism (custom simulspace and VR games until the end of time), escapism (loss of a loved ones leads them to write off physical concerns for awhile), freedom (go anywhere the mesh takes you—some have even beamed copies of themselves to far distant solar systems, hoping someone or something will receive their signal when they arrive), experimentation (forking and merging, running simulations, and weirder things), or because it is ensured immortality. 

\section{Everyday Mesh Mechanics} 

Everyone (and everything) is meshed in \textit{Eclipse Phase. } The following rules and concerns apply to standard mesh use. Note that various mesh-related terms are explained, along with other \textit{Eclipse Phase }concepts, under \textit{Terminology,} p. 25. 

\subsection{Mesh Interface} 

Characters have a choice of which interface to use, the entoptic interface of basic mesh inserts or the haptic interface of an ecto. The basic mesh inserts used by most users allows them to interact with AR, VR, XP, and the mesh at the 

speed of thought. This is the default method of mesh use and suffers no modifiers. They are, however, more prone to visual and operative impairments (virtual illusions denial-of-service effects) when hacked. 

Characters who use the haptic interface of an ecto, however, suffer a slight delay on their mesh activities due to manual toggling, physical controls, and physical interaction with virtual controls. In game terms, the use of haptics imposes a –10 skill modifier to all mesh tests where timing is rushed (particularly combat and or any sort of mesh use under pressure). Additionally, increase the timeframe for mesh-based Task Actions by +25\% when interfacing via haptics. On the positive side, ectos can be easily removed and discarded if compromised—for this reason, many hackers and security-conscious users use an ecto in addition to their mesh inserts, routing all high-risk traffic through the ecto as an extra line of defense. 

\subsection{Mesh ID} 

Every mesh user (and, in fact, every device) has a unique code called their \textit{mesh ID.} This ID distinguishes them from all other users and devices, and is the mechanism by which others can find them online, like a combination phone number, email address, and screen name. Mesh IDs are used in almost all online interactions which are often logged, meaning that your activities online leave a datatrail that can be tracked (p. 251). Fortunately for Firewall sentinels and others who value their privacy, there are ways around this (see \textit{Privacy and Anonymity,} p. 252). AIs, AGIs, and infomorphs also each have their own unique mesh ID. 

\subsection{Accounts And Access Privileges} 

Devices, networks (such as PANs, VPNs, and hardwired networks), and services require that every user that accesses them does so with an account. The account serves to identify that particular user, is linked to their mesh ID, and determines what access privileges they have on that system. There are four types of accounts: \textit{public, user, security,} and \textit{admin.} 

\begin{quotation} \textbf{Elite Exploits} 

The mesh gear quality rules allow for players and gamemasters to make a distinction between software tools, separating the open-source, stock-repertoire exploit tools of amateur hackers from cutting-edge military-grade penetration wares. While many characters will simply buy or otherwise acquire such programs, a hacker with the do-it-yourself ethic is likely going to want to design their own personalized applications, based on their playbook of closely-guarded intrusion/ counterintrusion methods. 

To reflect the efforts a hacker character makes by designing, coding, and modifying their own customized personal arsenals, they may make a Task Action Programming Test with timeframe of 2 weeks. If they succeed, they upgrade one of their software tools by one level of quality (i.e., from +0 to +10). Multiple Programming Tests can be made to enhance a program, but for each level add the target modifier as a negative modifier to the test (so upgrading a +0 suite to +10 is a –10 modifier on the Programming Test). 

Similarly, at the gamemaster’s discretion, software tools—particularly exploits—may degrade in quality over time, reflecting that they have become outdated. As a general rule, such programs should degrade in quality about once every 3 months. \end{quotation} 

\subsubsection{Public Accounts} 

Public accounts are used for systems that allow access (or access to parts of their system) to anyone 

on the mesh. Public accounts do not require any sort of authentication or login process, the user's mesh ID is enough. These accounts are used to provide access to any sort of data that is considered public: mesh sites, forums, public archives, open databases, social network profiles, etc. Public accounts usually have the ability to read and download data, and sometimes to write data (forum comments, for example), but little else. 

\subsubsection{User Accounts} 

User accounts are the most common accounts. User accounts require some form of authentication (p. 253) to access the device, network, or service. Each user account has specific access privileges allotted to it, which are tasks the user is allowed to perform on that system. For example, most users are allowed to upload and download data, change basic content, and use the standard features of the system in question. They are not, however, usually allowed to alter security features, add new accounts, or do anything that might impact the security or functioning of the system. As some systems are more restrictive than others, the gamemaster decides what privileges each user account provides. 

\subsubsection{Security Accounts} 

Security accounts are intended for users that need greater rights and privileges than standard users, but who don't need control over the entire system, such as security hackers and muses. Security access rights usually allow for reading logs, commanding security features, adding/deleting accounts, altering the data of other users, and so on. 

\subsubsection{Admin Accounts} 

Admin accounts provide complete control over the system. Characters with admin rights can do everything security accounts can, plus they can shut down/ reboot the system, alter access rights of other users, view and edit all log files and statistics, and stop or start any software available on the system. 

\subsection{Mesh Gear Quality} 

Not all gear is created equal, and this is especially true of computers and software, where new innovations are made on a daily basis. Keeping up-to-date with the last specs isn't too difficult, but on occasion the characters will get their hands on some old relic or find themselves in secluded or decrepit places with local systems and gear that aren't up to date. Likewise, they may acquire some cutting-edge gear straight from the labs or may run up against a state-of-the-art installation with next-generation defenses. To reflect this, mesh tests can be modified according to the state of the hardware or software being used, as noted on the Mesh Gear Modifiers table. 

\begin{table} \begin{tabular}{|r|l|} \hline

\hline{2}{|c|}{\textbf{Mesh Gear Modifiers}} \\ \hline

\textbf{Modifier} &\textbf{Software/Hardware} \\ \hline

–30 &Bashed-up devices, no-longer-supported software, relics from Earth or the early expansion into space \\ \hline

–20 &Malfunctioning/inferior devices, buggy software, pre-Fall technology \\ \hline

–10 &Outdated and low quality systems \\ \hline

0 &Standard ectos, mesh inserts, and software \\ \hline

+10 &High-quality goods, standard security-grade products \\ \hline

+20 &Next-generation devices, advanced software \\ \hline

+30 &Newly-developed, state-of-the-art, top-of-the-line technology \\ \hline

$>$+30 &TITANs and/or alien technology \\ \hline

\label{tab:mesh-gear-modifiers} \label{tab:mesh-gear-modifiers} \end{table} 

\subsection{Computer Capabilities} 

Computerized electronics can be broken down into three simple categories: \textit{peripherals, personal comput-} \textit{ers, }and \textit{servers.} In game terms, all are collectively referred to as \textit{devices.} 

\subsubsection{Peripherals} 

Peripherals are micro-computerized devices that don't need the full processing power and storage capacity of a personal computer, but benefit from online networking and other computerized functions. Peripherals may run software, but the gamemaster may decide that too many programs (10+) will degrade the system's performance AIs and infomorphs are incapable of running on peripherals, though they may access them. Peripherals only have public and user accounts (users accounts include security and admin functions; see p. 247). 

Common peripherals include: spimes, appliances, most cybernetic implants, individual sensors, and weapons. 

\subsubsection{Personal Computers} 

Personal computers account for a wide range of computer types, but essentially account for anything that has the capabilities evolved from generations and generations of personal computers to meet an everyday user's needs. Most personal computers are portable and tailored for use by multiple users at a time. Personal computers may run one AI or infomorph at a time. They may not run simulspace programs. 

Common personal computers include: mesh inserts, ectos, and vehicles. 

\subsubsection{Servers} 

Servers have much greater processing power and data management capabilities than personal computers They are capable of handling hundreds of users, multiple AIs and infomorphs, and they may run 

simulspace programs. Though few are portable, some of them come close. 

\subsection{Software} 

A wide manner of software is available for mesh users, from firewalls and AIs to hacking and encryption tools to tacnets and skillsofts. These programs are listed on p. 331 of the \textit{Gear }chapter. Like other gear, software may enable a character to perform a task they could not otherwise. The quality of the software may also be a factor, applying a modifier as appropriate (see \textit{Mesh } \textit{Gear Quality,} p. 247). 

Some software is equipped with digital restrictions to prevent it from being copied and shared with others. These restrictions may be defeated, but it is a time-consuming task, requiring a Task Action Programming Test with a timeframe of 2 months. Thanks to the efforts of the open source movement and numerous individual software pirates, however, quite a bit of software is available free or online. The availability of pirated software or freeware will depend on the local habitat and legalities. Finding it may be a matter of a simple search, or it may require some use of reputation to find someone who has it. Usually there is at least one local crime syndicate that is willing to help you out—for a price. 

\subsubsection{Software Compatibility} 

In most instances, software compatibility is not going to be an issue for characters. Gamemasters who like it as a plot device, however, can introduce compatibility problems in certain cases, whether this is done to increase drama, slow the characters down, or create obstacles that they must overcome. Such incompatibilities are more likely to arise when dealing with outdated systems or devices, or at least ones that are unlikely to have the latest patches and software updates Incompatibilities can also be used as a drawback to acquiring software from untrustworthy sources. 

Conflicting software issues are going to have one of two effects. Either the software will simply not work with certain devices, or it will inflict a –10 to –30 modifier due to instabilities and glitches. If the gamemaster allows it, a character may reduce this penalty by patching up the software, requiring a successful Programming Task Action (1 day). For every 10 points of MoS, reduce the incompatibility modifier by 10. 

\subsection{Traffic Filters And Mist} 

Mesh networks and AR are overrun with yottabytes of information. Though mesh inserts and ectos can deal with a lot of data traffic in terms of bandwidth and processing power, using filters to weed out unwanted traffic is quite simply a necessity. This is especially true of AR, where unwanted entoptics can clutter your vision and distract you. Nevertheless, entoptic spam of many flavors—advertisements, political screeds, porn, scams—do their best to bypass these filters, and in many areas the amount of unfiltered entoptics can be overwhelming—a phenomenon colloquially referred to as ``mist.'' 

At the gamemaster's discretion, mist can interfere with a user's sensory perceptions. This modifier can range from –10 to –30, and in some cases might be so distracting as to affect all of a character's actions. To lift the data fog, a character or muse must adjust their filter settings by succeeding in an Interfacing Test modified by the mist modifier. Alternately, the character can completely disable AR input, but this is likely to impede them in other ways. 

\subsection{Slaving Devices} 



For ease of use, as well as for privacy and security purposes, one or more devices may be slaved to each other. One device (usually the character's mesh inserts or ecto) takes the role of \textit{master,} while the other device(s) takes the role of \textit{slave.} All traffic to and from slaved devices is routed through the master. This allows a slaved device to rely on the master's security features and active monitoring. Anyone that wants to connect to or hacked into a slaved device is rerouted to the master for authentication and security scrutiny. Slaved devices automatically accept commands from their master device. This means that a hacker who penetrates a master system can legitimately access and issue commands to a slaved device, assuming their access privileges allow for it. 

PANs are typically formed by slaving all of a character's devices to their ecto or mesh inserts. Similarly, individual components of a security system (doors, sensors, etc.) are usually slaved to a central security node that serves as a chokepoint for anyone hoping to hack the system. The same is often true for other networks and facilities. 

\subsection{Issuing Commands} 

Characters may issue commands to any slaved device or teleoperated bot (see \textit{Shell Remote Control,} p. 196) with a Quick Action. Each command counts separately, unless the character is issuing the same command to multiple devices/drones. 

\subsection{Distance Lag} 

Anytime you extend your communications over great distances, you run into the risk of time lags. Most communications are restricted to ``local'' for this reason, which generally means your local habitat and any others within 50,000 kilometers. On planetary bodies like Mars, Venus, Luna, or Titan, ``local'' encompasses all of the habitats and linked mesh networks on that planetary body. 

If a character is searching the mesh beyond their local area, the most efficient way is to transmit a search AI (usually a copy of your muse) or a fork to the non-local area, which will then run its search and return completed results. This process does, however, add to the time of transmission to the timeframe (i.e., searching the mesh of a station 10 light-minutes away adds 20 minutes to the search as the search is transmitted over and the results are transmitted back). Since long-distance communications are sometimes interfered with 

or bumped for higher-priority items, the gamemaster can increase this time at their discretion. If the research involves correlation and fine-tuning the search parameters based on data accumulated from different locals, the timeframe may be exponentially increased due to the need for back-and-forth interaction. 

If the character is simply communicating with or accessing non-local networks, an appropriate time lag must be introduced between communications and actions The effects of this lag are largely up to the gamemaster as fitting current distances and other factors. 

\subsection{Accessing Multiple Devices} 

Meshed characters may connect to and interact with numerous devices, networks, and services simultaneously There is no penalty for doing this, but the character may only focus on one system at a time. In other words, you may only interact with one system at a time, though you may also switch between them freely, even within the same Action Phase. You could, for example, spend several Quick Actions to send a message with your ecto, tell your spime oven at home to start cooking dinner, and look up a friend's updated profile on a social network. You may not, however, hack into two separate systems simultaneously. 

Note that you may send the same command to multiple slaved devices or teleoperated drones with the same Quick Action, as noted above. 

\section{Online Research} 

The Research skill (p. 184) represents a character's ability to track down information in the mesh. Such information includes any type of digitized data: text, pictures, vids, XP, sensor feeds, raw data, software, etc. This data is culled from all manner of sources: blogs, archives, databases, directories, social networks, rep networks, online services, forums, chat rooms, torrent caches, and regular mesh sites of all kinds. Research is conducted using various public and private search engines, both general and specialized, as well as data indices and search AIs. 

Research has other uses as well. Hackers use it when looking for specific information on a network or device on which they have intruded. Likewise, since everyone inevitably uses and interacts with the mesh, Research skill is also a way to identify, backtrack, and/ or gather information on people as long as they have not hidden their identity, worked anonymously, or covered their identity with a shroud of disinformation. 

\subsection{Search Challenges} 

Due to the sheer amount of data available, finding what you're looking for may sometimes seem a daunting task. Thankfully, information is fairly well organized, thanks to the hard work of ``spider'' AIs that cruise the mesh and constantly update data and search indices. Additionally, information on the mesh is tagged with semantics, meaning that it's presented with code that allows a machine to understand the \textit{context }of that information as well as a human reader would. This helps AIs and search software correlate data more efficiently. So finding the data is usually not as difficult as analyzing it and understanding it. Finding specialized or hidden info or correlating data from multiple sources is usually the real challenge. 

Perhaps a larger issue is the amount of incorrect data and misinformation online. Some data is simply wrong (mistakes happen) or outdated, but the nature of the mesh means that such things can linger on for years and even propagate far and wide as they are circulated without fact-checking. Likewise, given 

the amount of transparency in modern society, some entities actively engage in disinformation spreading in order to clutter the mesh with enough falsehoods that the truth is hidden. Two factors help to combat this, the first being that data sources themselves have their own reputation scores, so that untrustworthy or disreputable sources can be identified and ranked lower in search results. Second, many archives take advantage of crowdsourcing—that is, harnessing the collaborative power of mesh users (and their muses) everywhere—to verify data integrity so that these sites are dynamic and self-correcting. 

\begin{quotation} \textbf{Search Capabilities} \\ Online research in Eclipse Phase is not the same as just googling something. Here are five ways in which it is vastly improved: 

\textbf{Pattern Recognition}: Biometrics and other forms of pattern recognition are efficient and intelligent. It is not only possible to run image recognition searches (in real-time, via all available spimes and sensor feeds), but to search for patterns such as gait, sounds, colors, emotive displays, traffic, crowd movement, etc. Kinesics and behavioral analysis even allow sensor searches for people exhibiting certain behavioral patterns, such as suspicious loitering, nervousness, or agitation. 

\textbf{Metadata}: Information and files online come with hidden data about their creation, alteration, and access. A photo’s metadata, for example, will note what gear it was taken on, who took it, when, and where, as well as who accessed it online, though such metadata may be easily scrubbed or anonymized. 

\textbf{Data Mash-Ups}: The combination of abundant computing, archived data, and ubiquitous public sensors enable intriguing correlations to be drawn from data that is mined and collated. In the midst of a habitat emergency such as a terrorist bombing, for example, the ID of everyone in the vicinity could be scanned, compared to data archives to separate out those who have a history of being in the vicinity at that particular time, with those remaining checked against databases of criminal/suspect history and their recorded actions analyzed for unusual behavior. 

\textbf{Translation}: Real-time translation of audio and video is available from open source translation bots. 

\textbf{Forecasting}: A significant percentage of what people do on any day or in response to certain situations conforms to routines, enabling easy behavioral prediction. Muses take advantage of this to anticipate needs and provide whatever is desired at the right moment and in the right context. The same logic applies to actions by groups of people, such as economics and social discourse, making predictions markets a big deal in the inner system. \end{quotation} 

\subsection{Handling Searches} 

Online research is often a crucial element of \textit{Eclipse } \textit{Phase} scenarios, as characters take to the mesh to research backgrounds and uncover clues. The following suggestions present a method of handling research that does not rely solely on dice rolls and integrates it with the flow of the plot. 

First, common and inconsequential information should be immediately available without requiring a roll at all. Most characters rely on their muses to handle searches for them, passing on the results while the character focuses on other things. 

For searches that are more detailed, difficult, or central to the plot, a Research Test should be called for (made either by the character or their muse). This test indicates the process of finding links to and/or accumulating all data that may in fact be relevant to the search topic. This test should be modified as appropriate to the obscurity of the topic, ranging from +30 for common and public topics to –30 for obscure or unusual intel. This initial search has a timeframe of 1 minute. If successful, it turns up enough data to give the character a basic overview, perhaps with cursory details. The gamemaster should use the MoS to determine the depth of the data provided on this initial excursion, with an Excellent Success providing some bonus details. Similarly, a Severe Failure (MoF 30+) may result in the character working with data that is incorrect or intentionally misleading. 

The next step is not so much accumulating links and data as it is analyzing and understanding the information acquired. This requires another Research Test, again modified by the obscurity of the topic. If the gamemaster allows it, complementary skills (p. 173) may apply to this test, providing bonus modifiers (for example, an understanding of Academics Chemistry would help research the effects of a strange drug). Muses may also perform this task, though their skills are typically inferior. As above, success determines the quality and depth of the analysis, with an Excellent Success providing the full story and potential related issues and a Severe Failure marking completely incorrect assumptions. The timeframe for this phase of research largely depends on two factors: the amount of data being analyzed and the importance to the storyline. Gamemasters 

need to carefully measure out their distribution of intel and clues to players. Give them too much too soon, and they may spoil the plot. Fail to give them enough, and they may get frustrated or pursue dead ends. Timing is everything. 

\subsection{Real-Time Searches} 

Characters may also set up ongoing mesh scans that will alert them if any relevant information comes up new or updated, or is somehow changed. This is a task usually assigned to muses for continuous oversight. Whenever such data arises, the gamemaster secretly makes a Research Test, modified by the obscurity of the topic. If successful, the update is noted. If not, it is missed, though the gamemaster may allow another test if and when the topic reaches a wider range of circulation or interest. 

\subsection{Hidden Data} 

It is important to remember that not everything can be found online. Some data may only be acquired (or may be more easily gotten) by asking the right people (see \textit{Networking,} p. 286). Information that is considered private, secret, or proprietary will likely be stored away behind VPN firewalls, in off-mesh hardwired networks, or in private and commercial archives. This would require the character to gain access to such networks in order to get the data they need (assuming they even know where to look). 

It's worth noting that some entities send out AIs into the mesh with the intent of finding and erasing data they'd rather hide, even if this requires hacking into systems to alter such information. 

\section{Scanning, Tracking, And Monitoring} 

Most users leave traces of their physical and digital presence all throughout the mesh. Accounts they access, devices with which they interact, services they use, entoptics they perceive—all of these keep logs of the event, and some of these records are public. Simply passing nearby some devices is enough to leave a trail, as near-field radio interactions are often logged. This electronic datatrail can be used to track a user, both to ascertain their physical location or to note their online activities. 

\subsection{Wireless Scanning} 

To interface with a wireless device or network, whether to establish a connection or for other purposes, the target device/network must be located first. To locate an active node, it's wireless radio transmissions must be detected. Most wireless devices automatically scan for other devices in range (see \textit{Radio and Sensor Ranges,} p. 299) as a matter of course, so no test is required. This means that it's trivial for any character to pull up a list of the wireless devices and networks around them, along with associated mesh IDs. Likewise, a muse or device can be instructed to alert the user when a new signal (or a specific user) comes into range. Detecting stealth signals (p. 252), however, is a bit more challenging. To detect a stealthed signal, the scanning party must actively search for such signals, taking a Complex Action and making an Interfacing Test with a –30 modifier. If successful, they detect the hidden emissions If the character aiming for stealth engages in active countermeasures, also requiring a Complex Action, then an Opposed Interfacing Test is called for (with the –30 modifier still applying to the scanning party). For covert devices that are only transmitting in short bursts, wireless detection is only possible during the short period the burst transmission is being made. 

\subsection{Physical Tracking} 

Many users willingly allow themselves to be physically tracked via the mesh. To them, this is a useful feature— it allows their friends to find them, their loved ones to know where they are, and for authorities to come to their aid in the event of some emergency. Finding their location is simply a matter of looking them up in the local directory, no test required (assuming you know who they are). Mesh positioning is accurate to within 5 meters. Once located, the position of the target can be monitored as they move as long as they maintain an active wireless connection to the mesh. 

\subsubsection{Tracking By Mesh ID} 

An unknown user's physical location can also be tracked via their online mesh activity—or more specifically by their mesh ID (p. 246). Network security will often trace intruders this way and then dispatch security squads to bring them in. To track an unknown user by their mesh ID alone requires a Research Test. If successful, they have been tracked to their current physical location (if still online) or last point of interaction with the mesh. If the character is in privacy mode (p. 252), a –30 modifier applies. 

\subsubsection{Tracking By Biometrics} 

Given the existence of so many spimes and public cameras and sensors, people may also be tracked by their facial profile alone using facial recognition software This software scans accessible video feeds and attempts to match it to a photo of the target. Given the sheer volume of cameras, however, and the typical range of false-positives and false-negatives, finding the target often boils down to luck. Priority can be given to cameras monitoring major thoroughfares, to narrow the search, but this risks missing the target if they avoid heavy traffic areas. The success of searches of this nature is best left to gamemaster fiat, but a Research Test can also be called for, modified appropriately by the range of the area being watched, whenever there is a chance the target may be spotted. Other biometric signatures may also be used for tracking this way, though these are usually less available than cameras: thermal signatures (requires infrared cameras), walking gait, scent (requires olfactory sensors), DNA (requires DNA scanners), etc. Each biometric scan requires a separate type of software. 



\subsection{Digital Activity Tracking} 

Tracking someone's online activities (meshbrowsing, entoptic interactions, use of services, messaging, etc.) is slightly more difficult, depending on what exactly you're after. Gathering information on a user's public mesh activities—social network profiles, public forums posts, public lifelogging, etc.—is handled just like standard online research (p. 249). 

\subsubsection{Tracking By Mesh ID} 

A more investigate search can attempt to use the target's mesh ID (p. 246), using it as a sort of digital fingerprint to look up where else they've been online. This primarily involves checking access/transaction logs, which are not always publicly accessible. This sort of search requires a Research Test, handled as a Task Action with a timeframe of 1 hour. 

\subsection{Sniffing} 

Wireless radio traffic is broadcast through the air (or space), meaning that it can be intercepted by other wireless devices. ``Sniffing'' involves the capture and analysis of data traffic flowing through the wireless mesh. 

To eavesdrop on wireless communications, you need a sniffer program (p. 331) and you must be within radio range (p. 299) of the target (alternately, you can access a device that is within radio range of the target, and sniff from that location). To capture the information you must succeed in an Infosec Test. If successful, you capture data traffic from any targeted devices in range. Note that sniffing does not work on encrypted traffic (including VPNs and anything else using public key cryptography) as the results are gibberish. Quantum encrypted communications cannot be sniffed. 

Once you have the data, finding the information you're looking for can be a challenge. Handle this as a standard Research Test (p. 245). 

\subsubsection{Remote Sniffing Via Mesh ID} 

Finally, a mesh ID may also be actively monitored to see what mesh activity it engages in. This requires special sniffer software (p. 331) and a Research Test. If successful, the monitoring will provide information on that user's public mesh activities (how much is determined by the gamemaster and the MoS), such as which sites they access, who they message, etc. It will not, however, uncover anything that is encrypted (unless the encryption is broken) or anything that takes place on a VPN (unless the VPN is hacked first), though it will show that encrypted communications and/or VPN use are taking place. 

\section{Privacy And Anonymization} 

Given how easily mesh activities are monitored, many users pursue privacy and anonymization options. 

\subsection{Privacy Mode} 

Characters who go into privacy mode hide their online presence and activities from others to a limited degree. The exact settings are adjustable, but typically involve masking their social profiles and presence to other users in the immediate vicinity, like having an unlisted phone number. Privacy mode can also be used to limit the use of mesh IDs and other data in access and transaction logs, applying a –30 modifier to attempts to research or track them by their online activity. 

\subsubsection{Stealthed Signals} 

Another tactic that can be taken for privacy is to stealth the wireless radio signals you emit. This method uses a combination of spread-spectrum signals, frequency hopping, and modulation to make your radio transmissions harder to detect with scanning (p. 251). Stealthing your signals is either a passive activity (Automatic Action, –30 modifier on Interface Tests to locate the signal) or an active one (Complex Action, requires an Opposed Test to locate). 

\subsection{Anonymization} 

Anonymization takes the issue of privacy a bit further. The user does not just hide their mesh ID, but they actively use false mesh IDs and take other measures to reroute and obfuscate their datatrail. Anonymization is a necessity both for clandestine operatives and those engaging in illicit mesh activities. 

\subsubsection{False Mesh IDs} 

The easiest method of making mesh activities anonymous is to set your muse to supply false mesh IDs in online transactions. Though illegal in many jurisdictions this is an easy task for any character or muse to do. Multiple false IDs are used, making it extremely difficult for anyone to tie all of the user's activities together. 

This method makes it extremely difficult for anyone to track the user's online actions. Someone attempting to track the character via these false mesh IDs must beat them in an Opposed Test, pitting their Research skill with a –30 modifier against the character's (or more likely, their muse's) Infosec skill. This is a Task Action with a base timeframe of 1 hour, adjusted higher according to the amount of activity they hope to track. If successful, the tracker manages to dig together enough correlating evidence and records of false IDs to get a picture of the character's activities (how thorough this picture is depends on their MoS). If the fail, the anonymous character has effectively camouflaged themselves in the mesh. 

Actively monitoring a character who is fluctuating their mesh ID with a sniffer program, or physically tracking them via the mesh, is next to impossible, as the continual shifting of IDs and intentional decoys make it too difficult to keep up. 

\subsubsection{Anonymous Account Services} 

A number of people—not just criminals, hackers, and secret agents—have an interest in keeping some of their affairs anonymous. To meet this demand, various online service vendors offer anonymous accounts for messaging and credit transfers. Some of these vendors 

are legit business (in places where it is legal), some are criminals operating illegally, others are hacktivists promoting the privacy meme, and still others are hypercorps or other organizations offering such services internally to their own staff/membership. 

The interaction between the vendor and user is encrypted and anonymous, with no logs kept, so even if the vendor's servers are hacked, an intruder will not find any leads. While some anonymous accounts are established for regular use, the truly paranoid use (multiple) one-time accounts for maximum security. One-time accounts are used for a single message incoming or outgoing) or credit transaction, and then are securely erased. 

Tracking an anonymous account is a practical impossibility, and something that only an extremely resourceful organization employing a systematic and expensive effort could attempt. 

\subsubsection{Disposable Ectos} 

Another option for those seeking privacy and security is to simply use disposable ectos. Using this method, all activity is routed through a specific ecto (using its mesh ID), the ecto is used for a limited period (until it gets hot), and then it is simply discarded or destroyed. 

\section{Mesh Security} 

Given the lessons of the Fall and the very real risk still posed by hackers, virii, and similar threats, network security is taken extremely seriously in \textit{Eclipse Phase. } Four methods are typically used: authentication, firewalls active monitoring, and encryption. 

\subsection{Authentication} 

Most devices, networks (PANs, VPNs, etc.), and services require some kind of authentication (a process by which a system determines whether the claimed identity of a user is genuine) before they grant an account and access privileges (p. 246) to a user. There are several different ways for a system to authenticate a user. Some are more reliable and secure than others, but for the most part, the more secure the method, the higher the operational expenses. 

\textbf{Account:} If you have access to an account on one system, this may give you automatic access to related systems or subsystems. This is typical of slaved devices (p. 248), where access to the master automatically grants you access to slaves. 

\textbf{Mesh ID:} Some systems accept mesh IDs as authentication This is extremely common with most public systems, which merely log the mesh ID of any user that wishes access. Other systems will only allow access to specific mesh IDs, but these are vulnerable to spoofing (p. 255). 

\textbf{Passcode:} This is a simple string of alphanumeric characters or logographic symbols, submitted in an encrypted format. Anyone with the passcode can access the account. 

\textbf{Biometric Scan:} This calls for a scan of one or more of the user's biometric signatures (fingerprint, palm print, retinal scan, DNA sample, etc.). Popular before the Fall, such systems have fallen out of use as they are impractical with synthmorphs or users that frequently resleeve. 

\textbf{Passkey:} Passkey systems call for some of encrypted code that is either hardwired into a physical device (that is either implanted or physically jacked into an ecto) or extracted from specialized software. Advanced passkeys combine hardwired encryption with physical nanotech etching to create a unique key. To access such systems, the passkey must either be acquired or somehow spoofed. 

\textbf{Ego Scan:} This system authenticates the user's ego ID (p. 279). 

\textbf{Quantum Key:} Quantum key systems rely on the unbreakable encryption of quantum cryptography (p. 254). 

\subsection{Firewalls} 

Firewalls are software programs (sometimes hardwired into a device) that intercept and inspect all traffic to and from a protected network or device. Traffic that meets specified criteria that designates it as safe is passed through, whereas all other traffic is blocked. In \textit{Eclipse Phase,} every network and device can be assumed to have a firewall by default. Firewalls are the main obstacle that an intruder must overcome, as discussed under \textit{Intrusion Tests,} p. 255. Like other gear, firewalls come in varying quality levels and so may apply modifiers to certain tests. 

\subsection{Active Monitoring} 

Instead of relying on authentication and firewalls alone, secure systems are actively monitored by a security hacker or a muse. These digital security guards inspect network traffic using a number of software tools and applications that flag conspicuous events. Active surveillance makes intrusions more difficult, since the interloper must beat the monitoring hacker/ AI in an Opposed Test (see \textit{Intrusion,} p. 254). Active monitoring also includes monitoring any devices slaved to the monitored system. Characters may actively monitor their own PANs if they so choose, though this requires a moderate level of attention (count as a Quick Action). It is far more common for a muse to actively guard a user's PAN. 

\subsection{Encryption} 

Encryption is an exceptionally effective extra layer of security. There are two types of encryption commonly used in \textit{Eclipse Phase:} public key cryptosystems and quantum cryptography. 

\subsubsection{Public Key Crypto} 

In public key cryptosystems, two keys are generated by the user, a public key and a secret key. The public key is used to encrypt messages to that user, and is made freely available. When messages are encrypted using that public key, only the secret key—controlled 

by the user—can decrypt them. Public key crypto is widely used both for encrypting data traffic between two users/networks/devices and for encrypting files. Due to the strength of the public key system algorithms, such crypto is essentially unbreakable without a quantum computer (see \textit{Quantum Code-} \textit{breaking,} p. 254). 

\subsubsection{Quantum Cryptography} 

Quantum key distribution systems use quantum mechanics to enable secure communications between two parties by generating a quantum key. The major advantage of transmitting information in quantum states is that the system itself instantly detects eavesdropping attempts as quantum systems are disturbed by any sort of external interference. In practical terms, this means that quantum encrypted data transfers are unbreakable and attempts to intercept automatically fail. Note that quantum crypto doesn't work for basic file encryption, its only use is in protecting communication channels. 

While quantum key systems have an advantage over public key systems, they are both more expensive and less practical. In order to generate a quantum key, the two communications devices must be entangled together on a quantum level, in the same location, and then separated. So quantum key encrypted communications channels require some setup effort, especially if long distances are involved. Since the implementation of quantum cryptographic protocols is an extraordinary expense, it is usually only adopted for major high-security communications links. 

\subsubsection{Breaking Encryption} 

What this means is that encrypted communications lines and files are very safe if using public key systems, and that data transfers are absolutely safe if using quantum crypto. Gamemasters should take note, however while this may be useful to player characters, it may also hinder them. If the characters need to get at something that is encrypted, they're going to need to figure out some way to get the secret key's passcode. Common methods include the old standbys of bribery, blackmail, threats, and torture. Other options involve espionage or social engineering to somehow acquire the passcode. Hackers could also find some other method to compromise the system and gain inside access, bypassing the encryption entirely. 

\subsubsection{Quantum Codebreaking} 

As noted above, quantum computers can also be used to break public key encryption. This requires an Infosec Task Action Test with a +30 modifier and a timeframe of 1 week (once started, the quantum computer finishes the job on its own; the user does not need to provide constant oversight). Gamemasters should feel free to modify this timeframe as fits the needs of their game. Note that quantum computers cannot break quantum-encrypted communications, only encrypted files. 

\section{Intrusion} 

The art of intrusion lies in penetrating a device's security The best methods involve infiltrating a system quietly without catching a watchdog's attention, by using exploits—code glitches, flawed security protocols—to create a path circumventing the target's defenses. When called for, however, a hacker can toss aside pretenses and attempt to brute-force their way in. 

\subsection{Preconditions} 

In order to hack a device, the hacker needs to establish a direct connection to the target computer system. If the hacker is making a direct wireless connection to the target, the target system must be wireless-capable and within range (p. 299), and the hacker must know the target is there (see \textit{Wireless Scanning,} p. 251). If the system is hard-wired, the hacker must physically jack in by using a regular jacking port or somehow tapping into a cable that carries the network's data traffic. If the hacker is accessing the target through the 

mesh, the target system must be online and the hacker must know it's mesh ID (p. 246) or otherwise be able to track it down (p. 251). 

\subsection{Circumventing Authentication} 

Rather than hacking in, an intruder can try to subvert the authentication system used to vet legitimate users. The easiest manner of doing this is to somehow acquire the passcode, passkey, or whatever authentication method the target uses (p. 253). With this in hand, no test is necessary to access the system; the hacker simply logs in just like a legitimate user and has all of the normal access privileges of that user. 

Lacking a passcode, the hacker can try to subvert the authentication system in one of two other ways: spoofing or forgery. 

\subsubsection{Spoofing Authentication} 

Using this method, the hacker attempts to disguise their signals as coming from the legitimate, authenticated user, rather than from themself. If successful, the system is fooled by this masquerade, accepting the hacker's commands and activity as if they came from a legitimate user. Spoofing is more difficult to pull off, but is very effective when it works. 

To spoof a legitimate user, the hacker must be using both sniffer and spoofing software (p. 331). The hacker must then monitor a connection between the legitimate user and the target system, and succeed in an Infosec Test to sniff the traffic between them (p. 252). Apply a –20 modifier if the user has security account privileges, –30 if they have admin rights (p. 247). If the connection is encrypted, this will fail unless the hacker has the encryption key. 

Armed with this data, the hacker then uses it to disguise their signals. This requires an Infosec Test, modified by the quality of the system's firewall and the hacker's spoofing program. If successful, communications sent by the hacker are treated as coming from the legitimate user. 

\subsubsection{Forging Authentication} 

Biometric and passkey systems used for authentication (p. 253) can potentially be forged hackers who are able to get a look at the originals. The means and techniques for doing so differ, and are beyond the scope of this book, but successfully forging such systems would allow a hacker to log in as the legitimate user. 

\subsection{Intrusion Tests} 

Hacking into a node is a time-consuming task. The target system must be carefully analyzed and probed for weaknesses, without alerting its defenses. Depending on the type of security in place, more than one test may be called for. 

Hackers require special exploit software (p. 331) to take advantage of security holes, but software does not a hacker make. What really counts is Infosec skill (p. 180), which is the ability to use, modify, and improvise exploits to their full advantage. 

\subsubsection{Defeating The Firewall} 

Lacking a passcode, the hacker must break in the old-fashioned way: discreetly scanning the target, look for weaknesses, and take advantage of them. In this case the hacker takes their exploit software and makes an Infosec Test. This is handled as a Task Action with a timeframe of 10 minutes. Various modifiers may apply, such as the quality of the exploit software, the quality of the Firewall, or the alertness of the target system. The gamemaster may also modify the timeframe, shortening it to reflect systems that are cookie-cutter common with known security flaws or raising it as fitting for a top-of-the-line system with still-unreleased defenses. 

By default, a hacker trying to break in this way is pursuing standard user access rights (p. 247). If the hacker wishes to obtain security or admin privileges on the system, apply a –20 or –30 modifier, respectively. 

If the Infosec Test succeeds, the intruder has invaded the system without triggering any alarms. If the system is actively monitored (p. 253), they must now avoid detection by that watchdog (see below). If there is no active monitor, the intruder gains the status of \textit{Covert} (see \textit{In-} \textit{truder Status,} p. 256). If the intruder scored an Excellent Success, however, their status is \textit{Hidden }(p. 256). 

\textbf{Probing:} Players may choose to take the time (p. 116) when probing the target for weakness and exploits In fact, this is a common procedure when a hacker wants to ensure success. 

\subsubsection{Bypassing Active Security} 

If a system is also actively monitored (p. 253), the hacker must avoid detection. Treat this as a Variable Opposed Infosec Test between the intruder and the monitor. The outcome depends on both rolls: 

If only the intruder succeeds, the hacker has accessed the node without the monitor or the system 

noticing. The hacker has acquired \textit{Covert} status (p. 256). If the hacker scored an Excellent Success, their status is \textit{Hidden }(p. 256). 

If only the monitor succeeds, the hacking attempt is spotted and the monitor may immediately lock the hacker out of the system before they manage to fully break in. The intruder may try again, but the monitor will be vigilant for further intrusions. 

If both succeed, the intruder has gained access but the monitor is aware that something strange is going on. The hacker acquires \textit{Spotted} status. 

If both fail, continue to make the same test on each of the hacker's Action Phases, until one or both succeed. 

\begin{quotation} \textbf{The Hacking Sequence} 

\begin{tabular}{ll} 1. Defeat the Firewall &Infosec Task Action (10 minutes) \\ 2. Bypass Active Security &Opposed Infosec Test \\ \indent a. Hacker Wins with Excellent Success, Defender Fails &Hidden status/+30 all tests (p. 256) \\ \indent b. Hacker Succeeds, Defender Fails &Covert Status (p. 256) \\ \indent c. Both Succeed &Spot Status/Passive Alert (p. 256) \\ \indent d. Defender Succeeds, Hacker Fails &Locked status/Active Alert (p. 256) \\ \end{tabular} \end{quotation} 

\subsection{Intruder Status} 

\textit{Intruder status} is a simple way of measuring an invader's situation when they are intruding upon a system. This status has an impact on whether the hacker has caught any attention or if they managed to remain unobtrusive. Status is first determined when the intruder access the system, though it may change according to events. 

Note that intruder status is a separate matter from account access privileges (p. 246). The latter represents what a user can legally do on a system. The former indicates how aware the system is of the hacker's true nature as an intruder. 

\subsubsection{Hidden} 

An intruder with Hidden status has managed to silently sneak into the system without anyone noticing. The system's security is totally unaware of their presence and may not act against them. In this case, the hacker is not using an account so much as they are exploiting a flaw in the system that grants them a nebulous, behind-the-scenes sort of presence in the system. The hacker effectively has admin access rights, but does not show up as an admin-level user in logs or other statistics. Hidden characters receive a +30 modifier on any efforts to subvert the system. 

\subsubsection{Covert} 

An intruder with Covert status has accessed the system in a manner that doesn't attract any unusual attention. For all intents and purposes, they appear to be a legitimate user with whatever access rights they sought. Only extensive checking will turn up any abnormalities. The system is aware of them, but does not consider them a threat. 

\subsubsection{Spotted} 

Spotted status indicates that the system is aware of an anomaly or intrusion but hasn't zeroed in on the intruder yet. The hacker appears to be a legitimate user with whatever access rights they sought, but this will not hold up under close scrutiny. The system goes on passive alert (inflicting a –10 modifier to the hacker's activities on that system) and may engage the hacker with passive countermeasures (p. 257). 

\subsubsection{Locked} 

Locked status means that the intruder—including their datatrail—has been pinned down by system security. The hacker has access and account privileges, but they have been flagged as an interloper. The system is on active alert (inflicting a –20 modifier on the hacker's actions and may launch active countermeasures (p. 257) against the intruder. 

\subsection{Changing Status} 

An intruder's status is subject to change according to their actions and the actions of the system. 

\subsubsection{Upgrading Status} 

A hacker can attempt to improve their status in order to better protect themself. This requires a Complex Action and an Infosec Test. If the hacker has Spotted status, this is an Opposed Test between monitor and intruder. If the hacker wins and scores an Excellent Success (MoS of 30+), they have upgraded their status by one level (for example, from Covert to Hidden). Intruders with Locked status may not upgrade. 

\subsubsection{Zeroing In} 

A security hacker or muse that is actively monitoring a system can take a Complex Action and attempt to hone in on a Spotted intruder. An Opposed Infosec Test is made between both parties. If the system's defender wins, the hacker is downgraded to Locked status. 

\subsubsection{Failing Tests} 

Any time an intruder scores a Severe Failure (MoF 30+) on a test involving manipulating the system, they are automatically downgraded one status level (from Covert to Spotted, for example). If a critical failure is rolled, they immediately give themselves away and achieve Locked status. 

\subsection{Brute-Force Hacking} 

Sometimes a character simply doesn't have time to do the job right, and they need to hack in now or never. In this case the hacker engages the target system immediately head on, without taking any time to prepare an attack. The hacker simply brings all of their software exploit tools to bear, throwing them at the target and hoping that one works. This is handled as an Infosec Test, but as a Task Action with a timeframe of 1 minute (20 Action Turns). The hacker receives a +30 modifier on this test. Many hackers choose to rush the job (see \textit{Task Actions,} p. 120), in order to cut this time even shorter. 

The drawback to brute-force hacking is that it immediately triggers an alarm. If the system is actively monitored the hacker must beat the monitor in an Opposed Infosec Test or be immediately locked out as soon as they break in. Even if they succeed, the hacker has \textit{Locked} status and is subject to active countermeasures. 



\section{Intrusion Countermeasures} 

If an intruding hacker fails to penetrate a system's defenses (i.e., they are Spotted or Locked, see p. 256), then the system goes on alert and activates certain defenses. The nature of the applied countermeasures depends on the capabilities of the system, the abilities of its security defender(s), and the policy of its owner/ admins. While some nodes will simply seek to kick the intruder out and keep them shut out, others will actively counterattack, seeking to track the intruder and potentially hack the intruder's own PAN. 

\subsection{Security Alerts} 

Security alerts come in two flavors: passive and active. 

\subsubsection{Passive Alert} 

Passive alerts are triggered when an intruder hits Spotted status. The system immediately flags a visual or acoustic cue to anyone actively monitoring the system and possibly the owner or admins. It immediately launches one or more passive countermeasures (see below). Depending on the system, extra security hackers or AIs may be brought in to help investigate. If the intruder is not encountered again or located within a set time period (usually about 10 minutes), the alarm is deactivated and the event is logged as an anomaly. Depending on the security level of the system, someone may analyze the logs at some point and try to ascertain what happened—and prevent it from happening again. 

All intruders suffer a –10 modifier for tests involving a system that is on passive alert. 

\subsubsection{Active Alert} 

An active alert is triggered when an intruder hits Locked status. The system immediately alerts the owners, admins, and monitoring security agents. Additional security assets (hackers and AIs) may be called in. The system also launches active countermeasure against the intruder (see below). Active alerts are maintained for as long as the intruder is present, and sometimes for a lengthy period afterwards just in case the hacker returns. 

\subsection{Passive Countermeasures} 

Passive countermeasures are launched as a precaution whenever an intruder acquires Spotted status. 

\subsubsection{Locate Intruder} 

A security hacker or AI monitoring a system may attempt to track down the source of the passive alert. See \textit{Zeroing In,} p. 256. 

\subsubsection{Re-Authenticate} 

When a passive alert is triggered, a firewall can be set to re-authenticate all active users, starting with the most recent. At the beginning of the next Action Turn, everyone on the system must take an action to log back in. For intruders, this means making an Infosec Test, modified by –10 for the passive alert, to satisfy the system that they are a legitimate user. 

\subsubsection{Reduce Privileges} 

As a protective measure, some systems will immediately reduce access privileges available to standard users, and sometimes security users as well. One common tactic is to protect all logs, backing them up and making sure no one has rights to delete them. 

\subsection{Active Countermeasures} 

Active countermeasures can only be launched if the intruder has acquired Locked status. 

\subsubsection{Counterintrusion} 

A security hacker or guardian AI can proactively defend a system by attacking the intruder's source. For this to occur, the intruder must first be successfully traced (p. 251). Once this occurs, the security forces can then launch their own intrusion on the hacker's home ecto/mesh inserts and/or PAN. 

\subsubsection{Lockout} 

A system that has locked onto an intruder may also attempt to lock them out. Lockout is an attempt to remove the compromised account, sever the connection between the two, and dump the hacker from the system. Lockout must be initiated by someone with security or admin privileges. An Opposed Infosec Test is made, with the intruder suffering a –20 modifier for being Locked. If the character defending the system succeeds, the intruder is immediately ejected from the system and the account they used will be placed on quarantine or deleted. That account will not be usable again until a security audit approves it and replaces the authentication. Any attempt to access the system from the same mesh ID as the intruder automatically fails. 

\subsubsection{Reboot/Shutdown} 

Perhaps the most drastic option for dealing with an interloper is to simply shut down the system. In this case, the system closes all wireless connections (if it has any), logs off any users, terminates all processes, and shut itself down—thereby locking out the intruder. The disadvantage, of course, is that the system must interrupt its activities. For example, shutting down your mesh inserts or ecto means losing all communication with teammates, access to augmented reality, and control over slaved/linked devices. Initiating a reboot/shutdown is only a Complex Action, but the actual process of shutdown takes anywhere from 1 Action Turn (personal devices) to 1 minute (large hardwired networks with multiple users), determined by the gamemaster. Rebooting a system takes the same amount of time to get started again. 



\subsubsection{Trace} 

For high-security systems, a popular countermeasure is to track the infiltrator's physical location via their mesh ID (see \textit{Physical Tracking,} p. 251). In most cases, habitat physical security is subsequently alerted and forwarded the position to take care of the criminal. 

\subsubsection{Wireless Termination} 

An alternative to shutdown or rebooting is simply to sever all wireless connections by shutting down the wireless capabilities of the system. The system will lose all active connections, but any intruders will be dumped. Wireless termination is a Complex Action to initiate and completes at the end of that Action Turn. Re-starting wireless connectivity takes 1 Action Turn. 



\begin{quotation} \textbf{Joint Hacking/Securing} 

Hacking will sometimes involve teams of attackers and/or teams of defenders. A hacker might be backed up by their muse or another team member with moderate Infosec skills. Hard networks are often defended and monitored by teams of highly-skilled security hackers and AIs. When intruding in or defending a computer system, operators must decide whether to act individually or in concert. 

Each approach has its tradeoffs. A team that chooses to breach or maintain a system’s security as a team effort must allocate one character (usually the team member with the highest Infosec skill) as the primary actor (see Teamwork, p. 117). Each additional character and muse adds a +10 modifier for each test (up to the maximum +30 modifier) but cannot spend time on other actions than those performed by the team leader. When acting in concert, teams may switch team leaders at any time, in case group members are specialized for certain tasks. 

Alternately, both intruding and defending teams may choose to act individually but for a joint goal. Each hacker must make intrusions on their own, with individual repercussions for detection and counterintrusion, which runs the risk of affecting all intruders if any one is Spotted or Locked. On the other hand, a team of intruders can pursue multiple actions simultaneously in a coordinated manner and may temporarily overwhelm available security. The same holds true for system defenders, who may accomplish more by splitting their actions, leaving some to monitor while others launch counterintrusion attacks and other countermeasures. \end{quotation} 



\section{Subversion} 

Once an intruder has successfully invaded a device or network, they can pursue whatever tasks they are interested in, as fitting that particular system. Depending on the type of account the intruder hacked, they may or may not have access privileges to do what they want to do. If their access rights allow it, the activity is handled like that of a legitimate user and no test is called for (unless the activity itself calls for some kind of test, such as Research). For example, a hacker who infiltrates a habitat's security system with a security account can monitor cameras, deactivate sensors, review recorded surveillance footage, and so on, as any legitimate user with security right would be allowed to do. Engaging in any sort of activity for which you don't have access rights is more difficult and requires hacking the system. This typically requires an Infosec Success Test, modified by the difficulty of the action as noted on the Subversion Difficulties table. In most cases this in not an Opposed Test even if the system is actively monitored, unless specifically stated otherwise. Failing such tests, however, will result in a change of the hacker's intruder status (see \textit{Failing Tests,} p. 256). Examples for different types of system subversion are given in the \textit{Subversion Examples} sidebar. This is not an exhaustive list, however, and gamemasters and players are encouraged to improvise game effects in case an action has not been explicitly described. 

\subsection{Augmented Reality Illusions} 

A hacker who has infiltrated an ecto, mesh inserts, or some other device with an AR interface may inject different kinds of visual, auditory, tactile, and even emotional illusions into the augmented reality of the device's user, depending on the type of interface used. How the hacked user will respond to the illusion depends on a number of factors, such as whether they are aware of the intruder (i.e., the hacker has Spotted or Locked status), what type of interface they are using (entoptic or haptic), and how realistic the illusion is. The best illusions are, of course, crafted in advance, using the best image and sensory manipulation tools available. Such illusions are hyper-realistic. Anyone making a Perception Test to identify them as fake 

suffers a –10 to –30 modifier (gamemaster's discretion An eclectic collection of software programs offer a diverse range of AR illusions. 

\begin{table} \begin{tabularx}{\hline}{|l|X|} \hline

\hline{2}{|c|}{\textbf{Subversion Difficulties}} \\ \multicolumn{2}{|c|}{Difficulty modifiers for common computer tasks} \\ \hline

\hline

–0 &Execute commands, view restricted information, run restricted software, open/close connections to other systems, read/write/copy/delete files, access sensor feeds, access slaved devices \\ \hline

–10 &Change system settings, alter logs/restricted files \\ \hline

–20 &Interfere with system operations, alter sensor/AR input \\ \hline

–30 &Shut system down, lockout user/muse, launch countermeasures at others \\ \hline

\label{tab:subversion-difficulties} \label{tab:subversion-difficulties} \end{table} 



\begin{table} 

\begin{tabularx}{\textwidth}{|l|X|} 

\hline

\hline{2}{|c|}{\textbf{Subversion Expamples}} \\ \multicolumn{2}{|c|}{In addition to the tasks noted under the Subversion Difficulties table,} \\ \multicolumn{2}{|c|}{these modifiers present some additional example actions.} \\ \hline

\textbf{Modifier} &\textbf{Task} \\ \hline

\hline{2}{|c|}{Hacking Bots/Vehicles} \\ \hline

–0 &Give orders to drones \\ \hline

–10 &Alter sensor system parameters, disable sensors or weapon systems \\ \hline

–20 &Alter smartlink input, send false data to AI or teleoperator \\ \hline

–30 &Lockout AI or teleoperator, seize control via puppet sock \\ \hline

\hline{2}{|c|}{Hacking Ectos/Mesh Inserts} \\ \hline

–0 &Interact with entoptics, befriend everyone in range, make online purchases using user's credit, intercept communications, log activity \\ \hline

–10 &Alter social network profile/status, adjust AR filters, tweak sensory interface, change AR skin, change avatar, access VPN \\ \hline

–20 &Block or shuffle senses, inject AR illusions, spoof commands to drones/slaved devices \\ \hline

–30 &Boot user out of AR \\ \hline

\hline{2}{|c|}{Hacking Habitat Systems} \\ \hline

–0 &Open/close doors, stop/start elevators, operate intercom \\ \hline

–10 &Adjust temperature/lighting, disable safety warnings, replace entoptic skin, lock doors, switch traffic timers \\ \hline

–20 &Disable subsystems (plumbing, recycling, etc.), disable wireless links, dispatch repair crews \\ \hline

–30 &Override safety cutoffs \\ \hline

\hline{2}{|c|}{Hacking Security Systems} \\ \hline

–0 &Move/manipulate cameras/sensors, locate security systems/guards/bots \\ \hline

–10 &Adjust patterns of sensor sweeps, view security logs, disable weapon systems \\ \hline

–20 &Delete security logs, dispatch security teams \\ \hline

–30 &Disable alerts \\ \hline

\hline{2}{|c|}{Hacking Simulspace Systems} \\ \hline

–0 &View current status of simulspace, simulmorphs, and accessing egos \\ \hline

–10 &Change domain rules, add cheats, alter parameters of story, alter simulmorphs, change time dilation \\ \hline

–20 &Eject simulmorph, alter/erase character AIs \\ \hline

–30 &Abort simulation \\ \hline

\hline{2}{|c|}{Hacking Spimes} \\ \hline

–0 &Get status report, use device functions \\ \hline

–10 &Adjust AI/voice personality settings, adjust timed operation schedule \\ \hline

–20 &Disable sensors, disable device functions \\ \hline

\end{tabularx} \end{table} 

Hackers may also improvise illusions on the fly, usually by patching in sensor data from others sources, though this is more difficult and more easily spotted (typically adding a +10 to +30 modifier to Perception Tests). The advantage is that the hacker can modify the illusion in reaction to the user's actions or environmental factors on the fly. AR illusion software, however, also offers some template illusions that can be modified and controlled in real-time via a connected interface. 

Whenever a user is bombarded with AR illusions, the gamemaster should make a secret Perception Test to see if they spot the deception. Even if they do, however, the character may still react to them. Almost anyone will duck when they see an object suddenly flying at their face, as their body reacts on its own before the brain comprehends that it's an illusion and not a threat. 

Aside from their deceptive value, illusions can be used to distract users or otherwise impair their perceptive faculties. For example, dark illusory clouds can obscure vision, ear-wrenching high-volume noises can make people cringe, and a persistent tickling sensation might drive anyone crazy. Such effects can apply a –10 to –30 modifier to Perception Tests and other actions, but the user can also adjust their filters and/or turn their AR off if necessary. 

\subsection{Backdoors} 

A backdoor is a method of bypassing a system's normal authentication and security features. It enables a hacker to sneak into a system by exploiting a flaw (which can take the form of an installed program or modification to an existing program or hardware device) that was integrated into the system previously, either by themself or another hacker (who shared the backdoor). 

To install a backdoor, the hacker must successfully infiltrate the system and succeed in both a Programming and an Infosec Test (or an Opposed Infosec Test if the system is actively monitored). The Programming Test determines how well the backdoor is crafted and hidden within system processes, while the Infosec Test represents incorporating it into the system without security noticing. Modify the Programming Test by –20 if the hacker wants to have security privileges when using the backdoor, –30 for admin. 

Once installed, using a backdoor requires no test to access the system—the hacker simply logs on as if they were a legitimate user, gaining Covert status. Anyone who is aware of a backdoor's details may use it. 

How long the backdoor lasts depends on many factors and is largely up to the gamemaster. Backdoors are only likely to be spotted during complete security audits, so more paranoid systems are likely to detect them earlier. Security audits may also occur when an intruder is Spotted but never Locked. Security audits are a Task Action with a timeframe of 24 hours. The character conducting the audit makes an Infosec Test to spot the back door. If the backdoor's installer scored an Excellent Success on their Programming Test, this Infosec Test suffers a –30 modifier. 

\subsection{Crashing Software} 

Intruders can attempt to crash software programs by killing the processes that run them. This requires a Complex Action and an Infosec Test. Note that some software is set to immediately respawn, but this can take from 1 Action turn to 1 minute, depending on the system. 

Hackers may crash AIs, AGIs, and even infomorphs this way, but the process is more difficult. In this case, an Opposed Infosec Test is made against the target, who is immediately aware they are under attack. Two consecutive tests must succeed in order to crash an AI, three in order crash an AGI or infomorph. If successful, the AI/infomorph immediately reboots, which generally takes 3 Action Turns, longer if the gamemaster chooses. 

\subsection{Eliminating Intrusion Traces} 

Hackers who have avoided being Locked may attempt to clean up all traces of their intrusion before they exit a system. This involves erasing incriminating data in the access and security logs, and otherwise hiding any evidence of system tampering. This requires a Complex Action and an Infosec Test, or an Opposed Infosec Test if the system is actively monitored. If successful the intruder has wiped anything that might be used to track them down later, such as mesh ID, etc. 

\subsection{Hacking Vpn} 

Virtual private networks (VPNs) are more challenging to hack than standard devices. Because they exist as an encrypted network within the mesh, accessing channels of communication within a VPN is all but impossible without the encryption key. This means any attempt to sniff the VPN traffic is also impossible without the key. 

The only way to hack a VPN is to hack into a device that is part of the VPN and running the VPN software. Once an intruder has access to such a device, they can attempt to access the VPN. The account the hacker has compromised may have VPN privileges, in which case they are in. If not, they must hack access, requiring an Infosec Test with a Minor modifier (–10). 

Once access to the VPN is acquired, the hacker may treat the VPN like any other network. They may hack other devices on the VPN, sniff VPN traffic, track other users on the VPN, research data hidden away on the VPN, and so on. 

\subsection{Scripting} 

A script is a simple program—a batch of instructions— that a hacker can embed in a system to be executed at a later pre-scheduled time or upon a certain trigger event, even without the hacker being present. When activated, the script will undertake a number of system operations limited by the abilities of the operating system and the access rights the hacker had when they implemented the script in the system. Scripts are 

a great way for a hacker to subvert a system without necessarily being in danger when they do it. 

Scripts can be programmed on the fly or preprogrammed When composing the script, the character must detail what system operations the set will call for, in what order and at what times (or at what trigger steps). The script cannot contain more steps/tasks than the character's Programming skill $\div$ 10 (round down). To program a script, the character must succeed in a Programming Test with a timeframe determined by the gamemaster. 

To load the script, the character must have successfully intruded in the system and must succeed in an Infosec Test (or an Opposed Infosec Test if the system is actively monitored). If successful, the script is loaded into the system and will run as programmed. 

Once the script is activated, it carries out the preprogrammed sequence of actions. The programmer's Infosec skill is used for any tests those actions call for. 

Inactivated scripts may be detected in a security audit, just like backdoors (p. 260). 

\begin{quotation} \textbf{Expample} 

Sarlo has infiltrated a security system and wants to arrange it so that a particular security sensor deactivates and a door unlocks at a set time, allowing his team to infiltrate a high-security area. He creates a script that will activate at 2200 hours with the following steps: \end{enumerate} \item At 2200 hours, disable security sensor \item Then unlock door \item At 2230 relock door \item Then re-enable security sensor \item Eliminate traces \end{enumerate} 

This script has 5 steps, which Sarlo can handle with his Programming skill of 70. Sarlo succeeds in his Programming and Infosec Tests, and the script is loaded. It will then activate at the appropriate time. Since Sarlo’s account did not have access rights to perform these actions, each will require an Infosec Test using Sarlo’s skill to succeed. 

\end{quotation} 



\section{Cyberbrain Hacking} 

Pods and synthmorphs (including some bots and vehicles are equipped with cybernetic brains. While this technology allows a transhuman ego to sleeve into and control these forms, they carry the disadvantage of being vulnerable to hacking, like any other electronic device. 

Cyberbrains are not wireless-enabled for security reasons, but they do have access jacks (p. 306) and are directly linked to mesh inserts. This means that in order to hack a cyberbrain, the hacker either must have direct physical access to the morph's body in order to jack in, or they must first hack into the mesh inserts and then break into the cyberbrain from there. 

Due to their importance, cyberbrains are equipped with numerous hard-coded security features that make intrusion very difficult. Apply a –30 modifier to all attempts to hack into and subvert a cyberbrain. (Note that the –30 modifier for hacking an admin account does not apply to cyberbrains.) Cyberbrains are treated just like other systems for intrusion and subversion purposes, but since they house the morph's controlling ego they present several unique hacking opportunities. 

\subsection{Entrapment} 

An intruding character can attempt to \textit{lock in} an ego, preventing it from evacuating the cyberbrain. The hacker (with the –30 modifier noted above) must beat the defending character or muse in an Opposed Infosec Test. If successful, the ego is prevented from transferring itself to another system. To fully pen the ego in, the ego character and its protecting muse must also be locked out (p. 257) from controlling the cyberbrain's system, otherwise the ego could potentially be freed. Trapped egos are quite vulnerable. They could, for example, be subject to enforced uploading, enforced forking, or psychosurgery. 

\subsection{Memory Hacking} 

All cyberbrains incorporate mnemonic augmentation (p. 307), or digitally recorded memories. A hacker who has accessed the cyberbrain can read, alter, or delete these memories with a successful Research or Interfacing Test (the –30 modifier applies). 

\subsection{Puppeteering} 

Most cyberbrains also incorporate a puppet sock (p. 307), enabling remote users to take over the pod or synthmorph body and control it via teleoperation or jamming (p. 196). This allows a hacker to seize control of the body and manipulate it remotely. To do so, the hacker must take a Complex Action and beat the defending character or muse in an Opposed Infosec Test; the hacker suffers the –30 modifier noted above. A defender who is not locked out may continue to fight for control of the morph, using a Complex Action. In this case, another Opposed Infosec Test is called for. This can result in a situation where the morph repeatedly slips control from the hacker to the defender, or perhaps slips into a catatonic state as the two sides battle it out. 

\subsection{Scorching} 

Having direct access to a cyberbrain opens the possibility for certain kinds of attacks that are normally infeasible due to the strict content filtering that occurs on the link between the cyberbrain and mesh inserts. One of these possibilities is \textit{scorching}—the use of damaging neurofeedback algorithms to harm the victim's mind. In order to make a scorching attack, the cyberbrain intruder must deploy a scorch program. To utilize a scorch program, the intruder must beat the defending Ego in an Opposed Infosec Test. The –30 modifier for cyberbrain hacking applies to the attacker. Several types of scorch programs exist, with different effects: cauterizers (damage), bedlams (stress), 

spasms (pain), nightmares (fear), and shutters (sensory deprivation). These are described on p. 332 of \textit{Gear.} 

\subsection{Shutdown} 

If a cyberbrain is shut down (p. 257), the morph immediately ceases activity, perhaps collapsing or rolling to a stop. Pods will appear to be in a coma. The ego, however, will be rebooted along with the cyberbrain. 

\subsection{Terminate Cortical Stack Feed} 

The cyberbrain feeds backup data to the cortical stack. This is a one-way connection, so the cortical stack may not be hacked, but the transfer of data may be cut off. This termination action requires an Opposed Infosec Test between the hacker (with the –30 modifier) and the defender. The ego's backup will not be updated for as long as the connection remains off. 

\section{Radio Jamming} 

Radio jamming is a method of transmitting radio signals that deliberate interfere with other radio signals in order to disrupt communications. In the highly-networked world of \textit{Eclipse Phase,} intentional jamming is often illegal, not to mention rude. 

Radio jamming does not require any special equipment other than a standard wireless device, such as an ecto or mesh inserts. Jamming can be \textit{selective } or \textit{universal.} Selective jamming targets a particular device or set of devices. In order to selectively jam, the character must have scanned the target device(s) (p. 251). Universal devices target all radio-equipped devices indiscriminately. 

Jamming simply requires a Complex Action and an Interfacing Test. If successful, affected devices within range have their radio communications disrupted— they are cut off from the mesh and may not communicate wirelessly. Wired devices are unaffected. 

Devices equipped by AIs will automatically attempt to overcome jamming, which requires a Complex Action (transhuman users may also do the same). In this case, a Variable Opposed Test is made between the jammer and defender. If the jammer wins, all communications are blocked; if the defender wins, they are unaffected. If both parties succeed, the defender's communications are impacted but not completely cut off. The gamemaster decides how much information the defender can get through, and how this situation affects mesh use. 

\subsection{Jamming Radar} 

Jamming can also be used to interfere with radar. In this case, if the jammer makes an Interface Test. If successful, the radar suffers interference, imposing a –30 modifier on all sensor-related tests. The entity operating the radar may attempt to overcome this interference by beating the jammer in an Opposed Interface Test. 

\section{Simulspaces} 

Simulspaces are virtual reality environments where the resolution advances beyond realistic high definition and into the hyper-real. The environments they create are comprehensive and authentic illusions, from aspects like lighting, day or lunar cycles, and weather down to minute details and sensations. Jacking into a simulspace is much like crossing over into a alternate world or reality, which is why simulspaces have become increasingly popular in entertainment. 

While simulspaces usually cannot harm characters immersed in them as the sensory algorithms are not intended to be offensive programs or routines, experiences in simulspaces can have a strong psychological impact on an ego, as the simulation is as close to reality as you can get. A character who is ``physically'' tortured within a simulspace will not be physically harmed, but the mental stress of the experience might still be sufficient to cause permanent traumas. 

\subsection{Simulmorphs} 

Characters access simulspace using an avatar-like persona called a simulmorph. This simulmorph is created by the simulspace, based on the domain rules of the simulation and certain characteristics of the morph or ego accessing the simulation. Depending on the simulation, this simulmorph may be customizable to varying degrees. 

While interacting with the simulation, treat simulmorphs as basic infomorphs for all rules purposes, even if the egos are still possessing another morph body in reality. 

When accessing a simulspace, muses are usually not transferred into the simulation, though they can potentially come along if domain rules permit it. In this case, muses are treated as separate characters within the simulspace with their own simulmorph body. 

Depending on the role a simulspace is intended to play in the story, the gamemaster may want to invent physical stats'' for the simulmorph bodies, especially if the characters are likely to spend a lot of time in the simulation These statistics can literally be made up—it is a virtual reality after all, and anything goes. Alternately, the gamemaster can simply wing it and invent any necessary statistics on the fly as the need for them comes up. 

\subsection{Immersion} 

When a character immerses themselves in a simulspace, they ``become'' the simulmorph. The character's physical body, typically secluded and protected in a vat or couch, slumps inertly. While immersed, they suffer –60 on all Perception Tests or attempts to take action with their physical morph. Characters can enter and leave the simulspace as will, but toggling in or out takes a Complex Action. 

If the simulspace crashes or the character is otherwise dumped from it, they immediately resume control of their own morph as normal. VR dumpshock is extremely jarring, and the character suffers 1d10 $\div$ 2 mental stress. 



\subsection{External Mesh Interaction} 

A character accessing a simulspace may still interact with the mesh (and through it, the outside world) assuming the domain rules allow for it. Any outside interactions are subject to time dilation issues, however For example, in a simulspace running faster than real time, holding a chat with someone in outside meatspace is excruciatingly slow, as real-world seconds translate into minutes in VR. If a character wishes to directly access other mesh nodes, they must toggle or log out of the simulspace. 

\subsection{Simulspace Rules} 

Since a simulspace is an alternate world whose realism matches reality, characters use their physical skills and aptitudes as if they were acting in the real world with few exceptions: \begin{itemize} \item Though intrusion and hacking can be represented as another layer of the simulation, there is no actual hacking within the simulspace (see \textit{Hacking Simulspaces}). 

\item Asyncs cannot use their psi abilities in simulspace, though such abilities can be simulated. 

\item Any ``physical'' damage taken in the simulspace is treated as ``virtual'' damage. While virtual injuries and wounds use the same mechanics, characters that die in a simulspace are usually simply ejected from the simulation. In some cases ``dead'' characters are brought into a white room and can re-enter or just watch the simulation, depending on the domain rules. 

\item Mental stress or trauma inflicted during a simulation carries over to the ego as real Lucidity damage. At the gamemaster's discretion, some mental stress may be reduced if the character is aware that they are in a simulation. \end{itemize} 



\subsection{Domain Rules} 

Anything goes in a simulspace, as dictated by the domain rules. A simulspace may range from approximating reality very closely to differing drastically Gravity might fluctuate, the visual light spectrum might not exist, characters might heal virtual damage effortlessly, simulmorphs may be capable of transmogrifying into other creatures, everything might be underwater—the possibilities are endless, limited only by imagination. In game terms, this allows the gamemaster to make up rules on the fly. 

\subsection{Cheating} 

As with any good game, simulspaces provide ways to cheat. Cheats are either built into the simulspace software or (externally) programmed in by a hacker. Cheats allow for a character to break the domain rules in some way. This may be a special power, a way to alter some environmental factor (like flying), altering the time dilation, some sort of power-up ability, a way to get info on other simulmorphs, or a short-cut through part of the simulation. In game terms, cheats might provide bonus modifiers to certain skill or stat tests made by a simulmorph. Cheating is usually forbidden Players who cheat in a simulspace game and who get caught may face eviction from the simulspace. 

\subsection{Hacking Simulspaces} 

Since simulspaces are complex virtual environments and often run on time dilation, hackers cannot hack them in a normal manner when they participate in the simulation. There are ways to affect and influence the simulation from within, but the degree of subversion that is achievable is limited. For this reason, hackers rarely enter into VR to hack. Hacking into the external system running a simulspace is just like breaking into any other system. Use all of the standard rules for intrusion and subversion. 

\begin{quotation} \textbf{Hacking Simulspace From Within Modifier Task} \\ \begin{tabularx}{\hline}{|rX|} \hline

–0 &Analyze simulation parameters, view domain rules, shape appearance of simulmorph, switch simulmorph character or morph type \\ \hline

–10 &Change probability of test outcomes, become invisible (“out-game”) to others \\ \hline

–20 &Interfere with simulation (e.g. make it rain, generate earthquakes), generate items, ignore domain rules, kill or lockout other simulmorphs \\ \hline

–30 &Go into god mode, command simulated characters, take over the simulation \\ \hline

\end{tabularx} \end{quotation} 

\subsubsection{Meddling From The Inside} 

Within a simulspace, a hacker's only choice for interacting with the VR controls is through the standard interface that any simulmorph can pull up. Typically used for standard user features like adjusting your simulmorph or chatting with or checking the status 

of other users, a clever hacker might find some ways to subvert the system. Such options are usually limited however, as a number of system controls and processes cannot be accessed and manipulated from the inside. Most of the hacker's options are going to involve meddling with the simulation and its specific domain rules or possibly gaining access to cheats. To make a change requires a successful Interface Test. Ultimately the gamemaster decides what the hacker can and cannot get away with, based on the limitations of that particular simulspace. Most simulspaces are monitored to prevent cheating and abuse, though the monitors are typically preoccupied with maintaining the simulspace as a whole, dealing with other users, etc. At the gamemaster's discretion, such a monitor might get to make an Interface Test (possibly with a modifier for distraction) to notice the hacker's efforts. 

\section{AI And Muses} 

AIs are sentient but specialized programs. Like other software, they must be run on a computerized system. Most AIs are run on bots, vehicles, and other computerized devices where they can assist transhuman users or operate the machine themselves. They are also commonly used to actively monitor computer systems against intrusion attempts. \textit{Muses} are AIs that specialize as personal companions, always at a character's virtual side every since they were a child. Sample AIs and muses can be found on p. 331 of \textit{Gear.} 

\subsection{AI Limitations} 

AIs feature a number of built-in restrictions and limitations. To start with, they can be loaded in the cyberbrains of pods and synthmorphs, but they may not be downloaded into biomorph brains. As software, they use the same rules as other software and may be shut down, restarted, copied, erased, stored as inert data, infected with virii, and reprogrammed. Due to their size and complexity, only one AI (or infomorph) may be run on a personal computer at a time (see \textit{Computer Capabilities,} p. 247), and they may not run on peripheral devices. While they possess cognition and intelligence, they are incapable of self-improvement and cannot expand their programming and skills on their own. Although –0 

Analyze simulation parameters, view domain rules, shape appearance of simulmorph, switch simulmorph character or morph type –10 

Change probability of test outcomes, become invisible (``out-game'') to others –20 

Interfere with simulation (e.g. make it rain, generate earthquakes), generate items, ignore domain rules, 

kill or lockout other simulmorphs –30 

Go into god mode, command simulated characters, take over the simulation they are not able to learn they do possess memory storage that grants them the ability to remember and a limited form of adaptation. AIs do not earn Rez Points, nor do they have Moxie. 

AIs have aptitudes no greater than 20 but are incapable of defaulting. If they don't possess a skill, they don't know how to do it. (At the gamemaster's discretion, they may default to field skills or similar skills as noted on p. 173 with a –10 to –30 modifier). They can use skills like any character in \textit{Eclipse Phase}, however they may not possess any Active skill at a rating higher than 40 or Knowledge skill higher than 90—the maximum amount of expertise that their skill software allows. 

While AIs are programmed with personality templates and empathy, they are generally less emotive and difficult to read (apply a –30 modifier to Kinesics Test made against them, when in pod bodies). When combined with non-expressive synthetic morphs, they are even more difficult (–60 modifier). Some AIs lack emotive capabilities altogether and are impossible to read with Kinesics skill. 

AIs do have a Lucidity and Trauma Threshold stat, and are capable of suffering mental stress and traumas. 

\subsection{Commanding AI} 

AIs and muses are programmed to accept commands from authorized users. In some circumstances, they may also be programmed to follow the law or some ethical code. Programming is never perfect, however, and AIs can be quite clever in how they interpret commands and act on them. In most cases, an AI will rarely refuse to follow a request or obey a command. Given that they also usually have a duty to protect the person commanding them, the AI may be reluctant to follow commands that could be construed as dangerous or having a negative impact on the user. Under certain circumstances, preprogrammed imperatives can force an AI to ignore or disobey their owner's commands (gamemaster's discretion). 

\section{AGI And Infomorphs} 

The term ``infomorph'' is used to refer to any ego in digital body, whether that be an AGI or the digital emulation of a biological mind (including backups and forks). The following rules apply to infomorph and AGI characters. 



\begin{quotation} \textbf{ROLEPLAYING MUSES} 

Muses should not be viewed as a mere tool for getting extra skills, but as an opportunity to enhance roleplaying. Though typical muse AIs are not complete intelligences (though they can be, see Infomorphs as Muses), their personality matrix is often quite sophisticated and they are very good at adapting to their user’s personality quirks. On the other hand, they share the same Real World Naiveté (p. 151) as AGI characters when it comes to understanding all the facets of transhuman behavior, social interaction, body language, or emotion. Their personalities are more non-human, abstract, alien, and less passionate than transhuman life forms, often leading to conceptual misunderstandings and miscommunications. Likewise, their creative capacities are limited, instead bolstered by an ability to calculate odds, run simulations and evaluate outcomes, and make predictions based on previous experiences. 

Depending on the user’s stance towards sentient programs, muses can be viewed as intelligent toys, followers, servants, slaves, friends, or pets, which should somehow be reflected in game play. Most transhumans have also acquired a tendency to bond with a muse mentally due to its omnipresence and devotion to the user (like bonding to a child or puppy that then grows to be an adult). Therefore the subversion or even destruction of a muse personality is sometimes even equated with rape or murder. \end{quotation} 



\subsection{Software Minds} 

At their core, infomorphs are just programs and so they are treated like other software in terms of rules. They must be run on a specific personal computer or server (see \textit{Computer Capabilities,} p. 247). If that device is shut down, the infomorph also shuts down into a state of unconsciousness, restarting along with the device (infomorphs may also shut themselves down, though it is rare that they do so). If the device is destroyed, the infomorph is killed along with it (unless their data can somehow be extracted from any surviving components, perhaps resulting in a \textit{vapor,} p. 274). Infomorphs may copy themselves, though in some places this is illegal and in most places is frowned upon as it raises numerous ethical and legal questions. For this reason many infomorphs that copy and transfer themself to run on a new device will thoroughly erase themselves off the old one. 

As digital beings, infomorphs have no physical mind, but it is a simple matter for them to possess an uninhabited synthmorph, taking up residence in the cyberbrain (see \textit{Resleeving Synthmorphs,} p. 271). They may also download into biomorph bodies according to standard resleeving rules (p. 271). Even when disembodied, they may interact with the physical world via the mesh, viewing through sensors, streaming XP feeds, communicating with characters, commanding slaved devices, and teleoperating/jamming drones. Infomorphs have a Speed of 3, reflecting their digital nature and their ability to act at electronic speeds. If an infomorph sleeves into a body, however, it takes on the Speed of that morph. 



\begin{quotation} \textbf{INFOMORPHS AS MUSES} 

Instead of relying on underdeveloped muses for aid and companionship, characters may prefer to have a full-fledged digital intelligence at their side, whether that be an AGI, a backed-up biological ego, or fork of the character’s own personality. Alternately, a character with a ghostrider module (p. 307) could have both, carrying a muse in their mesh inserts and an infomorph in the ghostrider module. 

This possibility is very useful for infomorph player characters, as they can ride along in someone’s head and participate in team affairs without needing a morph of their own. \end{quotation} 

\subsection{AGI Characters} 

Though AGIs were not born in a biological body, their programming encompasses the full spectrum of human personality, outlook, emotions, and mental states. AGIs are in fact raised in a manner similar to human children so that they are socialized much like humans are. Nevertheless, on a fundamental level they are non-humans programmed to act human. There are inevitably points where the programming does not mask or alter the fact that AGIs often possess or develop personality traits and idiosyncrasies that are quite different from human norms and often outright alien. Unlike standard crippled AIs, AGIs are capable of full-fledged creativity, learning, and self-improvement (at a slow but steady pace equivalent to humans). Just like other characters, they earn Rez points and may improve their skills and capabilities. AGIs suffer none of the skill limitations placed on weak AIs, using skills just like any other character. On an emotional level, AGIs run emotional subroutines that are comparable to biological human emotions AGIs are, in fact, programmed to have empathy and share an interest in human affairs and prosperity, and to place significant relevance on life of all kinds. In game terms, AGIs emote like humans (and so Kinesics may be used against them) and are vulnerable to emotionally manipulative effects, fear, etc. 









