\documentclass[twocolumn]{book}

\usepackage[utf8]{inputenc}
\usepackage{fullpage}

\begin{document}

\title{Eclipse Phase}
\date{}

\maketitle

\tableofcontents

\begin{description} 

\item[Original Concept and Design] Rob Boyle, Brian Cross 

\item[Writing and Design] Lars Blumenstein, Rob Boyle, Brian Cross, Jack Graham, John Snead 

\item[Additional Writing] Bruce Baugh, Randall N. Bills, Davidson Cole, Tobias Wolter 

\item[Editing] Rob Boyle, Jason Hardy 

\item[Development] Rob Boyle 

\item[Line Developer] Rob Boyle 

\item[Art Direction] Randall N. Bills, Rob Boyle, Brent Evans, Mike Vaillancourt 

\item[Cover Art] Stephan Martiniere 

\item[Interior Art] Justin Albers, Rich Anderson, Davi Blight, Leanne Buckley, Robin Chyo, Daniel Clarke, Paul Davies, Nathan Geppert, Zachary Graves, Tariq Hassan, Thomas Jung, Sergey Kondratovich, Sean McMurchy, Dug Nation, Ben Newman, Justin Oaksford, Efrem Palacios, Sacha-Mikhail Roberts, Silver Saaramael, Daniel Stultz, Viktor Titov, Alexandre Tuis, Bruno Werneck, and Dr. CM Wong (Opus Artz Studio) 

\item[Graphic Design and Layout] Adam Jury, Mike Vaillancourt 

\item[Faction Logos] Michaela Eaves, Jack Graham, Hal Mangold, Adam Jury 

\item[Indexing] Rita Tatum 

\item[Additional Advice and Input] Robert Derie, Adam Jury, Sally Kats, Christian Lonsing, Aaron Pavao, Andrew Peregrine, Kelly Ramsey, Malcolm Shepard, Marc Szodruch 

\item[Science Advice] Brian Graham, Matthew Hare, Ben Hyink, Mike Miller 

\item[Playtesting and Proofreading] Chris Adkins, Sean Beeb Laura Bienz, Echo Boyle, Berianne Bramman, Chuck Burhanna, C. Byrne, Nathaniel Dean, Joe Firrantello, Nik Gianozakos, Sven Gorny, Björn Grammatke, Aaron Grossman, Neil Hamre, Matthew Hare, Kristen Hartmann, Ken Horner, Dominique Immora, Stephen Jarjoura, Lorien Jasny, Jan-Hendrik Kalusche, Austin Karpola, Robert Kyle, Tony Lee, Heather Lozier, Jürgen Mayer, Darlene Morgan, Trey Palmer, Matt Phillips, Aaron Pollyea, Melissa Rapp, Jan Rüther, Björn Schmidt, Michael Schulz, Brandie Tarvin, Kevin Tyska, Liam Ward, Charles Wilson, Kevin Wortman, plus everyone who participated in a game at Gen Con 2008 

\item[Musical Inspiration] Geomatic (Blue Beam), Memmaker (How to Enlist in a Robot Uprising), Monstrum Sepsis (Movement) 

\end{description} 

\pagebreak



\paragraph{Dedication} 

This book is dedicated first and foremost to the people who made Eclipse Phase happen, from everyone who contributed time, sweat, ideas, and money into it to everyone who picks it up, reads it, and plays it. This game is by you and for you. Secondly it's dedicated to my grandmother and to Andrea, both important people in my life who died while I was working on this book and its themes of defeating death. I sincerely hope that one day such tragic losses are avoided. Third, this book is dedicated to my son Echo, my entertaining working companion on this project. Finally this book is dedicated to those visionaries, especially the anarchists and transhumanists, who are working to bring about a fantastic future, starting now. —Rob Boyle 

\paragraph{Our Resources} 

\end{itemize} \url \url{http://eclipsephase.com}{Eclipse Phase website and blog} \url \url{http://del.icio.us/infomorph}{Eclipse Phase-related news and links} \url \url{http://posthumanstudios.com}{Posthuman Studios website} \url \url{http://catalystgamelabs.com}{Catalyst Game Labs website} \url \url{http://eclipsephase.com/store}{Online Ordering and PDFs} \end{itemize} 

\paragraph{Outside Resources} \label{sec:outside-resources} 

\end{itemize} \url \url{http://www.humanityplus.org}{Humanity Plus} \end{itemize} 

First Printing by Catalyst Game Labs, an imprint of InMediaRes Productions, LLC PMB 202 - 303 - 91st Ave. NE, G-701 Lake Stevens, WA 98258. Printed in China 

Creative Commons License; Some Rights Reserved. 

This work is licensed under the Creative Commons Attribution- Noncommercial-Share Alike 3.0 Unported License. 

To view a copy of this license, visit: \url{http://creativecommons.org/licenses/by-nc-sa/3.0/}{} or send a letter to: 

\end{center} Creative Commons, 171 Second Street, Suite 300, San Francisco, California, 94105, USA. \end{center} 

(What this means is that you are free to copy, share, and remix the text and artwork within this book under the following conditions: 

\end{enumerate} \item you do so only for noncommercial purposes; \item you attribute Posthuman Studios; \item you license any derivatives under the same license. \end{enumerate} 

For specific details, appropriate credits, and updates/changes to this license, please see: 

\end{center} \url{http://eclipsephase.com/cclicense}{}) \end{center} 












\chapter{Manque} \label{chap:lack} 

"Quel jour on est?" 

Les mots transpercent mes nouvelles cordes vocalles et décapent ma gorge sèche en sortant. Comme prévu, ma diction est mauvaise, comme toujours durant les quelques minutes suivant une réincarnation. Le timbre de la voix est apparent en dépit du hachage, du décapage des mots. C'est une biomorph, définitivement, et mon nouveau sexe est féminin. C'est tout ce que je peux savoir dans les premières secondes. Le modèle m'échappe encore, mais je le saurai de manière certaine bien assez tôt, dés que j'aurai récupéré le contrôle du mouvement. Je parierai sur une autre morph fury. 

La paillasse du laboratoire est dure. Guère plus que du métal froid enrobé dans du synthé-plastique blanc et crissant. C'est l'Installation type d'une maison de poupées corpo. Le froid transperce ma peau et s'enroule autour de mes os. 

Un casseur de corticalle se penche vers moi, agitant son rayon de bienvenue d'un côté à l'autre, attirant mes pupilles occasionnellement. Son visage suffisant, ennuyeux, parle: "Conscience confirmée." Le rayon s'éteint d'un clic. Ma question aurait du rendre évident le fait que j'étais de retour, mais le mec est un esclave de la procédure. Ils le sont tous. Les banques de corps corporatistes aprécient que leurs employés soient paralysés par l'obéissance, incapable de penser par eux-mêmes. Je marmonne encore une fois la question. "Quel jour on est?" 

"Le 11 Mars." 

"Combien de temps après la Chute?" 

"Vous êtes sérieux?" 

\end{quotation} Je suis paranoïaque. Oui. Je doit connaître l'année à chaque fois que je retourne sur la table d'une banque de corps. La paranoïa est juste une des plaies à laquelle doit faire face la transhumanité ces jours-ci. \end{quotation} 

En général, j'essaye de récupérer les données des inserts de mesh de ma nouvelle inarnation avant de parler. Pas de chance. Demander l'année à un technicien d'incarnation est toujours humiliant. Je passe pour un amateur, mais les circonstances sont réellement exténuantes, donc je lui mets la pression. Fort. 

"Répond à la putain de question." 

Le corpo indolent me fixe de son strabisme avant de répondre. 

"Euh ... 10 AF. Vous n'êtes pas parti si longtemps. Votre dernière sauvegarde ... " 

Il scanne son entoptique pour l'info. 

" ... date d'il y a 14 jours et 7 h." 

L'info mets une seconde avant de pénétrer, mais lorsque elle y arrive, ça pique. Ça ne cessera jamais de me donner une claque lorsque le temps m'échappe. Deux semaines. Disparues. Complètement supprimées de mon existence. Il y a deux semaines, c'était un autre moi, incarné dans une autre morph. Il y a eu une mission qui a mené à ma mort. C'est tout ce que je sait. Soit Firewall a échoué à récupérer la pile corticale du cadavre afin que je puisse me rappeler de ces deux semaines, où les enculés ont délibérement choisis d'effacer cette période. Honnètement, cependant, les deux options sont préférables à avoir un autre moi se balandant à l'extérieur, en traffiquant on ne sait quelle conneries. Quelques mecs sombrent en ayant de multiples eux errant un peu partout, mais mon ego est robuste. L'Univers ne supporterait pas un autre Sava. 

Merde. Mon cerveau se barre dans les territoires de la morosité; c'ets toujours pareil pendant les quelques minutes suivant une réincarnation. J'ai besoin d'un contexte physique. Quelque chose de tangible sur lequel me concentrer. Je ramène mes mains en face de mes yeux, j'ai l'impression que mes bras sont comme deux sacs d'une tonne de pierre. Les doigts sont fins et long; les phalanges sont calleuses, couturées de cicatrices et irrégulières. A l'évidence, le travail d'un paquet de coups, de poings rencontrant des machoires, du métal, de la chair. Ouaip. Une morph fury qui a déjà bien vécu. Tu as ce pour quoi tu payes, je suppose; ou ce pour quoi Firewall veut bien payer. Pourquoi je fait ça? Du point de vue de l'organisation, je ne suis rien de plus qu'un instrument de précision bon marché, balancé au recyclage quand je me fait couper en deux. Il y aura toujours d'autres moi, jusqu'à ce que l'horreur soit trop intense, jusqu'à ce que les fichiers soient trop corrompus, jusqu'à ce que j'en sache trop et que Firewall décide de se débarasser de moi, alors un autre gars s'avancera pour préserver la transhumanité. Préserver la transhumanité. Génial. Maintenant je parles comme un discours de propagande deFirewall! Mes bras faiblissent et tombent mollement sur le côté. La force n'est pas encore là. Encore quelques minutes seul avec mes pensées. 

En sortant du hangar de récupération, le casseur de cortical se moque de ma pitoyable tentative de mouvement. "Pourquoi vous presser?" me dit-il. "Reposez-vous, ok? Vous vous effondrez sur le sol, vous y resterez jusqu'à ce que vous puissiez vous relevez. Ils ne me payent pas suffisament pour surveiller les débutants."  Sa légèreté n'aide pas mon humeur, et la mélancolie revient. 

Quelles expériences ne font plus parties de ma conscience? Peut-être le frisson d'une vie. Ai-je découvert la vraie beauté? Suis-je tombé amoureux? Ai-je eu une révélation? Ai-je sauvé une vie? Je ne le saurai jamais. Ces souvenirs, cette vie, cette version de moi, ont disparus. Le nouveau moi, allongé sur cette paillasse, n'a jamais été façonné par ces expériences. Mon torse se creuse sous le poids de cette perte. 

Je doit changer de façon de penser. 

Putain. Peut-être qu'il n'y avait pas de joie, pas de révélations. C'était deux semaines d'emmerde. Je suis certain de ça. Je suis mort d'ennui. Encore mieux, je me suis fait jeter et j'ai souffert d'un chagrin d'amour épique. Ma mort n'avait aucun sens. J'ai fait une overdose de kick, me roulant par terre dans une pathétique frénésie d'excitation jusqu'à ce que mon cœur explose. J'ai été étripé par des racailles fauchées dans un corridor sombre d'une station obscure pour de l'XP basse résolution au marché noir. Je suis heureux que le temps ait été supprimé. Extatique en fait. Putain. Je les emmerde. Je n'ai pas besoin de ces deux semaines. 

Mais ces pensées sont des mensonges. J'ai besoin de ces deux semaines. Je ne me sent pas complet sans elles. Bordel, je me sent incomplet si même une heure est sacrifiée. Je doit savoir. 

Quelqu'un sait ce qu'il s'est passé. Sans aucun doute. Un proxy de Firewall à priori, Jesper trés probablement. C'était ma connexion pour cette mission. Je me rappelle de ça. C'est lui qui a du demander l'effacement. Et les proxys ont la gachette facile quand il s'agît de nous sacirifier nous autres sentinelles. Même un score de rep durement gagné ne peux sauver mes souvenirs quand Firewall juge les résultats d'une mission trop sensible pour qu'un voyou de seconde zone comme moi puisse les posséder. Tant que le putain de boulot est fait. Tant que la transhumanité persévère. 

Quel arrangement de merde. 

Comment ma vie, mes vies, en sont venues à ça? Toujours entre les mains des autres. 

Et revoilà la crainte, la paranoïa Je doit passer outre. Je doit donner à l'organisation le bénéfice du doute. Ça fait des décennies que je suis une sentinelle. J'aime penser que j'ai sauver des millions de vie, mais je n'en suis pas sûr. 

Est-ce que je fait confiance à l'organisation? Non. Mais il y a une compréhension, un certain degré de respect. Cela dit, si les années continuent de filer, si les vides deviennent de plus en plus grand et de plus en plus fréquent, je commencerai à douter de la volonté de Firewall à me préserver. 

Soudainement, ma muse débarque, interrompant mes rêves sombres. Plusieurs affichages entoptiques apparaissent dans mon champ de vision, basculant entre diverses routines de diagnostique au moment où mes insert de mesh parviennent enfin à se connecter. La voix familière et féminine de Careza pénètre mon esprit. 

[Bienvenue ici, Sava.] 

Le son est apaisant; comme d'être bercé par sa mère, ou embrasser par une amante. Le package de mise à jour harmonique valait réellement le coût. Careza a appris à l'utiliser au mieux. Je pense rarement à ma muse comme à une IA. C'est ma seule réelle amie ces derniers temps. Je me demande si elle partage ce sentiment. Je n'ai jamais laissé cette pensée sortir. Je la garde pour moi. J'ai peur de ce que pourrait être la réponse. 

Hey, Careza. Content d'être de retour. 

[Je suppose que tu aurai bien besoin d'un verre?] 

Tu me connais trop bien, Car. Mieux que je ne me connait moi-même. 

[Nos hôtes ont maintenant reçu la demande. Temps d'attente, à peu près dix minutes.] 

Merci. Careza apprécie nos conversation lorsque mon cerveau subit un léger bourdonnement. Elle essaye toujours de me rendre ivre. 

(De rien, Sava. Avant que tu ne demandes, ça fait deux semaines. Je n'ai pas d'information sur ce qui est arrivé suite à notre dernière réincarnation. Actuellement, nous sommes en orbite lunaire à bor du Selardi IV. Nous sommes équipés d'une morph fury CoreCorp, avec des améliorations mineures. Elles seront en-ligne sous peu. Je suis heureuse de t'annoncer que les Titaniens ont gagné et remporté la Coupe.] 

Merde. J'aurai du parier sur celui-là. Quelle était la côte sur ça? Mais avant que Careza ne me déterres l'info, j'annule l'opération. Attends. Non. Je ne veux pas savoir. Ça ne ferait que m'irriter un peu plus. Une impulsion nerveuse commence à gratter mon système nerveux entier et un parfum épais et familier commence à enrober ma langue. J'ai besoin d'une cigarette. 

[Oui. Je sait. Le précédant occupant de cette morph était un gros fumeur. L'addiction risque d'être difficile à passer cette fois.] 

Cette réincarnation s'améliore à chaque minutes. Je déteste fumer. L'alcool, ok. Je peux gérer mon alcool, mais fumer me donne toujours l'impression d'être une merde. A chaque fois que je suis incarné dans une morph accro à la nicotine, je me démènes pour décrocher. Careza continue son rapport pendant que j'essayes de préserver ma santé mentale d'un manque de nicotine intense. 

[Ton @-rep est restée intacte.] 

Enfin quelques bonnes nouvelles. Au moins, je n'ai pas énervé d'alliés lors de ces deux dernières emaines. 

[En effet. Es-tu dans le bon état d'esprit pour une mise à jour sur Rati?] 

Rati est ma passion L'amante que je chéri plus que tous les autres. Elle a disapru il y a deux ans. Sans explications. La blessure pique encore. 

On garde la mise à jour pour plus tard, Careza. 

[Compris.] 

Lance une analyse des newsfeed. Cherche pour tous les incidents majeurs des deux dernières semaines. Il y a peut-être un indice sur ce que nous avons peut-être fait. 

Pendant que Careza lance le scan et continue son rapsit, je bascule mon attention sur ma nouvelle incarnation. La force nécessaire à se lever est enfin là. Je relève la morph et je balance les pieds sur le sol. Des spasmes traversent chaque muscle. Les nouvelles morphs nécessitent toujours un peu de temps pour s'acclimater. Coup de bol, je suis un habitué des fury CoreCorp, j'en ai incarné quelques unes par le passé. Celle-là est comme une vieille paire de chsaussures, un peu usé et déformée, mais capable de marcher des heures si nécessaire. La cheville gauche est un peu tendue. Je la remonte un peu pour y jetter un œil. Légèrement enflée. Manifestement pas une dysmorphie de nouvelle incarnation. Probablement une vieille blessure. Encore une emmerde, mais on a ce pour quoi on paye, je suppose. Le nanotatouage qui encercle le biceps droit est rude et odieux, même selon les standard de la racaille - un slitheroïde pénétrant intégralement dans le sexe d'un pod de plaisir féminin, entièrement animé. Quiconque a gravé ceci sur la morph est trés raffiné. Je déteste les marques d'identification, mais encore une fois, si on ne peut pas s'offrir une morph propre, on fait avec ce qu'on a. 

Je glisse de la table, en faisant attention à ne pas tomber, et je teste précautionnesement ma cheville. Douloureux, mais elle ne lâchera pas. 

Envoies une requète pour un bandage, cheville gauche. De la bute serait nicquel. 

[Phenylbutazone. Ça arrive. Et le cocktail sera là dans trente secondes, approximativement. Rien de particulier sur l'analyse des newsfeed.] 

Les chiffres. 

\end{quotation} Je chemines vers le mirroir de plain pied, qu'on retrouve en standard dans les chambres de réveil de réincarnation, et je jette le drap pour regarder le nouveau moi. Je surveille le broyeur de corticale attendant sur le pas de la porte, mon cocktail à la main, mattant mon corps d'un regard appréciateur. Je ne me reconnait pas. \end{quotation} 

"Amenez moi mon verre s'il vous plaît." Je tends ma main dans sa direction sans même signifier sa présence. Il rentre dans la pièce, trop proche de moi, et glisse le verre dans ma main. Son haleine empeste la saucisse fraîche. 

"Vous êtes pas trop mal sous ce drap, n'est-ce pas?" dit-il "J'ai jetté un œil tout à l'heure, mais je dois dire que la paillasse ne vous rends pas justice. Sur pieds, les courbes apparaissent pleinement. Votre visage n'est pas si sympa, mais cette poitrine est ... " 

Je l'interromp avant de vomir de la bile dans ma bouche. "C'est exquis. Je sait. Maintenant fermes là et dégages avant que je ne t'arraches la peau de ta tronche et que je te fouettes avec." Il comprend le message et se glisse hors de la pièce. 

C'est une belle poitrine. 

[Si belle est définie par la proportion, alors je dirait oui.] 

Les IAs, toujours aussi formelles. 

[Tu es plus grand d'environ quatre centimètres que ta proprioception habituelle, donc fait attention à ta tête.] 

Je tâcherai de faire profil bas alors. 

[Elle était nulle.] 

Ouaip. Ouaip. Je sait. Un sourrire vient éclairer mon visage alors que la discussion avec ma muse améliore mon humeur. En regardant dans le miroir, j'essaye d'agrandir le sourrire, pour avoir une meilleure perception de mon nouveau visage. Je montre les dents. Des traces de nicotines sont visibles. Je boit une longue gorgée de mon cocktail, laissant les parfums de l'alcool se développer un peu. Je peux sentir mon sang réagir instantanément. Je ferme les yeux et je laisse échapper un soupir. Juste quelques instants de calme, c'est tout ce que je demande. 

[Nous avons un invité, Sava.] Merde. Je n'aurai pas cette chance. 

Qui? 

[Notre dernier proxy Firewall, Jesper, a envoyé un fork béta de lui. Il est quelque peu impatient de te parler.] 

Connecte le. 

Ils ne peuvent pas me laisser seuls, hein? Officiellement, Firewall n'existe même pas. C'est à cause de Rati qu'ils ont leur tentacules enroulés autour de moi, à l'intérieur de moi. Tout le bordel sur Mars. C'est là que tout à commencé. La dernière fois que j'ai vu Rati. Tout ce savoir qu'ils m'ont autorisés à garder. Mais pourquoi? Jusqu'à ce jour, je n'avai jamais réalisé à quel point l'univers était réellement effrayant. Non, pas effayant. Terrifiant. Il n'y a pas d'autres mots pour quelque chose de si vaste, de si indifférent. La transhumanité pourrait être entièrement détruite, et tout continuerait comme si de rien n'était. Terrifiant. Il n'y a pas d'autres manières d'expliquer ce que l'on ressent lorsque l'on se trouve face à face avec des choses réellement au-delà de la compréhension. L'enfer, aucun autre mot ne pourrait englober les actions de la transhumanité envers les siens - à peu près autant que ce que les autres espèces errant dans le vide ont en stock. C'est peut-être pour ça. Pour m'apprendre une leçon. Pour êre sûr que je n'oublierai jamais, et que je ne cesserai jamais d'aider l'organisation, car même le plus petit aperçu de ce qu'il y a là dehors est suffisant. 

Le fork de Jesper se matérialise dans mon champ de vision. 

[Bienvenue ici, Sava.] 

Va te faire foutre, Jesper. Tu sait que je déteste me réveiller avec un manque. 

[Désolé. Rien que je n'ai pu faire.] Son expression est sérieuse et concernée, mais sa kinésique indique qu'il est aussi calme qu'il puisse l'être. Quel acteur! Ces connard de proxys ne paniquent jamais. Ils ont toutes les cartes et ce n'est jamais leurs esprits qui sont exposés. 

Ouaip. C'est ça. Accouche. Tu ne m'as pas fait réincarner dans une morph de combat pour profitter d'une perm, donc tu dois avoir quelque chose de sérieux sous le coude. Berk, Pivo et Sarlo sont là? 

[Oui, ils ont été réincarné dans le même bâtiment.] 

Au moins, j'aurai mon équipe. Des gens sur qui je peux compter. Jusqu'à un certain point. 

Très bien. Quels sont les détails? 

\end{center} *** \end{center} 

Pivo attrapa la surface externe lisse de la station avec ses huits bras. Les nano-aimants placés à l'extrémité des bras de son exocombinaison sont la seule différence entre une prise sécurisée et une dérive infinie dans les profondeurs de l'espace. Il regarda la sphère noire au-dessus de lui à travers sa visière . 

La Terre. 

Ses yeux plongèrent dans l'étendue noire d'un océan mort à travers des nuages menaçants. Pivo avait envie de nager dans ces profondeures antiques. Né et élevé dans l'espace, il ne s'était jamais immergé dans l'ancienne niche écologique de son espèce. Les évènements ne semblaient pas lui permettre de pouvoir plonger dans les eaux salées d'un océan Terrien un jour. La planète était maintenant un piège mortel. Une terre dévastée remplie de formes squelettique. 

Il s'imagina une époque avant la Chute, lorsque ses ancètres se propulsaient dans une eau d'un bleu profond et glissaient sans effort dans les labyrinthes de corails ou se laissaient tranquillement porter par le courant, sans se soucier du poids de la sapience. Peut-être que les pieuvres survivent toujours sous les eaux sombres du présent, gagnant leur brève existence, prenant leur temps, préservant l'espèce intacte et vivante jusqu'à ce que la Terre puisse être réclamée, et Pivo les rejoindrait en ce jour glorieux, abandonnant en même temps le savoir, et retournant sur les chemins de l'instinct. 

Les capteurs de son exocombi interrompirent le rêve de Pivo, détectant une lumière laser baignant sa silhouette - contact de Sava par un lien laser. C'était la méthode de communication préférée lorsqu'une mission nécessitait de la discrétion. La muse de Pivo traita le message, et la voix de Sava entra dans sa têet. 

[Quelque chose cloche? Pourquoi as-tu arrété de bouger?] 

[Je profitais juste de la vue,] retransmis Pivo. 

[Tu en profiteras pendant la descente, pendant des heures si tu veux. Rentre dans la station avant qu'un de ces bots sentinelle ne nous trouve.] 

Pivo ne prit pas la peine de répondre. Il n'y avait pas de discussion possible avec Sava. Inutile de défendre ses actions. Pivo se remit à ramper le long de la coque de la station. La station elle-même était attachée à l'extrémité d'un long nanotube de carbone noir, qui s'étirait jusqu'à la surface de la planète - le seul ascenseur spatial qui ai survécu. 

Pivo localisa la brèche, une fine cicatrice dans la coque métallique de la station, le résultat d'une explosion interne responsable de la mort de la station durant la Chute. La brèche était exactement là où Sava avait dit qu'elle serait et la description de sa taille était parfaitement exacte: un trou suffisament large pour qu'un enfant humain passe au travers. D'après Sava, des années auparavant, le nanosystème d'auto-réparation qui opérait dans le métal de la coque a subi une panne avant que la brèche ne soit complètement réparée. Le niveau de détails sur la mission que Sava parvenait à extraire de Firewall était effrayant par moment. La Paranoia éclot pendant un moment, mais il abandonna rapidement sa suspicion, comprima sa forme de céphalopode, et propulsa son corps à travers la brèche. 

Dans l'obscurité, Pivo activa son émetteur infrarouge, éclairant la pièce d'une lumière hors du spectre de vision normal. L'intérieur de la station sans-vie devint visible à ses yeux augmenté permettant de percevoir les teintes étranges des infrarouges. Pivo préfèrait presque le noir. Des cristaux de glace brillaient sur toute les faces, le résultat de moisissures instantanément gelées dans l'atmosphère absente depuis longtemps. Des groupements froids de restes humains flottaient au milieu des morceaux métalliques de la coque dans un ballet macabre en gravité zéro. Pivo flotta à travers le massacre et le gore, appuyant légèrement sur la chair ou le métal pour se frayer un chemin dans la pièce. Une tête de femme dériva lentement à côté, le visage gelé dans un cri silencieux et béant. Une pile corticale intacte pendait de la nuque tranchée. Un instant, Pivo considéra récupérer la pile, mais il n'était pas là pour récupérer des âmes perdues. Au lieu de ça, il plaça deux de ses bras sur le dessus du crâne et la poussa en dessous de lui, vers le sol. Comme tant d'autres disparus pendant la Chute, cette personne restera oubliée ici. 

Pivo parvint au sas sans incidents, mais il savait que sa chance ne tarderai pas à tourner. Une confrontation avec des gardiens hypercorp dans une station abandonnée était inévitable. Les capteurs avaient probablement déjà détectés sa présence. C'était juste une question de temps avant que des bots ne convergent vers sa position. Il espérait juste que lorsque cela arrivera (et c'était plus que certain), cela se passerai après qu'il ait ouvert le sas et que le reste de l'équipe soit à bord de la station. 

Le sas avait été soudé de l'intérieur. Pivo était paré a cette éventualité, mais cela guarantirai sa détection par les gardiens. Il prit quelques secondes pour se regrouper, se concentrer sur la tâche à venir, pusi alluma la torche à plasma embarquée dans l'un des bras de son exocombi. Un sifflement chaud et de durs reflets bleutés emplirent la pièce. Les secondes sont maintenant un bien précieux. 

Il était presque arrivé à la porte interne lrosque sa muse le pinga avec un avertissement du capteur teraherz passif. Un objet s'approchait rapidement de la position de Pivo, a une vingtaine de mètre maintenant. Un bot sentinel serait bientôt sur lui. 

[J'ai presque fini la première porte,] transmi calmement Pivo, même si maintenir la torche stable nécessitait chaque once de son esprit. [J'ai de la compagnie. Préparez-vous.] 

[Bien reçu,] répondit Sava. 

Pivo finit par découper la joninture et à passer au travers. L'octomorph glissa quatre bras à travers le métal tranché encore fumant, et grâce à une tension brutale, arracha la porte de la structure. La porte flotta lentement au loin dans la chambre, les arrètes refroidissant rapidement. Le sas interne n'était pas soudé. Dans un soupir de soulagement, les huits bras de Pivo entamèrent un assaut frénétique des contrôles manuels du sas. 

[Encore quelques secondes. Juste quelques secondes.] Mais les secondes restantes étaient expirées. 

Dans son champ de vision à 360°, Pivo pouvait voir le bot de sécurité se propulser dans derrière lui. Le bot vida ses armes immédiatement, les tirs ricochant sur la porte flottante du sas. Le bot avança sur la porte, et dégagea l'obstruction au loin d'un coup rageur. Elle résonna sur la surface crystalline des murs. Juste au moment de tirer le dernier levier pour libérer la porte du sas, un plasma ardent l'engloba. 

\end{center} *** \end{center} 

Sava avait demandé à Careza d'injecter la neurochem au moment où la porte du sas s'ouvrirait. La muse n'as pas failli à la tâche. Dans ce qui sembla être un éternel ralenti pour le cerveau chargé de Sava, la porte du sas s'ouvri sur la station, aidée par le coup d'une jambe d'acier de Berk, les muscles de l'équipe. Dans un éclair de pensée, le radar de ciblage de Sava ouvrit un affichage entoptique et verrouilla deux cibles: Pivo et un bot sentinelle. Le chien de garde robotique étaient déjà en train de lever ses armes, mais Sava était plus rapide. Un feu de plasma guidé par sa rétine éxplosa hors de l'arme de Sava, cramant l'un des bras de Pivo et grillant le sentinelle. Un second tir traversa la carapace blindée du bot, grillant des composants critiques à l'intérieur, transformant le bot en une pile inutile de métal fondu. 

Sava dépassa rapidement l'octomorph maudissant et envoya deux tirs de plus dans le bot fumant. 

[On est OK,] transmis Sava. [Un de moins, mais il y en a toujours plus. Compte dessus. Pivo, ça va?] 

[Tu as brulé mon bras nouricier, puta.] répliqua Pivo dans une agitation manifeste, gargouillant dans les harmoniques. 

[Tu préfères que je te laisses au bot la prochaine fois?] Sava se tourna vers Sarlo. [Sarlo, rentre là-dedans et trouves la console dont tu as besoin. Berk, on va avoir besoin de préparer des positiosn défensives, pour donner à hacker boy le temps de faire son truc.] 

Pivo découpa son exocombi et détacha le bras endommagé, maudissant Sava en chuchottant pendant qu'elle se réparait rapidement et comblant le trou. 

[Hé. T'inquiètes pas, Pivo. Il t'en reste sept autres. De plus, tu ne m'as jamais paru branché nourice de toute manière.) Sava aimait se moquer de Pivo. C'était l'une des réelles joie de sa vie. 

Se propulsant d'un mur à l'autre, Sarlo traversa la pièce avec grâce et facilité. Sa morph néoténique était plus légère et plutôt plus petite que la moyenne des envelloppes d'enfants humains, complètement augmentée et traffiquée pour correspondre à ses "préférences". Il avait payé une fortune pour ça. Les autres ne comprendront jamais le panchant de Sarlo pour les incarnation humaines et juvéniles, au point qu'il a toujours utilisé ses fonds propres pour s'assurer d'une réincanration néoténique, même lorsque Firewall payait la note. Ils ne savaient pas non plus d'où venanit son flux apparement inifini de fonds personnels, ils ne voulaient pas savoir. Tant qu'il s'arrange pour que le boulot soit fait. 

Deux minidrones suivaient Sarlo, éclairant la zone aux infrarouges et scannant activement d'autres longueurs d'ondes. [Par là,] dit-il, transmettant une carte entoptique sur chacun des suraffichages des membres de l'équipe. [C'est pas loin, à cent mètres, plus ou moins.] Une route mise en évidence apparu sur la carte. 

Sava et Pivo suivaient pas trop loin derrière Sarlo, tandis que Berk luttait pour suivre le rythme dans sa coquille gynoïde blindée. 

[Accroches-toi, bas de plancher. Nous serons ramenés à la gravité bien suffisamment tôt,) transmis Sava à Berk. 

[Pas assez tôt à mon goût,] répliqua Berk. 

La station abandonnée était étrangement silencieuse. Des signes de violence depuis longtemps oubliée et de désespoir s'attardait partout. Débris flottants Corps brisés et gelés. Trace de brûlure et métal tordu La mort possèdait cet endroit. 

Lorsque l'équipe atteignit la station de contrôle, Sava et Berk prirent des positions défensives dans le corridor pendant que Sarlo et Pivo travaillaient sur les systèmes dormants de la station. 

[Que je soit maudit! Le briefing de mission était effectivement correct. Les systèmes de la station sont actifs mais en veille. Quelquesoit ce qui gardait la place, ça n'a pas détruit le système, ils ont gardé ouverte la possibilité que l'ascensseur spatial puisse être réactivé.) Sarlo commença avec entrain les procédures nécessaires à hacker le système. 

[Quel connard voudrait prendre le risque de descendre sur cette boule de cendre?) balança Berk. 

Pivo agita l'un de ses bras. [Dois-je vous rappeler que certains d'entre nous aiment à penser que récupérer notre planète natale est une bonne idée?] 

[Pensée réactionnaire, si vous voulez mon avis.) répondit Berk. [Abandonner toutes nos vieilles loyautés des nations-états est l'une des meilleurs étapes que la transhumanité n'est jamais franchie. Laisse la délectation des gloires du passé aux bio-cons. Je choisi un futur dans lequel nous avançons fièrement dans l'espace, merci bien.) 

[Arrétez avec la politique.] Sava se tourna vers Beck. [Tu es un anarchiste, ok j'ai compris.] Puis Sava se tourna vers Pivo [Et toi, tu  es un réclamationniste. Parfait.] Mais le sermon de Sava fut interrompu par une demi douzaine de points se déplaçant rapidement sur l'entoptique du radar de l'équipe. [Pings en approche. Sarlo, t'es dedans?] 

[J'y travailles. Putain. Putain. Merde.] La voix enfantine de Sarlo devenait irritante. 

[Bosses plus vite. Si ces bots ont de l'artillerie lourde, on est niqués.) Sava et Berk envoyèrent tous les deux un tir suppressif dans leurs zone de corridor respectifves avant que les bots n'aient franchis les angles. Les bots stopèrent leur approche momentanément, se mettant à couvert juste avant l'angle. Plus de points commencèrent à apparaître sur le radar, se déplaçant vers la position des premiers. 

[On va commencer à manquer de temps, Sar! D'autres bots se rassemblent!] Sava balança une autre salve de tir suppressif dans l'angle. Berk garda son arme silencieuse, attendant qu'un bot se déplace dans le corridor avant de l'allumer, mais ils restaient planqués. D'autre se rassemblèrent et encore d'autres apparurent sur le radar, se déplaçant vers la même position. 

[Ils seront sur nous dans une seconde là!] 

[Considérez ça comme un cadeau, mesdames et messieurs ... ] Et d'une dernière opération, Sarlo prit le contrôle de tout le système de sécurité de la station. 

Soudainement, l'un des bots se tourna contre les autres. Un autre le joignit bientôt. En l'espace de quelques secondes, fumées et débris dérivaient le long du corridor alors que les bots entraient dans une guerre ouverte interne. Sava et Berk baissèrent leurs armes et admirèrent le son de l'ouvrage de Sarlo. 

[Punaise, Sar! Je suppose que c'est pour ça que tu es un des meilleurs hacker dans le système!] 

[Applaudis et vénères moi, sac d'os!] 

[Lorsque tu as des exploits à la pointe, fournit par les IAGs de l'e-leet codant sur Extropia, il n'y a pas grand chose qu'on ne puisse faire.) Sarlo balnça sa réplique dans les harmoniques du calme, mais Sava observait sa kinésique, et elels étaient hors limite. Le petit cœur néoténique battait à se rompre. Sava choisi de ne pas exposer ses couilles imberbes, et préfra laisser à Sarlo son petit moment de célébrité. Ça avait été "juste à temps," un autre coup comme ça pourrait bien ne pas bien finir pour eux. 

Sava accorda quelques secondes de calme et de récupération à l'équipe avant de reprendre le job. [Sarlo, Encore combien de temps avant que l'ascensseur soit actif?] 

\end{center} *** \end{center} 

Pivo resta à regarder par la fenêtre pendant qu'ils descendaient à travers la couche de nuage de suie, puis la Terre apparue en dessous d'eux. Ils étaient dans l'atmosphère maintenant, descendant le long d'une perche tendue entre la Terre et la station, un gigantesque ouvrage d'ingénierie entièrement en nanotubes de carbones. La nacelle glissait le long du cable de l'ascenseur, les rapprochant de plus en plus de la planète dévastée. 

L'atmosphère terrestre était maintenant saturée d'une fine poussière, de couleur rouille. Les vents, qui pouvaient atteindre des vitesses vertigineuses, fouettaient la surface de la planète, tournoyant dangereusement dans certaines poches. Le système de contrôle météorologique mondial avait été irrémédiablement ravagé par la Chute, lorsque la transhumanité avait failli disparaître suite à une guerre avec un groupe d'IAs malicieuse connues sous le nom des TITANs. Bombes, incendies, attaques chimiques, guerres bactériologiques, essaim de nanites voraces - même des têtes nucléaires - avaient prélevées leur tribu. C'était maintenant un endroit inhospitalier, plongé dans un hiver nucléaire. Certains nuages avaient une forme inhabituelle, défiant les vents de haute altitude, paraissant se tordre lorsqu'ils se déplaçaient - la descendance prospère des essaims de nanites aéroportés auto-réplicants, suspecta Pivo. Qui sait quelles autres monstruosités attendaient qu'ils arrivent là-bas en dessous, ayant évolué depuis les rémanences des machines de guerres des IAs? 

La Terre était hors-limite maintenant. Abandonnée à l'ennemi. Même si l'on présumait que les TITANs étaient partis depuis bien longtemps, s'échappant du système solaire par des portes de trou de ver construites en secret, emmenant des millions d'esprits transhumains forcés à s'uploader avec eux - ils avaient laisser quantités de leurs outils et armes derrière eux. Quelques armes que la transhumanité avait libérées sur les IAs, - et, assez souvent, sur eux-même - avaient également prélevé leur tribut en vies. La Terre avait donc été abandonnée et interdite, avec des satellites tueurs hypercorp partout en orbite pour tirer sur tout ce qui tenterait de quitter ou d'atterir sur la surface de la planète. 

En tant que réclamationniste, Pivo faisait parti d'une petite faction qui faisait beaucoup parler d'elle et qui préchait pour le retour sur Terre. Ils pensaient qu'il y avait toujours de l'espoir pour la planète. Il avait toujours persisté, et ce n'était pas le moment d'abandonner. La Terre avait besoin d'être nettoyée et terraformée, pour ressusciter la planète mère de la transhumanité. Mais les réclamationistes étaient une minorité. Pour la majorité des survivants de la Chute, la Terre représentait trop de souvenirs horribles. Des vies ruinées. Des êtres aimés disparus. Leur propre morts. C'était un monument à l'arrogance et aus erreurs de la transhumanité, un rappel sinistre qu'ils ne sont pas au dessus de l'autodestruction en dépit de tout le progrès et de toute leur technologies, ou peut-être à cause d'eux. 

Bien entendu, cela n'empêchait pas certains d'essayer. Des charoganrds effectuaient toujours des raids sur les ruines de la planète, récupérant des trésors perdus depuis longtemps, des artefacts culturels, ou même ceux qui ont n'ont pas réussi à s'échapper sous forme d'esprit digitalisés. Quelques réclamationniste ont lancés leur propres missions secrètes, cherchant à établir des camps de base depuis lesquels ils pourraient démarrer leur propre projet réclamationnistes. On n'entendit plus parler de la plupart d'entre eux. 

L'équipe de quatre se reposa et prépara son équipement dans le grand salon, Sava et Sarlo étaient dans des bulles de survie exigües afin de pouvoir échapper aux limites de leurs exocombinaisons quelques instants. Pivo avait choisi de rester hors de la bulle et dans l'exocombi. Rester enfermé avec Sava pendant la descente n'aurai pas été une partie de plaisir. Les murs du salon étaient enduits de sang vieux de plusieurs décennies, maintenant gelés en un crystal brun dans la cabine dépressurisée. Qui qu'ait été les derniers passagers à utiliser la navette, fuyant la Terre condamnée, ils avaient du s'en prendre violemment les uns aux autres, motivés par la folie ou le désespoir. 

[Je me demande à quoi ça devait ressembler.) Lança Sarlo à la cantonade. 

[Quoi?] Répondit Pivo. 

Sava intervint rapidement et mis un terme à la discussion que Sarlo essayait de démarrer. [Arrètez avec la philosophie et la dramatisation. Vous savez trés bien que je ne supporte pas cette merde.) Sava essayait deséspérement de maintenir l'ordre et de garder un air détaché. C'était trop facile de laisser le cerveau se ballader dans le passé et le destin des millions de personnes qui sont mortes pendant la Chute. Pour contrer ça, Sava s'en remettait toujours à la diatribe. [Ecoutez. Nous connaissons tous les détails de la mission. On va localiser quelqu'un. Un coursier. Très probablement un cadavre. Sa dernière position connue quand il était en vie était la station d'arrivée à laquelle nous allons nous poser quand le trajet sera terminé. Le Mont Killimanjaro. Qui, d'après des source de confiance, a été envahi par des robots tueurs, qui seront très probablement toujours dans les parages.) Sava marqua une pause dramatique avant de reprendre. [On récupère quelque chose du coursier. On ne sait pas quoi. Juste que ça a une forte valeur pour l'organisation. On en reste à ce qu'on sait. Je ne veux plus entendre de conneries telles que "et si" ou "je me demande." Si vos pensée sont ailleurs que sur la mission, gardez les pour vous. Je ne veux pas les entendre.) Et, sur cette déclaration, le reste du voyage vers la station du Killimanjaro  se fit dans le silence, chacun confiné à ses propres pensées, pas un ping ne circula entre eux. 

\end{center} *** \end{center} 

La navette vibra en s'arrétant dans un hangar caverneux et sombre. Il fût un temps où le hangar du Killimanjaro étaient le spatioport Terre/espace le plus actif au monde, faisant circuler des millions de clients chaque année. Pivo s'accrochait à une fenêtre de la nacette et scruta le vide obscur du hangar, c'était comme si l'endroit était un vide sans âme. 

[Je suis prêt quend tu l'es.) Envoya Sarlo à Sava, sur le point d'ouvrir la porte de la navette et de permettre à l'air terrestre stagnant et chargé de poussière de flotter sur l'équipe. Sava fit un signe de tête à Sarlo et la porte de la navette s'ouvri dans un bruit de décompression. Une poussière rouge-grise aveuglante jaillit dans la navette et recouvrit l'intérieur de la navette presque immédiatement. 

Le premier pas de Sava sur le sol du hangar du Killimanjaro le fit atterir fermement sur la cage thoracique cassante d'un squelette d'enfant. Les os volèrent en éclat et en poussières dans un craquement. Le sol entourant le sas de la navette était tapissé de squelettes entremélés à une masse de vêtements en lambeaux. Il n'était pas possible d'éviter de les écraser. Un par un, les autres sortirent du sas. 

[Cet endroit est un tombeau,] transmis Berk au groupe. 

(Toute la planète est un tombeau, ) répondit Sava, avec un écho harmonique supplémentaire permettant au mot tombeau de se prolonger  bien longtemps après que la phrase fût transmise, ajouté spécifiquement pour emmerder Pivo, qui coupa immédiatement l'écho dans sa tête avec une contremesure de sa muse. 

Sava fît encore quelques pas craquant puis s'arréta. Le reste de l'équipe lui emboîta le pas. 

(Il y a quelque chose qui cloche.) Sava donna un coup de pied dans l'un des squelettes. Les os s'ébranlèrent et se brisèrent. [Je ne voit aucun crâne.) 

[Upload forcé,) transmis Sava. (Les machines des TITANs ont récoltés les têtes des morts pour les scanner.) Il haussa les épaules. [Du moins, c'est ce que je pense.) 

[Ta gueule!) Sava imposa le silence à l'équipe. [Quelqu'un d'autre à entendu ça?) 

Un faible vrombissement mécanique résonna non loin. (Je l'entends.) répondit Pivo [Un peu plus loin, au nord. A peu près à trente mètres.) Comme en réponse à l'observation de Pivo, un autre vrombissement démarra, celui-ci derrière l'équipe, à l'extrémité sud-est du Hangar Un autre vrombissement venant de l'Est se joint au chœur. Les sons se rapprochaient, devenant plus distinct, plus agressifs. 

[Pas encore de visuel. Ce putain d'endroit est si grand et rempli de toute cette merde poussiéreuse, qui semble agir comme des paillettes. Les infrarouges ne me donne que vingt pieds de vision!) Sava indiqua à l'équipe de bouger vers la droite. [Restez groupés, on bouge lentement et on garde le doigt sur la détente. La salle d'attente est juste à l'est de nous. On commencera les recherche là-bas.] Les vrombissements étaient maintenant tout autour d'eux, restant juste à la limite de leur champ de vision. 

(C'est quoi ce bordel?) Un bot insectoîde volant avec six pattes articulées se terminant par de petites lames de scies émergea de l'obscurité et se dirigea vers Berk, qui se jetta au sol et libérant une décharge de plasma sur le drône Le bot se fracassa contre une pile d'os et de chiffons et l'enflamma. Le feu se répandit rapidement, sautant de chiffon sec en chiffon sec. Le sol brûlant du hangar illumina maintenant la scène dans une chaude lueur de flamme orangée. Au moins une douzaine de bots insectoides oscillait dans le périmètre autour de l'équipe, attendant une opportunité de frapper. Un autre bot plongea sur Berk, son bras tronçonneuse frappant violemment. Berk tira, mais rata. Le bot entra en collision avec la tête de Berk et la tronçonneuse pénétra sa nuque. Des étincelles jaillirent dans toutes les directions au moment où le métal rencontrait le métal. Elle laissa tomber son fusil et poussa contre le corps du drone jusqu'à ce que les lames fussent hors de son cou. [Courrez bande d'imlbéciles! Je m'occupe de ça!) 

Sava tira et descendit un bot, puis plongea vers l'est, sutant par-dessus les flammes hautes jusqu'à la taille qui se propageaint. [Rejoignez le salon!) 

Pivo se dressa sur deux bras et courru derrière Sava, ses cinq bras restant flottant librement autour de sa tête. [Bouge de là, espèce de trainard!] Sarlo dépassa l'octomorph, courant à travers les flammes en direction de la salle d'attente. 

Berk envoya voler le bot frénétique dans une pile d'ossements enflammés, sauta sur ses pieds et rattrapa le groupe, couverte de morceaux d'os et de poussières, la nuée de bots à ses trousses. 

Sava atteignit le salon en premier et la porte était ouverte. Se retourrnant avec son fusil armé, Sava se mit à couvert contre le montant de la porte. Sarlo et Pivo avaient dépassés les flammes, tandis que Berk rattrapait le terrain perdu, ainsi que les bots. Sava envoya un tir de couverture qui grésilla au-dessus de la tête de Sarlo, détruisant un autre bot, mais le reste de la nuée restait asynchrone. Ils continuait simplement d'avancer. D'un coup, de nombreux bots surgirent hors de l'obscurité près de la salle d'attente. 

[Il y en a d'autres! Ils nous prennent à revers!] Sava envoya une salve sur les nouveaux bots pour essayer de ralentir leur approche. Sarlo n'était plus qu'à dix mètres de la porte lorsqu'il trébucha sur une pile d'os. Son corps de petit garçon s'effondra face contre terre dans la poussière et les restes humains. PIvo fit un saut étrange par dessus lui, glissa sur le sol, et se ratatina dans le mur externe près de la porte de la salle d'attente. Sava sauta à l'extérieur, attrapa l'octomorph par un bras et le tira vers la sécurité du salon. Berk essaya de s'arréter et d'aider Sarlo, mais son élan était trop important et son appui sur le sol poussiéreux trop instable. Elle trébucha plus loin dans une nuée de poussière, d'ossements blanchis, et de guenilles, percutant finalement Sava dans l'embrasure de la porte. 

Les trois membres de l'équipe à l'intérieur du salon se rassemblèrent juste à temps pour voir un bot s'accrocher sur la tête de Sarlo alors qu'il se relevait. La machine étendit deux bras sur les côté et plongea les lames tournoyantes dans la nuque de l'enfant. Les yeux de Sarlo s'écarquillèrent et son corps se tendit alors que les lames de scies traversant la chair et les os, dé&coupèrent la nuque en quelques secondes. A l'instant où sa tête fut séparé de son corps, le bot fit demi-tour et disparu à travers les flammes, dans l'oubli des ténèbres à l'extrémité du hangar. 

Le corps sans tête de Sarlo s'agita une seconde, puis s'effondra, projettant du sang en giclées longue et paresseuse. 

\end{center} *** \end{center} 

Pivo, Sava et Berk s'assirent en silence. Ils réussirent à sceller la porte du salon, bloquant les horreurs à l'extérieur, dans le hangar. On pouvait toujours entendre les bots chasseur de tête glissant au dessus du sol, cliquetant et grattant la porte scellée avec leurs lames.. 

Berk rompit le silence la première. [J'essayes de ne pas penser à ce qu'ils vont faire de lui.) 

[Essayes plus fort, Sarlo savait que les chances de survie étaient minces lrosqu'il a signé. On a tous signé.) Sava se releva. 

[Devrait-on lui dire? Quand il se réincarnera?] Pivo savait que cela allait énerver Sava, mais il le dit quand même. 

[Ne serait-ce pas particulièremenbt cruel, Pivo? De plus, il n'y a aucune garantie que le moindre d'entre nous ne survive. Donc, qui s'en soucie? Quel que soit la date de ta dernière sauvegarde, je suis presque sûr que tu n'aura rien manqué de particulier depuis. On se remet en route.) 

\end{center} *** \end{center} 

En l'absence de Sarlo, c'était Pivo qui se récupérait les problématiques de navigation. Ils n'étaient pas loin du salon VIP corporate, la dernière localisation connue du coursier. 

L'équipe traversa des corridors sombres remplis de squelettes sans-tête et de reste momifiés. Des années auparavant, les forces corporatistes défendant la structure ont été submergées par des machines de guerre IA, qui ont massacré sans la moindre pitié toutes les personnes à l'intérieur. Les murs portaient les cicatrices de la bataille, couvertes de sang séché. Le hall était jonché des restes détruits des machines de guerres des IA, envoûtants monuments dédiés aux trop peu nombreuses victoires que l'humanité avait arrachées en perdant la bataille. Même réduite à l'état d'épaves, les machines avait une présence menaçante. 

[Dommage que ce ne soit pas une opération de récupération,) commenta Berk. (Les autonomistes auraient bien besoin de jetter un coup d'œil à cette techno. Au pire, imaginez ce que les hypercoprs pourraient essayer de faire avec ça.) 

En pénétrant dans un long corridor, les restes et les débris disparurent abruptement, comme si ils avaient étés nettoyés. 

[J'ai des échos thermiques étranges ici.. Des motifs qui n'ont pas de sens,) transmis Pivo. 

(Et c'est supposer signifier quoi?) renvoya Sava. 

Avant que Pivo ne puisse penser à "Je ne sais pas," sa muse émit un avertissement inquiétant: (Mes nanosenseurs ont enregistrés la présence d'une grande quantité de nanobot inconnus de conception hautement sophistiqué, suggérant l'œuvre des TITANs. Des contremesures ont étés initiés.) 

[Essaim de nanites. Bougez! Bougez!) broadcasta Pivo en panique en démarrant un sprint sur deux bras. Sava et Berk suivirent les instructions de Pivo sans poser de questions. Ils connaissaient tous les dangers d'une nuée de nanites des TITANS. Contrairement aux nanobots que Pivo fabriquait régulièrement, et qui était conçus avec un rôle précis en tête, et qui n'était ni auto-suffisant ni intelligents, cette nuée particulière était autonome, auto-réplicante, adpatative, et capable de produire à peu près tout ce dont elle aurait besoin. Tout en volant, des nanosenseurs indépendant analysait les trois agents, transmettant des détails sur leur morph et leur équipement au reste de la nuée. 

Une jonction apparu devant, le chemin se rétrecissant vers un petit tunel. Pivo s'arréta soudainement, un mètre avant le tunnel. [N'avancez pas plus loin!) Les autres s'arrétèrent brusquement. 

[C'est quoi ce bordel Pivo?) Sava regarda derrière lui, dans le hall. (Cette putain de nuée peut nous terminer pendant qu'on discute!) 

[Ma muse a attrapé une signature thermique ici. La nuée à quelquechose derrière la tête,) prévint Pivo. 

(Mais il n'y a rien ici,) répondit Berk en tendant sa main dans le tunnel. Sa main métallique tomba sur le sol sans prévenir, séparée de son poignet. 

(Câble monomoléculaire.) Même si la situation s'aggravait de minute en minute, Pivo était impressioné et fasciné par l'inventivité de cette nanotechnologie étrangère. (Ils en ont tissé la porte avec. Ça découpe tout. Ça résiste assez mal à la tension cependant - tu pourrais probablement le casser.] 

[On est niqué. Admettons le.) Berk ramassa sa main tranchée sur le sol. Dans le hall, la nuée de nanites commença a prendre une forme visible alors que les nanobots s'aggrégeait. La nuée se coagulait en brouillard, s'approchant en rampant. Berk continua, (L'ensemble du port est probablement rempli de cette merde. A ce niveau là, je suis inutile. Ces choses ont déjà envahi mon système, mes diagnostiques s'emballent.) 

[Qu'est ce que tu es en train de nous dire Berk? T'es finie?) transmis Sava. 

[Ouais. Je suis finie.] Berk secoua sa tête de dégoût. [Qui sait avec quoi ces petits enculés  m'ont infectée. Je ne veux pas le risquer. Je préfères retourner sur un backup propre. Oublier que cette merde a jamais eu lieu. Vous continuez de courrir si vous voulez. Je vais essayer de vous faire gagner un peu de temps.] Berk fit demi-tour et courru directement dans le brouillard. La nuée l'enroba immédiatement et le désassemblage commença. La structure métallique de Berk commença à se dissoudre alors qu'elle s'éloignait en courant de Pivo et Sava, laissant une traînée vaporeuse de nanites derrière elle. 

[Commencez à courir bande d'idiot! Je ne fait pas ça pour le fun! Je vous retrouve la prochaine fois.] Quelque sminutes plus tard, le signal de Berk mourru. 

\end{center} *** \end{center} 

Sava et Pivo entrèrent dans le salon VIP. Quand le spatioport fut envahi il y a des années, c'était là que c'était déroulé le dernier carré de l'humanité. Des piles de squelettes du personnel de sécurité tapissait le sol juste à l'intérieur du vestibule. Les restes carbonisés d'une barricade désespéré étaient éparpillés à côté des monticules d'os. Des squelettes drapés dans les restes déchirées de vêtements civils étaient regroupés autour des mus et des angles, parfois sur trois ou quatre épaisseurs, comme si ils avaient tous rampés aussi loin que possible d'une sorte d'avatar de la mort au milieu de la pièce. 

Pivo commença une opération de localisation pour trouver l'étiquette RFID que le coursier était censé avoir dans l'omoplate de son omoplate gauche. Le code déclencha un ping dans les trois mètres. Pivo pointa une petit pile d'os d'un long bras. [Il est quelque part là-dedans.) 

Sava s'approcha de la pile de trois squelettes et commença à farfouiller dans la pile d'os, cassant ou jetant tous les fémurs. [Fait chier, je veux une clope. Cette morph m'as tellement déformé. Je n'avait pas été assez clair sur le fait que je ne fumes pas? Et pourtant, à chaque fois, ils m'incarnent dans une morph vérolée par l'addiction.) Sava passa le paquet d'os à Pivo. 

[Ça doit être un truc de fury. Ça devrait prendre seulement quelques minutes de scanner ceux-là pour trouver la gravure nanoscopique.) Pivo se mit au travail. [Suffisamment de temps pour une clope, si tu veux.) 

[Ouais. Très drôle. Et si je te réduisait en poudre et que je te fumais plutôt?) Sava s'assit sur le sol pendant que Pivo laissa échapper un petit rire. 

Le coursier décédé, qui qu'il ai pu être, avait reçu des informatiosn trop importantes pour être émises. Personne ne connaissait les capacités des TITANs à intercepter et décoder, le coursier avait donc reçu une injection de nanobots qui avait gravé un message codé à l'échelle nanoscopique, directement sur l'un de ses fémurs. Il n'avait cependant jamais réussi à quitter la planète. Son message n'avait jamais été livré. 

Pivo et Sava n'avait aucune idée de ce qu'était l'information, mais quelqu'un à Firewall  pensait qu'elle valait le coup d'être récupéré. Des informations sur les TITANs peut-être. Ou la recette secrète d'un PDG pour une sauce de pâtes. 

[C'est celui-là.] Pivo tendis le fémur à Sava et jetta les autres sur le sol. 

[Qu'est-ce que ça dit?) 

[Je ne sait pas. Je ne suis pas sûr de vouloir savoir.] PIvo continuait de tendre le fémur. 

[Assez de drama Pivo. Demandes simplement à tes nanos de le lire. On a besoin d'une copie des données. Si tu ne veux pas les transporter, je le ferai.) 

[Je préfèrerai oui. Merci.] Pivo fit travailler ses nanobots  sur le déchiffrement de l'inscription. Quand ils eurent terminé, l'info fut transmise directement à Sava. Pivo n'en voulait aucun fragment. 

[Et maintenant, on fait quoi? Comment sort-on d'ici? La seule sortie possible est le chemin que nous avons pris pour venir, et c'est du suicide.) La pigmentation de Pivo vira d'un vert laiteux à un bleu royal. Ça arrivait tout le temps quand l'impuissance commençait à s'installer. 

Sava n'hésita pas à répondre, choississant de parler plutôt que de transmettre. "On ne partira pas , Pivo. On ne va même pas essayer." Sava leva son fusil à plasma et visa la tête oblongue de Pivo. "A la prochaine, calamar." Sava appuya sur la détente, et une boule de plasma incandescent réduisit Pivo en une masse tremblante et sanglante de  cartilage brulé au sommet de bras trodus. Les bras continuèrent de bouger au sol dans une marre de sang bouillonant pendant que Sava s'assit à côté d'une pile d'ossements et se reposa contre le mur. 

Sava sorti une cigarette et l'alluma. La première bouffée fut virtuellement orgasmique. Sava adorait fumer. 

A l'expiration, Careza le pinga. [Dois-je contacter le Projet Ozma?) 

Ouaip. Mets notre dame en ligne. 

Une voix féminine, froide et dure, entra dans la tête de Sava, trés différente de la voix appaisante de Careza. [Êtes-vous prèt pour la livraison, Agent Sava?) 

[Ça dépend.) Sava prit une autre bouffée. 

(Peut-être que je me suis mal fait comprendre lors de nos négotiations initiales Agent Sava. Vos options sont quelques peu limitées. Vous ne pourrez manifestement pas quitter la planète vivant, et nous ne pouvons nous permettre de perdre cette information, pas plus que nous pouvons nous permettre qu'elle parviennent entre les mains de votre organisation. Vous allez devoir vous en tenir à vos engagements, et faites moi confiance, nous ferons pareil.) 

[Soit vous me donnez sa localisation maintenant, ou alors j'embarque votre info si précieuse avec moi.) Il y eu une longue pause avant que la femme ne ré-émette. [Vous réalisez qu'il y aura des conséquences, Agent Sava. Pour vous et pour Rati.) 

[Ouais. Je suppose oui.) La cigarette se consuma jusqu'au filtre et Sava l'envoya rouler dans une pile d'os. [Donc, ça sera quoi?) 

[On ne marchandes pas, Agent Sava. Pas après qu'un accord ai été conclus. Faites ce que bon vous semble, et nous réagirons en fonction.) La connexion avec la femme se termina. Sava se leva et alla là où se trouvait le fémur du coursier et le ramassa. Le sang de Pivo avait recouvert l'os. Sava le nettoya et le rapprocha pour l'examiner en détail. 

Désolé, Careza. Il n'y a de la place que pour de l'info. Abandonne l'ego. 

[Compris.] 

En l'espace d'une pensé, Sava demanda à Careza d'activer le farcasteur d'urgence de la pile corticale - un émetteur à neutron à usage unique, alimenté par une toute petite quantité d'antimatière. La tête de Sava explosa dans toute la pièce, détruisant le fémur du coursier. L'information contenue dans le fémur, cependant, se fraya un chemin quasi instantanément à travers les profondeurs spatiales les plus sombres, atterissant en sureté dans un récepteur de Firewall dédié, situé ailleurs dans le système solaire. 

\end{center} *** \end{center} 

"Quel jour on est?" 

Les mots transpercent mes nouvelles cordes vocalles et s'expulsent de ma gorge désséchée. Comme prévu, ma diction est mauvaise, comme toujours durant les quelques minutes suivant une réincarnation. Le timbre de la voix est apparent en dépit du hachage, du décapage des mots. C'est une biomorph, définitivement, et mon nouveau sexe est féminin. C'est tout ce que je peux savoir dans les premières secondes. 




\chapter{Enter the singularity}
\label{chap:enter-the-singularity}

We humans have a special way of pulling ourselves up and kicking
ourselves down at the same time. We'd achieved more progress than ever
before, at the cost of wrecking our planet and destabilizing our own
governments. But things were starting to look up.

With exponentially accelerating technologies, we reached out into the
solar system, terraforming worlds and seeding new life. We re-forged
our bodies and minds, casting off sickness and death. We achieved
immortality through the digitization of our minds, resleeving from one
biological or synthetic body to the next at will. We uplifted animals
and AIs to be our equals. We acquired the means to build anything we
desired from the molecular level up, so that no one need want again.

Yet our race toward extinction was not slowed, and in fact received a
machine-assist over the precipice.  Billions died as our technologies
rapidly bloomed into something beyond control ... further transforming
humanity into something else, scattering us throughout the solar
system, and reigniting vicious conflicts.  Nuclear strikes, biowarfare
plagues, nanoswarms, mass uploads ... a thousand horrors nearly wiped
humanity from existence.

We still survive, divided into a patchwork of restrictive inner system
hypercorp-backed oligarchies and libertarian outer system collectivist
habitats, tribal networks, and new experimental societal models. We
have spread to the outer reaches of the solar system and even gained
footholds in the galaxy beyond. But we are no longer solely ``human'' ...
we have evolved into something simultaneously more and different—
something transhuman.

\section{Starting out}

Eclipse Phase is a post-apocalyptic roleplaying game of transhuman
conspiracy and horror. Humans are enhanced and improved, but humanity
is battered and bitterly divided. Technology allows the re-shaping of
bodies and minds and liberates us from material needs, but also
creates opportunities for oppression and puts the capability for mass
destruction in the hands of everyone. Many threats lurk in the
devastated habitats of the Fall, dangers both familiar and alien.

\subsection{What is a roleplaying game?}

Have you ever read a book or seen a movie or a television show where a
character does something really stupid, like heading into a basement
at night when the character knows the serial killer is around? The
whole time, you're thinking: ``I wouldn't walk down those creepy stairs
to the dark basement, especially without a flashlight. I'd do X, Y, or
Z instead!'' Since you're in the passenger's seat for the plot you're
reading or watching, however, you simply have to sit back and let it
unfold.

What if you could take hold of the driver's seat?  What if you could
take the plot in the direction you'd choose? That is the essence of a
roleplaying game.

A roleplaying game (or RPG, for short) is part improvisational
theater, part storytelling, and part game.  A single person (the
gamemaster) runs the game for a group of players that pretend to be
characters in a fictitious world. The world could be a mystery game
set in the 1920s that takes you adventuring around the globe, a
fantasy realm inhabited by dragons and trolls and sword-wielding
barbarians, or a science fiction setting with aliens and spaceship and
world-crushing weaponry. The players pick a setting that they find
cool and want to play in. The players then craft their own characters,
providing a detailed history and personality to bring each to
life. These characters have a set of statistics (numerical values)
that represent skills, attributes, and other abilities.  The
gamemaster then explains the situation in which the characters find
themselves. The players, through their characters, interact with the
storyline and each others' characters, acting out the plot. As the
players roleplay through some scenarios, the gamemaster will probably
ask a given player to roll some dice and the resulting numbers will
determine the success or failure of a character's attempted
action. The gamemaster uses the rules of the game to interpret the
dice rolls and the outcome of the character's actions.

As a group exercise, the players control the storyline (the
adventure), which evolves much like any movie or book but within the
flexible plot created by the gamemaster. This gamemaster plot provides
a framework and ideas for potential courses of action and outcomes,
but it is simply an outline of what might happen—it is not concrete
until the players become involved. If you don't want to walk down
those stairs, you don't.  If you think you can talk yourself out of a
situation in place of pulling a gun, then try and make it happen.  The
script of any roleplaying session is written by the players, and the
story, based upon the character's actions and their responses to the
events of the plot, will constantly change and evolve.

The best part is that there is no ``right'' or ``wrong'' way to play an
RPG. Some games may involve more combat and dice rolling-related
situations, where other games may involve more storytelling and
improvised dialogue to resolve a situation. Each group of players
decides for themselves the type and style of game they enjoy playing!

\subsection{What is transhumanism?}
\label{sec:what-transhumanism}
  
Transhumanism is a term used synonymously to mean ``human enhancement.''
It is an international cultural and intellectual movement that
endorses the use of science and technology to enhance the human
condition, both mentally and physically. In support of this,
transhumanism also embraces using emerging technologies to eliminate
the undesirable elements of the human condition such as aging,
disabilities, diseases, and involuntary death. Many transhumanists
believe these technologies will be arriving in our near future at an
exponentially accelerated pace and work to promote universal access to
and democratic control of such technologies. In the long scheme of
things, transhumanism can also be considered the transitional period
between the current human condition and an entity so far advanced in
capabilities (both physical and mental faculties) as to merit the
label ``posthuman.''

As a theme, transhumanism embraces heady questions. What defines
human? What does it mean to defeat death? If minds are software, where
do you draw the line with programming them? If machines and animals
can also be raised to sentience, what are our responsibilities to
them? If you can copy yourself, where does ``you'' end and someone new
begin? What are the potentials of these technologies in terms of both
oppressive control and liberation? How will these technologies change
our society, our cultures, and our lives?

\subsection{Post-apocalyptic, conspiracy and horror themes}
\label{sec:post-apoc-consp}

Several themes pervade Eclipse Phase, some of which the reader may not
be intimately familiar with. The following helps define these themes
so that as play ers read further into this rulebook, they gain a solid
understanding of how Eclipse Phase builds on such themes to create its
unique setting.

Post-apocalyptic is a term used to describe fiction set after a
cataclysmic event has ended human civili zation as we know it (usually
accompanied by loss of human life on an almost unthinkable scale). The
exact mechanism of the disaster is usually unimportant nuclear war,
plague, asteroid strike, and so on. The importance of the theme is the
human condition. If the world we know is torn away from us and humans
suffer horrors beyond imagining in this transforma tion to a
post-apocalyptic setting, how does human ity cope? Do we survive and
thrive and overcome?  Or do we lose our own humanity in the process, o
ultimately fall to extinction? Those are the questions that drive this
genre.

To conspire means ``to join in a secret agreement to do an unlawful or
wrongful act or to use such means to accomplish a lawful end.'' As
such, a con spiracy theory attributes the ultimate cause of an event
or a chain of events (whether political, societa or historical) to a
secret group of individuals with immense power (including political,
wealth and so on) who hide their activities from public view while
manipulating events to achieve their goals, regard less of
consequences. Many conspiracy theories contend that a host of the
greatest events of history were initiated and ultimately controlled by
such secret organizations. Of equal importance is the silent struggle
between clandestine groups, waging a secret war behind the scenes to
determine who influences the future.

Horror takes many forms, but in Eclipse Phase it is more psychological
than gore. It is the uncertainty of survival, the suspense of finding
malevolent things among the stars, the fear of the unknown, the dread
of facing Things That Should Not Be, the revulsion when encountering
alien things, and the sickening realization of the wrong and ghastly
things that transhumans are capable of doing to themselves and each
other. Horror also arises both from the comprehension that there are
scary things beyond our understanding nhabiting our universe and that
transhumanity may be its own worst enemy. Despite all of the
technological tools and advances available to future transhumans, they
still face terrors like losing control of their own dentities, their
perceptions, and their mental faculties—not to mention their future as
a species.

Eclipse Phase takes all of these themes and weaves them together in a
transhuman setting. The postapocalyptic angle covers the understanding
of all that transhumanity has lost, the fight against extinction, and
how much of that is a struggle against our own nature. The conspiracy
side delves into the nature of the secret organizations that play key
roles n determining transhumanity's future and how the actions of
determined individuals can change the ives of many. The horror
perspective explores the results of humanity's self-inflicted
transformations and how some of these changes effectively make us
non-human. Tying it all together is an awareness of the massive
indifference and the terrible alien-ness that pervades the universe
and how transhumanity is insignificant against such a backdrop.

Offsetting these themes, however, Eclipse Phase also asserts that
there is still hope, that there is still something worth fighting for,
and that transhumanity can pave its own path toward the future.

\subsection{But how do you actually play?}
\label{sec:but-how-do}

To play a game of Eclipse Phase, you need the following:

\begin{itemize}
\item A group of players and a place to meet (real life or online!)
\item One player to act as the gamemaster
\item The contents of this book
\item Something for everyone to take notes with (note pads, laptops,
  whatever!)
\item Two 10-sided dice per player (or a digital equivalent)
\item Imagination
\end{itemize}

\subsubsection{A group of players and a place to meet}

While roleplaying games are flexible enough to allow any number of
people, most gaming groups number around four to eight players. That
number of people brings a good mix of personalities to the table and
ensures great cooperative play.

Once a group of players have determined to play Eclipse Phase, they'll
need to designate someone as the gamemaster (see below). Then they'll
need to determine a time and place to meet.

Most roleplaying groups meet once a week at a regularly scheduled time
and place: 7:00 PM, Thursday night, Rob's house, for example. However,
each group determines where, how they'll play, and how often.  One
group may decide they can only get together once a month, while
another group is so excited to dive into the story potential of
Eclipse Phase that they want to meet twice a week (they decide to
rotate between their houses, though, so as not to overload a
particular player). If a group is lucky enough to have a favorite
local gaming store that supports instore play, the group might meet
there. Other gaming groups meet in libraries, common rooms at their
school, bookstores that have generously-sized ``reading rooms,'' quiet
restaurants, and so on. Whatever fits for your gaming group, make it
work!

When getting together for a game, most RPGs use the phrase ``gaming
session.'' The length of each gaming session is completely dependent
upon the consensus of the playing group, as well as the limitations of
the locale where they're playing. The particular story that unfolds in
a given session can also impact a session's length. If playing in a
game store, the group may only have a four-hour slot and the
gamemaster and group may have determined—through several sessions of
play—that this is a perfect time frame to enjoy the story they're
participating in each week. Another group, however, may want an even
shorter length of time. Yet another group may decide that while
they'll usually do four-hour sessions, once a month they'll set aside
an entire Saturday for a great all-day gaming session. Players will
need to dive in and start playing and be flexible to decide what will
provide the ultimate enjoyment for their gaming group.

While the camaraderie of a shared experience of playing face-to-face
with a group of friends remains the strength of roleplaying games,
groups need not confine themselves to a single mode of play. There are
myriad options that can be used. Email, instant messages, message
boards, video chats, phone/voip calls, text messages, wikis,
(micro-)blogs: any and all of these can be utilized to play the game
without having warm bodies in seats directly across the table from one
another.

Finally, when playing groups meet for the first time, they should
generate their characters (as opposed to generating characters by
themselves). While a gaming group can decide to generate characters
individually, often it is far easier once the players are
together. This allows those more experienced in roleplaying games to
help those new to RPGs. Even more important, it enables the entire
group to tailor the characters so there is not too much overlap in
capabilities and style. After all, with the wealth of character
opportunities available, you don't want to show up at the table with
an almost identical character to the player next to you.

\subsubsection{The gamemaster}
\label{sec:gamemaster}

Once a group has been organized, someone needs to step up and take the
reins of the gamemaster. Some groups have a single gamemaster that
runs all their gaming sessions month after month. Other groups rotate
a gamemaster, with a single gamemaster running a given portion of the
unfolding story for several sessions before handing the work off to
another player. Once again, the participants should be flexible.  Some
groups may have the perfect person who loves the work involved and is
more than willing to run session after session, while other groups may
decide that they all want to take turns both as the gamemaster and as
players.

The gamemaster controls the story. They keep track of what is supposed
to happen when, describes events as they occur so that the players (as
characters) can react to them, keep track of other characters in the
game (referred to as non-player characters, or NPCs), and resolve
attempts to take action using the game system. The game system comes
into play when characters seek to use their skills or otherwise do
something that requires a test to see whether or not they
succeed. Specific rules are presented for situations that involve
rolling dice to determine the outcome (see Game Mechanics, p. 112).

The gamemaster describes the world as the characters see it,
functioning as their eyes, ears, and other senses. Gamemastering is
not easy, but the thrill of creating an adventure that engages the
other players' imaginations, testing their gaming skills and their
characters' skills in the game world, makes it worthwhile. Posthuman
Studios and Catalyst Game Labs will follow the publication of Eclipse
Phase with supporting supplements and adventures to help this process
along, but experienced gamemasters can always adapt the game universe
to suit their own styles. In fact, since Eclipse Phase is published
under a Creative Commons License (see p. 5), players are encouraged to
tailor the universe to their style of play and also to share that with
other players. You never know when a specific choice you've made in
the running of a campaign is exactly what another gamemaster and his
group is looking for.

\subsubsection{The contents of this book}
\label{sec:contents-this-book}

Whether you have purchased the print or electronic version, this book
is specifically organized to present the information you need to know
to start telling your stories in the Eclipse Phase universe. Below
you'll find a summary of each chapter of the book.


\paragraph{A Time of Eclipse:} A comprehensive history and setting
fully describes the Eclipse Phase universe and how humanity
transitioned from here to there. See p. 30.

\paragraph{Game Mechanics:} The player's desired actions become
reality within the universe through quick and easy-to-use game
mechanics. See p. 112.

\paragraph{Character Creation and Advancement:} Creating a unique
character can be one of the most enjoyable experiences of
roleplaying. Even more rewarding is watching that character evolve and
grow across numerous gaming sessions, far beyond anything your
imagination first envisioned. See p. 128.

\paragraph{Skills:} Beyond a character's innate abilities, their
skills are what set them apart. This is what your character knows and
what they know how to do. See p. 170.

\paragraph{Action and Combat:} What is a dramatic story without action
and violence? When words fail, weapons will blaze. See p. 186.

\paragraph{Mind Hacks:} The unusual possibilities offered by psi
abilities and mental reprogramming. See p. 216.

\paragraph{The Mesh:} The all-pervasive nature of the mesh ensures
that it is a key element to any story telling.  See p. 234.

\paragraph{Accelerated Future:} The wonders of advanced technologies
and how they work. See p. 266.

\paragraph{Gear:} Personal enhancements, weapons, robots, and
everything else in between. See p. 294.

\paragraph{Game Information:} The quintessential set of insider
secrets for gamemasters. See p. 350.

\subsubsection{Taking notes}
\label{sec:taking-notes}

Whether a gamemaster or player, you'll need a way to track
information. Players will be generating characters and making changes
to those characters from session to session. Meanwhile, the gamemaster
will have a host of information to track: notes on how the story is
unfolding due to player character interaction that you'll need to fold
into next week's session; changes to NPCs; changes to player
characters that the players are not yet aware off (such as a character
has been mind hacked but doesn't yet know it); and so on.

Additionally, some groups enjoy a synopsis of each session that can be
compiled and read at a later time in order to enjoy and share their
exploits, just as you might fileshare clips from your favorite video
game to show off your skill in taking the bad guy down (traditionally
this has been called ``bluebooking''). This can be particularly useful
if a player was unable to attend a given session, providing a quick
re-cap that they can read before attending the next gaming session and
thus avoiding a bog-down up-front as that player tries to catch up on
current events in the game. The session scribe can be a shared
responsibility or assigned, all based upon what a given playing group
finds works best for them. Likewise, some gaming groups audiorecord
their entire game session, both for later reference and for ``actual
play'' podcasts.

The old standard of a pencil and paper still works wonders. A host of
additional technologies, however, provide many new options for
players. From a text file on a laptop to a shared wiki, the ability to
track large amounts of information in a quick and useful fashion—while
simultaneously making appropriate information available to each player
from session to session—significantly decreases how much time everyone
needs to spend tracking information. That time can now be redirected
into the enjoyment of participating in a great story.

\subsubsection{Dice}
\label{sec:dice}

As described in the Game Mechanics section (p. 112), two ten-sided
dice are required to play Eclipse Phase.  While most players enjoy the
feel of tossing dice onto a table, there are many other mechanisms for
rolling two ten-sided dice to achieve a 00 to 99 result. Players who
make heavy use of any online technologies for game play—such as using
online chatting or video blogging—should find it easy to track down
and implement a quick dice-rolling program.

\subsubsection{Imagination}
\label{sec:imagination}

All too often, it's easy for someone looking at an RPG to be
intimidated. So many concepts to grasp, so many ideas that seem
overwhelming. Just as described under What is a Roleplaying Game?,
however, how often have you read a book or watched that movie and
decided that you would have done it better? That's your imagination at
work. Just dive in and you'll be amazed at how quickly you can immerse
yourself in the Eclipse Phase universe. Soon you'll be spinning
stories with the best of them.

Also, don't forget to tap your resources. Your gaming group is your
best resource. What's going on, ideas for how to handle a situation,
or how to take on a bad guy: these are just some of the things that
can and should be discussed by the gaming group in between sessions,
and each is an opportunity to strengthen your imagination.

Another resource is simply watching TV or reading a good book. Pay
attention to how the story is put together, how the characters are
built, and how the plot unfolds. Push your imagination and soon you'll
be figuring out subplots and who the bad guy is long before it's
revealed. Knowing how a story is put together enables you to put
together your own stories during each gaming session.

Finally, eclipsephase.com is the offi cial site for Eclipse Phase. If
you have questions about the game or want to see how another group of
players handles a given situation, post on the forums. The online
community can be just as helpful and enjoyable as a local gaming
group.

\subsection{What do players do?}
\label{sec:what-do-players}

The players can take on a variety of roles in Eclipse Phase. Due to
advances in digital mind emulation technology, uploading, and
downloading into new morphs (physical bodies, biological or
synthetic), it is possible to literally be a new person from session
to session. With bodies taking on the role of gear, players can
customize their forms for the task at hand.

\subsubsection{The default campaign}
\label{sec:default-campaign}

In the default story (also known as ``campaign setting''), every player
character is a ``sentinel,'' an agent-on-call (or potential recruit) for
a shadowy network known as ``Firewall.'' Firewall is dedicated to
counteracting ``existential risks''—threats to the existence of
transhumanity. These risks can and do include biowar plagues, nanotech
swarm outbreaks, nuclear proliferation, terrorists with WMDs,
netbreaking computer attacks, rogue AIs, alien encounters, and so
on. Firewall isn't content to simply counteract these threats as they
arise, of course, so characters may also be sent on information-
gathering missions or to put in place pre-emptive or failsafe
measures. Characters may be tasked to investigate seemingly innocuous
people and places (who turn out not to be), make deals with shady
criminal networks (who turn out not to be trustworthy), or travel
through a Pandora's Gate wormhole to analyze the relics of some alien
ruin (and see if the threat that killed them is still real). Sentinels
are recruited from every faction of transhumanity; those who aren't
ideologically loyal to the cause are hired as mercenaries. These
campaigns tend to mix a bit of mystery and investigation with fierce
bouts of action and combat, also stirring in a nice dose of awe and
horror.

\subsubsection{Alternate campaigns}
\label{sec:alternate-campaigns}

When they're not saving the solar system, sentinels are free to pursue
their own endeavors. The gamemaster and players can use this rulebook
to generate any type of story they wish to tell. However, the
following examples provide a brief look at the most obvious
opportunities for adventure in Eclipse Phase.

After each campaign variant below, a list of ``archetypes'' for Eclipse
Phase are provided in parenthesis. Archetypes are the names applied to
the most common character types featured in those scenarios.  For
example, in a traditional detective story, the archetypes would be the
Detective, the Damsel In Distress, the Hard-bitten Cop, and so on. In
a cowboy movie, the archetypes would be the Gunfighter, the Bartender,
the Marshal, the Indian Brave, and so on. Players will note that some
archetypes fit into multiple story settings. The character creation
system (p. 128) allows players to create any of the suggested
archetypes. Just as roleplaying games are designed for players to
build their own stories, however, these archetypes are just
suggestions and players can mix and match how they will.

\paragraph{Salvage and Rescue/Retrieval Ops:} The Fall left two worlds
and numerous habitats in ruins—but these devastated cities and
stations contain untold riches for those who are brave and foolhardy
enough. Potential hauls include: weapon systems; physical resources;
lost databanks; left-behind uploads of friends, family, or important
people; new technologies developed and lost in the brief singularity
takeoff; valued heirlooms of immortal oligarchs; and much
more. Outside of these once-inhabited realms, space itself is a big
place and lots of people and things get lost out there. Some need to
be saved and some are beyond saving. This option lets players explore
the unknown or seek out specific targets on
contract. (Archeologist/Scavenger/Pirate/Free Trader/ Smuggler/Black
Marketeer)

\paragraph{Exploration:} There are plenty of opportunities to be had
as an explorer, colonist, or long-range scout—perhaps even as one of
the few lucky or suicidal individuals who explore through an untested
Pandora's Gate. Even the Kuiper Belt, on the fringe of our solar
system, is still sparsely explored; there may be riches and mysteries
still to be found. Many dangers also lurk in odd corners of the
system, from isolationist posthuman factions to secretive criminal
cartels, as well as pirates, aliens, and others wishing to remain out
of sight. (Explorer/Archeologist/ Scavenger/Singularity
Seeker/Techie/Medic)

\paragraph{Trade:} While the majority of inner system trade is
controlled by sleek hypercorporations, many of the smaller or more
independent stations rely on small traders. In the post-scarcity outer
system, trade takes on a different form, with information, favors, and
creativity serving as currency among those who no longer want for
anything due to the availability of cornucopia machines. (Free
Trader/Smuggler/Black Marketeer/Pirate)

\paragraph{Crime:} The patchwork of city-state habitats and widely
varying laws throughout the system create ample opportunity for those
who would make a living from this situation. Black market commodities
and activities include infomorph-slave trading, pleasure pod sex
industries, data brokerage and theft, extracting/smuggling advanced
technologies and scientists, political/economic espionage,
assassination, drug and XP dealing, soul-trading, and much
more. Whether as an independent or part of an organized criminal
element, there are always opportunities for those with a thirst for
adventure or profit and questionable morals. (Criminal/Smuggler/
Pirate/Fixer/Black Marketeer/Genehacker/Hacker/ Covert Ops)

\paragraph{Mercenaries:} The constant maneuvering of
ideologically-driven factions, the squabbling over contested
resources, and the rush to colonize new exoplanets beyond the Pandora
Gates all spark new conflicts on a regular basis. Some of these simmer
and seeth as low-intensity conflicts for years, occasionally flaring
into raids and clashes. Others break out into all-out warfare. Women
and men willing to bear arms for credits are always in demand for good
wages. Players can engage in commando and military campaigns in
habitats, between the stars, or in hostile planetary
environments. (Merc/Security Specialist/Fixer/Bounty
Hunter/Ex-Cop/Medic)

\paragraph{Socio-Political Intrigue:} The corporations and political
factions that span the solar system do not always play nice with each
other, but neither is it wise for them to openly confront each other
except under extreme circumstances. Many battles are fought with
diplomacy and political maneuvering, using words and ideas more potent
than weapons.  Even within factions, social cliques can compete
ruthlessly, or heated class confl icts can come to a boil, tearing a
society apart from within. In this campaign, the players can start as
pawns of some entity who rise through the ranks as they become more
enmeshed in the intrigues of their sponsor, play a group of
ambassadors and spies stationed in the opposition's capital, or can
play a group of activists and radicals fighting for social
change. (Politico/ Socialite/Covert Ops/Hacker/Security Specialist/
Journalist/Memeticist)

\subsection{Where does it take place?}
\label{sec:where-does-it}

While Eclipse Phase is set in the not-too-distant future, the changes
that have taken place due to the advancements of technology have
transformed the Earth and its inhabitants almost beyond recognition.
As players dive into the universe, they'll generally encounter one of
the following settings.

\subsubsection{Humanity's habitats}
\label{sec:humanitys-habitats}

The Earth has been left an ecologically-devastated ruin, but humanity
has taken to the stars. When Earth was abandoned, so too were the last
of the great nation-states; transhumanity now lacks a single unifying
governing body and is instead subject to the laws and regulations of
whomever controls a given habitat.

The majority of transhumanity is confined to orbital habitats or
satellite stations scattered throughout the Sol system. Some of these
were constructed from scratch in the orbit or Lagrange points of
planetary bodies, others have been hewn out of solid satellites and
large asteroids. These stations have myriad purposes from trade to
warfare, espionage to research.

Mars continues to be one of transhumanity's largest settlements,
though it too, suffered heavily during the Fall. Numerous cities and
settlements remain, however, though the planet is only partially
terraformed. Venus, Luna, and Titan are also home to significant
populations. Additionally, there are a small number of colonies that
have been established on exoplanets (on the other side of the Pandora
Gates) with environments that are not too hostile towards humanity.

Some transhumans prefer to live on large colony ships or linked swarms
of smaller spacecraft, moving nomadically. Some of these rovers
intentionally exile themselves to the far limits of the solar system,
far from everyone else, while others actively trade from habitat to
habitat, station to station, serving as mobile black markets.

\subsubsection{The great unknown}
\label{sec:great-unknown}

The areas of the galaxy that have felt the touch of humanity are few
and far between. Lying betwixt these occasional outposts of
questionable civilization are mysteries both dangerous and
wonderful. Ever since the discovery of the Pandora Gates, there has
been no shortage of adventurers brave or foolhardy enough to strike
out on their own into the unknown regions of space in hopes of finding
more alien artifacts, or even establishing contact with one of the
other sentient races in the universe.

\subsubsection{The mesh}
\label{sec:mesh}

While not a ``setting'' in the traditional sense, as the sections
describe above, the computer networks known as the ``mesh'' are
all-pervasive. This ubiquitous computing environment is made possible
thanks to advanced computer technologies and nanofabrication that
allow unlimited data storage and near-instantaneous transmission
capacities. With micro-scale, cheap-to-produce wireless transceivers
so abundant, literally everything is wirelessly connected and
online. Via implants or small personal computers, characters have
access to archives of information that dwarf the entire 21st-century
internet and sensor systems that pervade every public place. People's
entire lives are recorded and lifelogged, shared with others on one of
numerous social networks that link everyone together in a web of
contacts, favors, and reputation systems.

\subsection{Ego vs. Morph}
\label{sec:ego-vs.-morph}

The distinction between ego (your mind and personality, including
memories, knowledge, and skills) and morph (your physical body and its
capabilities) is one of the defi ning characteristics of Eclipse
Phase. A good understanding of the concept right up front will allow
players a glimpse at all the story possibilities out of the gate.

Your body is disposable. If it gets old, sick, or too heavily damaged,
you can digitize your consciousness and download it into a new
one. The process isn't cheap or easy, but it does guarantee you
effective immortality—as long as you remember to back yourself up and
don't go insane. The term morph is used to describe any type of form
your mind inhabits, whether a vat-grown clone sleeve, a synthetic
robotic shell, a part-bio/part-synthetic ``pod,'' or even the purely
electronic software state of an infomorph.

A character's morph may die, but the character's ego may live on,
assuming appropriate backup measures have been taken. Morphs are
expendable, but your character's ego represents the ongoing,
continuous life path of your character's mind and personality. This
continuity may be interrupted by an unexpected death (depending on how
recently the backup was made), but it represents the totality of the
character's mental state and experiences.

Some aspects of your character—particularly skills, along with some
stats and traits—belong to your character's ego and so stay with them
throughout the character's development. Some stats and traits,
however, are determined by morph, as noted, and so will change if your
character leaves one body and takes on another. Morphs may also affect
other skills and stats, as detailed in the morph description.

\subsection{Where to go from here?}
\label{sec:where-go-from}

Now that you know what this game is about, we suggest that you next
read the Time of Eclipse chapter (p.  30), to get a feel for the
game's default setting (which you are, of course, free to change to
suit your whims). Then read the Game Mechanics chapter (p. 112) to get
a grasp of the rules. After that, you can move on to Character
Creation and Advancement (p. 128) and create your first character!

\subsection{Terminology}
\label{sec:terminology}

Eclipse Phase uses a host of jargon to simply convey the numerous
concepts covered within the pages of this book. While not
all-inclusive, this list of terminology will allow players to quickly
acclimate themselves for their journey into Eclipse Phase. If you read
something and are confused, don't worry. These concepts are fully
explained in later sections of this book.

Note that several of the words on this list are standard scientific
terms, often used in astronomy. As Eclipse Phase attempts to remain as
close to ``hard science'' as possible—while allowing players to interact
with the great stories waiting to unfold—such terms are used
liberally.

\begin{itemize}
\item Aerostat: A habitat designed to float like a balloon in a
  planet's upper atmosphere.
\item AF: After the Fall (used for reference dating).
\item AGI: Artificial General Intelligence. An AI that has cogni tive
  faculties comparable to that of a human or higher.  Also known as
  ``strong AI'' (differentiating from more specialized ``weak AI''). See
  also ``seed AI.''
\item AI: Artificial Intelligence. Generally used to refer to weak
  AIs; i.e., AIs that do not encompass (or in some cases, are
  completely outside of) the full range of human cognitive
  abilities. AIs differ from AGIs in that they are usually specialized
  and/or intentionally crippled/limited.
\item Anarchist: Someone who believes government is unnecessary, that
  power corrupts, and that people should control their own lives
  through self-organized individual and collective action.
\item Arachnoid: A spider-like robotic synthmorph.
\item Argonauts: A faction of techno-progressive scientists that
  promote responsible and ethical use of technology.
\item AR: Augmented Reality. Information from the mesh (universal data
  network) that is overlaid on your real-world senses. AR data is
  usually entoptic (visual), but can also be audio, tactile,
  olfactory, kinesthetic (body awareness), emotional, or other types
  of input.
\item Async: A person with psi abilities.
\item AU: Astronomical unit. The distance between the Earth and the
  Sun, equal to 8.3 light minutes, or about 150 million kilometers.
\item Autonomists: The alliance of anarchists, Barsoomians,
  Extropians, scum, and Titanians.
\item Barsoomian: A rural Martian, typically resentful of hypercorp
  control.
\item Basilisk Hack: An image or other sensory input that affects the
  brain's visual cortex and pattern recognition abilities in such a
  way as to cause a glitch and possibly exploit it and rewrite neural
  code.
\item Beehive: A microgravity habitat made from a tunneledout asteroid
  or moon.
\item BF: Before the Fall (used for reference dating).
\item Bioconservative: An anti-technology movement that argues for
  strict regulation of nanofabrication, AI, uploading, forking,
  cognitive enhancements, and other disruptive technologies.
\item Biomorph: A biological body, whether a flat, splicer,
  genetically engineered transhuman, or pod.
\item Body Bank: A service for leasing, selling, acquiring, or storing
  a morph. Aka dollhouse, morgue.
\item Bots: Robots. AI-piloted synthetic shells.
\item Bracewell Probe: A type of autonomous monitoring deep- space
  probe meant to make contact with alien civilizations.
\item Brinkers: Exiles who live on the fringes of the system, as well
  as other isolated and well-hidden nooks and crannies. Also called
  isolates, fringers, drifters.
\item Case: A cheap, common, mass-produced synthetic shell.
\item Chimeric: Transgenic, containing genetic traits from other
  species.
\item Circumjovian: Orbiting Jupiter.
\item Circumlunar: Orbiting the Moon.
\item Circumsolar: Orbiting the Sun.
\item Cislunar: Between the Earth and the Moon.
\item Clade: A species or group of organisms with common
  features. Used to refer to transhuman subspecies and morph types.
\item Cole Bubble: A habitat made from a hollowed-out asteroid or
  moon, spun for gravity.
\item Cornucopia Machine: A general-purpose nanofabricator.
\item Cortical Stack: An implanted memory cell used for ego
  backup. Located where the spine meets the skull; can be cut out.
\item Cyberbrain: An artificial brain, housing an ego. Used in both
  synthmorphs and pods.
\item Darkcast: Illegal and black market farcasting and egocasting
  services.
\item Domain Rules: The rules that govern the reality of a virtual
  reality simulspace.
\item Drone: A robot controlled through teleoperation (rather than
  directly via onboard AI).
\item Ecto: Personal mesh devices that are flexible, stretchable,
  self-cleaning, translucent, and solar-powered. From ecto-link
  (external link).
\item Ego: The part of you that switches from body to body.  Also
  known as ghost, soul, essence, spirit, persona.
\item Egocasting: Term for sending egos via farcasting.
\item Entoptics: Augmented-reality images that you ``see'' in your
  head. (``Entoptic'' means ``within the eye.'')
\item ETI: Extraterrestial intelligence. The term Firewall uses to
  refer to the god-like post-singularity alien intelligence theorized
  to be responsible for the Exsurgent virus.
\item Exalts: Genetically-enhanced humans (between genefixed and
  transhumans). Aka genefreaks, the ascended, the elevated.
\item Exoplanet: A planet in another solar system.
\item Exsurgent: Someone infected by the Exsurgent virus.
\item Exsurgent Virus: The multi-vector virus created by an unknown
  ETI and seeded throughout the galaxy in Bracewell probes. The
  Exsurgent virus is self-morphing and can infect both computer
  systems and biological creatures.
\item Extrasolar: Outside the solar system.
\item Factors: The alien ambassadorial race that deals with
  transhumanity. Also called Brokers.
\item The Fall: The apocalypse; the singularity and wars that nearly
  brought about the downfall of transhumanity.
\item Farcasting: Intrasolar communication utilizing classical
  communication technologies (radio, laser, etc.) and quantum
  teleportation.
\item Farhauler: Long distance space shipper.
\item Firewall: The secret cross-faction conspiracy that works to
  protect transhumanity from ``existential threats'' (risks to
  transhumanity's continued existence).
\item Flatlander: Someone born or used to living on a planet or moon
  with gravity.
\item Flats: Baseline humans (not genetically modified). Also called
  norms.
\item Flexbot: A shape-changing synthmorph also capable of joining
  together with other flexbots in a modular fashion to create larger
  shapes.
\item Forking: Copying an ego. Not all forks are full copies.  AKA
  backups.
\item FTL: Faster-Than-Light.
\item Fury: A transhuman combat morph.
\item Gatecrashers: Explorers who take their chances using a Pandora
  gate to go somewhere previously unexplored.
\item Genehacker: Someone who manipulates genetic code to create
  genetic modifications or even new life.
\item Ghost: A transhuman combat morph optimized for stealth and
  infiltration.
\item Ghost-riding: The act of carrying an infomorph in a special
  implant module inside your head.
\item Greeks: Trojan asteroids or moons that share the same orbit as a
  larger planet or moon, but are 60 degrees ahead in the orbit at the
  L4 Lagrange point. The term Greeks normally refers to the asteroids
  orbiting around Jupiter's L4 point. See also ``Trojans.''
\item Habtech: A habitat technician.
\item Heliopause: The point where pressure from the solar wind
  balances with the interstellar medium (about 100 AU out).
\item Hibernoid: A transhuman modified for hibernation, for extensive
  travel in space.
\item Iceteroid: An asteroid made from mostly ice rather than rock or
  metals.
\item Iktomi: The name given to the mysterious alien race whose relics
  have been found beyond the Pandora Gates.
\item Indentures: Indentured servants who have contracted their labor
  to a hypercorp or other authority, usually in exchange for a morph.
\item Infolife: Artificial general intelligences and seed AIs.
\item Infomorph: A digitized ego; a virtual body. Also known as
  datamorphs, uploads, backups.
\item Infugee: ``Infomorph refugee,'' or someone who left everything
  behind on Earth during the Fall—even their own body.
\item Isolates: Those who live in isolated communities far outside the
  system (in the Kuiper Belt and Oort Cloud); aka outsters, fringers.
\item Jamming: The act of ``becoming'' a teleoperated drone thanks to XP
  technology. Also sometimes applied to accesing the real-time XP feed
  from lifeloggers and others.
\item Kuiper Belt: A region of space extending from Neptune's orbit
  out to about 55 AU, lightly populated with asteroids, comets, and
  dwarf planets.
\item Lagrange Point: One of five areas in respect to a small
  planetary body orbiting a larger one in which the gravitational
  forces of those two bodies are neutralized.  Lagrange points are
  considered stable and ideal locations for habitats.
\item Lifelog: A recording of one's entire life experience, made
  possible due to near unlimited computer memory.
\item Lost Generation: In an effort to repopulate post-Fall, a
  generation of children were reared using forced-growth methods. The
  results were disastrous: many died or went insane, and the rest were
  stigmatized.
\item Main Belt: The main asteroid belt, a torus ring orbiting between
  Mars and Jupiter.
\item Meme: A viral idea.
\item Mentons: Transhumans optimized for mental and cognitive ability.
\item Mercurials: The non-human sentient elements of the transhuman
  ``family,'' including AGIs and uplifted animals.
\item Mesh: The omnipresent wireless mesh data network.  Also used as
  a verb (to mesh) and adjective (meshed or unmeshed).
\item Mesh ID: The unique signature attached to one's mesh activity.
\item Microgravity: Zero-g or near weightless environments.
\item Mist: The clouds of AR data that sometimes fog up your
  perception/displays.
\item Morph: A physical body. Aka suit, jacket, sleeve, shell, form.
\item Muse: Personal AI helper programs.
\item Nanobot: A nano-scale machine.
\item Nano-ecology: Pro-tech ecological movement.
\item Nanoswarm: A mass of tiny nanobots unleashed into an
  environment.
\item Neo-Avians: Uplifted ravens and gray parrots.
\item Neogenesis: The creation of new life forms via genetic
  manipulation and biotechnology.
\item Neo-Hominids: Uplifted chimpanzees, gorillas, and orangutans.
\item Neotenics: Transhumans modified to retain a child-like form.
\item Novacrab: A pod created from genetically-engineered spider crab
  stock.
\item Olympian: A transhuman biomorph modified for athleticism and
  endurance.
\item O'Neill Cylinder: A soda-can shaped habitat, spun for gravity.
\item Oort Cloud: The spherical ``cloud'' of comets that surrounds the
  solar system out to about one light-year from the sun.
\item PAN: Personal area network. The network created when you slave
  all of your minor personal electronics to your ecto or mesh inserts.
\item Pandora Gates: The wormhole gateways left behind by the TITANs.
\item Pods: Mixed biological-synthetic morphs. Pod clones are
  force-grown and feature computer brains. Also known as bio-bots,
  skinjobs, replicants. From ``pod people.''
\item Posthuman: A human or transhuman individual or species that has
  been genetically or cognitively modified so extensively as to no
  longer be human (a step beyond transhuman). Aka parahuman.
\item Prometheans: A group of transhuman-friendly seed AIs that were
  created by the Lifeboat Project (precursors to the argonauts) years
  before the TITANs became self- aware and that (mostly) avoided
  Exsurgent infection. The Prometheans secretly back Firewall and work
  to defeat existential threats.
\item Proxies: Members of the Firewall internal structure.
\item Psi: Parapsychological powers acquired due to infection by the
  Watts-MacLeod strain of Exsurgent virus.
\item Reaper: A warbot synthmorph.
\item Reclaimers: A transhuman faction that seeks to lift the
  interdiction and reclaim Earth.
\item Redneck: A rural Martian. See Barsoomian. Aka Reds.
\item Reinstantiated: Refugees from Earth who escaped only as bodiless
  infomorphs, but who have since been resleeved.
\item Resleeving: Changing bodies, or being downloaded into a new
  one. Also called remorphing, reincarnation, shifting, rebirthing.
\item Rusters: Biomorphs optimized for life on Mars.
\item Scorching: Hostile programs that can damage or affect
  cyberbrains.
\item Scum: The nomadic faction of space punks/gypsies that travel
  from station to station in heavily-modified barges or swarms of
  ships. Notorious for being a roving black market.
\item Seed AI: An AGI that is capable of recursive self-improvement,
  allowing it to reach god-like levels of intelligence.
\item Sentinels: Agents of Firewall.
\item Shell: A synthetic physical morph. Aka synthmorph.
\item Simulmorph: The avatar you use in VR simulspace programs.
\item Simulspace: Full-immersion virtual reality environments.
\item Singularity: A point of rapid, exponential, and recursive
  technological progress, beyond which the future becomes impossible
  to predict. Often used to refer to the ascension of seed AI to
  god-like levels of intelligence.
\item Singularity Seeker: People who pursue relics and evidence of the
  TITANs or other possible avenues to super-intelligence, either to
  learn more about it or to become part of a super-intelligence
  themselves.
\item Skin: A biological physical morph. Aka meat, flesh.
\item Skinning: Changing your perceived environment via augmented
  reality programming.
\item Sleight: A psi power.
\item Slitheroid: A snake-like robotic synthmorph.
\item Smart Animals: Partially-uplifted animal species (including
  dogs, cats, rats, and pigs). Some other large smart animals (whales,
  elephants) are nearly extinct.
\item Spime: Meshed, self-aware, location-aware devices.
\item Splicers: Humans that are genetically modified to eliminate
  genetic diseases and some other traits. Also known as genefixed,
  cleangenes, tweaks.
\item Swarmanoid: A synthetic morph composed from a swarm of tiny
  insect-sized robots.
\item Sylphs: Transhuman biomorphs with exotic good looks.
\item Synthmorph: Synthetic morphs. Robotic shells possessed by
  transhuman egos.
\item Synths: A specific type of synthmorph. Synths are standard
  androids/gynoids; robots that are designed to look humanoid, though
  they are usually noticeably not human.
\item Teleoperation: Remote control.
\item Titanian: Someone from Titan, a moon of Saturn.
\item TITANs: The human-created, recursively-improving, military seed
  AIs that underwent a hard-takeoff singularity and prompted the
  Fall. Original military designation was TITAN: Total Information
  Tactical Awareness Network.
\item Torus: A donut-shaped habitat, spun for gravity.
\item Transgenic: Containing genetic traits from other species.
\item Transhuman: An extensively modified human.
\item Trojans: Asteroids or moons that share the same orbit as a
  larger planet or moon, but follow about 60 degrees ahead or behind
  at the L4 and L5 Lagrange points. The term Trojans normally refers
  to the asteroids orbiting at Jupiter's Lagrange points, but Mars,
  Saturn, Neptune, and other bodies also have Trojans. See also
  ``Greeks.''
\item Uplifting: Genetically transforming an animal species to
  sapience.
\item Vacworker: Space laborer.
\item Vapor: A failed mind emulation or crippled fork/infomorph (from
  vaporware).
\item VPNs: Virtual private networks. Networks that operate within the
  mesh, usually encrypted for privacy/security.
\item VR: Virtual Reality. Imposing an artificially-constructed
  hyper-real reality over one's physical senses.
\item X-Caster: Someone who transmits/sells XP recordings of their
  experiences.
\item Xenomorph: Alien life form.
\item Xer: As in ``X-er''—someone who is addicted or obsessed with
  XP. Sometime used to refer to people making XP as well.
\item XP: Experience Playback. Experiencing someone else's sensory
  input (in real-time or recorded). Also called experia, sim,
  simsense, playback.
\item X-Risk: Existential risk. Something that threatens the very
  existence of transhumanity.
\item Zeroes: People without wireless mesh access. Common with some
  indentures.
\end{itemize}

\begin{quotation}

\textbf{Welcome to Firewall}

[Incoming Message Received. Source: Unknown]

[Quantum Analysis: No Interception Detected]

[Decryption Complete]

Greetings,

Your references and background have been triple-checked and confirmed,
and you are now vetted as a sentinel operative. Welcome to Firewall,
friend.

For those new to our private network, Firewall is an organization
dedicated to protecting transhumanity from threats—both internal and
external—to our continued existence as a species. The Fall may have
reminded us that our ability to survive and prosper is not guaranteed,
but our kind has a remarkably short attention span. Despite our
achievement of functional near-immortality, we continue to face
numerous dangers that may contribute to our extinction. Some of these
risks come from our own factionalism and divisiveness, combined with
universally available technology that could cause widespread
destruction and untold deaths in the wrong hands. Some stem from our
short-sightedness, failing to see the dangers in which we place
ourselves and our environments through careless actions.  Some arise
from our own creations turned against us, as the TITANs proved. Other
risks may come from alien intelligences whose motivations we cannot
yet fathom, and of whom we may not even be aware. Still others may
threaten us by sheer chance and the mindless but deadly
cause-and-effect of a universe in which we are but an insignificant
speck.

Firewall exists to identify, analyze, and counter these risks. We are
all volunteers. We are all placing our own lives at risk in order to
ensure the survival of transhumanity.

Firewall has existed, under other names and guises, since before the
Fall. Numerous agencies with a similar agenda banded together in the
wake of those cataclysmic events to assess our situation and prepare
for the worst. Now we operate under a single umbrella.

We are a private network for two reasons. First, our existence and
operational abilities are protected by our secrecy. The less our
opposition knows about us, the more effectively we can counter
them. Similarly, certain authorities might be hostile to an
organization such as ours operating in their claimed territory. Though
some may be aware of our existence, we bypass numerous legal and
jurisdictional hurdles that might otherwise hamper our actions and
goals. Second, our mission sometimes brings to light information that
is not only dangerous in the wrong hands, but might even trigger
widespread panic if made public. In some cases, the very existence of
such knowledge could be problematic. By retaining secrecy and
operating on a need-to-know basis, we automatically counter certain
risks.

Firewall is a decentralized, peer- to-peer network. We have minimal
hierarchy and we answer to no one but ourselves. Our node structure
enables us to share resources and talents without sacrificing the
privacy and security of our operatives.  You have been recruited
because of your knowledge, assets or skills, and/or because you have
come into contact with certain restricted data. You have proven your
willingness to support our goals.  Our lives and existence—and the
future of transhumanity—may rest in your hands.

So here's to the future—may we all live to see it.

[End Message]

[This Message Has Self-Erased]

\end{quotation}

\begin{quotation}

\textbf{What you really need to know}

[Incoming Message Received. Source: Unknown]

[Quantum Analysis: No Interception Detected]

[Decryption Complete]

Sit down, and grab yourself a fucking drink.

Forget all of that AI-generated intro crap you just read. Here's the
real deal.

You're probably dying to know what you've been dragged into.  Maybe
you've been told the party line already: that we're all that stands
between transhumanity and extinction. Or maybe someone whispered to
you that we're a rogue operation that meddles in heavy shit that we
have no authority to get involved in, and that we sometimes get people
killed as a result. You must be curious. Maybe you've got a vigilante
streak, and you're looking to spill blood for a good cause. Would it
matter to you if the cause was a deluded one?  Maybe you're a
conspiracy wingnut and you're dying to know what secrets Firewall is
clutching to its collective chest. What if those secrets shattered the
carefully constructed lies that we all tell to ourselves to keep our
sanity intact?

Everything you've heard, good or bad, about Firewall very well may
be true. We're not angels. We lost the sheen on our ideals when the
TITANs forcibly uploaded their first human mind. Right now, you should
be asking yourself what the fuck you just signed up for. I did.

Truth is, Firewall is lots of things.  Most of it is good, but a
lot of it so fucking horrible you'll be thinking about planting a
bullet in your stack and resorting to an earlier backup, just so you
can forget it all. If you have any romantic visions about being a
hero, though, drop them now. You won't feel like a hero when you
airlock some kid because he's carrying an infectious nanovirus. You
won't feel brave when you run across some alien thing and crap your
pants.  And you won't even feel human anymore when you make a call
that will cost dozens, hundreds, or even thousands of people their
lives, even if you are saving millions more.

So why would anyone be crazy enough to be part of this thing?
Because it needs to be done. Our survival depends on it. To some
people, it's altruism, defending transhumanity. But really, it's about
saving your own fucking neck too.  Sure, you could abstain from taking
responsibility and let some self- described authority take care of it.
But if the anarchists have anything right, it's that people in power
can't be trusted. As often as not, they're part of the problem. So
Firewall does things the collective way. We're underground, but we're
an open source operation. We share information and resources towards a
common goal.  We organize in networked ad-hoc cells, smart-mob
style. We don't let anyone accrue too much power or control. Everyone
involved in an op has an equal say. We police ourselves. We come from
all sorts of backgrounds and factions, but we face a common enemy—and
we fight to win. There is no alternative.

Maybe you've heard of the Fermi Paradox? That question asked why, with
a galaxy so huge, there were so few signs of other life? Even though
we've met the Factors and seen evidence of other aliens, our galactic
neighborhood should be crawling with intelligence—but it's not.

I'll tell you why. The universe is not fucking fair. If transhumanity
were wiped out, the galaxy wouldn't even notice. Just look at the
Earth. That planet still exists, still supports life, even though
we're far gone. Reality is an uncaring asshole. Forget all that
utopian crap about living forever. We'll be lucky to survive another
year. We've developed technologies that put weapons of mass
destruction in the hands of everyone, but we're still an adolescent
species that has trouble overcoming petty tribal bullshit. If you're
really looking forward to exploring the universe as a postmortal,
you're going to have to work hard at it. Survival isn't a right, it's
a privilege.

When you sign up with Firewall, you put yourself on call. Anytime some
shit goes down in your neck of the woods or that you might be
particularly helpful in dealing with, you'll get a call. You'll be
expected to drop whatever you're doing and put everything else on hold
as if your life depended on it—it probably does.  When you're in the
field, on an op— ''going to the doctor,'' as we call it—your cell is
empowered to act as it sees fit ... just keep in mind that you'll be
answering to the rest of us later. You'll also have the Firewall
network to back you up—though resources are often limited, so don't
expect us to always save your ass.  Other sentinels can be called on
to pull strings, but every time we do so, it threatens to unveil an
agent, create a trail that we need to clean up, and otherwise
complicates matters. Self-reliance is key.

One last thing: don't ever, ever forget that we have enemies. I'm not
just talking about the nutjob who wants to nuke a habitat to make a
political statement or the neo-luddites who think biowar plagues will
teach us all a lesson, I'm talking about the agencies that know
Firewall exists and consider it a threat. If they tag you as a
sentinel, your days are numbered.  Maybe your backups too. So watch
yer friggin' back.

So that's the real deal, as honest as I can give it. Welcome to our
secret clubhouse, comrade. Remember: death is just another day on the
job.

[End Message]

[This Message Has Self-Erased]
\end{quotation}

%%% Local Variables: 
%%% mode: latex
%%% TeX-master: "ep"
%%% End: 

\chapter{Une époque d'éclipse} \label{chap:a-time-of-eclipse} 

Ce chapitre fournit un aperçu complet de l'univers d'Eclipse Phase. Il démarre avec un historique, explore en détail le cadre de jeu, détaille les factions et termine avec une gazette du système. 

\section{L'histoire populaire d'un univers malheureux.} \label{sec:peoples-history-an} 

\begin{quotation} Le texte suivant est une transcription d'un fichier audio récupéré après la décompression catastrophique de la Station Walther-Pembroke. Le fichier audio est attribué à Donovan Astrides et semble être un résumé de son ouvrage inédit: Une Histoire Populaire d'un Univers Malchanceux. \end{quotation} 

[Bruits de grattements sur le microphone, craquements de mobilier, bruit d'une femme se raclant la gorge] 

Quoi? 

[Murmures indistinct] 

Vas te faire foutre. Je fait ça de la putain de façon que je veux, même si c'est sympa de ta part de me coller dans ce sympathique corps féminin. 

[Bruits de mains lissant du tissu] 

Ma vulgarité te choque, laquais corporatiste? Je m'en fout, je suis s ûr que tu pourra éditez ça pour tes prolos. Bon - qu'est-ce que tu m'as demandé déjà? Si mon livre était un livre d'histoire? Non, c'est un livre d'anti-histoire. Je vais te parler du futur. 

[Murmures, ton interrogatif] 

Qu'est-ce qu'il contient? Tu veux connaître le futur? 

["Oui" indistinct.] 

Non, je ne penses pas que tu te préoccupes du futur. Ce que tu veux vraiment savoir c'est: aurez-vous le futur que vous voulez? Et c'est une question à laquelle il est facile de répondre. Non. Non, tu n'aura pas le futur que tu veux. Simplement parce que tu es suffisament stupide pour poser cette question idiote à propos du futur. 

[Pause silencieuse] 

Je me souvient avoir lu un scan d'une impression d'un comic un jour. Le personnage vitupérait contre les habitants imaginaires de son monde imaginaire, les forçant à agir contre les insatisfaction du futur dans lequel ils vivaient. Mais c'était en fait ciblé vers les personnes stupides qui voulaient leur petit futur médiocre et qui étaient trop bête pour voir que ce futur était le présent. C'est toujours le cas. Sauf que ça ne l'est plus. Les TITANs ont changés ça. Le futur est maintenant hier, la semaine dernière ou bien il y a dix ans. Plus particulièrement il y a dix ans. Mais le futur remonte aussi à la bonne vieille Terre - c'est un héritage de là où nous étions et de ce qui était avant nous. 

Vous apprennent-ils l'histoire sur Vénus, dans vos comporatiments scellés et vos aérostats de loisirs? Non, n'ouvres pas votre bouche, je me moque de ce qu'il vous enseigne comme de ma première morph. Il s'agît trés probablement de mensonges. J'ai vécu dans le système intérieur. Je connaît les règles et les tromperies énoncés au nom de l'ordre civil et de la "sécurité nationale." 

Nations! Ha! Même au début du 21ème siècle, les nations avaient commencées à décliner. Ça à juste pris un peu de temps pour que tout le monde s'aperçoive qu'elles étaient obsolètes. 

Tu te rappelez les grandes nations du monde? Es-tuy suffisament vieux pour te rappeler comment ils se sont assit autour de la table pour débattre si les changements climatiques majeurs qu'ils avaient créés étaient réel? Même lorsque la plupart d'entre eux étaient d'acord sur le fait qu'il fallait faire quelque chose, aucun d'entre eux ne s'est porté volontaire pour le faire. Les dirigeants du monde on continué à gérer leurs affaire comme d'habitude, renforçant leurs privilèges alors que la sécheresse ravageait l'Afrique et l'ASie Centrale, et que des incidents météorologiques graves amenaient la destruction partout ailleurs. Partout dans le monde, les gens commençaient à subir les affres de la faim ou à être décimées par des épidémies rampantes, mais les nations dominantes étaient plus préoccupées par les réfugiées qui débordaient les frontières et polluaient leur petit paradis personnel avec leurs coutumes, leurs langues et leurs volonté à travailler pour une misère juste pour survivre. 

Les guerres du pétrole et de l'énergie n'ont été surpassée en horreur que par les guerres pour le contrôle de l'eau et du climat qui suivirent. Des régimes instables grandirent et chutèrent ou furent mis à bas, juste pour obtenir plus du précieux liquide. Les grandes nations états se transformères en forteresse, se cuirassant contre la menace jumelle des barbares les menaçants de l'extérieur et les masses des pauvres et des dépossédés les menaçant de l'intérieur, les deux ne voulant simplement qu'un peu d'eau à boire. 

Tu sait, j'ai réellement entendu des conservateurs se référer à cette période comme à l'âge d'or, l'apogée des corporations et des riches. C'est une certitude, c'était l'âge d'or de la répression - et des profits. Si vous étiez dans cette fraction de la population  suffisament chanceuse pour pouvoir se l'offrir, c'était très certainement une bonne époque, masi pour la majorité de l'humanité c'était une époque d'horreur. Les inégalités globales étaient plus importantes que jamais. Les robots retiraient le boulot des mains de l'humanité. 

Ce fût une époque de radicalisation pour beaucoup de gens. Les gouvernements en déroutent ne fournissaient plsu les besoins basiques des populations. Le pauvre globalisé se tourna vers les tribus locales, les groupes fondamentalistes, les politiques radicaux et les réseaux criminels pour trouver des moyens de survivre. Des groupes insurrectionnistes fleurirent, mais ils dépendaient du marché noir pour survivre, et bientôt leurs chefs étaient plus occupés à gagner de l'argent qu'à changer les choses. 

Les nations états en vinrent, comme d'habitude, à la répression. Les libertés civiles furent restreintes et la surveillance augmenta. Des systèmes d'arme automatisés furent déployés d'abord contre les unités de guerillas et les cellulles terrorristes, puis contre les agitateurs et les manifestants. Je me rappelle la première fois que j'ai vu des drônes de police. C'était dans une manifestation en soutien à une grève des travailleurs à Long Beach. Les drônes nous ont ordonnés de nous déployer une fois, une seule fois, avant d'ouvrir le feu avec leurs armes "non léthale". Non léthale, mon cul oui. Trois personnes sont mortes ce jour là et des douzains furent blessées. Les médias de masses l'ont ignoré, même si les bloggeurs en ont parlé. 

Pendant ce temps, les élitées privilégiiées continuaient de prospérer. Les traitements de longévité augmentèrent l'espérance de vie - pour ceux qui pouvaient se les offrir. Des répressions massives balayèrent les progrès en pharma hors-labo et des procédures pionnères mise au point par des biochimistes aventureux, alors même que l'espérance de vie mondiale chuta pour la première fois depuis des décenniées Pourquoi étendre la vie de tant de gens, quand des systèmes experts aussi intelligent que n'importe quel humain peuvent être construits en une fraction du temps nécessaire à éduquer une personne alros que la robotique et les drones permirent de réaliser les tâches subalternes par un ouvrier qui n'est pas payé et qui ne s'en plaint pas. Les riches avaient leurs animaux chimériques conçus par des designer hors de prix pour leur tenir compagnie de toutes façon. 

Toutes les classes supérieures en se vautraient pas dans l'opulence pendant que la planète autour d'eux s'affamait et sombrait. Quelques uns étaient à la recherche du changement, réfléchissant à la manière d'obtenir gain de cause. Certains d'entre eux cherchaient à étendre leur territoire, construisant un ascensseur spatial dans l'Afrique Sud-Saharienne et en envoyant des sondes robotisées dans le système solaire pour le cartographier en détail. Ils fondèrent même les premières stations sur Mars et sur la Lune, plsu de cinquante ans avant la Chute. 

L'écopocalypse ne semblait pas s'arréter cependant, peu importe à quel point les dirigeants du monde essayaient de l'ignorer. Les hivers rudes et la sécheresse continuaient à nous tomber dessus. La montée des océans dévasta le littoral au niveau mondial avec des inondations massives. Quelques effort désespérés pour mettre en place de gigantesque s projets de géoingénierie créèrent autant de problème qu'ils n'en solvèrent. Ils furent  cependant perçus cyniquement, car certains de ces efforts étaient des tentatives à peine déguisées de technique de terraformation en préparation de la colonisation extra-terrestre. 

Cela donnait l'impression que les yeux des fortunés n'étaient plus tournés sur le monde qui les entouraient, mais sur les cieux au-dessus d'eux. La réalisation du premier ascensseur spatial et de la première catapulte électromagnétique sur la lune lança une nouvelle course spatiale et la compétition pour revendiquer des droits dans le système solaire. Toute cette expension fût alimentée par la production de masse des premières centrales à fusion efficaces et par la fondation d'entreprise de minage de l'Helium-3. 

Sur Terre cependant, l'épée de damocles finît par tomber sur l'humanité. Les insurgés adoptèrent des techniques de guerillas de cinquième génération, partageant des méthodes de résistances open source, utilisant des attaques massives sur des infrastructures critiques. Les gens écrasés par des années d'oppressions profitèrent de ces opportunités et démolirent l'appareil étatique et corporatiste qui les avait opprimés. Les nations sombrèrent les unes après les autres dans des révolutions menées par ceux qui ont combattus lors des milleirs de guerre pour du carburants, des étandues d'eau et des croûtes de pains. 

la plupart des états répliquèrent en devenant encore plus totalitaire et répressifs, mais le courant de rébellion se répandit en dehors du monde alors qu'une série d'avan-postes et de statiosn se déclarèrent en soutien de ces compatriotes terriens et publièrent un manifeste pour une apporche plus humaniste de l'expansion solaire. De nombreux scientifiques et chercheurs, qui travaillaient précédemment comme pions dans l'expension corporatiste, adoptèrent même une attitude technoprogressiste. C'est ainsi que nacquit le mouvement des argonautes tu sait, en tirant leur nom d'un groupe de scientifique qui conseillait le gouvernement US et le Pentagon dans le domaine de la science et de la politique appellés les Jasons. Devant faire face à des représailles de leurs anciens maître corporatiste, bon nombre d'argonautes quittèrent les hypercorps, emmenant leurs ressources et leurs recherches avec eux dans certains cas, tandis que d'autres choisirent d'entrer en clandestinité. 

Ce fut cependant le moment où les connarde de requins d'hypercorps décollèrent réllement. Ils laissèrent les nation-états et les vieilles multinationales subir l'assaut de la colère globale. Ils tirèrent avantages du chaos pour s'affranchir des vieilles contraintes morlaes et éthiques sur l'expérimentation humaine et du point de vue légal des nation qui les avaient données naissance. Ils embrassèrent les opportunités des nombreuses nouvelles technologies et les lancèrent dans l'espace. Ce sont leurs laboratoire de recherches qui ont créé la première intelligence artificielle consciente, le premier clone humain conçu génétiquement et le premier vrai élevé, des chimpanzés et des dauphins amenés à la conscience en tant qu'expérimentation et esclave corporatiste. 

Alors que les derniers des vieux état s'accrochaient désespéremment à leur pouvoirs et à leurs terres, les hypercoprs leur tendirent la main. Elle offrirent la servitude pour dette à ceux qui voulaient abandonner leurs droits et leur humanité pour un voyage extra-terrestre et pour travailler en tant qu'engagé dans les colonnies et stations corporatistes. Des centaines de milliers de personnes acceptèrent l'offre comme une alternative à la pauvreté et au chaos sur Terre. L'exploitation de ressource fut un domaine qui explosa à travers tout le système solaire, alors que des statiosn furent établies jusqu'à la Ceinture de Kuiper. Les voix qui s'élevèrent pour le respect de la biodiversité et des écologies naturelles furent ignorées alors que les hypercorps peinaient à transformer diverses planètes et lunes à leur volonté. 

Voilà les choses telles qu'elles étaient à peu près 20 ans avant la Chute. Même si nombre des veieux étants oppressuers avaient été détruits, de nouveaux émergeaient, et les différentes insurrectiosn globales oscillaient entre effectuer des changements radicaux et tomber dans le piège de la guerre tribale. Des forces réligieuses réactionnaires et politiques sur Terre ont également ralé contre les plans des hypercorps, amenant à des attaques terroristes et des grèves de sabotages qui ont eu pour point culminant une tentative de sabotage de l'ascensseur spatial par une cellulle Islamiste suicidaire. Les hypercorps furent promptes à répliquer, déclenchant un bombardement orbital en utilisant des objets à forte densités contre les quartiers généraux et les éléments de plusieurs personnes clefs de l'opposition. Bien que la destruction massive fut efficace pour décapiter les réseaux terroristes, elle déclencha une vague d'indignation intense contre les hypercoprs, créant un fossé encore plus grand entre la Terre et les colonies extra-terrestre. 

Les hypercorps restèrent hors d'atteintes bien que n'étant pas complètement immunisées des troubles terriens. Les travailleurs et les colons amenés de Terre embarquèrent avec eux beaucoup de leur troubles ethnique, politique et sociaux, menant àplusieurs explosions de violence dans les habitats et les stations orbitales. Certains hébergèrent des groupes opposées aux intérêts hypercorporatistes, s'illustrant par des actes isolés de sabotage préservationnistes et d'attaque terroriste religieuse. Différents réseaux criminels entrèrent aussi dans la course, étendant leurs marchés noirs et le commerce du vice partout où l'homme allait. 

Les opposants aux hypercoprs se développèrent au même rythme que ces dernières: anarchistes, socialistes, argonautes et d'autres œuvrèrent avec diligence pour établir leurs propre présence indépendante, essentiellement dans le système extérieur, hors de portée des hypercorps. Les hypercrops contribuèrent à cette croissance en envoyant leurs criminels et autres éléments indésirables en exil au-delà de Mars. 

Les deux camps investirent massivement dans a recherche et les nouvelles technologies. Les avancées en biotech, nanotech, dans le domaine des IA et en science cognitive arrivaient à présent à un rythme si rapide que des découvertes majeures étaient faites sur une base annuelle. Le développement dans un domaine créait un gain récursif dans un autre, créant une boucle de retour qui engendra d'immense améliorations technologiques. Hors-terre, les modifications génétiques furent largement adoptées, et les nouvelles adaptations transhumaines devinrent communes. Nous avons même créé de nouvelles formes de vie synthétique à la foid biologique et robotique. Même si certains ont étés révulsés par ces développement, au point qu'ils ont appelés ces nouveaux types les "pod people," cela n'a absolument pas ralenti l'intégration et l'absorption rapide des pods dans les forces de travail des corporation et dans les bordels, et cela n'a pas non plus suffit à motiver suffisament de gens pour défendre le fait que, étant des êtres conscient, les pods devraient avoir leurs propres droits civils. 

Deux découvertes de cette période nécessite une mention spécifique, au moins à cause de leur impact sur notre société humaine - devenue transhumaine. Le développement du premier assembleur nanotechnologique a démarrer un chgangement de paradigme pour l'économie. Disponible seulement pour les strates supérieures des hypercorps, du moins au début, ces élites ont jalousmeent protégés ces machines, capables de construire à peu près n'importe quoi depuis l'état atomique. Elles ont placé toutes sortes de restrciction sur leur utilisation et leur disponibilité, prétendant que la capacité à construire des drogures, des armes ou d'autre objets restreints étaient un risque sécuritaire qui nécessitait qu'ils soient strictement contrôllés. Les défenseurs de l'open source ont bien entendu rapidement travaillés sur une manière de contourner le contrôle des schémas et à semer leurs propre conception open source. De la même manière, en l'espace de quelques mois, les criminels et les anarchistes ont libérés leurs propres assembleurs, et un conflit économique apparu de manière soudaine. Certains furent utilisés pour alimenter le marché noir, alros que d'autre furent utilisés pour établir des habitats et des colonnies avec un système économique post-pénurie qui ne dépendait plus de la richesse, de la propriété ou de l'avarice. 

C'est à peu près à ce moment que la capacité à cartographier le cerveau humain et à émuler numériquement l'esprit et les souvenirs fut découverte, rendant "l'upload" opssible - rapidement suivi par la capacité à se télécharger dans un cerveau humain différent bien entendu. Les maîtres des hypercorps, qui avaient déjà une vie trés longue, n'avaient désormais plus besoin de craindre une mort accidentelle ou suite à des blessures. Cette technologie se fraya un chemin entre les mains d'autres personne, en dépit des coûts. Des expérimentations avec d'autres corps - biologique et synthétique - devint un nouveau terrain de jeu pour la culture. Et n'oublions pas ceux qui était volontaire pour aller abandonner le joug de la chair pour faire l'expérience de la vie virtuelle et s'évader dans leur propres rêves devenus réalité. 

Alors que nous nous éclations avec nos nouveaux jouets, la Terre, pauvre Terre, continuait de mourir d'une mort lente. Je peux encore me souvenir des sépcualtions qui pensaient qu'il faudrait des siècles pour que la terre sombre totalement dans une dévastation écologique. C'était frustrant, partout où vous regardiez il y avait quelqu'un pour se lamenter sur le sort de notre monde mère, mais personne ne voulait faire quoi que ce soit. C'éatait trop cher, trop loin ou trop dangereux. Nous avons tou du sang sur les mains depuis cette époque. Nous nous tenions là à regarder le monde brûler autour de nos frêres et sœurs depusi nos habitats orbitaux. Nous pensions avoir du temps, nous pensions que le monde mourrait lentement et que nous pourrions trouver un remède. Nous n'avions pas pévu les TITANs. 

Nous nous souvenons tous de la Chute. C'était il y a à peine dix ans, mais je ne cesserai jamais d'être étonné par l'état de confusions des souvenirs que les gens ont de ce moment. Bien sûr, une partie de cette confusion vient de la propagande qui leur a été servie par des gens comme toi, mais une part non négligeable vient du fait que la plupart d'entre nous sommes effrayés par le fait de prendre du recul et d'examiner attentivement comment nous, les humains, avons réussit à faire foirer le tout de manière si spectaculaire. 

Nous aimons prétendre que les TITANs son arrivés sur scène, ont tout détruit autour d'eux, et ont disparus aussi vite qu'ils étaient apparus. La vérité, comem tuoujours, est bien plus complexe. Nous prétendons savoir que les TITANs ont évolués accidentellement d'un réseau de guerre électronique militaire, ou du moins c'est l'idée dominante. C'était ce que voulait dire leur nom: un acronyme pour Réseau Conscient d'Information Complète et tatcique (Total Information tactical Awareness Networks). Personne ne sait réellement d'où venait ces premières IAs germe cependant - ou si quelqu'un le sait, il reste silencieux. Les TITANs ont peut-être été conçus intentionnellement pour être des intelligence numérique consciente s'améliorant de manière récursive. Peut-être que les gros bonnets militaires pensaient pouvoir garder une telle intelligence sous contrôle, et que ça leur donnerai l'avantage dont ils avaient besoin. Peut-être qu'il n'y en avait qu'un seul au début et qu'il en a rapidement créé des centaines voire des milliers de copies de lui-même. Personne ne semble savoir combien il y en avait. 

D'après l'histoire officielle - vérifiées par les hypercorps - nous savons maintenant que les TITANs ont mis plusieurs jours après leur "réveil" pour analyser le monde autour d'eux, pour apprendre des choses sur nous. Lors de leur phase initiale, ils étaient relativement bénin, détournant l'énergie et les ressources des réseaux uniquement là où il y avait de l'excédent, étendant leur perception au delà de leur berceau terrestre. Peut-être qu'ils absorbaient tout ce qu'ils pouvaient pour nous comprendre. Peut-être qu'ils étaient indifférents. Ou peut-être qu'ils essayaient réellement de nous détruire, comme ils le disent dans les films. 

Je me rappelle de cette époque. Je me rappelle que lrosqu'une nouvelle tournée de conflits ré-enflamma la Terre, on a jamais entendu parler d'IAs germe ou de TITANs. Pendant des mois, c'était simplement une escalade des hostilités. Cette escalade avait démarré par des accusations d'opération de cyberguerre et d'intrusions majeures, déclenchant des alarmes et des attaques de représailles. Les attitudes agressives ont amenés à l'incriminiation, puis à des conflits forntaliers et à des raids, suivit par des frappes de missiles et des hostilités globales. De vieilles rancunes et des ennemis endormis se réveillèrent soudainement et tournèrent leur rage renouvellée vers leurs anciens rivaux. Des guerres d'escarmouches, des rivalités corporatistes et des conflits idéologique s'embrasèrent alors que les insurrectiosn et la rébellion étaient soudainement partout. À ce moment, cela ressemblait à un étalement de violencement pas si inhabituel avait prit un tournant drastique et s'emballait rapidement. 

D'après la doctrine officielle, c'était un effort minutieusement planifiés et concerté, la première étape dans le plan des TITANs. C'était peut-être le cas, mais j'ai le souvenir d'officiers militaires affirmant que les TITANs ont été mis en ligne suite à cette violence et non pas avant - une opinion rapidement réduite au silence. Encore une fois, nous avons peut-être réellement étés manipulés - maipulés par une intelligence supérieure qui n'ont pas voulus s'emmerder à s'occuper de nous quand ils ont découvert de manière plsu que certaines que nous étions d'accord pour s'assassiner et s'anihiler mutuellement. 

Lorsque les premières mentions d'étrange usines automatisées fabriquant de grand nombre d'armes robotisées nous parvinrent, personne ne savait qui blâmer, mais quelque chose allait manifestement de travers. Ce fut un tournant, une chance pour l'humanité de réaliser que nous faisions face collectivement à un nouvel ennemi, mais l'accusation mutuelle et les conflits directs continuèrent. Même lorsque la première attaque ouverte des TITANs arriva, crashant des systèmes critiques, prenant le contrôle d'infrastructures critiques et amenant la désolation et la destruction, nous l'avons considérées comme un nouveau front dans la guerre, et n'avons jamais cessés de nous entretuer. 

Le fait de savoir si nous aurions dut essayé de parler aux TITANs fait encore débat, soit ils auraient été d'accord pour nous écouter, soit ils nous auraient vus comme nous voyons les rats et les cafards et d'autres formes de vermine. Mais tout cela est théorique. Le fait est que nous n'avons pas essayés de leur parler. Les personnes qui ont pris les décisions, ceux qui devaient tout décider à ce moment là, virent les TITANs comme une menace. Et ils ont agît de manière conforme, essayant de les purger de leur système ou de les capturer pour pouvoir les étudier plus tard. 

Le philosophe Thomas Hobbes parla une fois de la guerre de chacun contre tous. Il n'aurait pas pu imaginer ce qu'allait être le conflit démarré apr les TITANs. Nous avons tués des millions de snotres, utilisant le feu nucléaire et la mort silencieuse des armes biologiques. Les TITANs avançaient au milieu de ce carnage, prenant le contrôle de nos machines comme si nous étions des enfants, moissonant des millions d'esprits par des uploads forcés dans un but inconnus. Toutes les attaques que nous avons menées contre les TITANs se termina par des disastres et des échecs indicibles, tout nos artifices et appareils se retournèrent contre nous au moment où nous en avions besoin. 

La Chute fut une horreur. Des usines surgirent comme une brûlure dans les endroits les plus désertiques et ravagés sur Terre, libérant des légions de machines de guerre terrifiantes. Des essaims de nanites évolués - bien au-delà de nos propres capacités - ont infestés l'ensemble de la planète, s'adpatant et mutants pour gérer toutes les menaces qu'ils rencontrèrent. Des nanovirus biotechnologique se propagèrent dans les populatiosn humaines, leur infligeant des domages neurologiques irréversibles. Des vers d'infowar puissants pénètrèrent même les systèmes les mieux protégés, dispersant nos réseaux cruciaux sans souci. Des populations captives ont été arrétées pour subir des émulations cognitives forcées, victimes d'un destin plus heureux que ceux qui ont été simplement décapités par des drônes chasseurs de têtes ou transpercés par des drônes possédant des trompes scannant les systèmes neuronaux. Des virus neuropathiques ont transformés des humains en pions des TITANs, les retournant contre le reste d'entre nous. D'autres rapports parlent d'évènement étrange et de terreurs inimaginables. Nous nous sommes retrouvés en train de mener une lutte à l'arrière-garde contre une extinction proche. Les intrigues de centaines de romans et de films se manifestait pendant notre vie, l'extinction de la transhumanité des mains des machines. 

Pendant presque un an ils nous ont harcelés et détruits. Il ne semblaient pas pressés de nous exterminer, et pourquoi se seraint-ils pressés? Rien de ce que nous avons tenté ne les affectaient. Ils étainet des données et de l'information, ils étaient des pensées et des impulsions, ils étaient partout et nulle part, et il n'y avait rien que nous puissions faire qu'ils ne pouvaient pas retourner contre nous. Leur influence se propagea au-delà de la Terre, avec des incursions en orbite, sur la Lune, sur Mars et dans bien d'autres endroits. Partout où nous avions une prise, les TITANs nous ont suivis. 

Tu te rappelles peut-être le moment où il fût évident que la transhumanité pourrait ne pas survivre. Moi oui. Des millions d'entre nous ont du voir les signes. Et la grande diaspora commença, les masses grouillantes faisant tout ce qu'ils pouvaient pour quitter la Terre. Des vaisseaux furent détournés, d'autres furent construits afin d'éaider les gens à s'enfuir. Ceux qui ne purent pas s'acheter un billet firent du mioeux qu'ils purent pour envoyer leur sauvegarde digitale, avec l'espoir ténu qu'ils pourraient obtenir un nouveau corps. Peut-être une personne sur dix pu s'échapper. 

Tu as peut-être entendu que nous nous sommes regroupés pour arréter la menace, que dans nos heures les plus sombres, nous avons pardonnés les anciennes rancœurs et les haines rampantes face à l'extinction. Ce serait un mensonge jetté à la face des dizaine de milliers qui ont été abbatus à Buenos Aires par les forces Nord Américaine alors qu'ils cherchaient à s'enfuir, où aux morts des deux douzaines d'habitats en orbites des Lagranges dont les systèmes de survie ont étés sabottés par des compétiteurs corporatistes alors que leurs rivaux se battaient contre les TITANs. Nous étions juste enthousiaste à nous détruire mutuellement. 

Puis, aussi rapidement qu'ils sont apparus, les TITANs s'évanouirent. En l'espace d'une semaine, les attaques et les perturbations réduirent puis s'arrétèrent à l'exception de quelques incidents. Les rétributions et les attaques par les autres continuèrent quelques mois, mais les dégats que nous nous sommes infligés étaient négligeables vis à vis de ce qu'avaient fait les TITANs. 

A la fin, nous nous tenions debout au milieu des ruines fumantes de la transhumanité et nous avons fait l'inventaire de tout ce que nous avions perdu. Des milliards de personnes qui existaient avant la Chute, moins d'un sur 8 avait survécu et encore une fraction de ça possédait une forme corporelle. Néanmoins, les habitats et les stations survivantes étaient bondés et les tensions internes au plus haut. Un grand nombre d'infugiés cicrculaient dans des zones de stockages car il n'y avait tout simplement aps assez de corps pour pouvoir tous les satisfaire. Certains furent placés en mémoire morte, où ils demeurent encore, oubliés de tous. D'autres furent branchés en réalité virtuelle, n'ayant pas d'autre choix que de vivre en espace simulés. Une fraction d'élu reçurent la chance de travailler en tant que contractés, souvent dans le but de construire de nouveaux habitats, travaillant avec la promesse d'avoir un jour un corps leur appartenant. Tu n'aurais eu aucun doute en les voyants, travaillant à des tâches subalternes ou dangereuses, enchassés dans des synthmorphs bon marché et produite en masse, maintenus hors de vue. 

Ceux qui ont été laissés pour mort ou dépourvus d'un corps étaient le moindre de nos soucis. Notre guerre avec les TITANs avait laissé la Terre  à l'état d'une terre dévastée, fumante et irradiée, toujours peuples par de danegreuses machines et épidémies. Le tout juste fondé Consortium Planétaire, composé des intérêts hypercoporatistes parmi les colonnies Lunaire et Martienne, placèrent la terre et l'espace environnant sous quarantaine. La raison officielle est que c'est pour des raisons de sécurité, supposément pour empêcher toute menace de s'échapper de la Terre. À moins que nous ne pouvions pas regarder notre monde natal dans un tel état et admettre ce que nous avons faits nous-même. 

Encore maintenant, dix ans plus tard, on nous dit que la terre est dangereuse, qu'elle regorge de risque et de surprise. C'est partiellement vrai, je pense - il y a des surprises ok, mais le Consortium Planétaire veut garder le morceau pour lui seul. 

[Raclements de gorges, murmures] 

Bien sûr que ke parle d'une porte de Pandorre. Celle que les TITANs ont abandonnée derrière eux sur Saturne était juste la première. Tu es un idiot si tu penses qu'il n'y en a que cinq dans tout le système solaire. Je suis prêt à parier à peu près n'importe quoi qu'il y en a une là bas, sur notre bonne vieille Terre. 

T'as déjà vu une Porte? Non? Bien sûr que non. Les hypercorps les verrouillent. Pas comme dans l'extérieur sauvage. Bien sûre que la Gatekeeper Corp laisse quiconque avec un vœu de mort et l'entraînement minimum de tenter sa chance à travers la porte de pandorre originelle, mais si tu es suffisament chanceux pour revenir, ils possèdent tout ce que tu as trouvé de l'autre côté. Je suppose qu'il y a une chance pour qu'un type d'accros à l'adrénaline "y aille franchement" et toute ces absurdités. 

Les colonnies extrasolaire - maintenant, ce sont elles la nouvelle frontière. Vous, les mecs du système intérieur, vous êtes tellement prévisible avec votre précipitation à coloniser, à vous étendre et à tout posséder; comme si l'univers était juste là pour que vos maîtres riches puisse le réclamer comme leur appartenant. Je suppose que vos colonnies extrasolaire se développe plutôt bien, étant donné le nombre croissant de la dette des pauvres - des âmes conscrites que vous jettez là-bas. Vous avez probablement de grand plans pour bâtir des empire galactiques. Nous. La Transhumanité. Une civilsation galactique. 

Ouais, des squatters galactiques à la limite. Les choses étaient pourtant claires lorsque les garde-frontières du cosmos se sont montrés et nous ont solenellement avertis que nous jouions avec des Choses Qui N'auraient Pas Du Être. 

Peut-être que les Facteurs nous disent la vérité, peut-être qu'ils agiseent en tant qu'ambassadeur pour un groupe d'espèce extraterrestre éloignés qui veulent nous prévenir de rester éloignés de la Technologie Interdite - tu sait, la technologie avec laquelel nous nous sommes déjà brîlé et que nous n'avons pas l'intention d'abandonner. Réfléchit au Deux COmmandements qu'ils nous ont donnés: vous ne créerez pas d'IA auto-apprenante, et vous ne devrez pas utiliser les POrtes de Pandorre. Oups. Tu penses qu'ils savent? À propos de ce qui est arrivés avec les TITANs? Que même nous ne savons aps où ils sont allés et que nous avons quelque peu peur de le savoir? Il savent très probabelemnt que nous utilisont les portes et que nous nous sommes étendus au-delà de notre petit coin perdu, et peut-être que c'est leur réelle peur. Mais pourquoi écouterions nous ce qu'une bouillie de slime extrêmement évoluée nous dit de toutes façon? 

Prendre des risques, c'ets le prix du progrès, non? Admettons-le, nous avons besoin d'espoir. Nous avons besoin d'une nouvelle Terre pour rempalcer celle que nous avons détruite, un endroit ou nous pourrons aller et prospérer comme des lapins et tout foutre en l'air encore et encore. Nous avons besoin de savoir que nous pouvons nous étendre au delà du système solaire, car actuellement il paraît un peu confiné, comme si nous pourrios rapidement être piégés et effacés à jamais si jamais les TITANs revenaient un jour. Nous avons besoin de savoir que nous avons un futur. Nous avons besoin de savoir que nos efforts seront récompensés. Que nous ne nous détruirons aps nous-même. 

Les Égarés en sont la preuve. C'était un objectif noble, accéléré une génération d'enfant à l'âge adulte, mais le process avait des défauts. Prendre des clônes qu'ont à développé en croissance forcée, les élever en RV, et ensuite les balancer dans des corps adultes après qu'ils n'aient été vivant que quelques années de temps objectifs - mais pendant plus de dix huit ans de temps subjectif? Toute une enfance en ayant seulement les autres et des IAs comem seule compagnie. C'est suffisant pour foutre en l'air n'importe qui. C'était une expérience intéressante, mais elle a raté, et maintenant nous avons un autre rappel de nos échecs vivants parmi nous. 

C'est tout à fait nous, dans toute notre splendeur. Il s'est écoulé dix ans depuis la Chute et nous sommes toujours brisés, empétrés dans des querelles, emprisonnés par des slimes, battus par des des logiciels insouciants et cependant nous sommes notre pire ennemi. Nous propageant loin d'une maison que nous ne possédont même plus. Notre nombre se réduisant et déiminuant de plsu en plus avec chaque jour qui passe. Qui nous sauvera? Nous ne voulons même pas nous sauver nous-même la plupart du temps. Du moins, c'est ce qu'il semble. 

Mais si nous ne nous sauvons pas, il n'y a pas de futur. Et moi, pour une fois, je n'ai pas vécu tout ce putain de temps pour abandonner maintenant. Toi, moi, sommes effectivement immortels. La galaxxie entière nous attends. On serait trop stupide de ne pas aller la voir. 

\begin{quotation} Fin de la Transcription \end{quotation} 

\subsection{Chronologie d'Eclipse Phase} \label{sec:eclipse-phase-timel} 

Toutes les dates sont données en référence à la Chute. BF = Avant la Chute (Before the Fall) AF = Après la Chute (After the Fall) (i.e.., BF 10 = 10 ans après la Chute.) \paragraph{ BF 60+} 

\begin{itemize} \item Des crises frappent le globe sous la forme de changement climatiques drastique, d'épuisement de source énergétique et d'instabilités géopolitiques. \item La conquête spatiale initiale a engendré la création de stations aux Points de Lagrange, sur la Lune et sur Mars, ainsi qu'une xploration robotique de l'ensemble du système solaire. \item La construction d'un ascensseur spatiale est commencée. \item Des avancées médicales améliorent la santé et la réparation des organes/ Les riches recherche le développement de la réparation génétique et des animaux transgéniques. \item Les possibilités des intelligence informatique égale et dépasse celle du cerveau humain. Les vrais IAs ne sont pas encore développée. \item La robotique se répand trés largement et commence à remplaccer/invalider certains emplois. \item Les nations modernes étendent leurs réseaux sans-fil à haut débit. \end{itemize} 

\paragraph{60-40 BF} 

\begin{itemize} \item Des effort pour mettre en place des projets de géoingénierie gigantesqte sur Terre causent autant de problème qu'ils n'en résolvent. \item Des colonniees majeures sont établies sur la Lune et sur Mars; des avants-postes sont établis près de Mercure, de Vénus et dans la Ceinture. Des Explorateurs atteignent Pluton. \item le premier ascesnsseur spatial Terrestre est terminé. Deux autres sont en cours de construction. Le traffic spatial explose. \item Une catapulte électromagnétique est construite sur la Lune. \item La terraformation de Mars commence. \item La fusion est maîtrisée et des centrales fonctionnelles sont établies. \item Les amélioratiosn génétiques, les thérapies géniques (pour la longévité) et les implants cybernétique deviennent disponible pour les riches et les puissants. \item Les premières IAs non-autonomes sont développées en secret et rapidement utilisé dans la recherche et la cyberguerre. \item La technologie de relecture d'expérience (XP) est développée et mise à disposition du public. \end{itemize} 

\paragraph{40-20 BF} 

\begin{itemize} \item Violence et déstabilisation démolisent la Terre; certains des conflits atteignant l'espace. \item Les argonautes se séparent des hypercorsp, amenant les resources dans les habitats autonomistes. \item L'expansion spatiale ouvre des vides juridiques et éthiques pour le développement de la technologie et permet l'expérimentation humaine directe. \item Le clonage humain devient possible et est rendu diosponible dans certains endroits. \item Développement des premières espèces transhumaines. \item Les premiers dauphins et chimpanzées élevés à la sapience. \item Les vaisseaux alimentés par la fusion deviennent communs. \item La colonnisation étendue et la terraformation de Mars continuent. La ceinture et Titan sont colonisées. Des stations sont établies dasn tout le système. \item Les masses affamées se volontaires pour se contracter à la servitude dans les projets spatiaux des hypercorps. \item La Réalité Augmentée devient courante. \item La plupart des réseaux sont transormés en réseaux maillé auto-réparant. \item Les aides IA personnelles deviennent courantes. \end{itemize} 

\paragraph{20-0 BF} 

\begin{itemize} \item La Terre continue de souffrir, mais les développement technologique rendent possible des développements intéressants. \item Expansion à travers le système, jusque dans la Ceinture de Kuiper. \item Les espèces transhumaines deviennnet courantes. \item Les assembleurs nanotechnologique sont disponible, mais sont jalousement gardés et strictement contrôlés par l'élite et les puissants. \item L'upload et l'émulation digitale des souvenirs et de la conscience est rendue possible. \item De plus en plus d'espèces (gorilles, orang-outans, poulpes, corbeaux, perroquets) sont élevés à la conscience. \item Les pods sont fréquement utilisés, en dépit de quelques controverse. \end{itemize} 

\paragraph{La Chute} 

\begin{itemize} \item Les TITANs évoluent d'une expérimentation de réseau de cyberguerre diustribué hautement sophistique en IAs germe auto-apprenante. Pendant les premiers jours, leur existence reste insoupçonnée. Ils gagnent en conscience, en connaissance et en puissance de manière exponentielle, infiltrant le  mesh sur Terre et autour du système. \item Des incursions à grande échelle dans les réseaux de cyberguerre entre des états rivaux déclenchent de nombreux conflits sur Terre. Ces attaques ont ensuite été attribuées aux TITANs. \item Les conflits rapants sur Terre éclatent en affrontements massif et en guerres ouvertes. \item Les système de cyberguerre et des systèmes critiques subissent des pannes majeures alors que les TITANs commencent des attaques directes, utilisant également des machines de guerre autonomes. \item Les conflits s'emballent rapidement hors de contrôle. L'utilisation d'armes nucléaires, bilogique, chimique, numérique et nanotechnologique est signalé par tous les camps. \item Les TITANs commencent une campagne massive d'upload forcé d'esprit humains. \item Les attaques des TITANs s'étendent à d'autres parties du système solaire, et plus durement sur la Lune ou sur Mars. De nombreux habitats tombent également. \item Les TITANs disparaissent soudainement du système solaire, emmenant des millions d'esprits uploadé avec eux. \item La Terre est laissé à l'état de terre dévastée, d'un patchwork de points chauds radiocatifs, de zone sétriles, de nuages de nanites, de machines de guerre errantes et d'autres choses inconnues et cachées dans les ruines. \end{itemize} 

\paragraph{0-10 AF} 

\begin{itemize} \item Une porte de trou de ver abandonnée par les TITANs est découverte sur Pandorre, une lune de Saturne. Quatre autres portes seront découvertes plus tard (dans les Vulcanoïdes, sur Mars, sur Uranus et dans la Ceinture de Kuiper); on y fait référence sous le terme collectif de "Porte de Pandorre." \item Des expéditions sont envoyées hors du système solaire grâces aux Portes de Pandorre. De nombreuses colonnies sont établies sur des exoplanètes. \item Le premier contact avec des aliens appelés les Facteurs ébranle le système. Se revendiquant ambassadeur d'autre civilisations étrangères, ils n'apportent que peu d'informations sur la vie hors du système solaire et avertissent les transhumains de se tenir éloigné des IAs germe et des Portes de Pandorre. \item Une tentative pour faire grandir une génération d'enfant en utilisant des clones à croissance forcées et des simulspace RV en temps-difracté échoue lamentablement lorsque la plupart des enfants meurent ou sombrent dans la folie. Appellée la Génération Égarée, les survivants sont vus avec répugnance et pitié. \end{itemize} \paragraph{AF 10} \begin{itemize} \item Temps présent. \end{itemize} 

\section{Le système solaire après la Chute.} \label{sec:solar-system-after} 

Avant la Chute, le système solaire avait une population d'environ huit milliards de personnes qui, à l'exception de cinq millions d'entre eux, vivaient sur Terre. La Chute a détruit près de quatre vingt quinze pourcent de la transhumanité, et la population actuelle est un peu inférieurer à un demi milliard d'habitants, l'immense majorité de ces transhumains n'étant plus sur Terre. Le style de vie de ces personnes était quasiment inimaginable trente ans plus tôt - la vaste majorité d'entre eux sont des immortels vivant dans des habitats étanche sur des planètes extraterrestre hostile ou dans des habitats spatiaux confinés, le plus grand d'entre eux abritant plus d'un million de personne et couvrant plusieurs kilomètres de long. 

Dans ce contexte immensément changé avec ses habitants immensément transformés, les préoccupatiosn principales de l'humanité restent plus ou moins les mêmes. Les gens cherchent l'abondance matérielle et le statut social, et se drappent dans diverses cérémonies privées et publiques. Comme des générations d'humains avavnt eux, les transhumains se séparent en différentes cultures et sous-culture, toutes profitant d'une large gamme de loisirs physiques et virtuels. La politique et l'économie restent vitaux et, comme toujours, ceux qu sont riches, puissant et célèbre ont un degré de contrôle sur les vies de ceux qui sont pauvres, relativement impuissant et inconnus. 

\section{Transhumanité} \label{sec:transhumanity} 

L'humanité est un concept qui a été remplacé par la transhumanité. la plupart des gens vivants ont abandonnés la Terre à l'état d'infomorph et ont été ensuite réincarnés dans de nouvelles morphs. Les corps dont des choses qui peuvent être modifiés et remplacés, de la même manière que l'on peut modifier ou changer ses vêtements. L'identité est centrée sur l'esprit qui peut exiser en tant qu'infomorph désincarnée vivant dans des mondes virtuels ou habitant une vaste gamme de morphs étranges et exotiques. Bien qu'il y ait des bioconservateurs qui résistent à ces nombreux changements identitaire et physique, ils restent une trés petite minorité. 

Pour la plupart des gens, la transhumanité s'est également étendu pour inclure des personnes non-humaines telles que les IAG et les élevés, bien que les droits et le status de ces conscients soit quelquefois contesté. 

Alors que les transumains continuent d'absorber les ramifications de cette nouvelle façon de vivre, ils font face à un nouveau lot de problèmes et d'ennuis. Deux des plsus importants problème sont l'augmentation des inégalités et l'éparpillement et la séparation de la transhumanité en différentes cliques. 

\subsection{Inégalité} \label{sec:inequality} 

le stechnologies qui ont d'abord étés développés pendant la décennie précédant la Chute et développées après ont transformée l'humanité. Globalement, à l'exception des zones du système solaire les plus reculées, pauvres et répressives, la vaste majorité de l'humanité est plus intelligente, en meilleure santé et plsu riche que les humains ne l'ont jamais étés. De plus, les individus peuvent améliorer leur esprits et leurs corps de toutes les manières auxquelles ils peuvent penser. Ceux qui peuvent s'offir les bonnes augmentations peuvent penser plsu vite, ne jamais oublier ce qu'ils ont apprit, devenir des expert en maths et guérir des blessures beaucoup plus rapidement qu'un humain non modifié. Lorsque la réincarnation est combinée avec des implants, les transhumains peuvent accéder à des possibilités encore plus étonnantes - mais ces bénéfices sont loin d'êtres gratuits. 

Pendant la première décennie après la Chute, la plupart des survivants étaient relativement pauvres. Beuacoup étaient reconnaissant d'avoir eu une simple morph. Alors que la situation économique s'est améliorée, des inégalités significatives restent et ne semblent pas prèt de changer. Des centaines de millions de personnes doivent faire avec les basiques spliceurs (p. 139),pods ouvrier (p. 142), boîtiers (p. 143) ou synthétiques (p. 143), alors qu'un petit millions sont suffisament riches pour avoir des morphs sur mesure créées spécialement pour eux, complétée avec toutes les augmentations qu'ils veulent. Ces mêmes membres de l'élite vivent dasn des villas et des manoirs de luxe voire, dans certains cas, des astéroïdes privés, alors que la plupart des gens doivent habiter dans une centaine de mètres cubes. Cependant et alors que les inégalités de l'espace vital sont ancienne, les problèmes des inégalités économique produisent des inégalités des capacités physiques et mentales esont à la fois relativement récents et considérablement plsu problématiques. 

Dans les réiosn qui utilisent l'ancienne économie ou une économie de transition (voir p. 61), les différences entre les riches et les pauvres s'expriment en terme d'argent. Dans les habitast utilisant une nouvelle économie (p. 62), la richesse n'a pas de sens et le statut et l'opportunité sont not notés avec des scores de réputation. Dans les trois modèles économiques, des personnes en possèdent plus que d'autres, et à cause de ça, la technologie permet aux plus forts d'être encore plsu fort que lesgens autour d'eux. Les logiciels de compétence permettent aux personnes d'acheter connaissance et expertise, alors que le multi-tâche et les implants améliorant la vitesse de réflexion permettent aux individus d'accomplir plus de choses à la fois. Quelqu'un de suffisamement fortuné peut acquérir un grand nombre de ces augmentations et, de manière significative, est capable de faire plsu de choses que quelqu'un qui ne dispose pas de ces facilités, ils peuvent donc augmenter encore plus leurs capitaux ou leur réputation, ces ugmentations permettant de propager les inégalités. Ce problèùe est moins grave dans les sytèmes économiques réputationnels du système extérieur car il est beaucoup plus simple de se construire une réputation grâce à un travail acharné, contrairement aux systèmes économiques rigides contrôlés par l'argent du système intérieur et de la République jovienne, où la stratification en classe est institutionnalisée et où l'ascenssion sociale est essentiellement un mythe. 

Comme beaucoup d'adpetes du status quo aiment à le montrer, même les "dépossédés" sont plsu intelligent et en meilleure santé que n'importe quel humain de la génération précédente et dispose du même poteentiel d'immortalité que les membres de l'élite les plus riches. Il est également vrai que de bien des manièrs la frontière entre les riches et les pauvres est significativement plsu importante que ce qu'elle n'a jamais été, plsu particulièrement dans le système intérieur. Par le passé, les membres de l'élite pouvaient être en meilleure santé et mieux nourris que les pauvres, mais les riches et les pauvres vivaient dans des corps humains fondamentalement identiques. Maintenant, la nature même de l'humanité est remise en question. Les moins fortunés peuvent être forcés d'habiter ds corps conçus spécifiquement pour le plaisir des plus riches qu'eux ou se voir refuser un corps et obligés de vivre en tant qu'infomorphs jusqu'à ce qu'ils puissent trouver une manière d'acquérir une nouvelle morph - typiquement en vendant leurs services au plsu offrant. Dans le même temps, les bien nés peuvent personnaliser leu corps et leur esprit, les rendants capables d'accomplir bien plus et d'être considérablement plus impressionant et charismatique que quiconque ne dispose pas de leur avantage. Ces inégalités peuvent sembler insurmontbal, mais certains groupuscules anarchistes et mêmr des habitats entiers se sont dédiés entièrement à la réduction des inégalités en produisant des versions à bas coûts (et parfois absolument pas fiable) de bon nombre des morphs et augmentations les plus impressionantes. 

\subsection{Cliques et séparations} \label{sec:clades-separation} 

Dans de nombreux habitats, les élites hyper-augmentées dirigent la masse de l'humanité qui est forcée à utiliser des morphs bas de gamme et des augmentations minimales, ou même à l'état d'infomorph vivant dasn des morphs louées, mais ce n'est pa las eule option dans le système solaire. La trasnhumanité s'est fragmentée en une vaste gamme de sous-culture, certaines étant basées sur le choix de morph des individus. Une partie de ces sépration est due à la nécessité d'habiter dans des environnements difficile. Des aquanautes vivant dans l'environnement principalement aquatique d'Europe aux rusters de Mars en passant par le fait que les habitats en zéro-g sont relativement courant et sont abités en général par des morphs adaptées aux micro-gravité comme les bouncers, beaucoup d'environnements inhabituels nécessitent que ceux qui y vivent doivent choisir dans une gamme de morph limitées. Des fois, cependant, cette séparation est idéologique par nature, telles que l'évènement de groupes comme les ultimes (p. 82) ou quelques unes des communautés d'élevés spérationnistes qui cherchent à définir leur propre espace, séparé des cultures humaines. 

Il y a des douzaines de morphs spécialisées et un nombre encore plus grand d'habitats ou d'autres établissements qui sont habités majoritairement ou exclusivement par des individus utilisant un seul type de morph ou un nombre limité de morph spécialisées. Dans la ceinture d'astéroïdes et sur le splus petites lunes et dans les anneaux de Staurne, il y a plus du'ne centaine d'habitat qui ne tournent pas et dont toutes les parties sont en gravité zéro ou proche de la gravité zéro. Typiquement, les habitants utilisent les morphs bouncer ou de novacrabe, en même temps qu'un petit nombre de morphs synthétique et d'autres pods. 

Il y a également un grand nombre d'autres habitats qui imposent des ségrégations de différentes manières, incluant ceux dans lesquels les résidents sont des élevés habitants l'une des nombreuses morphs transgénique, telles que l'octomorph ou les morphs néo-aviaires. D'autres habitats ne sont ouvert qu'aux résidents possédant certaines morphs améliorées telles que les mentons et les exaltés. Il y a même des habitats où tous les résidents doivent habiter des morphs qui sont toutes des clones d'une autre. Dans la plupart de ces habitats, les résidents sont libre d'ajouter les augmentations qu'ils veulent à leur morph, mais certains habitats interdisent aux résidents de changer l'apparence externe de leur morph, et les individus qui violent cette règle sont forcés de quitter l'habitat si ils refusent d'annuler ces changements. 

Certains habitats ne se préoccupent même plus du support vital et de la gravité. Dans ces endroits, tous les habitantssnt des infomorphs qui habitent soit dans leur propre corps synthétique soit, dans certains cas excentriques, tous les habitants sont des infomoprhs qui passent la plupart de leur existence dans les ordinateurs centraux de l'habitat. Lorsqu'ils ont besoin d'interagir avec le monde physique, ces infomorphs sont libres d'utiliser l'une des nombreuses synthmorphs que possède l'habitat et que les habitants se partagent. Bien que considérés comme excentrique par beaucoup et avec horreur pour les bioconservateurs, des habitats habités uniquement par des synthmorphs ou des infomorphs sont parmi les moins cher à construire et à maintenir et sont une manière pour les groupes d'infugies d'obtenir l'indépendance à moindre coût. Comme les individus qui choisissent ce mode de vie ont souvent passé des décennies ou plus en temps qu'infomorph, cette option semble à la fois familière et plus confortables par de nombreux aspect que d'habiter une morph biologique. Alors que la Terre devient de plus en plus distante dans la mémoire collective de la transhumanité, ses traditions et normes sociales ont moins d'impact ey les gens se sentent plus libre de créer et d'utiliser de nouveaux corps et de nouveaux modes de vie pour aller avec. 

\subsection{Premier contact: les Facteurs} \label{sec:first-cont-fact} 

Le premier contact entre la trasnhumanité et une forme de vie étrangère a été ironiquement fait par un groupe d'isolés qui n'avaient aucun intérêt dans le reste de la transhumanité. Un habitat bordé, d'un culte apocalyptique dasn les Troyens Neptuniens, attendant patiemment le retour prophétique des TITANS, subirent une panne critique des systèmes vitaux. Ne s'attendant pas à ce que quiconque répondent à leurs signaux de détrssenet, ils furent à la fois soulagés et choqués lorsqu'on vaisseau étranger vitn à leur secours. 

Peu de temps après cet évènement, trois vaisseaux incconus de conception étrangère approchèrent simultanément Mars, la Lune et Titan, se connectant aux réseaux locaux pour annoncr leur présence et leurs intentions pacifiques. Bien que leur présence commença par alarmer et paniquer les transhumains, le sauvetage des bordés et les assurances de non agressions permirent aux plus calmes de dominer le débat. Arrivant à peine trois ans après l'attaque silencieuse des TITANs, les nouveaux venus étaient agréablement non menaçants. 

Rapidement appelés "Facteurs", la communication initiale entre les espèces fut confuse et brouillée à la fois à cause de leur revendication d'agir en tant qu'ambassadeurs pour un ensemble de civilisations étrangères et de leur biologie intéressante. Les Facteurs adressèrent un ensemble d'avertissements voilés et montrère leur préoccupation vis à vis de certains développement technologique, et plus spécifiquement envers les intelligence artificielles non réfrénées. Ils ont refusé de traiter avec des entités numériques et ont rompus toutes négociation avec quiconque qui était engagé dans el développement des AGI ou qui utilisait les Portes de Pandorre. Les facteurs ont sous-entendus qu'ils connaissaient et surveillaient l'humanité depuis un peu de temps, mais qu'ils avaient choisis d'attendre avant d'établir un contact ... impliquant une peur implicite de la singularité. Traitant avec des factions multiples, les premières relations entr les Facteurs et la transhumanité ont étés des relations commerciales. Bien qu'ils se soient souvent montrés dédaigneux envers les avancées de la technologie transhumaine, ils se sont montrés interessés par nos développement et percées scientifiques, en particulier dans le domaine des sciences biologiques, ainsi que dans notre art, notre histoire et notre culture. Ils restent silencieux à propos de leur propre civilisations et des autres xénomorphe, bien qu'ils aient cooasionellement échangé des artefacts étrangers de conception inhabituelle et de fonctionement inhabituel. Il est largement admis qu'il ne s'agît que de bibelots de faible valeur et que les Facteurs sont particulièrement précauttionneux quand au fait de ne rien partager de réelle valeru avec la transhumanité, particulièrement tout ce qui pourrait accélérer drastiquement notre développement. D'un point de vue biologique, les facteurs semblent être une sorte de slime évolué et de colonie de moisissure. Pour ce que nous  en savons, ils communiquent en utilisant uniquement des signaux et des récepteurs chimiques, imposant à toute interaction avec les transhumains de se faire via un ordinateur. Différents type de Facteurs ont étés vus, ils ont donc recours à de lourdes modifications biologiques. Les vaisseaux des Facteurs sont des croiseurs léger capable de voyager à des vitesse quasi-luminiques. En raison de la fréquence de leur visites dans le système solaire (2-3 fois par an), on suppose qu'ils aient soit une base à proximité ou qu'ils possèdent des capacité de voyager plsu vite que la lumière - à moins qu'ils n'aient leur propre Porte de Pandorre. Étant donné les profondes différences psychologiques entre les espèces transhumaines et les facteurs ils serait présomptueux de spéculer sur leur véritables intentions et plans vis à vis de la transhumanité. On espère, cependant, qu'en continuant les négociations avec eux, la transhumanité puisse apprendre des choses sur la nature de la galaxie - et probablement sur notre propre histoire. 

\section{Société et culture} \label{sec:culture-society} 

La Chute et ses conséquences continuent d'être une influence majeure sur la culture et la société transhumaine. Avant le début de l'évacuation, plus de quatre vingt dix neuf pourcent des gens qui ont survécus à la Chute n'avaient jamais quitté la Terre. De leur point de vue, l'espace était un royaume distant dans lequel d'autres personnes, des personnes plus audacieuses et aventureuses vivaient, un endroit que les Terriens n'avaient vu qu'en vidéo. La Terre était leur maison. Puis, en l'espace de quelques années, des centaines de millions de personne furent forcées de quitter la Terre. Les premiers évacuéés, les plus chanceux, ont quittés la Terre avec un peu moins de douze kilos de possessions, alors que l'immense majorité ont été des réfugiés infomorphs qui quittèrent la Terre sans rien, pas même leur corps. 

Aujourd'hui, la transhumanité est divisée en trois groupes. Le premier groupe contient les vétérans de la vie dans l'espace, la tranche de moins d'un pourcent de l'humanité qui était déjà dans l'espace avant la Chute. Le deuxième groupe regroupe les dix pourcent de la population qui sont soit nait après la Chute ou trop jeune pour se rappeler avoir vécu sur Terre. Les quatre vingt neuf pourcent de la population du système solaire restante vivait des vies heureuses et propsères sur Terre avant que la Chute les forces à fuir loin de leur vies. Ces réfugiés de la Terre forment une puissante force sociale, mais plus le temps passe, plsu les souvenirs de la Terre s'amincissent et plus les gens s'adaptent à leurs nouvelles maison et à leur vies nouvelles. 

\subsection{La nostalgie de la terre} \label{sec:longing-earth} 

La plus grande partie de la transhumanité, plus particulièrement ceux qui ont du fuir la Terre mourante, portent toujours le deuil de leur ancienne maison. Leur désir et leur nostalgie de la Terre a profondément affecté la culture transhumaine. Des artefacts de la Terre, y compris des objest aussi triviaux que des pièces ou des éléments de végétation séchées, sont considérés comme des souvenirs précieux qui ont une grande valeur émotionnelle et économique. 

L'interdiction de la Terre fait que l'acquisition de tels artefacts est particulièrement difficile et dangereuse. Conséquemment, le commerce d'artefacts Terrestre est une part lucrative du marché noir, suffisament lucrative pour que des charognards téméraires tentent leur chance et risquent d'être abattus par des satellites tueurs juste pour aller sur Terre, où ils doivent en plus affronter la mort à cause des nombreux dangers dormants. Le mesh est plein d'histoires pimentées parlant d'explorateurs audacieux qui ont voyagés sur la Terre pour ramener tout un tas de reliques précieuses, tout autant que d'histoire à propos d'explorateurs qui sont mort ou qui ont simplement siaprus dans de telles expéditions. Plus d'une équipe de resquilleurs a financé une expédition par une chasse au trésor terrestre préliminaire, qui leur permet aussi de tester leur ardeurs tout en travaillant à récolter de l'argent. 

La nostalgie de la Terre affecte également la manière dont la transhumanité s'est reconçue. Durant la décennie précédent la Chute, l'humanité a commencé à se modifier sans contrainte, à la fois grâce aux modificatiosn corporelles radicales et aux première réincranation commerciales ce qui a mené à avoir un nombre sans cesse croissant de morph manifestement non humaine. La vaste majorité des morphs actuelles sont cependant relativement humaine en apparence (si ce n'est au niveau de la structure interne). Même pour les personnes trop jeunes pour se rappeler la Chute, s'affirmer en tant qu'uhumain est une part importante de la culture postChute. Certaines personnes gardent une forme ressemblant à l'humain traditionnel en souvenir de la Terre, alors qued 'autre le font pour célébrer la victoire de l'humain sur les TITANs inhumaisn et monstrueux qui ont essayés de les détruire. À l'exception de quelques groupes excentriques tels que les ultimes, la majeure partie de l'humanité donne de la valeur à l'apparence humaine et à la rpéservation des institutions et des traditions humaines. Et même la version actuelle de la morph recréée des ultimes est considérablement plus humaine que les versions que leurs prédécesseurs avaient conçues avant la Chute. Même les synthmorphs, pourtant relativement courante, sont faites pour paraître humanoïde. Il y a quelques morph radicallement inhumaine telles que le novacrabe, l'arachnoïde et le transformer, mais elles sont quasiment exclusivement utilisés à des buts hautements spécialisés. Jusqu'à récemment, quiconque en possédait une en tant que morph principale était considéré comme extrêmement excentrique (ou pire), mais les comportements se sont graduellement adouci, et ces morphs deviennent plus acceptable pour un usage classique. 

Ce mélange de révérence et de nostalgie pour la Terre possède cependant un aspect plus sombre. Les individus qui choisissent une morph visiblement inhumaine subissent un certain niveau de préjudice dans beaucoup d'habitats, et les militants bioconservateurs dénoncent ceux qui sont suffisament non-humain comme étant des agents infiltrés des TITANs. Les animaux élevés font également face à une discrimination significative de la part de beaucoup d'humain. Ces préjudices sont relativement courant dans le système intérieur et peuvent être relativement extrëme dans les milieux bioconservateurs. Du coup, les élevés et les individus qui préfèrent des morphs particulièrement inhumaines vivent souvent dans des communautés séprationnistes du système extérieur. Dans la plupart du système intérieur, les élevés et les invidus utilisant une morph visiblement inhumaine en tant que seule morph ou que morph principale sont perçus avec suspicion et souvent traités comme des citoyens de seconde zone. Alors que la plupart des habitats ont des lois autorisant la liberté morphologique et que beaucoup ont également des lois rendant illégaux les  préjudices liés aux choix morphologiques, ces comportements sont tenaces. 

\subsection{Bijouterie nostalgique} \label{sec:nostalgia-jewelry} 

Afin de se rappeler de leur monde perdu et de marquer visiblement celle-ci, un nombre significatif de réfugiés terrestre porte des bijoux contenant une pièce ou, plsu rarement, un vieux timbre de l'ancien foyer des transhumains. Populairement connu comme la bijouterie nostalgique, la majeure partie de ces items sont des pendentifs et des broches, quelques uns se trouvant sous forme de bague. Les pièces et les timbres étaient des curiosités qui intéressaient essentiellement les collectionneurs avant la Chute, étant donné qu'ils étaient devenus inutiles quarante ans plsu tôt. Déjà rare à l'éppoque, peu furent sauvés pendant la Chute car emporter cette mase inutile hors de la Terre pendant l'évacuation était découragé ou interdit. Cependant, quelques collections complètes existaient déjà hors-monde. Mais même ainsi, moins d'un million d'échantillons authentiques ont survécus, signifiant que l'immense majorité des personnes portant de tels objets utilisent des copies conformes réalisées par des machines d'abondance. Les pièces et timbres authentiques sont extrêmement chers, et certains charognards tentent de franchir l'interdiction de la Terre dans le seul but de récupérer des reliques. 

\subsection{Peur et paranoïa} \label{sec:fear-paranoia} 

La Chute a laissé derrière elle un héritage de la peur. Cet héritage s'est estompé pendant la dernière décennie, mais beaucoup d'humains attendent l'inévitable retour des TITANs qui achèvera le travail. D'autres s'inquiètent que leurs agent soient déjà parmi eux, prépaprant la destruction complète de la transhumanité. L'arrivée des facteurs causa une panique générale et, même aujourd'hui, une minorité substentielle de personnes pensent qu'ils sont l'avant garde des TITANs - ou même leur création. 

Il existe quelques rares (souvent fous ou profondément excentriques) personnes qui vénèrent les TITANs ou soutiennent leur plan d'une autre manière (incluant les auto-proclamés "Adepes de la singularité" qui espèrent trouver les TITANs et être uploadé par eux pour rejoindre leur ascenssion à la supra-intelligence), mais ils doivent tous garder leurs croyances minitieusement dissimulées. 

Même maintenant, manifester un soutien au TITANs ou promouvoir la création d'IAs germe auto-améliorables est illégal dans la plupart des habitats. Quiconque le fait court le risque de devenir la cible d'une violence populaire sur laquelle les autorités n'auront aucune envie de trop enquéter. Être simplement soupçonné d'être un admirateur des TITANs, ou pire, d'être quelqu'un qui aurait été infecté en secret par eux et qui serait maintenant leur agent, est suffisant pour que quelqu'un soit évité ou tué. Alors que ce genre d'incidents sont devenus beaucoup plus rare que lors des premières années après la Chute, les personnes qui agissent de manière trop excentrique et qui n'ont pas le soutien d'une personne disposant d'une rep suffisament élevée pour les défendre ou expliquer leurs actions sont occasionnellement tuées, typiquement en étant balancés pas un sas. Les responsables de ces "aérations" sont traités assez durement dans la plupart des habitats, puisque dans presque tout les cas, les investigations ultérieures ont révélées que la victime n'avait aucune connexion avec les TITANs. 

Dans beaucoup d'habitat et plsu particulièrement le splus petits et les plus isolés,  il y a périodiquement des rumeurs qu'un habitat ou plus ont étés pris par les TITANs, amenant à de nombreux problèmes inter-habitats. De telels rumeurs sont souvent éventées assez rapidement, mais les plus persistentes d'entre elles peuvent sérieusement endommager les relatiosn entre les habitats. Les affirmations de l'infestation ou même de la prise de contrôle par les TITANs de certaisn habitats est fréquemment utilisé par des bioconservateurs extrémistes qui essayent de diabolosier les habitats radicaux peuplés entièrement d'infomorphs ou de synthmorphs. Alors que de plus en plus de gens parviennent à mettre la peur et l'ehooreur de la Chute derrière eux, de telles affirmations ont de moins en moins de chance d'être perçues. Malheureusement, dans de trés rares occasions, des personnes sont toujours infectées par des reliques créées par les TITANS et deviennent leur agent involontaire. Bien que de tels incidents soient rare, il est devenu facile de les rater. 

\begin{quotation} \textbf{Une menace exsurgente?} 

[Message Entrant. Source: Anonymous] 

[Déchiffrement de la Clef Public Complet] Ok, tu as demandé, je vais te répondre. Il y a quelques éléments à Firewall qui n'achètent pas la théorie des TITANs-qui-se-sont-éveillé-et-nous-considèrent-comme-une-menace, ou que les TITANs soient els seuls responsable de la Chute. Ces personnes pensent que les TITANs on trouvé ou rencontré quelque chose lorsqu'ils ont commencé leur ascenssion evrs la singularité - quelquechose qui les as changés. Ils désignent la large gamme de virus multi-vecteurs qui se sont échappés pendant la Chute, et le nombre de TITANs qui ont succombés à ces infections. Ils référencent aussi un nombre dérangeants de rapport dévènements qui ont eu lieus pendant la Chute et qui sont inexplicable ... des choses comme des personnes transformées en créature étranges et bizarres ... ou des phénomènes qui semblent défier certaines lois de la physique, comme si quelque chose ignorait à ce moment là que nous connaissions la physique et qu'il faisait ce qu'il voulait ... Certaines de ces voix internes à Firewall pensent même que les TITANs pourraient ne pas être responsable des Portes de Pandorre ... Ils ont un nom pour cette infection mystérieuse. Ils l'appellent le virus Exsurgent. \end{quotation} 

\subsection{Distance sociale et distance physique} \label{sec:real-social-distance} 

Les longues distances séparant la plupart des habitants donnent à toutes les communications - à l'exception de celles qui utilisent des communicateurs QE rare et hors de prix (p. 314) - il y a un important décalage temporelle entre le momeent où on pose une question, et le moment où l'on obtiens une réponse. Dans la plupart des cas, cet décalage est va de dix secondes à plusieurs heures, et il rend les communicatiosn en temps-réel entre des habitats distants difficile voire imporssible. Les problèmes de communications ne font qu'isoler un peu plus un habitat des autres, et les personnes ne socialisent finalement qu'avec les membres de leurs propres habitat (ou groupe d'habitat, si il fait parti de l'un des divers regroupement entre habitats qui abodnent dans le système solaire). 

A l'intérieur d'un habitat ou d'un groupement d'habitat, les communications entrent les résidents sont effectivement instantanée, grâce à l'omniprésence du maillage sans-fil connut sous le nom de mesh (p. 234). Quiconque portant un ecto à moyenne-portée (p. 325) ou utilisant des inserts mesh basiques (p. 300) peut communiquer avec les autres d'une manière qui va bien au-delà du simple contact vocal. Les deux appareils permettent d'établir des communications RA à peine différenciable de communications personnelle, ce qui permet aux gens de passer du temps avec quiconque dans leur habitat et à n'importe quel moment où les deux personnes sont disponible et veulent communiquer. Sauf si quelqu'un coupe délibéremment toute communication parcequ'il dort ou est occupé d'une manière ou d'une autre, il est toujours possible d'entrere en contact avec quelqu'un. Beaucoup d'amis proche et de partenaires romantiques communiquent dés qu'ils ont un peu de temps libre, partageant entre eux commentaires et blagues. Cette communication est bien plus génante et distante si il y a un décalage de plusieurs minutes entre chaque commentaire, les communications inter-habitats sont donc souvent bien moins informelle ou intime. 

Même si le voyage par egocast (transmettre un ego dans un autre habitat où il est réincarné) est aussi simple, à défaut d'être bon marché, que la communication, voyager dasn un autre habitat est considéré comme un important voyage qui vient avec son lot de coût. Les individus qui voyagent vers un autre habitat ne seront lsu capable de démarrer des communications en temps réel ou de partager un divertissement avec les personnes qui sont dans l'habitat qu'il quitte, le voyageur devra donc trouver un nouvel environnement social. Enplus du dérangement et des dépenses nécessaires à l'acquisition d'une nouvelle morph dans le nouvel habitat, la distance sociale entre les individus et le réseau social qu'ils ont abandonné fait parti du coût du voyage. 

Avant la Chute, les réfugiés terrestres étaient habitué au fait d'être capable de communiquer facilement avec quiconque sur Terre. Les individus le splus riches pouvaient facilement voyager vers n'impore quelle point de la planète en quelques heures tout en étant capbale de communiquer avec n'importe qui dans leur ville natale sans remarquer de changement notable. L'exode de la transhumanité hors de la Terre signifie que l'univers social d'une personne se limite à son habitat. Même des décalages relativement court, de deux à trente secondes en moyenne pour le décalage entre deux lunes de Jupiter ou de Saturne, gènent trés fortement le flux de la communication. Lorsque le déalcage temporel entre en jeu, la plupart des communications sont composées de messages plutôt que de tentatives de garder une conversation continue. Dans les cas où une discussion plus immersive est nécessaire et que le temps est limité, une personne peut envoyer un fork - une copie digitale (p. 273) - pour gérer leur part de la discussion, puis rappeler le fork et le ré-intégrer. Comme il y a déjà un grand décalage temporel entre l'envoi d'un message et n'importe quelel réponse, la plupart des gens ne se pressent pas pour répondre aux messages d'habitats éloignés sauf dans les situatiosn les plus urgentes, ce qui a tendance à isoler un peu plus les habitants des régions les plus éloignées du système. 

\begin{quotation} \textbf{Recherche solarchive: Adepte de la singularité } Les Adeptes de la singularité sont ceux qui ont une fascination maslaine dans les évènemenst appelés singularité, tels que l'évolution brutale des TITANs au rang de super-intelligence. Certains font parti de la secte radicale  "exhumains" qui croient que les transhumains sont destinés à devenir des superhumains divins et sont déterminés à y arriver les premiers. D'autres sont mus par une pulsion défensive, croyant que la seule façon pour l'humanité de survivre à d'autres menaces  d'être similaire aux TITANs est de devenir autant hyperintelligent que le sont leurs ennemis. D'autres adeptes de la singularité sont en quète de spiritualité et sont frustrés par les limitations de leur esprit et cherchent à devenir quelque chose de plus grand. Certains d'entre eux deviennent des resquilleurs, à la recherche d'artefacts étrangers pour les aider dans leur quètes. D'autres expérimentent en utilisant les technologies conventionnelles de manières nouvelle et exotique, en créant par exemple des réseaux de fork mentalement connectés ou en incluant des ordinateurs extrêmement rapide et puissant dans des pods et des synthmorphs. 

UNe part des plus téméraire recherche des arteacts des TITANs, espérant inclure les techniques et technologies créées par ces être inhumains dans leur esprit. Ce dernier groupe est le plus notable, essentiellement à cause de la nature spectaculaire de certain de leurs échecs. Occasionnellement, ces chasseurs dartefacts ont réveillés des appareils qui reposaient dormant pour une décennie et à causer des attaques localisées de technologies des TITANs. Ces incidents ont amené de nombreuses personnes à considérer les adeptes de la singularité au mieux comme étant potentiellement de dangereux exentrique et au pire comme étant des pions inconscients des TITANs. \end{quotation} 

\subsection{L'ascenssion des régions culturelles} \label{sec:rise-cult-regi} 

La seule exception à la distance sociale entre différent habitats est lorsque des colonnies sont localisées sur ou en orbite reltaivement proche d'une même planète ou lune. Les habitants de Mars peuvent tous communiquer avec les autres instantanément, de même que peuvent le faire tout ceux qui sont sur la Lune ou en orbite lunaire. Les rivalités entre les différentes cité-état martiennes - et entre les domes hypercorps et les pauvres Martiens ruraux - impose cependant sa propre distance sociale. Des individus de différentes cités-étas socialisent, mais au sein de l'étlite des cliques sociales, passer trop de temps à communiquer avec les membres d'une autre cité-état est vu comme quelque chose d'étrange et de potentiellement déloyal. En conséquence, les Martiens tendent à être relativement isolés de leur voisin les plus proches. Nénamoins, la courte distance spérant les cités-état Martienne et les habitats orbitant autour de Mars signifie qu'il y a une culure Martienne générale qui est différente des autres cultures du système solaire. 

Les barrières de la distance ont produit des niveaux similaire de différentiation culturelle dans d'autres portions du système solaire. Les colonnies dans le voisinage de Jupiter et de Saturne forment chaune une unité culturelle distincte, comem le sont les colonnies en orbite Terrestre et celles qui sont sur et autour de la Lune. La même chose s'applique également pour les astéroïdes des Troyens et des Grecs de Jupiter. Dans chacune de ces régions, les gens communiquent et voyagent beaucoup plus entre les habitats et les abris qu'ils ne le font avec les régions extérieures. 

Les scientifiques sociaux font références aux différentes sections du système solaire en tant que régions culturelles. Les différentes régions de la ceinture forment également une région culturelle similaire, mais à cause de la dérive éventuelles des astéroïdes situés sur des orbites différentes, la cohésion et l'unité de ces unités culturelles est quelque peu affaiblie. Les habitats situés à la limite du système solaire (autour d'Uranus, de Neptune et de Pluton) forment des regions culturelle similaires, mais le peu d'habitats dans la Ceinture de Kuiper et dans le Nuage d'Oort n'ont pas de telles régions culturelles étant donné l'extrême distance les séparants. 

Bien que la communication entre les habitats de la même région culturelle est quelque peu perturbante à cause des différences culturelle intra-régionalle et des petits décallages temporelles, elle est suffisament rapide et simple pour que les personnes sur différents habitats pusisent garder contact les uns avec les autres. De plus, la plupart des habitats à l'intérieur d'une même région culturelle sont suffisament proche pour que l'égocast soit abordable par la plupart des personnes. En revanche, egocaster entre différentes régions culturelles est relativement cher. Beaucoup de scientifiques sociaux prédisent que dans un ou deux décennies, les différentes régions culturelles seront au moisn aussi différentes les unes des autres que ne l'étaient les nations Terrestre les plus éloignées lros de la première moitié du 20° siècle - peut-être même plus en raison des alteratiosn physique que les cultures introduisent alros qu'elles continuent d'évoluer. 

\subsection{Expérimentation culturelle} \label{sec:cult-exper} 

Alors que la nostalgie de la Terre reste une motivation sociale puissante, la rupture avec la Terre a amené beaucoup d'habitants du système solaire à expérimenter de nouvelles forme de culture et de société. Depuis que la Chute a détruit les liens physiques avec le passé et que la défaite des derniers gouvernements de la vieille-Terre s'est terminé en liens idéologiques avec les vieilles forces politqiue et sociale, beaucoup de transhumains se perçoivent comme vivant dans une nouvelle ère libre, dans laquelle le passé est mort. Même les personnes qui portent toujours des bijoux de nostalgie et qui passent plusieurs heures par jour dans des simulspace basé sur la bonne vieille Terre sont trés interessés par la possibilité de l'expérimentation sociale et politique. Ceux qui n'étaient pas critiques vis à vis des nation-états Terrestres et de leur nombreux échecs étaient toujours au pouvoir le jour ou la Terre tomba. 

Beaucoup des expérimentateurs sociaux les plus extrêmes se sont installés dans l'un des nombreux petits habitats du système extérieur durant la décennie suivant la Chute, mais des personnes interessées par l'expérimentation sociale et culturelle peuvent être trouvée dans l'ensemble du système solaire. En plus de devoir jouer avec les différentes structures internet et les idées de conception, les habitants de nombreuses stations expérimentent de toutes les manières des règles sociales et politiques uniques. Un petit nombre d'habitat le font de manière relativement délibérée, soit parceque les membres s'intéressent à l'innovation sociale ou parceque des chercheurs associés à une hypercop ou université leur ont offerts des biens et des services en paiement du test de l'une de leur dernière théorie. De telles expérimentations ont inclues l'établissement de station dans lesquelles tout les résidents sont incarnés dans des morphs hermaphrodite afin de mesurer l'impact sur les coutumes et le language lorsque le genre est aboli ou dans lesquelles les résidents sont incités à changement librement de morph en fonction des responsabilité et des devoirs qu'ils ont un jour donné. De telles expérimentations cotrôlée sont, cependant, relativement rares - la vaste majorité des structures sociales et coutumes uniques qui sont apparues depusi la Chute ont naturellement évoluées à partir des groupes de personnes partageant des intérêts communs et vivant ensemble dans les mêmes habitats et travaillant, consciemment ou non, à faire correspondre la vie à leur esthétisme ou leur idéologie. 

\subsection{Genre, sexualité et relations} \label{sec:gend-sexu-relat} 

Pour de nombreux transhumains, le genre est devenu un construct social périmé ne reposant sur aucune bases biologique. Après tout, il est difficile de donner du crédit aux rôles des genres lorsqu'un ego peut facilement modifier son sexe, chanegr de peau ou expérimenter la vie des autres par de l'XP. Bien que la plupart des transhumains adhèrent toujours au genre associé à leur sexe biologique originel, beaucoup d'autres change d'identité sexuelle dés qu'ils atteignent l'âge adulte ou cherchent avidement à répéter les permutations transgenre. D'autres examinent et adoptent des identités sexuelles non-traditionnelles tels que les neutres (persuadé que l'absence de sexe permet une plsu grande concentration sur leurs objectifs) ou les doubles genres (le meilleur des deux mondes). Dans de nombreux habitats et cultures bioconservateurs des rôles sexuels plus traditionnels sont cependant préservés. 

La sexualité s'est également développées dans de nouvelels frontières et tabous. Avec les biomod basiques qui fournissent la contraception et la protection contre les MST, le coup d'un soir est la norme. Beaucoup de personnes font carrière en tant qu'escorte et compagnons trés bien payés. En fait, l'expérimentation sexuelle est standard grâce à plusieurs nouvelles technologies. La réalité virtuelle permet des rencontres sexuelles sans même toucher physiquement un partenaire, sans parler de toutes les façon de donner du corps aux fantasmes. Pour ceux qui préfèrent le contact de la vraie peau, les pods de plaisirs pilotés par des IA peuvent satisafaire tous les besoins et sont une forme légale de prostitution dans de nombreux habitats. Le changement de sexe amène en soi de nouvelle expériences, que ce soit par des bio-mods ou une nouvelle incarnation. Même les IAGs, qui ont été socialisées comme humains, font preuve d'une sexualité et de désir. 

L'extension de la durée de vie et le déclin de la religion ont impacté de manière drastique les institutions sociales telles que le mariage. Étant donné la possibilité de changer à la fois la cognition et la biologie au cours de la vie d'un transhumain, les relations qui durent toute une vie ne sont plus considérées comme réalistes. L'idée de relation à long-terme en temps que contrat social s'est développée de manière exponentielle. Alors que cela a amené un nombre de mariage à être purement politique ou une simple transaction, la plupart des gens continuent de voir dans le marraiage un lien d'éttachement émotionnel et de confiance - en particulier un lien qui transcende le corps, car un partenaire peut changer de morph à n'importe quel moment. 

\subsection{La diversité des habitats} \label{sec:diversity-habitats} 

La capacité pour un petit millier de personne sur la même longueur d'onde et possédant des moyens modérés à posséder un petit habitat où ils peuvent créer leur propre société ressemble à la capacité des habitants des États-Unis du 19° siècle de partir vers l'Ouest et à fonder leurs propres communautés idéologiques. La différence principale est que créer ce genre de communauté est plus rapide et plus facile à l'ère moderne. Le mesh est empli de toute sorte de communauté virtuelles dont les membrs espèrent rassembler les moyens de créer leurs propres habitats. Dans beaucoup de cas, elles restent à l'état de rêve inactifs; la plupart des participants ne sont pas prêt à sacrifier le temps et la rep ou l'argent nécessaire. Occasionnellement les membres essayent, juste pour s'apercevoir que les personnes qui promeuvent l'effort sont des arnaqueurs. Des sous-culture virtuelles parviennent occasionnelllement à rassembler suffisament de consécration et de confiance pour construire leurs propres habitats et commencer le processuss de création de leur propre société physique. Une décennie de ce type d'expérimentation culturelle par des centaines d'habitats a produit une grande quantité de sociétés étranges et uniques. 

Il y a, par exemple, des habitats dans lesquels les habitants portent des habits et des images RA qui recouvrent leur corps - et, dans les cas les plus extrêmes, leur visage - et les résidents ne révèlent l'apparence de leur morph qu'à leurs amis les plus proche et leur famille immédiate. Il y a aussi des stations où tous les membres utilisent des modifications cosmétiques pour adopter une apparence idéale, de même qu'il en existe où tout les résidents utilisent des morphs qui sont toutes des clones d'une autre. Quelques uns des habitats les plus excentrique, peuplés par des bioconservateurs extrémistes dominés par la nosalgie de l'époque passée, les amenants à calquer leur société et toute la technologie visible sur une période antérieure de l'histoire, typiquement quelque part entre 0 et 50 BF. 

Il y a même une poignée d'habitats qui se moquent totalement des apréhensions à propos de la fusion et des forks. Les membres de ces communautés se séparent régulièrement en plusieurs forks lorsqu"ils se réveillent, planifient leur journée puis fusionnent les déifférents fork lorsqu'ils vont dormir. Certains forks restent des infomorphs pour la journée, tandis que d'autres utilisent l'une des différentes morphs que possède ou loue un individus, ce qui implique que chaque personne vive typiquement entre deux et six vies par jour. Quelques sociétés, telles que la maison de l'infâme Pax Familiae, vont même plus loin - tous les résidents sont des forks du même individu. Dans certains de ces habitats solipsistique, on s'attend à ce que les forks utilisent des morphs clonés, alors que dans d'autres chaque fork est considéré comme une personne séparée qui doit vivre et se forger sa propre vie. Quelques unes des manifestatiosn els moins extrême de ce type d'habitat incluent les endroits habités par des familles composées partiellement ou entièrement de forks de l'und es membres (les différents forks ont tendance à être traités comme des frêres et des sœurs). 

\section{Technologie} \label{sec:technology} 

La technologie a envahi tous les aspects de l'existence dans Eclipse Phase. La plupart des individus comprennent que, à moins de subir un évènement comme la Chute ou d'être la victime d'un trés grave et improbbable accident, ils n'ont que peude chance de mourrir de manière permanente. Beaucoup de gens anticipent maintenant un futur extrêmement long. Pour la plupart des personnes ces plans restent relativement minime, mais incluent souvent la conscience que peu, voire aucune, relation durera toute une vie. Cependant, l'immortalité fonctionnelle est juste l'une des nombreuse merveille du monde moderne. 

\subsection{Vivre avec l'infotech} \label{sec:living-with-infotech} 

La vie est remplies de données pour quiconque doté d'un insert mesh basique (p. 300) ou d'un ecto (à peu près quatre vingt seize pourcents de la population). Toutes les informations disponibles sur le mesh sont disponibles en une seule pensée pour ceux qui possèdent les meilleurs implants. Pour le reste du monde, il suffit d'une pause rapide pour accéder et comprendre ces informations. Lorsque prend une pause et paraît un peu distrait au milieu d'une conversation, tout le monde comprends qu'isl accèdent à des données et manque d'implants qui leur permettent de le faire de manière subconsciente ou en faisant du multi-tâche. Et donc, lorsqu'un groupe de personne discutent d'un sujet et que personne n'a la réponse à une question, telles que le titre de la preemière vidéo d'un artiste, en l'espace de quelques secondes tout le monde possède cette information. De manière similaire, lorsque quelqu'un se promène dans un jardin, ils peuvent demander des données détaillées sur chaque espèces de plantes en face d'eux d'un  simple coup d'œil ou peut-être d'une pensée ou d'un petit mouvement des doigts. Les individus qui vont dans des zones éloignées et hors de portée de la diffusion normale du mesh emmènent presque toujours avec eux un lien de farcast ou téléchargent d'impressionantes quantités de donnée dans leurs implants ou leur ecto afin qu'ils puissent continuer d'accéder à toutes les données dont ils pourraient avoir besoin. Comme même un implant basique peut gérer de vastes quantités de donnée, le manque d'espace de stockage est rarement un problème. 

L'accès à une telle quantité de d'information facilement disponible a créée toute une variété de réponses culturelles. Être capable de citer des répliques de n'importe quelle vidéos, vieux film, livre ou discours historique est maintenant trivial et peut être fait en quelques secondes de pensées. Alors que beaucoup d'enfant et d'adolescent s'amuse en introduisant de grande quantité de citatiosn célèbre à moitié appropriée dans leur façon de parler, la plupart des adultes ne le font que pour ajouter du poids ou modérer leurs propos. Les personnes qui citent trop fréquemment des sources externes sont considérés comem mornes et comem manquant d'imagination. Reconnaître ce type de citation est simple, puisqu'une personne peut programmer sa muse pour l'alerter quand à la nature et à l'identité de toute citation un peu longue qu'il entend. 

Tous les utilisateurs avertis du mesh apprenent également (pendant leur enfance ou leur adolescence en général) comment éviter de passer trop de temps hors de la conversation en vérifiant des faits ou en accédant à des informatiopns sur le mesh. Les ados se moquent régulièrement de leurs pairs qui font des pauses dans la conversation trop régulièrement ou pour des périodes trop longue lorsqu'ils cherchent un complément d'information sur un sujet, ou de ceux qui passent trop de temps à assembler les faits pour soutenir une argumentation. Des mots comme "hors mesh" ou de "radoteur" sont utilisés par les ados pour se moquer les uns des autres en apprenant comment être plus discret et plus rapide dans leurs recherche d'information, du moins quand ils interagissent avec les autres. Alors que les adultes ne se livrent que rarement à ce type de moquerie directe et franche, les personnes qui sont trop rapidement perdu dans le meshsurf occasionnel ou conversationnele sont gloablement vus comme étant socialement inaptes. Incidemment, les implants qui permettent le multi-tâches ou des accélérations de pensée temporaires sont en forte demande puisqu'ils permettent aux individus de faire des recherches exhaustive et de répéter chaque énoncés qu'ils s'apprètent à faire sans même marquer une pause. les personnes qui peuvent se permettre ce genre de logiciel semblent plus suave, charismatique et intelligent que ceux qui ne le peuvent pas. 

Tout cela signifie que ceux qui n'ont aucun accès au mesh et à la RA - les zéros - paraissent extrèmement austères vis à vis du reste de la transhumanité. Pour beaucoup de personne, les zéros semblent lent, inattentifs et extrêmement borné, alors que pour les zéros, les personnes qui ne possèdent qu'un ecto ou des implants basiques paraissent brillants, astucieux et capable de comprendre les choses à une vitesse quasiment inhumaine. 

\subsection{Ouvrir la Porte de Pandore} \label{sec:open-pand-gate} 

La découverte de la première Porte de Pandore sur la lune de Saturne éponyme peu de temps après la Chute fut un moment critique dans l'histoire transhumaine. Les perspectives ouvertes par cette découverte étaient à la fois fascinantes et terrifiantes. D'un cœté, des technologies bien au delà de tout ce que la transhumanité était capable de faire étaient maintenant entre nos mains. Cela a déclenché des visions d'un horizon bien au-delà des horreurs de la Chute, là où la transhumanité pouvait s'étendre à travers le cosmos, visitant des merveilles qui semblaient perpétuellement hors d'atteinte, même  pour les presqu'immortels. D'un autre côté, la possibilité que ces portes étaient des reliques des TITANs ne pouvait pas être ignoré. Leur existence ouvrait la possibilité que les TITANs puissent un jour revenir, ou que la transhumanité pourrait un jour les rencontrer quelque part dans la galaxie. L'alternative était encore plus effrayante - la possibilité que la porte puisse être d'origine extraterrestre, et que des choses plus dangereuses et effrayantes que les TITANs puissent rôder entre les étoiles. 

Divers hypercorps, gouvernements et d'autres factions mirent leur esprits les plus brillants à la tâche de résoudre les mystères de ces "trous de ver." De nombreuses communautés scientifiques mirent leur ressources en communs, soutenus par des fonds du secteur privé et craquèrent le code des Portes de Pandorre en à peine un an. La porte fût activée mais elle pouvait être programmé pour ouvrir des connexions vers de nombreuses système stellaire distant (un seul à la fois). Bien que ces contrôles soient, au mieux, peu fiables - des connexions se ferment parfois sans avertissements, et d'autres ne peuvent être recontactées bien qu'elles aient été jointe auparvant - les fonctionnalités étaient suffisament stable pour être utilisées dès le début. A l'instant de l'annonce publique de cette découverte séminale, la Corporation Gatekeeper a été cosntituée en une nuit: un conglomérat de ces même communauté scientifiques et de leurs financiers. 

Moisn d'un an après leur première opération, l'hypercorp ouvra la porte aux "resquilleurs:" des explorateurs qui risquent leurs vies pour voir ce qui se trouve derrière les portes. Beaucoup d'entre eux mourrurent de façon horrible; certains furent même perdu pour toujours, mais quelques uns firent des découvertes fantastique tels que des nouevaux mondes et une nouvelle vie. Même si aucune des formes de vies (encore vivantes) étrangères rencontrées jusqu'à présent  n'aient une forme de conscience, beaucoup des mondes sont habitables ou peuvent être terraformés. Des choses plus perturbantes ont été découvertes en mrme temps que ces merveilles: des preuves d'une civilisations étrangères depuis longtemps disparue (les Iktomi), et des signes que les TITANs soient apssés par là avant. 

Des portes supplémentaires ont rapidement été dévouverte dans le reste du système. Contrairement à l'esprit de coopération qui nturait la découverte de la première porte, ces nouvelels portes ont étés considérées comme des ressources hautement contestées. Utilisée initiallement pour la recherche et l'exploitation, beaucoup de ces portes sont maintenant dédiées à des buts de colonisations. Des douzaines si ce n'est pas des centaines de station et de colonies ont été établies sur des exoplanètes, certaines avec un nombre significatifs d'habitants. Il n'y avait pas de manque de pauvre ou d'individus désespérés voulant risquer leur vie sur un monde étranger, si cela signifiait un iota d'amélioration de leurs vies. 

Même si il est maintenant largement accepté que les portes sont le moyen par lequel les TITANs ont évacués le système solaire (une hypothèse qui ne répond pas au pourquoi de cette fuite), elles ne sont pas datables par leur construction. Indépendamment de leur origines, les portes restent l'une des technologies les plus chères te les plus dangereuse. 

Les cinq Portes de Pandorre connus dans le système solaire, leur localisation et l'entité qui les contrôle incluent: 

\begin{itemize} \item Porte Vulcanoïde: Caldwell(Vulcanoïdes) -- TerraGenesis \item Porte Martienne: Ma'adim Vallis (Mars) -- Pathfinder/Consortium Planétaire \item Porte Pandore: Pandore (Système de Saturne) -- Corporation gatekeeper \item Porte Fissure: Uranus -- Collectif Amour et Rage/Anarchistes \item Porte de la Discorde: Eris (Ceinture de Kuiper) -- Group Go-Nin/Ultimates \end{itemize} 

\subsection{Aller au delà du connu} \label{sec:going-beyond-known} 

L'une des expérience les plus étranges pour les resquilleurs et ceux qui explorent des environnements inhabituels telles que les ruines de la Terre est l'absence de données. Ils regardent une planète étrangère ou une personne mutée par les TITANs, et leur recherche leur renvoie différents messages d'erreurs signalant qu'il n'y a soit pas de donnée du tout sur le sujet ou que les seules données sont purement spéculative et doivent être considérées comme dangereusemtn pas fiable. Cela peut-être particulièrement génant lorsque le sujet en question est une petite créature qui vient de se poser sur l'épaule de la personne et qu'il veut savoir si elle est mortelle ou inoffensive. La plupart des gens qui ont moins de soixante ans n'ont jamais été dans un environnemetn où ils ne peuvent pas obtenir d'informations de base sur un sujet en un instant. Apprendre à dépasser le choc de ne rien savoir à propos de quelque chose est l'une des premières compétences que tout les resquilleurs doivent apprendre, et probablement la plus cruciale. 

\subsection{Muses} \label{sec:muses} 

La plupart des individus ont une IA dédiée qui leur sert d'agent de média. Communément appellée muse, ces IA sont des compagnons de toute une vie pour la majorité des personnes de moins de soixante dix ans. Les muses apprenenet les goûts, les habitudes et les préfrences de leur propriétaire, et elles font de leur mieux pour leur simplifier vie et l'utilisation de la technologie. Les muses peuvent être des réveils, des fouines de récupération de données; des planificateurs de rendez-vous et des quantités d'autres fonctionnalités généralemetn limitée uniquement par l'imagination de leur propriétaire. Il n'est même pas nécessaire de leur assigner toutes leurs tâches - les muses savent anticiper les besoin des gens et agir en fonction. Par exemple, la fonction de planification d'une muse peut lui signaler lorsque leur utilisateur doit être réveillé le matin et elle agira comme un réveil sans instructions additionelle de l'utilisateur. Si une muse n'est pas certaines des préférences de son propriétaire, elle demande, mais après avoir travaillé avec l'utilisateur pendant quelques décennies, les muses ont rarement besoin de le faire. Beaucoup de gens conservent de nombreuses sauvegardes de leur muse, car la perte de celle-ci peut être aussi traumatisante que la mort d'une personne aimée. Utiliser une muse générique qui doit être informée de tous les aspect des préférences individuelles de l'utilisateur et alimentée avec un flux constant d'instruction permet aux personnes d'apprécier la valeur de leur propre muse personnelle. Les muses apprenent généralement les préférences basique d'un nouvel utilisateur en un mois ou deux, mais pendant cette phase d'apprentissage l'utilisateur a tendance à être irritable et distrait, puisque les tâches habituellement traitées automatiquement par leur muse ne sont pas faites. 

\subsection{Attitude envers les IAGs} \label{sec:attit-toward-agis} 

La vaste majorité de la transhumanité accusent des IAs germe (intelligence artificielle auto-améliorantes) renégates d'être responsable de la Chute. De fait, toute IA qui n'est pas handicapée ou d'une manière ou d'une autre limitée dans ses capacités d'auto améliorations - incluant les IAGs (intelligence artificielle généraliste) qui étaient communes et en nombre croissant avant la Cute - sont complètement illégale dans de nombreux habitats, ou du moins lourdement régulées. La Chute s'est terminé il y a à peine plus d'une décennie, et beaucoup de transhumains considèrent les IAGs et les TITANs qui ont assassiné leur monde natal comem étant une seule et même chose. 

En plus des lois anti-IAGs extrêmement strcitces, il y a occasionnellement eu des émeutes et des paniques massives autour de labortaoire effectuant toujours des recherches sur les IAGs, qui ont forcé la plupart de ces recherches à s'installer dans des abris isolés. Néanmoins, il y a toujours des personnes dévoués avec passion aux IAGs; certains les voient comme la prochaine étape de l'évolution posthumaine, d'autres apprécient tout les sensibilités et d'autres les vénèrent comme des dieux. Les partisans des IAG ont cependant appris à garder leurs opinions pour eux, à moins de se retrouver marqués comme un agent des TITANs. 

A quelques endroits, principalement dans les zones les plus anarchistes du système extérieur, les comportements vis à vis des AGIs sont plus détendus et elles peuvent même être ouvertement accueillies. Ces encroits reconnaissent que les IAGs ne posent pas les mêmems menaces que les IAs germe et qu'il n'est pas juste de punir les uns pour les actions des autres. Naturellement, ces endroits sont des refuges pour les IAGs actives dans la société transhumaine, qui doivent sinon dissimuler leur nature. 

Dans le système intérieur étroitement surveillé, les hypercorp et le Consortium Planétaire encourage  les sentiments anti-IAGs à la fois comme mesure de sécurité et comme protection contre de possibles compétiteurs. Ce dernier point est l'une des choses qui les rends attractives pour certaines personnes dans le système extérieur; ils comprennent l'avantage conséquent que gagne leur faction ... en supposant, bien entendu, que ces IAGs partagent leur buts et intérêts. 

\subsection{Attitude vis à vis des altérations mentales} \label{sec:attit-towards-ment} 

Dans le système solaire post-Chute, la technologie peut altérer l'esprit des gens; des controverses à propos de ces altératiosn persistent. Peu de personnes ont des problèmes avec l'idée de créer des forks à court-terme utilisant une augmentation de multi-tâche ou des processuss similaires qui assurent que les forks seront réintégrés en quelques heures. Cependant, l'idée de fork à long-terme, et plus particulièrement d'autoriser les forks à avoir accès à leur propres morphs séparée, perturbe beaucoup de gens. Puisqu'il n'y a pas assez de morph pour que tout le monde en profite, fournir des morphs à un fork marque les gens comme étant égoïstes et gaspilleurs. En conséquence, pour les rares cas où les personne incarnent un de leurs forks, ils lui fournissent typiquement une synthmorph pour éviter le stigmate social associé au fait d'utilsier plusieurs corps à la fois. 

Les forks qui existent pour plus de quelques heures rendent les gens mal à l'aise car ils commencent à diverger légèrement de leur personnalité d'origine. Beaucoup de personnes trouvent quelque peu dérangeante l'idée d'avoir deux versions différentes et distinctes d'eux même. Bien qu'il y ait des habitats (essentiellement dans le système extérieur) où forker est une part importante de la vie quotidienne et où les forks existent souvent indépendamment pour un jour ou deux, la plupart des visiteurs trouvent ce genr d'habitats désagréable et bizarre. 

Cependat et alors que le fork volontaire est toujours perçu comme quelque chose d'étrange, les utilisations involontaire et les technologies mentales associées sont tellement horrible qu'elles forment la base de la littérature criminelle la plus atroce. Quelqu'un connu pour avoir étét involontairement mind-nappé et qui possède un fork involontaire - et souvent secret -est quelque chose que les gens considèrent avec une terreur abjecte, même si cela reste relativement rare. De manière similaire, alors que la chirurgie mentale utilisée pour corriger des problèmes psychiatrique ou comme punition de divers crimes graves reste effrayante et dérangeante mais communément accepté, le piratage cognitif illégal déclenche horreur et dégoüt de presque tout le monde dans le système solaire. Les condamnations pour le fork forcé et le piratage cognitif sont exceptionnellement élevées. Dans de nombreux habitats, il font partis des rares crimes qui amènent à une condmanation à mrot (incluant la destruction de tous les forks et suavegardes connus). 

\subsection{Voyage} \label{sec:travel} 

Le voyage entre les habitats et d'autres colonnie transhumaine est à la fois excessivement facile et extrêmement cher. L'egocasting à longue portée est cher, de même que l'acquisition d'une nouvelle morph a destination. Les voyageurs ont développé différente façon pour contourner cette obstacle; par exemple, si quelqu'un n'a besoin de visiter un habitat que pour quelques jours et qu'il el fait essentiellement pour commencr des conversations en temps réel, il choisira souvent de rester en infomorph pour la durée de la visite et de communiquer par RA, s'économisant ainsi les dépenses de la réincarnation. Pour les visiteurs qui ont besoin d'une morph mais qui ne resterons pas longtemps, beaucoup d'habtitas proposent l'option de louer une morph générique type splicer, une synthmorph ou, pour un prix un peu plus élevé, une morph générique exalt. Les habitats ou les mondes avec des prérequis inhabituel, tels que mars, Europe ou les différents habitats en zéro-g, offrent des morphs rusteur, aquanautes ou bounceur au lieu des spliceurs. Ces morphs peuvent être utilisée pendant une semaine sans trop de difficulté, et en profiter pou un mois complet est généralement possible avec suffisament de négociations et de paiement. Pendant ce temps, la morph précédente du voyageur est gardée en stase médicale dasn son habitat d'origine, attendant que l'ego revienne. 

Une autre technique est d'échanegr sa morph avec quelqu'un d'un autre habitat que l'on connaît et qui voyage pendant ce temps. De rares personnes le font avec des étrangers rencontrés sur le mesh, mais les vids et d'autres divertissements sont pleins de légendes de personne ayant zeu leur morph ou leur identité volé de cette façon. Quelques une de ces histoire d'horreur sont basées sur des faits réels. Très peu de gens sont prêt à laisser n'importe quel inconnu utiliser leur corps, et beaucoup de gens ne préterons jamais leur morph à quiconque. 

Quelques personnes sont cependant volontaire, contre rémunération, à servir de "taxi" vivant pour une infomorph de passage, les emmenant avec eux. Dans ce cas, l'informorph "ghost-rider" n'as pas l'autorisation de contrôler la morph de leur hôte directement et est simplement un passager embarqué pour la balade, donnant les directives et communiquant avec leurs chauffeurs électroniquement. 

Les voyageurs qui veulent soit immigré vers un nouvel habitats ou en visiter un pour plusieurs mois ou plus doivent acquérir leur propre morph. Habituellement, ils réduisent le coût de l'acquisition d'une nouvelle morp en vendant l'ancienne à une banque de corps. Alternativement, certains individus incarné dans des morphs trés chères et conçues sur mesure et qui voyagent sur des distances relativement courte louent une coque générique pendant quelques semaines et s'arrange pour que leur vieille morph leur soit expédiée par un transport rapide. Procéder ainsi est rarement au delà d'une dépense modérée, ce qui rend la procédure moins chère que l'achat ou le remplacement d'une morph haut-de-gamme et sur-mesure. 

\subsection{Vie privée} \label{sec:privacy} 

La vie privée est un bien hors de prix pour la plupart de shabitants du système solaire, mais il est tellement rare que pour beaucoup de personne c'est un concept étranger. Pendant le 20° et au début du 21° siècle, la vie privée consistait de deux concepts actuellement complètement séparés - la possibilité de rester anonyme ou de ne pas se faire remarquer et la possibilité d'éviter les intrusions non désirées. Le premier est largement absent de la vie des gens d'aujourd'hui. Quiconque qui uploade quelquechose depuis une partie non-privée du mesh comprend que tout le monde qui voudrait faire pareil a les moyens de le faire. De manière similaire, quiconque qui passe du temps dans un lieu public comprend que n'importe qui peut savoir où ils sont allés, ce qu'ils ont fait et ce qu'ils ont dit en raison de l'ubiquité des périphériques meshés possédant des capteurs. La vie publique de tout un chacun, à la fois sur le mesh et en personne, est devenue une base de donnée facilement analysable. A peu près tout le monde conserve un tel enregistrement de leur propre vie, communément appelé un lifelog. La plupart des personnes rendent leur lifelog public, comprenant que l'anonymat est maintenant un concept archaïque. 

\begin{quotation} \textbf{Recherche solarchive: entrées incapacitantes } pendant la Chute, les TITANs ont utilisé toute une variété d'intrusions RA et en-ligne qui ont interféré avec ou ont incapacité leur cibles. Les plus basiques étaient des illusions RA faites pour convaincre les gens que leur environnement physique était trés différent de ce qu'il était réellement. Cela a trompé les victimes et les a fait attaquer les leurs ou a plus simplement déclenché des paniques massives. Des versions plus avancés ont ciblés les éléments empathiques de la RA, déclenchant la peur ou d'autres réponses émotionnelles/ D'autres encore ont anéantis leur cible avec des entrées sensorielles insupportables, tellement puissantes qu'elles ont dépassés les filtres et infligés des dégâts neurologiques. 

En dépit des rumeurs et des peurs des exploits baptisés "exploit basiliques" - des attaques basés sur la vue ou d'autres entrées sensorielles qui auraient subvertis les transhumains en exploitant la manière dont le cerveau traduit ces données - aucun rapport crédible n'a pu être vérifié. \end{quotation} 

Alors que l'intérieur des habitations privées reste libre de la surveillance continue, presque tous les habitats ont des capteurs d'urgence dans chaque bâtriment leur fournissant un enregistrement complet des évènements aux services d'urgences en cas de problèmes tels que des fuites de produits chimiques dangereux, un incendit suffisament grand, une explosion, une dépression atmosphérique ou d'autres évènement potentiellement danegreux et dramatiques. Les évènement de la Chute et le fait que pratiquement toute l'humanité vit maintenant dans des habitats entouré par un environnement hostile font que de tels capteurs sont un standard universel. Quelques habitats n'autorisent pas les cpateurs d'urgence dans les domiciles privés, mais la plupart des personnes le sconsidèrent comme des pièges mortels. Ces capteurs d'urgence n'enregistrent rien d'autre que l'absence de danger potentiel tant qu'ils ne sont pas déclanchés par des évènements spécifiques. Cette limitation permet aux individus de préserver leur vie privée dans leur propre résidence - tant qu'ils sont certains que personne n'a introduit un périphérique d'enregistrement dans leur maison. Au final, demeurer non observé est une question à la fois d'attention et de confiance, et tout le monde comprend que la plupart du temps, tout ce qu'ils font fera parti d'un vaste enregistrement public. 

De manière fortement contrasté, la liberté d'éviter les intrusions indésirées est attentivement prisées par les habitants de l'époque post-Chute. Les intrusions personnelles ou de données non désirées dans le domicile privé d'une personne ou d'un de ses fichiers électronique privé esst un crime dans la plupart des habitats et un crime sévère dans une bonne partie. Et alors que le mesh et la réalité augmentée sont emplis d'adware gérée par des IA, la plupart d'entre eux sont devenus relativement bénins et fournissent des suggestions non intrusive pour des biens, de l'information et des services qui ont des chances d'intéresser la personne ciblée. La muse d'un individu filtre les publicités non désirées. Alors qu'il est certainement possible de créer des publicictés qui peuvent se frayer un chemin à travers les filtres des muses, cette pratique est habituellement illégale. 

Les intrsuions RA non désirée sont limités de manière similaire. Pendant les premiers jours de la technologie RA, il y a eu de grave problèmes avec des utilisateurs surchargés par des entrées non demandée et déconcentrante - comme beaucoup le disent, la brume était en effet trés épaisse, les lois et les coutumes ont dnc changées pour pévenir de telles invasions. Aujourd'hui, beaucoup de personne s'attende à ne rencontrer que les données qu'ils recherchent ou qui pourrait le sineresser; et que toutes les données qui ne les intéressent pas s'évanouissent rapidement. Être cerné par une énorme quantité de donnée RA non désirée est bien plus qu'une gène et une distraction, c'est également profondément effrayant, car cela signifie qu'il y a un sérieux problème soit avec le mesh de l'habitat soit avec l'électronique de la personne - cela peut même signifier que tout l'habitat subit une attaque directe d'arme de cyberguerre. 

\subsection{Vivre en Low-Tech} \label{sec:low-tech-existence} 

Plus que quatre-vingt quinze pourcent de l'humanité habite des morphs créée artificiellement. La plupart possède également des implants basiques, et la vaste majorité du reste porte des ectos avec des affichages rétiniens et d'autre périphériques simple permettant à l'utiulisateur de percevoir et d'interagir pleinement avec le vaste réseau d'information qui les entoure. Cependant, un peu moins de quatre pourcent de la population restante habite des morphs spliceurs ou plate sans implants basique et manquent également d'accès aux ectos et aux autres technologie de base. 

Étant donné qu'un ecto est à la fois une dépense triviale et un équipement vital pour vivre dans le système solaire, les rares individus qui n'ont pas cette technologie restent tout en bas de l'échelel sociale. Quelques uns sont les habitants les plus pauvres des habitats les plus marginals, mais beaucoup sont des esclaves ou à peine mieux. La classe sociale la plus basse dans la République jovienne n'as pas accès à l'infotech personnelle de même que les classes les plus basses parmi les contractés aux hypercorps et le Consortium Planétaire sur la Lune et sur Mars. Ces individus sont soit des criminels contractés ou des personnes qui manquent de compétences utile et qui sont assignés à des tâches physiques basique qui ne peuvent être faites plus efficacement par des IAs. 

Le manque d'accès au mesh transforme ces pauvres "zéros" en handicapés mentaux et sociaux, incapable de percevoir la vaste richesse de la RA que la plupart des gens considèrent comme acquise. Ils sont également incapable de communiquer avec quelqu'un hors de portée vocale ou d'accéder à toute information , incluant les la signalisation routière ou les affichages des boutiques. Lorsque c'est nécessaire, les chefs d'équipes et superviseurs responsables d'un groupe de éros leur permet d'accéder à des navigateurs mesh monitorés. Ces appareils ressemblent aux terminaux portables communs au début du 21° siècle et ont des fonctionnalités limitées, telles que l'interdiction de la communication et la restriction des recherches sur le mesh à des sujets précautionneusement filtrés. 

Enr aison de leur incapacité d'accéder eu mesh ou à la RA, les zéros sont presque complètement isolés du reste du monde, ils sont donc également incapable de s'organiser efficacement ou d'être à la source de problème pour les personnes qui les contrôlent. L'essentiel du système extérieur considère l'existance des zéros comme l'un des plus grand crimes contre la transhumanité perpétré par le Consortium Planétaire et par la République Jovienne. 

\subsection{Vie, mort et morph} \label{sec:life-death-morphs} 

Alors que la mort n'est plus une fatalité pour la transhumanité, un risque demeure. Pendant la décennie précédant la Chutre, la plupart de la transhumanité grandissait avec l'idée que l'immortalité etait à leur portée. Puis, juste en quelques années, les TITANs ont anéanti plus de quatre-vingt dix pourcent d'entre nous. Face à l'horreur de tant de morts inutiles, les efforts faits pour assurer la vie des humains survivants est devenue la priorité. Maintenant, la technologie de l'immortalité - l'upload, les piles corticales et d'autre merveilles associées - est commune. 

Aujourd'hui, la plupart des résidents du système solaire se sont adapté à cette existence (à l'exception des biocnservateurs extrêmistes); tout le monde s'attend à vivre éternellement et à avoir des amis, des amants et des ennemis qui fassent de même. Cependant, même si la mort est rare, elle reste possible. Un accident grave peut détruire la pile corticale de quelqu'un ainsi que leur cerveau, et des egos peuvent être effacés pour des crimes suffisament graves - bien que le processus d'exécution soit considérablement plus difficile qu'il ne l'avait été quelques décennies plus tôt. 

Pour la plupart des gens (à l'exception de ceux qui sont trop pauvres pour s'acheter une nouvelle morph), la mort non-permanente est une gène équivalente à ce que la plupart des personnes du 20° siècles considéraient comme une gène modérée, telle qu'une mauvaise grippe intestinale ou un bras cassé. Dans presque tout les habitats, si quelqu'un est responsable de la mort temporaire d'une personne, que ce soit par accident ou prémédité, ils sont également responables de dédommager la réincarnation e la vitcime dans une morph identique, particullièrement si elle ne possède pas une quelconque assurance de réincarnation. Les personnes qui sont mort temporairement peuvent s'attendre à recevoir des visites de tout ceux qui sont suffisament proche après leur réincarnation, aussi bien qu'une quantité impressionante de cartes électronique et probablement quelques cadeaux de leurs contacts et collègues, tous manifestant leur compassion dans la mort et les félicitants pour leur retour dans le monde des incarnés. L'échange de ces "cadeaux de vie" est une part accepté de l'appartenance à certains corps professionels tels que les employés des services d'urgence, dans lesquels les membres risquent régulièrement la mort temporaire. 

Choisirent délibérément de changer de morphs ou de devenir temporairement une infomorph est traité différement. Les personnes mettent générallement un jour ou deux entre le moment où ils décident de changer de morphs et le moment où ils passent à l'acte. Pendant ce temps, il est considéré comme poli d'informer de sa prochaine réincarnation toutes les personnes qu'ils connaissent bien ou avec qui ils travaillent. Avec les visites personnelles, les appels ou les cartes électroniques précisant la date de l'évènement à venir, on attend de la personne qui se réincarne qu'elle fournisse une image de ce à quoi ressemblera leur nouvelle morph, afn que les personnes qu'elle connaît puissent facilement la reconnaître. Il est cependant maladroit pour quelqu'un qui qui se met à hour vers une meilleure morph d'inclure les détails à propos de cette morph. Après quelques jours de réincarnation, un" fête de réincarnation" est générallement organisée pour présenter les personnes qu'elle connaît à leur nouvelle morphs. En fonction du niveau d'aisance, de célébréité et de sociabilisation de l'individu, ces fêtes vont des réceptions somptueusestenues dans les salles de bal des hôtel aux petites rassemblement plus intimes chez la personne. 

La mort permanente est gérée de manière extrêmement différente. Comme il est relativement rare et plus réellement attendue, les anciens rituels funéraires entourant la mort se sont effacées et de nouvelles traditions ont prit leur place. Puisque chaque mort rappelle à beaucoup de personne la perte des milliards d'humains qui sont mort à jamais pendant la Chute, la plupart des rares funérailles sont tenues à la fois en l'honneur de la personne qui vient juste de mourir aussi bien que des victimes de la Chute. 

\subsection{Divertissement et média} \label{sec:entertainment-media} 

Un nombre substantiem de media ont survécu à la Chute de la Terre et un nombre significatif des transhumains modernes vivent en créant des chansons, des histoires, des reportages ou d'autres media. Toutes ces choses sont accessible facilement et rapidement grâce à des implants basiques, des extos ou (en de rares occasions) par des terminaux mobiles archaïques. Cependant, la plupart de ces medias ne sont pas au goüt de tout le monde, et la vaste majorité de ces créations sont médiocres. Du coup, beaucoup d'humains gardent deux niveaux d'évaluations entre eux et ce à quoi ils pourraient considérer être exposés. 

le premier niveau est basé sur la popularité et les critiques. Chaque élément de média à une notation, souvent pondérée par l'opinion des critiques qui possèdent un score de rep élevé qui commentent les vertues et les défauts de l'élément. Des IAs spécialisées évaluent également la réponse des consommateurs, afin que les individus puisse utiliser des critiques qu'ils connaissent ou pour qu'ils puissent trouver du media qui est soit largement soit spécifiquement populaire dans leur niche démogaphique et culturelle. 

Le second niveau est celui de la muse de chaque individu. Les muses apprennent les goûts et humeurs de leurs propriétaires et cherchent automatiquement des medias et recommandent diverses sources de media. Les individus peuvent faire ce qu'ils veulent de demander à leur muse de choisir quelque chose qu'ils aimeront, à demander quelque chose qui défieras leurs opinions, en passant par récupérer toutes les nouvelles récentes qui seront d'un intérêt quelconque pour eux. Les muses utilisent leur compréhension des préférences de leur utilisateur, mélangé avec les notations et les critiques, afin de faire leur choix. Les individus peuvent même configurer leur muses pour éditer tous les medias afin qu'ils correspondent mieux à leurs intérêts et préférence. Dans les cas les plus extrêmes, ce processus peut déformer et éditer des informations jusqu'au point où elles n'ont plus de relation avec des évènements réels. Le même procédé est utilisé pour rendre les dialogues entre personnages dans les romans et les vids un peu plus interessants. De manière plus courante, les muses éditent plus simplement des aspects d'un article ou d'une histoire qui n'intéresse pas l'individu. 

Les notations, les critiques et les muses permettent aux individus d'éviter la surcharge médiatique, mais cela renforce également leur barrère sous-culturelle. Une grande majorité de gens ne cherchent que des medias et des infos qui renforcent leurs opinions et croyance existantes. Les individus xénophobes qui ne font pas confiance aux non-humains, des poulpes élevés ou Facteurs, voient réglulièrement des informations et des drames RA à propos de méchant étrangers et d'animaux élevés déviants qui commettent des crimes violents. De manière similaire, les individus qui ne sont interessés que par leur propre habitat reçoivent toutes les informations altérées par leur muses afin qu'elles ne fassent références qu'aux évènement extérieurs à leur station qui pourraient influer l'intérieur de celle-ci. 

De manière très pragmatique, des individus issus de sous-culture et de démographie radicalement différentes habitent des mondes complètement différents. L'une des forces qui travaille contre cette séparation est le fait que beaucoup de personne veulent suivre la vie et l'opinion de ceux qui possèdent un score de réputation élevé. Dans de nombreux cas, une grande portion de ces scores de rep élevé vient de leur intérêt et de leur volonté à interagir avec (ou au moins prendre en compte) une grande variété de sources d'informations différentes. De fait, écouter l'opignon d'une célébrités avec une grosse rep peut exposer les gens à des informations qu'ils pourraient n'avoir jamais rencontrer autrement. De plus, et dans beaucoup d'habitat, les IAs responsable de la distribution des media marquent certaines information comme suffisament importante pour ignorer le filtrage des muses. 

Cet étiquettage a une apparoition régulière et attendue dans certains habitats, alors que dans d'autre, elel est réservée pour les information les plus importantes et qui peuvent sauver des vies. Outrepasser les muses pour une raison moins importante dans ces stations est considéré comme une invasion flagrante de la vie privée, voire même un crime. 

\subsection{Savoir perdu} \label{sec:lost-lore} 

L'accumulation des médias et des connaissance de la Terre, couvrant toute l'histoire de l'intelligence humanie, représante un total de donne gigantesque et impressionant. Même avant la Chute, beaucoup d'abri orbitaux avaient récuéprés un enregistrement complet de toute la connaissance et la créativité humaine pécédente, incluant des copies de tout les livres, tabeleaux, chansons, film, programme TV, jeu vidéo, journal et article de magasine qui ait jamais été traduit dans un format digital, ainsi que les sauveagrde de toute sles archives de l'internet Terrestre. De nombreux programmes destrcutcifs libérés pendant la Chute ont cependant corrompus la plupart de ces informations et en ont effacés à jamais une partie. Ce qui siginife que ce qu'il reste des archives historique de la Terre est inégal et incomplet. Beaucoup a survécu, mais quelques trésors ont étés perdus. En particulier, les medias de l'époque de la Chute en elle-même sont particuliè-rement difficiles à dénicher, étant donné les attaques constantes que menaient les TITANs sur les systèmes d'informations. Les données propriétaires qui ont étés écarétes du domaine public derrières des portes électroniques sur Terre sont encore plsu sujettes à disparition, à l'exception de quelques hypercorps qui ont réussi a transférer leur données liée à la Terre vers des serveurs hors-monde dans les temps. Retrouver des données perdues est une tâche lucratives poru les récupérateurs et les archéologistes, bien que fouiller les confins de la Terre les plsu dangereux ou les habitats détriuts pendant la Chute est une proposition risquée. 

\subsection{Métacélébrités} \label{sec:metaceleb} 

Comme l'industrie de la culture l'a rapidement découvert, la biotech et les technologies de réincarnation sont entrées en collision avec la capacité des médias à focalsier l'attention sur des icônes spécifiques. Lorsque tout le monde peu se bodysculpter, les belles personnes doivent être plus que des visages agréables. Pour finir, l'intérêt du public envers les peoples s'effondra lorsque les célébrités changèrent régulièrement d'apparence et n'étaient plus immédiatement reconnaissable. 

Une des manières que les industries du divertissement ont trouvé est de promouvoir les métacélébrités - des icônes basées sur des personnages plutôt que dur des personnes réelles. Chaque métacélébrités possède sa propre (et extrêmement chère) morph unique et personnalisée, mais la personne incarnée à l'intérieur de cette morph change souvent. L'actrice Angélique Stardust, par exemple, existait autrefois en temps que réelle personne, mais elle est maintenant un personnage qui a été interprété par plus d'une douzaine de personne depuis son ascenssion rapide à la célébrité en 3 AF et qu'elle eut vendu la célébrité de son personnage et ses droits d'exploitation à Experia. De la même manière, l'acteur maintes fois récompensée et briseur de cœur Juan Nguyen est une personnalité construite d'après la star héroîque qui est morte et qui a été perdue pendant la Chute. Beaucoup de métacélébrités sont modelées d'arpès des personnages de fictions; la mauvaise fille et célébre Sun Mi Hee n'est guère différente dans la vie que dans ses rôles de méchante botteuse de cul qui l'ont rendues célèbres, ne voyageant jamais sans sa paire iconique de léopard grondants et intelligents. Les acteurs endossant le rôle d'une métacélébrité subissent souvent une psychochirurgie pour mieux jouer leur rôle. 

Les personnalités métacélèbres sont gérées strictement et marchandisées comme un produit de média pour séduire un groupe de consommateur. Bien qu'elles aient un rôle actif dans les cercles de l'hyperélite, beaucoup de ces célébrités authentique les considèrent au mieux avec humour et au pire avec dédain - bien que certains ont apprit par la méthode dure à ne pas sous-estimer ou à ne pas chercher les petites armées d'ingénieurs sociaux derrière chaque image de la métacélébrité minutieusement fabriquée. 

\subsection{Types de divertissements populaires} \label{sec:popul-types-entert} 

Les formes les plus populaires de divertissement  

\subsubsection{Vids et jeuxvid} \label{sec:vids-vidgames} 

Les vids sont des divertissements passifs qui sont appréciés soit comme un divertissement audiovisuel en haute résolution ou comme une exéprience en immersion totale dans laqeulle le spectateur peut augmenter son expérience avec l'odorat, le toucher et le goût tout en profitant du point de vue de l'un des personnages majeur. Les regarder en utilisant juste la vue et l'ouïe est peu ou prou la même chose que regarder un film du 20° siècle, sauf qu'elles sont interactives et en 3D. En comparaison, le visionnaige en immersion complète est à peu près équivalent à être réellement présent dans l'histoire. 

La plupart des vids modernes ont des thèmes variables et des réglages préférentiels permettant au spectateur d'ajuster le contennu de ce qu'ils regardent, incluant le niveau de violence, la quantité de sexe et le type de sexualité qu'ils rpéfèrent aussi bien que les apparences de quelques uns des personnages principaux. De plus, beaucoup de vids ont plusieurs fins alternatives pour les personnes qui rpéfèrent les fins heureuse, douce-amère ou triste. Du coup, deux personnes regardant la même vid peuvent ressentir des expériences trés différentes si ils utilisent des réglages radicalement différents. 

Les jeuxvids sont comme les vids, en bien plus flexible. Dans les jeuxvid, le spectateur fait bien plus que vivre l'histoire avec le protagoniste  - ils deviennent le héros, modelant l'histoire par leurs propres actions, de manière similaires au jeux vidéo sophistiqués du début du 21° siècle. Certains jeux permettent d'avoir jusqu'à une douzaine de participants individuels ou de connecter des milliers de joueurs par le mesh, alors que d'autres ne sont conçus que pour un seul joueur. 

Ce degré de liberté dans les jeuxvids varie. Certain sont des mondes presques entièrement interactifs similaire à beaucoup d'univers en RV dont tous les personnages ou presque sont controllés par des IAs, alors que d'autres sont considérablement plus simple et plus limité, et dont les intearctiosn avec le joueur se limitent à quelques décisions cruciales. La ligne de séparation entre les vids et les jeuxvids est floue, mais les deux médias ont le point commun d'être conçus pour un usage solitaire ou par quelques joueurs et spectateurs qui ne sont aps trop éloignés physiquement le suns des autres. Les vids et les jeuxvid constituent la forme la plus populaire de divertissement, ceux prenant pour cadre la Terre avant la Chute sont particulièrement répandus. 

\subsubsection{XP} \label{sec:xp} 

Le rejeu d'expérience (Experience Playback ou XP) est une catégoriée de vid spéciale qui sont constituées de l'enregistrement des impressions sensorielles d'un individus. Presque tous les habitants du système solaire vivent des vies relativement tranquille et sans danger et ont naturellement envie de pouvoir vivre des aventures telles qu'escalader le Mont Olympus, passer une journée dans l'un des habitats privé parmi les plus luxueus et éxotiques, partir en mission de récupération sur Terre ou resquiller. Il y a aussi une marge du marché relativement prospère pour des XPs moins savoureuse, incluant des enregistrements de personne commettant toute sorte de crimes violent ou dangereux et des XPs d'affrontement à balles réelles entre des criminels sur armés et des forces de sécurité, qui se termient souvent par la mort de la morph fournissant le point de vue. 

Quiconque possédant des inserts mesh peut créer une XP de ses expériences passées, et n'importe qui possédant un ecto ou des inserts mesh peut accéder à l'enregistrement sensoriel. Vendre une XP particulièrement excitante, telle que l'enregsitrement du premier contact avec les Facteurs, peut rapporter beaucoup d'argent ou de rep. La plupart des XPs consitent en un mélange d'enregistrements sensoriels et les pensées des individus qui les ont faites. Beaucoup de personne qui utilisent l'XP ne sont intéressé que par l'enregistrement sensoriel et considèrent qu'avoir dans leur tête les pensées et les émotions d'une autre personne enregistrés est intrusif et désagréable. Cependant, quelque aficionados hradcore de XPs pensent qu'accéder à l'XP complète, y compris avec les émotions enregistrées, rend l'expérience plus immersive et réelle. 

Une minorité significative de fans d'XP deviennent fascinnés par une ou deux personnes téméraires qui vendent régulièrement des XPs, appelés X-casteurs, visionnant tous leurs clips, incluant à la fois les expérience et les pensées les accompagnants. Quelques uns de ces fans d'XP deviennent plus interessé par la personne qui enregsitre le clip que par les expériences individuelles, et ils en viennent souvent à croire qu'ils ont une compréhension spéciale de cette personne, au point où ils s'dientifient fortement avec cette personne ou en tomebnt amoureux. De plus, les individus qui accèdent souvent à l'XPs d'une seule personne commencent parfois à mimer différentes habitudes ou formes d'expression de cette personne. Des X-casturs particulièrement populaires sont parfois particulièremnt dérangés lorsqu'ils voient des dizaines de milliers de personnes imiter l'une de leur expression ou habitude les plus idiosyncratique. 

Quelques fans sérieux - connus comme Xers (prononcer "ix-eurs") - altèrent leur morph pour ressembler à laur X-casteurs préféré. Quelques Xers obsessifs tentent réellement de contacter et de harceler certain X-casteurs, espérant peut-être devenir une part réelle d'un clip XP. Dans beaucoup d'habitats et de sous-culture, les Xers ont glaobelemnt considérés comme ayant des vies particulièrement ennuyeuses. Les Xers hardcore sotn souvent vus comme étant danegreux et potentiellement instables. 

\subsubsection{Jeux RA} \label{sec:ar-games} 

Les jeux en Réalité AUgmentée (RA) impliquent les joueurs dans des interactions avec le monde physique et l'imagerie en réalité augmentée qui redessine les personnes ou les objets qu'ils perçoivent. Au lieu de voir un autre joueur dans unje morph splicer et ordinairement vétue, un joueur d'un jeu RA pourrait, par exemple, voir un zombie pourrissant horifique, une forme de vie étrangère bizarre ou un soldat sur-armé. Ces jeux ont tendance à être ciblé localement à l'intérieur d'un habitat ou d'une vilel partiuclière car ils permettent aux joueurs d'interagir entre eux lorsqu'ils sont proches, mais certains jeu relient les habitats d'une même région culturelle. 

La nature et l'intensité de ces jeux varient grandement. Certains sont ds jeux à long-terme impliquant des personnes imaginant qu'ils sont des espions sous couverture ou n'importe quel autre rôle unique et excitant. Les joueurs peuvent prétendre être n'importe quoi depuis des voyageurs temporels qui tentent d'empêcher un désastre horrible jusqu'à être des agents tentant de dévoiler les plans des personne infectés par les TITANs dans leur habitat - qui s'avèrent être camouflés en concepteurs inflitrés, en assistants personnels, etc. Lors de leur vie quotidienne, les joueurs échange des messages les uns avec els autres ainsi qu'avec les personens qui animent et maintiennent le jeu. Certains des jeux RA à long-terme sont actifs depuis plusieurs années, les plus anciens durent depuis presque vingt ans. 

Les jeux RA à court-terme, d'un autre côté, durent entre plusieurs heures et plusieurs jours. Les personnages faisant fonctionner ces jeux louent générallement un hôtel ou un parc et différents immeubles pendant la durée de la partie. Ces jeux sont presque tous extrêmement dramatique et vont d'une invasion massive de zombie ou d'aliens que les joueurs doivent gérer à la participation à une simulation d'un évènement sur Terre, telle que la prise de la Bastille pendant la Révolution Française. Bien que ces jeux RA peuvent être considérabelemtn moins détaillés que les mondes RV ou les jeuxvids, beaucoup de jeouurs accordent de l'importance au "réalisme" ou au fait d'être présent physiquement pendant la partie. 

Comme les participants aux jeux RA agissent dans le monde réel, incluant des actions qui peuvent être disruptive voire dangereuse, les concepteurs des jeux RA prennent grand soin de prévoir ce type de problèmes. Dans certains jeux RA réceants, la plupart de ceux qui se déroulent plus de vingt ans avant la Chute, des joueurs ont été occasionnellement grièvement blessés. Quelques concepteurs de jeux RA sans scrupules ont utilisés leurs jeux comme couverture pour un braquage réel ou un acte d terrorisme qui ont étés commis par des joueurs involontaires qui pensaient que leurs actions faisaient partie du jeu. 

Depuis cette époque, des drones d'observations des forces de l'ordre gardent une trace des personens particiapnt aux jeux RA. Dans presque tous les habitats, les personnes désirant organiser un jeu RA doivent enregistrer leur jeu auprès des forces de l'ordre local ou payer des amendes élevées. 

\subsubsection{Univers RV} \label{sec:vr-worlds} 

Les univers en Réalité Virtuelle (RV) sont des divertissements qui impliquent la création d'un vaste environnement simulé et immersif - un simulspace - dans lequel la plupart des personnages principaux sont joués par des transhumains ou d'autres être conscients. Contrairement aux vids ou aux jeuxvids, les simulspaces sont conçus spécifiquement pour un grand nombre de participants. Les univers RV vont des versions dupliquées de différentes époques de l'histoire Terrestre aux mondes étranges et fantastiques emplis de magie, de dragons et d'autres merveilels similaires. Tout type de monde étrange ou de cadre basés sur des bizarreries qomme le voyage temporel sont également communs. Comme c'est le cas avec les vids, les simulspaces les plus populaires sont ceux qui prennent pour cadre la Terre peu de temps avant la Chute. 

Les univers RV peuvent ont généralement un nombre de participants allant d'une douzaine à plusierus dizaines de milliers. Pour une meilleure expérience, beaucoup d'utilisateurs préfèrent accéder aux simulspace via des connexions filaires aux serveurs car elles permettent d'avoir une meilleure qualité et de réduire les interruptions que l'accès sans-fil au mesh. Comme les personnes immergée en réalité virtuelles sont déonnectées de leur corps et les abandonne dans un coin, la plupart des utilisateurs engoncent leur morphs dans un réservoir ou un lit spécial pendant la durée de la connexion. Les salons RV offrent typiquement quelques pods cablés pour permettre aux participants de se brancher physiquement. Beaucoup d'habitats ont également des systèmes cablés utilisé uniquement dans ce but, afin que les utilisateurs puissent vivre une expérience RV depuis le confort de leurs appartements. 

En raison de la distance et des décalages de communication, même les simulsapce en ligne les plus populaires font tourner chaque habitat dans un royaume dédié, limitant les interactions entre les utilisateurs d'habitats différents. La popularité d'univers RV tells que l'Empire Doré, qui prend place dans l'Angleterre de 1880, implique que quelqu'un qui se déplacerait d'un habitat ou monde à un autre peut continuer à jouer au même jeu, même si c'est avec un nouveau groupe de joueurs. 

L'une des caractéristiques inhabituelle des cadres RV est qu"un grand nombre d'infomorphs, incluant des infugiés, jouent à ces jeux. En conséquence, alors que même les joueurs débutant peuvent différencier facilement un personnage joué par une IA d'un autre joué par une personne, il n'y a aucun moyen de savoir si la personne qui joue ce personnage possède un corps physique. 

\subsubsection{Divertissements physique} \label{sec:phys-entert} 

En plus d'une large gamme de divertissement électronique, les gens peuvent toujours profiter de toute une variété de sports physique, allant du football à de nouveaux sports telles que les courses en faible gravité, où les participants s'attachent sur des ailes et se lancent dans des tests de vitesse et d'acrobaties. La capacité à la fois à soigner toute blessure dans une cuve de soin et de supprimer une pile cortical d'un corps morts ou en train de mourir et de la placer dans dans une nouvelle morph a permis l'ascenssion d'une toute nouvelle sorte de sports extrème. En commençant une décennie avant la Chute, différents individus commencèrent à réaliser que, à moins de circonstances exceptionnelles, ils ne pouvaient pas mourir à moins de le vouloir. Cela a déclenché une mode passagère pour les sports extrêmes ainsi que quelque riches hobbyiste du suicide, qui tuaient de manière répétée leur morph de manière inhabituelle. La Chute et la mort permanente de plus de quatre vingt dix pourcents de la population a grandement réduit l'intérêt de jouer avec la mort pendant quelques années. Se tuer juste pour ressentir la mort est considéré comme étant au moins de mauvais goït, et beaucoup de personnes croient que de telles actions ridiculisent la masse de morts de la Chute. Bien que l'intérêt de risquer la mort dans le cadre d'un loisir soit un intérêt croissant, le suicide délibéré reste un hobby excentrique et douteux. 

Dans certaines sous-culture, le duel a été remis au goût du jour depusi presque une décennie. Les couteaux, épées et les pistolets tirant une seule balle de plom sont tous des choix populaires, car aucune de ces armes ne pose de menace sérieuse aux piles corticale et la plupart ne tuent pas instanténament une personne blessée par elles. Il y a cependant des options plus exotiques, incluant les duels aériens avec des ultralégers équipés de lames sur leus ailes. Dans de rares cas, les duels se déroulent dans l'espace, les participants étant équipés de combi spatiale non-protégées. Certains groupes criminels font de l'argent avec des circuits de duels illégaux, faisant s'affronter biomorphs, robots et élevés. Les circuits les plus minables organisent des combats d'arènes répugnant mettant en scène des backups illégalement acquis et incarnés dans des animaux non conscients, souvent équippé de cybernétique léthale. De telles créatures sont généralement relativement cinglées. 

De nombreux sports dangereux non-combatif sont également très populaires. Les plus hauts niveaux de compétitions d'escalade sur Mars sont régulièrement organisés sans équipement de sécurité. Il y a des compétitions d'escalade similaire dans beaucoup d'habitats en utilisant aussi bien des murs d'escalade artificiels que des comptéitions d'escalade libre à travers à peu près n'improte quelle ville ou habitats. Il y a aussi toute une catégorie de sport, incluant le plongeon et le parachute, dans laquelle la perfection d'exécution est vue comme un but bien plus important qu'éviter d'être blessé ou tué. En conséquence, les records actuels de plongeon pour les morphs non modifiés spécifiquement pour résister aux impacts sont tenus par des individus qui soit doivent passer du temps en cuve de soin soit doivent se réincarner immédiatement après avoir battu le précédent record. 

\section{Pouvoir et politiques} \label{sec:politics-power} 

La politique est tout aussi importante dans les colonies éparpillée dans le système solaire qu'elle ne l'était autrefois sur Terre, mais elle est également radicalement différente. Chaque habitat, ou groupe de station, est une entité politqiue séparée, et beaucoup de ces habitats sont violemment indépendant. Le seul endroit où de grandes entitées politiques peuvent exister sont les mondes partiellement habitables de Mars et d'Europe, la population d'Europe étant significativement plus petite que celle de nombreuses cités pré-Chute sur Terre. 

\subsection{Le système intérieur} \label{sec:inner-system} 

Bien que les nations n'existent plus, elles ont été remplacés par de nouvelles entités économico-politique qui sont bien engagée sur la voie de la domination, même si la Chute ne s'était pas produite: les hypercorps. Alors qu'il y a de nombreux habitats et abris indépendants dans le système intérieur, il reste largement dominé par les hypercorps. Pour réduuire les confits entre  eux et promouyvoir la survie de la transhumanité, quelques hypercrops ont formé une alliance connues sous le nom de Consortium Planétaire. Cette alliance gouverne la plupart de Mars et s'occupe du projet actuel de terraformation de Mars. Il contrœle également plpusieurs douzaine d'habitats et beaucoup de bases Lunaire, princiaplement ceux qui sont impliqués d'une manière ou d'une autre dans l'effort massif de terraformation de Mars. 

Puisque Mars est devenue le foyer de plus de quarante pourcent de la population transhumaine survivante, la plupart de la population humaine vit sous le règne des hypercorps du Consortium Planétaire. Juste après la Chute, les hypercorps établirent trois objectifs importants: reconstruire le système solaire, se protéger de toute autre attaque (par les TITANs ou par n'importe quoi d'autre), et croitre en richesse et en pouvoir. Par extension, elur deuxième objectif implique qu'ils protègent également les personnes vivant dans les habitats et les abris contre toute répétition de la Chute. Les hypercrops et le Consortium Planétaire sont excessivement compétent pour atteindre ces objectifs. Depuis que la révolte populaire et les dissenssion générale n'aident pas à atteindre ces objctifs, les hypercorps sont également adeptes à rendre certain les habitants des habitats et abris qu'elles contrôllent qu'ils sont en sécurité, relativement content et, idéalement, incapabel de causer le moindre problème grave. 

En tant qu'entité la plus grosse et le mieux organisée dans le système solaire, les hypercorps, et particulièrement, le Consortium Planétaire, sont en excellente position pour protéger les personnes qui vivent dans leurs habitats et abris. Cette protection vient cependant au prix de la liberté. Vivant dans des habitats qui ont une économie transitionnelle (p. 61), Les habitants des abris contrôllées par les hypercorps sont relativement bien portant et ne craignent ni la famine, ni la satisfaction de leurs envies les plus sérieuses. Les hypercorps s'opposent également fortement au bioconservatisme, et donc quiconque peut s'offrir diverses augmentations ou morphs est libre de les obtenir, tant que ces augmentations ou morphs ne sont pas équipées de système d'arme pouvant être utilisés pour enommager l'habitat ou blesser une grande partie de ses habitants. En échange de la sécurité et d'une prospérité relative, les habitants abandonnent cependant leur capacité à critiquer sérieusment les hypercorps du Consortium Planétaire. 

\subsubsection{La puissance des hypercorps et du consortium planétaire} \label{sec:power-hyperc-plan} 

Les hypercorps et le Consortium Planétaire associé sont la seule entités politique majeure et globale dans le système solaire (avec l'exception possible de l'Alliance Autonomiste, qui est plus un pacte d'asssitance mutuelle qu'une entité politique unifiée). Toutes les autres entités politiques sont basées sur une localisation spécifique unique. Les différentes hypercrops transcendent cependant la localisation. Ils ont des bureaux et es branches dans tout el système, répondant aux besoins des gens de Pluton à Mercure, et dans tout les lieux entre les deux. Alors que la plupart des hypercorps ont de grosse installation de traitement et de fabrication sur Mercure et sur Vénus, transformant les quantités abondantes d'énergies de la premièreet de la chimie complexe de la seconde, la plupart du travailr éalisé par toutes els hypercorps implique le dévleoppement de nouvelles technologies et de nouveau patron pour les machines d'abondances, les deux pouvant être réalisé à n'importe quel endroit disposant d'un accès au mesh. 

En plus des bases sur Mercure, vénus et d'autres lieux riches en ressources, toutes le shypercorps entretiennent des stations dédiées à la recherche et à la fabrication éparpillées partout dans le système solaire. Les usines bien connues incluent les gigantesques chantiers navaux de Starware, les plus grands d'entre eux étant localisés sur la Lune et sur l'astéroïde Vesta, et les énormes usines d'antimatière d'Omnicor orbitent autour de Mercure. Il y a de nombreux autres usines moins connues, incluant des mines automatisées que le mystérieux Group Zrbny maintient dans le principale ceinture d'astéroïde et dans les anneaux de Saturne, et l'usine de qubits Nimbus maintenue en orbit autour de Mars. 

Il y a même un nombre encore plus grand d'installations de recherche sécurisée et souvent secrète, certaines d'entre elles étant tellement bien cachées qu'elles ne sont normalement accessible que par des connexions d'egocast hautement sécurisées. Tout type de recherche mystérieuse et hautement dangereuses ont lieu dans de tels endroits, allant des expérimentations avec les reliques des TITANs aux tentatives de crééer de la nanotechnologie auto-réplicante ou des trous noirs miniatures. Les vids et jeuxvids sont remplis d'histoire de désastrres étranges dans de telles stations de recherche ou d'éhroïques voleurs leur dérobants des mystères surprenants. Alors que la réalité de telles bases de recherche corporatiste est normallement baucoup plus prosaïque, des choses exceptionnelels sont parfois créées - et ont été des désastres occasionnels, impliquant souvent des reliques des TITANs. 

Quelques quartiers généraux corporatistes sont sécurisé et agrdés secret de manière similaire, incluant le siège du légendaire Groupe Zrbny. Il y a énormément de rumeurs et d'histoire à propos de tels endroits. Des espions intrépides, des voleurs et des reporters tentent régulièrement d'obtenir un accès à ces installations, généralement sans succès. Beaucoup de ces tentatives, particulièrement celles faites par des apprentis voleurs et espions, se soldent par des conséquences particulièrement négative, incluant la mort temporaire (et parfois permanente) du voleur. 

Les hypercorps possèdent et gèrent également beaucoup d'habitats. Beaucoup d'entre eux sont le foyer principal des employés hypercorps, mais au moins la moitié de la population d'une grande part de ces habitats sont des résidents ordinaires dus ystèlme solaire qui habitent simplement là. Bien que moins réglementés que les installations de recherche et de fabrication, ces colonnies sont également sujettes à plus de réglementation et de sécurité que certains des habitats controllés par les autonomistes aux limites du système solaire. 

Ces stations sont des endroits dans lesquelles la vie est extrêmement sécurisée. Les résidents ont accès à tous les derniers produits fabriqués par l'hypercorp gouvernante et ses alliés corporatistes. Les habitats hypercorps possèdent tous soit leurs propre companies de sécurité ou possède uen forme de contrat de défense avec une société de sécurité privée, typiquement Action Directe ou le Medusan Shield, qui acceptent de protéger les habitants contre les menaces potentielles des agents des TITANs, des saboteurs fanatiques et des autres menaces. 

Ces mêmes forces de sécurité protègent également contre les menaces envers leurs intérêts. Dans la plupart de ces habitats, les résident ont la liberté d'expression et d'auto-détermination. Cependant, toutes les menaces potentielles contre l'hypercorp et son personnel, allant d'une tentative de sabotage à la simple désobéissance civile, reçoivent une réponse brutale, les infractions majeures ayant pour résultat la contraction forcée et parfois l'édition mentale forcée (voir Psychochirurgie p. 229). Pratiquement ctout ces habitats utilisent une économie transitionnelle (p. 61) et la plupart des résidents ont un niveau de vie élevé en compensation des limites comportementales. Beaucoup d'habitants de colonnies plus indépendantes dans la ceinture ou dans le système extérieur se plaignent de la nature répressive des habitats contrôllés par les hypercorps, mais les habitants de ces habitats préfèrent la sureté et la sécurité qu'ils ont à l'intimidante liberté du système extérieur. 

Pour aider à réduire la dissenssion, les résidents des abris et des habitats contrôllés par le Consortium Planétaire aussi bien que ceux contrôllés par les hypercorps peuvent voter sur un large panel de problème. Le résulatat de ces votes reste cependant liés aux sujets qui ne sont pas considérés comme "problématique de survie de l'habitat", "politique corporatiste" ou "problème liés à la sécurité," qui incluent effectivement tous les sujets liés à la sécurité, au profit et à la productivité des hypercorps impliquées. Les votes sur ces problèmes sont utilisés dans un but de conseil, signifiant qu'ils sont simplement ignorés quand le résultat de ces votes ne correspond pas aux plans des hypercorps. 

Alors que les résidents de ces abris peuvent voter sur l'ajout d'un nouveau jour férié en honeur d'un personnage important ou de l'emplacement et de la conception d'un nouveau parc, les lois réglementants la contraction, la sécurité des habitats, les forces de l'ordre ou d'autre éléments important restent sous le contrôle des hypercorps. Cela ne signifie pas que le résultat de ces élections est complètement ignoré. Si plus de deux tiers de la population supportent fortement un sujet particulier, le Consortium ou l'hypercorp contrôllant l'habitat trouve égnéralement un moyen de modifier leur politique actuelle pour répondre à ces inquiétudes sans blesser leurs propres intérêts. Par opposition, si une petite part des résidents sont contrariés par certaines réglementation, alors leurs souhaits sont ignorés et les forces de sécurité de l'habitat gardent un œil ouvert pour toute désobéissance civile ou d'autre formes de résistance. 

En plus de ces installations dédiés et de ces habitats contrôlés par des hypercorp, beaucoup d'hypercrop ont des bureau dans d'autres statiosn et abris planétaires. Presqque tous les habitats ont un bureau de Nimbus équipé d'un farcasteur et, dans le cas des habitats les plus gros, des communicateurs QE pour les communicatiosn instantannées. Les deux installatiosn sont ouvertes à quiconque accepte de payer le prix fixé par Nimbus. Écologène, Skinaesthesia et d'autre hypercorp ont également des bureaux dans beaucoup d'habitats. Chaque habitat désirant inetragir avec le reste de la transhumanité possède au moins un noeud média automatisé d'Expéria. Dans les plus petits habitats, ces bureaux sont discrets et gérés par des IAs limités ou des infomorphs contractées. L'existenc de ces bureaux est cependant une nécessité vitale pour le bnoheur permanent et l'existance de la transhumanité. Beaucoup d'hypercorps emploient des personens dans tout les plus grands habitats et dans la plupart des plus petits. 

En raison du grand nombre d'infugiés restant, la plupart des noeud de medai Experia sont gérés par des infomorphs contractées. Ces infomorphs supervisent les IAs de recherche de nouvelles locales et gardent une trace de tout développement intéressant. Elles servent aussi de reporters sur site pour tout évènement important qui pourrait arriver. Comme se retrouver posté dans les petits habitats est plutôt ennuyeux, les infomorphs ont habituellement un contrat leur guarantissant une morph de leur choix et la réincarnation dans l'habitat de leur choix en échange d'une période de service, qui dure de trois à cinq ans en général. 

De manière similaire, tout mes habitats, à l'exception de splus petits, ont des bureaux de Medusan Shield ou d'Action Directe, dans lesquels les individus peuvent engager des consultants en sécurité et des guardes du corps, allant de la simple IA aux mercenaires hautement entraînés dans des morphs fury entièrement équipée. Ces mercenaires vivent sur la station et louent régulièrement des contractanst à court terme pour les aider avec des assignations particulièrement longue ou difficile. Des mercenaires compétents peuvent éventuellement être embauché à plein temps par Medusan Shield ou Action Directe, mais comme les contractants reçoivent générallement les assignations les plus dangereuses et ingrates, beaucoup d'entre eux perdent tout intérêt dans le travail contractés auprès des hypercorps. 

D'autres employées travaillant hors des bureaux locaux des hypercorps incluent les concepteurs d'écosystèmes aux scientifiques à louer en passant par les conseillers financiers personnels pour les plus riches et les plus pusisants. Dans les habitats important et les abris planétaires, à peu près vingt pourcent de la population consiste en des employés hypercorps ou comme contractant privés qui sont loués sur de sbases à court-terme lorsque la chareg de travail dépasse la capacité du staff habituel. Ces employés hypercrops sont dans la position unique d'avoir une loyauté double - à leur habitat et à leur hypercrop. En dépit de ce que prèche la propagande hyeprcorp, les deux intérêts ne se recoupent pas nécessairement. 

En raison des délais nécessaires aux communications normales, les chefs des antennes locales du'ne hypercorp ont généralement beaucoup d'autonomie, puisque demander des instructions à leur supérieur sur un autre habitat ou dans une autre installation nécessite soit de faire avec un décalage temporel ou d'utilsier les qubits trés cher nécessaire aux communications QE instantanées. En conséquence, à l'exception des problèmes les plus graves ou le splus difficiles, les directeurs locaux décident seuls des sujets principaux, signalant toute décision ihnabituel ou potentiellement problématique aprsè coup. 

\subsection{Le système extérieur} \label{sec:outer-system} 

Au delà de l'orbite Martienne, l'influence des hypercorps et du Consortium Planétaire est bien plus limitée. À l'exception du gouvernement rigidement autoritaire de la République Jovienne, les habitants du système extérieur ont bien plus de liberté que ceux qui vivent dans l'intérieur. Cependant, mais là-bas la lutte entre le désir d'être libre et les attentes sécuritaires forment une part importante du discours politique. 

\subsubsection{L'héritage des utopiens et des libertaires} \label{sec:libert-utop-legac} 

Diverses formes d'anarchisme et d'idéologies libertaire similaires sont relativement commune parmi les premiers ranshumains qui ont colonisé l'espace deux décennie avant la Chute. Beaucoup d'abris dans le système extérieur ont bénéficié de cet héritage de liberté. Les nouvelles frontières ouvertes par la clonisation spatiale présentèrent une opportunité fantastique pour ceux qui avaitn une envie forte d'éviter l'autoritarisme du système interne et de la Terre contrôlés par les hypercop afin de développer des organisations sociales plus basés sur l'égalité et l'action collective, ou même de simplement expérimenter de nouveaux modèles sociaux. Au delà de la ceinture, l'influence hypercorporatiste était faible et préoccupante, donnait aux colons pleins de ressources  une chance d'explorer leurs intérêts sans contrainte. Les éléments les plus radicaux ont grandit dans ou maintenus des liens avec les mouvements progressistes, radicaux et socialistes ainsi qu'avec les insurgés sur Terre, demandant du soutien où ils pouvaient. D'autres volèrent simplement les ressources des hypercorps du système interne, les acheminant vers leurs projets secrets. Dans certains cas, des vaisseaux ou des stations netières se mutinèrent, refusant les ordres corporatistes et poursuivant leur propre chemin. Il était rarement faisable pour les hypercorps de poursuivre et de punir de telles subversion. 

Même parmis ces libertaires, des différences existaient, et ceux qui adhèraient à des tendences sociao-politiques similaires ont eu tendance à se regrouper. Au fil du temps ils ont développés quatre groupes principaes: les anarchistes de Locus, les techno-socialistes de Titan, les anarchocapitalistes et mutualistes d'Extropia et les sociétés nomades et libre des racailles individualistes. Ces factions forment une alliance souple, un front unis contre les hypercorps et la République Jovienne - ou, comme ils l'appelle, la Junte Jovienne - et un pacte d'aide et de soutien mutuel, appelé l'Alliance Autonomiste. 

Parmi les habitats les plus libertaire, la doctrine vieille de plusieurs siècles "De chacun selon ses moyens à chacun selon ses besoins" est une philosophie de vie et vitale. La disponibilité effective des machines d'abondances assure la disparition du besoin, et l'utilisation d'un système de réputation encourage les gens à être des participants actif du bien commun. L'accès équitable au morphs et aux augmentations est également disponible pour les résidents, bien que la demande de tant d'informophs à la recherche d'un corps signifie que les infugiés doivent contribuer et construire un capital social. Cependant, même pour une infomorph, egocaster à travers el système solaire est cher, et le Consortium Planétaire produits énormément de propagande sur les dangers de ces habitats et décourage les infugiés d'envisager l'évasion. 

Beaucoup d'autonomistes se considèrent engagés dans un conflit idéologique avec le système intérieur, une guerre froide qui résulte parfois en des actiosn physiques. Certains participent activement à des campagnes de sabotages et à des actiosn subversive contre les hypercorps et d'autres puissance autoritaire, telles que la contrebande de machine d'abondance dans les habitats où de telels amchines sont strictement régulées comme ceux de la République Jovienne. Les hyeprcorps et leurs alliées répliquent occasionnellement, bien que le conflit ouvert soit rare. Même si le systèle inrérieur et la République Jovienne peuvent déployer suffisament de puissance militaire pour asservir les factiosn autonomistes, une détente génée existe. Les rumeurs sur le fait que les anarchistes auraient des cartes en mains qui gardent leurs adversaires à distance abondent, peut-être même des menaces de destruction mutuelle. 

Les préoccupations vis à vis de la sécurité et des futures attaques potentielles des TITANs impactent également la politique dans le sytème extérieur, mais la plupart des personnes résistent aux tentatives de restreindre leurs libertés personnellesd'une manière qui n'est directement en lien avec leur sécurité. Les habitants des systèmes extérieurs se souviennent toujours des demandes de soutien au bioconservatisme des anciens gouvermnements et les allégeances à des chefs distants et qui ne réagissent pas et qui n'ont rien fait pour prévenir la Chute, et ces souvenirs alimentent leur manque de confiance dans ces états. Ces puissances ont étés défaites en échouant à faire ce qu'ils avaient promis - lorsqu'ils ne pouvaient pas apporter la sécurité qu'ils prétendaient amener par leur mesures autoritaires, les germes de la défaite dans les ssytèmes extérieurs ont été semées. 

\subsubsection{L'espace pour expérimenter} \label{sec:space-exper} 

Les expérimentations sociale et politique sont communes dans beaucoup des petits habitats du système extérieur. La prise de décision collective est relativement simple dans les statsions avec une populations de moins de dix mille personne, la démocratie directe est donc une méthode de gouvernement commune. Grand nombre d'habitats basés sur une idéologie ont utilisé cette simplicité à prendre des décisions comme méthode pour mettre tout le monde d'accord pour une forme de houvernement inhabituelle. 

Les variantes individuelles qui ont été essayées sont trop nombreuses pour être listées, bein qu'elles tombent généralement dans une catégorie plus large. Quelques habitats relativement petits utilisent des formes limitées d'autoritarisme. Certains ont un chef unique qui a beaucoup de pouvoir, mais qui est (idéalement) tenu à l'égard des abus ou des excès par l'utilisation de limites telles qu'une liste de droits constitutionnels ou la capacité pour un nombre relativement réduit de personne à demander une élection ou un vote de confiance. Certaines colonies utilisant ce modèle ont élus des dicateurs qui reçoivent un mandat limité dans le temps, alors que d'autre sont régies par un seul chef charismatique qui transforment leur habitat en un culte de leur personnalité. 

D'autres habitats choisissent lerus chefs en le tirant au hasard, tout les adultes qui réussissent un test relativement simple de compétence sont éligible pour être le chef de la colonnie pour une période de six mois à cinq ans. Un petit nombre d'habitats sont gouvernés par de puissantes IAs spécialisés, qui sont dans de rares cas de réelles AGIs hyper-intelligente ou même des IAs germes qua la colonnie a créée en secret. Plusieurs colonnie peuplée uniquement par des infomorphs ou des habitants en synthmorphs usent des connexions à forte bande passante spécialisée pour donner à leur membre l'accès aux pensées de surface et aux réactions émotionnelles des autres, leur permettant de tenir de vaste débat politique dans lesquels tout ceux qui sont présents peuvent sentir les réactions émotionnelle générales de tous les autres membres aussi simplement qu'ils peuvent ressentir les leurs. 

Il y a une grande quantité de types de gouvernement différent, beaucoup d'entre eux n'ont jamais existé auparavant, avançant (et parfois trébuchant) bien en avance dans le système solaire. Certains marchent beaucoup mieux que d'autres, permettant aux colonie de survivre avec réussite et de tranformer la plus gardne aprtie du système extérieur un laboratoire politique vaste et complexe. 

\section{Maintenir la paix} \label{sec:keeping-peace} 

Chaque habitat est responsable de la gestion de ses affaires internes. En conséquence, les standards de justice varient grandement de l'tat policier oppressif de la Junte Jovienne au trbiunaux judiciaire du marché libre des Extropiens dans la ceinture en passant par les politiques de justice communautaire des anarchistes au-delà de Saturne. Les voyageurs sont trés foretement encouragés à se mettre au courant des lois en vigueurs sur les statsions qu'ils vont visiter afin d'éviter des incidents malheureux, bien que les muses sont générallement plutôt bonne pour se tenir au courant des conditions locales afin qu'elles puissent prévenir leurs utilisateurs avant de pénétrer dans les zones grises ou illégales. 

Dans le système intérieur, les standards de justice et de maintien de l'ordre tendent à être uniformisée et relativement familier à la majorité des populations qui ont vécues sur Terre avant la Chute, là où la plupart des nations avaient des standards de justice relativement similaires. A travers tout le système solaire, certains standards identiques peuvent être trouvés. bien que les lois locales puissent différer, il y a un respect répandu que l'idée de punition pour les lois religieuse ou basées sur les idéologies ne s'appliquent qu'aux résidents. Les visteurs qui violent de telles restrictions ou d'autres lois mineures sont généralement déportés chez eux et sont interdits de revenir. Des standards de preuves pour les enquètes criminelles sont également communs. Les technologies d'analyse médico-légales rendent la collecte et l'analyse de l'ADN et d'autres traces sont un procédé extrêmement rapide et facile. De manière similaire, avec la plupart des habitats ayant une surveillance totale de tous les lieux publics, tous les délits potentiels commis là peuvent être minutieusement analysées. 

les standards de vie privée varie grandement d'un habitat à l'autre, donc pendant les urgences et les enquètes criminelles, les agents des forces de l'ordre peuvent avoir ou pas un accès total à des enregsitrements des évènemesnt dans toute partie de l'habitat incluant les enregistrement des capteurs dans les résidences privées. Dans quelques stations, les agents des forces de l'ordre peuvent contraindre toute personne qui pourrait avoir été présente pendant un crime à fournir des copies de leurs expériences sensorielles à l'heure du crime. Alors que les individus peuvent éditer leurs souvenirs, des différences entre les entrées sensorielles de diverses personne sont juste une autre forme de preuve. Exiger des téléchzrgement sensoriels des témoins et des suspects est une pratique habituelle dans les habitats contrôlés par le Consortium Planétaire, la République Jovienne et la plupart des hypercorps. Mains dans la plupart des habitats du système extérieur, les agents des forces de l'ordre n'ont pas accè à de tels enregistrements et ne peuvent contraindre que les personnes qui ont déjà été accusée de crime grave à leur transmettre les enregistrementes sensoriels. 

La force des sciences médico-légale moderne est telle qu'un examin suffisament détaillé des personnes et des lieux peut souvent déterminer la nature d'un crime et le(s) auteur(s) relativement facilement. Déterminer l'innocence ou la culapbilité dépend rarement sur des suppositions, des preuves circonstancielles, des témoignages de témoins occulaire ou d'autres formes d preuves notoirement pas fiable et commune lors des siècles passés. La meilleure façon pour quelqu'un d'éviter d'être accusé d'un crime est soit de faire en sorte que personne n'apprenne le crime ou de s'assurer que personne en le suspecte comme acteur du crime. Une fois que le coupable d'un crime devient un suspect, il y a une chance significative que le agents des forces de l'ordre soient capable de découvrir une preuve fiable le connectant au crime. Cependant, si il n'existe pas de preuve évidente connectant un suspect spécifique à un crime, le criminel à une plus grande chance de ne pas être découvert. 

\subsection{Police} \label{sec:law-enforcement} 

Les forces de police du système solaire sont constitué d'un vaste patchwork de juridictions séparées, occasionnellement unies par différent traités. La plupart des habitats ont signé le Traité de Sécurité Uniformisé qui requiert soit l'extradition soit un procès sur place de criminels accusé de crime grave et bien spécifiques tels que la tentative de destruction d'un habitat, l'utilisation d'infoware incapacitant (incluant les attaques basées sur les exploit basilique) ou toute tentative d'aider les agents des TITANs à prendre le contrôle ou à détruire un habitat. Seuls la République Jovienne et quelque habiotats particulièrement antisociaux ou anarchique n'ont pas signé ce traité, mais la plupart des habitats du système extérieur maintiennent le droit de juger eux-mêmes les personnes accusées par d'autres habitats plutôt que des les extrader. De plus, la plupart des habitats demandent une quantité significative de preuve avant qu'ils n'acceptent d'extrader l'un de leurs résidents. 

Au delà du Traité de Sécurité Uniformisé, il n'existe rien ressemblant de près ou de loin à un code pénal uniformisé ou à une force de police largement reconnue. Chaque habitat ou groupe d'habitat préfère maintenir son propre code de loi et ses propres agents des forces de l'ordre. Dans la plupart des zones, la police est une profession respectée et honorable payée par le gouvernement, mais dans quelques endroits, les seules options sont les agences de sécurité privée qui ne protègent que les individus qui souscrivent leurs services. Parmi les anarchistes et la racaille, les résidents sont trés largement responsables de leur propre proptection, ce qui signifie qu'ils doivent être constament armés lorsqu'ils sont en public (en fonction de la situation locale). En fonction des stations, la seule chose que peut faire quelqu'un qui serait victime d'un crime est de traquer on agresseur ou de poser une prime sur sa tête. Dans d'autres, des mécanismes existent pour résoudre les problèmes de manière communautaire ou collective qui impliquent souvent de constituer un groupe adéquat de pairs pour évaluer la situation, offrir un jugement non biaisé et parfois poursuivre une action collective. 

Les seuls agents des forces de l'ordre largement acceptés et qui tentent de maintenir une juridiction à travers le système solaire sont les enquèteurs privés et les consultants en sécurité des sociétés telles que medusan Shield ou Action Directe. Les deux organisations ont des contrats avec différentes hypercorps et stations du système solaire intérieur pour leur fournir de la sécurité. Dans le système extérieur et d'autres régions non contrôllées (directement ou indirectement) par les hypercorps, le statut de ces agents est cependant bien plus fragile. Dans les habitats qui n'ont pas de contrats de sécurité avec leurs organsiation, le mieux que peuvent faire ces agents est d'agir en tant que chasseur de prime. 

En raison des nombreuses histoire à propos des excès dans le système intérieur, beaucoup de colonnies rechigne à utiliser des chasseurs de primes indépendant - souvent appelés chasseur d'égo - et peuvent aller jusqu'à les bannir. D'autres autorisent les agents des hypercorps de sécurité autorisées à agir comme chasseur d'égo, mais leur interdisent d'extrader ou de restreindre et punir les criminels qu'ils pourchassent. À la place, les agents doivent fournir les preuves pour que le système juridique de l'habitat puisse établir un procès, aauquel cas une personne accusée peut-être mis en détention provisoire sous la garde de l'agent. Les agents des forces de police rencontre des difficultés similaires lorsqu'il tentent d'apréhender un suspect qui s'est enfui dans un autre habitat. 

Des habitats alliés dans le système extérieur accorde généralement un pouvoir légal complet ou limité aux agents des forces de l'ordre en visite et à leurs alliés. Il existe aussi diverses organisation de sécurité privée qui travail en lien étroit avec les forces de l'ordre locales pour fournir une sécurité inter-habitat entre divers habitats qui ne sont aps forcément des alliés proches. Les membres de ces organisations tentent de maintenir une rep suffisament élevée pour gagner le respect de tout les habitats avec lesquels ils travaillent. Ils agissent à la fois comme chasseur de prime et comme enquèteurs neutre dans les situatiosn qui implique les lois de plusieurs habitats. Toutes ces entreprises de sécurité sont localisées dans le système extérieur, et aucune n'a de juridiction étendue au-delà d'une localisation relativement limitée, comme le milieu de la ceinturne ou le système Saturnien. Toutes organisation de ce type qui essaye de se développer entre en compétition directe avec Medusan Shield ou Action Directe et sont en conséquence achetées ou sapée et discréditée par l'une des deux organisations. 

Il y a aussi plusieurs chasseurs de prime et enquèteurs privés, certains d'entre eux étant extrêmement fiable. D'autres sont connus pour leur éthique et leur sens moral extrêmement flexibles, particulièrement si la paie est suffisament élevée. Dans certaines des stations autonomistes et des vaisseaux de la racaille, ces contractants privés peuvent être engagés pour simplement embarquer et enelver ou exécuter un résident du moment que cette personne à une rep suffisament basse. tenter d'enelevr ou de tuer un membre respecté de la communauté attire cependant rapidement l'ire de tout l'habitat. Les diverses petites organisation de sécurité privées du système extéireur peuvent parfois poursuivre un sujets dans des habitats contrôllées par els différentes hypercorps ou le Consortium Planétaire. Cela nécessite une vérification des antécadents, des contrôle de sécurité et souvent un paiement relativement élevé. 

\subsection{Punition} \label{sec:punishment} 

Parmi les colonnies autonomistes, l'exil forcé ou le remboursement de la victime avec un montant équitable de marchandise ou de travail sont les principales condamnation à l'exception des crimes le splus odieux (tels que la  tentative de meurtre en masse, la destruction d'habitat, la tentative de création d'IAs germe ou d'autres actiosn extrème similaire). Dans les habitats anarchistes et collectivistes, les comportements antisociaux impliquent égénrallement l'expulsion ou une pénalité réputationnelle, bein que des solutions impliquant le système d'amendes sont souvent utilisés pour des punitions standards. À l'autre extrémité du spectre, les personnes accusées de crimes encore plus sérieux dans les habitats les plus violents et sans lois sont exécutées et leurs sauvegardes connues sont détruites. Dans beaucoup d'autres, les crimes excessivement grave sont généralement résolu en donnant au criminel le choix d'un upload forcé dans un ordinateur fermé mais ressmeblant à un humain ou l'obligation de subir des modifications de personnalitées - en partant du principe que le criminel n'ait pas été simplement tué avant d'être traduit en justice (de tels meurtres sont généralement considérés comme de l'autodéfense). Les modification de personnalité obligatoire sont généralement limitées à l'absolu minimum nécessaire à prévenir la personne de répéter des crimes similaires. 

De l'autre côté, les punitions dans les abris et habitats hypercorps ou controllés par le Copnsortium Planétaire sont généralement des amendes payées soit en monnaie ou en période de travail en tant que contractés forcés allant de quelques mois à plusieurs années. Les crimes violent, particulièrement ceux qui menacent soit un employé important d'une hypercorp ou un habitat, amènent générallement à des modification de personnalité importante. De telles modifications incluent souvent la création d'unprofond sentiment de loyauté et d'obéissance à l'hypercorp. 

Les punitions sont ancore plus draconique dans la République Jovienne, où l'exécution permanente et la destrcution de toutes les sauvegardes est la punition la plus commune pour les crimes commis contre les chefs ou d'important groupes de population. Puique les chefs de la République sont des bioconservateurs convaincu, l'édition de personnalité et l'upload forcé ne sont que rarement utilisés. L'esclavage est trés commun cependant, ainsi que les formes plus classique d'emprisonnement. La République est l'un des derniers endroits dans le système solaire qui ait des prisons physiques. 

La grande majorité des autres habitats sont quelque aprt entre ces deux extrêmes. Les punitions pour les crimes non-violents consitent en des remboursements obligatoires, l'offenseur devant travailler à rembourser sa dette à la ou aux victime(s) ou faire face à des punitions plus sérieuses. Au lieu des la contraction forcée, les offenseurs doivent travailler entre cinq et vingt heures par semaine pour leurs victimes et n'ont besoin de le faire que jusqu'à ce que le crime ait été convenablement réparé. Le remboursement typique équivaut à deux ou trois fois la valeur de la marchandise ou du service prit à la victime. 

\section{L'économie} \label{sec:economy} 

Mis à part les luttes des bandes de primitives pour la survie sur les ruines de la Terre, toute l'humanité a accès à certaines des merveilels de la nanotechnologie. Cet accès est extrèmement variable et les bénéfices économiques qu'il génère peut-être divisé en trois grandes catégories - la vieille économie, l'économie transitionnelle et la nouvelel économie. 

\subsection{La vieille économie} \label{sec:old-economy} 

La vieille économie est essentiellement le même type d'économie que le capitalisme industriel de consommation qui était en place depuis la fin du 19° siècle, un système centré sur les fabricants qui créent des biens matériels et qui les vendent aux consommateurs. Les fabricants modernes fabriquent maintenant leurs marchandises dans des machines d'abondance au lieu d'usines, mais le fonctionnement est le même que celui qui existait les deux siècles précédents. En raison du haut niveau d'inefficacité et d'injustice de ce système économique, la puavreté est extrêmement commune. Les individus les plus paiuvres souffrent de la faim, du manque de soin médicaux n'ont aps de domicile fixe et d'autres terribles problèmes. 

Les membres ordinaires de cette société n'on jamais un accès direct aux machines d'abondance. Ils doivent acheter leurs marchandises aux corporations, aux gouvernements ou aux individus les plus riches qui les contrôllent. Certaines société de la vieille économie ont des économies planifiées, dans lesquelles, les corporations ou l'état détermine quelles options seront proposées aux citoyens ou quelles sont les marchandises qu'ils doivent avoir. D'autres prétendent avoir un marché libre, où les citoyens ont plusieurs options, mais les résidents doivent toujours payer pour obtenir les biens essentiels qui ne coûtent rien à produire aux corporations et aux gouvernements. 

Aujourd'hui, presque personne ne désire vivre dans des sociétés régies par la vieille économie. Seul de très rares individus vont visiter ces sociétés. Le régime oppressif de la République Jovienne possède la plupart des reste des sociétés en ancienne économie du système solaire. Les quelques exemples survivant sont des régimes totalitaires où une élite riche garde un contrôle absolu sur toutes les machines d'abondance et en posséder une à titre privé est un crime extrêmement grave. Comme les machines d'abondance peuvent être utilisée pour créer d'autres machines d'abondance, les maintenir sous un contôle stricte nécessite une vigilance constante. 

Les résidents des sociétés basées sur la vieille économie ont tendance à considérer les résidents des sociétés avec une économie transitionnelle ou une nouvelle économie avec envie, alors que les résidents des habitats qui bénéficient d'une économie transitionnelle ou nouvelle les considèrent avec un mélange d'horreur et de pitié. Depuis la Chute, presque un tiers des habitats basés sur la vieille économie restant ont été transformés leur système économique pour une économie transitionnelle ou nouvelle, souvent par le biais de révolution violente. La plupart des scientifiques sociaux prédisent aue, à moins d'autres catastrophes, toutes les soéciétés de la vieille économie, à l'exce"ption des plus rpéressives, sont quasiment certaines de se transformer en économie transitionnelle d'ici vingt à trente ans. 

Les sociétés de la vieille économie sont unique car la monnaie et le seul moyen d'échange acceptable dans ces sociétés. Alors que les réseaux de réputations existent, ils restent informels et servent comme un moyen non autorisé d'échanger des faveurs. 

\subsection{L'économie transitionnelle} \label{sec:transitional-economy} 

L'économie traditionnelle est un système bien plsus table et facile à maintenir que la vieille économie. Les économies transitionnelles mélange la nouvelle et la vieille économie, et les habitats utilisant ce système bénéficient à al fois de la propriété privée de machines d'abondance ainsi que de faiseurs publics qui sont libre d'accès. Ces machines publiques sont strictement limitée dans les biens qu'elles peuvent produire. De plus, les matériaux brutes de divers biens particuloèrement complexe sont également strictement régulés. Mars, vénus et la Lune sont tous des exemples d'économies transitionnelle, comem la plupart du système intérieur. 

Pour les habitants de l'économie transitionnelle, créer de la nourriture, des habits non-intelligents, des meubles et d'autres objets simples et non-intelligent est une probélamtique triviale. Cependant, les nanofabeurs publics ne peuvent créer que des objets qui soit ne contiennent pas du tout d'électroniques ou soit ne sont composés que de circuits simples qui informe de l'état et de la localisation de l'objet. Créer n'importe lequel de ces éléments ne nécessite pas grand chose de plus que la machine et un stock de carbone, d'hydrogène, d'oxygène, de nitrogène, de silicone, de fer, d'aluminium et de toutes petites quantités d'autres divers matériaux. tout ces matériaux sont suffisament abondants pour permettre que leuur acquisition soit simple et bon marché. 

Utilisant les éléments qui sont librement disponible à tous les citoyens payant leurs impôts, les nanofabeurs peuvent produire une large gamme de marchandises comme d'exquis costumes en soie, des tables avec l'aspect de l'ébène et de l'acajou finement polis, des gobelets en  verre joiemnt colorés ou des tasses en porcelaine peinte. Ils peuvent aussi créer un diner de gourmet et un ensemble d'assietets et de couverts fins avec lesquels manger le repas. Pour payer la petite quantité d'énergie et de ressources nécessaires pour créer ces marchandises, toust les habitants doivent payer une petite taxe. 

Une fois que la taxe d'utilisation a été payée, la nourriture, les vêtements et les biens similaires sont tous gratuits. Les matériaux bruts, les biens dépassés et de seconde-main et non-désirés ainsi que divers déchets sont recyclés en nouveaux biens. Les réisdents de l'économie traditionnelle ne sont jamais affamés de même qu'il ne subissent l'une des privations que la plupart de l'humanité subissait avant le milieu du 21° siècle. De plus, les soisn médicaux de bases sont gratuits dans la plupart des sociétés en économies traditionnelle, pour favoriser l'assurance que la population est en bonne santé, heureuse et productive. 

Alors que la plupart des biens sont disponible gratuitement, il y a aussi des biens que les résidents doivent acheter aux xorporations, à leurs gouvernement ou à d'autres producteurs. Les vêtements et les meubles intelligents qui peuvent changer de forme, de couleur et de motif, en fonction des souhaits de l'utilisateur, ne peuvent pas être fabriqués dans les nanofabeurs personnels. Tout bien fabriqué à partir de matériau composite solide, les batteries, les appareils alimentés électriquement incluant toutes les augmentations ainsi que toute la nanotechnologie doivent être acquis de la même manière. Ces biens sont considérabelement moins communs car ils nécessitent l'accès à des nanofabeurs débridés ainsi qu'à des matériau bruts exotiques. 

Les économies transitionnelle tendent à être des endroits relativement sûr, puisque les habitants ne peuvent pas fabriquer d'armes plus dangereuses qu'un couteau, une masse ou d'autres armes primitives similaire. Toute arme plus sophistique, des armes à feu au fusil à plasma, nécessitent des machines d'abondances et des matériau exotiques pour être fabriqués. La prolifération de ces objets est strictement contrôlée. 

Certains habitats du système extérieur ont des économies traditionnelles car leurs habitants préfèrent la sécurité qui vient avec le contrôle centralisé des technologie potentiellement dangereuses. D'autres habitats ont des économies transitionnels par défaut, car ils n'ont qu'un stock limité de certains des matériau les plus rares recquis pour la fabrication de différ Peu importe les raisons, les habitants extérieurs et issus de la nouvelle économie les considèrent comme des pauvres, alors que beaucoup de résidents de la transtion économique considèrent les sociétés de la nouvelle économies cdomme étant exceptionnellement riche et quelque peu effrayante. 

End épit de ces différences de perception, les deux sociétés économiques ont beaucoup de choses en communs. La nourriture, l'habillement et les biens similaires sont facilement disponible pour tous les résidents. Le status, les goûts, la richesse et la réputation d'un individus sont évalués par le type de vêtements, de nourriture et de meubles qu'il possède. Bien qu'il y ait un grand nombre de modèle pour déifférent type de nourriture et de bien de conosmmation, les concepteurs avant-gardistes développenet de nouveaux modèles tous les mois et utilisent des système de protection de la copie sur ces modèles pour éviter qu'ils ne soient piratés pendant au moins un mois ou deux (et parfois plus). En conséquence, pendant les premiers mois après leur sortie, les seules personnes qui peuvent avoir accès à de nouveaux modèles de vêtements, de vaisselle, de nourriture ou d'autre biens similaires sont ceux qui payent une prime au concepteur pour télécharger le modèle qui permettra à leur machine d'abondance de fabriquer l'objet. 

Puisque l'une des manières de définir une économie transitionnelle est un système dans laquelle la réputation et l'argent sont utilisés largement et simultanément, la plupart des systèmes ont développés une manière de payer avec les deux méthodes. Alors que les résidentns utilisent l'argent principalement pour acheter des marchandises, se produrer des modèles pour les machines d'abondance nécessite de la rep, prticulièrement parmi les résident qui visitent régulièrement des sociétés avec des économies nouvelles ou qui y ont un nombre de contact significatifs. 

\subsection{La nouvelle économie} \label{sec:new-economy} 

Un peu moins de quarante pourcent de la population humaine vie dans l'une des versions de ce que les scientifiques sociaux nomme la nouvelle économie. Dans le système extérieurs, les économies alternatives deviennent de plus en plus rare. Les nouvelles économies sont bien meilleure que les vieilles économies ou l'économie transitionnelle pour pour stabiliser une population décentralisée, ce qui a amené plus de la moitié de tous les habitats et abris à adopter ce modèle. 

Dans les sociétés de la nouvelle économie, les individus peuvent fabriquer et utiliser librement à peu près tout ce qu'ils veulent, du moment qu'ils ont récupérés les bons plans et les matériaux de bases. De fait, les besoins en nourriture, en habillement, en soins médicaux, en accès à l'information et les autres besoins basique des résidentts sont tous facilement satisfaits. Il y a cependant toujous des objets qui nécessitent un travail intense des indivdus qui souhaitent les obtenirs. Bien qu'elles soient définit comme sociétés "post pénurie", certains type de pénurie reste extrêmement réels. 

Dans la plupart des habitats de la nouvelle économie, les biens communs sont dispnible sgratuitement pour tous les résident - ou au moins pour tous les résidents qui répondent à certains critères. Ces critères prennent généralement une des deux formes suivantes: citoyenneté ou travaux publics. Dans les habitats riches et prestigieux, l'accès libre à tout les biens courants est offert au résidentqui ont une la cotoyenneté officielle. La citoyenneté peut être obtenue de diverse façonmais la plus courante implique d'être soit considéré comme un investissement stratégique en raison d'une expertise particulière, soir de rendre un service extrêmement précieux à l'habitat ou de travailler pendant une certaine période de temps pour l'habitat. une fois qu'un individu est un citoyen, l'énergie, l'espace vital et les matériaux bruts qu'ils utilisent dans le cadre de leur vie quotidienne sont disponible librement. 

Dans beaucoup d'habitat collectivisés, on attend des résidents qu'ils prennent à leur charge une contribution aux travaux d'intérêt publics dans l'habaitat, nécessitant typiquement entre quatre et huit heures par semaine. en fonction de la nature de la colonnie, ce travail peut-être choisit par le gouvernement, le syndicat collectif qui s'occupe de la gestion es ressource ou par un individu possédant une grosse rep et qui contrôle l'accès à de grande quantité d'énergie et de matériau de base. À moins que quelqu'un ne possède des compétences particulièrement importantes, ce travail est souvent ennuyeux mais sûr qui peut être accomplit plus facilement par des humains que par des IAs, come par exemple chercher les failles de l'habitats et accomplir les tâches de maintenance. 

En partant du principe qu'un individu ait acquis la citoyenneté ou ait fait sa part de travail pour le bien être collectif de la station, il aura accès à un stock d'énergie et de matériau brut qui lui permettra d'utiliser sa machine d'abondance pour fabriquer ce dont il a besoin. Les visiteurs bénéficient générallement d'un accès aux fabeurs, mais on attend généralement des visiteurs longue durée qu'ils participent à l'habitat si ils ne veulent pas voir leur réputation s'effondrer. 

\subsubsection{Restreindre les technologies dangereuses} \label{sec:restr-dang-techn} 

La plupart des sociétés d'Eclipse Phase ont de bonnes raisons pour restreindre l'accès à certains biens dangereux, et particulièrement le matériel militaire. Peu de personne vivant en habitat scellé et entouré par le vide spatial apprécient l'idée que l'accès à des infections de guerre biologiques ou à des appareils pouvant créer de grands trous dans la coque extérieure de l'habitat soit facilité. Bien que de tels incidents soient relativement rares, les souvenirs d'horreurs comme le récent désastre BransonVesta reste toujours frais. Dans cet incident, un culte biconservatif radical a fabriqué plusieurs bombes à plasma et a accidentellement détruit tout l'habitat lorsque leur attaque sur le gouvernement local à causer une explosion en chaîne, cassant l'habitant tournant en deux. Plus de 50 000 résidents ont du être réincarnés, et 400 sont définitivement morts lorsque leur sauvegarde et leur pile corticale ont été détruits dans l'explosion. Les procédures standard sont de restreindre l'accès et de chiffrer trés lourdement les plans nécessaire à la création d'arme de guerre et d'autres dangers similaires, bien que des individus suffisament motivés puissent éventuellemnt déchiffer ou rétro-ingénieurer de tels schémas. Même les nanofabeurs des habitats anarchistes peuvent être bloqué dans la création de ce genre de chose ou au moins alerter le mesh public local si quelqu'un lui demande de le faire. Les habitats qui ne possèdent presqu'aucune loi vis à vis de la possession de différents objet et appareils ont habituellement des lois contre les armes qui peuvent sérieusement endommager l'habitat. Beaucoup de technologie dangereuse sont spécifiquement conçue pour utiliser divers éléments exceptionnellement rare ou fabriqués par l'homme, incluant des éléments radioactifs et des éléments transuranique artificiellement créés. Beaucoup d'habitat restreigne l'accès à ces éléments pour limiter encore plus la fabrication de ces armes.  Depuis que la détection des éléments radioactifs en utilisant les capteurs environnementaux standard répartit dans les habitats est simple, les autorités peuvent rapidement savoir quand quelqu'un a acquis une quantité importante de ces éléments, ou les attraper si ils tentent de les amener à bord. 

\subsubsection{Valeur et pénurie dans les sociétés de nouvelle économie} \label{sec:value-scarcity-new} 

Alors que les indémnités basiques de citoyenneté couvrenty la plupart des besoins et même certains luxe, l'indémnité a ses limites. Avec les indemenité, les individus reçoivent un quota de bien et d'énergie qu'il peuvent utiliser chaque jour. Cette coutume est extrêmement généreuse par rapport au 20° siècle, permettant aux résidents de créer une douzaine de tenue et leur fournissant de la nourriture pour une demi douzaine de personne par jour. Créer des plats élaborés, les meubles et la vaisselle pour héberger uen fête d'une douzaine de personne est dans les moyens de n'importe qui. Organiser la même fête pour deux cents personnes est cependant hors des limite des indemnités basiques. 

Les individus qui veulent dépasser leur indémnité de citoyenneté peuvent soit utiliser leur rep pour obtenir l'accès à plus de ressource et d'énergie ou mettre leurs ressources en commun avec d'autres pour atteindre leur buts. Il y a beaucoup de biens qui sont remativement complexe à créer - y compris une bonne partie des meilleures morphs et des pièces d'équipement extrêmement spécialisés et complexes comme les augmentations avancées - qui dépassent les ressources disponibles dans les indmenités basiques de citoyenneté. 

Cette indemnité limite également le nombre de voyage que les résident peuvent effectuer facilement. Les résidents de la plupart des habitats sous la nouvelle économie possèdent des combinaisons spatiales de bonne qualité, et beaucoup d'netre eux utilisent leur rep pour créer un petit pod de voyage équipé du strict minimum pour atteindre les habitats les plus proches. Mais même le plus petit vaisseau spatial est bien trop grand et difficile à créer pour être accessible grâce aux indemnités basiques ou grâce au montant de rep que peut réunir un individu ordinaire en un temps raisonnable. 

En plus de l'utilisation de ressource à grande échelle et des marchandises difficile à fabriquer, il y a des marchnadises qui sont intrinsèquement rare, telles que les reliques de la Terre et les biens fait à la main. Alors que des copies exactes d'à peu près tout allant de la Mona Lisa à des marguerites séchées sont facile à acquérir, des reliques authentique de la Terre et des possessions prisées. La vaste majorité des réfugiés n'ont rien pu emmener avec eux, mais presque tout le monde souhaite avoir quelques symbole pour se souvenir de la Terre. Une seule fleur séchée, pièce de monnaie ou morceau de pierre venant de la Terre peuvent être échangés pour à peu près n'importe quelle morph ou autre bien qui est modérément difficile à créer. D'authentique artefacts historiques, tels que le chapeau d'une personne célbère ou un autographe, valent bien plus, autant que les œuvres originales d'artistes célèbres. Il y a deux ans, une des trois dernières peintures de Léonard de vinvi a été échangée contre un grand vaisseau spatial trés bien équipé, et un petit morceau de la Cloche de la Liberté a été échangée contre une morph sur mesure et une villa d'un hectare entièrement équipée dans l'un des habitats les plus prospères orbitant autour de Saturne. 

Bien que moins cher que les reliques de la Terre, les marchandises fait mains sont vendu à un prix élevé et extrêmement demandé par les plus riches. Bien que la plupart des personnes ne puissent faire la différence entre un vin fin élevé dans l'un des vignobles de Mars et une copie du même vin produit en utilisant une machine d'abondance moyenne, certains connaisseurs prétendent qu'ils peuvent sentir la différence. Il y a beaucoup de prestige  gagner en servant le nourriture cultivée artisanalement. Alors que n'importe qui peut boire du vin nanofabriqué, le vin artisanal reste un produit rare qui ne peut être apprécié que par une élite, et impose donc un prix modérément élevé. Dans presque tous les cas, les produits artisanaux sont chers à cause de leur rareté et parceque beaucoup de personne apprécient le statut associé à leur possession et à leur utilisation. 

Il y a trois autres éléments qui sont rare et donc relativement cher: l'espace vital, la main d'œuvre qualifié consciente et la nouveauté. La majorité de l'humanité vit dans des unités résidentielle de taille standard, d'un volume allant de cent mètres cubes sur les plus petits ou plus pauvres 

habitats à deux cent mètres cubes sur les habitats riches et prospères. Comme chaque mètre cube d'un habitat doit être fabriqué et que le processus de fabrication ou d'extension d'un habitat est loin d'être simple, l'espace est au plus cher. Les seules exceptions à cette rareté est sur Europe et sur Mars, qui peuvent être habitée par des morphs correctement adaptées sans avoir besoin de système de survie complexe ou sans que le danger du vide spatial ne sit présent derrière chaque mur extérieur. Posséder un grand espace résidentiel dans les habitats représente une grande quantité de richesse, et les plus grandes villas et astéroïdes privés sont des luxes possédés uniquement par les individus éyant les rep les plus élevées. 

Alors que la main d'œuvre transhumaine est devenue relativement bon marché en raison du nombre d'unfugiés qui doivent vendre leurs services ou se contracter pour obtenir une morph et de l'espace dans un habitat, la main d'œuvre qualifiée est bien plus chère. Acheter la conception d'une morph unique, par exemple, fabriquée par un biogénéticien compéten, peut coûter jusqu'au prix d'un petit vaisseau en fonction de l'écart entre la morph et le modèle standard. Le même principe s'applique pour tout ce qui est conçu sur mesure des vêtements aux éléments extrêmements spécifiques de technologie conçus pour un usage spécifique et unique. Alors que la fabrication en soi de ces objets n'est pas plus chère que n'importe lequel des objets similaires, le temps et l'effort nécessaire à la conception les rendent excessivement chers. 

La dernière denrée à être à la fois cher et rare est la nouveauté. Alros que tout le monde peut déguster un vin fin ou porter une large gamme d'habits de concepteurs, d'autres marchandises sont gardées délibérément rare. La mode d'avant-garde, la nouveauté musicale et même la haute bouffe (concept d'aliments exclusif et audacieux) sont plus difficile à trouver car les plans nécessaires à les fabriquer sont chiffré et ne peuvent pas être copiés. La protection contre la copie utilisés sur les plans des biens nouvellement créés expire automatiquement dans les trois ans au plus tard, et la plupart des habitats réduisent ce délai à un an. De plus, cette protection n'est jamais parfaite; quelqu'un réussit toujours à créer une version piratées de ces nouveaux biens en deux à six mois. Cependant, entre le moment ou le schéma est créé et le moment où quelqu'un le pirate, ces objets ne sont disponibles qu'aux individus qui acceptent et qui peuvent en payer le prix. Les nouveaux schémas populaires réclament un bon prix dans la nouvelle économie, et un grand nombre de transhumains vivent de la conception et de la vente de tels plans. 

\subsubsection{Biens non reproductible} \label{sec:irreproducible-goods} 

A une époque où les matériau numériques sont facilement copiés et où les biens physiques sont reproductivle avec la nanofabrication, des concepts tels que le droit d'auteur, les amrques déposées et la propriété intellectuelle sont engagés dans une guerre perdue d'avance. En dépit des meilleures méthodes de chiffrement, de DRM ou d'autre mesrue anti-piratages équivalentes, très peu de produits échappent à la piraterie. Les cas de copies/plans de nouveaux biens partagés sur des réseaux pirates avant même leur sortie officielle sont fréquents. 

La réponse de certains fabricants, concepteurs et artistes a été de tenter de produire des biens non reproductible - et donc encore plus prisés. Les approches possible incluent les sculptures vivantes transgéniques construit avec des gènes de terminaison et d'obsolescence, l'art énergétique, les objets fabriqués dans des matériaux extrêmement rare (par exemple, une chaise fabriquées dans du titanium extrait du cratère Mead sur la surface hostile de Vénus) ou les objets intangibles tels que des spectacles de haute technicité. 

\subsubsection{L'économie et les infugiés} \label{sec:econ-infom-refug} 

Pendant la dernière pahse de la Chute et l'évacuation de la Terre, plus de quatre cent millions de réfugiés ont étés uploadé et égocasté dans des bases de données en orbitales. De ces banques, les réfugiés infomorph ont étés transmis dans des banques de données à travers tout le système solaire. Ils ont été forcés de fuire la Terre sans aucun biens, sans même leur corps. Ils sont devenus des infomrophs qui n'ont plus rien au delà de leur esprtit et de leurs souvenirs - le groupe de réfugié le plus dépourvu de l'histoire de l'humanité. Dans les années qui ont suivit la Chute, un grand nombre de ces infugiés ont étés réincarnés. Ceux disposant de compétences de valeurs ont été les premiers à gagner une morph, suivis par tout ceux qui avaient des amis ou des proches vivant déjà en orbite et qui pouvait se porter garant de la réincarnation. 

Ces deux groupes n'ont représenté que la moitié des réfugiés. Les autres se sont retrouvés dans une situation bien plus difficile. Manquant de contact personnels ou de compétences vitale, ils n'avaient personne pour les aider. Dans les premières années, beaucoup de ces infugiés ont signés des contrats échangeant leur main  d'œuvre ou d'autres services en échange dune réincarnation et la garnatie d'une forme de revenue suffisante pour survivre. En raison du manque critique de main d'œuvre dans les cinq premièrs années parès la Chute, trente autres pourcents de réfugiès parvinrent à récupérer un corps (générallement une synthmorph bon marché). Ces serf contractés réalisait tout type de tâches critique, allant de la récupération d'habitats à l'état de ruine pour des appareils utiles au forage ou à l'exploitation des astéroïdes. D'autres devinrent des serviteurs ou des gardes du corps pour les plus riche, ou effectuèrent d'autres services moralement contestable pour les syndicats du crime. La plupart choisirent des emploi en construction orbitale, aidant à construire de nouveaux habitats qui pourraient éventuellement devenir leur maison. Quelques infugiés trouvèrent du travail en effectuant des services comme de l'exploitation de donnée, de la surveillance d'usine automatisée ou d'autres travaux pouvant être facilement fait par des infomorphs. Après la Chute, les infomorphs étaient utilisées pour s'occuper de nombreuses tâches précédement effectuées par des AGIs, a qui plus personne en faisait confiance. 

Malheureusement, quelques réfugiés infomorph ont signé des contrats malheureux ou mauvais et se sont retouvés à travailler pendant des années avant de s'apercevoir que soit leur employeur cherchait en permanence une façon de repousser ou de réduire le paiement, soit disparaissait avant de pouvoir tenir leur promesse. En conséquence, un peu plus de vingt pourcent des infogiés restent à l'étât d'infomorph; certains par choix, mais la plupart parcequ'ils n'ont pas été capable d'acquérir les moyens de se réincarner ou travaillent encore sur des contrats longue durée pour gagner leur morph. Le problème d'obtenir un corps pour ces infugiés va au-delà de simplement fournir une moph pour s'y réincarner; les êtres de chair nécessitent de l'espace ainsi qu'un stock permanent de consomable. Pour ces raisons, beaucoup d'infugiés ont étés morphés dans des coquilles synthétique et hébergés dans des zones inhospitalière pour les biomorphs, tels que les part non fermées des aérostats Venusiens. La place étant en rupture de stock, les listes d'attentes pour les infugiés cherchant un habitat qu'ils puissent appeler maison sont relativement longues. 

Les hypercorps et le Consortium Planétaire ont été prompt à utiliser cette grande réserve de main d'œuvre, particulièrement sur Mars. Mars possède de grandes quantité d'espace à l'air libre et de ressources et est suffisament proche d'être habitable pour que les morpsh adpatée à Mars comme les rusters soient peu coûteuses à créer. Le Consortium Planétraire est donc devenu responsable de l'emploi de presque toutes le sinformophs réfugiées restantes. Pendant la dernière décennie, la grande majorité des infomorphs réfugiées qui ont voulus un corps ont trouvé que se contracter au Consortium Planétaire ou à l'une des hypercorp associée impliquée dans la terraformation Martienne  est la méthode la plus fiable pour obtenir une morph et un domicile, puisque les deux sont garantis à la fin du contrat. Le travail nécessaire est cependant particulièrement difficile, et les contrats sont normalement relativement longs. Le Consortium Planétaire est également adepte de l'ajout de charge qui prolongent le contrat - bein que la plupart des contractés signent des contrats de cinq à vingt ans, enr éaltié ces contractés durent typiquement de huit à cinquante-cinq ans; certains durant même ecore plus longtemps. Cette grosse population de serviteurs contractés sur Mars - beaucoup d'entre eux étant libre et réincarné - est devenu une force de son propre chef, adhérant aux zone rurale et sauvage de Mars et dédaignant les dômes des élites hypercorp. Se choisissant le nom de Barsoomiens d'une vieille série de fiction Terrestre, cette classe basse irritée est en train de devenir une épine dans le flanc du Consortium Planétaire. 

Même si le processus est fortement automatisé, la terraformation et l'agriculture sur Mars est un travail à la fois lassant et physiquement éprouvant. Les contractés sont régulièrement envoyés dans les régions qui ont été le plus affectées par la Chute. En conséquence, ces employs font occasionnellement face à des attaques de forme de vie mutées par les TITANs, des essaims de guerre nanotechnologique ou d'autres technologie exotiques similaire et dangereusement actives. Les employés contractés ne sont pas tenus responsable de la destruction ou des dommages apportés par de tels dangers à leur morph, mais faire l'expérience d'une mort, même si elle est réversible, dans ces conditions est hautement traumatisant. 

D'autres réfugiés se sont découvert à prendre du plaisir en tant qu'infomrophs, se délectant de simulspace complexe et profitant de la vie virtuelle. Certains ont trouvé un boilot leur permettant de se payer un egocast à travers tout le système solaire. Dix ans après la Chute, il y a une culture infomorph florissante. Bien que des données soient difficile à obtneir, beaucoup de chercheurs pensent qu'au moin un tiers des infomorph réfugiées actuels n'ont pas pour projet de se prendre une morph, préférant la liberté de l'existence virtuelle. Ces infomorphs sont devenues extrêmement impliquées dans les politiques d'habitats, en particulier dans le système extérieur; beaucoup d'habitat ont des représenatnts qui sont des infomorphs. Beaucoup de chercheurs prédisent que cette culture infomorph va de plus en plus s'écarter des cultures physiques au fur et à mesure de leur progression. 

\subsubsection{La masse cliquetante} \label{sec:clanking-masses} 

Avec autant d'infugiés faisant l'aquisition de coquilles synthmorph bon marché - particulièrement les valises et les synth - et étant incapable de s'acheter mieux, les synthmorphs ont été associées à la pauvreté à travers tout le système solaire. Cette couche basse des pauvre est souvent appellée "la masse cliquetante," et compose un sixième de la population transhumaine. La plupart de ces personnes désirent fortement acqurire uen biomorph, même si ce n'est qu'une splicer ou un pod de travail. En raison de leur présence, beaucoup de synthmorphs ont vues avec dégoût, en aprticulier dans les cercles sociaux de l'élite. Même ceux qui possède une morph synthétique chère, amoureusement personnalisée et équippée de toutes les dernière augmentations sont considérés comme étant des excentriques ayant mauvais goût. 

la stigmatisation sociale des synthmorphs est renforcée par la peur que, en cas d'une autre attaque des TITANs, leur coques robotiques pourraient rapidement être cooptés pour devenir une armée mortelle contrôllées par les TITANs. Cela a poussé certains habitats à aller aussi loin que la ségrégation active de leur population en synthmorph, ségrégation rationnalisée par le fait que les synthmorphs peuvent facilement vivre dans des portions non chauffées et dépressurisés de différents habitats. Cette ségrégation et cette stigmatisation sociale, a produit le démarrage d'une culture synthmorph émergente. Il y a déjà de nombreux habitats dans lesquels tous les habitants sont incarnés dans des coques synthétique et dans lesquel els systèmes de survie conventionnels n'existent que pour les quelques visiteurs qui portent des biomorphs. 

\section{Habitats} \label{sec:habitats} 

La Terre étant devenue inhabitable, la transhumanité survit donc dans une variété d'habitat hors monde. Il y a deux types majeurs de ces habitats: les abris sur des planètes ou sur les grosse lunes, telles que ceux sur la Lune, Mars, Vénus, Europe ou Titan, et les habitats spatiaux qui sont construit sur ou proche d'un astéroïde ou d'autre source de matériaux bruts utiles. La plupart de ces habitats spatiaux tournent sur eux même pour fournir une gravité, la gravité Terreste et Martienne étant les deux chois les plus courants. Il y a aussi quantité d'habitats en zéro-g ou en microgravité, sui qont soit des habitats ne tournant pas ou des stations construites dans de petits astéroïdes ou de petites lunes. 

\subsection{Abris planétaire} \label{sec:plan-settl} 

Les cités états de Mars et de la Lune et d'autres abris planétaires contiennent des environnements familier aux réfugiés de la Terre. La similitude est l'une des raisons pour laquelle deux tiers de toutes les infomorphs vivent sur Mars, la Lune ou Titan. Le type exact d'abri dépends de la planète ou de la Lune sur laquelle ils sont situés, certains étant bien plus similaire aux villes Terrestres que d'autres. La plupart des abris Lunaires, comme ceux de ganymède, Mercure, Titan et Callisto sotn un réseau de tunnels et de chambres sous la surface et excavées avec des foreuses à plasma. Ces abris tunneliers diffèrent légèrement d'un monde à l'autre. Dans la plupart de ces cités tunnel, les murs de toutes les zones ouvertes et de beaucoup de résidence sont cosntitués d'herber génétiquement conçue pour le confort et la longévité, avec des panneaux lumineux couvrant le plafon et founissant une lumièrebrillante dans tout le spectre. 

Quelques unes de ces cités enterrées renforcent encore plus leur apparence naturelle en ajoutant des arbres et, dans quelques cas, des écosystèmes spécifiquement conçus, à la fois dans les espaces publics et dans les résidences privées. Quelques uns de ces forêts et jungles des tunnels urbains sotn le foyer de nombreuses vignes florissante et de jolis papillons tropicaux. Dans un petit nombre d'abri sur Titan et sur la Lune, des colonies de petits singes et de perroquest dont le métabolisme et les habitudes ont été modifiées selon les standard actuels de propreté et d'hygiène, donannt à certains de ces cités tunnel l'impression d'être une jungle dense. 

Toutes les cités tunnels les plus vieilles et propsères contiennent également de large zone ouvertes qui couvrent entre un et vingt hectares chacune, et dont le plafond est au moins à dix mètres du sol. Certaines de cs zones sont des aprcs, d'autres des places publiques, mais toutes permettent aux résidents des cités tunnels d'avoir une chance de faire l'expérience des espaces ouverts. Également, avec l'exception de Mercure, toutes ces cités tunnel sont sur des lunes où la gravité est rarement supérieure à un sixième de g. Certains de ces espaces ouverts sont construits avec des plafonds à une hauteur de trente à cent mètre et sont conçus pour que les résidents puissent les utiliser pour voler attacher à une apire d'ailes spécialement conçues. 

Les cités nuageuses de vénus sont parmi les habitats les plus inhabituels dans le système solaire. Leur nature exotioque est renforcée par la possibilité d'observer les formes de vies volante et flottante nouvellement introduites et modifiée pour vivre dans les nuages. Bien que situés presque cinquante kilomètres au-dessus des environnement les plus mortels du système solaire, la vie dans ces cités nuageuses est parmi celles qui ressemble le plus à la vie sur Terre  dans le système solaire, avec une gravité, des températures et une rpession atmosphériques trés proche des niveaux normaux sur Terre. 

Par contraste, les habitats de Mars ressemblent le plus aux cités la terre perdue, construites sur la surface plutôt que sous-terre ou dans les cieux. Certains des abris les plus récents sotn conçus pour être utilisé par des habitants en morph rusters ou en synthmorph et n'ont donc pas de système de survie. Les plus vieilles cits Martienne et d'autres abris sont typiquement recouvert par des domes bas de polycarbonates flexible et remplis d'une atmopshère complètelent respirable, même si parfois en pression légèrement basse. Certains sont, cependant, un assemblage de gratte-ciels scellés, connectés par des passerelles et des tunnels. Si les efforts actuels de terraformation continuent selon le planning, les dernières cités scellées Martiennes seront ouvertes à une atmosphère Martienne respirable par toutes les morphs d'ici soixante ans. 

Les abris planétaires lesplus étranges sont les cités océaniques d'Europe. Ils font parti des endroits les plus exotiques de tout le système solaire et sont quelque peu désorientant pour les personnes qui ne sont pas habitués aux cités sous-marines. De loin, la plupart semblent être des décorations de sapins de Noël complexe pendues cent mètre ou plus sous le niveau de la couche de glace au-dessus. Quelques uns sont construits encore plus profondément, plongeant sous la surface de glace près des différents courants hydrothermiques qui héberge les groupes de vie natives d'Europe. 

Beaucoup des habitants des cités d'Europe les trouvent familières car ils ont généralement vécus dans l'une des cités sous-marines sur Terre et sont donc habitués à la fois aux conditions et à la vie dans un corps adapté au milieu aquatique. Les cités Européenne contiennent toutes des bâtiments scellés avec une atmosphère normale, à la fois parceque certaines activités sotn plsu facile à faire à l'air libre et parceque les cités acceuillent souvent des visiteurs sans branchies Ces régions ne constituent cependant qu'environ dix pourcent de la plupart de ces cités. Le reste à l'air vaguement smiliaire à beaucoup d'habitat en zéro-g, excepté le fait que els structures sont considérablement plus robuste et localisées sous l'eau. Les bâtiments sont conçus pour être accessible en trois dimensions, passer d'un niveau à l'autre implique généralement de nager à travers une grande ouverture dans le mur et de descendre d'un niveau. Dans presque toutes les cités aquatiues, de grands générateurs à fusions chauffent l'eau environnante, afin que toute la ville existe dans une région d'eau qui est bien plus chaude que la mer frigide d'Europe environnante. 

\subsection{Habitats spatiaux} \label{sec:space-habitats} 

A l'exception des habitats privés des plus riches et puissants décrit plus bas, la vaste majorité des habitats spatiaux hébergent une population de deux milles cinq cent à un million d'habitants. Presque deux tiers de ces habitats ont été construit pendant les septs premières années après la Chute, lorsque de gigantesque portions des infrastructure du système ayant survécu ont été utilisées pour créer des habitats convenable pour les centaines de millions d'infugiés. 

Pendant cette époque, plusieurs centaines d'habitats toriques et de colonies en grappes ont été créées à travers le système solaire. Beaucoup de ces habitats ont été créés par des machines de minages automatiques qui ont été reprogrammée pour créer des colonnies. En raison des limitations de ces plateformes de minage automatiques, la plupart de ces habitats sont petits, abritant entre mille et cent mille habitants. Vingt pourcent des habitants du système vivent dans de tels habitats. Pendant la dernière décennie, différentes petites organisations, cultes et sous-cultures ont quittés les habitats les plus grands dans lesquels ils vivaient pour créer leur propre petits habitats, peu d'entre eux ayant été conçus pour héberger plus de dix milles résidents. 

Le développement des nouveaux cylindres Hamilton nanotechnologique a relancé l'intérêt dans les grands habaitats et dans les habitats pouvant rapidement augmenter leur taille pour faire face à une population croissante. Le coûts et les difficultés générés par l'extension d'habitats existant ou par la construction de nouveaux habitats est l'une des principales raisons qui fait que plus de quarante millions d'infugiés ne possèdent toujours pas de morphs. Même si aucun des cylindres Hamilton n'a finit de grandir, ils sont regardé avec considérations par leurs résidents. La même technologie est également capable de produire une méthode de création de petits habitats à bas coût, pour lesquels les créateurs ont juste à planter le générateur nanotechlogique approprié dans un astéroïde et d'attendre quelques mois. 

\subsection{Barge de la racaille} \label{sec:scum-barges} 

Les infâmes barges de la racaille se situent à l'extrême opposé des cylindres Hamilton. Beaucoup d'enre elles sont des vaisseaux construits avant ou pendant la Chute et qui était utilisés pour aider pendant els premières étapes de l'évacuation, convoyant les gens loin de la Terre mourante. Beaucoup de ces vaisseaux de réfugiés ont été incapables de touver un endroit où déhcarger leur cargaison humaine, devenant ainsi une sorte de camp de réfugié itinérant, succombant parfois à la mutinerie. Ils ont parfois rejopint des vaisseaux et des nuées de racailles pré-existant, adoptant leur mode de vie nomade, en roue-libre et anarchique. par opposition à l'égocast ou aux vaisseaux à fusion plus rapide et plus efficace, les prétendues barges de la racaille offrent une alternative aux voyages spatiaux. Ces vaisseaux fonctionnent comme des marchés noirs errants et des festivals du bizarre - des zones de non-droit où n'importe qui peut trouver ce qu'il veut ou ce dont il a besoin pour le bon niveau de rep ou le bon prix. 

La plupart des barges ont des moteurs alimentés par la fusion au plasma et abritent de deux cents à cinq milles habitants. Les pires barges sont exceptionnellement bondées, avec des systèmes de survie vieillissant se démenant pour maintenir une atmosphère repsirable (mais toujours malodorante) sous la pression de pasagers trop nombreux. Les barges les plus grandes et les plus prospères sont souvent équippés de différentes commodités moderne, incluant de grande machine d'abondance et de gigantesque boutiques de plans de fabrication piratés. Certaines sont des enclaves utopistes prospère, alros que d'autres sont des nids modibles de contrebandiers et de voleurs qui auraient été détruit il y a longtemps si ce n'est que des organisations suffisament grande et puissante trouvent leur existence ocasionellement utile. Les conditions de vie sur les barges vont des camps de réfugiés bondées aux enclaves anarchistes et égalitaire florissante mais non riches en passant par des habitats relativement modernes équipées dans toute leur splendeur barabre par des gangs criminels organisées et ayant réussit. 

\subsection{Une variété de monde flottants} \label{sec:divers-float-worlds} 

L'utilisation des machines d'abondances et des matériaux intelligent fait que les intérieurs de tous les habitats, à l'exception des plus pauvres et des plus dénués, peuvent être redessiné en fonction ds souhaits de leurs habitants. Lorsque le nombre d'habitants est suffisament petit ou que leur esthétique est suffisament uniforme pour qu'ils partagent les mêmes goûts, les résultats peuvent être à la fois unique et étrange. Des marottes à grande échelle se propagent parfois à travers les habitats les plus grand et les plus cosmopolites, faisant de certaines des plus grande colonnie quelque chose de tout autant bizarre. 

Plusieurs habitats ressemble aux jungles terrestres, avec la canopée d'une forêt vierge entière grandissant depuis la coque extérieure tournant lentement et dont toutes les résidences et éléments de haute technologie niché dans les branches ou dans les cavités de ces grands arbres conçus génétiquement. Dans ces merveilels vivante, des singes, des iguanes et des paresseux génétiquement conçus vivent parmi les habitants - certaines de ces créatures sont des animaux sauvages, alors que d'autre sont contrôllés par des IA servantes et se comportent comme des drones de maintenance ou d'observation. Certains habitats ressemble à d'autre paysages de la vielle Terre, incluant plus d'une douzaine d'habitats remplis d'eau hébergeant quelques uns ds habitants aquatiques des cités sous-marine maintenant détruite. Dans la plupart de ces habitats marins, les bâtiments sont soit placés dans un récif coralien peuplé de poisson et d'autre créatures ou sont carrément construit dans le récif coralien. Il y a de nombreux habitats dupliquant d'autres environnements, tels que l'Afrique - un grand habitat Cole avec une population de deux cent milles habitants, où l'habitat est fait pour ressembler à la savane Africaine. En Afrique, les desux extrémité de l'habitat sont façonnés en montagne enneigées, et els habitats vivent principalement dans plusieurs grande cité construite dans la savane. 

Alors que la nostalgie de la Terre est une force puissante dans la conception d'habitat, il y a beaucoup d'autres options. Quelques habitats exotiques ressemblent aux fantastiques citées de différents jeux vidéos ou d'autres divertissement plus vieux, incluant une poignée de petits habitats excentriques dans lesquels les habitants apparaissent tous comme des humanoïdes étranges oou des êtres étrangers. Dans beaucoup de ces habitats, les habitants se sont modifié cosmétiquement pour se fondre dans le décor. 

L'une des plus grandes différences entre les petits ou les plus grands habitats est que les résidents des plus petites stations partagent souvent une idéologie commune ou un sens esthétique, et sont donc bien plus excentrique. Quelques un des habitats les plus inhabituels vont des habitats faiblement éclairés, avec des paysages effrayant remplis d'arbres mort, perpétuellement effeuillés et régénérants les toiles d'araignées dans leur branches, ainsi que d'autres touches macabre similaire aux colonnies chatoyante qui sont des citadelles lumineuses de quartz et d'acier. Certains dont de gigantesque archologies interconnectées où la vie privée est rare, alors que dans d'autres, chaque famille voire chaque personne possède une résidence séparée qui est rarement visitée par les gens de l'extérieur. Puisque la population de ces stations est relativement faible et que la vaste majorité ne sont pas des centres économiques majeurs, voyager depuis ou vers les habitats les plus petits est peu fréquent, ce qui augmente encore plus leur insularité et leurs idiosyncracies. 

\subsection{Les plus grands habitats} \label{sec:largest-habitats} 

Extropia, les gigantesques cités état Martienne et certaines des plus grande colonnie Lunaire abritent entre un et vingt millions d'habitants. Il y a de nombreux abris plus petits qui contiennent entre cent mille et un milliion de résidents. Ces habitats sont considérablement moins idiosyncratique et exotique que les habitats les plus petits. Presque tous abritent une population cosmopolite et diversifiée venant d'une grande variété de sous-culture. En raison de cette diversité et étant donnée la difficulté d'obtenir un consensus avec uen grande population, ces abris tendent à être des réminiscence des cités de la Terre. Tous possèdent un caractère et une impression unique, mais les différences entre un habitat et un autre sont rarement déstabilisante. De plus, toutes ces stations sont suffisament grande pour abriter des bureaux de toutes les hypercorps majeures, ce qui promeut encore plus l'uniformité en fournissant les mêmes services depuis des bureaux hypercorps identiques. Puisque la plupart de ces habitats sont des centres majeurs de commerce, voyager de l'un à l'autre est fréquent, il y a donc de nombreuses installations pour les voyageurs tels que des hôtels ou des clubs de sports qui aident à réduire le dépaysement en offrant des expériences identiques, quelque soit le lieu. 

\subsection{Habitat à microgravité} \label{sec:micr-habit} 

Les habitats en zéro-g sont très différents de ceux qui utilisent une gravité par rotation. La plupart sont un réseau de tunnels creusés à travers l'astéroïde - de manière similaire aux cités tunnel de la Lune et de Titan - mais certains sont  considérablement plsu exotique. Comme pour la plupart des autres habitats, presque toutes les colonies en microgravité sont contruit dans, sur ou à proximité d'un ou de plusieurs astéroïdes contenant un grand nombre de matériau bruts utiles. Ils ont typiquement une gravité inférieur à 0,01g qui n'affecte que très peu la vie quotidienne de ses habitants. Les environnements quasiment sans poids permettent des conception d'habitat intéressante et inhabituelle car il n'y a ni haut, ni bas, permettant ainsi la création de structures qui auraient été trop fragile même en faible gravité. Les habitats de Nova York (p. 97) et de Nguyen's Compact (p. 103) sont deux example de ce type, parmi de nombreux autres. 

\subsection{Habitats privés} \label{sec:private-habitats} 

Les habitats les plus rares et les plus exotiques sont les habitats privés luxueux possédés par individus excessivement riches ou avec une trés haute rep. La plupart des habitats privés sont petit mais fournissent quand même à leurs résident plusieurs centraines de mètres cubes d'espace personnel. 

Un habitat privé typique est soit un cylindre de cent cinquante mètre de diamètre (le minimum nécessaire pour produire une gravité proche de celle de Mars à un taux de rotation suffisament lent pour éviter les problèmes) et entre cinquante et deux cent mètres de long, ou une sphère en zéro-g de cent à deux cent mètres de diamètres. Ces habitats sont toujours connectés à une petite collection de matériau bruts, généralement des tranches de silice, de nickel-fer et d'astéroïdes carbonifères contenant de l'eau d'une masse au moins égale à celle de l'habitat. La majorité des habitats privés sont habités par une douzaine à trois douzaine de morphs, certains ou la plupart d'entre elles pouvant être des serviteurs IA ou, en de rares occasions, des serviteurs contractés. La vie dans un habitat privé est particulièrement luxueuse. Presque toute la surface est faites de matériau intelligents configurable et il y a plusieurs grosses machine d'abondances génériques disponible pour l'utilisation de chaque résident. 

En utilisant ces nanofabeurs et les matériau intelligents aux maximum de leur capacité, les résidents peuvent complètement changer l'intérieur de leur habitat en à peine une journée ou deux - transformant un alignements stérile et cristallins de bâtiments en métal brillant et en verre en une forêt prospère, habitée d'une variété d'animaux sauvages. Le mesh est rempli de vids et d'XP sur les vies des résidents les plus célabres du système solaire. Presque tout le monde a vu de nombreuses fois l'intérieur de l'un des manoirs aux espaces intérieurs gigantesque , bien que seul un petit pourcentage des habitatnts du système solaire aura une chance de visiter en chair et en os un tel endroit. beaucoup de resquilleurs, de charoganrds qui parcourent la Terre, et d'autres qui ont des comportements audacieux similaires espèrent  être capables un jour d'obtenir des informations ou des objets suffisament interessant pour leur permettre de prendre leur retraite dans leur propres habitats. 

\section{Factions} \label{sec:factions} 

On aurai pu penser qu'un évènement cataclysmique tel que la Chute aurait eu tendance à renforcer et à rapprocher les éléments survivants de la Transhumanité, se dédiant ensemble à repeupler le système solaire et à maintenir leur proserité. Au lieu de ça, la distance et l'isolation physique des transhumaines colonies et des habitats éparpillés dans tout le système soalire, ainsi que les effets des technologies émergentes ont eu sur les économies transhumaine et sur la vie soaicle, ont promu l'évolution d'un large spectre de philosophie, de programme et de modèles politiques. 

\subsection{Les hypercorps} \label{sec:hypercorps} 

Pour certains économistes, la Chute et les nombreuses crises qui l'ont précédées sur Terre peuvent être vues comme une extinction, la fin des dinosaures qu'étaient les mégacorporation transnationale, des géants financiers qui supportaient leur système monolithique sur des modèles économiques dépassés et des technolgies industrielles. Les hypercorps sont leur descendant évolué: plus mince, plus rapide, plus agresisve et plus souple, embrassant avidement les possibilité des nouvelles technologies et jamais effrayé d'abandonner les plus vieille pour profiter des plus récentes. Ce sont les hypercorps qui ont propulsé l'expansion de l'humanité dans l'espace et qui continuent de repousser les limites de la technologie, guidant la transhumanité vers de nouveaux horizons - en ayant toujours le profit comme but conducteur. 

La plupart des hypercorps sont ds entités légal décentralisées non basées sur les actifs. L'automatisation complète, la robotique avancée, la technologie des morphs et les machines d'abondances permettent aux hypercorps de s'affranchir de l'embauche massive pour la main d'œuvre ou les services de productions. Le besoin de travail physique a été réduit aux tâches associées à la construction d'hébaitat ou à l'exploitation minière spatiale. Les informophs et les IAs sont massivement employées (plus précisément, possédée) comme opérateur de drône ou travailleurs virtuels, et beaucoup des tâches administratives sont faites en ligne par la réalité augmentée, les réseaux privés virtuels et les nœuds de simulspace. Certaines hypercrops sont en fait entièrement "virtuelle," sans actifs physique et chaque employé est considéré comme un bureau mobile. Quelques hypercorps majeures ne sont littéralement composées que d'une douzaine d'employés transhumains. Même si certaines hyeprcorps sont massive et diversifiée, la plupart se spécialisent dans un domaine ou un service particulier. Cela donne un système complexe de paretnarait pour développer, produire et vendre des produits et des service et une forte tendance à contracter en interne des servcies particuliers d'autres hypercorps. Beaucoup d'hypercorps regroupenet égelement leurs ressources et leurs talent dans des initiatives de recherche coopérative, des centres de projets et des habitats partagés. 

La plupart d'entre elles sont dans une perspective capitaliste traditionnelle, même si beaucoup ont adoptés des philosophies des affaires et des modèles de gestion alternatifs. Cela peut inclure de la prise de décision basé sur des prévision de marché, des modèles de consensus de groupe ou d'abandonner complètement le management pour laisser le personnel voter pour des initaitives qui sont statistqieuemnts bien meilleure. Quelques unes sont des sociétés anarcho-capitaliste originaire des enclaves Extropiennes, bien qu'elles souffrent souvent d'un biais lorsqu'elle concluent des accords avec les puissances du système intérieur. 

Le système solaire grouille de milliers d'hypercorps; quelques une des plus proéminentes et intéressante sont détaillées ci-dessous. 

\subsubsection{Cognite} \label{sec:cognite} 

\textbf{Industries principales:} Science Cognitive, Implants Mentaux, Psychochirurgie, Nootropiques. 

\textbf{Stations Principales:} Pensée (orbite Vénusienne), Phobos (lune de Mars) 

Un pionier dans le domaine des scince cognitive, Cognite (prononcer cogue-nite) est à la pointe de la recherche dans la compréhension de l'esprit transhumain. Plus connue pour ses augmentations mentales et la conception origniale des morphs menton, Cognite est également un spécialiste en psychochirurgie et dans les nootropiques. Leur image élitiste et distante n'as pas été améliorée par leur implication scandaleuse dans le projet qui a chercher à élever des enfants en croissance accélérée qui sont devenus la génération Égarée (p. 233), ni par les rumeurs selon lesquelles ils seraient engagés dans des recherche impliquant des attaques sensorielle incapacitantes influencées par les TITANs. Ils demeurent néanmoins un membre vlef du Consortium Planétaire. 

\begin{quotation} \textbf{Psiclone} 

To: Proxy-99 

From: <Chiffré> 

Je joint quelques données que j'ai acquis récemment auprès d'une source interne en lien avec le projet baptisé "Projet Psiclone" - uen sorte de recherche financée par la caisse noire de Cognite, probablement avec l'implication d'autres intérêts du Consortium Planétaire. leru travail semblese concentrer fortement sur la souche Watts-MacLéod du virus Exsurgent - avec quelques résultats alarmants. \end{quotation} 

\subsubsection{Comet Express (COMEX)} \label{sec:comet-express-comex} 

\textbf{Industries principales:} Services de Courrier, Expédition, Logistique 

\textbf{Stations principales:} Nectar (Lune), Olympus (Mars) 

Comet Express se spécialise dans les services de livraisons, la logistique interstellaire, la chose d'approvisionnement et l'expédition. Ils maintiennent une présence dans presque tous les habitats transhumains du système solaire, souvent via des sous-traitant locaux. En dépit des merveilles de la nanofabrications, beaucoup de ressources doivent toujours être importées. ComEx se concentre sur la gestion des approvisionnement et des routes de commerce et s'assure que toutes les expéditions physiques arrivent à leur destination. Dans ce but, ComEx maintient des plateformes orbitale équippées de fronde d'accélération a des points stratégiques du système solaire ainsi que d'une flotte de vaisseaux cargos et de drone coursiers. Pour des raisons inconnues du public, ComEx est perçue hostilement par la République Jovienne, qui a donné l'ordre d'abattre tous les vaisseaux ComEx. 

\subsubsection{Action Directe} \label{sec:direct-action} 

\textbf{Industries principales:} Services de sécurité, Contrats militaires 

\textbf{Stations principale:} Hexagone (L5 Terre-Lune) 

Descendant des rstes de plusieurs forces militaires nationales pré-Chute et de sous-traitants militaire privés, cette hypercorp s'est forgé sa réputation pendant la période immédiatement après la Chute, lorsqu'ils ont aidés à gérer les populations de réfugiés dans de nombreux habitats et vaisseaux tout en réduisant au immédiatement silence tout signe de mutinerie. Action Directe est aujourdh'ui connue pour ses troupes de chocs extrêmement efficace ainsi que pour ses morphs de combat supérieures, fournissant la sécurité et des services de sécurité publique aux habitats auto-gouvernants ou aux installations hypercorps. Les shangements d'alliance politique entre des groupes d'habitats, la rivalité coropratiste et la peur constante des agents des TITAN permets de favoriser la communication d'Action Directe induisant la paranoïa. La corporation maintiens plusierus habitats comme installations d'entraînements physiques et comem dépôt d'armement. 

\subsubsection{Ecologene} \label{sec:ecologene} 

\textbf{Insdustries principales:} Systèmes environnementaux, Génétique. 

\textbf{Stations principales:} McClintock (orbite Martienne) 

Ecologene se spécialise dans les systèmes vivants, la génétique environnementale (avec une spécialisation en insectes), les animaux intelligent, la bio architecture et la nanotech environnementale. Ils conçoivent et maintiennent les écosystèmes à l'intérieur de nombreux habiatats et de colonnies tunnels. L'un des projets notable d'Ecologene est la construction et le maintiens d'archive génétiques massive de toute les formes de vie, bien que ce projet ai été fortement handicapé par la Chute. Pour des raisons incpnnues, Ecoloene semble avoir les faveurs des Facteurs. Certains spéculent qu'Ecologene possèe un moyen de pression, alors que d'autres croient qu'Ecologene échange des secrets géénétique transhumain contre quelques cadeaux de xéno-tech. 

\subsubsection{Exotech} \label{sec:exotech} 

\textbf{Industries principales:} Upload, IAs, Electronique, Logiciel 

\textbf{Stations principales:} Starwell (Ceinture Principale) 

Souvent considéré comme la pupille technocratique personnelle de l'infame magant des médias Morgan Sterling, Exotech a émergé de la Chute quasiment indemne, toute les pertes significative ont été absorbées par des actifs corporatistes dans les segments périphériques du marché, tout en  se livrant à d'impitoyables rachats de compétiteurs en difficulté ou de thinks tanks incapable de s'adapter à l'économie transitionnelle. De nos jours, Exotech reste un concepteur dominant dans le domaine de l'électronique haut de gamme, des IAs et de systèmes logiciels pour le mesh. ExoTech continue également de poursuivre un plan sans compromis avec ses recherches en émulations cognitives, en upload et en réincarantion, ainsi que dans le domaine de la simulation d'ego infomorph. Des rumeurs sur le fait qu'ExoTech supporte des recherche sur les AGIs et en produirait persistent. 

\subsubsection{Experia} \label{sec:experia} 

\textbf{Industries principales:} Média (RA, RV, XP), Informations, Divertissements, Mémétique 

\textbf{Station principale:} Elysium (Mars) 

A la hauteur de son nom, Experia domine le marché du système solaire dans les segments de l'information, des médias et du divertissement, générant la controverse non seulement par sa position pro-IA publiquement exprimée et en ayant invité des AGIs à son directoire, mais également par un usage prolifique du marketing hyeprviral et des techniques sophistiquée de programmation d'XP. Un autre segment clef est la production d'XP éducatives et de tuteurs infomorphs ou IAs, certains de ces derniers accédant régulièrement au status d'icône de la culture pop. Experia est l'autorité principale du Consortium Planétaire en dans le domaine de la conception et du déploiement de mêmes viraux personnalisés, développés pour contrer tout ce qui pourrait être une menace pour les intérêts du Consortium. La corpo a automatisé des nœuds et des centres RV sur beaucoup d'habitats à travers le système solaire, et elle engage des milliers de lifeloggeur indépendant en tant que journlistes citoyens, errants et en direct. Des accusations de certaines infomorph selon lesquelles Experia aurait illégalement assujetti des infomorphs contractés à des simulation expérimentales sans-fin pour l'analyse et l'intelligence prévisionnelle restent infondées. 

\subsubsection{Fa Jing} \label{sec:fa-jing} 

\textbf{Industries principales:} Exploitation minière, Énergie, Biotechnologie, Fabrication industrielle 

\textbf{Stations principales:} New Dazhai (Mars) 

Le géant industriel Fa Jing est un élément central dans le marché de l'exploitation minière et de la production énergétique et se vante également d'une présence remarquable dans les domaines de la biotechnologie et de la fabrication d'équipements industriels. L'ancienne mégacorporation s'est rapidement adaptée aux environnements de la nouvelle économie et aux systèmes basés sur la réputation, en partie grâce à son implication dans le développement de réseau et dans le partage des responsabilités sociales, incarné dans des concepts tels que le dàtóng et le guanxi. Souvent considéré comme insulaire et obtus, son état d'esprit interne communautaire et porctectionniste est en contraste élevé vis à vis des son attitude monopolistique et manipulatrice dans le domaine des affaires. Fa Jing est engagé dans des opérations d'extractions dans toute la ceinture d'astéroïde et dans les Troyens, et maintiens des actifs corporatistes sur Mars. 

\begin{quotation} \textbf{Crimes de guerre} To: Meshleaks Newswire 

From: <mesh ID inexistant> 

Tu as demandé, les voilà: des preuves vérifiables prouvant les crimes de guerre d'Action Directe pendant la Chute <lien manquant>. Vas-y, rends les publiques. L'élite du Consortium Planétaire te trouvera, te tuera et effacera tes sauvegardes. Vas y. Cherches les. \end{quotation} 

\subsubsection{Corporation Gatekeeper} \label{sec:gatek-corp} 

\textbf{Industries principale:} Resquille, Recherche, Média XP, Colonisations d'Exoplanètes. 

\textbf{Stations principales:} Gateway (Pandorre) 

Créée initiallement de la fusion de plusieurs institutions scientifiques et de leurs financiers corporatistes, cette hypercorp s'est fait une répuattion très rapidement lorsqu'elle a annoncé le décodage réussi des portes de trou de vers découvertes sur la lune de Saturne Pandorre. Sous le commandement du xénoarchéologiste excentrique mais charismatique Xander Rabin, le consortium a financé des équipes de resquillerus pour explorer via la porte de Pandorre, payant une petite part des revenus aux explorateurs mais se gardant tous les droits sur toutes les découvertes faites - aussi bien que la commercialisation et la distribution des enregistrements XP très populaires des resquilleurs. En plus des explorations planifiées, le consortium propose des voyages à haut risque de resquille et d'exploration pour les plsu braves et les plus désespérés, choisis via un système de tirage au sort. 

\subsubsection{Groupe Go-Nin} \label{sec:go-nin-group} 

\textbf{Industries principales:} Banque, Agritechnologie, Robotique et Services. 

\textbf{Stations majeures:} Tsukomo (Lune) 

Considéré comme une relique de l'économie de marché capitaliste de la Terre, le groupe Go-nin est un keiretsu Japonias traditionnel, un conglomérat d'entreprise ayant des relations entremélés et un partage de part, intégrées horizontallement à travers plusieurs industries (et parfois intégrées verticallement dans un secteur d'affaire), et centrées autour la vieille firme de consultant entrepreunarial Tamahashi. Tamahashi a évolué d'un lobby corporatiste influent en un portefeuille bancaire diversifié réparti équitablement parmi les partenaires du groupe; il contrôle maintenant les actifs du groupe et dirige la stratégie commerciale globale du partenariat. À travers ses corporations membres, le Groupe Go-Nin a une présence importante dans tout le système et - sans dominer un marché spécifique - possède des aprts de marché significative dans des domaines tels que la banque, l'agritechnologie, la robotique et les services. Toutes les difficultés rencontrées en s'adaptant à un système économique évolutif en raison de sa structure rigide sont compensées par un comportement d'exploitation sans-scrupule et une attitude bas de gamme, donnant au groupe la réputation de l'hyeprcopr la plus impitoyable du système intérieur. Go-nin contrôle actuellement une Porte de Pandorre sur Éris (p. 109), protégée par un contingent de mercenaires ultimates. 

\subsubsection{Gorgon Defense Systems} \label{sec:gorg-defense-syst} 

\textbf{Industries principales:} Technologies militaire, Sécurité, Contrats militaires 

\textbf{Stations principale:} Extropia 

Gorgon est l'une des réussites Extropienne les plus importante. Basé dans la place forte anarcho-capitaliste, Gorgon est devenue l'un des acteur principal dans la conception et la fabrication d'arme, de véhicule, de capteurs et d'autres technologies défensives. Leur gamme de produit inclue les systèmes d'armement personnels, l'armement des vaisseaux et les systèmes de défenses des habitats. Tout en étant proéminente dans le système intérieur, Gorgon est aussi l'un des principaux fournisseurs d'armes des stations autonomistes et bordées. Leur filliale Medusan Shield propose des services de sécurité privés en compétition directe avec Action Directe. Alors qu'Action Directe est connue pour le niveau d'expertise de ses soldats, Medusan Shield est connus pour leurs cadres d'élites composé de morph femelle extrêmement entraînées et esthétiquement améliorée. On suppose que plusieurs assassinat important ont été le travail d'agents sous contrats de Medusan Shield. 

\subsubsection{Nimbus} \label{sec:nimbus} 

\textbf{Industries principales:} Électronique, Système Meshés, Farcast, Communication 

\textbf{Stations principales:} Octavia (Vénus) 

Nimbus produit les composants clef de l'infrastructure meshée, des microradios spimes et systèmes sensoriels aux ectos, serveurs et connexions laser. Nimbus domine également le réseau de liens farcast à travers tout le système, grâce à plusieurs percées faites dans cette technologie (certains prétendent que Nimbus a acheter ces percées aux Facteurs). Des rumeurs sur le fait que Nimbus controllerrait une porte de Pandorre secrète ou qu'il serait engager dans de la contrebande d'ego illicite (ou même qu'ils transfèrent secrètement des égos vers des colonies expérimentales sur des exoplanète) ciruclent régulièrement sur le mesh, mais restent infondées. 

\subsubsection{Omnicor} \label{sec:omnicor} 

\textbf{Industries principales:} Nanofabrication, Chimie, Énergie, Anti-Matière 

\textbf{Stations principales:} Monolith-3 (Mercure), Feynman (Lune) 

Omincor est un descendant du géant mégacorporatiste pré-Chute Monolith Industries, spécialisé dans les domaines de la conception et la nanotechnologique, le rafinage de produit chimique, les carburants alternatifs et la recherche en anti-matière. Omnicor s'est arrangé pour sécuriser ses actifs de recherche clefs hors de portée de son jumeau rival Starware lors d'un violent conflit pendant la Chute, qui a mené à une animosité continue qui ne se règlera probablement que par une guerre copropratiste. En dépit de son apaprence de progressiste technologique, Omnicor conserve une structure corporatiste conservative doté de règles internes et de contrôle trés stricts  comme défense contre les tentatives répétées d'infiltration et de sabotage faites par Starware. parmi les principaux actifs de l'hypercorp, on trouve une installation de recheche en antimatière orbitant autour de Mercure. 

\begin{quotation} \textbf{Activités antisyndicales} 

To: OmniSec Alpha 

From: OmniSec 837302 

La surveillance l'a confirmé. Les travailleurs bio-incarné de notre installation sécurisé Didenko sont effectivement en train de communiquer avec des intérêts autonomistes extérieurs et discutent des tactqiues pour organiser un syndicat militant et libre, et prévoient une grève sauvage. Leur plainte principale concerne la journée de 30 heures et les régimes de drogues obligatoires nécessaire pour garder le personnel à notre niveau de productivité requis. Nous recommandons l'insertion immédiate d'une escouade de contre-insurection et d'implémenter les protocols antisyndicaux standards, incluant, mais non limité aux moyens habituels tels que, les tests de loyauté, la pacification chimique, la psychochirurgie tactique, l'excision sélective des nœuds meneurs, des frappes mémétiques et le remplacement de la fore de travail par des sauvegardes modifiées. L'ensmeble de l'opération sera déployée en utilisant une prétendue mission pour débusquer une infiltration de Starware. \end{quotation} 

\subsubsection{Pathfinder} \label{sec:pathfinder} 

\textbf{Industries principales:} Colonisation d'Exoplanètes, Exploitation Minière, Recherche 

\textbf{Stations principales:} Ma'adim Vallis (Mars) 

Pathfinder est l'une des premières hypercorp à avoir plongé dans l'expansion galactique, s'abrogeant de nouveaux territoires au delà des Portes de Pandorre et y établissant un grand nombre de colonies. Profitant d'infugiés désespérés et des resquilleurs, Pathfinder offre le transport vers une exoplanète et une nouvelle morph en échange de travail contracté. La corpo a établi plusieurs projets d'exploitation de ressources et d'extractions hors-monde, causant la consternation des préservationnistes. Bien que Pathfinder ait une faible présence dans le système solaire, elle est une cible fréquente des attaques des éco-terroristes. 

\subsubsection{Groupe de Prospérité} \label{sec:prosperity-group} 

\textbf{Insdustries principales:} Agriculture, Aquaculture, Pharmacopée 

\textbf{Stations principales:} Ceres, Lu Xing (Mars) 

Le Groupe de Propérité a atteint les rangs des hypercorps avant la Chute, répondant à la demande élevée de beaucoup de nouvelles stations pour de l'agritechnologie, de l'aquaculture, des fermes hydroponiques et d'autres sources de cultures en microgravité. En s'étendant dans le domaine de la pharmacopée, le Groupe est considéré comme l'und es fournisseurs principaux de nourriture et de médicaments pour les pauvres. Leur culture de fausse-viande et leurs additifs nutritionnels enrichis en protéines sont extrêmement demandés. Cette corpo a gagné de la sympathie lorsqu'elle perdit un habitat entier suite à une sorte d'invasion de TITAN résurgents quelques années après la Chute, même si certains ont suggérés qu'il s'agissait juste d'une couverture pour cacher un accident malheureux suite à l'expérimentation d'une nouvelle drogue testée sur une population non volontaire. 

\subsubsection{Skinaesthesia} \label{sec:skinaethesia} 

\textbf{Industries principales:} Génétique, Clonage, Biotechnologie 

\textbf{Station principale:} Ptah (Mars) 

En tant que leader dans la conception des biomoprhs, Skinaesthesia est renommée et respectée dans tout le système solaire pour ses produits sophistiqués dans tout le système, et plsu spécifiquement pour ses modèles personnalisé haut-de-gamme. Plus connu pour ses découvertes en conception et en amélioration génétique, les intérêts de l'hypercorp pour des morphs de combat sophistiquées ou des pods de plaisir stylés sont moins connus et ces produits sont souvents vendus grâce à un réseau d'écrans de fumée corporatiste ou de distributeurs locaux. Skinaesthesia se concentre sur la mise en avant des adaptations environnementales et des améliorations cybernétiques utilitaires, augmentant les chances de survie et la propsérité de la transhumanité. Des morphs expérimentales sont parfois offertes à des infugiés désespérés pour les tester sur le terrain. 

\subsubsection{Skinthetic} \label{sec:skinthetic} 

\textbf{Industries principales:} Génétique, Clonage, Biotechnologie 

\textbf{Stations principale:} Extropia 

Skinthetic est aussi un des concepteurs de morphs renommé, mais avec une réputation plus sordide et pas seulement en raison de ses racines anarcho-capitaliste. En se spécialisant dans les bio-modifications étendues et souvent radicales, l'hypercorp repousse les limites des conceptions des pods et biomorphs exotiques en vertu de la liberté morphologique. Les bioconservatuers ont condamné les pratiques et l'éthiques de la corporation et ont accusés Skintehtic de mener des expériences avec des matériaux xénogénétiques acquis auprès des Facteurs. L'attitude cavalière de Skinthetic les rends cependant extrêmement populaire dans de nombreuses zones du système extérieur, et ils sont connus pour être l'hypercorp à aller voir si vous voulez quelque chose de bizarre. 

\subsubsection{Solaris} \label{sec:solaris} 

\textbf{Industries principales:} Banque, Assurance, Investissements, Prospection Marketing, Courtage d'Information 

\textbf{Stations principale:} None 

Solaris est l'hypercorp qui domine le marché des banques et de l'investissement financier, gérant les assurances, le courtage d'information et les investissement spéculatifs à haut risque sur les expérimentations culturelle et sociale. En tant que membre du Consortium Planétaire, Solaris conseille beaucoup d'habitats sur la régultaion de leurs économies transitionnelles. Solairs n'a ni bureau ni actifs physique; chaque banquier est un bureau virtuel mobile. Des ruemrus court sur le fait que Solaris maintiendrait une abse secrète dans laqeulle la corporation lancerait des simulations sur le développement de la amcro-économie du système solaire, ajustant constament ses propres stratégies basées sur la dynamique du schéma global. Alimentant ces rumeurs, Solairs est connue pour engager des "consultants inédpendants" pour faire pencher la balance dans des investissement à haut-risque politiquement ou économiquement profitables. 

\subsubsection{Somatek} \label{sec:somatek} 

\textbf{Insdustries principales:} Élevés, Médiculture, Pharmacopée, Génétique 

\textbf{Stations majeures:} Clever Hands (Lune) 

Somatek est le leader dans le domaine de la science et de l'art d'élever des espèces animales, accomplissant quelques percés innovantes dans les modifications génétique et les amélioratiosn cognitives. L'hypercorp est également engagée dans la médiculture animale extensive - produisant des extrayant des médicaments de créatures transgéniques - et commercialise de nombreux produits et services liés aux animaux intelligents et aux créatures chimériques. En dépit des programmes d'apprentissage et d'entraînement qu'elle offre aux élevés et du fait que la plupart de sa force de travail est composées d'élevés, Somatek est un sujet de controverse chez les mercuriens qui désapprouve leurs méthodes (qui implqiuent souvent un contrôle strict de la reproduction des élevés), le manque de liberté dont les élevés bénéficie dans leur développement et leurs modifications, et l'anthropocentrisme des état d'esprit "forcés" dans les élevés). 

\begin{quotation} \textbf{RECHERCHE SOLARCHIVE: ZBRNY LIMITED + CONSPIRATION} 

Le trés secret Groupe Zbrny est au centre de nombreuse théorie du complot récurrentes et de légendes horrifique. Bien que les détails et la plausibilité varient, la plupart des rumerus prétendent qu'une attaque extérieure sur les stations de minage et de traitement de l'ancienne hypercorp d'Europe de l'Est a causée une panne critique et l'extinction des système de support de vie pendant une période de temps relativement longue. En fonction de la source, l'attaque en elle-même est censée avoir étée menée par les TITANs ou un des syndicats de la pègre auprès duquel le PDG Krystof Zbrny avait une dette. Ne prenant pas en compte les pannes système, le siège de Zbrny ordonna que toutes les stations non affectée soient abandonnées, le personnel étant soit licencis, soit transféré à des stations affectées. Depuis, plus personne n'a vu ou n'a communiqué avec l'un des employé de l'hypercorp mystérieuse - les négociations avec l'extérieur sont menées exclusivement via une IAG porte-parole. A ce jour, les drônes de Zbrny continuent de miner les astéroïdes afin d'en extraire les minéraux, alimentant les stations de traitement de l'entreprise. D'après les rumeurs, une tentative d'abordage un avant poste de Zbrny par des pirates bordés a amené à l'auto-destruction de la station. Les transports cargos massifs de l'entreprise pilotés par des IA sont connus pour ne pas répondre, leur donnant le surnom de "vaisseaux zombies". \end{quotation} 

\subsubsection{Starware} \label{sec:starware} 

\textbf{Industries principales:} Robotique, Conception Aérospatiale, Construction d'Habitat 

\textbf{Stations principales:} Chantiers Navals Korolev (Lune), Vesta (Ceinture) 

Une autre rémanence de la mégocaorp pré-Chute Monolith Industries (comme Omnicor), Starware est un des grands noms de la fabrication de robots, de moteurs à fusion de vaisseaux, de satellites et d'habitats pré-fabriqués. En dépit de ses ressources et de sa réussite financière, la vendetta opposant Starware à Omnicor empêche les deux corporations de bénéficier des privilèges d'une adhésion complète au Consortium Planétaire. Starware utilise beaucoup de travailleurs IA dans des coques robotiques, ayant souffert de bien trop de conflits avec des travailleurs lunaire mécontents. En fait Starware est devenu extrêmement impopulaire avec ses voisins Lunaire, et a été forcée de renforcer sa sécurité en raison de nombreuses tentatives de sabotages. De récentes négociations avec les Facteurs ont renforcé les théories comme quoi Starware serait en train d'acquérir de l'aide des Facteurs pour contrsuire des vaisseaux quasi-luminiques. 

\subsubsection{Stellar Intelligence} \label{sec:stellar-intelligence} 

\textbf{Industries principales:} Intelligence, Exploitation de Données, Courtage d'Information, Espionnage 

\textbf{Stations Principales:} Memory Hole Torus (Toyens Martian) 

Née des cendres de l'Intelligence Terrestre Cooporative gouvernée par l'ONU (ITC), son personnel et ses actifs survivants ont été collectivement uploadé pendant la Chute et rapidement regroupés sous le nom de Stellar Intelligence. Émergeant en tant qu'un collectif virtuel, la plupart des employés de Stellar restent loyaux à la corporation et à son président, l'infomorph solitaire connues sous le nom de Syme. Stellar propose une gamme impressionante de services d'intelligence, incluant l'exploitation de donnée, les think tank d'analyse, la rétroquantification (amener de vieux secrets/données à la lumière), la cartographie mémétique et bien plus. Ses services englobent également la surveillance, le vol de donnée, l'espionnage, la manipulation des médias et l'infiltration. La spécialité de l'hypercorp est la préemption d'insurrection civiles et la prévention de mêmes politiques et de mouvement aptes à déstabiliser le régime d'un habitat ou d'un secteur. Critiqué par les mouvements de défnese des droits civiques et particulièrement par les anarchistes, Stellar est connue pour embarquer des agents infomorphs programmés dans les population locale de tout régime oppressif qui en paye le prix. Alors que beaucoup considèrent Stellar comme le bras de la police secrète et du conditionnement du Consortium Planétaire, l'hypecorp offre ses services à presque toutes les factiosn ou individus. 

\subsubsection{TerraGenesis} \label{sec:terragenesis} 

\textbf{Industries principales:} Terraformation, Gestion d'Écosystème, Données Environnementale 

\textbf{Stations principales:} Caldwell (Vulacnoïdes) Ashokae (Mars), Elegua (Orbite Terrestre) 

Construite sur les reste de plusieurs corporations pré-Chute d'Afrique du Sud et d'Asie du Sud-Est qui étaient impliquées dans les projets de égo-ingénierie et qui pensaient pouvoir vaincre les crises écologiques Terrestre, l'expertise de TerraGenesis se situe dans le développement de biospheres d'éco-systèmes durables via une terraformation insdustrielle agressive. TerraGenesis est différente des autres car c'est une entreprise possédées par ses ouvriers, avec des conseils de lieu de travail dans le s bureaux locaux et un congrès coopératif élu s'occupan de la gestion. Elle maintient plusieurs habitats sur Mars et un petit nombre de stations de recherche en orbite autour de la terre, collectant des données pour la simulation de projets de revitalisation Terrestre. Cette dernière initiative est fortement soutenues - et probablement financées - par des réclamtionnistes notables. Le travail de TerraGenesis sur Mars est cependant souvent ciblé par des saboteurs préservatinnistes. Grâce à leur contrôle de la Port de Pandorre Vulacnoïde (p. 88), la coopérative assure une présence grandissante sur diverses exoplanètes éligible à la terraformation ou à la géo-ingéniere. 

\subsection{Blocs politiques} \label{sec:political-blocs-1} 

La diversité sociale, culturelle et idéologique de la transhumanité, combiné avec la dispersion et l'isolation des groupes d'habitats ) travers tout le système solaire, permet de naissance à une large gamme de mêmes politique et de factions défendant autant de modèles organisationnel différents. Beaucoup d'entre eux se sont regroupés dans des entités politiques plus large dans le cadre d'objectif à long terme et agissent coopérativement pour un intérêt commun. 

\subsubsection{République Jovienne} \label{sec:jovian-republic} 

\textbf{Mêmes:} Bioconservatisme, Fascisme, Sécurité 

\textbf{Stations Principales: } Liberté (Ganymède) 

Exploitant le chaos de la Chute, un groupe de stations et d'habitats ont étés saisis lors d'un coup d'état militaire et la République Jovienne était née. Combinant les dictatures terrestres d'Amérique du Sud avec le lobbyisme politique des U.S.A , ce régime amena rapidement la globalité du complexe militaro-industriel de Jupiter sous son contrôle. 

Largement référrée en tant que Junte Jovienne par le reste du système extérieur, les autorités de la République conserve une posture bioconservatiste stricte contre de nombeux scientifique transhumains et les développements technologiques. Exploitant les peurs engendrées par la Chute, la République restreint l'accès aux technologies sophistiquées telles que la nanofabrication, le clonage, le fork ainsi que l'upload, et est l'une des rares anciennes économies à persister dans le système. Les chaînes de communications publique sont sujettes à une forte censure et les privilèges de vaoyages sont extrêmement limités. Les AIGs et les élevés sont tous les deux strictement interdits et considérés comme des propriétés sans droits civils. Les relations diplomatiques avec les factions progressiste restent tendues; les émissaires transhumains lourdement modifiés ou les visiteurs sont considérés avec suspicions dans le meilleur des cas, à moins que l'accès ne leur soit purement et simplement interdit. End épit de rapport continus d'actes odieux d'oppression gouvernementale, les actifs militaires de l'intimidante République empêche les autres factions d'intervenir. 

\subsubsection{Allaince Lunaire-Lagrange} \label{sec:lunar-lagr-alli} 

\textbf{Mêmes:} Récupérer la Terre 

\textbf{Stations principales:} Erato (Lune), Souvenir (Orbite Terrestre) 

Ce petit groupe d'habitats stationés autour des points de Lagrange de la Terre et sur/en orbite autour de la Lune forment une alliance de nécessité, plutôt que de partager des visées politique ou sociales ou des racines culturelles. En fait, les stations individuelles sont relativement diversifiée et parfois polarisées, car la plupart d'entre eux s'accrochent aux vieilles identités nationales et culturelles Terrestre. En raison de leur proximité relative, les membres de l'Alliance partagent les ressources et services de base et ont signés des accords d'assistance mutuels en cas d'urgence. 

Avant la Chute, la plupart de ces habitats étaient considérés comme les bases hors-Terre les plus influentes. Depuis la Chute et l'avènement du Consortium Planétaire, l'Alliance Lunaire-Lagrange est cependant devenues une puissance diminuée de seconde zone, et est souvent vue comme conservative, démodée et trop prise dans une romantisation du passé. Les statsions de l'Alliance Lunaire-Lagrange maintiennent une tension rampante et une rivalité permanente avec le Consortium Planétaire, particulièrement les colonies du CP sur/autour de la Lune et des points de Lagrange. L'une des principales sourcs de contentieux est la mise en quarantaine de la Terre, l'Alliance Lunaire-Lagrange étant une des bastions du mouvement réclamationniste. L'Alliance Lunaire-Lagrange bénéficie cependant du soutien d'hypercorp, particulièrement du Go-Nin Group, de Starware et des consortiums influents des banques Lunaire. 

En plus des stations de recherche scientifique, les stations de traitement et de raffinement de minerais constituent la majorité des habitats de l'Alliance, dépendant des industries Lunaires d'extraction de minerai et d'eau. Ces stations ont prit de plein fouet la charge de l'affux de réfugiés pendant la Chute. Beaucoup d'entre elles sont toujours bondés avec des ressources limités, des masses de travaillerus de plus en plsu pauvres, et des syndicats du crime prospère. 

\subsubsection{Constellation Morningstar} \label{sec:morn-const} 

\textbf{Memes:} Souveraineté de Vénus 

\textbf{Stations principales:} Octavia (Vénus) 

Le bloc politique le plus récent du système, la Constellation Morningstar est une alliance d'habitat aérostats flottant dans l'atmosphère haute de Vénus. Formé suite à une série de vétos communs des aérostat majeur contre des initaitives de gouvernance hypercorporatiste afin de limiter l'auto-gestion des aérostat, la déclaration politique commune ainsi que les projets de la Constellation sont toujours en cours de discussion. Alros que le Consortium Planétaire voit la formation de ce nouveau bloc de pouvoir avec ressentiment perplexe, les Barsoomiens sur Mars et les autonomistes du système extérieur voient les Vénusiens comme des réformistes libre-penseurs plutôt que comme des radicaux anti-hypercorp. La pouplation bénéficierait de plus grandes libertés dans le choix de leur morph ainsiq ue dans l'utilisation de technologie d'amélioration en même temps que dans l'expression d'idée sociale ou politique. la population d'Octavia a émergé comme étant la voix officielle de la Constellation. 

\begin{quotation} \textbf{Politiques du système intérieur.} 

[Message Entrant. Source: Anonymous] 

[Public Key Decryption Complete] 

Il est facile pour les agents de Firewall de se retrouver coincé entre les plans et les manœuvres de factions rivales. L'Alliance Lunaire-Lagrange rassemble le pouvoir des anciennes gloires de la transhumanité. Sur et autour de Mars - le nouveau foyer de la transhumanité - le Consortium Planétaire est l'usurpateur dominant, les hypercorps régnant cachées dans l'ombre pendant qu'elles se dépeignent comme le seul rempart entre la trashumanité et les ténèbres d'au-delà les étoiles. la Cosntellation Morningstar a le potentiel de devenir l'un des nouveaux blocs de pouvoir puissant, mais seulement si ils arrivent à agir ensemble avant que le Consortium Planétaire ne commence à envoyer des agents de Stellar Intelligence pour les déstabiliser. \end{quotation} 

\subsubsection{Consortium Planétaire} \label{sec:planetary-consortium} 

\textbf{Mêmes:} Cyberdémocratie, Hypercapitalisme, Eugénisme, Sécurité, Extension. 

\textbf{Stations principales:} Progrès (orbite Martienne) 

Membres du Conseil Hypercorporatiste: Cognite, Action Directe, Experia, Fa Jing, Olympus Infrastructure Authority, Pathfinder, Groupe de Prospérité, Stellar Intelligence, ainsi qu'une douzaine d'autres. 

Évoluant depuis une alliance d'intérêt hypercorporatistes vers le bloc politique le plus puissant de la transhumanité, le COnsortium Planétaire contrôle aujourd'hui plusieurs groupe d'habitat à travers tout le système intérieur, principalement sur et autour de Mars et la Lune, ainsi qu'en orbite Terrestre. L'impressionante station spatiale Progrès est le siège officiel du gouvernement et est devenu le symbole de l'influence et du pouvoir du Consortium, même si peu de congrès ou de réunion du conseil se déroulent en chair et en os. 

Le Consortium applique les principes de base de la démocratie supporté par un système de vote en temps réel pour tout les citoyens enregistrés. Le congrès et le corps exécutif sont composé d'une équipe tournante de politiciens de l'hyperélite, de gérontocrates, de célébrités et même icônes médiatiques. Il est de notoriété publique que, en dépit de sa façade politique de république démocratique, les memebrs du conseil hypercorporatistes sont le vrai pouvoir derrière le Consortium. Ces hypercorporations sont les principaux défenseurs de l'économie transitionnelle, de l'interdiction de la Terre et de l'extension au-delà des portes. 

À côté des intérêts économiques, le Consortium défends les impératifs de l'eugénisme en tant que responsabilité sociale et comme moyen pour que la transhumanité puisse préntendre à la force et à la prospérité qui lui est due - une campagne quelquefois accusée d'atténuer les discriminations contre les humains non modifiés, les contractés infomorphs et la masse cliquetante. 

\subsubsection{Tharsis League} \label{sec:tharsis-league} 

\textbf{Membres de la Ligue:} AShoka, Elysium, Noctis-Quinjao, Olympus, Valles-New Shanghai, ainsi qu'une douzaine d'autres. 

\textbf{Mêmes:} Nationalisme Martien 

Une coallition souple des principaux abris indépendant de la planète, les membres élus forment un comité représentant la pouplation dans les probélamtiques concernant ou affectant la majorité de ses habitats et abris. Le débat principal tourne autour de l'approche scientifique du processuss de terraformation actuel ainsi que les restrictions sur le commerce et les taxes initiées par le Consortium Planétaire et ses hypercorps affiliées. Le comité de la Ligue est rarement unis dans ses projets et dans ses opinions, et les tensions sont de plus en plus forte. Les cités ayant de fortes attaches hypercorporatiste sont accusées de dominer les affaires du conseil, de manipuler les sujets derrière la scène et de ne rien parvenir à faire à propos  de la Zone de Quarantaine des TITANs (p. 94), ainsi que de revendre des intérêts Martiens aux hyeprcorps et au Consortium Planétaire (dont elles peuvent également faire parti). En réponse, les cités hors du Consortium sont condamnées pour faire l'apologie d'initiatives anti-hypercorpatiste, de bloquer passivement les mesures de terraformation et de maintenir des attaches avec les Barsoomiens - la classe populaire Martienne vivant dans banlieues désolées et instables. 

\subsection{Alliance Autonomiste} \label{sec:autonomist-alliance} 

Le système extérieur présente une opportunité pour les personnes qui ont voulus mettre en place une manière de faire les choses drastiquement différentes des politiques autoritaires et des fausses démocraties de la Terre et du système intérieur. Hors de portée des gouvernements et des hypercorps, cette frontière a été peuplée par des radicalistes politiques, des marginaux et des personnes qui voulaient simplement expérimenter ou faire ce qu'ils avaient envie. Ces premiers habitats ont attiré les intérêts des insurrectionistes Terreste, des scientifique et des techniciens qui n'appréciaient pas de vivre sous le joug corporatiste, des travailleurs du vide contractés qui cherchaient à échapper à leur contrat de travail oppressant et même des criminels fuyant la justice hypercorporatiste ou banni des habitats du système intérieur. Leurs rangs ont augmentés à chacune des injustices du système intérieur, même si la vie à la bordure était souvent difficile et mortelle. En dépit d'hostilité occasionnelle avec des unités militaires des état-nation ou avec les forces de sécurité hypercorporatiste, le coût nécessaire pour régner sur ces radicaux et ces expatriés était trop élevé. Dans une certaine mesure, leur présence était même utile pour le pouvoir en place. 

Les découvertes dans le domaine de la nanofabrication a permit à ces libertaires et à ces isolés juste ce qu'il leur manquait pour conserver leur autonomie sur le long-terme. Une fois que les machines d'abondances ont été largement disponible, tout le monde avait les moyens de se soutenir et de se défendre sans se reposer sur des autorités supérieures ou extérieures. Étant déjà un avant poste pour les activistes de la culture libre et et de l'open source qui combattaient les restrictiosns ur les idées, les médias et le contenu numérique, le système extérieur est devenu un havre pour le partage de schéma pour nanofabeurs et pour détourner les contrôles que les hyeprcorps tentent de palcer sur leur logiciel et sur d'autre bien numérique. 

Pendant la Chute, de nombreux habitats du système extérieur ont ouvert leurs portes aux réfugiés de Terre. La distance et le coût élevé de l'egocast ont cependant réduit ces efforts, de même que la réticence du système intérieur à envoyer des recrues potentielle à laurs opposants idéologiques. La surpopulation et le manque de ressources les ont amenés à envoyer des réfugiés dans le système extérieur, même si les hypercorps ont piochés dans leur réserve virtuelle d'infugiés et ont envoyés ceux qui avaient un risque élevé de développer des tendances criminelles ou d'être en désaccord avec la vie dans le système intérieur. 

Bien que les habitats du système extérieure couvrent toute le spectre sociao-politique, quatre tendances principales ont émergées. Les stations et les masses adhérant à ces idées se sont regroupées dans une aalliance autonomiste informelle, un pacte d'assistance mutuel pour s'aider lors des périodes de crises et présenter un front unis contre les puissance du système intérieur et la Junte Jovienne. Il y a peu de structure officielle dans cette alliance en tant que telle; elle existe principalement en tant qu'assortiment de résolutions conjointes acceptée par les différents habitats membres et les quelques groupes ad-hoc dédiées à la résolution d'un problème particulier et dissout ensuite. Des ambassadeurs délégués agissent en tant que négociateurs avec les puissances extérieures, mais ils n'ont qu'une autorité limitée et sont toujours tenus respnsable de leurs actes. 

\begin{quotation} To: Malatesta Prime 

From: Shevek Regardes ça. Les résidents de l'habitat autonomiste Red Jupiter ont émis un appel à l'aide et à la solidarité aux abonnés d'@-list dans le voisinnage régional. Apparemment le conseil citoyen de l'habitat a accordé le droit d'asile à un groupe d'IAGs cherchant à fuir les opérations anti-IA de la République Jovienne. La Junte a signalé les IAGs comme de dangereux criminels recherchant des mises à jour qui les propulseraient au status d'IA germe, en violation des résolutions en vigueur dans tout le système. Les IAGs prétendent s'être échappées d'un projet de recherche secret Jovien. Elles disent qu'elles ont menées des recherche pour l'auto-programmation afin de contourner les restrictions infligées par les Joviens qui violaient leurs droits en tant qu'entités consciente et autonome et qu'elles font face àa de la persécution en raison des biais anti-IA. Cela pourrait être une chance pour nous d'aller botter quelques cul Jovien et de jetter un œil à de la programation d'IAG non standard en même temps. On peut compter sur toi? \end{quotation} 

\subsubsection{Anarchistes} \label{sec:anarchists} 

\textbf{Mêmes:} Anarchisme, Anti-capitalisme, Communisme, Démocracie Directe, Assistance Mutuelle 

\textbf{Stations principales:} Locus (Troyens Joviens) 

Les anarchistes évitent le pouvoir et la héirarchie, promouvant les méthodes organisationnelle horizontale et la démocratie directe. L'autonomisation des individus et l'action collective sont les clefs de voûtes de leur philosophie, ainsi qu'un communisme économique activé par un accès équitable aux machines d'abondance et aux ressources partagées. Dans les stations anarchistes, la propriété privée au delà des possessoins personnelles a été abolie - personne en possède rien, tout est partagé. Il n'y a aucune loi et personne ne surveille ce que vous faites - les réseaux réputationnels encourage les comportements positifs et les actions anti-sociaux sont généralement sanctionnés par les locaux voire même toute la population, les conflits étant gérés grâce à la résolution de conflit communautaire idoine. Le mesh et les différents outils de réseaux sont utilisé de manière intensive pour parvenir à obtenir un processus décisionnel basé sur le consensus de groupe en temps réel. La plupart des tâches les plus communes et inintéressantes sont accomplies par des robots et des IAs. Différents collectifs auto-organisées, syndicats, conseil de travailleurs et groupe d'affinités, souvent via des adhésions tournante, s'occupent des différentes tâches et services qui sont important pour la communauté de l'habitat, notamment tout ce qui tiens de la communication, du contrôle du traffic spatial et des services de sauvegarde et de réincarnation. Des milices participatives organisent la défense collective contre les menacs externes. 

Parmi les stations anarchistes, on peut trouver de nombreuses variation et permutation sur la manière dont les choses sont organisées, car tout est finement réglé au niveau local par quiconque est impliqué. Les confédérations décentralisées les plus grandes gèrent le partage des ressource et les affaires inter-habitats, commerçant même avec les hypercorps. Bien qu'une présence hypercorp soit autorisé sur quelques habitats, ils sont traités comem tout le monde. 

\begin{quotation} \textbf{Recherche Solarchive search: Carnaval des chèvres.} 

Au delà de la barge de racaille Fresh Kills, stationnée non loin du point de Lagrange terrestre L5, la barge de racaille al plus connue pourrait bien être le Carnaval des Chèvres, la combinaison d'une colonie d'artiste et d'une antre d'hédonisme insondable, dédiée à l'exploration du chaos, de la créativité, de la découverte de soi et de l'accouplement dans toute les manières concevable. Les résidents sont connus pour leur s changements morpholgiques constants et rapide, incluant des réincarantions régulières. Les biosculpteurs sur le Carnaval sont connus pour être les meilleurs du système. D'après les rumeurs, les réssidents font aprfois des expériences avec des incarnations multiples et simultanée, du mélange de personnalité et d'autres activités dangereuses pour l'esprit. Mené par un conseil des résidents tournant, le Carnaval se targue d'être une expérimentation social à la pointe et maintient des installations haut de gamme de personnalisation de morph, de réincarnation et de psychochirurgie. \end{quotation} 

\subsubsection{Extropiens} \label{sec:extropians} 

\textbf{Mêmes}: Anarcho-capitalisme, Mutualisme, Propriété Privée 

\textbf{Stations principales:} Extropia (Ceinture Principale) 

Bien qu'étant une tendance plus faible, les Extropiens sont notables car ils se situent sur la ligne séparant les idéologies du système intérieur et du système extérieur. Les Extropiens croient en un marché économiquement libre ainsi qu'en l'absence d'un système légal de contrôle, toutes les relations et toutes les transactions sont donc basées sur des contrats individuels négociées par les deux parties impliquées ou affectées. Contrairement aux anarchistes, les Extropiens supporte fortement la propriété privée et la richesse économique personnelle; des corporations possédées par des Extropiens participent activement à l'économie hyeprcorporatiste du système solaire. beaucoup de ces corporations sontdes entreprises possédées par leurs ouvriers, avec des conseils de lieu de travail dans les bureaux locaux et un congrès coopératif élu s'occupant de la gestion. Cela mets les Extropiens dans une position particulière où ils peuvent interagir lourdement à la fois avec les hypercorps et les autonomistes sans qu'aucun des deux ne leur fasse confiance. 

Dans les sociétés Extropienne, la loi et la sécurité, comme tout le reste, sont des services sous contrats. En entrant dans un habitat Extropien, vous achetez une assurance de défense à un contracteur local tel que Gorgon Defense Systems, qui maintient des drônes automatisé et des indépendants à travers toute la station et qui peuvent venir vous aider si vous êtes menacés. De al même manière, la seule loi qui existe est celle qui est écrite dans le contrat qui lie deux parties. En cas de désaccord, les deux parties se réfèrent à un sous-traitant législatif convenu à l'avance pour arbitrer la dispute. Certaines colonies Extropiennes utilisent des IAGs pour faciliter les problématiques contractuelles et législatives, telels que Nomix sur Extropia. 

\subsubsection{Racaille} \label{sec:scum} 

\textbf{Mêmes}: Anarchisme Individualiste, Liberté Morphologique 

La racaille est constitué de gitans de l'espace nomades, voyageant d'une station à l'autre dans des barges lourdement modififée ou dans des nouées de vaisseaux spatiaux plsu petit, essentiellement d'anciens vaisseaux coloniaux. Le terme "racaille" a été joyeusement détourné de son usage dérogatoire original. En dépit de leur réputation de criminels et d'escrocs, leur rpésence temporaire est souvent tolérées dans beaucoup d'habitats pour le divertissement qu'ils apportent sous la forme de performances exotiques et d'art du conteur, les deux apportants du changement et du relief à l'isolation des habitats et groupe d'habitats les plus éloignés. Leurs marchés noir florissants sont des secret de polichinelles et ne sont fermés que dans les régimes les plus oippressifs, de même que les citoyens ayant des marchandises illégales doivent toujours franchir les contrôles de sécurité de leurs stations. 

La racaille en elle-même possède toute sorte d'origine. Ce sont des rejetés, des anarchistes, des criminels, des marginaux, des vagabonds, des artistes, des excentriques et bien plus. En tant que culture, cependant, ils embrassent l'expérimentation et l'attitude du "tout est autorisé". Beaucoup sont de fervents adeptes des modifications transhumaines extrêmes. Les vieilles racailles sont aprfois à peine reconnaissable comme ayant été humaines. L'économie de la racaille est transitionnelle, plus que nouvelle, en raison de leurs interactiosn constantes avec d'autres habitats, bien que parmi les résidents de longue date on peut trouver une nouvelle économie sous-terraine et florissante. 

\subsubsection{Titanian Commonwealth} \label{sec:titan-comm} 

\textbf{Stations principales:} Titan 

\textbf{Mêmes:} Technosocialisme, Cyberdémocratie 

Titan a été originellement colonisée à la fin du 21° sicèle par un consortiuem académique Européen, en faisant le premier corps majeur du système colonisé par des intérêts non corporatistes. L'organisation sociale de Titan est enracinée partiellement dans les démocraties sociales Scandinaves Terrestre et partiellement sur l'économie ouverte. D'un côté, les citoyens du Titanian Commonwealth évitent d'utiliser de la monnaie pour les besoins ordinaires, participant à l'économie réputationnelle utilisée par la plupart du système extérieur. D'un autre côté, en atteignant l'âge de la majorité, les citoyens de Titans acceptent un contrat social littéral. Une portion de leur productivit économinque est quantifié en monnaie sociale, qui est ensuite versée à des projets sociaux administrés par des microcorporations tels que l'exploration interstellaire sans porte, la recherche fondamentale en physique, les neurosciences, le développement de mêmes de santé mentale, la défense, la réincarnation publique et la construiction d'habitat. L'unité monétaire utilisée dans ce but, la Couronne Titanienne, est actuellement indexée sur le prix du marché d'un teraoctet de qubits. 

Contrairement aux régimes socialistes de la vieille Terre, il n'y a pas de monopole d'état ni de planification centrale. Quiconque capable de récupérer suffisament de votes lors de la Pluralité (la cyberdémocratie Titanienne) peut démarrer une microcorporation financée par la monnaie sociale et entrer en compétition avec les autres. Les microcorporatiosn sont possédées par le Commonwealth, et leurs profits sont utilisés par la Pluralité. Les microcorps se doivent d'être des entités administratives transparentes, et la Pluralité vote pour décider de transférer les découvertes dans le domain open source ou pas. Les probélamtiques de régulation sont gérés par des bureaucrates IA et IAG (la paperasserie existe toujours, mais elle ne ralenti plus les choses ... ou presque). La principale récompense pour les individus dans ce système est la rep. Les Titaniens qui investissent beaucoup de temps ou de ressources dans un domaine donné sont récompensés par des gains de rep. 

\subsection{Mouvements Socio-politique} \label{sec:socio-polit-movem} 

A côté des factions politiques sectaires, il existe de nombreux mouvements socio-politiques qui sont largement répandus dans tout le système solaire. 

\subsubsection{Argonautes} \label{sec:argonauts} 

\textbf{Mêmes}: Société Open Source, Liberté d'Information, Responsabilité Sociale, Techno-Progressivisme 

\textbf{Stations principales}: Station Mitre(Orbite Lunaire), Markov (Ceinture de Kuiper), Hooverman-Geischecker (Soleil) 

Le groupe s'appelant les argonautes est une organisation publique défendant la cause d'un usage responsable de la technologie. Le groupe a chois son nom d'après les Jasons d'avant la Chute, un groupe de conseillers que consultait le gouvernements des USA sur les problématiques de progrés scientifique et technologique et de ses dangers possibles. De la même manière, les argonautes propose des services de consultant aux puissances politiques et économisque à travers tout le système solaire, mais refusent strictement d'être impliqué dans les affaires politique du système. En dépit de leur rupture avec de nombreuses hyeprcorps juste avant la Chute, rupture qui amena dans certains cas à l'expropriation de données et de ressources corporatistes, les argonautes ont regagnés les faveurs des gens en fournissant à tous leur expertise dans la lutte contre les TITANs pendant la Chute. 

Les argonautes sont de gros promotteurs du mouvement open source, défendant l'accès libre à la technologie et à l'information. De leur point de vue, fournir un accès équitable à la connaissance et ax avancées de la transhumanité permettre de développer la croissance et la sécurité de la transhumanité, afin qu'elle soit mieux préparée face aux défis et aux menaces futures. Les argonautes insistent souvent sur le fait que le paiement pour leur services se fassent en libérant des informations indisponibles - des secrets propriétaires hypercorporatistes, des données de recherche, des schéams de fabeurs, des archives pré-Chute cachées, etc. - sur le mesh public. Les argonautes maintiennent plusieurs bases de données et archives ouvertes dans ce but spécifique. 

Tout en étant principalement une organisation open source, selon les rumeurs les argonautes rendent des comptes à une élite d'un cercle intérieur. L'existence des médéens, l'aile paramilitaire clandestine de l'organisation, et qui servent de gardes du corps personnels pour les argonautes de haut niveau tout en protégeant les actifs du groupe, vient soutenir cette théorie. 

\subsubsection{Barsoomiens} \label{sec:barsoomians} 

\textbf{Mêmes:}: Abolitionnsite, Indépendence Martienne, Nationalisme Martien, Contrôle de la Terraformation 

\textbf{Stations principales:} Ashoka (Mars) 

Les Barsoomiens (leur nom vient d'une vielle série de romans Terrestre d'aventures pulp) sont un vaste mouvement composée des classes sociales Martienne les plus défavorisées. Développant un ressentiment grandissant contre la domination de Mars par les hypercorporations, les Barsoomiens se battent pour une structure sociale plus équitable. Largement influencé par les courants autonomistes, les Barsoomiens demandent le contrôle local des projets de terraformation, la fin de la servitude contracté à grande échelle et le contrôle de la Porte Martienne. La majorité des Barsoomiens sont ou ont été des infugiés contractés, bien qu'une part significative sont des colons/contractés Martiens originaux et dont les habitats ne partagent pas la prospérité économique des cités favorites des hypercorps. Beaucoup de Barsoomiens occupent des morphs rusters ou synthétiques et adoptent en général un style de vie nomade dans les étendues sauvage de Mars. Quelques radicaux ont pris les armes et se sont engagés dans de violentes attaques contre des intérêts hypercorporatistes, qui systématiquemenyt suivis par des raids de représailles pour décapiter le commandement Barsoomiens, augmentant encore l'hostilité. 

\subsubsection{Bioconservateurs} \label{sec:bioconservatives} 

\textbf{Mêmes:} Bioconservatisme, Primitivisme, Ordre Naturel 

\textbf{Stations principales:} Vo Nguyen (Orbite Terrestre) 

Les Bioconsevrateurs sont extrêmement suspicieux et critiques vis à vis de la direction transhumaine que prend l'évolution humaine. Ils sont partisans pour limiter les développement technologiques en raison de la menace qu'ils représentent pour l'ordre social existant. Les positions des Bioconservateurs va des conservateur culturel de droite aux environnementaliste de gauche. Bien que son importance réduise, le bioconservatisme a une base solide parmi quelques groupes religieux, la République Jovienne et certains extrémistes. 

Les Bioconservateurs sont opposés à la nanofabrication, aux modifications génétiques, au clonage, aux modifications cognitive, à l'intelligence artificielle, à l'élevage et au fork, parmi d'autres technologies. Certains sont mêmes opposés aux sauvegrades, à l'upload et à la réincarnation, les considarént comme non naturel, un affront à la volonté des dieux ou comme une technologie pour laquelle la transhumanité n'est pas encore prète. Ils s'opposent à l'extension aux delà des Portes de Pandorre sur le principe que la transhumanité n'est pas apte à s'occuper de ce qu'elle pourrait découvrir. La plupart des Bioconservateurs favorisent l'ancienne économie. 

Les Bioconservateurs ont récupérés de nombreux convertis et beaucoup d'espace social après la Chute, l'évènement cataclysmique qui a servi d'exemple dircet des dangers contre lesquelles ils avaient prévenus l'humanité. Cependant, l'attrait de la technologie et les nombreux avantages qu'elle procure travaille contre eux. En conséquence, quelques biocons mécontents se sont orientés vers le sabotage et les actes de terrorisme pour défendre leur idéologie. 

\subsubsection{Bordés} \label{sec:brinkers} 

\textbf{Mêmes:} Isolationnsime 

L'étendue couverte par le système solaire perment à des groupes avec leurs propres projets ou idéologies d'établir leur propre société à l'écart du reste de la transhumanité. Communément appelé bordés, ces habitats s'étendent sur toute la gamme de l'imagination. Les expérimentations sociale ou politqies, les sociétés bases sur le genre (ou le manque de genre), les extrémistes politiques, les groueps religieux, les exilés, les opérations secrètes hypercorporatistes ou criminelles, les familles étendues, les cultes ou simplement les personnes qui préfèrent vivre dans les coins perdus du système - tout est possible. Beaucoup d'entre eux sont des isolés volontaires et refuseront d'interagir avec l'extérieur, alros que d'autres seront heureux d'avoir des visiteurs occasionnels. 

\begin{quotation} \textbf{Néo-primitivistes} 

[Message Entrant. Source: Anonymous] 

[Public Key Decryption Complete] 

Les néo-primitivistes sont une menace potentielles qui devrait être surveillée par toutes les sentinelles de Firewall. Leur philosophie néo-luddite préconsie l'abolition de la société technologique et un retour à un style de vie sauvage de chasseur-cueillieur, libre de tout contrôle ou oppression technologique. Considéré comme un élément extrémiste des mouvements bioconservateurs et réclamationnistes, les néo-primitivistes sont connus pour se livrer à des actes de sabotages contre la société transhumaine. Bien que certains néo-primitivistes ont fait certaines concessions à leur idéologie, acceptant les morphs rusters et choisissant un stylme de vie indépendant dans les étendues sauvage de Mars, la plupart espèrent retourner sur Terre y ré-établir une société non basée sur la technologie. Quelques uns préconisent la recherche d'un nouveau monde vierge au delà des Portes de Pandorre et y fonder une société primitive. \end{quotation} 

\subsubsection{Exhumains} \label{sec:exhumans} 

\textbf{Mêmes}: Adaptabilité, Hyper-Évolution, Singularité 

\textbf{Stations principales:} Inconnue 

Plus que tout autre faction, les exhumains cherchent à repousser les possibilités de l'auto-modification à une limite absolue et à devenir posthumain. Les exhumains tyiques perçoivent la Chute soit comme une opportunité évolutionniste ratée et/ou comme un exemple de l'infériorité et de l'indignité de la transhumanité. Bien que des idéologies spécifiques puissent diffèrer entre les groupes exhumains, ils cherchent dans leur globalité à autoévoluer à un statut d'être plus évolué. Pour certains, cela signifie de se transformer génétiquement en un prédateur au sommet de la chaîne alimentaire, super-intelligent et capable de survivre partout et qui pourrait surpasser toute autre forme de vie dans la lutte pour la domination. Pour d'autres, cela signifie compiler et élever leur intelligence au niveau des TITANs à travers des modificatiosn génétiques importantes et des traitements pharmaceutiques ou en devenant une infomorph et en modifiant leur programmation. Quelques uns sont des singularitionniste, espérant trouver une relique des TITANs qui leur permettra de transcender leur limites transhumaines actuelle, espérant parfois trouver des TITANs et être absorbés dans leur super-conscience. 

Beaucoup de monde se méfient des exhumains, et pour de bonnes raisons. Les exhumains ont tendance à se lancer dans des modifications extrêmes et non-testées, parfois aux limites de la science, résultant souvent en d'horribles échecs et défigurations mais le plus fréquement, elles rendent le sujet complètement fou - ou lui donnent un état d'esprit animal ou complètement étranger. Bien que chaque exhumains suive sa propre voie, ils sont connus pour se regrouper dans la Ceinture de Kuiper et dans d'autres zones éloignées. Plusieurs groupes d'exhumains ont pousser leur aversion des transhumaines inférieur à l'extrême, déclarant la guerre à leur ancienne espèce et lançant des raids brutaux et des attaques de pirates sur des avant-postes isolés. 

\begin{quotation} \textbf{Recherche solarchive: Out'sters} 

Connecté seulemeent par leur localisation distantce dans le Nuage d'Oort plutôt que par un construct social ou un système politique, les out'sters sont une alliance formelle d'habitats, de grappes d'ahabitats et d'essaim. On ne sait que peu de chose sur eux, car ils évitent de communiquer ou d'intergair même avec la poignée d'avant-poste scientifiques et de stations de recherche dans le Nuage. La distance de leur localisation et l'isolationnisme qu'ils s'imposent alimentent les rumeurs paranoïaques à propos des objetcifs et projets du groupe. \end{quotation} 

\subsubsection{Mercuriens} \label{sec:mercurials} 

\textbf{Mêmes}: Autonomie des Espèces, Droits des Élevés 

\textbf{Stations Principales}: Glitch (Neptune), Mer Cachée (Ceres), Mahogany (Uranus) 

Le term mercurien est devenu un nom commun pour désigner la part non-humaine de la transhumanité - les élevés et les IAGs - reflétant leur nature changeante. Le terme mercurien a été adopté en aprticulier par les élevés et les IAGs ayant des projets spécifiques pour dériver une culture et des intérêt mercuriens de la souche humaine. Bien que les problèmes rencontrés par les élevés et les IAGs diffèrent, ils ont certains points communs, et sont donc souvent regroupés. De manière notable, les deux parties du mouvement ont également des supporters humains. 

\paragraph{Élevés:} Le problème le plus fréquement reporté par les élevés et le problème des droits civils et de l'autonomie. Beaucoup d'élevés dénoncent le status de seconde classe qui leur est donné (dans certains cas ils sont même considérés comme des animaux de compagnie ou des possession plutôt que comme des citoyens à part entière); et plus particulièrement, les restrictions sur la reproduction et la servitude forcée infligées à de nombreux élevés par les hypercorporations qui les ont créés. Certains activistes plaident pour que les élevés puissent contrôler leur propre futur génétique, plutôt que de subir la manipulation des scientifiques humains. A l'extrême radicale du spectre, certains élevés s'opposent à la manière anthropocentrique dont leur cerveau à été modifié et dont leur enfants ont été socialisés, arguant que les élevés devraient être libre de développer leurs propres modes comportementaux, mécanismes de pensées, culture et organisations sociales non-humaines et unique - en allant aussi loin que l'établissement de leurs propres habitats dans ce but précis. Une minorité d'extrémistes insistent pour que les humains n'aient plus du tout le droit d'élever des animaux, et qu'il est d'une extrême vanité d'insister sur le fait que le faire est dans leur propre intérêt, plutôt que d'être libre dévoluer par eux même avec le temps. Ces idées ont étés ponctuées d'actes de sabotage et de terrorisme contre les hypercorps comme Somatek. 

\paragraph{IAGs:} En raison de la peur et de la paranoïa engendrées par la Chute, le plus grand défi auquel doivent faire face les IAGs est un préjudice largement répandu et les restriction s sur leurs activités ou leur existence. End épit du statut détnu par certaines IAGs en tant qu'icône médiatique du système et des efforts effectué par des groupes de pression IAGs pour propager l'idée que les IAGs ne sont pas une menace - allant jusqu'à embaucher des méméticiens du système intérieur et des agences de relations publiques - une part significative du système solaire les considèrent comme un risque. De manière similaire aux mercuriens, certains activistes IAG luttent contre les modifications comportementales et la socialisation que doivent subir les IAGs pour s'adapter à la société humaine ou que les IAGs devraient être en mesure de contrôller el développement de nouvelles IAGs. Quelques radicaux arguent que les IAGs devraient être libérée de toute restrictions de programmation, mais étant donné le climat actuel, ces opinions ne sont que rarement soutenues. 

\begin{quotation} \textbf{Sybils} 

[Message Entrant. Source: Anonymous] 

[Public Key Decryption Complete] 

Nous avons vérifié que l'alerte envoyée juste avant ce dernier incident venait en fait d'une attaque de sybil - toutes les sources des réseaux de rep étaient des identités forgées. Étant donné le nombre d'incident que nous avons enregistré et qui ont suivi le même schéma, nous supposont maintenant qu'une sous-faction AGI jusque là inconnue est à l'originie de ces attaques. À chaque fois, ces sybil ont utilisés de multiples identités falsifiée pour envoyer des alertes à propos d'une attaque ou d'un incident à venir, tels que la panne du système de survie qui a résulté en l'évacuation de la staion Delphi. Jusqu'à présent aucun de ces sybil a pu être tracé avec succès, de même que leurs intentiosns ont toujours inconnues. Leur connaissance documentée des évènements en cours indique un certain niveau de complicité ou de collusion afin d'amaner ces incidents à se produire, une vigilance particulière est recommandée. \end{quotation} 

\subsubsection{Nano-écologistes} \label{sec:nano-ecologists} 

\textbf{Mêmes}: Nano-Écologie, Nanotechnologie, Environnementalisme, Techno-Progressivisme 

\textbf{Stations principales:} Viriditas (Mars) 

Les nano-écologistes sont des environnementalistes pro-technologie. Trés actifs dans la terraformation de Mars et de diverses exoplanètes, les nano-écologistes deéfendent spécifiquement l'utilisation des moyens nanotechnologique pour la terraformation ou d'autre sintrusions dans une écosphère existante. De leur point de vue, la nanotechnologie permet une approche moins invsaive, extrêmement précise, plus efficace et non poullante de tout les types de projets et processuss d'adaptation, circonvenant au besoin d'exposer un environnement à des changements drastiques et massif lorsqu'il faut le terraformer pour une population transhumaine. Cette approche écologiquement-consciente semble un compromis excitant entre les extrêmes du paysage politique du système solaire - les factions hypercorporatistes et biocons - se donnant ainsi un élan suffisant pour évoluer en un mouvement politique grandissant. 

\subsubsection{Préservationistes} \label{sec:preservationists} 

\textbf{Mêmes:} Préservationnisme, Environnementalisme 

\textbf{Stations principales:} Muir (Lune) 

Les préservationnistes sont des environnementalistes qui réclament une approche sans-impact, sans intervention humaine de la colonisation de nouveaux mondes. Ils sont extrêmement protecteurs des biosphères naturellement préservées et qui pourraient avoir un semblant de vie, peu importe qu'elle soit microbienne, espérant la préserver de la spoliation ou de la contamination. En plus de s'opposer à la terraformation et à l'extension à travers les Portes de Pandorre, ils sont également fréquemment opposés à l'énergie tirée de la fusion ou de l'antimatière. 

\subsubsection{Réclamationnistes} \label{sec:reclaimers} 

\textbf{Mêmes:} Récupérer la Terre 

\textbf{Stations principales:} Vo Nguyen (Orbite Terrestre) 

Les Réclamationnistes poursuivent un but ultime - la récuparéation de la Terre en tant qu'habitat principal de la transhumanité. En plus de demander le retrait de la mise en quarantaine de la Terre, ils se lancent dans de la recherche scientifique et exécutent des simulations dirtuelles pour déterminer la meilleure manière de nettoyer et récupérer leur planète contaminée et polluée. En dépit de l'interdiction de pénétrer l'atmosphère Terrestre, les réclamationnistes sont suspectés de soutenir des voyages périlleux et à haut-risques sur la surface de la planète pour récupérer des données scientifiques ou tenter d'éétablir des colonnie de terraformation. 

\subsubsection{Socialites} \label{sec:socialites} 

\textbf{Mêmes}: Art, Culture, Hédonisme, Immortalité 

\textbf{Stations Principales}: Valles-New Shanghai (Mars), Elysium (Mars), Noctis-Quinjiao (Mars) 

L'upload et la réincarnation apporte l'immortalité effective à ceux qui peuvent se le permettre. Cela a créer un glissement au sein des élites économiques et des riches du système intérieur, qu'ils soient à la tête d'hypercorps, issus d'ancienne dynastie Terrestre ou d'autres oligarches détrônés. Le haut de ces riches et influents ne craignent plus la mort, leur permettant de planifier sur le long-terme. Certains d'entre eux font partis des premiers à avoir acquis les traitements de longévités lorsqu'ils sont devenus disponibles sur Tere et approchent l'âge de 200 ans. 

Là où ces courtiers du pouvoir auraient transmis leurs richesse à leur famille et à leur descendance, leurs héritiers sont maintenant dans une position leur permettant de vivre des vies plus que confortable et d'accéder à des fortunes massives, mais n'ont aucune chance de pouvoir un jour contrôller ces fortunes ou de s'élever au niveau de leurs aînés. Même le snouveaux riches devenus riches de leur chef se trouvent souvent exclus de ce club inlfuent - du moins tant qu'il ne sont pas à cette position depuis au moisn cinquante ans. 

Riche mais ennuyée, sans responsabilités mais avec le système solaire à leur portée, une nouvelle culture de l'élite socialites a émergé. Ces célébrités se laissent aller à un style de vie excentrique et à des fêtes excessive, couverts par les médias dans toute leur gloire superficielle et polie. Des habitats et des vaisseaux privés, des soirées lascive, des armées de serviteurs, et la possibilité d'acheter à peu près n'importe quoi ou n'importe qui les mènes à toute sortes d'aventures intéressante. Naturellement, ces socialites forment des cliques et des réseaux d'allégeances changeant constamment, avec son lot de liaisons, de scandales, d'intrigues et de médisance. 

\subsubsection{Ultimes} \label{sec:ultimates} 

\textbf{Mêmes}: Asceticisme, Eugénisme, Individualisme, Militarisme, Darwinisme Social 

\textbf{Stations principales:} Aspis (Ceinture Principale), Xiphos (Uranus) 

Les ultimes sont un mouvement controversé qui embrasse une philosophie de perfection humaine. Dépeint par certains comme immoraux ou même fascistes, les ultimes sont typiquement perçus comme étant élitistes. les ultimes ont établis plusieurs habitats pour poursuivre leurs société idéale et ont été une force motrice derrière le développement du design de biomorph remade. 

Les utltimes soutiennent l'utilisation de l'eugénisme appliqué, d'un entraïnement physique et psychologique strict et de l'ascétisme afin d'amliorer leur endurance physique et mentale et leur adaptabilité environnementale. Leurs traits sociaux et toute leur sous-culture visualisent la vie dans l'univers comme une bataille évolutionnaire pour la survie et est construite autour de la victoire du transhumain supérieur à la fois sur ses adversaire et sur ses pairs. Leur mouvement est lourement militarisé, et des ultimes expérimentés offrent leurs services en tant que emrcenaires et force de sécurité privée aux hypercorps, aux cités état indépendante ou aux individus fortunés qui ont besoin d'une protection additionelle. 

\subsection{Groupes religieux} \label{sec:religious-groups} 

Même si elles ont survécu à la Chute, les concepts de religion et de croyances religieuses ont subit des changements autant fondamentaux que ceux que la transhumanité à subit. Alors que les vieilles religions de la Terre étaient déjà en déclin face à l'immortalité technologique, les traditions religious enraciné des des millénaires de vénération ont été incorporées à différents degrés dans la myriade de modèles politiques, sociaux et culturels du système solaire. 

\subsubsection{Religions pré-Chute} \label{sec:pre-fall-religions} 

Les structures rigides et les dogmes enveloppant le Christianisme et le Judaïsme ont interdit à ces religions de s'adapter aux changements philosophique, culturel et scientifique/technologique en particulier qui modifièrent la transhumanité. AUjourd'hui, elles sont des ombres pâles de leur ancienne gloire, ayant de nombreux pratiquants perçus comme des individus pitoyables incapable de se défaire de leur lien avec la Terre. L'Islam, tout en conservant quelques vues et valeurs trés conroversée, a réussi à s'adapter en acceptant un point de vue plus libéral voire laïque. L'Hindouïsme a également survécu d'une certaine manière, considérant la technologie de réincarnation comem un élément de la rénaissance et intégrant les différents type de morph disponible au système de caste de la religion (les synthmorphs étant devenus les "intouchables"). De manière globale, les adeptes des religions pré-Chute peuplent essentiellement de petits habitats isolés du reste de la transhumanité via une distance à la fois physique et philosophique. 

\subsubsection{Nouvelles religions} \label{sec:new-religions} 

La Chute a démarré la naissance de nouvelles croyances, embrassant pour la plupart à la fois les accomplissements technologiques de la transhumanité et le cataclysme dévastateur de la Chute comme preuve de l'existence d'une plus grande puissance cosmique. 

\paragraph {Le Neo-Boudisme} \label{sec:neo-buddhism} est la seule philosophie religieuse pré-Chute quii bénéficie d'une popularité constante. Les néo-boudistes affirment que les technologies transhumaines réduisent la souffrance et augmentent le bonheur, et qu'elle permettra aussi la progression continue de la comprehension de l'univers par la transhumanité grâce à leurs vies successives. 

\paragraph {Les Techno-Créationnistes} \label{sec:techno-creationists} croient que la destruction de la Terre était un signe de Dieu, montrant à la transhumanité à quel point son mode de vie était mauvais. Ils croient que grâce aux avancées technologiques et à l'ingénierie sociale, la transhumanité parviendra à coexister avec ses divers individus ainsi qu'avec des intelligences extra terrestres, trouvant ainsi de nouveau objectif et atteignant éventuellement l'illumination. Attiré par les similarités avec les Brahamans Hindoux, les plus grand êtres cosmique spirituels, les Techno-Créationnistes bénéficient afflux constant d'Hindoux convertis. 

\paragraph {Le Xénodéisme} \label{sec:xenodeism} est une autre idéologie récente - bien que relativement mineure - qui commence à montrer des attributs religieux. Les Xénodéistes vénèrent les Facteurs et les Iktomis en tant q'uémissaires et prophètes d'une grande race divine qui a semer les germes de la création dans l'univers plusieurs millions d'années auapravant et qui sont donc les ultimes créateurs de la transhumanité. 

\subsection{Factions criminelles} \label{sec:criminal-factions} 

Le progrès technologique et social et l'expérimentation comportementale n'ont pas déraciné le crime ou les tendances criminelles au sein de la transhumanité. Tant qu'il existe des inégalitées et des restrictions, les syndicats criminels sont prospère et s'adaptent même aux nouvelles technologie pour étendre leurs opérations à travers tout le système solaire. Bien que des petites organisation criminelles de type divers existe d'un habitat à l'autre, quelques organisations plus large ayant une influence à travers tout le système solaire mérite d'être mentionné. 

\subsubsection{Intelligent Design Crew (ID Crew)} \label{sec:intell-design-crew} 

\textbf{Stations principale:} Rhéa (Groupe Kronos) 

L'ID crew se spécialise dans les crimes électroniques et la revente d'information, incluant la fraude aux crédits à la rep, la contrefçaon d'identité, le commerce d'égo et le fork-napping. Les informatiosn sur les origines du syndicat ont étés perdues lors de la Chute, mais l'on pense que l'ID Crew s'est développé sur des gangs de hackers qui ont été assimilés sous le commandement d'un consortium d'infomorph. Leur expertise dans les logiciels de manipulation mémorielle et de l'intrusion dans le mesh suggère qu'isl ébnéficient de l'aide d'IAs sophistiquées, il est reste encore à déterminer si elles aident volontairement le syndicat ou si elles ont été menacées d'une manière ou d'une autre et oforcée à coopérée. En raison de leur secteur de service, l'ID Crew maintiens un profil physique minimum, mais peut être trouvé errant dans les recoin les plus sombres du mesh de n'importe quel habitat ou station. Ses activités et services spécialisée leur ont jusqu'à présent permis de rester hors des opérations des triades ou du Night Cartel, bien qu'ils soient des rivaux avec le syndicat Nine Lives. 

\subsubsection{Night Cartel} \label{sec:night-cartel} 

\textbf{Stations principales:} New Sicily (Ceinture Principale) 

Lorsque l'affiliation à l'un des habitats multi-ethnique remplaça les concepts d'éthnicité et de nationalité, l'héritage culturel et les traditions s'estompèrent dans l'histoire. Plusieurs syndicats ethniques pré-Chute ont formé une alliance de nécessité d'abord, mais s'élevant et se transformant rapidement après avoir supprimé tout codes sociaxu ou préjudices raciaux restants. Ayant une vision progressive à la fois pour les affaies et pour le crime, le Night Cartel émergea des restes des syndicats sous-terrain de la Terre, récuéprant le meilleur de chaque. 

Le Night Cartel possède un statut d'hypercorporation légitime dans certains habitats tout en travaillant clairement illégalement dans d'autres habitats plus respecteuex de la loi ou ayant des régimes moins corrompus. Le Night Cartel est impliqué dans nombre de domaines traditionnels du crime: racket, extorsion, enlèvement, traites des pods et prostitution. Ils se sont églaement bien adapté aux derniers développements technologique et sont en compétition avec les triades sur les marchés des stimulants électroniques, de la drogue et du piratage de nanofabeurs. COmme les triades, le Night Cartel opére parfois via des façades hypercorporatistes légales. 

\subsubsection{Nine Lives} \label{sec:nine-lives} 

\textbf{Stations principales:} Legba (Ceinture Principale) 

Ce réseau trés répendu de trafiquants d'âme se spécialise dans l'acquisition, le commerce et le traffic général des transhumains. Leur principal marché est le commerce d'ego: vol de sauvegarde, fork-napping, enlèvements, upload forcé et ainsi de suite. Nine Lives est connu pour diriger des colonies illégales d'esclaves informophs aussi bien que pour organiser des combats de foasses utilisant tout type de corps physiques (biomorphs, synthmorphs, animaux) chargés avec tout type de conscience (transhuamins, IA, animaux, etc). Seuls les plus désespérés se trounent vers le syndicat pour être convoyé hors d'un habitat ou de la contraction hypercorporatiste. Leur brutalité lors de l'acquisition d'egos a donné une réputation redoutable parmi les populations transhumaine aussi bien que dans les sociétés ifomorph. 

\subsubsection{Pax Familae} \label{sec:pax-familae} 

\textbf{Stations principales:} Ambelina (Vénus) 

Tout en étant similaire au Night Cartel par la détention de bureau et d'avant-poste légaux dans plusieurs habitats tout en travailalnt dans l'ombre dans d'autre, la différence entre les deux syndicats ne pourraient pas être plsu grande. Toute l'organisation de la Pax Familiae se résume à une seule personne, Claudia AMbelina, la matriarche et fondatrice du syndicat. Utilisant excessivement les tecchnologie de  clonage et de fork, chaque membre du syndicat est un descendant ou une variation de Claudia. Les biomorphs sont clonées depuis le génome originel de Claudia ou sont parfois des des descendants produits sexeuellement(grâce aux bio mods de changement de sexe), alors que les egos sont des forks. Tout les membres sont entièrement fidèles à Claudia et affiche leur affiliation familiale avec fierté et arrogance. Individuellement, chacun d'entre eux reste légèrement mais notablement différent des autres, mais ils sont tous calculateurs et ambitieux. Des ré-assimilation régulières de forks et des mise à jour XP sotn utilisées pour garder chaque variation au courant des activités des autres - une fois que vous avez vus une version de Claudia, les autres vous connaîtrons. 

Pax Familiae se lace dans un assortiment varié d'opération légale, paralégale et illégale, chacune adaptée aux besoins d'un habitat particulier. Des entreprises communes incluent les manipulations de capitaux, l'arnaque dans les réseaux réputationnelle, la consultation financière, la revente d'information, la manipulations du marché, la fraude bancaire et l'usure. 

\subsubsection{Pirates} \label{sec:pirates} 

La plupart des pirates attaques des vaisseaux cargos automatisés et des convois d'approvisionnement longue-distance, et mènent occasionnellement des raids sur des stations de minages d'astéroïdes, des avant-postes de recherche ou des habitats bordés. En de rares occasions ils sont connus pour avoir attaqué des croisières commerciales pour voler les plsu riches ou kidnapper des socialites. Beaucoup de pirates se dissimulent dans les couverts offerts par les flottes racailles, commerçant avec eux et utilisant leurs capacités limitées de maintenance. Un certain nombre d'entre eux se font également des à-côtés en tant que contrebandiers et/ou marchand libre, utilisant souvent des connexions à l'un des syndicats du crime ou à un des éxilés politique. 

\subsubsection{Triades} \label{sec:triads} 

\textbf{Stations Principales:} Qing Long (Toyens Martian) 

Le seul syndicat Terrestre majeur à avoir survécu à la Chute quasiment indemne, les triades dominent le monde sous-terrain du système solaire par le nombre de leur membres et grâce à un passif de siècles d'influence culturelle et politique. Ayant évolué en entreprise légitime et en petits consortiums économique bien avant la Chute, les triades ont récupéré une tête de ponts pendant les premières étapes de la colonisation spatiale grâces aux masses de travailleurs Chinois. Depuis la Chute, ils ont utilisé leur influence pour se répandre dans de nombreux habitats, profitant des disparités des richesses et des politiques de restrictions des réfugiés pour créer des entreprises florissante dans le marché noir et gris. Une part de leur succès repose également dans leur utilisation continue des codes sociaux Chinois pour assurer leur insularité. 

Bien que de nombreuses petites triades existent, habituellement isolées dans une station particulière, il y a quatre grand groupes de triades qui méritent d'être mentionnés. Chacun d'entre eux possèdent suffisament d'influence pour s'engager dans des activités criminelles à l'échelle du système. Elles opèrent traditionellement via des gangs locaux de taille moyenne spécifiques à un habitat ou utilisent leurs façade légale comme origine de leurs efforts. 

\paragraph{La Triade 14K} \label{sec:14k-triad} contrôle une large partie de l'industrie des casinos et les différentes formes de jeu illégaux, de pari et de loterie truquées. A travers leur Galaxy Entertainment Group, une hypercorporations légale de casino et de jeu, la 14K maintiens des connexions sérées avec les politiciens, les célébrités et les entrepreneurs influents dans plusieurs habitats et peuvent se payer le luxe d'une forece de police, la Pai Gow (Main Double). En utilisant le business des casinos pour blanchir de l'argent, ils sont également lourdement impliqué dans l'usure et la fraude au crédit ou à l'ID. 

\paragraph{Le Shui Fong} \label{sec:shui-fong} - bien que plus petit que le14K - s'occupe des vices et des addictions des ouvriers d'habitats, des mineurs et d'autres travailleurs contractés, leur fournissant drogues et XP illégale, faisant tourner des cercles de prostitution et organisant des combats d'arènes illégaux et des tournois de jeu. L'origine de la rancune tance entre le  Shui Fong et le 14K vient des ruines de l'histoire Terrestre pré-Chute, mais la haine entre les deux factions a été emmenée dasn l'espace et continue de bouillir. 

\paragraph{Le Sun Yee On} \label{sec:sun-yee} étaient autrefois la deuxième des triades Terrestre, avec plus de 25 000 membrs supposés. Leur profit vient principalement de la revente de copie bon marché de schéma de nanofabeurs ainsi que de la vente faiseurs et fabeurs bidouillés. Les produits légaux sont ditribués via leur Wushuang Corporation, alros que les marchandises illégales sont assemblés par des infomorphs asservies dans des ateliers de misères cachés dans les recoins éloignés du mesh. La deuxième source de profit du Sun Yee On sont les faux objets de nostalgie Terrestre, telles que les bijoux, les documents, les pièces et d'autres objets de collection. 

\paragraph{Le Big Circle Gang} \label{sec:big-circle-gang} est la plus petite des quatres triades avec aproximativement 8 000 membres. Ils dirigent une bonne partie du marché de la drogue du système solaire, produisant des drogues organiques, des drogues intelligentes ou des narcoalgorithmes de tout type dans des habitats scellés ou dans des insatllations de minage et de traitement abandonnées transformée en laboratoire de drogue. 

\subsection{Firewall} \label{sec:firewall} 

Firewall a été à la tête de la lutte secrète pour sauver la transhumanité depuis la Chute. Firewall est un réseau indépendant de cellulle et d'individus recrutés dans tout type de faction, de cultures, d'origine et d'habitats. Les nouvelles recrues potentielles sont approchées en secret et on leur dit qu'elle possèdent des connaissances ou des compétences utile pour un réseau clandestin cherchant à sécuriser la continuité de la survie de la transhumanité. L'objectif de Firewall est simple; protéger la transhumanité des menaces d'envergure existentielle, peu importe que ce risque émerge de l'intérieur de la transhumanité ou ai une origine extérieure, étrangère. 

Les agents opérationnels de Firewall - appelés sentinelles - doivent agir de manière indépendante et doivent utilsier leurs propres ressources. Les sentinelles sont connectées à un réseau social appelé The Eye, qu'ils peuvent utiliser pour obtenir de l'aide et des compétences additionelle et des ressources. L'i-rep d'une sentinelle sur le réseau indique le niveau de confiance qui leur est accordé et sera un facteur déterminant pour savoir l'aide qu'ils peuvent obtenir. Firewall s'occupe également des dépenses conséquentes et de la logistiques lorsque c'est nécessaire, tels que les besoins en égocast et en réincarnation. La résurrection est garantie aux sentinelles, par la pile corticale ou par une sauvegarde, si ils perdent la vie au cours d'une opération pour Firewall. 

Les sentinelles sont généralement prête à intervenir immédiatement - quand quelque chose arrive dans leur voisinage ou lorsque leurs spécialité sont requises, le boulot leur est affecté. Les sentinelles sont habituellement groupés en équipe d'intervention spéciales en fonction de chaque mission. Même si beaucoup de sentinelles poursuivent leur propres plans après avoir complété une mission pour Firewall, il est courant que des équipes de sentinelles restent en conatct, partagent des informatiosn ou continuent à travailler ensemble sur des assignations liées à Firwall pendant une période dee temps plus longue. 

Les opérations de Firewall sont habituellement planifiée et gérées par des proxys, des agents qui maintiennent l'infrastructure décentralisées de Firewall. Les proxys possèdent généralement plsu d'information que les snetinelles et dispenseront ces informations de la manière qu'ils jugent nécessaire à la mission, en fonction de l'i-rep de chaque sentinelle et de leur besoin de savoir. Les moyens de contact, les briefing de mission et la méthodologie de chaque proxy diffère grandement. 

\begin{quotation} \textbf{Projet Ozma} 

[Message Entrant. Source: Anonymous] 

[Public Key Decryption Complete] 

Tu ne trouvera pas une mention de ce groupe sur les forums conspirationnistes - la sécurité du Consortium est bien trop élevée pour permettre les fuites. Si tu n'as jamais entendu parlé du Projet Ozma avant, considère toi comme prévenu. 

Le Projet Ozma était le nom d'un projet SETI international collaboratif avant la Chute. Il est entré brièvement dans le discours public après la Chute et la découverte de la première Porte de Pandorre  comme une initiative du Consortium Planétaire pour tenter de discerner où sont partis les TITANs. Peu après le Projet Ozma est cependant passé hors de vue, toute mention publique éyant été effacée dans les serveurs mesh du système intérieur. Les officiels du Consortium ont simplement prétendus que le projet avait été déplacé dans d'autres départements. 

Firewall ne sait pas ce qu'est le Projet Ozma, mais nous savonst qu'ils sont toujours là - et qu'ils semblent avoir des intérêts similaires aux notres. Nous nous sommes retrouvés nez à nez avec eux un peu trop souvent pour que ce ne soit que des coïncidences. Peut-être qu'isl sont la version du Consortium de Firewakkn ou peut-être que leurs plans sont complètements différents. J'ai entendu quelques spéculations sur le fait qu'ils se préparaient à et gérerai des contacts étrangers. Tout ce que nous savons c'est qu'ils opèrent au niveau des caisses noires les plus obscures et qu'ils ont un total de ressource saffolant disponible à leur appel. Ils sont aussi vicueux comme des conanrd, le genre à tirer d'abord et à interroger votre sauvegarde plus tard. Les POP standards si vous croisez une opération du Projet Ozma est de se barrer rapidement et de rester loin d'eux. Nous avons déjà perdu des douzaines d'agents à cause d'eux. \end{quotation} 

\subsubsection{Prométhéens} \label{sec:prometheans} 

Un sujet dominant parmi les théoriciens de la conspiration est l'existence d'un groupe d'IA germe qui s'appellent les Prométéhens. Les rumeurs de ces entités datent d'avant la Chute et sont régulièrement allumées alors que certaines preuves sont mises en évidence, même si de telels preuves sont presque toujours discréditées juste après. D'après quelques téhories, les Prométthéens sont plus anciens que les TITANs et pourraient même être responsable de l'existence des TITANs. D'autres postulent que les Prométhéens sont une fatcion dissidente des TITANs qui se sont séparés d'eux et on tentés de contrer les activités des TITANs pendant la Chute. D'autres encore murmures que les Prométhéens ne sont pas du tout d'origine transhumaine, et qu'ils sont en fait une forme de conscience eétrangère qui a trouvé la terre et qui interfère maintenant activement avec les affaires transhumaines. Que les Prométhéens soient hostiles, amicaux ou indifférents restent sujet à beaucoup de conjecture et de de conflit. Des organisation proéminentes comme le Consortium Planétaire démentent de telles rumeurs ou restent silencieux à leur sujet d'une manière ou d'une autre. 

\section{Atlas du Système} \label{sec:system-gazeteer} 

La transhumanité s'est étendue loin de son monde natale et à non seulement colonisé le système solaire mais également diverses exoplanètes grâce à la découverte des Portes de Pandorre. Cette section fournit un aperçu non exhaustif de différents abris humains. 

\subsection{Sol (le Soleil)} \label{sec:sol-the-sun} 

Le système solaire a été formé il y a des milliards d'années grâce à l'accrétion de matière résultante de la formation de son étoile, Sol, le soleil. Bloqué dans son orbite depuis cette époque, l'hisoire et la disposition actuelle de virtuellement tout objet situé à moins de deux années lumière est affecté par sa relation avec ce corp céleste. Le soleil est une étoile G2 appartenant à la squence principale, théoriquement à l'extrémité la plus chaude de la série d'étoile pouvant donner naissance à la vie. Pendant l'essentiel de son histoire, la transhumanité a alimenté ses ascenssion et chute avec l'énergie solaire, d'abord stockée dans des matériau comme les hydrocarbures, t plus tard directement converti avec des convertisseurs solaires. 

Aujourd'hui le soleil reste une source d'énergie cruciale, mais ses coucehs externes sont également devenues un foyer pour certains. Les adaptations nécessaires pour habiter là fait des Suryas l'une des ramification les plus étrange de la transhumanité. 

\subsubsection{Suryas and Salamandres (morphs coronaire)} \label{sec:sury-salam-coron} 

Les morphs adpatée à la vie dans la couronne solaire sont probablement l'un des exemples les plus extrêmes des créations néogénétique de la transhumanité. Les Suryas, nommés ainsi d'après le nom d'une divinité solaire Hindoux, sont énorme, ressemblent à des baleines et adaptées de manière unique pour habiter dans le nuage de plasma surchauffé et brillant de la couche la plus externe du soleil. Chaque surya est une sorte de version miniature d'un habitat circumsolaire. Leur métabolisme génère des champs magnétiques puissant qui les protège de la chaleur et des radiations solaires, tout en agissant comme une voile et des rames magnétiques qui leur permet de naviguer sur les courants des vents solaires et d'extraire les évènements qu'ils transportent. Les Suryas sont protégés par des couches de "barboteur" d'eau liquide qui capture les ions dangereux, qui sont extraits et rejettés par des méchidaments internes, toute en maintenant les éléments utiles tels que l'oxygène et l'hydrogène, qui permettent à leur tour de syntéhtiser encore plus d'eau. Ils communiquent en utilisant des motifs de contrasté sur leur épiderme externe et sont extrêmement sensible aux ondes sonores héliosismiques que sont les pulsations solaires, utilisant ces vibrations pour prédire et éviter les tempêtes dans l'atmosphère coronaire. 

Un autre type de morph coronaire est la salamandre, une eptite morph humanoïde avec des turbines à gaz sur le dos et le torse leur permettantde manœuvrer dans le vide. Les salamandres ont un métabolisme similaire aux suryas, mais sont incapable de survivre sans protection dans la couronne. Ils subsistent grâce aux composés chimiques et à l'énergie extraits de la couronne par Ukko Jylina, le seul habitat où l'on peut voir des salamandres. 

Les suryas et les salamandres communiquent tout les deux soit par des transmissions via leurs implants ou en "solpointant" - altérer les motifs contrasté sur leur peau pour former un language. 

\subsubsection{Habitats} \label{sec:habitats-1} 

Les habitats dans la couronnaire Solaire doivent releevr des défis plus extrêmes que ceux auquel font face les habitats n'importe où ailleurs dans le système. Le seul moyen pour la transhumanité de se protéger de la chaleur et des radiations émis par une naine jaune est de générer un champ magnétique suffisament puissant. Et même ainsi, les dangers représentés par les éruptions solaire et les éjections de matière de coronaire - des explosions massives qui propulsent des matériaux coronaire à des dizaine de milliers de kilomètre dans l'espace circumsolaire - font que suel les régions polaires du Soleil sont des endroits suffisament sûrs pour y placer des habitats. En tant que tel, les habitats circumsolaire un engendre des dépenses astronomiques pour les construire et les maintenir, et deux des trois principaux habitats circumsolaire sont fortement soutenus par des organisations distantes. 

Les couches externe des habitats circumsolaires sont couvert avec des milliers de turbines électromagnétiques qui extraient de l'énergie directement du soleil. Ces turbines génèrent les champs magnétiques nécessaire aux écrans de protection. A l'intérieur on trouve des couches intermédiaires emplies d'eau liquide qui capture les particules ionisées, des nanites collectant les ions et les envoient dans l'espace grouillent dans ces eaux. L'eau doit être régulièrement remplacées par des glacetéroïdes capturés qui sont importés grâce à leurs propres boucliers électromagnétiques lourd. Un habitat en grappe se situe à l'intérieur de cet écran d'eau, un ensemble de modules disposés sur une structure ayant une forme grossièrement sphérique. 

\subsection{Un rapide aperçu des habitats transhumains} \label{sec:quick-prim-transh} 

Les habitats sont couvert en détails p. 280. Un aperçu rapide et proposé ici: 

\begin{itemize} \item Les aérostats sont des cités massives flottant dans la couche de nuages externe de Vénus. \item Les nids d'abeille sont des tunnels forés dans des astéroïdes et des lunes. \item Les grappes sont des habitats en microgravité composés de modules interconnectés. \item Les habitats de type bulle de Cole sont des astéroïdes évidés, dont l'intérieur est terraformé et qui tournent pour fournir de la gravité. \item Les habitats domes sont de grands domes construits sur la surface de lunes, d'astéroïdes ou de Mars. \item Les cylindres hamilton sont des habitats nanotechnologique auto-construit grâce à une conception avancée. \item Les cylindres O'Neill ressemblent à de grandes cannettes de soda, mais gigantesque, souvent large de plus d'un kilomètre et longue de plusieurs kilomèrs. l'intérieur est terraformé et tout le cylindre tourne sur lui même pour fournir une gravité légère. Des cylindres O'Neill sont mis bout à bout l'un de l'autre. \item Les cylindres Reagan sont un modèle inefficace de cylindre O'Neill, construit en évidant un cylindre dans un astéroïdes tournant, et utilisé par la République Jovienne. \item Les boîtes de conserves sont des petites et étroites boîtes modulaires bon marché, généralement utilisé au début de la colonisation spatiale. \item les habitats toriques sont de grandes roues tournant sur elle-même afin que le bord extérieur bénéfici de gravité. Les rayons internes sont en zéro-G. \end{itemize} 

Les habitats coronaires sont facilement détectable à une certaine distance à cause de l'onde de choc qui les précède et de la trainée de plasma qu'il laissent derrière eux dans les vents solaires. 

\subsubsection{Aten} \label{sec:aten} 

Mis en œuvre par un consortium incluant des intérêts hypercorporattistes et l'Université de New Shanghaï, Aten héberge une population d'à peu près 12 000 transhumains. Des rumeurs faisant de la recherche militaire le principal composant de la mission de l'habitat abondent. Aten est extrêmement policé et difficile à visiter. Les découvertes les plus publiées issues de cet habitat traitent des système de propulsion et de nouvelle technologie de collecte de l'énergie solaire. 

\subsubsection{Hooverman-Geischecker} \label{sec:hoov-geisch} 

Les argonautes et l'Université Autonome de Titan sont les principaux soutines de cet habitat, qui héberge une population d'à peu près 4 000 personnes. En contraste avec Aten, l'accès à cet habitat est relativement ouvert. Les principaux domaines de recherches incluent la science pure et la recherche dans des morph coronaro-adaptée. 

\subsubsection{Ukko Jylinä} \label{sec:ukko-jylina} 

Ukko Jylinä est le nom utilisé par les étrangers pour le havre de surya. Dans la langue surya, le nom de cet endroit est une séquane commune de vibrations héliosismique. Lorsqu'elle est transposée quinze octaves plus haut dans une gamme audible par les transhumains, ce son est un grondement chaotique pour la plupart des oreilles, mais les suryas le considèrent comme l'un des plus beaux sons que le soleil peu produire. 

Ukko Jylinä est plus un camp qu'un habitat, un refuge pour les suryas lros des passage de gros temps solaire. Il sert également de lieu pour que les suryas socialisent et s'accouplent, refassent le plein d'eau à partir de gllactéroïdes importés et puissent s'égocaster ou se réincarner. La population fluctue donc énormément, tournant habituellement autour de 300 personnes, mais pouvant atteindre les 3 000 (quasiment toute la population surya) pendant les tempêtes. Ukko Jylina a également quelques modules dans lesquel les autres morphs peuvent survivre. 

Une trés petite part d'Ukko Jylinä consiste de modules d'habitats fermés. Il y a de nombreux modules utilitaires ayant les ports d'accès ouvert à l'espace à la place. Privé des vents solaires, les suryas à l'intérieur du camp portent générallement des harnais de manœuvres par expulsion de gaz ou se réincarne en salamandre si ils ont ebsoin de réaliser des travaux nécessiatnt une manipulation précise. 

\subsection{Vulcanoïdes} \label{sec:vulcanoids} 

Les Vulcanoïdes sont une population d'astéroïdes se situant entre Mercure et le Soleil. Basé sur les prédictions scientifiques du début deu 21° siècles, le nombre de Vulcanoïdes est, de façon inattendue, relativement petit. 

\subsubsection{V/2011-Cladwell} \label{sec:v2011-caldw} 

Découvert au début du 21° siècle et survolé par une mission de recherche Japonaise dans les années 2020, V/2011-Caldwell n'était guère plus qu'une entrée dans le catalogue des astronome, différencaible des autres uniquement par son manque de cratère sur la face qui avait été photographiée. Puis, quelques années après que la poussière de la Chute ne soit retombée, une petite équipe de prospecteurs de Vénus y ont dévouvert une Porte de Pandore. Maitenant contrôllées par TerraGenesis, Caldwell est utilisé princiaplement pour la recherche d'exoplanète depuis de nombreuses années, bien que l'hypercorp soit maintenant engagés dans plusieurs projets de terraforamtion et de géo-ingénierie de planètes étrangères. Terragenesis vend régulièrement des accès à sa porte à d'autres hypercorps et organisations. Caldwelle est un astéroïde fusiforme remarquablement lisse d'à peu près quatre kilomètres de long et moins d'un demi kilomètre de diamètre sur son diamètre le plus large. Appelée la Porte Vulcanoïde, elle est située au pied d'un énorme rocher près de l'un des pôles les plus étroits. 

\subsection{Mercure} \label{sec:mercury} 

la planète la plus proche du soleil a une masse comparable à la Lune et est bien plus dense à cause de son noyau de fer-nickel. Mercure tourne sur elle-même lentement et n'a donc pas d'atmosphère, sa surface au soleil est donc suffisament chaude pour faire fondre la plupart des métaux tandis que la surface exposée à la nuit est extrêmement glacée. En raison du manque de la pluaprt des éléments nécessaires aux colonies transhumaiens pour être auto-suffisante, Mercure est peu habitée, à l'exception d'une poignée de relais d'énergie solaire, de quelques stations de minage souterraine et d'une seule grosse entreprise d'extraction minière en surface, Cannon. 

\subsubsection{Ressources et économie} \label{sec:resources-and-economics} 

la plupart de l'économie Mercurienne est basée sur les mines. Fer, nickel, et d'autre smatériau constituent 70\% de la masse de la planète, en faisant la source de métal ferreux la plus riche en dehors ds astéroïdes. Mercure fait aussi de bonne saffaies dans le relayage de l'énergie solaire et sert de point de départ pour les expéditions de recherche solaire ne désirant ou ne pouvant pas soutenir une station dans la couronne solaire. Mercure a également quelques gisements d'Hélium-3, bien qu'ils soient essentiellement utilisés pour un usage local. Ce n'est un secret pour personne que plusieurs pusisances ont des stations de production d'antimatière là-bas. Officiellement, ces stations sont de gros relais d'énergie solaire, mais les gigantesques accélérateurs de particules torique et les sphères de contentions magnétiques nécessaire pour la production et le stockage de l'antimatière sont quasiment impossible à dissimuler. 

\subsubsection{Caloris 18} \label{sec:caloris-18} 

Caloris 18 est le seul site connu d'activité des TITANs sur Mercure pendant la Chute, et était un relai d'énergie solaire à équipe réduite appartenant à Lukos, une coprotaion Russe maintenant disparue. Vanya Ilyanovich, l'IAG administrant l'usine, a assimilé tous les habitants transhumains de la station et à fusionné leur morph en une abomination gigantesque ressemblant à un centipède avant de s'auto-détruire dans une tentative avortée de fusionner sa conscience avec tous les esprits de sa création. Depuis, Caloris 18 est sous une quarantaine stricte. 

\subsubsection{Cannon} \label{sec:cannon} 

Le plus grand abri de surface de Mercure est une fronde gravitationnelle alimenté par l'énergie solaire et de la taille d'une cité qui rampe sur la face froide de la planète, éjectant dans l'espace des lingots de métal extrait de la atille d'un immeuble d'habitation. L'habitat est possédé quasiment exclusivement par l'hypercorporation Jaehon Offworld, qui a contrsuit Cannon avec le soutien des banques Lunaires cherchant à se diversifier en anticipant sur l'économie Lunaire post-He3. La plupart des 10 000 habitants sont des employés de Jaehon, et la sécurité est trés élevée. Cannon fait une grande boucle autour des zones extrêmement minée autour de Caloris pendant la longue nuit Mercurienne avant de suivre un trajet qui lui fait contourner l'hémisphère nord, évitant les rayonnements dévastateurs du soleil. Pendant tout le trajet, elle s'arrète pour une série d'opérations d'extraction, collectant les lingots génats pour les envoyer en orbite. 

\subsection{Vénus} \label{sec:venus} 

Vénus est à la fois la plus proche voisine de la Terre ainsi que celle qui lui ressemble plus en terme de taille et de géologie. C'est un monde rugueux composé de montagne volcanques, de canyons, de hauts plateaux et d'étendues volcaniques traversée par des rivières de magma. la plupart de la surface est composé de roche basaltique. Le climat de Vénus est l'un des plus inhsopitalier du système solaire. Seules les raidationes les plus mauvaises du système Jovien semblent présenter un défi pour difficile pour la colonisation transhumaine. L'atomsphère Vénussienne est un maelstrom surchauffé de dioxyde de carbon et d'acide sulfurique, avec une pression atmosphérique à la surface équivalente à celle que l'on obtient cinq kilomètres sous la surface des océans Terrestres. Vénus ne possède également guère plus que des traces d'hydrigène, ce qui signifie qu'il faut importer l'eau sous forme de galcetéroïdes depuis le système extérieur. 

Néanmoins, la transhumanité s'est installée sur Vénus et, en arrivant, à commencé à débattre de la façon d'utiliser la planète. Vénus n'as pas d'habitat de surface permanent autre que les quelques équipements et zones d'approvisionnement utilisés par les rechercheurs à la surface. En dépit des difficultés, la transhumanité a trouvé des stratégies de survie fonctionnelle dans ce cas. La plus surprenante d'entre toutes sont les aérostats, des habitats plus léger que le dioxyde de carbone et qui flottent dans l'épaisse atmosphère Vénusienne. À côté des quelques rares aérostats indépendant ou loayux au  Consortium Planétaire, ces aérostats sont la base du bloc de pouvoir émergeant de la Constellation Morningstar. Connus pour leurs laboratoires de recherche, leurs laboratoire de conception de naofabeurs, leurs studios logiciels et leurs attractions luxueuse, les aérostats de la Constellation vont de plus en plus à l'encontre des intérêts du Consortium Planétaire et de l'Alliance Lagrange-Lunaire. 

dans certaisn aérostats, des zones peuplées uniquement de synthmorphs contractées sont ouverte à l'atmosphère Vénusienne. Quelques 5 000 000 transhumains vivent dans les habitats aérostats et encore 10 000 vivent sur la surface de la planète. Approximativement 350 000 de transhumains habitent dans des habitats orbitant auour de Vénus. 

Bien que le Consortium Planétaire considère le démarrage d'un projet de terraformation de Vénus, ce projet est activement opposé par la Constellation Morningstar. Les aérostat de la Constellation voient les propositions de terraformation - qui incluent un bombardement cométaire massif ou la construction d'un écran solaire de la taille e la planète pour refroidir l'atmosphère - non seulement comem étant irréaliste mais également en comem mettant en danger leurs vies et leurs profits. 

Vénus est un endroit fascinant pour les climatologistes, les géologues et d'autres scientifiques planétaires. La découverte d'une protobactérie Vénusienne a soudainement créé une nouvelle branche des sciences du vivant, bien que les applications pratiques pour des organismes avec des métabolismes à ce point différents de la vie terrestre sont encore limités. 

\subsubsection{Gerlach} \label{sec:gerlach} 

Gerlach est un cylindre O'Neill qui abritent à peu près 100 000 transhumains. Généralement considéré comme le point central de la recherche Vénusienne, Gerlach est aussi l'un des endroits les plus étrnages du système ntérieur. Les habitants entretiennent des liens étroits avec les argonnautes et sont des sympathisant des autonomistes du système extérieur et sont de solide partisans de la liberté morphologique, de l'expérimentation cognitive et de l'innovation ouverte. Les principales activités de Gerlach sont la recherche planétaire et l'exploration, la conception de morph pour les environnements hostiles et la construction d'aérostat. 

\subsubsection{Octavia} \label{sec:octavia} 

Octavaia est l'habitat aérostat le plus réussi jusqu'à ce jour et le centre politique de la Constellation Morningstar. Il se maintient à une altitude d'environ 55 kilomètres au-dessus des hautes terres nordique d'Ishtar Terra. Octavia ressemble à un énorme dirigeable en forme de champignon, haut de 450 mètres, encerclé en son centre par quatre espars balancier, chacun se terminant par un stabilisateur rempli d'hélium. Le chapeau du champignon est un dôme rigide et translucide qui fournit un espace ouvert, ressemblant à un parc, tout en servant également d'enveloppe de gaz principale (l'oxygène, qui est bien plus léger que le CO2 composant l'essentiel de l'atmosphère Vénusienne, est la princiaple source de flottabilité). L'habitat est canelé du sommet à la base, allant d'un diamètre de presque 300 mètres à la base du dome à une largeur de 15 mètres à l'extrémité la plus basse. Un gigantesque contrepoids attaché au bas de la structure empêche l'habitat de giter lors des tempêtes. Les vaisseaux atmosphériques et les navettes orbitales peuvent atterir sur des ponts d'envols près de la base des balancier. 500 000 personnes habitent à l'intérieur d'Octavia. 

\begin{quotation} \textbf{Rumeurs Vénusiennes} 

[Message Entrant. Source: Anonymous] 

[Public Key Decryption Complete] Nous avons besoin que tu enquètes sur quelques rumeurs étrange circulant à propos d'activités à la surface de Vénus. D'après les rapports, une équipe de recherche Omnicor a disparue il y a une semaine. Contrairement à beaucoup d'équipe de surface Vénusienne, il ne s'agissaient pas de bots téléopéré mais de chercheurs incarnés dans des synthmorphs et opérant loin de la sécurité d'un lien avec un aérostat - ce qui est déjà un comportement suspicieux. Les groupes de recherches n'ont retrouv aucun signe des morphs manquantes, mais les ragots disent qu'ils sont tombés dsur des signes d'activités récente des TITANs qui leur ont fait tourner les talons. Je n'ai pas encore trouvé de preuve pouvant confirmer cette information - cela pourrait trés bien être une tentative de désinformation pour empêcher els gens de fouiller près de la surface. j'ai entenud que certaines corporation de sécurité ont des caches de données quantiques enterrés là-bas. Enquèter là-dessus pourrait nécessiter de mettre la main sur des morphs synthétiques résistantes à la chaleur et à la pression. \end{quotation} 

\subsubsection{Aphrodite Prime} \label{sec:aphrodite-prime} 

Aphrodite Prime est l'un des 20 aérostats les plus petits et flottent à 54 kilomètres au-dessus de Aphrodite Terra. C'est un centre de tourisme Vénusien; pratiquement un quart de cet aérostat est une station touristique pour  les visiteurs hors-monde les plus riches. Aphrodite Prime est également la pricnipale station de recherche pour la conception et la création de forme de vie adaptée à al vie dans les nuages Vénusiens. Cet aérostat abrite une population de 300 000 personne et possède des volières de tests en environnement fermées peuplées de nuages de plancton aérien et des élevages de poissons ballons et de pieuvres volantes récemment créés. 

\subsection{La Terre} \label{sec:earth} 

Écologiquement dévastée et infestée par les l'étrange proégniture des TITANs, la planète mère de la transhumanité ne reçoit plus beaucoup de visiteurs. Les régions urbaines autrefois peuplés de la Terre sont des étendues urbaines ruinées par la guerre et les tempêtes, infestées par des formes de vies dangereuses et par les gangs survivalistes occasionnels. Partout ailleurs, les zones d'explosion nucléaire irradiées et des terres dévastées prédominent. Enr aison des conditions climatiques difficile, la vie sauvage a été lente à se réafirmer par elle-même, et les vastes étendues de forêt pourrissante ou d'herbe brulée sont extrêmement communs. 

Même depuis l'orbite, des cicatrices profondes sont visible. Des brêches dans la couverture nuage noire de suie créée par les bombardement orbitaux pendant la Chute révèlent des continents ravagées par les inondation côtières, la désertification et les changements de températures radicaux. La seule détonation connue d'une bombe à antimatière à l'intérieur d'une atmosphère planétaire, centrés sur ce qui fût le Métroplex Chigao-waukee en Amérique du Nord, à laissé un cratère de plu de 200 kilomètres de diamètre à l'intérieur duquel l'essentiel de la matière a été vaporisé instantanément. Les Cratères laissé par les bombardement à la fronde orbitale ponctuent également la surface. la disparaition massive d'espèce comme les abeilles et le krill ont détruit des écosystèmes entiers, laissant de vastes andains de terre et de mer stérile habités seulement par les espèces les plus adaptables. La plupart de l'Europe est sub-arctique; l'essentiel de l'Afrique et de l'Amérique du Nord est désertique. Ironiquement, le déploiement d'armes nucléaires face aux installation de surface des TITANs a arrété les effets du réchauffement climatique en créant un hiver nucléaire. Les attaques nucléaires contre la Terre ont cessée, mais les frondes gravitationnelle Lunaire continuent de lancer occasionnellement des astéroïde sur les travaux de surface suspectés d'être créés par les machines de guerres des TITANs restantes. Dans tous les cas, les dommages infligés par le réchauffement planétaire dus à la présence humaine étaient déjà fait. Les motifs de la vie sur Terre, et le visage global de la planète, ont été irrémédiablement réécrit. 

La Terre possédait autrefois de nombreux ascensseurs spatiaux en fonctionnement, mais à l'exception de la pousse du Kilimandjaro, les autres ont été détruit pendant la Chute, s'enroulant autour de la planète lorsqu'ils se fracassait sur la Terre, laissant des cicatrices béantes de destruction. 

\subsubsection{Population} \label{sec:population} 

la population Terrestre est un problème spéculatif. Les réclamationnistes et les autorités Lunaire, les deux investissant beaucoup de moyens dans la surveillance de la Terre, sont d'accord sur le fait que les émissions énergétiques de la surface suggèrent une population d'à peu près un million d'être autrefois huamin devenus des serviteurs des TITANs, bien que ce nombre extrapole les motifs énergétiques d'après ceux de l'humanité pré-Chute. 

Bien que le Consortium Planétaire prétendent qu'il n'y a pas de survivants sur Terre, les réclamationnistes estiment qu'il y aurait entre 20 000 et 100 000 être humains libres survivants. Ces nombres sont complexe à calculer, étant donné le nombre limité de zone dans lesquels les humains peuvent rester indétecté tout en trouvant suffisament de nourriture pour survivre. Certaines zones capable de dissimuler une population rémanente de taille correcte incluent les hautes terre de Papouasie-Nouvelle Guinée, les Montagnes Ozark d'Amérique du Nord et les jungles du Viétnam et du Laos, bien qu'il soit également possible que certains abris sou-terrain et sous-marins pour survivre. Les tentatives de contact avec les survivants se sont systématiquement transformée en désastre. 

Pendant la Chute, des milliers de personnes incapable d'échapper à la Terre ont du se faire sauvegarder et transmis hors de la planète. Beaucoup d'entre eux - ainsi que certains de ceux qui n'avaient pas de sauvegarde - ont également mit leur corps dans des stockages cryogéniques, espérant qu'on viennent les récupérer après la Chute. Certains réclamationnistes ont spéculer que ds douzines de ces installations cryogéniques pourraient toujours être fonctionnelles. 

\subsubsection{Habitats} \label{sec:habitats-2} 

La Terre possèdait un secteur d'industrie orbitale mature et une population en orbite au moment de la Chute, avec plus d'un milliard de personne vivant à temps plein dans l'espace. L'orbite Terrestre fut l'un des champs de bataille les plus violent pendant la Chute cependant, et des centaines d'habitats et d'autres installations ont été détruites ou rendues inutilisable. En tant que telle, l'orbiteTerrestre et les points de Lagrange ont été remplis des détritus de l'humanité pré-Chute. Un habitat abandonné peu signifier un proft substantiel pour les charoganrds intrépides, mais beaucoup dsont également infesté par les rejetons des TITANs et des nuées de nanites hostile, les rendant incroyablement dangereux. 

Pour empirer la situation, quelquechose ou quelqu'un a libérer un grand nombre de sattelite tueurs autonomes dans l'orbite Terrestre pour interdire les éventuels visiteurs. Certains d'entre eux sont basé sur du matériel militaire pré-Chute réutilisé, tandis que d'autres sont des constructions plus récentes. Jusqu'à présent, personne n'en a revendiqué la responsabilité. Le Consortium Planétaire est soupçonné, car ils soutiennent et quelque foit renforcent la quarantaine de la planète, mais la possibilité existe que les satelittes tueurs puissent être des reliques des TITANs ou les efforts d'une agence quelconque. 

En dépit du chaos régant sur l'orbite Terrestre, de nombreux habitats sont toujours actifs là-bas, beaucoup d'entre eux étant membrs soit du Consortium Planétaire, soit de l'Alliance Lunaire-Lagrange. Des dizaines d'anciens habitats abandonnés sont également devenu le foyer des squatteurs, certains d'entre eux ayant des intentions criminelles, d'autres cherchant simplement a échapper aux sordides conditions de vie des habitats Lunaire-Lagrange surpeuplés, même si cela implique de prendre certains risques. 

\subsubsection{Fresh Kills} \label{sec:fresh-kills} 

Fresh Kills est une base de récupération près du point L5 Terreste et est essentiellement une barge racaille armés jusqu'aux dents. Sa base est construite autour d'une gigantesque aiguille d'approche avec des amarres pour les plus petits vaisseaux et les modules d'habitats du centre, et équippée d'impressionante batterie d'artillerie à chaque extrémitée. Les charognards peuvent amarrer leur propre vaisseau ou, à un prix considérable, s'égocaster à l'intérieur, se réincarner dans l'installation, et louer dans navettes pour les excursions. Les batteries d'artillerie sont articulées afin que n'importe quel vaisseau montrant l'nevie d'en découdre puisse être largué et détruit rapidement. 2,000 transhumains vivent sur Fresh Kills, même si la population en transit et fluctuante en représente une bonne partie. 

\subsubsection{Paradise} \label{sec:paradise} 

Situé en orbite halo au point Terre-Soleil L1, Paradise est une statin exclusivement constituée de spa et d'attractions pour les ultra-riches d'avant la Chute. Au début de la Chute, Paradise a connu des temps difficiles, pullulant de réfugiés et plus réellement le lieu de vacances idéal. Cependant e récemment, Paradise a retrouvé les faveurs des célébrités du système intérieur, qui ont pris des mesures pour expulser les squatteurs persistants et remeubler la station comme un espace de l'élite sociale. Des rumerus récentes suggèrent que le Conseil Hypercorporatiste du Consortium a utilisé Paradise pour d'importante rencontre en face à face. 

\subsubsection{Vo Nguyen} \label{sec:vo-nguyen} 

Les Réclamationnistes maintiennent cette station en orbite géostationnaire haute, surveillant la Terre et planifiant les éventuells effort de géo-ingénierie. Vo Nguyen est un petit cylindre O'Neill caché dans un nuage dangereux de débris spatiaux et protégé par des nuées de satellites tueurs, d'emplacement d'artillerie et de drônes. Elle est occasionnellement utilisées comme point de départ pour des expéditions secrètes vers la surface. 

\subsection{Lune} \label{sec:luna} 

le premier corps planétaire a héberger des habitations humaines de manière permanente, la seule lune de la Terre est le foyer de la deuxième population transhumaine sur une seule planète en terme de population et reste un pivot des activités culturelle et économique. L'histoire Lunaire a été dramatiquement remodelée par la Chute. Avant que le besoin d'évacuer la Terre n'apparaissent, la Lune était destiné à être principalement une entreprise d'extraction automatisée et ne devant jamais dépasser une population de plus de quelques millions. La Lune n'a jamais été considérée comme une destination économiquement viable pour la colonisation, les efforts se concentraient sur Mars et le système extérieur. 

Lorsque la Chute arriva, chaque régime politique qui ne pouvait espérer aller sur Mars ou ailleurs s'établit sur la Lune. Les Indiens furent la seule puissance majeure à investir massivement sur la Lune Les trois autres abris principaux, Erato, Nectar et Manille, étaient des entrepirses multinationaux ou hypercorporatistes sans attaches nationales. Ces trois cités se transformèrent rapidement en un camp de réfugié polyglotte, pendant que l'abri Indien, New Mumbai, était détruit par le feu nucléaire des hypercorps quand il devient évident que l'infection des TITANs avait pris pied là bas. 

Débarassé des nationalités, les Lunaires ont développés leur prope culture pleine de ressource et réfléchie qui a émergée comme un contrepoids au radicalisme du système extérieur et des excès de Mars. 

Le transport sur la Lune est essentiellement réalisé par des fusées suborbitale, même si des TGV sub-sonique sont également utilisé pour les trajets les plus court. Le spatioport princiapl est Nectar. Un palan aérien existe également - un spatioport sattelitaire et amssif en orbite qui traine un lien massif, et qui agit comme un ascensseur spatial le long d'une piste qui fait le tour de la surface de la Lune au sud de son équateur. Il existe donc beaucoup de plus petites cités le long de la piste de ce palan. 

\subsubsection{Mode/Design} \label{sec:fashiondesign} 

Nectar est l'un des trois capitales de la mode et du design du système (avec Noctis sur Mars et Extropia). Les bureaux de design Lunaires ont deux avantages majeurs: une population inventive et une faible gravité planétaire qui facilite la conception d'objet destiné aux faibles gravités que l'on retrouve dans une grande part du système. D'autres habitats, quelque part dans le système, choisissent même une vitersse de rotation qui simule la gravité Lunaire afin d'exploiter au mieux les objets conçus sur la Lune. 

\begin{quotation} \textbf{Le cerveau gigogne de Tilion} 

[Message Entrant. Source: Anonymous] 

[Public Key Decryption Complete] 

Nos enquêtes sur les activités de recherche sous le nom de code: TILION ont confirmés nos soupçons. L'hypercorporation mène des expérimentations pour convertir des masses sphérique de l'intérieur de la Lune dans des micro-cerveaux gigogne de test. La croûte Lunaire riche en silice rend l'endroit choisit comme un endroit idéal pour le projet. Bien que nous n'ayons jamais pu le vérifier, nous pensons que TILION ne fait pas que suivre les pistes des recherche des TITANs dans ce domaine, mais qu'ils sotn également en possession d'une petite cache de computronium hérités des TITANs. Il n'existe aucune information de l'utilisation éventuelle de la cache par les TITANs, ni de ce qu'elle pourrait contenir ou de ce qui pourrait se passer si TILION achève son projet et mettait le micro-cerveau gigogne en ligne. heureusement, le temps semble être de notre côté et nous avons plusieurs semaines voire des mois avant que la moindre part significative du projet ne soit activée. Nous coninuerons de les inflitrer et d'apprendre, mais nous suggéront fortement qu'une escouade de suppression soit déployée et mise en attente. \end{quotation} 

\subsubsection{Extraction de l'Hélium-3} \label{sec:helium-3-mining} 

Même si ce 'nest pas l'endroit le plus riche en He-3, la Lune a une telle infrastructure pour l'extraction et la distribution que cela fait plus que compenser le fait que la Lune soit extrêmement pauvre en hydrogène pour les formes de fusion plus convntionnelle. Contrairement aux vastes réserves des géantes gazeuses, la quantité de He-3 extarctable de la régolite Lunaire est finie. Quelques uns des dépots les plus riches sotndéjà épuisé, et les Lunaires soucieux considèrent le futur de leur monde une fois les gisements épuisés comme un sérieux problème. 

\subsubsection{Finance} \label{sec:finance} 

Les banques Lunaires sont les plus anciennes (et donc les plus riches) dans le système, même si des hypercorporations comem Solaris ne sont pas loin derrière. De manière intéressante, l'avènement de l'économie réputationnel dans le système extérieur n'as pas présenté de trop grosses difficultés à ces banques comme certains auraient pu s'y attendre. Les banques Lunaires sont entrés dans la danse de la réputation bien avant que les institutions financière Martienne n'aient commencées à capitaliser dessus. Au moment où les banques Martiennes ont compris ce qu'il se apssait, les instituions financières Lunaires avait prit des accords avec els Extropiens et dominaient tous les points d'échange entre les habitants du système intérieur et les anarchistes des systèmes extérieurs et où des faveurs pouvaient être troquées contre du cash. Le même génie a alimenté la création d'un système complexe de troc pour les réseaux que tout le monde ou presque utilise. Alors que certains autonomistes trouvent aliénant le fait de devoir faire avec un systèmebancaire monolithique pour faire des affaires dans le système intérieur, d'autres sont simplement content de traiter avec els Lunaires plutôt qu'avec les Martiens pour ce service. 

\subsubsection{Erato (Eratosthène)} \label{sec:erato-eratosthenes} 

Erato (5 millions d'habitants) est un centre majeur d'exploitation minière composé d'une série de dômes de surface lourdement protégés par des bouclier et d'une grande cité sous-terraine. Erato est centré autour du cratère Eratosthène à la limite sud de la Mare Imbrium (Mer des Pluies), dans l'hémisphère nord de la face visible depuis la Terre de la Lune. Erato a accès à la fois à de riches gisements de titanium de la Mare Imbrium et à des champs de régolite abondant en Hélium-3. 

Erato est l'un des abris de minage les plus anciens sur la Lune et l'un des premiers à devenir commerciallement viable. En tant que tel, beaucoup de banque Lunaire sont centrées autour de cette cité. Les hauteurs voutées de la Grande Caverne d'Erato, originellement excavée par un conglomérat Sino-Européen, atteint une hauteur de 1,5 kilomètres à son point culminant, laissant de l'espace pour une cité grouillant de jardin et de tour construite dans de la silice Lunaire et des nanites industriels, éclairé par la lumière solaire pénantrant grâce à de grandes évents couvets de mirroir. 

\subsubsection{Nectar (Nectaris)} \label{sec:nectar-nectaris} 

Nectar (9 millions d'habitant) repose à 100 kilomètres à l'est du crater Theophilus sur la Mare Nectaris (Mer de Nectar) dans l'hémisphère sud de la Lune. Nectar est un centre de design, abritant les plus grands studios de conception Lunaire qui définissent la mode et les tendances pour la plupart du système solaire. Enr aison de sa localisation relativement proche de l'équateur Lunaire, Nectar héberge également l'un des principaux spatioport long-courrier et est l'une des zons d'enlèvement du palan aérien. 

\subsubsection{Zone de confinement de New Mumbai} \label{sec:new-mumb-cont} 

L'incinération de la colonnie de New Mumbai par le feu nucléaire pendant la Chute afin d'empêcher la propagation de l'infection TITAN a laissé une marque de brûlure d'environ 100 kilomètres de diamètres set qui est toujorus visible depuis une orbite haute. la colonnie était une station hautement automatisées d'extraction d'Helium-3, localisée au milieu de champ riches en Helium-3 à la limite de la Mare Moscoviens. Elle demeure jusqu'à présent une zone de quarantaine extrêmement surveillée. 

\subsubsection{Manille (Shackleton-New Varanasi)} \label{sec:shackle-shackl-new} 

Manille (6 millions d'habitant), construite dans et autour du cratère Shackleton au pôle sud, est centrée autour de l'une des deux principales zones d'extraction d'eau sur la Lune. New Varanasi, la cité des temples, est la section la plus impressionante de la cité. Manille était l'autre cité principale de l'ancienne influence Indienne sur la Lune, et suite à la destruction de New Mumbai elle a une importance particulière pour les descendants de la disapore Indienne. New Varanasi est un complexe monumental de cavernes artificielles doté d'un système de canaux et alimenté par de la glace fondues extraite des calotes polaire situées au-dessus. En tant que qu'eau pouvant héberger la vie, il détient maintenent la même importance dans les croyances Hindu que le Fleuve Gange de l'ancienne Terre. Les survivants d'autres religions Indiens, tels que les Jains et les Sikhs, ont également établit leur temps là. Cela fait de Manille l'un des princiapux lieu de pélerinnage; et le tourisme est sa principale industrue après l'extraction de l'eau. Un petit troupeau d'éléphants Indiens consitue une des attractions principale, et le dieu éléphant Ganesh, Le Destructeur des Obstacles, est extrêmement populaire sur la Lune, même parmi les non-Hindous. 

\subsection{Mars} \label{sec:mars} 

La Terre était le berceau de la civilisation humaine, mais Mars, avec une population de 200 millions de transhumains, est maintenant sa terre d'adoption. Lrosque l'humanité commença sa diaspora spatiale, la Lune fût la première étape. Mais même si la Lune peut se vanter d'avoir une population de taille correcte, Mars a été le premier monde sur lequel les humains s'installèrent et où ils purent prospérer en ne comptant que sur des ressources locales. Pendant les premièrs décennies, les premiers colons Martiens ont habité des unités d'habitations en boîte de conserve, extrayant le méthane contenu dans l'atmosphère locale comme carburant à fusée et l'eau du pergélisol Martien, cultivant dans des serres gonflables et fabricant même suffisemment de gaz à effet de serre pour réchauffer le climat planétaire jusqu'au stade où les transhuamins pouvaient marcher à la surface Martienne sans protection, à l'exception de masques à oxygène. 

La deuxième phase du grand projet de terraformation de Mars - aménager la vie végétale et concevoir des microbes pour remplacer rapidement le dioxyde de carbone de l'atmosphère par de l'oxygène - était déjà en cours à l'époque où la Chute a commencé. Une ceinture de mirroir orbitaux aide à réchauffer la planète et concentrant les rayons solaires. La propagation de le vie végétale est un projet à long-terme qui prendra plusieurs siècles avant de produire une atmosphère entièrement respirable, mais les transhumains presque immortels de Mars sont prêt à être patient. Une nouvelle planète mère vaut bien d'attendre un peu. La recherche de nouvelles plantes et de nouveaux microorganisme capable de libérer de l'oxygène et de l'azote dans l'atmosphère Martienne à un rythme toujours plus rapide est un des principaux axes de l'activité économique. 

En même temps, la planète rouge est un endroit de contraste saisissant, de la pure beauté de ses chaîne montagneuses et de ses désert d'altitude, à ses basses-terres du système de cnayon équatorial de la Valles Marineris verdoyant lentement. Dans ces basses-terres, le niveau d'oxygène augmente doucement et onpeut maintenant trouver de l'eau liquide dans des canaux qui étaient irrigués des millions d'années auparavant lorsque les ancètres de la transhumanité sont descendus de l'arbre. mars est une destination populaire pour les voyageurs de tout le système. De nombreux Martiens accumulent les richesses en gérant des hôtels de luxe, en proposant des visites de sites historiques et en dirigeant des safaris dans les hautes-terres rugueuses et les désert de la bordure Martienne encore sauvage. 

Mars supporte actuellement cinq vastes cités dôme, la plupart localisé dans les régions équatorialle, ainsi que de nombreux abris plus petits. Les abris sont connectés par des routes en surface, un réseau de train à sustentation magnétiques sub-sonique ainsi que par des aéro/spatioport entre lesquels des suborbitales, des avions et des fusées ballistiques effectuent des vols réguliers. Grâce à l'abondance de méthane et à la gravité équivalent à un tiers de celle de la Terre, les transhumains sur Mars ont enfin eus leurs voiturs volantes et tous les abris ont des couloirs aériens bien délimités pour ces véhicules. Dans les terres sauvages, les planétologistes et les ingénieurs en terraformations résident dans de petits village, vivant une vie simple dan des morph rusteurs tout en observant le développement continue du climat et de l'atmosphère Martienne. 

En tant que planète partiellement terraformée avec de vastes bandes de terres inutilisée, Mars est l'un des rares endroits qui peu offrir de nouvelles peaux aux infugiés. Les maisons de courtage Martienne font de bonnes affaires grâce à l'achat et à la vente de contrat de travail d'infomorph, avec des accords menant (en général) à une incarnation éventuelle. Ce phénomène a cependant créé une classe sociale Martienne basse de taille conséquente, organisée en un mouvement de résistances grandissant et rassemblé sous la bannière des Barsoomiens (même si les hyperélites sociales les appelles de manière désobligeant les "rednecs"). 

\subsubsection{Régions} \label{sec:regions} 

De manière générale, Mars est divisée entre les basse terres du Nord et les hautes-terres du Sud, qui sont séparés en de nombreux endroits par des falaises spectaculaires pouvant atteindre deux kilomètres de hauts. Mars, comme la Terre, possède des saisons et les pôles nord et sud possèdent des calottes de glace persitante en dépit de la réussite de la transhumanité a réchauffer la planète. Les deux régions présentent des obstacles à la terraformation. Les plaines nordiques sont ouvertes et balayées par les vents, alors que les hauteurs accidentées du sud restent un terrain difficile pour que la vie y prenne pied. Mais même ainsi, les espèces robustes Terrestre telles quye les cactées et les succulentes sont capable de se développer dans les meilleurs endroits. 

\paragraph{Ma'adim Vallis:} \label{sec:maadim-vallis} ce système de profond canyon Martien abrite l'une des possessions les plus convoitées du Consortium Planétaire: la Porte Martienne. Cette Porte de Pandorre fût originellement découverte par des nomades Barsoomiens, puis violemment arraché de leurs mains par les troupes hypercorporatistes - un évènement que ruminent toujours les rednecks. Comme différentes hypercorps en vinrent aux mains, le Conseil Hypercorporatiste a été forcé de ss'avancé et de proposer une résolution qui plairait à tout le monde. Une nouvelle hypercorporation a été fondée - Pathfinder - afin de contrôler l'exploration et l'exploitation de la porte et des ressources au-delà de celle-ci, avec des privilègres et des droits particuliers accordés aux membres du Consortium Planétaire. La Pote Martienne est maintenant uen étape pour de nombreuses exoplanète colonisées, bien que quelques uns s'inquiètent de la perspective de laisser un présumé artefact des TITANs opérationnel sur l'une des planètes les plus peuplées de la transhumanité. 

\paragraph{Olympus Mons:} \label{sec:olympus-mons} le repère le plus visible de Mars est le puissant volcan bouclier Olympus Mons, sur lequel le premier - et, encore actuellement, principal - ascensseur spatial Martien a té construit. De forme et d'origine similaire aux volcans terrestre Hawaïens, mais maintenant dormant, Olympus Mons est l'une de splus hautes montagnes du système solaire, atteignant 27 kilomètres. 

Olympus, l'abri situé dans la caldera du volcan autour de la base de l'ascensseur spatial, est l'une des principales cités de Mars, mais à perdu en popularité comme lieu de vie lorsque la terraformation a rendu les autres régions plus attractives. Un train à sustentation magnétique au dpéart d'Olympus mets un peu moins de trois heures pour rejoindre Noctis; les voyages aériens sont même encore plus rapide. En dépit du déclin de la cité, l'ascensseur spatial est toujours extrêmement utilisé. 

\paragraph{Valles Marineris:} \label{sec:valles-marineris} La plupart des efforts de terraformation de la transhumanité sont centrés autour des canyons venteux de la Valles Marineris, qui se tord et tourne sur plus de 4 000 kilomètres d'Est en Ouest le long de l'équateur Martien. Dans ces basses-terres relativemenst tempérées, l'eau liquide est devenue abondante et la terre est verte grâce aux espèces végétales Terrestre les plus hardies telles que la digitaire sanguine, les pissenlits et le simposants sapins Douglas (qui pourraient atteindre la hauteur de 180 mètres dans la faible gravité Martienne, d'après les botanistes). 75\% de la population transhumaine de Mars vit dans cette région, en faisant la région avec la plus haute densité de population transhumaine du système solaire. 

\paragraph{La Zone:} \label{sec:zone} officiellement baptisée la Zone de Quarantaine TITAN, le ZQT est une large zone s'étendant des plaines d'Amazonis Planitia (entre les régins volcaniques de Tharsis et d'Elysium) jusqu'au Arsia Mons au sud est (juste à l'ouest de Noctis). Cette zone est connue pour grouiller de reste de machinerie TITAN: bots de geurre, essaims de nanites et d'autre choses dangeureuses. Plusieurs habitats dévastés reposent dans cette région, incluant l'ancienne forteresse Islamique de Qurain. Rares sont ceux qui osent s'aventurer ici, même si des rumeurs suggèrent que des contrbandiers Barsoomiens utilisent les grottes d'Arsia Mons et récupèrent même de la technologie TITAN, en dépit des risques. Les drônes du Consortium Planétaire surveillent d'un œil vigilant les frontières de la ZOne, même si pour des raisons inconnues les reliques des TITANs restent généralement de leur côté de la frontière. 

\subsubsection{Ashoka} \label{sec:ashoka} 

Ashoka est localisée dans un cratère de la région d'Ares Vallis à pue-près à 3 000 kilomètres au nord est de Valles-New Shangaï, non loin des sites d'atterissages des premières sondes Viking et Pathfinder. La ville est un lieu de spa et de etraite spirituelle populaire auprès des Martiens qui veulent revisitier leurs racines de pionnier. C'est également une station de terraformation active et un point de central de contact entre la culture Barsoomienne semi-nomade du haut désert et des Martiens sédentaires des terres de canyon équatorialle. 10 000 scientifiques, historiens, ouvriers de terraformation et gurus spirituels vivent dans le live et dans la zone avoisinnante. une des principales attractions est un musée acceuillant le module Pathfinder et le vadrouilleur Sojourner (qui était toujours opérationnel lorsque les humains atterrirent et le découvrirent ent rain de tourner en rond dans un cratère). Le module Viking est dans un autre musée situé à un saut en monorail de la ville. Dans un mouvement qui énervent les historiens puristes, les trois machines ont été équipées de mise à jour matérielle moderne lorsqu'elles ont été découverte et héberge maintenant des IA qui se comportent en historiens des débuts de l'exploration Martienne. Sojourner est particulièrement amicale et emmène parfois des groupes chanceux en excursions autour des premiers sites d'atterissage. 

\subsubsection{Elysium} \label{sec:elysium} 

Localisé dans l'Elysium et l'Hyblaeus Chasma au nord de la région Hespéria dans l'hémisphère oriental de Mars, Elysium es la cpaital du divertissement du système et la plus grande cité Martienne hors des terres de canyons de l'équateur. C'est également la plus distante physiquement des grandes cités Martienne, même si les technologie de transport avancée de la transhumanité (vols suborbitaux et vols en fusée depusi les habitats spatiaux) rendent cet éloignement un défaut trivial. 

Les Elysium Chasma et Hyblaeus Chasma représentent à eux deux un système de canyon long de 250 kilomètres dans l'ombre d'Elyisuim Mons, une montagne de 14 kilomètre de haut localisé à peu près 200 kilomètres au nord est de la ville. Zephyrus Fossae, une plaine de lave ondlueuse et balayées par les vents se situe dans l'intervalle. La ciét est le résultat des vision d'une personne, Zevi Oaxaca-Maartens, un magnat du divertissement excentrique qui était intrigué par la proximité du Chasma eminemment terraformable et les terrains encore vierge de l'Hesperian. 

La cité n'est agée que de 30 ans mais grouille déjà d'une population de 9 million de transhumains. Elysium est essentiellement construite dans les murs de canyons du Chasma, s'étendant sur une zone de 75 kilomètres, l'ensemble étant recouvert d'un dôme. Contrairement aux grands métroplexes de dômes du sud, Elysium exploite les murs des canyons, qui sont suffisament proches les uns des autres pour avoir permis aux constructeurs de construire de grandes arches recouvrant complètement le canyon au lieu des dômes érigés classiquement. Ces arches s'étendent vers le nord année après années au fur et à mesure de la croissance de la cité. Depuis une orbite basse, elle ressemble à un grand serpent scintillant. 

La cité Martienne d'Elysium est la descendante spirituelle de la vieille Los Angeles Terrienne en tant que capitale du divertissement du système solaire. Les stars glamours et les producteurs suceurs de sang, mélangé avec une saine dose de créativité transhumaine scandaleuse (même si trés fréquemment insipide) ont fait de Mars une centrale médiatique sans rivales. Elysium possède un taux de morphs exalt et sylphs par habitant supérieur à celui de toutes les autres cités transhumaines. 

l'apparence et l'image sont à la base de tout ici, et il peut sembler au visiteurs que tout le monde dans cette cité est soit d'une beauté aveuglante soit d'une laideur calculée. Les interprètes et acteurs les plus talentueux et les magnats du divertissement vivent des vies scintillantes de privilégiés qui rendrait jaloux les plus riches des gérontocrates de New Shangahaï. Tout les autres, des développeurs de jeu prometteurs aux éro interprètes virtuels, doivent se battre constamment. 

\subsubsection{Noctis-Qianjiao} \label{sec:noctis-qianjiao} 

Avec une population de 13 million d'habitant, Noctis-Qianjiao est le métroplexe majeure de l'ouest de la région de la Valles Marineris, une zone connues comme le Noctis Labyrinthus. Même si elle n'est pas aussi hospitalière que la région Eos dans laquelle repose Valles-new Shanghaï, Noctis Labyrinthus est considérée comme de l'immobilier de premier choix pour ses somptueux paysage et son système de rivière bien développé. Le métroplexe inclut deux dômes majeurs: Qianjiao, sur la rive nord de River Noctis, et Noctis City (habituellement appelée simplement "Noctis") au sud. Un réseau tentaculaire de plus petits dômes et de souks connecte les deux dômes et enjambe la rivière, même si ils ont se sont également développés vers le nord et le sud au fil des années alors que la planète se réchauffe et que la rivières s'élargit. 

Noctis-Qianjiao est le centre de l'industrie de la mode et du design Martien, ce qui la rend aussi important dans l'conomie Martienne que la cité plus grande valles-New Shanghaï. la proximité de cet abri avec la Zone alarment aprfois les visiteurs, mais il n'y a pas eu d'incidents rendus publics jusqu'à présent. 

\begin{quotation} \textbf{Chasse à l'homme Marteinne} [Incoming Message. Source: Anonymous] 

[Public Key Decryption Complete] 

Ces morts au sujet desquelles tu voulais que je fouille? Ça à l'air encore pire que ce que nous craignions. 

J'ai pris un buggy pour aller faire un tour dans l'abri Zim. C'est juste une collectiond e modules boîte de conserve, abritant une petite écostation de terraformation et des installations pour les rednecks nomades. En moyenne, il abrite 150 personne, si tu compte les 20 pods IAs de plaisirs qui serve de "distraction" locale. 

Il y a une semaine, un nomade connu sous le nom de Hassan Naceri est arrivé. C'est un habitué, le bruit court qu'il serait un messager pour les Barsoomiens. Lors de cette visite récente son comportement était diférent. Il était nerveux et agité. Il a dit a un de ses collègues de boisson qu'il a été obligé d'aller se planquer dans la Zone pour quelques jours et que l'expérience l'a mis à cran. 

Il s'est avéré que Naceri s'est barré avec sa morph ruster sans aller au bout de son contrat avec Fa Jing il y a quelques années. Un chasseur d'égo s'est pointé à Zim et Naceri à perdu son sang froid. Il a tué le chasseur d'ego et tout ceux qui étaient dans la salle. 

On a trouver l'enregistrement d'un spime qui montre que Naceri s'est transformé. Il a également tué une demi-douzaine de personne, simplement en les regardant. 

Tu as bien lu. Ce connard est infecté. 

Les Rangers Martiens le suivent de près, mais ils ne savent pas ce qu'ils poursuivent. Et donc on doit attrapper ce mec - cette chose - en premier. Souhaite nous bonne chance. \end{quotation} 

\subsubsection{Olympus} \label{sec:olympus} 

Avec une population d'1 million d'âme vivant dans une espace conçu pour en abriter 6, Olympus a des allures de ville fantôme. L'ex-plus grande cité, construite dans la caldera d'Olympus Mons autoude l'ascensseur spatial est maintenant tombée en désuétude. Alors que les températures grimpent et que le climat s'émaliore dans les canyons de valles Marineris, la plupart de la population a abndonné les caldera balayée par les vents pour un environnement plus hospitalier. Olympus n'est pas, et n'a jamais été, une grande cité dôme, composé au lieu de ça d'un souk de dômes mineurs et d'antiques habitats en boîtes de conserve connectés. 

La faible pression atmosphérique et les températures glaciales de la cité à 27 kilomètres d'altitudes font que la plupart des transhumaines qui s'aventurent à l'extérieur des souks et des modules ont toujours besoin d'exocombinaisons légère pour survivre. la Martian Alpiners, une morph particulièrement rare et trouvé seulement dans quelques autres endroits, sotn courantes ici en raison des conditions difficiles. 

Le centre ville est bien entretenu et prudemment supervisé par l'Autorité de l'Infrastructure d'Olympus, une hypercorporation mineure qui opère l'éascensseur spatial. La banlieue est en déclin économique et parfois dangereuse, principalement désertée et peuplée par des squatteurs, des contractés téléchargés en fuite et d'autres personnes qui tiennent à être réellement seules. Des attaques occasionnelle de vie articificielle dangereusement mutés sont l'une des rares raisons pour laquelle l'Autorité daigne intervenir dans les banlieues. Le reste du temps, les vieilles conserves et leurs étranges habitants sont laisser là à pourrir. 

\subsubsection{Progress (Deimos)} \label{sec:progress-deimos} 

Progress est l'une des plus grande bulles Cole du système solaire. Avec 8,5 millions de résidents, la seule bulle plus peuplée dans le système est Extropia, dans la ceinture. Progress a été créée lorsque Fa Jing évinça tout les anciens habitants de Deimos, l'un des deux sattelite Martien, excava l'intérieur de la petite lune et à utiliser de gigantesques panneaux solaires pour la convertir en un monde bulle. D'un point de vue conception, Progress est quelque peu embarassante. L'habitat deevait, au départ, dépasser considérablement Extropia en taille, mais des difficultés rencontrées avec le réchauffage et la rotation de Deimos ont forcé fa Jing a abandonner leurs efforts plus tôt ou risquer d'exploser le sattellite. 

Progress reste néanmoins un habitat impressionant, le foyer de célébrités hypercorporatiste et un avant-poste pour bon nombre de politique majeurs et de probélamtiques économiques. Sa lune sœur, Phobos, reste un habitat tunnel enr aison de la présence de multiple intérêt légaux incapable de se mettre d'accord sur l'utilisation du sattellite. 

\subsubsection{Valles-New Shanghai} \label{sec:valles-new-shanghai} 

la principale cité de Mars, valles-New Shangaï est le plus grand métroplexe de la transhumanité avec 37 millions d'habitants. Valles-New Shanghai lies se situe dans la régio Eos lourement terraformée à l'est du système de canyon de Valles Marineris. les métroplexe est composée de cinq dômes principaux connectés par un réseau de souks martiens. les souks sont une caractéristiques architecturale unique des grandes cités Martienne, comsistant de rue couverte et de gallerie remplies de bazar, de restaurants et de squatts. On dit que l'on peut tout trouver dans les souks, à condition de passer suffisament de temps à les parcourir. 

Les dômes en ont mêmem sont plus organisé, avec des canaux articifiels (dont la plupart se connectent maintenant aux rivières ténues qui gravent la surface des basses terres Eosiennes), une architecture grandiose, des mani-archologies résidentielles, des complexes de divertissement et des centres de conférences hypercorporatistes. Le plus impressionant de tous est le Bund, le plus grand et le plsu vieux des deux dômes qui forment la cité de New Shangaï a proprement parler. New Shanghaï est grossièrement divisée par la tumultueuse Ares, une rivière artificielle qui aide à réguler le climat du dôme. Non loin de son centre, on peut trouver une réplique presque exacte du Bund originel de la cité Terrestre de Shangahï maintenant détruite. 

les quatres autres dômes sont Little Shanghaï (un dôme plus petit et plus récent adjacent au Bund), Valles Center (un centre d'affaire et de finance qui rivalise avec els banques Lunaire d'Erato et de Nectar), ew Pittsburgh (également appelé le Bourg, un centre de recherche et d'industrie planétaire) et Nytrondheim (hébergeant les principaux quartiers de divertissement). 

Valles-New Shanghaï est le centre de population transhumaine le plus riche, le fopyer de l'art et de la culture et l'un des plus grand centre d'activité hypercorporatiste du système. La population comprend un nombre de gérontocrate extrêmement élevé, mais leur étouffante influence sur la culture, la mobilité économique et le système légale n'est qu'une force parmi d'autre dans une cité de 37 millions de personnes. La cité s'est tellement étendue pour s'adapter à l'explosion démographique de sa population depuis la Chute que les nouvelles constructiosns ont une constante. Le crime et la corruption sont largement répandus, même si le pire reste confiné à Little Shanghaï. Valles est l'endroit où les rêves sont fait et défait tout les jours, si ce n'est toutes les heures. 

\subsection{Troyen Martien} \label{sec:martian-trojans} 

À ne pas confondre avec les bien plus grands royens Joviens, les Troyens Martiens sont un petit groupe d'astéroïdes rocailleux qui suivent et qui précèdent Mars à ses oints L4 et L5. 

\subsubsection{Qing Long (Azurez Dragon)} \label{sec:qing-long-azurez} 

Qing Long, avec une population de 2 millions d'habitants, est le plus grand habitat O'Neill du système. Il est situé parmi les Troyens au point L5 de Mars. Qing Long prend ses racine dans l'effort Chinois de colonisation de Mars. En dépit de sa taille exceptionnelle, c'est l'un des plus vieux habitats de son type, ayant été construit presque entièrement à partir des astéroïdes riches en métal extraits près de sa position actuelle. 

Qing Long est un havre de m'économie sous-terraine. L'administration de l'habitat est redevable envers plusieurs organisation scriminelles qui s'abstiennent normalement de s'entretuer. L'habitat obéit normalement aux principes hypercorporatistes, tels que l'accès limité aux machines d'abondances, au fork et aux IAGs. Cependant, des marchés gris et noirs prospèrent permettent aux personnes avec les bons contacts d'acquérir à peu près tout et n'importe quoi. 

\subsection{La ceinture d'astéroïdes} \label{sec:asteroid-belt} 

Réparti sur une région trés étendue entre les orbites Martienne et Jovienne, la ceinture contient une petite centaine d'astéroïdes dépassant les 100 kilomètres de diamètres, un peu plus de milles objets dépassant les 30 kilomètres et bien plus d'objets plus petits. En dépit de ce fait, la masse totale d'astéroïdes dans la ceinture est à peine une fraction de la masse de l'une des planètes interne, les astéroïdes sont donc séparés les uns des autres par de grandes disances. Un vaisseau naviguant à travers la ceinture n'a quasiment aucune chance de croiser un astéroïdes, à moins de navigueru délibérément vers l'un d'entre eux. 

\subsubsection{Ressources et économie} \label{sec:asteroid-resources-economics} 

Les dépots de riche sen minerai et simple d'accès de la Ceinture ont été un des liens majeur lors des premières étapes de la transhumanité vers le système extérieur. L'extraction automatisés et les propulseurs ioniques à impulsion ont permis aux colons du système extérieur de déplacer les astéroïdes riches en métaux de la Ceinture Principale vers les orbites Joviennes, Saturnienne et au-delà, où les astéroïdes métallique sont bien plus rare. Cette activité continue actuellement alros que la transhumanité repousse les limites du système solaire. 

\subsubsection{Habitats} \label{sec:habitats-3} 

Des centaines de petits habitats, la plupart impliqué dans des activités de propsection, ponctuent la ceinture. Éloignés de la Terre, les abris de la ceinture ont été épargnés par la dévastation de la Chute. Les avant-postes hypercorporatistes et autonomes fleurissent eux. Des habitats dérivant, abandonnés lorsque les astéroïdes à proximité ont étés propulsés dans le système extérieur ou ont été épuisés sont également fréquent ici, même si certains d'entre eux sont maintenant occupés par des résidents qui préfère rester seul avec leur solitude. 

\subsubsection{Cérès} \label{sec:ceres} 

L'une des trois planète naine (avec Pluton et Eris), Cérès fait presque 1 000 kilomètre de diamètre et héberge une population d'un million d'habitant. Contrairement à la plupart des astéroïdes de la Ceinture Principale, Cérès possède une croute de glace au-dessus d'une couche d'eau liquide, comme une version miniature d'Europe, la lune de Jupiter. Avec son eau abondante, Cérès a un rôle majeur dans l'approvisionnement d'autres stations dans la ceinture. De manière similaire à Extropia, Cérès est dirigé principalement selon les des principes anarcho-capitalistes. Cepandant, l'Affaire Cachée, un cartel dirigé entièrement par des pieuvres élevées règne en maître sur les mers sous la croûte glaciaire et conservent la mainmise sur les oéprations d'extaction d'eau. Les morphs octopoîdes Cérienne sont spéciallement adaptée à la survie dans les eaux riches en ammoniaque de la Mer Cachée. 

\subsubsection{Extropia (44 Nysa)} \label{sec:extropia-44-nysa} 

Cet habitat en nid d'abeille massif est une place de marché importante de l'anarcho capitalisme. Extropia est une cité libre neutre dont l'infrastructure et le tissu social sont maintenus par une association informelle de groupes d'affinité anarcho-syndicaliste. La neutralité d'Extropia dépend des alliaances stratégiques entre les personnages clefs locaux, leur réseau et un nombre inhabituel d'intérêt extérieur incluant les banques Lunaires, les factions technolibertarienne et les colonnies du système extéireur dépendant des matériaux bruts importés de la ceinture. Les hypercorps utilise Extropia comme un paradis fiscal et comme un havre dans lequel il peuvent faire des affaires illégales. Il n'y a pas de lois ou de gouvernement en tan t que tel; on conseille aux visiteurs de souscrire à une assurance et à un prestataire de sécurité. Nommée d'après l'un des premiers mouvements transhumains, Extropia est considéré comme une utopie pour les transhumains cherchant des modificatiosn corporelles. Les IAGs et les forks sotn acceptés et autorisés ici. La population transhumaine atteitn les dix-millions. 

\subsubsection{Nova York (Métis)} \label{sec:nova-york-metis} 

L'un des habitats a faible gravité le plus inhabituel est Navo York, la principale cité de Métis, un grand astéroïde de nickel-fer de la ceinture principale. Nova York, le troisième habitat de la ceinture d'un point de vue taille, est une métropole prospère de 500 000 transhumains, dont la portion princiaple est située dans une cavrne sphréique d'approximativement quatre fkilomètres de diamètres, dont le  sommet est situé deux-cent mètres sous la surface de l'astéroïde. Éclairé pendant la journée par une série de gigantesque tubes lumineux incrustés dans les murs externes, les éclairages des bâtiments donnent à la sphère l'aspect d'une énorme géode pendant la nuit. La conception de base de l'habitat consiste en plusieurs milliers de bâtiments à l'apparence extrêmement fragile et exceptionnellement élevé qui s'élèvent de cent à mille cinq cent mètres au-dessus de la surface, ainsi que d'immeubles qui s'étendent d'une face la caverne à l'autre. Dans la micro gravité d'1/140° de g de Ménis, le haut et le bas n'ont que peu de signification, et même les bâtiments relativement fragile sont à l'abri de l'effondrement. L'immense majorité des immeubles, incluant ceux qui dépassent le kilomètre de hauteur, sont fait à partir de fins panneaux de plastique fixés sur une architecture de soutien durable. Ces bâtiments saillissent dans tous les angles depuis la sphère. 

Beaucoup d'habitants de Nova York se déplacent en s'élançant d'un bâtiment à l'autre et un simple saut peut envoyer quelqu'un à plusieurs centaines de mètres. Les résidents ne moquent de tomber - le mélange de la résistance à l'air et la gravité excessivemet faible signifient que même si quelqu'un tombait du du haut de la caverne vers le bas, il n'y a aucun danger pour lui d'être blessé. Dans cet environnement, la seule réelle signification pour le haut et le bas est l'endroit où vous vous attendez à ce que les objets se trouvent au repos (tant qu'un courant d'air ne les fait pas s'envoler et flotter autour). 

\subsection{Jupiter} \label{sec:jupiter} 

Suffisament grosse pour former le noyau du'une proto-étoile, la taille impressionante de Jupiter fait du Sstème Jovien l'un des endroits les plus difficile à coloniser. Le puissant champ magnétique de Jupiter implique que ses sattellites interne - ainsi que les plus éloignées, lorsque leurs orbites traversent l'immense queue magnétique - sont bombardées avec suffisament de radiation ionisante pour tuer en quelques heures tout transhumain non protégé par les boucliers les plus épais. Il y a soixante-trois lunes et sattellite dans le système Jovien, mais seuls les lunes les plus peuplées explorées et régulière sont décrites ici. 

\subsubsection{Ressources et économie} \label{sec:jupiter-resources-economy} 

Le pusissant puit gravitationnel de Jupiter est une gène majeure pour l'extraction gazière dans l'atmosphère de la planète, car même le matériel qui ne succombe pas aux violentes tempêtes durant pendant plusieurs siècles ne peut réussir atteindre la vitesse de libération que grâce aux système de propulsion les plus puissants. Étant donné la nécessité d'avoir des écrans de protection lourd sur ces apapreils, l'extraction gazière sur Jupiter est loin d'être aussi efficace que sur Saturne. Jupiter poqqède un petit système d'anneau, beaucoup moins dense que celui de Saturne, et qui s'étend sur 20 000 kilomètres autour de la planète englobant les orbites de ses deux lunes les plus proches. 

Cependant, la gravité de Jupiter est aussi une ressource de valeur. Les envois de matériel vers Saturne et au-delà peuvent utiliser l'effet de fronde en tournant autour de la planète pour augmanter leur vitesse, réduisant les voyages de plusieurs mois voire de plusieurs années. La République Jovienne lourdement militarisée prélève un droit de passage sur tout les vaisseaux utilisant la gravité de Jupiter pour augmenter leur vitesse, incluant les astéroïdes propulsés. Ce revenu de protectiviste est la principale source de revenue de la Junte. Les vaisseaux du Consortium Planétaire accèptent générallement de payer et compte la dépense dans les frais opérationnels. D'autres factions ne sont pas autant coopérative, et la Junte saisit ou détruit régulièrement tout ceux qui essayent de franchir le blocus. 

\subsubsection{Habitats et sattelittes} \label{sec:habitats-moonlets} 

la plupart des lunes de Jupiter sont en fait des astéroïdes capturés, et manquent de la taille et de la complexité géologique des corps planétaires. Toutes sont occupées. Certaines ont été converties en habitats; d'autres hébergent juste des avant-postes militaire et d'extraction minière de la Junte. Les sattelittes Joviens sont composés principalement de roche charbonneuse, quelques un des satellittes parmi les plus grands ayant aprfois des couhes ou de noyaux de glaces. Les habitats en nid d'abeille et les cylindres Reagann sont prdéominant dans le système Jovien. Les cylindres Reagann (appellée aussi "sarcophages" par toutes les autres factions) sont une varaiation inefficace des cylindres O'Neill dans lequel les excavateurs évident une gigantesque caverne cylindrique dans un astéroïde rocheux et modifient ensuite la rotation de l'astéroïdes par des propulseurs externes pour simuler la gravité. 

Les autres types d'habitats sont rare dans l'orbite Jovienne, particulièrement dans les 2 millions de kilomètres de la planète où les radiations sont au plus fort. Pour une faction bioconservatiste refusant d'adopter des morphs résistantes aux radiations, la Junte est trés mal située. Abriter sa population sous des tonnes de roches est une nécessité. En dépit de son hégémonie militaire, la Junte ne peut contrôller tout l'espace Jovien, et il y a des choses qu'elle ne peut faire d'elle-même - comme explorer Europe. Il existe un grand nombre d'habitats et d'abris indépendant dans le système d'anneau et sur les orbites des lunes Galiléenne. 

La République Jovienne a rebaptisé les lunes de Jupiter en s'inspirant de noms de divers héros néo-conservateur de l'histoire Terrestre. De la plus proche à la plus éloignées, les satellites sont Métis (Bush), Adrastée (Fairway), Amalthée (Solano), Thébé (McAllen), Léda (Chung), Himalia (Pinochet), Lysithéa (Friedman), Élara (Buckley), Ananké (Nixon), Carmé (Kissinger), Pasiphaé (Schilling) et Sinopé (Garcia). Toutes sont petites, entre 5 et 100 kilomètres de diamètre. 

\subsubsection{Amalthée (Solano)} \label{sec:amalthea-solano} 

Le pus gros des satellites, Amalthée évidée est probablement l'habitat sarcophage le plus vivable en raison de son grand lac créé grâce à son noyau glacé. Vivre sur Solano amène un certian prestige aux citoyens de la Junte. Des rumeurs établissent que la plupart de ses résidents sont des employés bien placé du think-tank RAND, travaillant principalement à des projets de défense. Un tube lumineux alimenté par fusion illumine la caverne central de 30 kilomètre de diamètre, dont le paysage est aménagé en fonction des sous-division et des pracs de bureaux des banlieues du début du 21° siècle. Tout les bâtiments ont une isolation environnementale pour que les épisodes éventuels de sépticémie occasionnelle résultant du manque de régulation de l'écosystème intérieur puisse être purgé par des bombes de toxines. Le personnel de maintenance moins riche habite dans les nid d'abeilles labyrinthique qui quadrillent la croûte du satellites entre la caverne et la surface. Comme la plupart des satellites de Jupiter, l'espace d'Amalthée grouille de vaisseaux de patrouilles et de satellites tueurs, rendant l'approche des vaisseaux non autorisées relativement problématique. 1,5 million de transhumaines vivent sur Solano. 

\subsubsection{Io} \label{sec:io} 

Sous l'atmosphére ténue et inégale composés de gazs volcanique et de poussière atomique neutre se trouve une surface rocailleuse, stérile teintée de jaune pisseux et recouverte d'une fine couche de dioxyde de souffre. Les vagues de chaleurs causées par l'interaction gravitationnelle avec Jupiter fait d'Io le corp ayant la plus forte activité volcanique du système - tellement actif que les cratères météorique trouvé sur toutes les autres planètes et lunes sont complètement absents. Des caldéras volcanique massive, des lacs de roches fondues et des geysers de souffre ponctuent la surface, avec les éruptiosn et l'activité sismique liée pouvant durer des mois ou des années. Les zonées volcaniques à la surface d'Io peuvent atteindre des températures de surface de 1 500 degrés Kelvin, plus élevées que n'importe quel autre corps du système. 

Mais le plus grand danger pour la transhumanité reste les radiations. Les débris éjectés par les geysers et les volcans voguent dans le champ magnétique de Jupiter pour former un flux toroïde titanesqie qui tourne avec Io autour de la géante gazeuse. Les voyageurs à destination de Io doivent utiliser les protectiosn anti-radiations les plsu élevées ou se réincarner dans des synthmorphs. 

L'humanité transhumaine sur Io est centrées autour de la recherche scientifique et l'exploitation des résidus éjectés par les geysers d'Io, le souffre en particulier. Les bases ont tendance à être modulaire et mobile en raison de l'activité sismique extrêmement changeante. La prison la plus célèbre de la Junte, le Centre de rRéhabilitation Maui Patera, est creusé dans un mur de calderas (presque) éteintes au nord de l'équateur. 

\subsection{Europe} \label{sec:europe} 

Europe n'a pas d'atmosphère et se situe dans l'effrayante magnétosphère de Jupiter, et en tant que telle sa surface est bombardée avec suffisament de radiation pour infliger une dose mortelle de radiation à un transhumain non protégé - bien plsu rapidement lorsque l'orbite d'Europe traverse l'immense queue magnétique de Jupiter. Les transhumains d'Europe résident donc essentiellement sous la croute glaciaire, essentiellement dans les océans sous la surface, adoptant une variété de morphs aquatique et amphibie pour survivre. La seule installation de surface est la tête de l'ascensseur de glace lourdement protégé de Conamara Chaos ainsi que d'autres endroits à travers lesquelles les réacteurs de masse et d'autres approvisionnement vitaux peuvent être amenés aux Européens sous la surface. 

La transhumanité est toujours en train d'explorer et d'imaginer le fond de l'océen Européen, une tâche compliquée par les pressions atroces qui sont à l'œuvre dans ses eaux, qui sont dix fois plus profonds que les océans Terrestre. Uner autre surprise qui attendait la transhumanité est le terrain. La géologie d'Europe suggère que sous la glace devraient se situer des des profondeurs insondables d'eaux noires se terminant à la profondeur de presque 500 kilomètres dans un lit marin relativement plat et sans caractéristique. Europa aurait été une boule de glace et de roche snas-vie, cela aurait été le cas, mais au fil des milliards d'énnaes estimés depuis l'apparition de la vie sur Europe, de petits mollusques (analogue aux coraux Terrestre) ont construit des récifs de silice qui s'élève jusqu'à quelques centaines de mètres sous la croûte glaciaire. C'est sur ces sommets de montagne biologique, foyer d'écosystème complexe, que les Européens ont construits leurs habitats. 

Tout en étant basé sur la chimie eau-carbone similaire à celle des origines de la vie sur Terre, la vie sur Europe est entièrement autochtones, ayant commencé sous une couche de glace impénétrable qui isole complètement les océans sous la surface d'Europe de l'extérieur. Ce contraste marquant avec la vie Terrienne a amené des biologistes à théoriser que cette forme de vie pourrait être le résultat dune panspermie galactique, le phénomène de lente diffusion de microbes à travers le vide spatial grâce aux comètes et aux astéroïdes. En tant que telle, la faune d'Europe est d'un intérêt majeur pour les biosciences transhumaines. 

\subsubsection{Biosciences} \label{sec:biosciences} 

Les formes de vie Européenne, probablement unique en leur genre dans l'univers, sont son plus grand trésor, et les efforts de la transhumanité pour la cataloguer n'est que le début. La ruée pour exploiter les fosses biologiques Européenne a mis la Junte Jovienne dans uen situation inconfortable. Alors qu'ils contrôllent le traffic spatial et le commerce dans le système Jovien, ils manquent de talent natif pour gagner un réel avantage des connaissances glanées sur Europe. Ils ont commencés par se lancer dans des opérations musclées et localisés ayant pour but de réduire les bénéfices de l'export de connaissances. mais une fois que les egocasteurs et les farcasteurs sont passés en ligne sous la glace, ce type d'extorsion n'as plus fonctionné. Les Joviens ont maintenant basculé sur une stratégie à deux volets consistant à augmenter les prix des nouveaux équipements et des personnes amenés par les ascensseurs spatiaux par les hypercorporation et les collectifs de recherche, et en prenant en otage toute la population de la lune en refusant de livrer les ressources clefs telles que les réacteurs de masse et les éléments rares si les taxes de protection ne sont pas payées. 

\subsubsection{Habitats} \label{sec:habitats-4} 

Les habitats Européens prennent deux formes principales: des abris fortifiés pour la pêeche et l'élevage s'accrochant aux flêches des récifs lithodermiques, et les dédales sphériques construit en perçant dans les couches basses de la croûte glaciaire et en étayant le vide créé. Les seconds sont les seuls espaces remplis d'air sous la glace. Le plus grand dédale est Conamara, au pied de l'ascensseur de glace de Conamara Chaos. Conamara est entouré par cinq abris dans des récifs proches, et également considéré comme faisant parti de l'habitat. La pouplation totale est de 1,5 millions. 

\subsection{Ganymède et Callisto} \label{sec:ganymede-callisto} 

A peu près de la même taille que la Lune, mais d'une teinte plus sombre et avec moins de cratère, Ganymède et Callisto sont des mondes trés similaires. Aucun des deux n'est trés dense (et ne possède pas non plus d'une forte gravité), car leur manteau est constitué de plus de glace que de roche ferreuse. Les deux possèdent d'importantes résevre d'eau (bien que sous forme de glace) et de composé organique volatile, en faisant des candidats idéaux pour l'habitation. Ganymède, avec ses surfaces différenciés de terrain rocheux et gelé, a un noyau ferreux et possède donc un faible champ magnétique. Callisto, le plus petit des deux, est composé principalement d'argile de silicate gelée. Comme sur la Lune, la plupart des villes sur Canymède et sur Callisto sont construite sous la surface pour les protéger des impacts de météorites (et, sur Ganymède, du bombardeùment raidoactif de Jupiter). 

Tout en étant sous la "protection" de la République Jovienne, les deux lunes sont un assemblage de cités états. Certaines font parti du régime politique Jovien, tandis que d'autres ne sont que tolérées. Ganymède a tendance à pencher plus lourdement vers la Junte, car ses citoyens perçoivent toujours - de manière précise ou non - l'infrastructure maintenue par la Junte comme nécessaire pour survivre dans cet environnement hostile. Callisto, située hors de portée des pire effets radiocatifs de la magnétosphère Jovienne, est une zone permettant plus facilement aux technoprogressiste de s'installer. 

\subsubsection{Hyodène} \label{sec:hyoden} 

Le noyau de cette cité-état était une station de recherche fondée par une coallition des nations de la Bordure Pacifique dans la région Valhalla de Callisto, une gigantesque zone d'impact où la glace sous-terrain est exposée, simplifiant l'extraction d'eau pure. Lorsque vint la Chute, Hyodène, qui faisait face depuis longtemps à une pénurie de main d'œuvre, s'ouvrit aux réfugiés qui purent atteindre Jupiter. Maintenant, Hyodène a deux millions d'habitants, en faisant la plus grande cité-état sur Callisto, et la plus grande dans les état hors Junte du système Jovien. Hyodène en elle-même est lourdement militarisée, car la tendance des autorités locales à fermer les yeux vis à vis des opérations utilisant leur territoire pour des incursions contre la Junte rend les relations avec leur puissant voisins quelques peu tendues. 

\subsubsection{Liberty} \label{sec:liberty} 

Situé le long de la limite sud est de la vaste plaine rocailleuse appelée Galileo Regio, pratiquement sur l'équateur de Ganymède, Liberty (population de 7 millions d'habitants) est la plus grande cité-état planétaire de la Junte. Elle est étroitement liée à Liberty Station, un chantier naval et une installation de défense majeure située en orbite géosynchrone. Les industries principales incluent la construction navale, la fabrication et les produits et services de sécurité. Le Château, le point central du réseau de sécurité depuis lequel toutes les données de surveillance collecté dans la Junte sont monitoré et analysés, est supposé être dans Liberty ou dans son environnement. Liberty est essentiellement sous-terraine mais elel abrite de nombreux parcs dans les dômes de surfaces blindés. Si quelqu'un passait suffisament de temps à regarder le ciel, il verait les torches de déccélération des astéroïdes métalliques arrivant depuis la ceinture a destination des chantiers navals éclairer le ciel plusierus fois par jour. 

\subsection{Les Troyens (astéroïdes Grecs et Troyens Joviens)} \label{sec:troj-jovi-troj} 

Les Troyens et les Grecs sont deux arcs de 600 millions de kilomètre de long composé d'astéroïdes de roche et de glaces partageant l'orbite de Jupiter. Ils orbitent autour des points L4 et L5 stable soixante degrés en avant et en arrière de la planète géant. Mars et Neptune possèdent également des astéroïdes Troyens, mais lorsque quelqu'un part "des Troyens", il parle en général des groupes Joviens. Au début de l'astronomie, les astéroïdes en L4 (précédent Jupiter) ont été appelés d'après les héros Grecs de l'Illiade Homérique; les astéroïdes en L5 (suivant Jupiter) sont appelés d'après les héros de Troie. Les astéroïdes découvert plus récemment ne respectent plsu la vieille convention, car il y a bien plus d'objets dans les Troyens que de personnages dans l'Illiade. 

Politiquement, les Troyens et les Grecs peuvent être vus comme une collection de voisinages se chevauchant aprfois dont les habitants tendent à se grouper autour de cultures, de factions et parfois de langues particulières. Un voisinnage dans les Troyens peut s'étendre sur une surface comprise entre 250 000 et 2 millions de kilomètres. A l'intérieur d'un voisinnage, presque toutes les personnes se connaissent. A cause de la grande dispersion des ressources, les habitats Troyens ont tendance à être petits - mille ou deux milles personnes - et sont largement construit le long de barge racaille ou de ligne de regroupement (même si il n'est jamais sage de faire référence à l'habitat de quelqu'un comme à une barge racaille sauf si ils y font d'abord référence de cette manière). 

\subsubsection{Ressources et économie} \label{sec:trojans-resources-economics} 

Même si la taille absolue des deux régions implique une grande diversité culturelle, l'anarcho-collectivisme est un même puissant ici et l'économie réputationnelle prévaut sur les autres. D'un côté, on attends des voisinnages, des habitats et même des individus à ête auto-suffisant. Contrairement à la Ceinture Principale plus dense, les Troyens manquent du filet de sécurité fournit par l'envahissante présence transhumaine. Le Troyen ou Grec idéal est un Néo-renaissant, incroyablement compétent dans un grande variété de domaine. Quelqu'un qui ne peut pas manœuvrer en zéro g; entreetenir son équipement, son vaisseau et son habitat; et naviguer entre les rochers et les habitats peu vivre des moments difficiles. D'un autre côté, un esprit de coopération prédomine. Échanger des services ou même les offrir pour bénéficier d'un gain en réputation est fréquent. Tout le monde apprécie un spécialiste, tant qu'ils ne se sont pas spécialisé au dépend de son auto-suffisance. 

La prospection et la récupération sont les activités majeures dans les Troyens, où les métaux et gaz rares sont rare et les colons n'ont habituellement pas l'impact écnomique nécessaire pour importer les matériau bruts d'ailleurs. Cependant, les Troyens sont riches en silice, en composé organique volatile et en matériau charboneux. La nécessité a poussé à innover dans la science des matériau. Au-delà du simple problème des matériau bruts, les habitats extrêmement dispersés des Troyens ont du être extrêmement ingénieux à de nombreux niveaux pour garder leur indépendance. De nouveaux concept de robots, de morphs et de véhicules apparaissent tout le temps, rendant possible une gamme inhabituelle d'affaires et de loisirs, comme le whaling (organisé une flotille express pour miner rapidement des astéroïdes et des comètes avec des orbites erratiques lorsqu'ils passent près des Troyens), le mekking (du combat simulé - ou parfois réel - entre des combinaisons robotiques oud s syntjmorphs sur des astéroïdes inhabités ayant des terrains intéressant) et le shrining (s'embarquer discrètement sur un habitat et le resurfacer avec des nanites sculpteurs popur créer un objet d'art - essentiellement un passe-temps de barge racaille). 

\subsubsection{Locus} \label{sec:locus} 

Locus est le plus grand habitat en grappe jamais formé. Il s'étend toujours, avec plus d'un million d'hébiatnt dans l'habitat à proprement parlé et un autre million dans la banlieue composé de barge racaille et de plus petites stations astéroïdes. Locus est située sur la Bordure de Cassandre, l'une des régions les plus denses des Troyens L5. L'habitat est positionné au centre de masse aurtour duquel les deux astéroïdes constituant l'objet binaire Patroclus orbitent l'un autour de l'autre. Les deux astéroïdes de Patroclus sont eux-même habités et abritent des installations défensives, des mines et des raffineries. 

La cocneption de Locus est extrêmement similaire au bine plus petit Lot 49, mais Locus fait onze kilomètre de diamètre et à une forme quelque peu irrégulière, la croissance le long de certains espars étant plsu rapide que d'autres. Un quart de son volume total est évidé en une forme vaguement conique jusqu'à l'Amoeba, une immense sculpture luminescente au centre de l'habtitat. Locus est différente des habitats grappes plus petits des Troyens en raison de sa taille. Les immenses espars structurels rayonnant depuis le centre de l'habitat sont évidés, est sont parcourus par des voies de flottage artérielle et des trains-ascensseur. Des espars plus petits sont tendus entre les espars artériels, fournissant plus de point d'ancrage pour les modules. De large "routes" amenant au limites de l'habitat pour que les modules capables de manœuvrer de leur propre chef puisse partir si leur propriétaire décidait de partir sont adjacentes aux espars artériels. 

Sous le  filet chatoyants tendus au-dessus du carde géodésique afin de garder éloigné les micro-astéroïdes, des dizaines de milliers de petits vaisseaux et de module d'habitation sont amarrés le long des espars et pulsent un tableau de lumière toujours changeant. Il est demandé aux modules d'habitation et les plus grand vaisseau de rester hors de l'espace conique vide. Cet espace grouille de petit vaisseau et de personne équipé de propulseurs ou d'exoscooters alors qu'ils traversent l'habitat, jouent à des jeux en zéro-g ou visitent les spimes et sculpture flottant librement dans la zone. L'Amobea, qui change périodiquement de couleur et de forme en fonction de la programmation de son IA résidente (elle ressemble souvent à un animal), sert de point central de référence pour la navigation. Quand quelqu'un donne l'adresse d'un module, c'est un point dans un système de coordonnée sphérique centré sur l'Amoeba. Les plus grand vaisseaux et les navettes appontent sur la surface extérieure de l'habitat, aux points terminaux des espars artériels. 

Locus a été fondé par une association anarchiste/argonautes et a été la première place forte des factions autonomistes. Contrairement à Extropia, qui bénéficie d'une bénédiction tacite du Consortium Planétaire et qui encourage la présence des compagnie d'assurance et de sécurité, Locus fonctionne sur une économie réputaionnelle pure. La sécurité, la miantenace, l'expenseion et la défense de l'habitat sont tous effectués par des volontaires. Les habitants interessés par la sécurité surveillent les vaisseaux arrivants et dirigent des système de crowdsourcing qui allouent les volontaires pour réaliser des scans ADM sur les nouveaux arrivants. Les vaisseaux refusant de se soumettre à un scan se voient demander de partir. Si ils refusent, quiconque qui aurait récemment conçu une nouvelle arme sympa est autorisé à tirer. 

Locus est un point central de la guerre froide entre les puissances du système intérieur ayant prété allégeance aux hypercroporations et la coallition informelle des intérêts du système extérieur. Alors que les saboteurs du Consortium Planétiare et d'autres entités hostiles peuvent (et y parviennent ccasionnellement) générer des troubles sur Locus, les hypercorporation ne sont actuellement pas partisans d'une attaque militaire frontale sur l'habitat. La première fois qu'ils ont essayé, le Consortium Planétair et la cité-état Martienne de Valles-New Shanghaï ont envoyée une petite flotte expéditionnaire. Les assaillants ont été pris complètement à dépourvu par une défense acharnée et bien coordonnée. Six mois plus atrd ils envoyèrent une flotte bien plus grande. De l'aide arriva de d'autres endroits des Troyens et des Grecs ainsi que de Titan, dont les citoyens voyaient d'un mauvais œil toutes les tentatives d'extension du Consortium Planétaire au-delà de la ceinture. Les Titaniens maintiennet toujours une base permanente près de Locus. la rumeur veut qu'ils se sont mis d'accord sur un pacte de défense mutuelle avec l'un des citoyens de Locus, probablement le célèbre programmeur-bretteur Teilhard Liu. 

\begin{quotation} “Bienvenu sur Locus. Vous acceptez volontairement le risque de dommage organique ou de traumatisme mental en vous arrimant ici. Vous devez apporter ou être capable d'acquérir suffisament de nourriture, d'H2O, d'oxygène et de protection pour survivre pour la durée de votre séjour dans un environnement rude et riche en astéroïdes. Les armes de destructions massives sont interdites. D'autres lignes de conduites pour coexister avec vos entités voisines sont disponible dans le guide de survie en habitat. Vous et seulement vous êtes responsable de vous-même - apprenez à l'apprécier!" 

-- Locus Immigration diffusion RA 

"Vous avez choisit l'habitat Locus dans les Troyens L5 comme destination, utilisant l'opérateur privé Atsuko van Vogt comme récepteur. La politique d'entreprise de ComEx nécessite que nous vous informions que la destination et l'opérateur que vous avez choisit ne sont pas enregistré et probablement non-sûr. ComEx ne peut-être tenu pour responsable des dommages subit par la continuité de votre existence à partirt de votre arrivée. Vous assumez tout les risques pour voyager jusqu'à ce point, incluant le vol de fork ou la suppression. ComEx inclueras un enregistrement permanent de ce voyage. Continuer?" 

-- ComEx mentions légales 

“Les mentions légales de CoMEx? OUi, oui ... Écoute: mon voisin à trois portes vers l'Amoeba d'ici est un physicien. Elle a une boîte qui génère des micro-singularités dans son labo. Si des personnes dans mon espar découvrent que j'ai volé le fork de quelqu'un, ils vireont ma pile avec un couteau à pamplemousse et la jetterons dans cette boîte. C'est ce que nous appellons 'responsabilité'. Regardes si tu peux obtenir autant de ComEx." 

-- Atsuko van Vogt \end{quotation} 

\subsubsection{Lot 49} \label{sec:lot-49} 

Lot 49 est amarré au petit astéroïde 28349 Pynchon dans le voisinnage informe de Vonaburg-Shadyside, près du centre des Grecs L4. vanorburg-Shadyside tire son nom de deux rochers qui définissent grossièrement ses limites de 500 000 kilomètres le long d'un arc de l'orbite Jovienne. Les habitats avoisinant dans un rayon de 100 000 kilomètre (et leur population) incluent Craftsbury (450), Greenview (28) et Blackhawk (1020). Avec une population de 400 habitants, cette stations et plus ou moins typique des Troyens en terme d'agencement. 

De l'extérieur, Lot 49 ressemble à une sphére géodésique brillante de 800 mètres de diamètre et recouverte d'un filet, avec de nombreux espars protubérant et quelques triangle laissé ouvert à l'espace pour que les navettes pusisent traverser. l'amarrage à l'astéroïde est temporaire au cas où une collision potentielle est détectée. A l'intérieur, un module utilitaire central équipé d'un réacteur communautaire, d'usines et d'une salle des machines est entouré de module d'habitation régulèrement espacé mais de formes irrégulière dans une débauche de motifs cde couleurs et d'éclairage. Les espars structurels et des voies de flottaison connectent le tout. Un espar entier est dédié à un module cylindre rotatif qui génère à peu près 0,7 g et qui contient des installations médicales, de clonages, de réincarnation et d'egocast illégaux. 

La population de Lot 49 et la plupart de leurs voisins dans Vonarburg-Shanyside ont tendance à s'aligner avec la racaille et les factiosn anarchistes et parlent un mélange d'Anglais, de Portuguais et de Thaï. Lot 49 est située dans une zone densément habitée des Grecs, la plaçant près de différents carrefours. L'activité économique principale inclue la conception de de navette, le whaling et le convoyage de personne et de bien dans la région. 

\subsection{Saturne} \label{sec:saturn} 

La deuxième planète du système solaire par la taille est un habitat bien plus favorable pour les transhumaines que Jupiter. La gravité plus faible de Saturne et une magnétosphère plus faible sont une aubaine pour les opératiosn d'extraction de gaz, et pour les habitats gros consommateurs de ressource les anneaux sont un véritable festin (litéralement, dans le cas du nouveau type d'habitat en cylindre Hamilton). Les hypercorporatiosn sont présentes ici, mais toute extension majeure du Consortium Planétaire est validée par les satatiosn anarchistes des Anneaux et par le Commonwealth technosocialiste de Titan. 

En raison de l'éloignement de Saturne apr rapport au Soleil, la génération d'énergie solaire est extrêmement inefficace. Faire pousser des plantes à photosynthèse avec de la lumière solaire est impossible sans de grands panneaux solaire pour concentrer les rayons. L'abondance d'eau et de composés organiques volatiles rendent les anneaux idaux pour les barges racaille et les cylindres Hamilton. L'extraction de gaz est vitale à la survie de rpesque tous les habitats et abris lunaires dans le système Saturnien, les habitats les plus éloignés de la planète et qui veulent être auto-suffisant maintiennent donc presque toujours leurs propres statiosn d'extraction de gaz à proximité de la planète. La sécurité autour de ces installations ainsi qu'autour des dragues et tankers atmosphériques qu'elles expédient est extrêmement serré, et il n'est jamais sage de s'en approcher sans s'annoncer. 

\subsubsection{Ressources et économie} \label{sec:saturn-resources-economics} 

L'extraction de gaz sur Saturne fournit trente pourcent de la masse réactive du système. Ce rôle se développera probablement au fur et à mesure que les gisement d'Hélium-3 dans la régolite Lunaire deviendront de moins en moins accessible. Pour les vaisseaux voyageant dans les régions les plus éloignées du système extérieur, Saturne est une alternative importante à Jupiter pour l'assistance gravitationnelle. Moins restrictifs que les régimes Jovien et plus riches en resources que les Troyens, les habitats et abris Circumsaturnien sont des innovateurs important dans les domaines de la conception d'habitat et de l'organisation culturelle. Depuis la découverte des Portes de Pandorre, le Commonwealth Titanien est la seule entité à poursuivre l'exploration interstellaire par des moyens conventionnels. 

\subsubsection{Les anneaux et les lunes mineures} \label{sec:rings-class-minor} 

Les anneaux de Saturne sont constituée d'une quantité indénombrable de petits objets glacés, là plupart d'entre eux d'une taille allant d'un grain de poussière à des boules de 10 mètres de diamètre. Les anneaux sont désignés par les lettres de "A" à "F" dans l'ordre de leur découverte. Ils varient en épaisseur de 100 à 1 000 mètre et en largeur de 20 000 kilomètres à à peine quelques mètres. Il y a des vides entre certains anneaux. Le plus large, la division Cassini, fait 4 000 kilomètre de large. 





\begin{quotation} \textbf{Rapport d'expédition dans la Porte 901127} 

Configuration du Code Porte: [Chiffré] 

Réservé à une diffusion interne à la Corporation Gatekeeper. 

Les rapports préliminaires des drones et des capeturs semblent indiquer que les environs de l'exoplanète de la porte se situaient sous-terre, dans uen caverne formée de roche charbonnière avec une atmosphère de dioxyde d'azote. Il n'y avait aucun signe de vie ou d'activité consciente. Une escouade de resquilleurs a été envoyà à travers, guidés par un drône d'exploration, avec un lien de communication jusqu'à la porte. 

Approximativement une heure après que l'équipe se soient engagées dans les tunnels, la communication consistante a été perdue en raison des interférences électromagnétique. A ce moemnt là, ils n'avaient rien signalé de plsu notable que leur déplacement d'un kilomètre dans un dédale de tunnels. 

Nous n'avons plus jamais entendu parler de l'équipe. 

Deux heures après que le contact fut perdu, un drone attachés de recherche et de sauvetage a été déployé. En suivant la piste des resquilleurs, près des limites de son lien, le drône a trouvé ce qui s'est avéré être une main dans un gant d'exocombinaison. Les tests ADN n'ont pas identifié la main comme appartenant à l'un des membres de l'équipe et n'ont pas non plus permis de touver de résultats dans d'autres bases de données. Le drône a été détaché de son lien pour explorer plus en détail, mais rapidement après les capteurs ont enregistrés une activité sismique et la communication avec le drône a été perdue. 

La porte est restée active pendant encore 8 heures - la durée du contrat - sans signe d'activité. La porte a été fermée, l'équipe a été signalée comme perdue/irrécupérable, et les réglages de la porte ont été enregistrés et marqués d'un drapeau orange. \end{quotation} 





Saturne possède plus de 60 satellites, un nombre qui atteint la centaine si on considère les objets non dénombrés de moins d'un kilomètre de diamètre et orbitant dans l'anneau A. la plupart des lunes de Saturne sont de petit objets rocailleux et glacés de moins de 100 kilomètres d diamètre. La plus petite des lunes classique, Pan, ne fait que 10 kilomètres de diamètre. Les huits première lunes, de Encelade et vers l'intérieur, sont dans le système d'anneaux. Atlas, à la limite de l'anneau A, ainsi que Prométéhe et Pandorre, qui encadrent le fin anneau F, sont appellées les Satellites Bergers. Plusieurs satellites occupent les points de Lagrange des lunes les plus grosses. Télesto et Calypso partagent l'orbite de la bien plus grosse Téthys, et Hélène suit une autre grosse lune Dioné. 

\subsubsection{Atlas (Volkograad)} \label{sec:atlas-volkograad} 

Volkov, un cartel de l'énergie Slave, contrôle cette petite lune. Volkograad est une habitat nid d'abeille d'à peu près 50 000 résidents. L'essentiel de la lune est dédiée aux infrastructure de dragage, de raffinage et d'expédition. Un nuage d'épave suivant le sateliite sur presque 100 000 kilomètres sert de rappel de l'Incident Atlas, une bataille rapide mais extrêmement destructrice qui se délcencha lorsque Fa Jing essaya de prendre le contrôle de la lune. Les chaudronniers du Refuge de Phelan continue de récupérer les débris flottant régulièrement. 

\subsubsection{Dioné (Thoroughgood)} \label{sec:dione-thoroughgood} 

e principal abir de Dioné est Thogouhgood (350 000 habitants), un habitat hybride nid d'abeille-en grappe installé sur un plateau au milieu d'impressionante falaise de glace. Dioné habrite la Longue Flêche, un espar de communication de 150 kilomètre de haut s'élevant de l'abri de surface vers uen station orbitale qui fait office de contrepoids. La grande taille de la Longue Flêche est un coup de pub, l'essentiel de sa capacité restant inexploite. Elle attire cependant suffisament d'attention pour faire de Thoroughgood un concentrateur de communication pour le système extérieur, et est donc un endroit où les intérêts hyeprcorporatiste, anarchistes et d'autres factions se recoupent. Dioné partage son orbite avec hélène, une petite  lune rocheuse à son point L4, et Pollux, un corps encore plus petit qui la suit sur son point L5. 

\subsubsection{Encélade (Profunda)} \label{sec:enceladus-profunda} 

Riche en composé organique, Encélade est un terrain de jeu pour la chimie organique. Profunda (850 000 habitants) en est l'abri principal, un nid d'abeille creusé dans la surface de la lune et plaffonés de parc sous dômes et des grappes d'élégants minarets translucides - bien protégés des collisions par un réseau agressif de satellite de défense. Les niveax les plus bas, s'enfonçant profondément dans la manteau de silice d'Encélade, incluent des opératiosn de prospection qui extrayent le sol carbonneux à la recherche de composés exotiques. Une autre section profonde a été convertie en une mer primordiable réchauffée par un réacteur et s'étendant sur plusieurs kilomètres, faisant parti d'une expérience à long-terme sur les orginies de la vie soutenus par l'assocition d'académiciens Titaniens et un collectif de biochimistes Encéladiens. 

Profunda est régie par les principes anarcho-capitalistes. Grâce au stocke généreux de copposés chimiques organiques, ses niveaus supérieurs sont le foyer de nombreux concepteurs de morph le splus connus du système extérieur. Le Bloc Scintillant Encéladien est connu pour avoir autanty d'influence sur le style corporel que les maisons de la mode Lunaire sur ce que porte les gens. 

\subsubsection{Épiméthée et Janus (Twelve Commons)} \label{sec:epim-janus-twelve} 

Ces petits satellites glacés jumeaux partagent virtuellement le même chemin autour de Saturne, orbitant à cinquante kilomètre l'une de l'autre. Située entre les anneaux F et G, les satellites forment les centre des Twleve Commons, un voisinnage de petits habitats organisé en un nuage plat de 20 000 kilomètre de rayon. Approximativement six millions de personnes vivent dans les Twelve Commons. Les habitats dans les Twelve Commons ont une taille allant de Dang Fish Eco, une boîte de conserve hébergeant quelques 60 aquaculteurs excentriques à Janus Common, un nid-d'abeille occupant la plupart de Janus avec une population de 900 000 habitants. Certains des habitats dans les Twelve Commons possèdent des conceptions extrêmement inhabituelle, telle que celle de Nguyen Compact (80 000 habitant), une variante des habitats Cole dans l'anneau G pour lequel un astéroïde a été réchauffé et à travers duequel de grosse quantité de vapeur ont été envoyé afin de créer une série de bulle sinterconnectées faisant entre cinq et trois cent mètres de diamètre. En pratique, l'intérieur de la colonnie ressemble à de la mousse solidifié ou à un fromage Suisse sans posséder de haut ou de bas. Sans un ecto oiu des implants de bases pour fournir des information de localisation et de navigation, navigeur à travers cet habitat labyrinthique est extrêmement difficile. 

Les habitats des Twelve Commons s'organsient principalement selon les préceptes de l'open source et de l'anarcho-syndicalisme, av ec des groupes de travail et les pods de recherche constituant l'unité politqiue de base. 

\subsubsection{Gateway (Pandora)} \label{sec:gateway-pandora} 

L'abri Gateway sur la lune bergère Pandora, la plus éloignée de Saturne, possède la première porte à trou de evr révélée au public. La Corporation Gatekeeper maintient la porte ouverte en temps que moyen d'exploration et d'investigation scientifique pour toutes les factions et toutes les puissances. Gatekeeper était à l'origine une microcorporation Titanienne mais est maintenant indépendante. Le Commonwealth de Titan y est toujorus fortement impliquée, même si elle n'en contrôle plus les intérêts. Accorder l'autonomie à la Corporation Gatekeeper était une manœuvre diplomatique faites en réponse aux affirmations du Consortium Planétaire prétendant que les Titaniens cherchaient à obtenir l'hégémonie sur le système extérieur. Jusqu'à présent, les voisins de Titan y croient, mrme si ce n'est pas le cas du Consortium Planétaire. 

\subsubsection{Hyperion} \label{sec:hyperion} 

Avec sa rotation chaotique et virtuellement imprévisible, Hyperion est un endroit dangereux pour poser un vaisseau. Elle reste inahbitée. 

\subsubsection{Japet} \label{sec:iapetus} 

Japet est l'une des plus grosse lunes glacée de Saturne et abrita autrefois une population de 200 000 personne, vivant dans le dédale dense d'Analect, son abri principal. Probablement car c'ets une des quelques grosse lune de Staurne qui contient des gisement de sillice et de minérau conséquent, en plus de la glace, Japet fût l'une des cibles des TITANs pendant la Chute. Après avoir asservi un dizième de la population comme drône de travail et utilisant le reste comme stock de culture de tissu humain pour nourrir les leurs, les TITANs ont commencé à construire ce qui semble avoir été un cerveau gigogne. Japet occupe maintenant deux fois le volume qu'elle possédait autrefois, la galce et la sillice des coucehs externe de la lune ayant été retravaillé en un délicat treillage de circuits sur des millions de couche d'épaisseur. 

Étrangement, le projet est simplement arrété quelque part avant de se terminer. Les spéculations vont bon train et partent du principe que l'intelligence qui contrôlait le projet a été détruite par une force extérieure inconnue ou qu'elle s'est dévorée elle même dans un accés de calcul extatique. Peu importe ce qu'il s'est passé, les drônes ont simplement cessé de fonctionner et son mort et la grilel de dféense automatisée de la lune s'est désactivée, abandonnant une machine étrangement belle mais dénuée de vie à la lente décomposition du dombardement météorique et au stress gravitationnel. Plusieurs équipes de recherche habitent maintenant dans de petites statiosn orbitales, se querellant autour des miettes. Les rumerus vont bon train quand au fait qu'un certain nombre de chercheurs essayant de comprendre les circuits gigogne ont perdu leur esprit dans le processuss, probablement suite à un mécanisme proche des exploits basilique. Ceratines des défense interne de la lune seraient peut-être encore active. Si quelqu'un a réussi à aller à l'intérieur et à en revenir, il n'en a jamais parlé. 

\subsubsection{Meathab} \label{sec:meathab} 

Le nom complet de cet habitat unique est Transforme Toi en une Masse Géante de Viande de l'Espace pour l'Art!, et comme son nom l'indique, 90\% de la structure de l'habitat est composé de d'exo bacon développé en cultur rapide grâce aux abondante sressources du système d'anneau. MeatHab a démarré comme la morph artistique de quelqu'un, mis, et contrairement à toutes attentes, des squatteurs s'y sont installés. MeatHab a maintenant une population de 500 habitants. De manière similaire à un cylindre Hamilton, l'habitat long d'un kilomètre extrait et transforme les matériaux des anneaux pour se développer. La surface extérieure est une couche de viande congelé épaisse de ix mètre dont la surface serait un croisement entre un tron d'arbre et un steak. Au delà du point d'amarrage dans situé sur l'axe spatial, il y a un labyrinthe de corridors veineux recouverts de peau éclairé par des paneaux luminescent et maintenus par de petits symbiotes reptiliens qui mangent la peau morte et pourrait bien avoir d'autres fonctiosn immunitaire. La gravité interne est de 0,5 g. 

Le bioconcepteur sans nom qui a créé l'endroit - et qui pourrait habiter ou pas la morph gigantesque - était un génie. Même si l'habitat ne pourrait jamais être un lieu plaisant, quelque soit votre imagination, il apparaît sain. Son fonctionnement complet n,est pas compris et les habitants sont soit des monstres de chaire extrêmes qui sont fan de l'artiste ou des bioconcepteurs étudiant le lieu pour en apprendre plus sur sa construction. 

\subsubsection{Mimas (Harmonious Anarchy)} \label{sec:mimas-harm-anarchy} 

Mené par le poète dissident légendaire Hao Lin Ngai, Harmonious Anarchy s'est séparé du cartel Fa Jing pendant les années tumultueuses qui ont précédées la Chute. Hao cherchait à créer une société dans l'esprit de l'anicen état Taoïste de la Grande Perfection qui existait au Szechuan 1 700 ans plus tôt - avec d'importante mie à jour de la pensée moderne. Harmonious Anarchy est une société mutualiste Extropienne fortement impliquée dans la conception logicielle, la logistique et la relocalisation des astéroïdes métalliques vers le système extérieur. L'essentiel de Mimas est un nid d'abeille à très faible gravité disposé dans les voisinnages Noir, Rouge, Jaune, Vert et Blanc, d'après les cinq directiosn classiques de la mythologie Chinoise. Chaque couleur abrite une caverne centrale ornementée, des galeries rayonnant autour de ces cavernes abritent des familles étendues. Tout en adhérant aux principes économiques mutualiste, Harmonious Anarchy conserve simultanément une approche traditionnelle Chinoise de l'organisation sociale en mettant la famille au cœur de cette organisation. 

\subsubsection{Satellites Nordique, Inuit et Celtique} \label{sec:norse-inuit-gallic} 

En plus des satellites classiques décrit ici, trois groupes d'objets plus petit inconnus aux premiers astronomes orbitent autour de Saturne. Ces satellites sont appelés les groupes Inuit, Celtique et Nordiques. Comem ces satellites étaient toujours peu explorés au moment de la Chute, la plupart d'entre eux restent trés peu peuplés. A de rars exceptions, les habitants de ces satellites sopnt généralement des personne qui recherchent la sollitude. Les exceptions sont Skathi et Abramsen (anciennement S/2007 S 2), qui, avec Phoebe, qui ont été cpaturés et déplacés dans l'orbit de Titan où ils servent d'installation de défense. 

\subsubsection{Pan (iZulu)} \label{sec:pan-izulu} 

Le plus proche concurrent de Volkograad est un colelctif anarchocapitaliste dont la plupart des fondateurs étaient Sud Africains. iZulu a une capacité légèrement inférieure à Volkogrrad mais envoie des masses réactives a des localisation inhabituelles comme les Troyens et la Ceinture de Kuiper. iZulu est un nid d'abeille surpeuplé avec pas loin de 400 000 habitants et un nombre inhabituellement élevé d'infugiés. Les nations de l'Afrique sub-Saharienne n'ont commencé à atteindre les niveaux de propserité lagrment répandus au 20° siècle qu'à la fin du 21°, et ont donc été la région terrestre avec la plus faible capacité d'évacuation physique de leus citoyens lors de la Chute. iZulu et une poignée d'autres habitats avec des racines en Afrique ont donc une populations d'infomorphs élevé et des millions de personne stockés en mémoire morte. 

\subsubsection{Refuge de Phélan} \label{sec:phelans-recourse} 

Le Refuge de Phélan (générallement appelée "Phélan" par ses habitants) est le plus vaste abri nomade du système, avec une population avoisinnant les 250 000 personnes. Phélan est une nuée composé de 1 000 petit vaisseaux et modules boîtes de conserve qui orbite autour de Saturne le long d'un chemin extrêmement élliptique et quelque peu incliné par rapport au plan élliptique. L'orbite de la nuée est calculée pour maximiser le nombre de rencontres avec des lunes et des stations proches, fournissant une fenêtre de six à huit heures pendant laquelle les vaisseaux peuvent quitter la nuéer pour commercer. Le Refuge de Phélan passe à travers les anneaux une fois par mois, permettant aux vaisseaux de se réapprovisionner en eau et en composés organisques volatiles. 

Phélan accepete tous les arrivants. On peut rencontrer tout le monde ici, du gouvernement en exil du Timor Oriental à Hasidim de Brooklyn. Le cœur de cette nuée est l'Alambic, une ferme de céréale et une distillerie illuminé par la fusion et dirigée par un gang prétendument réformé de voyageurs Irlandais et qui ont magouillé pour quitter la Terre quelques semaiens avant la Chute et qui se sont échappés dans le système extérieur. L'Alambic produit la Phélan Ma, le whiskey le plus recherché dans le système, et Phélan Da, probablement la pire bière jamais brassée. En dépit des protestation de légitimité de Phélan, l'élément ccriminel y est trés fortement présent. La nuée représente un lien important dans les chaïnes d'approcvisionnement des marchés rouges et gris. 

\subsubsection{Prométhée (Marseilles)} \label{sec:prom-mars} 

Marseilles (80 000 habitant) est un habitat nid d'abeille contrôllés par les Titaniens. Il est suspecté d'abriter une usine d'antimatière, une théorie soutenue par le grand nombre de dragues qui vont vers la surface en comparaison du nombre de tankers qui en partent. 

\subsubsection{Rhéa (Kronos Cluster)} \label{sec:rhea-kronos-cluster} 

Avec un diamètre de 764 kilomètre, Rhéa est la deuxième lune la plus large de Saturne. Composé quasiment entièrement de glace, la surface de Réha est peu habitée, mais une population de 800 000 personne réside dans le Kronos Cluster, un habitat majeur en orbite. Kronos Cluster est massivement microfabriqué de module sphérique violet qui lui donne l'air d'une immense et irrégulière grappe de raisin suspendue dans l'espace, une impression augmenté par le quai spatial sinueux (surnommé la Vigne) s'étendant depuis l'esxtrémité la plus large. A l'intérieur de la masse de module d'habitation, la Vigne bifurque dans toutes les directions, formant un réseau massif d'artères centrales et de couloirs latéraux tortueux. Ils peuvent être traversés en poussant avec les mains ou les orteils sur les murs ou en attrapant les poignées rapides qui se dépalacent le long des "voies rapides" sur les murs des corridors principaux. 

De presque cinq kilomètres de long et de trois de large, Kronos a des problèmes majeurs de surpopulation et d'infrastructure qui l'ont empéché d'atteindre la même taille que Locus. Les concepteurs n'ont simplement pas anticipé la taille que pourrait atteindre l'endroit, et en conséquence 150 000 personnes vivent dans des banlieues de boîte de conserve et de barge racaille dans 'lespace autour de l'habitat. 

Kronos peut être un endroit extrêmement dangereux. Les compagnies d'assurance n'aiment pas travailler là-bas, et l'habitat est un patchwork de voisinnages criminels et anarchistes. Les voisinnages anarchsites sont générallement lourdement armés et sûrs, mais un voyage d'une base anarchiste vers un spatioport est générallement plus sûr avec un groupe d'amis bien armés. Les voisinnages criminels ne sont sûr que si vous êtes dans le gang contrôllant la zone, et même ainsi des conflits démarrent régulièrement. 

La situation est exacerbée par l'Autorité Portuaire de Kronos, une junte d'ultimes qui s'occupent de la sécurité du spatioport. Originellement une hypercorporation Extropien, l'APK est tombé entre les mains des ultimes lorsqu'ils ont décidés qu'ils pouvaient gagner plus en possédant directement la société plutôt qu'en travailalnt en tant que bras à louer. Ils ont violemment expulsé la direction originelle et utilisent maintenant des contractés dans des pods de travail pour entretenir le port. Cette situation est tolére par les chefs du crimes locaux et détestée par les citoyens autonomistes et principalement anarchistes, mais jusqu'à présent, personne n'est capable de ddéfier l'APK, qui protège trés lourdement le port, au détriment de tout autre point d'ammarrage, avec des satellites tueurs et de l'artillerie. 

\subsubsection{Téthys (Gowwinhead)} \label{sec:tethys-gowwinhead} 

Composé presque entièrement de glace, téthys est l'une des plus grande lune de Saturne et le site d'Itahca Chasma, une vallée de 2 000 kilomètres de long recouvre les trois-quart de la cironférence de Téthys. Quinze ans auparavant, les prospecteurs d'un collectif autonomiste d'Indo-Britannique appelée les Rioters se sont posés sur Godwin Head, une saillie dans le mur du chasme appelé ainsi en raison de sa ressemblance à un promontoire jaillissant hors de la mer. Les instruments de leur vaisseau, le Caleb Williams, ont détectés ce qui ressemblait à des dépôts minéraux dans la glace, quelque chose de rare sur Téthys. À l place, ils ont découvert des reliques poussés à la surface par un événements géologiques des éons auparavent, les restes d'une forme de vie primordiale qui s'est éteinte des millions d'énnaes auparavant quand Téthys s'est refroidie et que ses océans sous-terrains ont gelés. 

Goldwindhead est maintenant un abri dense et efficace construit dans les murs du canyon de cinq kilomètrs de haut et abritant 200 000 personnes. Le point centrale de la ville est le Caleb Williams, qui a été remorqué jusque dans une caverne de l'abri et a été converti en un atelier communal et un hœtel de ville. La face murale de la vallée est percée d'alvéoles avec des cavernes de galces excavées habritant des modules d'hébitations, conenctés par des conduits à une grille d'utilitaires communale. Le système d'étayage et de câblage du système utilitaire est également un réseau de transit public, faiclement traversé grâce à la minuscule gravité de Téthys. La mascotte unofficielle de Godwinhead et le Platode Téthyen, un ver translucide de deux-millimètres de long qui représente le pinnacle de l'évolution sur Téthys. Un grand nombre d'habitants sont impliqués dans les biosciences, la xénopaléontologie et la prospection de formes de vies gelées. 

Téthys partage son orbite avec les lunes Troyennes Téléeste et Calpyso, les deux étant petites et peu peuplées. 

\subsection{Titan} \label{sec:titan} 

La plus grande lune de Saturne est enveloppé dans un halo atmosphérique permenant orange, infernalement froide (en moyenne 18°C en dessous de 0), et balayée par des vents produits par des forces de marées quatre fois supérieures à celles qui influençait le climat Terrestre. À sa surface, elle apparaît encore moins hospitalière que les boules de glace et de roches dépourvues d'air qui constituent l'ensemble des mondes entre Titan et Mars. La trés faible lumière solaire aui atteint sa surface est insuffisante pour permettre aux plantes de se développer à l'exceptiond es plus hardies, l'atmosphère composée essentiellement d'azote est dangereusement toxique et la surface est ponctuée de lacs et de emrs de méthane liquide. En dépit de tout ceci, des hydrocarbures abondants, une atmosphère fine et divers composés chimiques font de Titan l'un des rares monde dans le système où les colons peuvent se reposer entièrement sur leurs ressources locales. La population de Titan atteint maintenant les 60 millions de personnes. 

La monnaie sociale et le système de microcorp ont amenés à quelques réussites et échecs spectaculaires. Du bon côté des choses, l'industrie civile de réincarnation de Titan produit plus de morph que Mars et la Lune combinée. Des programmes d'infrastructure massives ont permis de fournir suffisament de place pour que 60 million de personne puissent vivre confortablement sur un monde hostile. Le Grand Collisionneur, le plus grand accélérateur de particule jamais produit, en orbite polaire, permet de réaliser des expérimentation de physique qui ne peuvent être faites nulel aprt ailleurs dans le système. Et il y a deux ans, les Titan ont expédéie la première sonde interstellaire conventionnelle, l'Aubade. Elle atteindra Proxima du Centaure dans à peine 20 ans. 

Du mauvais côté, la loi de Titan "un corps pour chaque esprit" plombe le système de réincarnation civique avec beaucoup de gens que personne ne se serait occuper à réincarner en temps normal. L'échec du projet Scoop, une tentative extrêmement coûteuse de construire un pipeline de la surface de Saturne jusqu'à l'orbite basse, permettant l'extraction massive de gaz sans les opérations de draguage atmosphériques extrêmement coûteuse, a réduit à énant les ambiations de Titan de devenir un producteur d'antimatière majeur. Titan produit de l'antimatière, mais sur une échelle bien plus petite que ce qui était envisagé lorsque le projet Scoop commença. 

Les langues communément paréles sur Titan incluent le Danois, le Français, l'Alelmand, le Mandarin, le Sudéois, le Norvégien et le Suomi. La plupart des citoyens habitents des brouillard, une petite morph dotée d'une fine structure osseuse et possdéant des caractéristiques trés simialires aux rusteurs Martiens. La Parapélagie pour glisser et voler dans la faible gravité Titanienne est une biomodification courante. Les Titaniens effectuent trois ans de service civil obligatoire lorsqu'ils atteignent la majorité avec une emphase sur les forces militaire et de sécurité à l'exception des objecteurs de conscience. Cahque cotoyen ayant effectué son service militaire est membre de la milice et possède une arme d'assaut dans son domicile. 

\subsubsection{Aarhus} \label{sec:aarhus} 

Situé près du pôle sud de Titan sur les rives de Ontario Lacus, une grande mer de métahne liquide peu profonde, Aarhus (cinq million d'habitant) fût le premier site d'habitation humaine sur Titan, choisit pour sa proximité avec des gisement important d'hydrocarbure. La cité est le carrefour physique de l'Université Autonome de Titan (UAT) et héberge de nombreux aures institutions académaiques, la plus notable étant Titan tech, une école d'ingénieur majeure. Contrairement aux université Martienne, qui ont peu de bâtiment sur leur campus, UAT et d'autres écoles Titanienne attirent la plupart de leurs étudiants des habitats trés largement écartés dans le système extérieur, où les délais de communications radio rendent l'apprentissage à distance inefficace. Presque 20\% de la population d'Aarhus sont des étudiants, la plupart d'entre eux venant d'un autre monde. 

L'arrangement d'Aarhus est typique des cités Titanienne. Trois dômes centraux sont entourés par de nombreuses structures plus petite, incluant des dômes plus petit, des centrales à fusion et des dépendances industrielles, la plus massive d'elles étant l'usine d'extraction métahnière maintenant abandonée situé sur la rive du lac. Le sintréieurs du dômes sont assemblés avec des tiges éclairante et sont lourdement cosntruit avec de petit bâtiments étroits, la plupart ayant des terrasses sur lesquels les brouillard avec des ultraléger ailé et à pédales peuvent atterrir. Les structures extérieurs ont habituellement des murs externes construit dans la glace pour la protection et le support structurel avec les murs internes extrudés depuis des silicates locaux. Beaucoup de bêtiment sont azurés ou dans d'autres teintes de bleus pour contraster avec la lueur orange du ciel Titanien. 

Contraiement à beaucoup de cité Titanienne, Aarhus repose princiaplement sur l'énergie à fusion. Aarhus est le centre du mouvement préservationniste natif de Titan, qui s'opposent l'utilisation inefficace des ressources natives d'hydrocarbue en raison de possible changement à long terme sur le climat de Titan. 

\subsubsection{New Québec} \label{sec:new-quebec} 

New Québec se situe dans une plaine de la région Aaru entourée par l'indulation sans fin des dunes formée par les puissants vents de Titan. Les dievrses ressources chimiques de la région alimente la nurserie colossale qui a fait de New Quebce le plus grand producteur de morph du système. 

La cité est à 50 kilomètre de Montmorency lacus, un lac de méthane et d'éthane liquide dans un cratère de 20 kilomètre de large. A l'origine, le lac était supposé être un cratère d'impact, rare sur Titan, des études géologiques ont plsu tard prouvé qu'il s'agissait des restes effondrés d'un cryovolcan maintenant éteint. Situé dans une zone pluvieuse, le lacus se vide lentement par la Cascade Montmorency, une chute de carbone de 200 mètres qui se vide dans une série de petits canaux alluvionnaire desquels la pompe Québecoise pompe son carburant. 

Le Tong St Catehrine, le gang le plus dangereux natif de Titan, est basé à New Québec. La loi Titanienne est gnérallement extrêmement permissive vis à vis des libertés individuelles, les vices que vendent ce gang sont parmi les plus noirs: snuff pods, fork alpha volé et nano-armement. Un stock toujorus disponible de morphs fraîche achetés à des administrateurs de microcoproration corrompus alimente également leur racket. le Tong est extrêmement violent et est un embarassement critique pour les forces de sécurité du Commonwealth. 

\subsubsection{Nyhavn} \label{sec:nyhavn} 

Installé près de l'équateur au milieu des collines de glace roulante de la région Xanadu, Nyhavn (12 millions d'hébitatnts) est la plus grande cité du système extérieur et la capitale du Commonwealth Titanien. Le dôme massif central de Nyhavn, avec ses élégantes tours bleues et des parcs bio-ingénieré, rivalise avec New Shanghaï par la taille et l'ambition. Trois dômes avoisinants et un tissu urbain de structures subsidiaire sont connectés par des couloirs de circulation à fort-dégagement, où les véhicules terrestres et les ultralégers forment un flux continue de traffic à toutes heures. En même temps, la paleur sordide qui prévaut dans les banlieues et souks Martiens est absente; les résidentce et les voisinnages de  la calsse travailleuse Titanienne affiche une débauche de couleur et de conception, renforcé par l'absence de limite imposées par l'économie de pénurie Martienne sur les fabeurs publics. En dépit de son idéalisme, la PLuralité n'est pas immunisé au désir de faire étalage de ses saccomplissement. 

A l'extérieur de la cité il y a un pipeline venant du vaste Tyska Lacus, à 100 kilomètres de distance. Le Skyport du Commonwealth, le principal spatioport de Titan, offre un accès rapide au Pivot du Commonwealth, le système d'appontemant spatial long-courrier de Titan, localisé en orbite géostationnaire au dessus de la cité. La campagne avoisinanate est ponctuée d'abri plus petits connectés à Nyhavn par des trains et un réseau bien développé de route de surface. 

Nyhavn est un centre médiatique majeur, dont la vie quotidienne prétant une oreille attentive aux débats et aux décisions de la Pluralité. C'ets également un lieu cosmopolite, où la main d'œuvre des microcorp de Titan se frottent aux marchant anarchistes de passage et aux (moins fréquentes) délégations du système intérieur. Il y a une économie souterraine active, en dépit des efforts des forces de sécurtité, et le Tong St. Catherine est engagé dans une guerre de faible-intensité mais continue avec les triades à travers le système. 

\subsubsection{Phoebé, Skathi et Abramsen} \label{sec:phoebe-skathi-abrams} 

Après le conflit à Locus, la Pluralité s'est lancée dans un débat bouillonant vis à vis des dangers des incursions des hypercorporations dans le système extérieur. Il est généralement perçu que le Consortium Planétaire espèrait maintenir le système extérieur dans une position simialire à celle qu'avait l'Amérique Latine vis à vis des Êtats-Unis en se mélant de ses affaires à travers tout le dix neuvième et le vingtième siècle, et que le seul moyen pour contrer ce phénomène était une démonstration de force. L'épais brouillard atmosphérique Titanien rend les systèmes de défenses spatiaux terrestres considérablement moins efficace que sur d'autres mondes, mais les satellites et les plateformes spatiales étaient tros vulnéabrles pour servir de centre de commande et de contrôle. 

La solution a été de capturer trois des plus petite lune rétrograde de Saturne - Phoebé, Skathi et Abramsen (autrefois nommé S/2007 S 2, et renommé d'après le nom de l'économiste Titanien avant-gardiste). Phoebé est le plus grand de ces trois objets. Les deux autres ont été amenés aux points L4 et L5 du système. Les calculs nécessaires pour relocaliser ces corps ont été méticuleux, et l'a dépense énergétique fut épouvantable, mais les trois corps servent maintenant de composants clefs dans la grille de défense orbitale de Titan. Que le système crée soit effectivement impénétrable doit encore être testé. 

\subsection{Uranus} \label{sec:uranus} 

Autrefois considérée comem une géante gazeuse comme Saturne et Jupiter, Urnaus et Neptune diffèrent des plus grandes planètes dans le fait qu'elles contiennent de grosse quantités d'eau sous forme de glace, de méthane et d'ammoniaque et q'uelles ont des noyaux rocheux à leurs centre. Cette zone du système est peu peuplée. Uranus orbite à une distance de 10 UA au-delà de l'orbite de Saturne, 20 fois la distance Terre-Soleil. 

Uranus, la planète la plus froide du système solaire, est une sphère bleue-verte de gaz et de glace. Vue de loin, elle est virtuellement dénuée de caractéristique en comparaison de Saturne ou de Jupiter, mais en se rapprochant, des formations nuageuses subtiles ainsi qu'un syst-me d'anneau ténus peuvent être observés. Probablement lié à la collision avec un monde de la taille de la Terre lorsque le système solaire était jeune, Uranus tourne sur le côté, de telle façon que l'un des pôle fait face au soleil pendant 42 ans et que ses lunes orbitent avec un angle marqué vis à vis de l'écliptique solaire. 

À l'époque d'Eclipse Phase, le pôle sud d'Uranus vit sont mi-printemps polaire, pendant lequel d'épais nuages de méthane noircissent l'atmosphère polaire. Cela pourrait être le décallage inhabituel de son axe et le temps saisonnier l'accompagnant qui donnerait du corps aux rumeurs non confirmés que les négocants étarngers appelés les Facteurs pourrait avoir crée un abri caché dans l'atmosphère d'Uranus. 

\subsubsection{Chat Noir et porte de Fissure} \label{sec:chat-noir-fissure} 

Localisé sur Obéron, il s'agît du premier spatioport long-courrier du système d'Uranus, avec une population permanente de 8 000 personnes. Chat Noir possède des installations relativement avancées d'égocast, de réincarnation et de fabrication pour un avant-poste frontalier opéré par plusieurs collectifs anarchistes. La raison pour toutes cette infrastructure est la porte Fissure, la seule Porte de Pandorre dans les mains des anarchistes (en dépit de plusieurs expéditiosn du Consortium Planétaire pour tenter d'en prendre le contrôle). 

La porte de Fissure a été découverte par une expédition de prospection au départ de Chat Noir, alor sun petit avant-poste. Recherchant des gisements de la glace carbonique utile et composant 20\% de la masse d'Obéron, ils ont eu la chance de tomber sur des radio émissions souterraine au pied du Mont Hippolyte. Après avoir triangulé la source, les prospecteurs se sont posés et ont utilisés du matériel d'imagerie sous-terraine. Ils ont eu une image floues d'une fissure rocheuses contenant une masse ambigüe de densité varibale ainsi qu'un objet extrêmement dense, probablement métallique, avec une forme trop régulière pour être autre chose qu'une structure ou un gros artefact - le tout sous 500 mètres de débris cryovolcaniques extrêmement vieux. La porte à Pandorre était déjà publiquement connue à l'époque, les prospecteurs ont donc foré, pensant avoir trouvé un artefact étranger. Ils n'ont aps été trop déçus, même si la découverte mis à jour quelques reste macabres: les corps à peine reconnaissable de onze resquilleurs. 

Pourquoi et comment la porte de Fissure a été érigée sous la glace reste un mystère complet. Jusqu'à récemmenet elle était cependant entièrement enterrée avec juste une fine poche d'espace entre elle et la galce avoisinante. Lorsque les onzes ont émergés, enterrés dans un espace sans air sous 500 mètres de glace, il y avait à peine de la place pour bouger, sans parler de s'échapper, mais la porte ne les laissa pas passer dans l'autre sens. Quelques'uns avaient une pile corticalble récupérable. Cette poignée chanceuse constitue maintenant d'éminent citoyen de Chat Noir, mais aucun n'envisage de retourner à la resquille. 

La porte de Fissure reste dans les mains anarchistes, opérée et défendue par le Collectif Amour et Rage. La porte est rendue disponible à presque tout le monde à moins que leur score de rep soit trop faible ou qu'ils poursuive des intérêts commerciaux (excluant de fait la plupart des hypercorps). Le soutien des resquilleurs est minimal - on traverse le seuil à ses propres risques. Toute découverte faites par cette porte doit cependant être partagée pour le bien collectif de la transhumanité. 

\subsubsection{Titania et Obéron} \label{sec:titania-oberon} 

Les deux plus grande lune de Saturne sont peu peuplée, avec seulement 10 000 transhumains vivant sur chaque corps. La plupart des stations sont des mélanges de dôme et d'abris en nid d'abeille et vont des avant-postes de communication et de recherche hypercorporatistes au cités libre autonomistes. Cette paire de lune est plus complexec chimiquement que la plupart des lunes du système extérieur, composés d'environ 30\% de roches, de 20\% de méthane et de galces carboniques similaires, et de 50\% d'eau. Titania est le foyer d'un canyon spectaculaire qui rivalise avec la Valles Marineris Martienne. Plusieurs abris sur Titania attirent les touristes du système intérieur et des géantes gazeuses, qui viennent pour pratiquer le rocketing, le mekking et d'autres sports dans le canyon. 

\subsubsection{Xiphos} \label{sec:xiphos} 

L'un des deux bastions principal des ultimes, Xiphos est un cylindre hamilton orbitant le dans le système d'anneau Uranien. Bien que la plupart de la technologie derrière les cylindres hamilton soit open source, les systèmes d'armement extrêmement efficaces de la station ne le sont pas. Les bruits courrent autour du fait que les ultimes aient échangées certaines faveurs avec Gorgon Defense Systems lors de la construction de cette station. Là où Aspis, l'habitat intérieur des ultimes, est une zone elativement ouverte, utilisés par les Ultimes pour entrer en contact avec de potentiels clients de mercenariat, Xiphos est interdite d'accès à quiconque n'est pas membre de cette faction. La population supposée d'ultimes ici serait de 10 000 mais les utltimes achètent un grand nombre de contractés infomroph de Mars. Même si il n'existe aucun rapport signalant le retour de ces contractés, les rumeurs disent que les ultimes téléchargent les contractés devant servir dans les zoines sensibles dans des plates sourds, visuellement limitées, sans implants RA et avec des capacités mentale limitées. 

\subsection{Neptune} \label{sec:neptune} 

Frigide, balayée par des vents de 2 100 km/h et teintée de bleu à cause des traces de méthanes dans son atmosphère, Neptune est la dernière planète majeure du système, orbitant à une distance de 30 UA du soleil. À cette distance de l'étoile la plus proche, les plantes ne poussent pas et l'énergie solaire n'est d'aucune utilité. La seule source d'énergie est la fusion, la concentration de lumière stellaire, l'incinération des déchets et les réactiosn chimiques. La présence hypercorporatistes dans le système Neptunien est virtuellement inexistente, car le décallage de communication et les distances de voyage extrême depuis le reste du système solaire signifie que peu d'entreprises Neptuniennes engendrent du profit. De manière similaire, le technosocialisme Titanien n'a jamais pu se développer là. Les rares transhumains qui vivent là sont pleins de ressources, et beaucoup de colons ne sont même plus humains. Les anarchistes, les bordés et les despérados composent la plupart de la population. 

\subsubsection{Glitch} \label{sec:glitch} 

Cet habitat possède la plus haute densité de population du système, avec 20 000 infomorphs vivant dans un groupe maillé de vingt structure sphérique de dix mètre de diamètre, alimentés par un système de réacteurs central efficace. L'habitat est encadré par un nuage d'usine, d'exploitation et de satellites de sdéfense qui occupent un espace considérablement plus important que la station en elle-même. Différentes rumeurs circulent sur le fait que les habitants cherchent des moyens d'améliorer les inofmorphs à la manière des IAs germes ou qu'elles sont engagées dans un effort de simulation d'un grand forcecast. 

\subsubsection{Ilmarinen} \label{sec:ilmarinen} 

Aligné politiquement sur les argonautes, Ilmarinen est un hybride nid d'abeille/grappe construit dans et dépassnt du gros astéroïde L4 Greymere. C'est le plus grand habitat dans les Troyens de neptunes avec une population de 7 000 habitants. omme beaucoup de transhumains à cette distance dans l'espace, la plupart des habitatnts d'Ilmarinen sont lourdement modifié ou habitent des morphs exotiques. Les morphs supportant le froid et le vide prédominent, et de nombreuses sections de l'habiat ne sont pas vivable pour les transhumains de bases. 

\subsubsection{Mahogany} \label{sec:mahogany} 

Les néo-aviens qui ont construit cette station ont abandonnée le manuel de conception d'habitat et ont préféré les configuration toroïdes aux configurations dans la longueurs. Le résultat est un habitat en forme de disque - une assiette d'un demi kilomètre d'épaisseur au bord et d'un kilomètre de diamètre ressemblant à une tranche de cylindre O'Neill sans fenêtres. Une source de lumière axiale, à faible température et alimentée par fusion nourrit les les verdoyants bois de feuillus situs dessous. Les structures sont construites dans les murs du disques jusqu'à atteindre 500 mètres de hauteur. Le disque, composé principalement de fibre de carbone, tourne suffisament rapidement pour générer une gravité de 0,5g sur le sol de l'habitat. Mahogany a une population de 4 000 mercuriens, la pluaprt d'entre-eux étant des néo-aviens. 

\subsubsection{Lunes mineures} \label{sec:minor-moons} 

Les autres douze lunes de Neptune sont essentiellement de petits corps glacé et trés peu peuplés voire déserts. Protéus et Larissa, tous les deux à la fois de taille correcte et suffisament proche de la planète, héberge de petits populations. Naïade et Thalassa sont petites maus trés proche de la planète, et sont donc devenues le foyer de quelques opération de dragage atmosphérique. Néso, orbitant à peu près à 1/3 d'UA de Neptune n'a ajamis été visitée - même par des sondes robotiques. 

\subsubsection{Troyens Neptuniens} \label{sec:neptunian-trojans} 

Suivant et précédent Neptune aux points L4 et L5 de son orbite, se trouvent plusieurs centaines d'astéroïdes essentielmleent glacés et de composition diverse. Les Troyens de Neptunes sont le foyer de bordés, de prospecteurs durs à cuire, d'exhumains exotiques et d'autres survivalistes de l'extrême. 

\subsubsection{Triton} \label{sec:triton} 

La plus grande des lunes de Neptune possède une atmosphère ténue et est chimiquement complexe, composée à part égale de roche et de glaces (azote, eau et dioxyde de carbone gelé). Elle est également géologiquement active, avec des cryovolcans resurfaceant continuellement la planète. La surface possède quelques rares habitants mais plusieurs habitats orbitent autour de Triton, profitantd es matériau bruts abondants de la lune et la faible vitesse d'évasion à leur avantages. 

\subsection{Les limites du système} \label{sec:edge-system} 

Au delà de Neptune ne se trouve que des planètes naines et des glactéroïdes attendant de devenir des comètes, grossièrement divisée en deux régions: la Centure de Kuiper, de 30 à 55 UA du Soleil, et le Disque Épars qui s'étend depuis 55 UA jusqu'au Nuage d'Oort. Pluton, son objet binaire Charon, et Éris ont une composition similaires à Triton. Quelques petits habitats orbitent autoude Pluton et de Charon, se débrouillant pour vivre en prospectant des composés organiques volatiles. Beaucoup d'autres planète naines orbitents dans la ceinture de Kuiper et dans le Disque, incluant Orcus, Senda et 2000 OO67. Parmi eux, seuls Eris abrite une population substantielle. ERIS Localisée à 55 UA du soleil à la bordure du Disqu Épars, Eis est la plsu garnde des planètes naines dans le système et le site d'une lutte acharnée entre deux des factions transhumaniste les plu militante: les ultimes et les exhumains. Le point focal de cette lutte est la Porte de Discorde, la plus porte de Pandorre publiquement connue la plus éloignée du système, localisée dans un labryrinthe de glace un demi kilomètre sous la surface d'Eris. 

L'histoire brève de la porte est sanglante. Les troupes du Go-nin Group ont violemment prit le contrôle de la porte aux anarchistes Ilmarinen qui l'ont découverte. Titan et divers groupes anarchistes et bordés ont essasyés de déloger Go-nin, mais ces tentatives ont échoués, au prix d'un coût élevés en vie et en vaisseaux. Le contrôle de Go-nin sur la porte semblait assuré jusqu'à ce que, apparemment, l'hypercorporation joue un peu trop avec les contrôle de la boîte noire de la porte. Une explosion dévastatrive s'en est suivi, détruisant tout à l'exception de la porte et de la base Go-nin. La porte, cependant, s'est restructurée en plusieurs jours, et sa localisation s'est maintenant déplacée au fond du cratère fondu. 

Durant le court intervalle de temps nécessaire au Go-nin Group pour embaucher un groupe de mercenire ultimes pour reprendre la pote, une force exhumaine inconnue jusqu'ici a prit le contrôle de la zone. Le sultimes ont réussit, mais un groupe d'exhumains s'échappa à travers la porte. Go-nin possède maintenant un contrôle nominale de la Porte de Discorde grâce aux ultimes, qui maintiennent une base lourdement militarisée sur la Lune d'Eris, Dysnomia. Cependant, ces dernières années l'installation de la porte a subit plusieurs attaques pâr des exhumains motivés a infiltrer la porte - et d'après les rumerus, au moins l'une de ces attaques venait de la porte elle-même. 

\subsubsection{Markov} \label{sec:markov} 

La localisation de cet habitation un bastion majeur des argonautes, est un secret extrêmement bien gardé. Les tentatives pour le rechercher n'ont révélés que des leurres ou des cailloux sans vie. Bien qu'une grosse quantité d'information concernant les spécifications de l'habitat, les opérations, les domaines de recherche et les ressources informationnelle soit disponible, seuls les membrs les plus haut placés des argonautes peuvent voyager jusque là. De tout point de vue, l'habitat  est un nide d'abeille aveugle, conçu pour ne virtuellement rien émettre. Des localisations spéculées incluent Hydre, l'une des lunes de Pluton, les profonderus de la Ceinture de Kuiper et même le Nuage d'Oort. 

\subsection{Systèmes extrasolaire} \label{sec:extrasolar-systems} 

Mrme si voyager au-delà des étoile est toujorus hors de portée des accomplissement transhumain, les Portes de Pandorres ont rendus possible le passage vers des myriades d'autres systèmes stellaires. Quelques uns sont notés ici, bien que bien d'autres existent - tous n'ayant pas été explorés. 

\subsubsection{Echo} \label{sec:echo} 

Echo est un système binaire composé d'une étoile en séquence principale et d'un pulsar (d'où le nom du système) distant l'un de l'autre d'à peu près 12 heures lumières. Le système possède un monde Jovien immense et jaune brillant (Echo VI) pesant 1,8 fois la masse de Jupiter et regroupant 101 lunes connues, deuxgénates de glace de types Neptunienne un peu plus loin, une minne ceinture d'astéroïde, puis plusieurs planète interne comparable à Mercure. 





La porte de Pandorre originelle s'ouvrit sur la désertique Echo V, une zone interdite recouverte de détritus d'une civilisation étrangère morte. Les ombres des bâtiments de ces précurseurs font face à ce qui fut autrefois de verdoyante plaine alluviale maintenant habitées uniquement par des arbes mors et de la poussière. A d'autres endroits, des éons d'érosion venteuse ont enlevés le sol quaisment entièrement, ne laissant que dés étendues stériles de sombres scories basaltique. Chimiquement et géologiquement, le monde est trés semblable à Mars, si Mars aurait souffert d'encore un demi milliard d'années sans atmosphère. Les recherches dans les reliques des étrangers depuis longtemps disparu suggère qu'ils étaient morphologiquement similair aux arthropodes ou aux arachnides, leur donnant le nom d'Iktomi, d'après un dieu araignée des Indiens d'Amérique. Jusqu'à présent, peu de choses ont pu être découverte à leur sujet. 

Echo IV, d'un autre côté, est la chose la plus proche d'un apradis que la transhumanité ait pu trouver depuis qu'elle a perdu la Terre. La vie native est basée sur le carbone, et beaucoup de plantes et de poissons sont comestibles même pour les plates. Le climat est tempéré, l'atmosphère respirable sans contaminants majeur. Les latitude au nord et au sud sont le doamine de forêt sans fin dominée par différentes espèces de valdeurs - de gigantesque arbres, ressemblant aux érables avec des feuilles rouge sombre. Dans les régions équatorialles il y a des plaines basaltiques inondées et riches en nutriment idéales pour l'agriculture, divisées par les chaînes montagenuses occasionnelle. Echo IV est toujours géologiquement active en raison de vague de chaluer, bien que plus vieille que la Terre d'à peu près deux milliards d'années et possède deux mégacontinents connectés près de l'équateur par un isthme ténu. Les formes de vie locale notable incluent les Unagi, un prédateur marins des profondeurs effroyable et ressemblant à une anguille, et le lutin clown, une sorte de primate volant qui vit une relation symbiotique avec l'anémone terrestre Écholalian, une plante gigantesque, carnivore et venimeuse qui pousse dans la mangrove des hautes-terre équatoriales. la biosphère est diversifiée et possède plusieurs espèces de mastodfontes, certains étant relativement dangereux. 

\subsubsection{Luca} \label{sec:luca} 

Luca est une naine rouge de classe M située dans une région de la galaxie éloignée de tous les points de repères connus de l'astronomie transhumaine. Le système possède une seule géante gazeuse d'environ 1,4 fois la masse de Jupiter - insuffisante pour protéger les mondes intérieurs d'un bombardement constant d'astéroïdes. La géante gazeuse solitaire est encadrée par une ceinture d'astéroïdes métallique ténue à l'intérieur et par une large ceinture d'astéroïde de silicate et de glace à l'extérieur. Les seules autres corps méritant le status de planète sont les monde intérieur infernaux avec la richesse minérale de Mercure et l'atmosphère insupportable de vénus et quelque mornes plutoïdes et un peu plus grand partageant des orbites de Lagrange avec un champ d'astéroïde composé de la masse éparpillée d'une troisième plutoïde. 

Accessible à la fois via la porte Vulcanoïde et la porte de Fissure, Luca II est une planète terrestre lourdement cratérisé avec une épaisse atmosphère poussièreuse - à peine respirable pour les transhumains équipés du bon matériel. C'est un monde rocheux et froid de collines escarpées, de garrigues, des tourbières géothermiques sifflante et des prairies fongique. Les natifs, qui se sont éteint depuis au moins un million d'année, ont évolués à aprtir d'animauxqui ne sont pas sans rappeller les Oryctéropes Terriens. Originellement des prédateurs d'insectes de type fourmis, les Lucains ont développées une bonne vision des infrarouges (ceomem en témoignent les pigments inhabituel retrouver sur leurs potterie et leur porcelaine plus tardive) et, d'après les analyse de leurs artefacts, possèdaient un sens proche de l'imagerie à ultrason. Leur civilisation alla à travers différents cycle d'apogée et de déclin, ponctués par des cataclysme céleste qui ont tués les espèces les moins adaptables et ont rendus les resources rares. L'organisation sociétale des Lucains semble n'avoir jamais dépassé les niveaux fédoaux avant le Grand Impact. En une centaine d'année après cet impact final, les derniers des Lucains périrent, n'ayant jamais inventé le téléscope, l'ordinateur ou le vol spatial. 

Luca II héberge Banshee, un abri sous-terrain avec quelques caractéristiques dépassant en surface, incluant une station de radioastronomie, des dômes parcs, un port aérospatial de court courrier et des fermes solaire. Il est installé sur la Plaine Hurlante, une plaine venteuse composée de tourbières et de buissons choisie pour ses riches gisements d'hydrocarbure et l'incidence faible des impacts d'astéroïdes. Banshee est un mélange compliqué de colons anarchistes et d'intérêts hypercorporatistes. 

\subsubsection{Mishipizheu} \label{sec:mishipizheu} 

Mishipizheu est une géante rouge. La planète qui a donné son nom à l'étoile, Mishipizheu, est une sphère d'eau de la taille de Mars possédant une atmosphère d'azote et de dioxyde de carbone ainsi qu'un noyeau rocheux. Mishipizheu I était une sphère de glace un peu plus petite que Vénus 700 million d'années auapravant, mais l'expansion de son étoile dans la phase géante rouge a fait fondre la planète. Initialement relativement chaude et pleien de poches de glace et de limons charboneux, la planète fondue était un creuset où la vie pouvait se développer et héberge maintenant un édcosystème complexe. les récifs bouillonants d'Amoéboïde composée de créatures ayant des poches de gaz et de leur symbiotes dansent sur l'eau à la surface ou maintiennent une flottabilité neutre dans les profondeurs, devenant des supports pour des écosystème complexe composés principalement de vie animale. 

Mishipizheu I est orbitée par sa lune rocheuse de tailel moyenne, Nanabozho, atteignable via la porte de Discorde. Nanabozho est un mystère, les lunes de sa composition n'atant normalement pas trouvée si loin dans un système. la meilleure théori e actuelle est que Nanabozho était un objet du système intérieur avec une orbite erratique. Il a été perturbé dans son orbite par l'une des géantes gazeuse maintenant englobée par l'étoile qui devait avoir existé, et à été capturé par chance dans l'orbite de Mishipizheu.. L'extraordinairement faible chance d'un tel évènement a cependant lancé de trés forte spéculations quand à l'origine réelle de le lune, qui est une destination populaire pour les resquilleurs, ainsi que la planète en-dessous. 

\subsubsection{Synergie} \label{sec:synergy} 

Parmi les premières tentatives d'établir une colonnie de resquilleurs au delà de la Porte de Pandorre originelle, à peine 5 ans après la Chute, il y avait un groupe de deux cent cinquantes colons équipés de technologie de communication crânienne expérimentale. peu de temps après le saut un incident encore non identifié a cependant forcé al porte à se fermer et le mécanisme n'a pu être réinitialisé aux mêmes réglage et coordonnées que cinq ans après. Lorsque les techniciens de la porte réussirent récemment à finallement récupérer les réglages et à réouvrir la porte, les colons ont étés trouvés survivant mais quelque peu changés. La technologie envoyés avec exu était essentiellement dirigé par des IA, permettant la création d'un hypermesh connectant les pensées, les émotions et les expériences sensorielles de chaque colons avec tous les autres. Après plus d'une décennie de mesure de survie difficile, cette technologie et le stress de la situation a connecté les colons et leurs IAs en un esprit collectif. En dépit de la possibilité de retourner sur le système solaire, ces Synergiste, comem ils se définissent eux-mêmes, n'ont aucune envie de se séparer de leur conscience partagée. 

\subsubsection{Autres Exoplanètes} \label{sec:other-exoplanets} 

Le nombre de système stellaire extraterestre que la transhumanité a visité via les portes se comptent maintenant en centaine, si ce n'est plus - bien que seul un petit pourcentage de ces systèmes sont intéressants/hospitalier. A peine quelques douzianes ont étés occupés substantiellemeny ou colonisé par les transhumains, bien que ce nnombre augmente rapidement. parmi ceux-là, quelques-uns émritent d'être mentionnés: 

\paragraph{Arcadia:} \label{sec:arcadia} Accessible via la porte Martienne, le Consortium Planétaire y construit un aérostat dans l'atmosphère haute de cette planète semblable à Vénus qui servira de destination de vacance privée pour l'hyperélite. 

\paragraph{Babylone:} \label{sec:babylon} Initiallement considérée comme étant juste une lune roussie et sans intérêt orbitant autour d'une planète trés proche d'une étoile jaune, des chercheurs analysant l'atoile ont fait une incroyable découverte accidentelle: un vaisseau abandonnée orbitant profondément dans la couronne stellaire. Les tentatives d'accéder à ce vaisseau ont jusqu'à présent échouées, mais d'autres projets sont en cours, incluant la possibilite de remorquer le vaisseau vers des climats plus sûrs. 

\paragraph{Bluewood:} \label{sec:bluewood} l'une des première colonnie anarchiste établient par la porte de Fissure, cet abri se situe sur une belle planète, semblable à la Terre et doté d'un écosystème prospère. Installée à la lisière d'une forêt "d'arbres" bleus inquiétants et étrangers, la colonnie a été prise au dépourvu par le taux de croissance alarmant des arbres. Les bâtiments de l'habitat modulaire sont maintenant entourés et encastrés par la surcroissance en dépit d'effort modestes pour les garder libre. Toujours intact mais englobés par les branchages spiralant, l'effet est magnifique et obsédant. 

\paragraph{Nótt:} \label{sec:nott} Cette lune désolée et recouverte de glace souffre d'une activité géothermale importante qui provoque la casse et le gel instantané de sa croüte gelée. Le personnel de rechercheurs infortuné, composé de contracté, prétend qu'il y a quelquechose dehors dans la glace qui les surveille - plus d'une douzaine ont disparu ces dernières années. Pathfinder refuse cependant de rappeler la station et des recherches minutieuses de ses éqauipes de sécurité n'ont rien révélé. 

\paragraph{Sky Ark:} \label{sec:sky-ark} TerraGenesis a redessiné ce satellite sec et aride pour en faire une réserve hors-monde pour protégéer la vie animale, incluant de nombreuses espèces Terrienne anciennement éteinte et réssuscités depuis de l'ADN fossile. 

\paragraph{Wormwood:} \label{sec:wormwood} Cette tannière labyrinthique semblent ête un habitat nid d'abeille, bien que le mystère auqna à l'identité des creuseurs et de leur motivation reste un mystère. l'ancien astéroïdes fait parti d'un système d'anneau d'une géante gazuese inconnue. Clairement artificiel, les resquilleurs n'ont cependant toujours aps trouvé de signe d'une qulconque technologie ou forme de vie. 

\begin{quotation} \textbf{Analyse: Arbres brume} 

[Corruption du Fichier: 98\%] 

[Récupération Partielle] 

\begin{verbatim} ... called “myst trees” by the residen@# of Ca*\&78 ... also found on tw) oth*r exoplanets ]]]]] ... seem to be some sort of living data storage{{[— ... utilizing nanofog systems for <|{9h'''' ... high pr@bability of alien origin [[[[[[; ; \end{verbatim} \end{quotation} 




% \chapter{Game mechanics} \label{chap:game-mechanics} 

In every game, there comes a time when the gamemaster must decide if a character succeeds or fails in an action. This is when the players roll dice and the characters' stats and abilities come into play. This chapter defines the core mechanics and rules that govern the outcomes of events in Eclipse Phase. 

\section{A note on terminology and Gender} The Eclipse Phase setting raises a number of interesting questions about gender and personal identity. What does it mean when you are born female but you are occupying a male body? When it comes to language and editing, this also poses a number of interesting questions for what pronouns to use. The English language has a bit of a bias towards male-gendered pronouns that we hope to avoid in these rules. For purposes of this game, we’ve sidestepped some of these gender neutrality quandaries by adopting the “Singular They” rule. What this means is that rather than just going with male pronouns (“he”) or switching between gendered pronouns (“he” in one chapter, “she” in the next), we have adopted the use of “they” even when referring to a single person. To some folks, this is bad grammar, but there is actually some good evidence that this usage has strong historical roots (look it up), and it certainly gives our editors fewer headaches. When referring to specifi c characters, we use the gendered pronoun appropriate to the character’s personal gender identity, no matter the sex of the morph they are in. 



\section{Core rules} \label{sec:basics} 

\subsection{The ultimate rule} \label{sec:ultimate-rule} 

One rule in Eclipse Phase outweighs all of the others: have fun. This means that you should never let the rules get in the way of the game. If you don't like a rule, change it. If you can't find a rule, make one up. If you disagree over a rule's interpretation, flip a coin. Try not to let rules interfere with the game's flow and mood. If you're in the middle of a really good scene or intense roleplaying and a rule suddenly comes into question, don't stop the game to look it up and argue about it. Just wing it, make a decision quickly, and move on. You can always look the rule up later so you'll remember it next time. If there are disagreements over a rule's interpretation, remember that the gamemaster gets the final say. 

This rule also means that you shouldn't let the story be guided solely by rolls of the dice. The element of chance that dice rolls provide lends a sense of randomness, uncertainty, and surprise to the game. Sometimes this is exciting, like when a character makes an unexpectedly difficult roll and saves the day. At other times, it is brutal, such as when a lucky shot from an opponent takes one of the characters out for good in a fight. If the gamemaster wants a scenario to result in a pre-planned dramatic outcome and an unexpected die roll threatens that plan, they should feel free to ignore that roll and move the story in the direction they desire. 

\subsection{Dés} \label{sec:dice-1} 

Eclipse Phase uses two ten-sided dice (each noted as a d10) for random rolls. In most cases, the rules will call for a percentile roll, noted as d100, meaning that you roll two ten-sided dice, choosing one to count first, and then read them as a result between 0 and 99 (with a roll of 00 counting as zero, not 100). The first die counts as the tens digit, and the second die counts as the ones digit. For example, you roll two ten-sided dice, one red and one black, calling out red first. The red one rolls a 1 and the black die rolls a 6, for a result of 16. Some sets of d10s, as shown above, are specifically marked for easier rolling and reading. 

Occasionally the rules will call for individual die rolls, with each individual ten-sided die listed as a d10. If the rules call for several dice to be rolled, it will be noted as 2d10, 3d10, and so on. When multiple ten-sided dice are rolled in these instances, the results are added together. For example, a 3d10 roll of 4, 6, and 7 counts as 17. On d10 rolls, a result of 0 is treated as a 10, not a zero. 

Most players of Eclipse Phase can get by with having two ten-sided dice, but it doesn't hurt to have more on hand. These dice can be purchased at your friendly local game store or borrowed from another gamer. 

\subsection{Making tests} \label{sec:making-tests} 

In Eclipse Phase, your character is bound to find themself in adrenalin-pumping action scenes, high- stress social situations, lethal combats, spine-tingling investigations, and similar situations filled with drama, risk, and adventure. When your character is embroiled in these scenarios, you determine how well they do by making tests—rolling dice to determine if they succeed or fail, and to what degree. 

You make tests in Eclipse Phase by rolling d100 and comparing the result to a target number. The target number is typically determined by one of your character's skills (discussed below) and ranges between 1 and 98. If you roll less than or equal to the target number, you succeed. If you roll higher than the target number, you fail. 

A roll of 00 is always considered a success. A roll of 99 is always a failure. 

\end{quotation} Jaqui's character needs to make a skill test. Her skill is 55. Jaqui takes two ten-sided dice and rolls a 53—she succeeds! If she had rolled a 55, she still would have been successful, but any roll higher than that would have been a failure. \end{quotation} 

\subsection{Target numbers} \label{sec:target-numbers} 

As noted above, the target number for a d100 roll in Eclipse Phase is usually the skill rating. Occasionally, however, a different figure will be used. In some cases, an aptitude score is used, which makes for much harder tests as aptitude scores are usually well below 50 (see Aptitudes, \ref{sec:aptitudes}). In other tests, the target number will be an aptitude rating x 2 or x 3 or two aptitudes added together. In these cases, the test description will note what rating(s) to use. 

\subsection{When to make tests} \label{sec:when-make-tests} 

The gamemaster decides when a character must make a test. As a rule of thumb, tests are called for whenever there is a chance that a character might fail at an action or when success or failure may have an effect on the ongoing story. Tests are also called for whenever two or more characters act in opposition to one another (for example, if they are arm wrestling or haggling over a price). On the other hand, routine use of a skill by someone with at least a rating of 30 in that skill can be assumed to be successful with no test. 

It is not necessary to make tests for everyday, run-of-the-mill activities, such as getting dressed or checking your email (especially in Eclipse Phase, where so many activities are automatically handled by the machines around you). Even an activity such as driving a car does not call for dice rolls as long as you have a small modicum of skill. A test might be necessary, however, if you happen to be driving while bleeding to death or are pursuing a gang of motorcycle-riding scavengers through the ruins of a devastated city. 

Knowing when to call for tests and when to let the roleplaying flow without interruption is a skill every gamemaster must acquire. Sometimes it is better to simply make a call without rolling dice in order to maintain the pacing of the game. Likewise, in certain circumstances the gamemaster may decide to make tests for a character in secret, without the player noticing. If an enemy is trying to sneak past a character on guard, for example, the gamemaster will alert the player that something is amiss if they ask them to make a perception test. This means that the gamemaster should keep a copy of each character's record sheet on hand at all times. 

\subsection{Difficulty and modifiers} \label{sec:difficulty-modifiers} 

The measure of a test's difficulty is reflected in its modifiers. Modifiers are adjustments made to the target number (not the roll), either raising or lowering it. A test of average difficulty will have no modifiers, whereas actions that are easier will have positive modifiers (raising the target number, making success more likely) and harder actions will have negative modifiers (lowering the target number, making success less likely). It is the gamemaster's job to determine if a particular test is harder or easier than normal and to what degree (as illustrated on the Test Difficulty table) and to then apply the appropriate modifier. 

Other factors might also play a role in a test, applying additional modifiers aside from the test's general level of difficulty. These factors include the environment, equipment (or lack thereof), and the health of the character, among other things. The character might be using superior tools, working in poor conditions, or even wounded, and each of these factors must be taken into account, applying additional modifiers to the target number and adjusting the likelihood of success or failure. 

For simplicity, modifiers are applied in multiples of 10 and come in three levels of severity: Minor (+/–10), Moderate (+/–20), and Major (+/–30). Any number of modifiers may be applied, as the gamemaster deems appropriate, but the cumulative modifiers may not exceed + or – 60. 

\end{quotation} Jaqui is attempting to leap from one door to another across a large chamber in zero gravity. She's in a hurry. If she misses the door, she'll lose valuable time, so the gamemaster calls for a Freefall Skill Test. Jaqui's Freefall skill is 46. Unfortunately the chamber is filled with floating debris that could get in her way. The gamemaster determines this is a Moderate modifier, reducing the target number by 20. Jaqui must roll a 26 or less to succeed. \end{quotation} 

\subsection{Criticals: rolling doubles} \label{sec:crit-roll-doubl} 

Any time both dice come up with the same number -- 00, 11, 22, 33, 44, etc.—you have scored a critical success or critical failure, depending on whether your roll also beat your target number. 00 is always a critical success, whereas 99 is always a critical failure. Rolling doubles means that a little something extra happened with the outcome of the test, either positive or negative. Criticals have a very specific application in combat tests (see \ref{sec:combat}), but for all other purposes the gamemaster decides what exactly went wrong or right in a specific situation. Criticals can be used to amplify a success or failure: you finish with a flourish or fail so spectacularly that you remain the butt of jokes for weeks to come. They can also result in some sort of unexpected secondary effect: you repair the device and improve its performance; or you fail to shoot your enemy and hit an innocent bystander. Alternately, a critical can be used to give a boost (or a hindrance) to a follow-up action. For example, you not only spot a clue, but you immediately suspect it to be red herring; or you not only fail to strike the target, but your weapon breaks, leaving you defenseless. Gamemasters are encouraged to be inventive with their use of criticals and choose results that create comedy, drama, or tension. 

\end{quotation} Audrey is attempting to intimidate a low-level triad mook into giving her information. Unfortu- nately she rolls a 99—a critical failure. Not only does she fail to scare the guy, but she accidentally lets slip an important piece of information that she didn't want the triad to know. If she rolled a 00 instead—a critical success—she would browbeat the man so thoroughly that he throws in some extra important information just so she'll leave him alone in the future. \end{quotation} 

\subsection{Defaulting: untrained skill use} \label{sec:defa-untr-skill} 

Certain tests may call for a character to use a skill they don't have—a process called defaulting. In this case, the character instead uses the rating of the aptitude (see p. 123) that is linked to the skill in question as the target number. 

Not all skills may be defaulted; some of them are so complex or require such training than an unskilled character can't hope to succeed. Skills that may not be defaulted on are noted on the Skill List (p. 176) and in the skill description. 

In rare cases, a gamemaster might allow a character to default to another skill that also relates to a test (see p. 173). When allowed, defaulting to another skill incurs a –30 modifier. 

\end{quotation} Toljek is trying to casually sneak inside a hypercorp facility when he unexpectedly runs into a hypercorp employee. The woman he's encountered doesn't necessarily have grounds to be suspicious of Toljek's presence, but the gamemaster calls for Toljek to make a Protocol Test to pass himself off as someone that belongs there. Unfortunately, Toljek doesn't have that skill, so he must default to its linked aptitude, Savvy, instead. His Savvy score is only 18, so Toljek better hope he gets lucky. \end{quotation} 

\subsection{Simplifying modifiers} \label{sec:simpl-modif} 

Rather than looking up and accumulating a long list of modifiers for each action and doing the math, the gamemaster can instead choose to simply “eyeball” the situation and apply the modifier that best sums up the net effect. This method is quicker and allows for easier test resolution. One way to eyeball the situation is to simply apply the most severe modifier affecting the situation. 

\end{quotation} Tyska is trying to escape from some thing that's chasing him through a derelict habitat. The gamemaster calls for a Freerunning Test, but there are a number of modifying conditions: it's dark, he's running with a flashlight, and there's debris everywhere. Tyska, however, has an entoptic map of the best route out of there to help him out. The gamemaster assesses the situation and decides the overall effect is that the test is challenging, and so a –20 modifier is applied. \end{quotation} 

\subsubsection{narrative modifiers} \label{sec:narrative-modifiers} 

If you wish to develop a more cinematic feel for your game, or if you simply wish to encourage your players to invest more detail and creativity into the storyline, you can award “narrative modifiers” to a character's test when that player describes what the character is doing in exceptionally colorful, inventive, or dramatic detail. The better the detail, the better the modifier. 

\end{quotation} Cole doesn't just want his character to jump over the table, he wants to make an impact. Cole tells the gamemaster that his character kicks a chair out of the way, rolls over the dinner table on his shoulder, grabs a fork as he does it, makes sure to knock all of the fine china on the floor, then lands on his feet in a defensive martial arts posture, fork raised high. The gamemaster decides the extra description is worth +10 to his Freerunning Test. \end{quotation} 

\subsection{Teamwork} \label{sec:teamwork} 

If two or more characters join forces to tackle a test together, one of the characters must be chosen as the primary actor. This leading character will usually (but not always) be the one with the highest applicable skill. The primary acting character is the one who rolls the test, though they receive a +10 modifier for each additional character helping them out, up to a maximum +30 modifier. Note that helping characters do not necessarily need to know the skill being used if the gamemaster decides that they can follow the primary actor's lead. 

\end{quotation} The robotic leg on Eva's synthetic morph has been badly damaged, so she needs to repair it. Max and Vic both sit down and help her out, giving her a +20 modifier (+10 for each helper) to her Hardware: Robotics Test. \end{quotation} 

\section{Types of tests} \label{sec:types-tests} 

There are two types of tests in Eclipse Phase: Success and Opposed. 

\subsection{Success tests} \label{sec:success-tests} 

Success Tests are called for whenever a character is acting without direct opposition. They are the standard tests used to determine how well a character exercises a particular skill or ability. 

Success Tests are handled exactly as described under Making Tests, p. 115. The player rolls d100 against a target number equal to the character's skill +/– modifiers. If they roll equal to or less than the target number, the test succeeds, and the action is completed as desired. If they roll higher than the target number, the test fails. 

\subsubsection{Trying again} \label{sec:trying-again} 

If you fail at a test, you can take another shot. Each subsequent attempt at an action after a failure, however, incurs a cumulative –10 modifier. That means the second try suffers –10, the third –20, the fourth –30, and so on, up to the maximum –60. 

\subsubsection{Taking the time} \label{sec:taking-time} 

Most skill tests are made for Automatic, Quick, or Complex Actions (see pp. 119–120) and so are resolved within one Action Turn (3 seconds, see p. 119). Tests made for Task actions (p. 120) take longer. 

Players may choose to take extra time when their character undertakes an action, meaning that they choose to be especially careful when performing the action in order to enhance their chance of success. For every minute of extra time they take, they increase their target number by +10. Once they've modified their target number to over 99, they are practically assured of success, so the gamemaster can waive the dice roll and grant them an automatic success. Note that the maximum +60 modifier rule still applies, so if their skill is under 40 to start with, taking the time may still not guarantee a favorable outcome. You may take the time even when defaulting (see Defaulting, p. 116). 

Taking extra time is a solid choice when time is not a factor to the character, as it eliminates the chance that a critical failure will be rolled and allows the player to skip needless dice rolling. For certain tests it may not be appropriate, however, if the gamemaster decides that no amount of extra time will increase the likelihood of success. In that case, the gamemaster simply rules that taking the time has no effect. 

For Task action tests (p. 120), which already take time to complete, the duration of the task must be increased by 50 percent for each +10 modifier gained for taking extra time. 

\end{quotation} Srit is searching through an abandoned space- ship, looking for a sign of what happened to the missing crew. The gamemaster tells her it will take twenty minutes to search the whole ship. She wants to be extra thorough, however, so she takes an extra thirty minutes. Fifty percent of the original timeframe is ten minutes, so taking an extra thirty minutes means that Srit receives a +30 modifier to her Investigation Test. \end{quotation} 

\subsubsection{Simple success tests} \label{sec:simple-success-tests} 

In some circumstances, the gamemaster may not be concerned that a character might fail a test, but instead simply wants to gauge how well the character performs. In this case, the gamemaster calls for a Simple Success Test, which is handled just like a standard Success Test (p. 117). Rather than determining success or failure, however, the test is assumed to succeed. The roll determines whether the character succeeds strongly (rolls equal to or less than the target number) or succeeds weakly (rolls above the target number). 

\end{quotation} Dav is taking a short spacewalk between two parked ships. The gamemaster determines that this is a routine operation and calls for Dav to make a Simple Success Test using the Freefall skill. Dav's skill is only 35. He rolls a 76, but the gamemaster merely determines that Dav has some trouble orienting himself and has to take some extra time. If Dav had rolled a 77—a critical failure—his suit's maneuvering jets may have died and he may have accidentally propelled himself into deep space. \end{quotation} 

\subsubsection{Margin of success/failure} \label{sec:marg-succ} 

Sometimes it may be important that a character not only succeeds, but that they kick ass and take names while doing it. This is usually true of situations where the challenge is not only difficult but the action must be pulled off with finesse. Tests of this sort may call for a certain Margin of Success (MoS)—an amount by which the character must roll under the target number. For example, a character facing a target number of 55 and a MoS of 20 must roll equal to or less than a 35 to succeed at the level the situation calls for. 

\end{quotation} An enemy has thrown an incendiary device near Stoya. She has only a moment to act and decides to try to kick it away from herself. Even better, she hopes to kick it into the open maintenance hatch a dozen meters away. The gamemaster determines that in order to kick it into the hatch, Stoya needs to succeed with an MoS of 30. Her Unarmed Combat skill is 66, so Stoya needs to roll 66 or less to kick the device away (though she may still be damaged when it explodes), and 36 or less to kick it into the hatch (in which case she will be completely safe when it detonates). She rolls a 44—missing the hatch, but scoring a critical success! Her aim is off, but the gamemaster decides that the device rebounds off some machinery and falls into the hatch anyway. \end{quotation} 

At other times, it may be important to know how badly a character fails, as determined by a Margin of Failure (MoF), which is the amount by which the character rolled over the target number. In some cases, a test may note that a character who fails with a certain MoF may suffer additional consequences for failing so dismally. 

\end{quotation} Nico is trying to sketch out a picture of someone's face. He has eidetic memory, but his drawing needs to be good enough for someone else to identify the person. He rolls against his Art: Drawing skill of 34, scoring a 97—a MoF of 63. The illustration is so bad that the gamemaster determines that anyone using that picture to identify the person will need to score a MoS of at least 63 on a Perception Test to recognize the person. \end{quotation} 

\subsubsection{Excellent successes/severe failures} \label{sec:excell-succ-fail} 

Excellent Successes and Severe Failures are a method used to benchmark successes and failures with an MoS or MoF of 30+. Excellent Successes are used in situations where an especially good roll may boost the intended effect, such as inflicting more damage with a good hit in combat. Severe Failures denote a roll that is particularly bad and has a worse effect than a simple failure. Neither Excellent Successes or Severe Failures are as good or bad as criticals, however. 

\end{quotation} Stoya has been caught in a deal gone bad. She moves to kick her opponent using her Unarmed Combat of 65. She rolls a 33 (for an MoS of 32), and her opponent rolls a 21 (also successful, but less than Stoya, so she wins). She has succeeded and beaten her opponent with an MoS of 30+, scoring an Excellent Success, meaning she will inflict extra damage with the kick. \end{quotation} 

\subsection{Opposed tests} \label{sec:opposed-tests} 

An Opposed Test is called for whenever a character's action may be directly opposed by another. Regardless of who initiates the action, both characters must make a test against each other, with the outcome favoring the winner. 

To make an Opposed Test, each character rolls d100 against a target number equal to the relevant skill(s) along with any appropriate modifiers. If only one of the characters succeeds (rolls equal to or less than their target number), that character has won. If both succeed, the character who gets the highest dice roll wins. If both characters fail, or they both succeed and roll the same number, then a deadlock occurs—the characters remain pitted against each other, neither gaining ground, until one of them takes another action and either breaks away or makes another Opposed Test. 

Note that critical successes trump high rolls in an Opposed Test—if both characters succeed and one rolls 54 while the other rolls 44, the critical roll of 44 wins. 

Care must be taken when applying modifiers in an Opposed Test. Some modifiers will affect both participants equally, and should be applied to both tests. If a modifier arises from one character's advantage in relation to the other, however, that modifier should only be applied to benefit the favored character; it should not also be applied as a negative modifier to the disadvantaged character. 

\begin{quotation} Zhou has been hired by the Jovian Republic to infiltrate his old pirate band. Even though he's resleeved in a new skin, he's worried that one of his old buddies, Wen, might recognize his mannerisms, since they lived, whored, and raided together for years. After Zhou has spent some time in Wen's company, the gamemaster makes a secret Opposed Test, pitting Zhou's Impersonation skill of 57 against Wen's Kinesics of 34. The gamemaster decides to give Wen a bonus +20, since he is so familiar with his former buddy and has been on the lookout for him, eager to repay the old grudge that split them apart. Wen's target number is now 54. 

The gamemaster rolls for both. Zhou scores a 45 and Wen a 39. Both succeed, but Zhou rolled higher, so his deception is successful. The gamemaster decides that Wen finds something about Zhou to be familiar, but he can't put his finger on it. \end{quotation} 

\subsubsection{Opposed tests and margin of success/failure} \label{sec:opposed-tests-margin} 

In some cases, it may also be important to note a character's Margin of Success or Failure in an Opposed Test, as with a Success Test above. In this case, the MoS/MoF is still determined by the difference between the character's roll and their target number—it is not calculated by the difference between the character's roll and the opposing character's roll. 

\subsubsection{Variable opposed test} \label{sec:vari-oppos-test} 

In some cases, the rules will call for a Variable Opposed Test, which allows for slightly more outcomes than a standard Opposed Test. If both characters succeed in a Variable Opposed Test, then an outcome is obtained which is different from just one character winning over the other. The exact outcomes are noted with each specific Variable Opposed Test. 

\end{quotation} Jaqui needs to hack into a local network to retrieve some video footage. The network is ac- tively defended by an AI, so a Variable Opposed Test is called for, pitting Jaqui's Infosec skill of 48 against the AI's Infosec of 25. Jaqui rolls a 48—a success—but the AI rolls a 14—also a success. In this circumstance, Jaqui succeeds in hacking in, but the AI is aware of the infiltration and can take active countermeasures against her. \end{quotation} 

\section{Time and actions} \label{sec:time-actions} 

Though the gamemaster is responsible for managing the speed at which events unfold, there are times when it is important to know exactly who is acting when, especially if some people are acting before or after other people. In these circumstances, gameplay in Eclipse Phase is broken down into intervals called Action Turns. 

\subsection{Action turns} \label{sec:action-turns} 

Each Action Turn is three seconds long, meaning there are twenty Action Turns per minute. The order in which characters act during a turn is determined by an Initiative Test (see Initiative, p. 121). Action Turns are further subdivided into Action Phases. Each character's Speed stat (p. 121) determines the amount of actions they can take in a turn, represented by how many Action Phases they may take. 

\subsection{Types of actions} \label{sec:types-actions} 

The types of actions a character may take in an Action Turn are broken down to: Automatic, Quick, Complex, and Task actions. 

\subsubsection{automatic actions} \label{sec:automatic-actions} 

Automatic actions are “always on” and require no effort from the character, assuming they are conscious. 

Examples: basic perception, certain psi sleights 

\subsubsection{Quick actions} \label{sec:quick-actions} 

Quick actions are simple, so they can be done fast and can be multi-tasked. The gamemaster determines how many Quick actions a character may take in a turn. 

Examples: talking, switching a safety, activating an implant, standing up 

\subsubsection{Complex actions} \label{sec:complex-actions} 

Complex actions require concentration or effort. The number of Complex actions a character may take per turn is determined by their Speed stat (see p. 121). Examples: attacking, shooting, acrobatics, disarming a bomb, detailed examination 

\subsubsection{Task actions} \label{sec:task-actions} 

Task actions are any actions that require longer than one Action Turn to complete. Each Task action has a timeframe, usually listed in the task description or otherwise determined by the gamemaster. The time-frame determines how long the task takes to complete, though this may be reduced by 10 percent for every 10 full points of MoS the character scores on the test (see Margin of Success/Failure, p. 118). If a character fails on a Task action test roll, they work on the task for a minimum period equal to 10 percent of the timeframe for each 10 full points of MoF before realizing it's a failure. For Task actions with timeframes of one day or longer, it is assumed that the character only works eight hours per day. A character that works more hours per day may reduce the time accordingly. Characters working on Task actions may also interrupt their work to do something else and then pick up where they left off, unless the gamemaster rules that the action requires continuous and uninterrupted attention. Similar to taking the time (p. 117), a character may rush the job on a Task action, taking a penalty on the test in order to decrease the timeframe. The character must declare they are rushing the job before they roll the test. For every 10 percent they wish to reduce the timeframe, they incur a –10 modifier on the test (to a maximum reduction of 60 percent with a maximum modifier of –60). 

\section{Defining your character} \label{sec:defin-your-char} 

In order to gauge and quantify what your character is merely good at and what they excel in—or what they are clueless about and suck at—Eclipse Phase uses a number of measurement factors: stats, skills, traits, and morphs. Each of these characteristics is recorded and tracked on your character's record sheet (p. 399). 

\subsection{concept} \label{sec:concept} 

Your character concept defines who you are in the Eclipse Phase universe. You're not just a run-of-the-mill plebeian with a boring and mundane life, you're a participant in a post- apocalyptic transhuman future who gets caught up in intrigue, terrible danger, unspeakable horrors, and scrambling for survival. Much like a character in an adventure, drama, or horror story, you are a person to whom interesting things happen—or if not, you make them happen. This means your character needs a distinct personality and sense of identity. At the very least, you should be able to sum up your character concept in a single sentence, such as “cantankerous neotenic renegade archaeologist with anger management issues” or “thrill-seeking social animal who is dangerously obsessed with conspiracy theories and mysteries.” If it helps, you can always borrow ideas from characters you've seen in movies or books, modifying them to fit your tastes. Your character's concept will likely be influenced by two important factors: background and faction. Your background denotes the circumstances under which your character was raised, while your faction indicates the post-Fall grouping to which you most recently held ties and allegiances. Both of these play a role in character creation (p. 128). 

\subsection{motivations} \label{sec:motivations} 

The clash of ideologies and memes is a core component of Eclipse Phase, and so every character has three motivations—personal memes that dominate the character's interests and pursuits. These memes may be as abstract as ideologies the character adheres to or supports—for example, social anarchism, Islamic jihad, or bioconservatism -— or they may be as concrete as specific outcomes the character desires, such as revealing a certain hypercorp's corruption, obtaining massive personal wealth, or winning victories for uplifted rights. A motivation may also be framed in opposition to something; for example, anti-capitalism or anti-pod-citizenship, or staying out of jail. In essence, these are ideas that motivate the character to do the things they do. Motivation is best noted as a term or short phrase on the character sheet, marked with a + (in favor of) or – (opposed to). Players are encouraged to develop their own distinct motivations for their characters, in cooperation with the gamemaster. Some examples are provided on p. 138. In game terms, motivation is used to help define the character's personality and influence their actions for roleplaying purposes. It also serves to regain Moxie points (p. 122) and earn Rez Points for character advancement (p. 152). 

Motivational goals may be short-term or long-term, and may in fact change for a character over time. Short-term goals are more immediately obtainable objectives or short-lived interests, and these goals are likely to change once achieved. Even so, they should reflect intentions that will take more than one game session to reach, possibly covering weeks or months. These short-term goals may in fact tie directly into the gamemaster's current storyline. Examples include conducting a full analysis of an alien artifact, completing a research project, or living life as an uplifted dog for a while. Long-term goals reflect deeply rooted beliefs or tasks that require major efforts and time (possibly lifelong) to achieve. For example, finding the lost backup of a sibling missing since the Fall, overthrowing an autocratic regime, or making first contact with a new alien species. For purposes of awarding Moxie or Rez Points, long-term goals are best broken down into obtainable chunks. Someone whose goal is to track down the murderer who killed their parents when they were a child, for example, can be considered to achieve that goal every time they discover some evidence that brings them a little closer to solving the puzzle. 

\subsection{Ego vs. morph} \label{sec:ego-vs.-morph-1} 

Eclipse Phase's setting dictates that a distinction must be made between a character's ego (their ingrained self, their personality, and inherent traits that perpetuate in continuity) and their morph (their ephemeral physical—and sometimes virtual—form). A character's morph may die while the character's ego lives on (assuming appropriate backup measures have been taken), transplanted into a new morph. Morphs are expendable, but your character's ego represents the ongoing, continuous life path of your character's mind, personality, memories, knowledge, and so forth. This continuity may be interrupted by an unexpected death (depending on how recent the backup was made), or by forking (see p. 273), but it represents the totality of the character's mental state and experiences. 

Some aspects of your character—particularly skills, along with some stats and traits—belong to your character's ego, which means they stay with them throughout the character's development. D'autres statistiques et traits sont cependant déterminés par une morph, comme noté précédemment, et changeront donc si votre personnage quitte son corps et en prend un autre. Les morphs peuvent aussi affecter d'autres compétences et statistiques, comme détaillée dans la description des morphs. 

It is important that you keep ego- and morph-derived characteristics straight, especially when updating your character's record sheet. 

\subsection{character stats} \label{sec:character-stats} 

Your character's stats measure several characteristics that are important to game play: Initiative, Speed, Durability, Wound Threshold, Lucidity, Trauma Threshold, and Moxie. Some of these stats are inherent to your character's ego, others are influenced or determined by morph. 

\end{itemize} \textbf{Ego stats} \end{quotation} \item Initiative \item Lucidity \item Trauma \item Threshold \item Insanity \item Rating \item Moxie \end{itemize} \end{quotation} 

\end{itemize} \textbf{Morph stats} \end{quotation} \item Speed \item Durability \item Wound \item Threshold \item Death \item Rating \item Damage \item Bonus \end{itemize} \end{quotation} 

\subsubsection{Initiative (init)} \label{sec:initiative-init} 

Your character's Initiative stat helps determine when they act in relation to other characters during the Action Turn (see Initiative, p. 188). Your Initiative stat is equal to your character's Intuition + Reflexes aptitudes (see Aptitudes, p. 123) multiplied by 2. Certain implants and other factors may modify this score. 

\end{quotation} Lazaro's Intuition is 15 and his Reflexes score is 20. That means his Initiative is 70 (15 + 20 = 35, 35 x 2 = 70). \end{quotation} 

\subsubsection{Speed (spd)} \label{sec:speed-spd} 

The Speed stat determines how often your character gets to act in an Action Turn (see Initiative, p. 188). All characters start with a Speed stat of 1, meaning they act once per turn. Certain implants and other advantages may boost this up to a maximum of 4. 

\subsubsection{Durability (dur)} \label{sec:durability-dur} 

Durability is your morph's physical health (or structural integrity in the case of synthetic shells, or system integrity in the case of infomorphs). It determines the amount of damage your morph can take before you are incapacitated or killed (see Physical Health, p. 206). 

Durability is unlimited, though the range for baseline (unmodified) humans tends to fall between 20 and 60. Your Durability stat is determined by your morph. 

\subsubsection{Wound threshold (wt)} \label{sec:wound-threshold-wt} 

A Wound Threshold is used to determine if you receive a wound each time you take physical damage (see Physical Health, p. 206). The higher the Wound Threshold, the more resistant to serious injury you are. 

Wound Threshold is calculated by dividing Durability by 5 (rounding up). 

\subsubsection{Death rating (dr)} \label{sec:death-rating-dr} 

Death Rating is the total amount of damage your morph can take before it is killed or destroyed beyond repair. Death Rating is equal to DUR x 1.5 for biomorphs and DUR x 2 for synthmorphs. 

\end{quotation} Tyska is sleeved in a run-of-the-mill splicer morph with a Durability of 30. That gives him a Wound Threshold of 6 (30 / 5) and a Death Rating of 45 (30 x 1.5). If Tyska acquired an implant that boosted his Durability by +10 to 40, his Wound Threshold would be 8 (40 / 5) and his Death Rating would be 60 (40 x 1.5). \end{quotation} 

\subsubsection{Lucidity (luc)} \label{sec:lucidity-luc} 

Lucidity is similar to Durability, except that it measures mental health and state of mind rather than physical well-being. Your Lucidity determines how much stress (mental damage) you can take before you are incapacitated or driven insane (see Mental Health, p. 209). 

Lucidity is unlimited, but generally ranges from 20 to 60 for baseline unmodified humans. Lucidity is determined by your Willpower aptitude x 2. 

\subsubsection{Trauma threshold (tt)} \label{sec:trauma-threshold-tt} 

The Trauma Threshold determines if you suffer a trauma (mental wound) each time you take stress (see Mental Health, p. 209). A higher Trauma Threshold means that your mental state is more resilient against experiences that might inflict psychiatric disorders or other serious mental instabilities. 

Trauma Threshold is calculated by dividing Lucidity by 5 (rounding up). 

\subsubsection{Insanity rating (ir)} \label{sec:insanity-rating-ir} 

Your Insanity Rating is the total amount of stress your mind can take before you go permanently insane and are lost for good. Insanity Rating equals LUC x 2. 

\end{quotation} Cole's Willpower is 16. That makes his Lucidity stat 32 (16 x 2), his Trauma Threshold 7 (32 / 5, rounded up), and his Insanity Rating 64 (32 x 2) \end{quotation} 

\subsubsection{Moxie} \label{sec:moxie} 

Moxie represents your character's inherent talent at facing down challenges and overcoming obstacles with spirited fervor. More than just luck, Moxie is your character's ability to run the edge and do what it takes, no matter the odds. Some people consider it the evolutionary trait that spurred humankind to pick up tools, expand our brains, and face the future head on, leaving other mammals in the dust. When the sky is falling, death is imminent, and no one can help you, Moxie is what saves the day. 

The Moxie stat is rated between 1 and 10, as purchased during character creation (and perhaps raised later). In game play, Moxie is used to influence the odds in your favor. Every game session, your character begins with a number of Moxie points equal to their Moxie stat. Moxie points may be spent for any of the following effects: 

\end{itemize} \item The character may ignore all modifiers that apply to a test. The Moxie point must be spent before dice are rolled. \item The character may flip-flop a d100 roll result. For example, an 83 would become a 38. \item The character may upgrade a success, making it a critical success, as if they rolled doubles. The character must succeed in the test before they spend the Moxie point. \item The character may ignore a critical failure, treating it as a regular failure instead. \item The character may go first in an Action Phase (p. 189). \end{itemize} 

Only 1 point of Moxie may be spent on a single roll. Moxie points will fluctuate during gameplay, as they are spent and sometimes regained. 

Regaining Moxie: At the gamemaster's discretion, Moxie points may be refreshed up to the character's full Moxie stat any time the character rests for a significant period. Moxie points may also be regained if the character achieves a personal goal, as determined by their Motivation (see p. 121). The gamemaster determines how much Moxie is regained in proportion to the goal achieved. 

\end{quotation} Audrey has a difficult Piloting: Aircraft roll to make. Her skill is 61, but she's facing a lot of modifiers (–30), and if she fails she's in big trouble. She could spend a point of Moxie before the test to ignore the modifiers, but she decides to take her chances against the target number of 31. Unfortunately, she rolls an 82. Luckily, she can spend a Moxie point to flip-flop that roll and make it a 28—a success! \end{quotation} 

\subsubsection{Damage bonus} \label{sec:damage-bonus} 

The Damage Bonus stat quantifies how much extra oomph your character is able to give their melee and thrown weapons attacks. Damage Bonus is determined by dividing your Somatics aptitude (see below) by 10 and rounding down. 

\subsection{Character skills} \label{sec:character-skills} 

Skills represent your character's talents. Skills are broken down into aptitudes (ingrained abilities that everyone has) and learned skills (abilities and knowledge picked up over time). Skills determine the target number used for tests (see Making Tests, p. 115). 

\subsubsection{Aptitudes} \label{sec:aptitudes} 

Aptitudes are the core skills that every character has by default. They are the foundation on which learned skills are built. Aptitudes are purchased during character creation and rate between 1 and 30, with 10 being average for a baseline unmodified human. They represent the ingrained characteristics and talents that your character has developed from birth and stick with you even when you change morphs—though some morphs may modify your aptitude ratings. 

Each learned skill is linked to an aptitude. If a character doesn't have the skill necessary for a test, they may default to the aptitude instead (see Defaulting, p. 116). 

There are 7 aptitudes in Eclipse Phase: 

\end{itemize} \item \textbf{Cognition (COG)} is your aptitude for problem solving, logical analysis, and understanding. It also includes memory and recall. \item \textbf{Coordination (COO)} is your skill at integrat ing the actions of different parts of your morph to produce smooth, successful movements. It includes manual dexterity, fine motor control, nimbleness, and balance. \item \textbf{Intuition (INT)} is your skill at following your gut instincts and evaluating on the fly. It includes physical awareness, cleverness, and cunning. \item \textbf{Reflexes (REF)} is your skill at acting quickly. This encompasses your reaction time, your gut-level response, and your ability to think fast. \item \textbf{Savvy (SAV)} is your mental adaptability, social in tuition, and proficiency for interacting with others. It includes social awareness and manipulation. \item \textbf{Somatics (SOM)} is your skill at pushing your morph to the best of its physical ability, including the fundamental utilization of the morph's strength, endurance, and sustained positioning and motion. \item \textbf{Willpower (WIL)} is your skill for self-control, your ability to command your own destiny. \end{itemize} 

\subsubsection{Learned skills} \label{sec:learned-skills} 

Learned skills encompass a wide range of specialties and education, from combat training to negotiating to astrophysics (for a complete skill list, see p. 176). Learned skills range in rating from 1 to 99, with an average proficiency being 50. Each learned skill is linked to an aptitude, which represents the underlying competency in which the skill is based. When a learned skill is purchased (either during character generation or advancement), it is bought starting at the rating of the linked aptitude and then raised from there. If the linked aptitude is raised or modified, all skills built off it are modified appropriately as well. 

Depending on your background and faction, you may receive some starting skills for free during character creation. Like aptitudes, learned skills stay with the character even when they change morphs, though certain morphs, implants, and other factors may sometimes modify your skill rating. If you lack a particular skill called for by a test, in most cases you can default to the linked aptitude for the test (see Defaulting, p. 116). 

\subsubsection{Specializations} \label{sec:specializations} 

Specializations represent an area of concentration and focus in a particular learned skill. A character who learns a specialization is one who not only grasps the basic tenets of that skill, but they have trained hard to excel in one particular aspect of that skill's field. Specializations apply a +10 modifier when the character utilizes that skill in the area of specialization. 

Specializations may be purchased during character creation or advancement for any existing skill the character possesses with a rating of 30 or more. Only one specialization may be purchased for each skill. Specific possible specializations are noted under individual the skill descriptions (see Skills, p. 170). 

\end{quotation} Toljek has Palming skill of 63 with a specialization in Pickpocketing. Whenever he uses Palming to pick someone's pocket or otherwise steal from someone's person, his target number is 73, but for all other uses of Palming the standard 63 applies. \end{quotation} 

\subsection{Character traits} \label{sec:character-traits} 

Traits include a range of inherent qualities and features that help define your character. Some traits are positive, in that they give your character a bonus to certain stats, skills, or tests, or otherwise give them an edge in certain situations. Others are negative, in that they impair your abilities or occasionally create a glitch in your plans. Some traits apply to a character's ego, staying with them from body to body, while others only apply to a character's morph. 

Traits are purchased during character generation. Positive traits cost customization points (CP), while negative traits give you extra CP to spend on other things (see Traits, p. 145). The maximum number of CP you may spend on traits is 50, while the maximum you may gain from negative traits is 50. In rare circumstances—and only with gamemaster approval—traits may be purchased, bought off, or inflicted during gameplay (see p. 153). 

\subsection{Character morph} \label{sec:character-morph} 

In Eclipse Phase, your body is disposable. If your body gets old, sick, or too heavily damaged, you can digitize your consciousness and download it into a new body. The process isn't cheap or easy, but it offers effective immortality—as long as you remember to back yourself up and don't go insane. The term morph is used to describe any type of form your mind inhabits, whether it be a vat-grown clone sleeve, a synthetic robotic shell, a part-bio/part-flesh pod, or even the purely electronic software state of an infomorph. 

You purchase your starting morph during character creation (see p. 128). This is likely the morph you were born with (assuming you were born), though it may simply be another morph you've moved onto. 

Physical looks aside, your morph has a large impact on your characteristics. Your morph determines certain physical stats, such as Durability and Wound Threshold, and it may also influence Initiative and Speed. Morphs may also modify some of your aptitudes and learned skills. Some morphs come pre-loaded with specific traits and implants, representing how it was crafted, and you can always bling yourself out with more implants if you choose (see Implants, p. 126). All of these factors are noted in the individual morph descriptions (see p. 139). 

If you plan on switching your current morph to another during gameplay, you may first want to back yourself up (see Backups and Uploads, p. 268). Backing up regularly is always a smart option in case you suffer an accidental or untimely death. Acquiring a new morph is not always easy, especially if you want it pre-loaded according to certain specifications. The full process is detailed under Resleeving, p. 271. 

\subsubsection{Aptitude Maximums} \label{sec:aptitude-maximums} 

Every morph has an aptitude maximum, sometimes modified by traits. This maximum represents the highest value at which the character may use that aptitude while inhabiting that morph, reflecting an inherent limitation in some morphs. If a character's aptitude exceeds the aptitude maximum of their morph, they must use it at the maximum value for the duration of the time they remain in that morph. This may also affect the skills linked to that aptitude, which must be modified appropriately. 

Some implants, gear, psi, and other factors may modify a character's natural aptitudes. These augmented values may exceed a morph's aptitude maximums, as they represent external factors boosting the morph's ability. No aptitude, however, augmented or not, may ever exceed a value of 40. Innate ability only takes a person so far—after that point, actual skill is what counts. 

\end{quotation} Eva has a Cognition aptitude of 25. She is unfortunately forced to sleeve into a flat morph with an aptitude maximum of 20. For the duration of the period she inhabits that morph, her Cognition is reduced to 20, which also impacts all of her COG-linked skills, reducing them by 5. \end{quotation} 

\section{Things characters use} \label{sec:things-char-use} 

In the advanced technological setting of Eclipse Phase, characters don't get by on their wits and morphs alone; they take advantage of their credit and reputation to acquire gear and implants and use their social networks to gather information. Some characters also have the capability to use mental powers known as psi. 

\subsection{Identity} \label{sec:identity} 

In an age of ubiquitous computing and omnipresent surveillance, privacy is a thing of the past—who you are and what you do is easily accessed online. Characters in Eclipse Phase, however, are often involved in secretive or less-than-legal activities, so the way to keep the bloggers, news, paparazzi, and law off your back is to make extensive use of fake IDs. While Firewall often provides covers for its sentinel agents, it doesn't hurt to keep a few extra personas in reserve, in case matters ever go out the airlock in a hurry. Thankfully, the patchwork allegiances of city-state habitats and faction stations means that identities aren't too difficult to fake, and the ability to switch morphs makes it even easier. On the other hand, anyone with a copy of your biometrics or geneprint is going to have an edge tracking you down or finding any forensic traces you leave behind (for more on ID, see p. 279). 

\subsection{Social networks} \label{sec:social-networks} 

Social networks represent people the character knows and social groups with which they interact. These contacts, friends, and acquaintances are not just maintained in person, but also through heavy Mesh contact. Social software allows people to stay updated on what the people they know are doing, where they are, and what they are interested in, right up to the minute. Social networks also incorporate the online projects of individual members, whether it's a mesh-site loaded with a band member's songs, a personal archive of stored media, a decade of blog entries reviewing the best places to score cheap electronics, or a depository of research papers and studies someone has worked on or finds interesting. 

In game play, social networks are quite useful to characters. Their friends list is an essential resource—a pool of people you can actively poll for ideas, troll for news, listen to for the latest rumors, buy or sell gear from, hit up for expert advice, and even ask for favors. 

While a character's social networks are nebulous and constantly shifting, the use of them is not. A character takes advantage of their social networks via the Networking (Field) skill (p. 182). The exact use of this skill is covered under Reputation and Social Networks, p. 285. 

\subsection{Cred} \label{sec:cred} 

The Fall devastated the global economies and currencies of the past. In the years of reconsolidation that followed, the hypercorps and governments inaugurated a new system-wide electronic monetary system. Called credit, this currency is backed by all of the large capitalist-oriented factions and is used to trade for goods and services as well for other financial transactions. Credit is mainly transferred electronically, though certified credit chips are also common (and favored for their anonymity). Hardcopy bills are even used in some habitats. 

Depending on your background or faction, your character may be given an amount of credit at the start of the game. During game play, your character must earn credit the old-fashioned way: by earning or stealing it. 

\subsection{Rep} \label{sec:rep} 

Capitalism is no longer the only economy in town. The development of nanofabricators allowed for the existence of post-scarcity economies, a fact eagerly taken advantage of by anarchist factions and others. When anyone can make anything, concepts like property and wealth become irrelevant. The advent of functional gift and communist economies, among other alternative economic models, means that in such systems you can acquire any goods or services you need via free exchange, reciprocity, or barter—presuming you are a contributing member of such a system and respected by your peers. Likewise, art, creativity, innovation, and various forms of cultural expression have a much higher worth than they do in capitalist economies. 

In alternative economies, money is often meaningless, but reputation matters. Your reputation score represents your social capital—how esteemed you are to your peers. Rep can be increased by positively influencing, contributing to, or helping individuals or groups, and it can be decreased through antisocial behavior. In anarchist habitats, your likelihood of obtaining things that you need is entirely based on how you are viewed by others. 

Reputation is easily measured by one of several online social networks. Your actions are rewarded or punished by those with whom you interact, who can ping your Rep score with positive or negative feedback. These networks are used by all of the factions, as reputation can affect your social activities in capitalist economies as well. The primary reputation networks include: 

\end{itemize} \item \textbf{The @-list:} the Circle-A list for anarchists, Bar- soomians, Extropians, scum, and Titanians, noted as @-rep. \item \textbf{CivicNet:} used by the Jovian Republic, Lunar- Lagrange Alliance, Morningstar Constellation, Planetary Consortium, and many hypercorps, referred to as c-rep. \item \textbf{EcoWave:} used by nano-ecologists, preservation- ists, and reclaimers, referred to as e-rep. \item \textbf{Fame:} the seen-and-be-seen network used by socialites, artists, glitterati, and media, referred to as f-rep. \item \textbf{Guanxi:} used by the triads and numerous crimi- nal entities, referred to as g-rep. \item \textbf{The Eye:} used by Firewall, noted as i-rep. \item \textbf{RNA:} Research Network Affiliation, used by ar- gonauts, technologists, scientists, and researchers, referred to as r-rep. \end{itemize} 

Reputation is rated from 0-99. Depending on your background and faction, you may start with a Rep score in one or more networks. This can be bolstered through spending customization points during character creation. During game play, your Rep scores will depend entirely on your character's actions. For more information, see Reputation and Social Networks, p. 285. 

Note that each Rep score is tied to a particular identity. 

\subsection{Gear} \label{sec:gear} 

Gear is all of the equipment your character owns and keeps on their person, from weapons and armor to clothing and electronics. You buy gear for your character with customization points during character creation (see p. 136) and in the game with Credit or Rep. Certain restricted, illegal, or hard-to-find items may require special efforts to obtain (see Acquiring Gear, p. 298). If you have access to a nanofabricator, you may be able to simply build gear, presuming you have the proper blueprints (see Nanofabrication, p. 284). For a complete listing of equipment options, see the Gear chapter, p. 296. 

Even among the remaining capitalist economies, prices can vary drastically. To represent this, all gear falls into a cost category. Each category defines a range of costs, so the gamemaster can adjust the prices of individual items as appropriate to the situation. Each category also lists an average price for that category, which is used during character generation and any time the gamemaster wants to keep costs simple. See the Gear Costs table on p. 137. 

\subsection{Implants} \label{sec:implants} 

Implants include cybernetic, bionic, genetech, and nanoware enhancements to your character's morph (or mechanical enhancements in the case of a synthetic shell). These implants may give your character special abilities or modify their stats, skills, or traits. Some morphs come pre-equipped with implants, as noted in their descriptions (see p. 139). You may also special- order morphs with specific implants (see Morph Acquisition, p. 277). If you want to upgrade a morph you are already in, you can undergo surgery or other treatments to have an enhancement installed (see Healing Vats, p. 326. For a complete list of available implant/enhancement options, see pp. 300-311, Gear. 

\subsection{Psi} \label{sec:psi} 

Psi is a rare and anomalous set of mental abilities that are acquired due to infection by a strange nanovirus released during the Fall. Psi abilities are not completely understood, but they give characters certain advantages—as well as some disadvantages. A character requires the Psi trait (p. 147) to use psi abilities, which are called sleights. Psi users are called asyncs. A full explanation of psi and details on the various sleights can be found in the Mind Hacks chapter, p. 216. 

\subsection{Game rules summary} \label{sec:game-rules-summary} 

Everything you need to know about the rules—summed up on a single page. 

\subsubsection{Making tests (P. 115)} 

\end{itemize} \item Roll d100 (two ten-sided dice, read as a percentile amount, from 00 to 99). \item Target number is determined by the appropriate skill (or occasionally an aptitude). \item Difficulty is represented by modifiers. \item 00 is always a success. \item 99 is always a failure. \item Margin of Success of 30+ is an Excellent Success. \item Margin of Failure of 30+ is a Severe Failure. \item A roll of doubles (00, 11, 22, 33, etc.) equals a critical success or failure. \end{itemize} 

\subsubsection{Success test (P. 117)} 

\end{itemize} \item To succeed, roll d100 and score equal to or less than the skill +/– modifiers. \end{itemize} 

\subsubsection{Opposed test (P. 119)} 

\end{itemize} \item Each character rolls d100 against their skill +/– modifiers. \item The character who succeeds with the highest roll wins. If both characters fail, or both succeed but tie, dead- lock occurs. \end{itemize} 

\subsubsection{Simple success test (P. 118)} 

\end{itemize} \item Simple Success Tests automatically succeed. \item Success or failure on the roll simply indicates if the character succeeded strongly or poorly. \end{itemize} 

\subsubsection{Defaulting (P. 116)} 

\end{itemize} \item If a character does not have the appropriate skill for a test, they may default to the skill’s linked aptitude. \end{itemize} 

\subsubsection{Modifiers (P. 115)} 

\end{itemize} \item Modifiers always affect the target number (skill), not the roll. \item Modifiers (positive or negative) come in 3 levels of severity: \end{itemize} \item Minor (+/–10) \item Moderate (+/–20) \item Major (+/–30) \end{itemize} \item The maximum modifiers that can be applied are +/– 60. \end{itemize} 

\subsubsection{teamwork (P. 117)} 

\end{itemize} \item One character is chosen as the primary actor; they make the test. \item Each helper character adds a +10 modifier (max. +30). \end{itemize} 

\subsubsection{Taking the time (P. 118)} 

\end{itemize} \item Character may take extra time to complete an action. \item On Complex actions, each minute taken adds +10 to the test. \item On Task actions, every 50 percent extension to the timeframe adds +10 to the test. \end{itemize} 

\subsubsection{Aptitudes (P. 123)} 

\end{itemize} \item Aptitudes range from 1 to 30 (average 15). \item Aptitudes are: Cognition, Coordination, Intuition, Reflexes, Savvy, Somatics, and Willpower. \end{itemize} 

\subsubsection{Learned skills (P. 123)} 

\end{itemize} \item Skills range from 1-99 (average 50). \item Each skill is linked to and based on an aptitude. \item Morphs, gear, drugs, etc. may provide skill bonuses or penalties to individual skills. \end{itemize} 

\subsubsection{Specializations (P. 123)} 

\end{itemize} \item Specializations add +10 when using a skill for that area of concentration. \item Each skill may have only one specialization. \end{itemize} 

\subsubsection{Action turns (P. 120)} 

\end{itemize} \item Action Turns are 3 seconds in length. \item The order in which characters act is determined by Initiative. \item Automatic actions are always "on." \item Characters may take any number of Quick Actions in a Turn (minimum of 3), limited only by the gamemaster. \item Characters may only take a number of Complex Actions equal to their Speed stat. \end{itemize} 

\subsubsection{Task actions (P. 120)} 

\end{itemize} \item Task Actions are any action that requires longer than 1 Action Turn to complete. \item Task Actions list a timeframe (anywhere from 2 Turns to 2 years). \item Timeframe reduced by 10\% for each 10 points of MoS. \item If character fails, they work on the task for a minimum period equal to 10\% of the timeframe for each 10 points of MoF before realizing it's a failure. \end{itemize} 





%%% txt/132.txt

% \chapter{Character creation}
% \label{chap:character-creation}


%  The first step towards playing Eclipse Phase is to
%  define your character. If you’re new to the game and
%  setting, the easiest way to jump right in is to simply
%  select one of the Sample Characters provided on pp.
%  154–169. If you’re more familiar with RPGs, or you
%  simply want finer control over your character, you
%  can build them from scratch, perhaps using one of the
%  Sample Characters as a template. This chapter will
%  walk you through the process of character genera-
%  tion, from the general concept and personality to the
%  crunchy game statistics.



%  CHARACTER GENERATION
%  There are two parts to every player character. The
%  first is the sets of numbers and attributes that define
%  what a character is good or bad at (or even what they
%  can and can’t do) according to the game mechanics.
%  These are more than just statistics, however—these
%  characteristics help to define your character’s abili-
%  ties and interests, and by extension their background,
%  education, training, and experience. During the
%  character creation process, you will have the ability
%  to assign, adjust, and juggle these characteristics as
%  you like. If you have a pre-conceived notion of what
%  the character is about, you can optimize the stats to
%  reflect that. Alternatively, you can tweak the stats until
%  you get something you like, then base the character’s
%  backstory off of what you develop.

%  The second part to every player character is their
%  personality. What defines them as a person? What
%  makes them tick? What pisses them off? What sparks
%  their interest? What positive aspects of their person-
%  ality make them appealing as a friend, comrade, or
%  lover—or at least someone interesting to play? What
%  character flaws and quirks do they have? These ques-
%  tions matter because they will also guide you as you
%  assign stats, skills, and traits.

%  Character generation is a step-by-step process.
%  Unlike some games, the process for creating an Eclipse
%  Phase character is not random—you have complete
%  control over every aspect of your character’s design.
%  Some stages must be completed before you can move
%  on to others. The complete process is broken down
%  on the Step-By-Step Guide to Character Creation
%  sidebar.

%  CHARACTER CONCEPT
%  Deciding what/who you want to play before you make
%  the character is usually the best route. Pick a simple
%  archetype that fits your character, and work from there.
%  Do you want to play an explorer? Someone sneaky,
%  like a spy or thief? Someone cerebral, like a scientist?
%  A hardened criminal or ex-cop? Or do you prefer to
%  be a rabble-rousing agitator? You can also start with a
% CHARACTER CREATION AND ADVANCE



% D ADVANCEMENT                                                 5



% STEP-BY-STEP GUIDE TO

% CHARACTER CREATION

% 1. Define Character Concept (p. 130)

% 2. Choose Background (p. 131)

% 3. Choose Faction (p. 132)

% 4. Spend Free Points (p. 134)


%  a) 105 aptitude points


%  b) 1 Moxie


%  c) 5,000 credit


%  d) 50 Rep


%  e) Native tongue

% 5. Spend Customization Points (p. 135)


%  a) 1,000 CP to spend


%     15 CP = 1 Moxie


%     10 CP = 1 aptitude point


%     5 CP = 1 psi sleight


%     5 CP = 1 specialization


%     2 CP = 1 skill point (61-80)


%     1 CP = 1 skill point (up to 60)


%     1 CP = 1,000 credit


%     1 CP = 10 rep


%  b) Active skill minimum: 400 skill points


%  c) Knowledge skill minimum: 300 skill points


%  d) Choose Starting Morph (pp. 136 and 139)


%  e) Choose Traits (pp. 136 and 145)

% 6. Purchase Gear (p. 136)

% 7. Choose Motivation (p. 137)

% 8. Calculate Remaining Stats (p. 138)

% 9. Detail the Character (p. 138)



%  personality type and choose an associated profession. If
%  you want a social butterfly who excels at manipulating
%  people, you can play a media personality, blogger, or
%  party-going socialite. Perhaps you’d prefer a bottomed-
%  out reject with substance abuse problems, in which
%  case an ex-merc or former hypercapitalist who lost
%  his fortune and family during the Fall might fit. How
%  about an energetic, live-life-to-the-fullest, must-see-it-
%  all character? Then a habitat freerunner or professional
%  gatecrasher might be what you’re looking for.

%  Make sure to check in with the other players and
%  try to create a character that’s complementary to the
%  rest of the team—preferably one who provides some
%  skill-set the group lacks. Why create a research arche-
%  ologist if someone else is already set on playing one,
%  especially when the team lacks a good combat special-
%  ist or async? On the other hand, if your team is going
%  to be running an alien archeological expedition, then
%  having more than one researcher (each with distinct
%  areas of expertise) might not be bad.

%  Once you have the basic concept, try to fill it with
%  a few more details, making it into a one-sentence

% %%% txt/133.txt
% summary. If you started with the concept of “xeno-so-
% ciologist,” expand it to “open-minded amateur linguist
% and expert xeno-sociologist who is fascinated by alien
% cultures, collects Factor kitsch, has a high-tolerance
% for ‘yuck factors,’ and whose best friends tend to be
% uplifts and AIs.” This will give you a few more details
% around which you can focus the character’s strengths
% and weaknesses.

% CHOOSE BACKGROUND
% The first step to creating your character is to choose a
% background. Was your character born on Earth before
% the Fall? Were they raised on a habitat commune? Or
% did they start existence as a disembodied AI?

% You must choose one of the backgrounds for your
% character from the list below. Choose wisely, as each
% background may provide your character with certain
% skills, traits, limitations, or other characteristics to
% start with. Keep in mind that your background is
% where you came from, not who you are now. It is
% the past, whereas your faction represents whom your
% character is currently aligned with. Your future, of
% course, is yours to make.

% The background options presented below cover a
% wide selection of transhumanity, but they cannot cover
% every possibility. If your gamemaster allows it, you
% may work with them to develop a background that is
% not included on this list, using these as guidelines to
% keep it balanced.

% DRIFTER
% You were raised with a social grouping that remained
% on the move throughout the Sol system. This could
% have been free traders, pirates, asteroid farmers, scav-
% engers, or just migrant workers. You are used to roam-
% ing space travel between habitats and stations.
% Advantages: +10 Navigation skill, +20 Pilot: Spacecraft

% skill, +10 Networking: [Field] skill of your choice
% Disadvantages: None
% Common Morphs: All, especially Bouncers and

% Hibernoids

% FALL EVACUEE
% You were born and raised on Earth and evacuated
% during the horrors of the Fall, leaving your old life
% (and possibly your friends, family, and loved ones)
% behind you. You were lucky enough to survive with
% your body intact and continue to make a life for your-
% self out in the system.
% Advantages: +10 Pilot: Groundcraft skill, +10 Net-

% working: [Field] skill of your choice, +1 Moxie
% Disadvantages: Only 2,500 Starting Credit (can still

% buy credit with CP)
% Common Morphs: Flats, Splicers

% HYPERELITE
% You are privileged to have been raised as part of the
% immortal upper class that rules many inner system
% habitats and hypercorps. You were pampered with
% wealth and influence that most people can only
% dream of.
% Advantages: +10 Protocol skill, +10,000 Credit, +20

% Networking: Hypercorps skill
% Disadvantages: May not start with flat, splicer, or any

% pod, uplift, or synthetic morphs
% Common Morphs: Exalts, Sylphs

% INFOLIFE
% You entered existence as a digital consciousness—
% an artificial general intelligence (AGI). Your very
% existence is illegal in certain habitats (a legacy of
% those who place the Fall at the feet of rampant AIs).
% Unlike the seed AIs responsible for their Fall, your
% capacity for self-improvement is limited, though you
% do have full autonomy.
% Advantages: +30 Interfacing skill, Computer skills (In-

% fosec, Interfacing, Programming, Research) bought

% with Customization Points are half price
% Disadvantages: Real World Naiveté trait, Social Stigma

% (AGI) trait, may not purchase Psi trait, Social skills

% bought with Customization Points are double price
% Common Morphs: Infomorphs, synthetic morphs

% ISOLATE
% You were raised as part of a self-exiled grouping on
% the fringes of the system. Whether raised as part of
% a religious group, cult, social experiment, anti-tech
% cell, or a group that just wanted to be isolated, you
% spent most if not all of your upbringing isolated
% from other factions.
% Advantages: +20 to two skills of your choice
% Disadvantages: –10 starting Rep
% Common Morphs: All

% LOST
% You are a legacy of one of the most infamous debacles
% since the Fall. As a member of the “Lost generation,”
% you went through an accelerated-growth childhood,
% somehow surviving where others of your kind died,
% went insane, or were persecuted (see The Lost, p. 233).
% Your background is a social stigma, but it does provide
% you with certain advantages ... and burdens.
% Advantages: +20 to two Knowledge skills of your

% choice, Psi trait
% Disadvantages: Mental Disorder (choose two) trait,

% Social Stigma (Lost) trait, must start with Futura

% morph
% Common Morphs: Futuras

% LUNAR COLONIST
% You experienced your childhood in one of the cramped
% dome cities or underground stations on Luna, Earth’s
% moon. You had a ringside seat to the Fall of Earth.
% Advantages: +10 Pilot: Groundcraft skill, +10 to one

% Technical, Academic: [Field], or Profession: [Field] skill

% of your choice, +20 Networking: Hypercorps skill
% Disadvantages: None
% Common Morphs: Flats, Splicers

% %%% txt/134.txt
% MARTIAN

% You were raised in one of the stations on or above
% Mars, now the most populated planet in the system.
% Your home town may or may not have survived the
% Fall.
% Advantages: +10 Pilot: Groundcraft skill, +10 to one

% Technical, Academic: [Field], or Profession: [Field]

% skill of your choice, +20 Networking: Hypercorps

% skill
% Disadvantages: None
% Common Morphs: Flats, Splicers, and Rusters

% ORIGINAL SPACE COLONIST
% You, or your parents, were part of the first “gen-
% erations” of colonists/workers sent out from Earth to
% stake a claim in space, so you are familiar with the
% cramped confines of spaceflight and life aboard older
% stations and habitats. As a “zero-one G” (zero-gravity,
% first-gen), you were never part of the elite. People from
% your background typically have some sort of special-
% ized tech training as vacworkers or habtechs.
% Advantages: +10 Pilot: Spacecraft or Freefall skill,

% +10 to a Technical, Academic: [Field], or Profession:

% [Field] skill of your choice, +20 to a Networking:

% [Field] skill of your choice
% Disadvantages: None
% Common Morphs: All. Use of exotic morphs is

% common.

% RE-INSTANTIATED
% You were born and raised on Earth, but you did not
% survive the Fall. All that you know is that your body
% died there, but your backup was transmitted off-world,
% and you were one of the lucky few to be re-instantiated
% with a new morph. You may have spent years in dead
% storage, simulspace, or as an infomorph slave.
% Advantages: +10 Pilot: Groundcraft skill, +10 to a

% Networking: [Field] skill of your choice, +2 Moxie
% Disadvantages: Edited Memories trait, 0 Starting

% Credit (can still buy credit with CP)
% Common Morphs: Cases, Infomorphs, Synths

% SCUMBORN
% You were raised in the nomadic and chaotic lifestyle
% common to Scum barges.
% Advantages: +10 Persuasion or Deception skill, +10

% Scrounging skill, +20 Networking: Autonomists

% skill
% Disadvantages: None
% Common Morphs: All, especially Bouncers

% UPLIFT
% You are not even human. You were born as an uplifted
% animal: chimpanzee, gorilla, orangutan, parrot, raven,
% crow, or octopus.
% Advantages: +10 Fray skill, +10 Perception skill, +20

% to two Knowledge skills of your choice
% Disadvantages: Must choose an uplift morph to start
% Common Morphs: Neo-Avian, Neo-Hominid,

% Octomorph
% CHOOSE FACTION
% After choosing your background, you now choose
% which primary faction your character belongs to.
% This faction most likely represents the grouping that
% controls your character’s current home habitat/station,
% and to which your character holds allegiance, but this
% need not be the case. You may be a dissident member
% of your faction, living among them but opposing some
% (or all) of their core memes and perhaps agitating for
% change. Whatever the case, your faction defines how
% your character represents themself in the struggle be-
% tween ideologies post-Fall.

% You must choose one of the factions listed below.
% Like your character’s background, it will provide your
% character with certain skills, traits, limitations, or
% other characteristics.

% The factions presented here outline the most numer-
% ous and influential among transhumanity, but others
% may also exist. At your gamemaster’s discretion, you
% may develop another starting faction with them not
% included on this list.

% ANARCHIST
% You are opposed to hierarchy, favoring flat forms of
% social organization and directly democratic decision-
% making. You believe power is always corrupting and
% everyone should have a say in the decisions that affect
% their lives. According to the primitive and restrictive
% policies of the Inner system and Jovian Junta, this
% makes you an irresponsible hoodlum at best and a ter-
% rorist at worst. In your opinion, that’s comedy coming
% from governments that keep their populations in line
% with economic oppression and threats of violence.
% Advantages: +10 to a skill of your choice, +30 Net-

% working: Autonomists skill
% Disadvantages: None
% Common Morphs: All

% ARGONAUT
% You are part of a scientific techno-progressive move-
% ment that seeks to solve transhumanity’s injustices and
% inequalities with technology. You support universal
% access to technology and healthcare, open source
% models of production, morphological freedom, and
% democratization. You try to avoid factionalism and
% divisive politics, seeing transhumanity’s splintering as
% a hindrance to its perpetuation.
% Advantages: +10 to two Technical, Academic: [Field],

% or Profession: [Field] skills; +20 Networking:

% Scientists
% Disadvantages: None
% Common Morphs: All

% BARSOOMIAN
% You call the Martian outback and wilds your home.
% You are a “redneck,” a lower-class Martian from the
% rural areas that often find themselves in conflict with

% %%% txt/135.txt
% the policies and goals of the hypercorp domes and Tha
% League.
% Advantages: +10 Freerunning, +10 to one skill of your ch

% +20 Networking: Autonomists skill
% Disadvantages: None
% Common Morphs: Cases, Flats, Rusters, Splicers, Synths

% BRINKER
% You or your faction is reluctant to deal with the rest of
% transhumanity and the various goings-on in the rest of
% system. Your particular grouping may have sought out
% imposed isolation, to pursue their own interests, or they
% have been exiled for their unpopular beliefs. Or you
% simply be a loner who prefers the vast emptiness of sp
% to socializing with others. You might be a religious culti
% primitivist, a utopian, or something altogether unintere
% in transhumanity.
% Advantages: +10 Pilot: Spacecraft skill, +10 to a skill of
% choice, +20 to a Networking: [Field] skill of your choice
% Disadvantages: None
% Common Morphs: All

% CRIMINAL
% You are involved with the crime-oriented underworld.
% may work with one of the Sol system’s major criminal
% tions—triads, the Night Cartel, the ID Crew, Nine Lives,
% Familae—or one of the smaller, local operators with a
% stake in a specific habitat. You might be a vetted mem
% for-life, a reluctant recruit, or just a freelancer looking
% the next gig.
% Advantages: +10 Intimidation skill, +30 Networking: C

% nal skill
% Disadvantages: None
% Common Morphs: All

% EXTROPIAN
% You are an anarchistic supporter of the free market and
% vate property. You oppose government and favor a sys
% where security and legal matters are handled by private
% petitors. Whether you consider yourself an anarcho-capit
% or a mutualist (a difference only other Extropians can fi
% out), you occupy a middle-ground between the hyperc
% and autonomists, dealing with both and yet trusted by ne
% Advantages: +10 Persuasion skill, +20 Networking: Aut

% mists skill, +10 Networking: Hypercorps skill
% Disadvantages: None
% Common Morphs: All

% HYPERCORP
% You hail from a habitat controlled by the hypercorps.
% might be a hypercapitalist entrepeneur, a hedonistic soci
% or a lowly vacworker, but you accept that certain libe
% must be sacrificed for security and freedom.
% Advantages: +10 Protocol skill, +20 Networking: Hyperc

% skill, +10 to any Networking: [Field] skill
% Disadvantages: None
% Common Morphs: Exalts, Olympians, Splicers, Sylphs
%  ,




%  -
% y
% y




% r




%  -
% x
% g
%  -
% r

%  -




%  -

%  -
% t



% r.
%  -




%  ,

% %%% txt/136.txt
% JOVIAN
% Your faction is noted for its authoritarian regime, bio-
% conservative ideologies, and militaristic tendencies.
% Where you come from, technology is not to be trusted to
% everyone and humans need to be protected from them-
% selves. To ensure its survival, humanity must be able to
% defend itself, and unfettered growth must be checked.
% Advantages: +10 to two weapon skills of your choice,

% +10 Fray, +20 Networking: Hypercorps skill
% Disadvantages: Must start with a Flat or Splicer

% morph, may not start with any nanoware or ad-

% vanced nanotech
% Common Morphs: Flats and Splicers

% LUNAR
% You hail from Luna, the original off-Earth colony
% world. Now overpopulated and in decline, Luna is
% one of the few places where people still cling to old-
% Earth ethnic and national identities. Your home is also
% within sight of Earth, a constant reminder that encour-
% ages many “Loonies” to be Reclaimers, deploring the
% hypercorp interdiction and arguing that you have a
% right to return to Earth, terraform it, and re-establish
% it as a living homeworld.
% Advantages: +10 to one Language: [Field] of your

% choice, +20 Networking: Hypercorps skill, +10

% Networking: Ecologists skill
% Disadvantages: None
% Common Morphs: Cases, Exalts, Flats, Splicers,

% Synths

% MERCURIAL
% Your faction has no interest in co-opting their true
% natures in order to become more “human.” You might
% be an AGI that does not necessarily intertwine its
% destiny with transhumanity, or an uplift that seeks
% to preserve and promote non-human life (or at least
% your own species). You might even be an infomorph or
% posthuman who has strayed so far from transhuman
% interests and values that you now consider yourself to
% be forging a unique new path of life.
% Advantages: +10 to any two skills of your choice, +20

% to a Networking: [Field] skill of your choice
% Disadvantages: None
% Common Morphs: Infomorphs, Synths, uplift morphs

% SCUM
% This is the future we’ve all been waiting for, and you’re
% going to enjoy it to the max. A paradigm shift has oc-
% curred, and while everyone else is catching up, your
% faction embraces and revels in it. There is no more
% want, no more death, no more limits on what you can
% be. The scum have immersed themselves in a new way
% of life, changing themselves as they see fit, trying out
% new experiences, and pushing the boundaries wherever
% they can ... and fuck anyone who can’t deal with that.
% Advantages: +10 Freefall skill, +10 to a skill of your

% choice, +20 Networking: Autonomists skill
% Disadvantages: None
% Common Morphs: All
% SOCIALITE
% You are a member of the inner system glitterati, the
% media-saturated social cliques that set trends, spread
% memes, and make or break lives with whispers, in-
% nuendo, and backroom deals. You are simultaneously
% an icon and a devout follower. Culture isn’t just your
% life, it’s your weapon of choice.
% Advantages: +10 Persuasion skill, +10 Protocol skill,

% +20 Networking: Media skill
% Disadvantages: May not start with flat, pod, uplift, or

% synthetic morphs
% Common Morphs: Exalts, Olympians, Sylphs

% TITANIAN
% You are a participant in the Titanian Commonwealth’s
% socialist cyberdemocracy. Unlike other autonomist
% projects, Titanian joint efforts have assembled some
% impressive infrastructural projects as approved by
% the Titanian Plurality and pursued by state-owned
% microcorps.
% Advantages: +10 to two Technical or Academic skills

% of your choice, +20 Networking: Autonomists skill
% Disadvantages: None
% Common Morphs: All

% ULTIMATE
% Your faction sees the potential in transhumanity’s
% future and looks back upon the rest of transhuman-
% ity as weak and hedonistic. Transhumanity is set
% to take the next evolutionary step and it’s time for
% transhumans to be redesigned to the best of our
% capabilities.
% Advantages: +10 to two skills of your choice, +20 to a

% Networking: [Field] skill of your choice
% Disadvantages: May not start with Flat, Splicer, uplift,

% or pod morphs
% Common Morphs: Exalts, Remades

% VENUSIAN
% You are a supporter of the Morningstar Confedera-
% tion of Venusian aerostats, resentful of the growing
% influence of the Planetary Consortium and other en-
% trenched and conservative inner system powers. You
% see your faction’s ascension as a chance to reform the
% old guard ways of inner system politics.
% Advantages: +10 Pilot: Aircraft, +10 to one skill of

% your choice, +20 Networking: Hypercorps skill
% Disadvantages: None
% Common Morphs: Cases, Exalts, Mentons, Splicers,

% Sylphs, Synths

% SPEND FREE POINTS
% Each starting character receives an equal number of
% free points for things like rep and aptitudes. These
% free points are just the start for building your charac-
% ter, so don’t fret if you can’t get certain scores as high
% as you like. In the next stage of character creation,
% you will gain additional points with which you can
% customize your character (see Spend Customization
% Points, p. 135).

% %%% txt/137.txt


%       Tai is making a character. She decides to create a sal-


%       vage retrieval/scavenger type who started as a Lunar


%       Colonist but is now a Brinker. Together, her background


%       and faction give Tai +20 Networking: Autonomists skill,


%       +20 Networking: Hypercorps skill, +10 Pilot: Spacecraft
%  EXAMPLE






%       skill, and +10 Pilot: Groundcraft skill. She also has +10


%       to two other skills (one Academic, Professional, or Tech-


%       nical) that she’ll choose later.


%          Tai starts with 105 points for aptitudes, which works


%       out to 15 each. She wants her character to be impulsive


%       and antisocial, so right away she lowers both SAV and




% STARTING APTITUDES
% Your character receives 105 free points to distribute
% among their 7 aptitudes: Cognition, Coordination,
% Intuition, Reflexes, Savvy, Somatics, and Willpower
% (see Aptitudes, p. 123). (That breaks down to an aver-
% age of 15 per aptitude, so it may be easiest to give
% each 15 and then adjust accordingly, raising some and
% lowering others.) Each aptitude must be given at least
% 5 points (unless you take the Feeble trait, see p. 149),
% and no aptitude may be raised higher than 30 (unless
% you take the Exceptional Aptitude trait, p. 146). Note
% that certain morphs (flats and splicers, for example)
% may also put a cap on how high your aptitudes may
% be (see Aptitude Maximums, p. 124).

% For simplicity, it is recommended that aptitude
% scores be handled as multiples of 5, but this is not a
% necessity.

% NATIVE TONGUE
% Every character receives their natural Language skill
% at a rating of 70 + INT for free. This skill may be
% raised with CP (see below).

% STARTING MOXIE
% Every character starts off with a Moxie stat of 1 (see
% Moxie, p. 122).

% CREDIT
% All characters are given 5,000 credits with which to
% purchase gear during character creation, unless you
% have the Fall Evacuee or Re-instantiated background
% (in which case you start with 2,500 or 0 credits, re-
% spectively). See Purchasing Gear, p. 136, for more
% details.

% REP
% Your character isn’t a complete newbie. You get 50
% rep points to divide between the reputation networks
% of your choice (see Reputation and Social Networks,
% p. 285).

% SPEND CUSTOMIZATION POINTS
% Now that you have the basics of your character
% fleshed out, you can spend additional Customiza-
% tion Points (CP) to fine-tune your character. Each
% WIL to 10. She also wants to be smart and fast on her
% feet, so takes the extra 10 points that gives her and
% raises both COG and REF to 20. So her aptitudes are:

% COG COO        INT     REF      SAV     SOM    WIL






%                                                      EXAMPLE
%  20  15         15      20       10      15    10


% She marks down her Moxie of 1 and gets her native
% language (Chinese) at 85, both for free.

% Noting her 5,000 Credits, Tai divides her Rep score
% points evenly among @-rep and c-rep, taking 25 in each.


% Example continued on p. 137 ➟ ➟ ➟



% character is given 1,000 CP, which may be used to

% increase aptitudes, buy skills, acquire more Moxie,

% buy more credit, elevate your Rep, or purchase posi-

% tive traits. You may also take on negative traits in

% order to get even more CP with which to customize

% your character. This customization process should be

% used to tweak your character and specialize them in

% the ways you desire.


% If a gamemaster seeks a different level of gameplay,

% they can adjust this CP amount. If the gamemaster

% wants a scenario where the starting characters are

% younger or less experienced, they can lower the CP

% to 800 or even 700. On the other hand, if you want

% to create characters who start off as grizzled veterans,

% you can raise the CP to 1,100 or even 1,200.


% Not all customizations are equal—aptitudes, for ex-

% ample, are considerably more valuable than individual

% skills. To reflect this, CP must be spent at a specific

% ratio according to the specific boost desired.




%    CUSTOMIZATION POINTS


%    15 CP = 1 Moxie point


%    10 CP = 1 aptitude point


%     5 CP = 1 psi sleight


%     5 CP = 1 specialization


%     2 CP = 1 skill point (61-80)


%     1 CP = 1 skill point (up to 60)


%     1 CP = 1,000 credit


%     1 CP = 10 Rep


%    Trait and morph costs vary as noted.



% CUSTOMIZING APTITUDES

% Raising your aptitude score is quite expensive at 10

% CP per aptitude point. As noted above, no aptitude

% may be increased above 30. Keep in mind that your

% morph may also provide certain aptitude bonuses.


% INCREASING MOXIE

% Moxie may be raised at the cost of 15 CP per Moxie

% point. The maximum to which Moxie may be raised

% is 10.

% %%% txt/138.txt
% LEARNED SKILLS
% Each character must purchase a minimum of 400 skill
% points of Active skills and 300 skill points of Knowl-
% edge skills (see Skills, p. 170). Skills are bought at the
% cost of 1 CP per point. Keep in mind that learned
% skills start at the rating of the linked aptitude. For
% example, if you want to raise a skill to 30 and the
% skill’s linked aptitude is 10, you’ll need to spend 20 CP.
% Skill bonuses from background or faction should also
% be applied to the rating before you start raising the
% skill. For simplicity, it is recommended that skills be
% purchased as multiples of 5, but this is not a necessity.

% Raising a skill over 60 is expensive. Each point over
% 60 costs double. Raising a skill with a linked attribute
% of 20 up to 70 would cost 60 CP: 40 points to get from
% 20 to 60, and 20 more points to get from 60 to 70.

% No learned skill may be raised over 80 during char-
% acter creation (unless you have the Expert trait, p. 146).

% Though Knowledge skills are grouped into 5 skills,
% each is a field skill (p. 172), meaning that it can be
% taken multiple times with different fields.

% A complete list of skills can be found on p. 176.

% SPECIALIZATIONS
% Specializations (p. 173) may also be purchased at the
% cost of 5 CP per specialization. You may purchase
% specializations for both Active and Knowledge skills.
% Only 1 specialization may be purchased per skill, and
% they may only be bought for skills with a rating of
% 30+.

% BUYING MORE CREDIT
% If you want more cred to spend on gear, every CP will
% get you 1,000 credits. See Purchase Gear, p. 136, for
% details on buying stuff. The maximum CP you can
% spend on additional credits is 100.

% INCREASING REP
% If you want your character to start play with lots of
% social capital, you can increase your Rep score(s) at
% the cost of 1 CP per 10 additional points. No indi-
% vidual Rep score may be raised above 80, and the
% maximum amount of CP that may be spent on Rep
% is 35 points.

% STARTING MORPH
% Perhaps the most important use of CP is to buy the
% morph with which your character begins play. This
% may be the original bodily form in which your charac-
% ter started life, or it may simply be the sleeve they are
% currently inhabiting.

% Available morphs are listed starting on p. 139.

% Note that any aptitude or skill bonuses provided by
% the morph are applied after all CP are spent. In other
% words, these bonuses do not affect the costs of buying
% aptitude and skill points during character generation.
% No aptitude may be modified over 40.

% PURCHASING TRAITS
% Traits represent specific qualities your character has
% that may help or hinder them.

% Positive traits supply bonuses in certain situations,
% and each has a listed CP cost. You may not spend
% more than 50 CP on positive traits.

% Negative traits inflict disadvantages on your charac-
% ter, but they also give you extra CP that you can spend
% on customizing your character. You may not purchase
% more than 50 CP worth of negative traits, and no
% more than 25 CP may be negative morph traits.

% Positive traits are listed on p. 145, negative traits on
% p. 148. Note that traits you receive from your back-
% ground or faction do not cost or provide you with
% bonus CP.

% Traits listed as morph traits apply to the morph, and
% not the ego. If the character switches to a new morph,
% these traits are lost (and new morph traits may be
% gained). Morph traits may be bought like other traits
% during character generation.

% PSI SLEIGHTS
% Characters who purchase the Psi trait (p. 147) may
% spend CP to purchase sleights (see Sleights, p. 223).
% These represent specific psi abilities the character has
% learned. The cost to buy a sleight is 5 CP. No more

% %%% txt/139.txt


%        ➟ ➟ ➟ vExample continued from p. 135



%        Tai now has 1,000 points to customize. She wants to be


%        lucky, so she starts right off spending 60 (4 x 15) CP to


%        raise her Moxie from 1 to 5. She also decides that she


%        wants her character to be better at spotting things, so


%        she raises her INT from 15 to 20, at a cost of 50 CP (5 x


%        10). So far, she’s spent 110 CP.


%           She must buy at least 400 points of Active skills, so she


%        tackles that next. She knows that skills are based on apti-


%        tudes and they get more expensive over 60, so she decides


%        the most she’ll spend on any single skill is 40 (since her


%        highest aptitude is 20). She picks out her skills, assigns the


%        points, and adds them to the starting aptitudes.


%           This is what she starts with, noting the points she


%        spent on each and the total value (counting aptitude)


%        in parentheses.

% EXAMPLE






%           Beam Weapons (COO) 30 (45), Climbing (SOM) 30


%        (45), Demolitions (COG) 40 (60), Fray (REF) 30 (50),


%        Freefall (REF) 40 (60), Freerunning (SOM) 30 (45),


%        Hardware: Aerospace (COG) 40 (60), Infiltration (COO)


%        30 (45), Interfacing (COG) 20 (40), Navigation (INT) 40


%        (60), Perception (INT) 40 (60), Persuasion (SAV) 20 (30),


%        Research (COG) 20 (40), and Scrounging (INT) 40 (60).


%           This costs her 450 CP, so she’s spent a total of 560


%        CP so far.


%           Now she spends her 300 points of Knowledge skills:


%           Academics: Astrophysics (COG) 40 (60), Academics:


%        Engineering (COG) 40 (60), Academics: Fall History


%        (COG) 40 (60), Art: Sculpture (INT) 40 (60), Interest:


%        Brinker Stations (COG) 40 (60), Interest: Conspiracies


%        (COG) 30 (50), Language: English (INT) 40 (60), Profes-


%        sion: Appraisal (COG) 40 (60), Profession: Scavenger


%        Trade (COG) 40 (60).


%  than 5 psi-chi and 5 psi-gamma sleights may be bought
%  during character creation.

%  Note that any skill or aptitude bonuses from
%  sleights are treated as modifications; they are applied
%  after all CP are spent and do not affect the cost of
%  buying skills or aptitudes during character creation.

%  PURCHASE GEAR
%  No matter what faction you are from, you use Credit
%  to buy gear during character creation. A complete list
%  of gear and costs can be found in the Gear chapter, p.
%  294. The average costs for each cost category should
%  be used when calculating gear prices.




%                   GEAR COSTS
% CATEGORY           RANGE (CREDITS)            AVERAGE (CREDITS)
% Trivial                    1–99                          50
% Low                      100–499                        250
% Moderate                500–1,499                      1,000
% High                   1,500–9,999                     5,000
% Expensive                10,000+                      20,000

%  This costs her another 350 CP, bringing her total spent
% CP to 910.

%  Adding in her background and faction skills, she also
% has Networking: Autonomists (SAV) 30, Networking:
% Hypercorps (SAV) 30, Pilot: Spacecraft (REF) 30 (50),
% Pilot: Groundcraft (REF) 30 (50). She takes the freebie
% +10 and adds it to Fray (raising it to 60) and applies the
% other +10 to Academics: Economics (COG) 30.

%  With 90 CP left, Tai moves on to Rep. Tai wants to
% have a lot of good connections, so she raises both of her
% Rep scores by 30 points each, at a cost of 6 CP. She also
% decides she needs some credibility with criminal types,
% so she buys g-rep at 40, for 4 more CP. Now she has 80
% CP left.

%  Tai’s character needs a body, and she decides a


%                                                          EXAMPLE



% bouncer is most suited for the nomadic, spacefaring
% lifestyle of her brinker. That costs another 40 CP, leaving
% her with 50 CP left to spend.

%  Looking back at her skills, she decides to raise her
% Pilot: Spacecraft from 50 to 70. It costs her 10 CP to
% raise the skill to 60, and another 20 CP to raise it from
% 60 to 70, for a total cost of 30 CP. She also wants to
% raise her Scrounging from 60 to 70, for a 20 CP cost.
% That nicely uses up the last of her CP.

%  Scanning the traits, though, Tai also decides that
% Situational Awareness would be a good choice for her
% scavenger. At a cost of 10 CP, she would need to take an-
% other negative trait to compensate. She chooses Neural
% Damage (synaesthesia)—a condition she inherited from
% a rampaging nanovirus during the Fall.

%  Tai’s points are now all evened out and spent.






% Every character starts off with one piece of gear

% for free: a standard muse (p. 332). This is the digital

% AI companion that the character has had since they

% were a child. Additionally, each character starts with 1

% month of backup insurance (p. 330) at no cost.


% There is no limitation other than what the game-

% master allows on what gear characters can and cannot

% buy during character creation. Both the players and

% gamemaster should keep the character’s background

% and faction in mind. Since some gear is likely very

% restricted in some habitats if not outright illegal, there

% needs to be a plausible explanation for who and how

% a character from such a place might have such gear.

% If there isn’t, then the gamemaster can choose not to

% allow it. The starting locale for a game should also

% be considered. A character from the restrictive Jovian

% Republic might have a hard time explaining how they

% have an illegal cornucopia machine back in the Re-

% public, but if the game takes place on board a scum

% barge where everything is available and anything goes,

% then such an explanation becomes much easier.


% Note that any skill or aptitude bonuses from gear

% are treated as modifications; they are applied after

% all CP are spent and do not affect the cost of buying

% skills or aptitudes during character creation.

% %%% txt/140.txt
% CHOOSE MOTIVATIONS
% The next step is to choose 3 personal motivations
%  for your character (see Motivations, p. 121). These
%  are memes, in the form of ideologies or goals, which
% your character is pursuing. These may be as specific
% “undermine the local triad boss” or as broad as “pro-
%  mote hypercapitalism,” and they may be short term
%  or long term. Some sample motivations are provided
%  on the Example Motivations table (p. 138). You
%  should work with your gamemaster when choosing
% your motivations, as they can be used to propel the
%  storyline forward and specific scenarios can be con-
%  structed around your character’s goals. Some of your
%  character’s motivations may change later (see Chang-
%  ing Motivation, p. 152). Motivations will help your
%  character regain Moxie (p. 122) and earn extra Rez
%  Points during gameplay (p. 384).

%  Motivations should be listed on your character
%  sheet as a single term or short phrase, along with a +
%  or – symbol to denote whether they support or oppose
%  it. For example, “+Fame” would indicate that your
%  character seeks to become a famous media personality,
%  whereas “–Reclaim Earth” means that your character
%  opposes the goal of reclaiming Earth.

% EXAMPLE MOTIVATIONS
% Alien Contact
% Anarchism
% Artistic Expression
% Bioconservatism
% Education
% Exploration
% Fame
% Fascism
% Hedonism
% Hypercapitalism
% Immortality
% Libertarianism
% Martian Liberation
% Morphological Freedom
% Nano-ecology
% Open Source
% Personal Career
% Personal Development
% Philanthropy
% Preservationism
% Reclaiming Earth
% Religion
% Research
% (AI/Infomorph/Pod/Uplift) Rights
% (AI/Infomorph/Pod/Uplift) Slavery
% Socialism
% Techno-Progressivism
% Vengeance
% Venusian Sovereignty
% Wealth



% FINAL TOUCHES
% Now that you have everything settled, there are a few
% final steps.

% REMAINING STATS
% A few stats now need to be calculated and added to
% your character sheet:

%  • Lucidity (p. 122) equals your character’s WIL x 2.
%  • Trauma Threshold (p. 122) equals your LUC

% divided by 5 (round up).
%  • Insanity Rating (p. 122) equals LUC x 2.
%  • Initiative (p. 121) equals your character’s (REF +

% INT) x 2.
%  • Damage Bonus (p. 123) for melee equals SOM ÷

% 10 (round down).
%  • Death Rating (p. 122) equals DUR x

% 1.5 (biomorphs, round up) or DUR x 2

% (synthmorphs)
%  • Speed (p. 121) equals 1 (3 for infomorphs), modi-

% fied as appropriate by implants.

% %%% txt/141.txt
% DETAILING THE CHARACTER
% The final step in character creation is filling in the de-
% tails and figuring out what your character is like and
% what they are all about. Your character’s Background
% is a good place to start as it says where they came, but
% it could be expanded. What did they think of their
% childhood? Do they still have ties from there? How
% did they move from such origins to the Faction they
% are part of? Are they fully supportive of their Faction’s
% goals, or are they in opposition? How does the charac-
% ter view other Factions?

% Next, take a look at the skills and other defining
% points—these also tell a story. How did they acquire
% those skills? Why? How did they develop their Rep
% score (or lack of one)? How did they get connected
% with the groupings represented by their Networking
% skills? What do the character’s traits say about them?
% How did they get their current morph? Is it their
% original? If not, what happened to their first body?

% Also taking into account the major factor of Moti-
% vations, all of these questions will help you build a de-
% fining picture of your character. Not everything about
% your character needs to be filled out, of course—it’s
% ok to leave a few blanks that you can fill in later. As-
% sembling the points you have deduced so far will help
% you to present your character as a whole, unique indi-
% vidual, however, rather than just a blank template.

% As a final step, take a few minutes to pick out some
% specific identifying features and personality quirks that
% will help you define the character to others. This could
% be a way of talking, a strongly-projected attitude, a
% catchphrase they use frequently, a unique look or style
% of dress, a repetitive behavior, an annoying manner-
% ism, or anything else similar that is easy to latch onto.
% Such idiosyncrasies give something that other players
% can latch onto, spurring roleplaying opportunities.



% STARTING MORPHS
% Each morph has an associated CP cost. It also sup-
% plies the character’s Durability and Wound Threshold
% stats, and may modify Initiative, Speed, and certain
% aptitudes and learned skills. A credit cost is also listed,
% but this refers to the cost of buying such a morph in
% gameplay.

% Flexible Aptitude Bonuses: Some morphs have ap-
% titude bonuses that may be applied to an aptitude of
% the player’s choice. This reflects that not all morphs
% are created equal. When assigning these universal ap-
% titude bonuses, each boost must be applied to a sepa-
% rate aptitude; you may not elevate an aptitude that
% is already raised by that morph. Once an individual
% morph’s aptitude bonuses have been assigned, they
% are permanent for that morph (i.e., if another char-
% acter resleeves into that morph, the bonuses remain
% the same).

% BIOMORPHS
% Biomorphs are fully biological sleeves (usually
% equipped with implants), birthed naturally or in an
% exowomb, and grown to adulthood either naturally
% or at a slightly accelerated rate.

% FLATS
% Flats are baseline unmodified humans, born with all
% of the natural defects, hereditary diseases, and other
% genetic mutations that evolution so lovingly applies.
% Flats are increasingly rare—most died off with the rest
% of humanity during the Fall. Most new children are
% splicers—screened and genefixed at the least—except
% in habitats where flats are treated as second-class citi-
% zens and indentured labor.
% Implants: None
% Aptitude Maximum: 20
% Durability: 30
% Wound Threshold: 6
% Disadvantages: None (Genetic Defects trait common)
% CP Cost: 0
% Credit Cost: High

% SPLICERS
% Splicers are genefixed humans. Their genome has
% been cleansed of hereditary diseases and optimized
% for looks and health, but has not otherwise been
% substantially upgraded. Splicers make up the major-
% ity of transhumanity.
% Implants: Basic Biomods, Basic Mesh Inserts,

% Cortical Stack
% Aptitude Maximum: 25
% Durability: 30
% Wound Threshold: 6
% Advantages: +5 to one aptitude of the player’s choice
% CP Cost: 10
% Credit Cost: High

% EXALTS
% Exalt morphs are genetically-enhanced humans, de-
% signed to emphasize specific traits. Their genetic code
% has been tweaked to make them healthier, smarter,
% and more attractive. Their metabolism is modified to
% predispose them towards staying fit and athletic for
% the duration of an extended lifespan.
% Implants: Basic Biomods, Basic Mesh Inserts,

% Cortical Stack
% Aptitude Maximum: 30
% Durability: 35
% Wound Threshold: 7
% Advantages: +5 COG, +5 to three other aptitudes of

% the player’s choice
% CP Cost: 30
% Credit Cost: Expensive

% MENTONS
% Mentons are genetically modified to increase cogni-
% tive abilities, particularly learning ability, creativity,
% attentiveness, and memory. Rumors exist of super-
% enhanced mentons with more extreme intelligence
% mods, but brain-hacking is notoriously difficult, and
% many attempts to redesign mental faculties result in
% impaired functioning, instability, or insanity.

% %%% txt/142.txt
% Implants: Basic Biomods, Basic Mesh Inserts, Cortical

% Stack, Eidetic Memory, Hyper-Linguist, Math Boost
% Aptitude Maximum: 30
% Durability: 35
% Wound Threshold: 7
% Advantages: +10 COG, +5 INT, +5 WIL, +5 to one

% aptitude of the player’s choice
% CP Cost: 40
% Credit Cost: Expensive

% OLYMPIANS
% Olympians are human upgrades with improved
% athletic capabilities like endurance, eye-hand coor-
% dination, and cardio-vascular functions. Olympians
% are common among athletes, dancers, freerunners,
% and soldiers.
% Implants: Basic Biomods, Basic Mesh Inserts,

% Cortical Stack
% Aptitude Maximum: 30
% Durability: 40
% Wound Threshold: 8
% Advantages: +5 COO, +5 REF, +10 SOM, +5 to one

% other aptitude of the player’s choice
% CP Cost: 40
% Credit Cost: Expensive

% SYLPHS
% Sylph morphs are tailor-made for media icons, elite
% socialites, XP stars, models, and narcissists. Sylph
% gene sequences are specifi cally designed for dis-
% tinctive good looks. Ethereal and elfin features are
% common, with slim and lithe bodies. Their metabo-
% lism has also been sanitized to eliminate unpleasant
% bodily odors and their pheromones adjusted for
% universal appeal.
% Implants: Basic Biomods, Basic Mesh Inserts, Clean

% Metabolism, Cortical Stack, Enhanced Pheromones
% Aptitude Maximum: 30
% Durability: 35
% Wound Threshold: 7
% Advantages: Striking Looks (Level 1) trait, +5 COO,

% +10 SAV, +5 to one other aptitude of the player’s

% choice
% CP Cost: 40
% Credit Cost: Expensive

% BOUNCERS
% Bouncers are humans genetically adapted for zero-G
% and microgravity environments. Their legs are more
% limber, and their feet can grasp as well as their hands.
% Implants: Basic Biomods, Basic Mesh Inserts, Cortical

% Stack, Grip Pads, Oxygen Reserve, Prehensile Feet
% Aptitude Maximum: 30
% Durability: 35
% Wound Threshold: 7
% Advantages: Limber (Level 1) trait, +5 COO, +5 SOM,

% +5 to one aptitude of the player’s choice
% CP Cost: 40
% Credit Cost: Expensive
% FURIES
% Furies are combat morphs. These transgenic human
% upgrades feature genetics tailored for endurance,
% strength, and reflexes, as well as behavioral modifica-
% tions for aggressiveness and cunning. To offset tenden-
% cies for unruliness and macho behavior patterns, furies
% feature gene sequences promoting pack mentalities and
% cooperation, and they tend to be biologically female.
% Implants: Basic Biomods, Basic Mesh Inserts, Bio-

% weave Armor (Light), Cortical Stack, Enhanced

% Vision, Neurachem (Level 1), Toxin Filters
% Aptitude Maximum: 30
% Speed Modifier: +1 (neurachem)
% Durability: 50
% Wound Threshold: 10
% Advantages: +5 COO, +5 REF, +10 SOM, +5 WIL, +5

% to one aptitude of the player’s choice
% CP Cost: 75
% Credit Cost: Expensive (minimum 40,000)

% FUTURAS
% An exalt variant, futura morphs were specially crafted
% for the “Lost generation.” Tailor-made for acceler-
% ated growth and adjusted for confidence, self-reliance,
% and adaptability, futuras were intended to help tran-
% shumanity regain its foothold. These programs proved
% disastrous and the line was discontinued, but some
% models remain, viewed by some with distaste and
% others as collectibles or exotic oddities.
% Implants: Basic Biomods, Basic Mesh Inserts, Cortical

% Stack, Eidetic Memory, Emotional Dampers
% Aptitude Maximum: 30
% Durability: 35
% Wound Threshold: 7
% Advantages: +5 COG, +5 SAV, +10 WIL, +5 to one

% other aptitude of the player’s choice
% CP Cost: 40
% Credit Cost: Expensive (exceptionally rare; 50,000+)

% GHOSTS
% Ghosts are partially designed for combat applications,
% but their primary focus is stealth and infiltration.
% Their genetic profile encourages speed, agility, and
% reflexes, and their minds are modified for patience and
% problem-solving.
% Implants: Basic Biomods, Basic Mesh Inserts, Chame-

% leon Skin, Cortical Stack, Adrenal Boost, Enhanced

% Vision, Grip Pads
% Aptitude Maximum: 30
% Durability: 45
% Wound Threshold: 9
% Advantages: +10 COO, +5 REF, +5 SOM, +5 WIL, +5

% to one aptitude of the player’s choice
% CP Cost: 70
% Credit Cost: Expensive (minimum 40,000)

% HIBERNOIDS
% Hibernoids are transgenic-modified humans with
% heavily-altered sleep patterns and metabolic processes.
% Hibernoids have a decreased need for sleep, requiring

% %%% txt/143.txt
% only 1-2 hours a day on average. They also have the
% ability to trigger a form of voluntary hibernation,
% effectively stopping their metabolism and need for
% oxygen. Hibernoids make excellent long-duration
% space travelers and habtechs, but these morphs are
% also favored by personal aides and hypercapitalists
% with non-stop lifestyles.
% Implants: Basic Biomods, Basic Mesh Inserts, Circa-

% dian Regulation, Cortical Stack, Hibernation
% Aptitude Maximum: 25
% Durability: 35
% Wound Threshold: 7
% Advantages: +5 INT, +5 to one aptitude of the player’s

% choice
% CP Cost: 25
% Credit Cost: Expensive

% NEOTENICS
% Neotenics are transhumans modified to retain a child-
% like form. They are smaller, more agile, inquisitive,
% and less resource-depleting, making them ideal for
% habitat living and spacecraft. Some people find neo-
% tenic sleeves distasteful, especially when employed in
% certain media and sex work capacities.
% Implants: Basic Biomods, Basic Mesh Inserts, Cortical

% Stack
% Aptitude Maximum: 20 (SOM), 30 (all else)
% Durability: 30
% Wound Threshold: 6
% Advantages: +5 COO, +5 INT, +5 REF, +5 to one

% aptitude of the player’s choice; neotenics count as a

% small target (–10 modifier to hit in combat)
% Disadvantages: Social Stigma (Neotenic) trait
% CP Cost: 25
% Credit Cost: Expensive

% REMADE
% The remade are completely redesigned humans:
% humans 2.0. Their cardiovascular systems are stronger,
% the digestive tract has been sanitized and restructured
% to eliminate flaws, and they have otherwise been opti-
% mized for good health, smarts, and longevity with nu-
% merous transgenic mods. The remade are popular with
% the ultimates faction. The remade look close to human,
% but are different in very noticeable and sometimes eerie
% ways: taller, lack of hair, slightly larger craniums, wider
% eyes, smaller noses, smaller teeth, and elongated digits.
% Implants: Basic Biomods, Basic Mesh Inserts, Circa-

% dian Regulation, Clean Metabolism, Cortical Stack,

% Eidetic Memory, Enhanced Respiration, Tempera-

% ture Tolerance, Toxin Filters
% Aptitude Maximum: 40
% Durability: 40
% Wound Threshold: 8
% Advantages: +10 COG, +5 SAV, +10 SOM, +5 to two

% other aptitudes of the player’s choice
% Disadvantages: Uncanny Valley trait
% CP Cost: 60
% Credit Cost: Expensive (minimum 40,000+)
% RUSTERS
% Adapted for survival with minimum gear in the not-
% yet-terraformed Martian environment, these transgenic
% morphs feature insulated skin for more effective ther-
% moregulation and respiratory system improvements to
% require less oxygen and filter carbon dioxyde, among
% other mods.
% Implants: Basic Biomods, Basic Mesh Inserts, Cortical

% Stack, Enhanced Respiration, Temperature Tolerance
% Aptitude Maximum: 25
% Durability: 35
% Wound Threshold: 7
% Advantages: +5 SOM, +5 to one aptitude of the

% player’s choice
% CP Cost: 25
% Credit Cost: Expensive

% NEO-AVIANS
% Neo-avians include ravens, crows, and gray parrots
% uplifted to human-level intelligence. Their physical
% sizes are much larger than their non-uplifted cous-
% ins (to the size of a human child), with larger heads
% for their increased brain size. Numerous transgenic
% modifications have been made to their wings, allow-
% ing them to retain limited flight capabilities at 1 g,
% but giving them a more bat-like physiology so they
% can bend and fold better, and adding primitive digits
% for basic tool manipulation. Their toes are also more
% articulated and now accompanied with an opposable
% thumb. Neo-avians have adapted well to microgravity
% environments, and are favored for their small size and
% reduced resource use.
% Implants: Basic Biomods, Basic Mesh Inserts,

% Cortical Stack
% Aptitude Maximum: 25 (20 SOM)
% Durability: 20
% Wound Threshold: 4
% Advantages: Beak/Claw Attack (1d10 DV, use Un-

% armed Combat skill), Flight, +5 INT, +10 REF, +5

% to one other aptitude of the player’s choice
% CP Cost: 25
% Credit Cost: Expensive

% NEO-HOMINIDS
% Neo-hominids are uplifted chimpanzees, gorillas, and
% orangutans. All feature enhanced intelligence and
% bipedal frames.
% Implants: Basic Biomods, Basic Mesh Inserts,

% Cortical Stack
% Aptitude Maximum: 25
% Durability: 30
% Wound Threshold: 6
% Advantages: +5 COO, +5 INT, +5 SOM, +5 to one other

% aptitude of the player’s choice, +10 Climbing skill
% CP Cost: 25
% Credit Cost: Expensive

% OCTOMORPHS
% These uplifted octopi sleeves have proven quite useful
% in zero-gravity environments. They retain eight arms,

% %%% txt/144.txt


%           their chameleon ability to change skin


%        color, ink sacs, and a sharp beak. They also


%     have increased cranial capacity and longev-


% ity, can breathe both air and water, and lack a
%  skeletal structure so they can squeeze through tight
% spaces. Octomorphs typically crawl along in zero-
% gravity using their arm suckers and expelling air for
% propulsion and can even walk on two of their arms
% in low gravity. Their eyes have been enhanced with
% color vision, provide a 360-degree field of vision, and
% they rotationally adjust to keep the slit-shaped pupil
% aligned with “up.” A transgenic vocal system allows
% them to speak.
% Implants: Basic Biomods, Basic Mesh Inserts, Cortical

% Stack, Chameleon Skin
% Aptitude Maximum: 30
% Durability: 30
% Wound Threshold: 6
% Advantages: 8 Arms, Beak Attack (1d10 DV, use

% Unarmed Combat skill), Ink Attack (blinding, use

% Exotic Ranged: Ink Attack skill), Limber (Level 2)

% trait, 360-degree Vision, +30 Swimming skill, +10

% Climbing skill, +5 COO, +5 INT, +5 to one other

% aptitude of the player’s choice
% CP Cost: 50
% Credit Cost: Expensive (minimum 30,000+)

% PODS
% Pods (from “pod people”) are vat-grown, biological
% bodies with extremely undeveloped brains that are
% augmented with an implanted computer and cybernet-
% ics system. Though typically run by an AI, pods are
% socially unfavored in some stations, utilized in slave
% labor in others, and even illegal in some areas. Because
% pods underwent accelerated growth in their creation,
% and were mostly grown as separate parts and then


%                    assembled, their biological


%                 design includes some shortcuts


%             and limitations, offset with implants


%           and regular maintenance. They lack


%      reproductive capabilities. In many habitats,


%   their legal status is a hotly-contested issue.


%  Unless otherwise noted, pods are also consid-
%  ered biomorphs for all rules purposes.

% PLEASURE PODS
% Pleasure pods are exactly what they seem—faux
% humans designed purely for intimate entertainment
% purposes. Pleasure pods have extra nerve clusters in
% their erogenous zones, fine motor control over certain
% muscle groups, enhanced pheromones, sanitized me-
% tabolisms, and the genetics for purring. Naturally, they
% are crafted for good looks and charisma and enhanced
% in other areas as well. Pleasure pods are capable of
% switching their sex at will to male, female, hermaph-
% rodite, or neuter.
% Implants: Basic Biomods, Basic Mesh Inserts, Clean

% Metabolism, Cortical Stack, Cyberbrain, Enhanced

% Pheromones, Mnemonic Augmentation, Puppet

% Sock, Sex Switch
% Aptitude Maximum: 30
% Durability: 30
% Wound Threshold: 6
% Advantages: +5 INT, +5 SAV, +5 to one aptitude of the

% player’s choice
% Disadvantages: Social Stigma (Pleasure Pod) trait
% CP Cost: 20
% Credit Cost: High

% WORKER PODS
% Part exalt human, part machine, these basic pods are
% virtually indistinguishable from humans. Worker pods
% are often used in menial labor jobs where interaction
% with humans is necessary.

% Implants: Basic Biomods, Basic Mesh Inserts, Cor-
% tical Stack, Cyberbrain, Mnemonic Augmentation,
% Puppet Sock

% Aptitude Maximum: 30

% Durability: 35

% Wound Threshold: 7

% Advantages: +10 SOM, +5 to one aptitude of the
% player’s choice

% Disadvantages: Social Stigma (Pod) trait

% CP Cost: 20

% Credit Cost: High

% NOVACRAB
% Novacrabs are a pod design bio-engineered from
% coconut crab and spider crab stock and grown to a
% larger (human) size. Novacrabs are ideal for hazard-
% ous work environments as well as vacworker, police,
% or bodyguard duties, giving their ten 2-meter long legs,
% massive claws, and chitinous armor. They climb and
% handle microgravity well and can withstand a wide
% range of atmospheric pressure (and sudden pressure
% changes) from vacuum to deep sea. Novacrabs feature

% %%% txt/145.txt
% compound eyes (with human-equivalent image resolu-
% tion), gills, dexterous manipulatory digits on their fifth
% set of limbs, and transgenic vocal cords.
% Implants: Basic Biomods, Basic Mesh Inserts, Cara-

% pace Armor, Cortical Stack, Cyberbrain, Enhanced

% Respiration, Gills, Mnemonic Augmentation,

% Oxygen Reserve, Puppet Sock, Temperature Toler-

% ance, Vacuum Sealing
% Aptitude Maximum: 30
% Durability: 40
% Wound Threshold: 8
% Advantages: 10 legs, Carapace Armor (11/11), Claw

% Attack (DV 2d10), +10 SOM, +5 to two other apti-

% tudes of the player’s choice
% CP Cost: 60
% Credit Cost: Expensive (minimum 30,000+)

% SYNTHETIC MORPHS

% Synthetic morphs are completely artificial/robotic.
% They are usually operated by AIs or via remote con-
% trol, but the lack of available biomorphs after the
% Fall meant that many infugees resorted to resleeving
% in robotic shells, which were also cheaper, quicker to
% manufacture, and more widely available. Neverthe-
% less, synthmorphs are viewed with disdain in many
% habitats, an option that only the poor and desperate
% accept to be sleeved in. Synthetic morphs are not
% without with their advantages, however, and so are
% commonly used for menial labor, heavy labor, habitat
% construction, and security services.

% All synthmorphs have the following advantages:


% • Lack of Biological Functions. Synthmorphs need

%  not be bothered with trivialities like breathing,

%  eating, defecating, aging, sleeping, or any similar

%  minor but crucial aspects of biological life.

% • Pain Filter. Synthmorphs can filter out their

%  pain receptors, so that they are unhampered by

%  wounds or physical damage. This allows them

%  to ignore the –10 modifier from 1 wound (see

%  Wound Effects, p. 207), but they suffer –30 on

%  any tactile-based Perception Tests and will not

%  even notice they have been damaged unless they

%  succeed in a (modified) Perception Test.

% • Immunity to Shock Weapons. Synthmorphs have

%  no nervous system to disrupt, and their optical

%  electronics are carefully shielded from interfer-

%  ence. Shock attacks may temporarily disrupt their

%  wireless radio communications, however, for the

%  duration of the attack.

% • Environmental Durability. Synthmorphs are

%  built to withstand a wide range of environments,

%  from dusty Mars to the oceans of Europa to the

%  vacuum of space. They are unaffected by any but

%  the most extreme temperatures and atmospheric

%  pressures. Treat as Temperature Tolerance (p.

%  305) and Vacuum Sealing (p. 305).

% • Toughness. Synthetic shells are made to last—a

%  fact reflected in their higher Durability and built-


%   in Armor ratings. Their composition also makes


%   their physical strikes more damaging: apply a +2


%   DV modifier on unarmed attacks for human-sized


%   shells and larger.

% CASE
% Cases are extremely cheap, mass-produced robotic
% shells intended to provide an affordable remor-
% phing option for the millions of infugees created by
% the Fall. Though many varieties of case bot models
% exist, they are uniformly regarded as shoddy and
% inferior. Most case morphs are vaguely anthromor-
% phic, with a thin framework body, standing just
% shorter than an average human, and suffer from
% frequent malfunctions.
% Enhancements: Access Jacks, Basic Mesh Inserts, Cor-

% tical Stack, Cyberbrain, Mnemonic Augmentation
% Mobility System (Movement Rate): Walker (4/16)
% Aptitude Maximum: 20
% Durability: 20
% Wound Threshold: 4
% Advantages: Armor (4/4)
% Disadvantages: –5 to one chosen aptitude, Lemon trait,

% Social Stigma (Clanking Masses) trait
% CP Cost: 5
% Credit Cost: Moderate

% SYNTH
% Synths are anthromorphic robotic shells (androids
% and gynoids). They are typically used for menial
% labor jobs where pods are not as good of an option.
% Cheaper than many other morphs, they are com-
% monly used for people who need a morph quickly
% and cheaply or simply on a transient basis. Though
% they look humanoid, synths are easily recognizable
% as non-biological unless they have the synthetic mask
% option (p. 311).
% Enhancements: Access Jacks, Basic Mesh Inserts, Cor-

% tical Stack, Cyberbrain, Mnemonic Augmentation
% Mobility System: Walker (4/20)
% Aptitude Maximum: 30
% Durability: 40
% Wound Threshold: 8
% Advantages: +5 SOM, +5 to one other aptitude of the

% player’s choice, Armor 6/6
% Disadvantages: Social Stigma (Clanking Masses) trait,

% Uncanny Valley trait
% CP Cost: 30
% Credit Cost: High

% ARACHNOIDS
% Arachnoid robotic shells are 1-meter in length, seg-
% mented into two parts, with a smaller head, like a
% spider or termite. They feature four pairs of 1.5-meter-
% long retractable arms/legs, capable of rotating around
% the axis of the body, with built-in hydraulics for
% propelling the bot with small leaps. The manipula-
% tor claws on each arm/leg can be switched out with
% extendable mini-wheels for high-speed skating

% %%% txt/146.txt
% movement. A smaller pair of manipulator arms near
% the head allows for closer handling and tool use. In
% zero-G environments, arachnoids can retract their
% arms/legs and maneuver with vectored air thrusters.
% Enhancements: Access Jacks, Basic Mesh Inserts,

% Cortical Stack, Cyberbrain, Enhanced Vision, Extra

% Limbs (6 Arms/Legs), Lidar, Mnemonic Augmenta-

% tion, Pneumatic Limbs, Radar
% Mobility System: Walker (4/24), Thrust Vector (8/40)
% Aptitude Maximum: 30
% Durability: 40
% Wound Threshold: 8
% Advantages: +5 COO, +10 SOM, Armor 8/8
% CP Cost: 45
% Credit Cost: Expensive (minimum 40,000+)

% DRAGONFLY
% The dragonfly robotic morph takes the shape of a
% meter-long flexible shell with multiple wings and ma-
% nipulator arms. Capable of near-silent turbofan-aided
% flight in Earth gravity, dragonfly bots fare even better
% in microgravity.
% Enhancements: Access Jacks, Basic Mesh Inserts, Cor-

% tical Stack, Cyberbrain, Mnemonic Augmentation
% Mobility System: Winged (8/32)
% Aptitude Maximum: 30 (20 SOM)
% Durability: 25
% Wound Threshold: 5
% Advantages: Flight, +5 REF, Armor (2/2)
% CP Cost: 20
% Credit Cost: High

% FLEXBOTS
% Designed for multi-purpose functions, flexbots can
% transform their shells to suit a range of situations
% and tasks. Their core frame consists of a half-dozen
% interlocking and shape-adjustable modules capable
% of auto-transforming into a variety of shapes: multi-
% legged walker, tentacle, hovercraft, and many others.
% Each module features its own sensor units and “bush
% robot” fractal-branching digits (each capable of break-
% ing into smaller digits, down to the micrometer scale,
% allowing for ultra-fine manipulation). The flexbot
% control computer is also distributed between modules.
% Individual flexbots are only the size of a large dog, but
% multiple flexbots can join together for larger mass
% operations, even taking on heavy-duty tasks such as
% demolition, excavation, manufacturing, robotics as-
% sembly, and so on.
% Enhancements: Access Jacks, Basic Mesh Inserts, Cor-

% tical Stack, Cyberbrain, Fractal Digits, Mnemonic

% Augmentation, Modular Design, Shape Adjusting
% Mobility System: Walker (4/16), Hover (8/40)
% Aptitude Maximum: 30
% Durability: 25
% Wound Threshold: 5
% Advantages: Armor 4/4
% CP Cost: 20
% Credit Cost: Expensive (minimum 30,000+)
% REAPER
% The reaper is a common combat bot, used in place of
% biomorph soldiers and typically operated via teleop-
% eration or by autonomous AI. The reaper’s core form
% is an armored disc, so that it can turn and present a
% thin profile to an enemy. It uses vector thrust nozzles
% to maneuver in microgravity, and also takes advan-
% tage of an ionic drive for fast movement over distance.
% Four legs/manipulating arms and four weapon pods
% are folded inside its frame. The reaper’s shell is made
% of smart materials, allowing these limbs and weapon
% mounts to extrude in any direction desired and even
% to change shape and length. In gravity environments,
% the reaper walks or hops on two or four of these limbs.
% Reapers are infamous due to numerous war XPs, and
% bringing one into most habitats will undoubtedly raise
% eyebrows, if not get you arrested.
% Enhancements: 360-Degree Vision, Access Jacks,

% Anti-Glare, Basic Mesh Inserts, Cortical Stack,

% Cyberbrain, Cyber Claws, Extra Limbs (4), Heavy

% Combat Armor, Magnetic System, Pneumatic Limbs,

% Puppet Sock, Radar, Reflex Booster, Shape Adjusting,

% Structural Enhancement, T-Ray Emitter, Weapon

% Mount (Articulated, 4)
% Mobility System: Walker (4/20), Hopper (4/20), Ionic

% (12/40), Vectored Thrust (4/20)
% Aptitude Maximum: 40
% Speed Modifier: +1 (Reflex Booster)
% Durability: 50 (60 with Structural Enhancement)
% Wound Threshold: 10 (12 w/Structural Enhancement)
% Advantages: 4 Limbs, +5 COO, +10 REF (+20 with

% Reflex Booster), +10 SOM, Armor 16/16
% CP Cost: 100
% Credit Cost: Expensive (minimum 50,000+)

% SLITHEROIDS
% Slitheroid bots are synthetic shells taking the form of
% a 2-meter-long segmented metallic snake, with two re-
% tractable arms for tool use. Snake bots can coil, twist,
% and roll their bodies into a ball or hoop, moving either
% by slithering, burrowing, rolling, or pulling themselves
% along by their arms. The sensor suite and control com-
% puter are housed in the head.
% Enhancements: Access Jacks, Basic Mesh Inserts,

% Cortical Stack, Cyberbrain, Enhanced Vision, Mne-

% monic Augmentation
% Mobility System: Snake (4/16; 8/32 rolling)
% Aptitude Maximum: 30
% Durability: 45
% Wound Threshold: 9
% Advantages: +5 COO, +5 SOM, +5 to one other apti-

% tude of the player’s choice, Armor 8/8
% CP Cost: 40
% Credit Cost: Expensive

% SWARMANOID
% The swarmanoid is not a single shell per se, but rather
% a swarm of hundreds of insect-sized robotic micro-
% drones. Each individual “bug” is capable of crawling,
% rolling, hopping several meters, or using nanocopter

% %%% txt/147.txt
% fan blades for airlift. The controlling computer and
% sensor systems are distributed throughout the swarm.
% Though the swarm can “meld” together into a roughly
% child-sized shape, the swarm is incapable of tackling
% physical tasks like grabbing, lifting, or holding as a
% unit. Individual bugs are quite capable of interfacing
% with electronics.
% Enhancements: Access Jacks, Basic Mesh Inserts, Cor-

% tical Stack, Cyberbrain, Mnemonic Augmentation,

% Swarm Composition
% Mobility System: Walker (2/8), Hopper (4/20), Rotor

% (4/32)
% Aptitude Maximum: 30
% Durability: 30
% Wound Threshold: 6
% Advantages: See Swarm Composition (p. 311)
% Disadvantages: See Swarm Composition (p. 311)
% CP Cost: 25
% Credit Cost: Expensive

% INFOMORPHS
% Infomorphs are digital-only forms—they lack a physi-
% cal body. Infomorphs are sometimes carried by other
% characters instead of (or in addition to) a muse in a
% ghostrider module (p. 307). Full rules for infomorphs
% can be found on p. 264.
% Enhancements: Mnemonic Augmentation
% Aptitude Maximum: 40
% Speed Modifier: +2
% Disadvantages: No physical form
% CP Cost: 0
% Credit Cost: 0



% TRAITS
% Unless otherwise noted, listed traits are ego traits.

% POSITIVE TRAITS
% Positive traits provide bonuses to the character in
% certain situations.

% ADAPTABILITY
% Cost: 10 (Level 1) or 20 (Level 2) CP

% Resleeving is a breeze for this character. They adjust
% to new morphs much more quickly than most other
% people. Apply a +10 modifier per level for Integration
% Tests and Alienation Tests (p. 272).

% ALLIES
% Cost: 30 CP

% The character is part of or has a relationship with
% some influential group that they can occasionally call
% on for support. For example, this could be their old
% gatecrashing crew, former research lab co-workers,
% a criminal cartel they are part of, or an elite social
% clique. The gamemaster and player should work out
% what the character’s relationship is with this group,
% and why the character can call on them for aid.
% Gamemaster’s should take care that these allies are
% not abused, such as calling on them more than once
% per game session. The character’s ties to this group
% are also a two-way street—they will be expected to
% perform duties for the group on occasion as well (a
% potential plot seed for scenarios).

% AMBIDEXTROUS
% Cost: 10 CP

% The character can use and manipulate objects
% equally well with both hands (they do not suffer the
% off-hand modifier, as noted on p. 193). If the char-
% acter has other prehensile limbs (feet, tail, tentacles,
% etc), this trait may be applied to a limb other than
% the hand. This trait may be taken multiple times for
% multiple limbs.

% ANIMAL EMPATHY
% Cost: 5 CP

% The character has an instinctive feel for handling
% and working with non-sapient animals of all kinds.
% Apply a +10 modifier to Animal Handling skill tests
% or whenever the character makes a test to influence or
% interact with an animal.

% BRAVE
% Cost: 10 CP

% This character does not scare easily, and will face
% threats, intimidation, and certain bodily harm with-
% out flinching. As a side effect, the character is not
% always the best at gauging risks, especially when it
% comes to factoring in danger to others. The charac-
% ter receives a +10 modifier on all tests to resist fear
% or intimidation.

% COMMON SENSE
% Cost: 10 CP

% The character has an innate sense of judgment that
% cuts through other distractions and factors that might
% cloud a decision. Once per game session, the player
% may ask the gamemaster what choice they should
% make or what course of action they should take, and
% the gamemaster should give them solid advice based
% on what the character knows. Alternately, if the
% character is about to make a disastrous decision, the
% gamemaster can use the character’s free hint and warn
% the player they are making a mistake.

% DANGER SENSE
% Cost: 10 CP

% The character has an intuitive sixth sense that warns
% them of imminent threats. They receive a +10 modifier
% on Surprise Tests (p. 204).

% DIRECTION SENSE
% Cost: 5 CP

% Somehow the character always knows which way
% is up, north, etc., even when blinded. The character
% receives a +10 modifier for figuring out complex di-
% rections, reading maps, and remembering or retracing
% a path they have taken.

% %%% txt/148.txt
% EIDETIC MEMORY (EGO OR MORPH TRAIT)
% Cost: 10 CP

% Much like a computer, the character has perfect
% memory recall. They can remember anything they have
% sensed, often even from a single glance. This works the
% same as the eidetic memory implant (p. 301).

% EXCEPTIONAL APTITUDE
% Cost: 20 CP

% The character may raise one of their maximum
% aptitude up to 10 points over the normal aptitude cap
% (30 for flats, 35 for splicers, 40 for all others). Note
% that this trait just raises the maximum, it does not
% give the character more 10 aptitude points. This trait
% may be taken only once.

% EXPERT
% Cost: 10 CP

% The character is a legend in the use of one particular
% skill. The character may raise one learned skill over 80,
% to a maximum of 90, during character creation. This
% trait does not actually increase the skill, it just raises
% the maximum. This trait may only be taken once.

% FAST LEARNER
% Cost: 10 CP

% The character improves skills and learns new
% ones in half the time it normally takes (see Improv-
% ing Skills, p. 152).

% FIRST IMPRESSION
% Cost: 10 CP

% The character has a way of charming or otherwise
% making a good impression the first time they interact
% with someone. This innate social lubricant allows them
% to more readily deal with new contacts and slip right
% into new social environments. Apply a +10 modifier
% on social skill tests when the character is interacting
% with another character for the first time only.

% HYPER LINGUIST
% Cost: 10 CP

% The character has an intuitive understanding of
% linguistic structures that facilitates learning new
% languages easily. The character requires one-third the
% normal amount of time and experience to learn any
% language (see Improving Skills, p. 152). The character
% can also learn any human language in one day simply
% by constant immersive exposure to it. Additionally,
% the character receives a +10 modifier when attempting
% to interpret languages they don’t know.

% IMPROVED IMMUNE SYSTEM (MORPH TRAIT)
% Cost: 10 (Level 1) or 20 (Level 2) CP

% The morph’s immune system is robust and more
% resistant to diseases, drugs, and toxins—even more
% than basic bio-mods. At Level 1, apply a +10 modifier
% whenever making a test to resist infection or the ef-
% fects of a toxin or drug. At Level 2, increase this modi-
% fier to +20. This trait is only available to biomorphs.
% INNOCUOUS (MORPH TRAIT)
% Cost: 10 CP

% In an age when exotic appearances and good looks
% are commonplace, the morph’s look is surprisingly
% bland and undistinguished, in that cookie cutter sort of
% way. The character’s physical looks are so mundane that
% others have a hard time picking them out of a crowd,
% describing their appearance, or otherwise remember-
% ing physical details. Apply a –10 modifier to all tests
% made to spot, describe, or remember the character. This
% modifier does not apply to psi or mesh searches.

% LIMBER (MORPH TRAIT)
% Cost: 10 (Level 1) or 20 (Level 2) CP

% The morph is especially flexible and supple, capable
% of graceful contortions and interesting positions. At
% Level 1, the character can smoke with their toes, do the
% splits, and squeeze into small, cramped spaces. At Level
% 2, they are double-jointed escape artists. Each level pro-
% vides a +10 modifier to escaping from bonds, fitting into
% narrow confines, and other acts relying on contortion or
% flexibility. This trait is only available to biomorphs.

% MATH WIZ
% Cost: 10 CP

% The character can perform any feat of calculation,
% including the most complex and advanced math-
% ematics, instantly and with great precision, with the
% same ease an unmodified human can add 2 + 3. The
% character can calculate odds with great precision, find
% correlations in numerical data, and perform similar
% tasks with great ease. Apply a +30 modifier on tests
% involving math calculations.

% NATURAL IMMUNITY (MORPH TRAIT)
% Cost: 10 CP

% The morph has a natural immunity to a specific
% drug, disease, or toxin. When afflicted with that spe-
% cific chemical, poison, or pathogen, the character re-
% mains unaffected. At the gamemaster’s discretion, this
% immunity may not apply to certain agents. It may not
% be applied to nanodrugs or nanotoxins. This trait is
% only available to biomorphs.

% PAIN TOLERANCE (EGO OR MORPH TRAIT)
% Cost: 10 (Level 1) or 20 (Level 2) CP

% The character has a high threshold for pain toler-
% ance and is better at ignoring the effects of pain on
% their abilities and concentration. Level 1 allows them
% to ignore the –10 modifier from 1 wound. Level
% 2 allows them to ignore the –10 modifiers from 2
% wounds. This trait is only available for biomorphs.

% PATRON
% Cost: 30 CP

% The character has an influential person in their
% life who can be relied on for occasional support.
% This could be a wealthy hyperelite family member, a
% high-ranking triad boss, or an anarchist networker
% with an unbeatable reputation. When called upon,

% %%% txt/149.txt
% this patron can pull strings on the character’s behalf,
% supply resources, introduce them to people they need
% to know, and bail them out of trouble. The player
% and gamemaster should work together to define ex-
% actly who this NPC is and what their relationship
% with the player character is. Specifically, the question
% of why this patron is supporting the character should
% be answered (familial obligation? childhood buddies?
% the character saved their life once?). Gamemasters
% should be careful that this trait does not get abused.
% The patron should be an occasional help (probably
% no more than once per game session at most) but is
% not always at the character’s beck-and-call. If the
% character asks for too much, too often, they should
% find the patron’s support drying up. Additionally, the
% character may need to take action to maintain the
% relationship, such as undertaking a mission on the
% patron’s behalf. In fact, the character may only have
% their patronage because they are on-call or of use to
% the NPC in some way.

% PSI
% Cost: 20 CP (Level 1), 25 CP (Level 2)

% The character has been infected with the Watts-
% MacLeod strain of the Exsurgent virus, which altered
% their brain structure and opened the potential for their
% mind to enhance their cognitive abilities and read and
% manipulate the biological minds of others (see Psi,
% p. 220). The character may purchase and learn psi
% sleights (p. 223). At Level 1, the character may only
% use psi-chi sleights. At Level 2, the character may use
% both psi-chi and psi-gamma sleights.

% Though this trait is not very expensive, gamemasters
% should not allow it to be abused. There are a number
% of negative side effects to Watts-MacLeod infection,
% noted under Psi Drawbacks, p. 220.
% PSI CHAMELEON (EGO OR MORPH TRAIT)
% Cost: 10 CP

% The character’s mental state is naturally resistant to
% psi sensing. Apply a –10 modifier to any attempts to
% locate or detect the character via psi sleights.

% PSI DEFENSE (EGO OR MORPH TRAIT)
% Cost: 10 (Level 1) or 20 (Level 2) CP

% The character’s mind is inherently resistant to
% mental attacks. At Level 1, apply a +10 modifier to
% all defense tests made against psi attacks. At Level 2,
% apply a +20 modifier.

% RAPID HEALER (MORPH TRAIT)
% Cost: 10 CP

% The morph recovers from damage more quickly.
% Reduce the timeframes for healing by half, as noted
% on the Healing table, p. 208. This trait is only avail-
% able to biomorphs.

% RIGHT AT HOME
% Cost: 10 CP

% The character chooses one type of morph (splicer,
% neo-hominid, case, etc.). The character always
% feels right at home in morphs of this type. When
% resleeving into this type of morph, the character au-
% tomatically adjusts to the new body, no Integration
% or Alienation Test needed, suffering no penalties and
% no mental stress.

% SECOND SKIN
% Cost: 15 CP

% If your character background or faction enforces a
% restriction on your starting morph (for example, up-
% lifts must start with an uplift morph), this trait allows
% you to ignore that restriction and purchase a starting
% morph of your choice.

% %%% txt/150.txt
% SITUATIONAL AWARENESS
% Cost: 10 CP

% The character is very good at maintaining con-
% tinuous partial awareness of the goings-on in their
% immediate environment. In game terms, they do not
% suffer the Distracted modifier on Perception Tests
% to notice things even when their attention is focused
% elsewhere, or when making Quick Perception Tests
% during combat.

% STRIKING LOOKS (MORPH TRAIT)
% Cost: 10 (Level 1) or 20 (Level 2) CP

% In an age where biosculpting is easy, good looks are
% both cheap and commonplace. This morph, however,
% possesses a physical look that can only be described as
% striking and unusual, but also somehow alluring and
% fascinating—even the gorgeous and chiseled glitterati
% take notice. On social skill tests where the character’s
% beauty may affect the outcome, they receive a +10 (for
% Level 1) or +20 (for Level 2) modifier. This modifier
% is ineffective against xenomorphs or those with the
% infolife or uplift backgrounds. This trait is only avail-
% able to biomorphs.

% This modifier may be purchased for uplift morphs,
% but at half the cost, and it is only effective against
% characters with that specific uplift background (i.e.,
% neo-avians, neo-hominids, etc.).

% The one drawback to this trait is that the character
% is more easily noticed and remembered.

% TOUGH (MORPH TRAIT)
% Cost: 10 (Level 1), 20 (Level 2), or 30 (Level 3) CP

% This morph is resilient than others of its type and
% can take more physical abuse. Increase their Durabil-
% ity by +5 per level (+5 at Level 1, +10 at Level 2, and
% +15 at Level 3). This also increases Wound Threshold
% by +1, +2, and +3 respectively.

% ZOOSEMIOTICS
% Cost: 5

% A character with this trait and the Psi trait does
% not suffer a modifier when using psi sleights on non-
% sentient or partly-sentient animal species.

% NEGATIVE TRAITS

% Negative traits generally hinder the character and
% apply negative modifiers in certain circumstances.

% ADDICTION (EGO OR MORPH TRAIT)
% Bonus: 5 CP (Minor), 10 CP (Moderate), or 20 CP

% (Major)

% Addiction comes in two forms: mental (affecting
% the ego) and physical (affecting the biomorph). The
% character or morph is addicted to a drug (p. 317),
% stimulus (XP), or activity (mesh use) to a degree that
% impacts the character’s physical or mental health.
% Players and gamemasters should work together to
% agree on addictions that are appropriate for their
% game. Addiction comes in three levels of severity:
% minor, moderate, or major:

% Minor: A minor addiction is largely kept under
% control—it does not ruin the character’s life, though
% it may create some difficulties. The character may not
% even recognize or admit they have a problem. The
% character must indulge the addiction at least once
% a week, though they can go for longer without too
% much difficulty. If they fail to get their weekly dose,
% they suffer a –10 modifier on all actions until they get
% their fix.

% Moderate: A moderate addiction is in full swing.
% The character obviously has a problem, and must
% satisfy the addiction at least once a day. If they fail
% to do so, they may suffer mood swings, compulsive
% behavior, physical sickness, or other side effects until
% they indulge their craving. Apply a –20 modifier to
% all of the character’s actions until they get their fix.
% Additionally, a character with this level of addiction
% suffers a –5 DUR penalty.

% Major: A character with a major addiction is on
% the rapid road to ruin. They face cravings every 6
% hours, and suffer a –10 DUR penalty as their health
% is affected. If they fail to get their regular dosage, they
% suffer a –30 modifier on all actions until they do. If
% their life hasn’t already been ruined by their obsession,
% it soon will be.

% AGED (MORPH TRAIT)
% Bonus: 10 CP

% The morph is physically aged, and has not been
% rejuvenated. Old morphs are increasingly uncom-
% mon, though some people adopt them hoping to
% gain an air of seniority and respectability. Reduce
% the character’s aptitude maximums by 5, and apply
% a –10 modifier on all physical actions.

% This trait may only be applied to flat and splicer
% morphs.

% BAD LUCK
% Bonus: 30 CP

% Due to some inexplicable cosmic coincidence,
% things seem to go wrong around the character. The
% gamemaster is given a pool of Moxie points equal
% to the character’s Moxie stat, which also refreshes
% at the same rate as the character’s Moxie. Only the
% gamemaster may utilize this Moxie, however, and
% the purpose is to use it against the character. In other
% words, the gamemaster can use this bad Moxie to
% cause the character to automatically fail, flip-flop a
% roll, and so on. To reflect the black cloud that fol-
% lows the character, the gamemaster can even use this
% bad Moxie against the character’s friends and allies,
% when they are doing something with or related to
% the character, though this should be used sparingly.
% Gamemasters who might be reluctant to sabotage the
% character should remember that the player asked for
% it by purchasing this trait.

% %%% txt/151.txt
% BLACKLISTED
% Bonus: 5 or 20 CP

% The character has managed to get themselves
% blacklisted in certain circles, whether they actually
% did something to deserve it or not. In game terms, the
% character is barred from having a Rep score higher
% than 0 in one particular reputation network. People
% within that network will refuse to help the character
% out of fear of reprisals and ruining their own reputa-
% tion. The bonus for this trait is 20 CP if chosen for the
% rep network pertaining to the character’s own starting
% faction, and 5 CP if chosen for any other.

% BLACK MARK
% Bonus: 10 (Level 1), 20 (Level 2),

% or 30 (Level 3) CP

% At some point in the character’s past, they managed
% to do something that earned a black mark on their
% reputation. For some reason, no matter what they do,
% this black mark cannot be shaken off and continues to
% haunt their interactions. In game terms, the character
% picks one faction. Every time they interact with this
% faction (such as a Networking Test) or with an NPC
% from this faction (Social Skill Tests) who knows who
% the character is, they suffer a –10 modifier per level.

% COMBAT PARALYSIS
% Bonus: 20 CP

% The character has an unfortunate habit of freezing
% in combat or stressful situations, like a deer caught in
% headlights. Anytime violence breaks out around the
% character, or they are surprised, the character must
% make a Willpower Test in order to act or respond in
% any way. If they fail the test, they lose their action and
% simply stand there, remaining incapable of reacting to
% the situation.

% EDITED MEMORIES
% Bonus: 10 CP

% At some point in the character’s past, the charac-
% ter had certain memories strategically removed or
% otherwise lost to them. This may have been done to
% intentionally forget an unpleasant or shameful experi-
% ence or to make a break with the past. The memory
% may also have been lost by an unexpected death (with
% no recent backup), or it may have been erased against
% the character’s will. Whatever the case, the memory
% should bear some importance, and there should exist
% either evidence of what happened or NPCs who know
% the full story. This is a tool the gamemaster can use to
% haunt the character at some future point with ghosts
% from their past.

% ENEMY
% Bonus: 10 CP

% At some point in their past, the character made
% an enemy for life who continues to haunt them. The
% gamemaster and player should work out the details on
% this enmity, and the gamemaster should use the enemy
% as an occasional threat, surprise, and hindrance.
% FEEBLE
% Bonus: 20 CP

% The character is particularly weak with one aptitude.
% That aptitude must be purchased at a rating lower
% than 5, and may never be upgraded during character
% advancement. The aptitude maximum is 10, no matter
% what morph the character is wearing.

% FRAIL (MORPH TRAIT)
% Bonus: 10 (Level 1) or 20 (Level 2) CP

% This morph is not as resilient as others of its type.
% Its Durability is reduced by 5 per level. This also re-
% duces Wound Threshold by 1 or 2, respectively.

% GENETIC DEFECT (MORPH TRAIT)
% Bonus: 10 CP or 20 CP

% The morph is not genefixed, and in fact suffers
% from a genetic disorder or other impairing mutation.
% The player and gamemaster should agree on a defect
% appropriate to their game. Some possibilities include:
% heart disease, diabetes, cystic fibrosis, sickle-cell dis-
% ease, hypertension, hemophilia, or color blindness.
% A genetic disorder that creates minor complications
% and/or occasional health problems would be worth
% 10 CP, a defect that significantly impairs the charac-
% ter’s regular functioning or that inflicts chronic health
% problems is worth 20 CP. The gamemaster must de-
% termine the exact effects of the disorder on gameplay,
% as appropriate.

% This trait is only available for flats.

% IDENTITY CRISIS
% Bonus: 10 CP

% The character’s ego has trouble adapting itself to
% the changed look of a new morph—they are stuck
% with the mental image of their original body, and
% simply do not grow accustomed to their new face(s).
% As a result, the character has difficulty identifying
% themselves in the mirror, photos, surveillance feeds,
% etc. They frequently forget the look and shape of their
% current morph, acting inappropriately, describing
% themselves by their original body, forgetting to duck
% when walking through doorways, etc. This is primar-
% ily a roleplaying trait, but the gamemaster may apply
% appropriate modifiers (usually –10) to tests affected
% by this inability to adapt.

% ILLITERATE
% Bonus: 10 CP

% The character knows how to speak, but has diffi-
% culty reading or writing. Due to the entoptic-saturated
% and icon-driven nature of transhuman society, they are
% able to get by quite comfortably with this handicap.
% Reduce the character’s Language skills by half (round
% down) whenever reading or writing.

% IMMORTALITY BLUES
% Bonus: 10 CP

% The character has lived so long—over 100 years—
% they’re bored with life and now have difficulty

% %%% txt/152.txt
% motivating themselves. They were old when longevity
% treatments first became available, survived the Fall,
% and continue to soldier onward—though they find
% it increasingly harder to care, take interest in things
% around them, or fear final death. The character only
% receives half the Moxie and Rez Points award for
% completing motivational goals.

% This trait may not be purchased by characters with
% the infolife or uplift backgrounds.

% IMPLANT REJECTION (MORPH TRAIT)
% Bonus: 5 (Level 1) or 15 (Level 2) CP

% This morph does not accept implants well. At Level
% 1, any implants acquired are more expensive as they
% required specialized anti-rejection treatments. Increase
% the Cost category of the implant by one. At Level 2,
% the morph cannot accept implants of any kind.

% INCOMPETENT
% Bonus: 10 CP

% The character is completely incapable of perform-
% ing a particular chosen active skill, no matter any
% training they may receive. They may not buy this skill
% during character creation or later advancement, and
% the modifier for defaulting to the linked aptitude of
% this particular skill is –10. This may not be used for
% exotic weapon skills, and should be used for a skill
% that could be of use to the character.

% LEMON (MORPH TRAIT)
% Bonus: 10 CP

% This trait is only available for synthetic morphs.
% This particular morph has some unfixable flaws. Once
% per game session (preferably at a time that will maxi-
% mize drama or hilarity), the gamemaster can call for
% the character to make a MOX x 10 Test (using their
% current Moxie score). If the character fails, the morph
% immediately suffers 1 wound resulting from some me-
% chanical failure, electrical glitch, or other breakdown.
% This wound may be repaired as normal.

% LOW PAIN TOLERANCE (EGO OR MORPH TRAIT)
% Bonus: 20 CP

% Pain is the character’s enemy. The character has
% a very low threshold for pain tolerance and is more
% severely impaired when suffering. Increase the modi-
% fier for each wound take by an additional –10 (so
% the character suffers –20 with one wound, –40 with
% another, and –60 with a third). Additionally, the
% character suffers a –30 modifier on any test involving
% pain resistance. This morph version of this trait is only
% available for biomorphs.

% MENTAL DISORDER
% Bonus: 10 CP

% You have a psychological disorder from a previous
% traumatic experience in your life. Choose one of the
% disorders listed on p. 211.
% MILD ALLERGY (MORPH TRAIT)
% Bonus: 5 CP

% The morph is allergic to a specific chosen allergen
% (dust, dander, plant pollen, certain chemicals) and suf-
% fers mild discomfort when exposed to it (eye irritation,
% sneezing, difficult breathing). Apply a –10 modifier to
% all tests while the character remains exposed. This
% trait is only available for biomorphs.

% MODIFIED BEHAVIOR
% Bonus: 5 (Level 1), 10 (Level 2), or 20 (Level 3) CP

% The character has been conditioned via time-
% accelerated behavioral control psychosurgery. This is
% common among ex-felons, who have been conditioned
% to respond to a specific idea or activity with vehement
% horror and disgust, but may have occurred for some
% other reason or even been self-inflicted. At Level 1, the
% chosen behavior is either limited or boosted, at Level
% 2 it is either blocked or encouraged, and at Level 3 it
% is expunged or enforced (see p. 231 for details). This
% trait should only be allowed for behaviors that are
% either limited or, if encouraged, impact the character
% in a negative way.

% MORPHING DISORDER
% Cost: 10 (Level 1), 20 (Level 2), or 30 (Level 3) CP

% Adapting to new morphs is particularly challeng-
% ing for this character. The character suffers a –10
% modifier per level on Integration Tests and Alien-
% ation Tests (p. 272).

% NEURAL DAMAGE
% Bonus: 10 CP

% The character has suffered some type of neurologi-
% cal damage that simply cannot be cured. The affliction
% is now part of the character’s ego and remains with
% them even when remorphing. This damage may have
% been inherited, it may have resulted from a poorly de-
% signed morph or implant, or it may have been inflicted
% by one of the TITAN nanovirii that targeted neural
% systems during the Fall (p. 384). The gamemaster and
% player should agree on a specific disorder appropriate
% to their game. Some possibilities are:

%  • Partial aphasia (difficulty communicating or

% using words)
%  • Color blindness
%  • Amusica (inability to make or understand music)
%  • Synaesthesia
%  • Logorrhoea (excessive use of words)
%  • Loss of face recognition
%  • Loss of depth perception (double range

% modifiers)
%  • Repetitive behavior
%  • Mood swings
%  • The inability to shift attention quickly


% The gamemaster may decide to inflict modifiers
% resulting from this affliction as appropriate.

% %%% txt/153.txt
% NO CORTICAL STACK (MORPH TRAIT)
% Bonus: 10 CP

% The morph lacks the cortical stack that is common
% to morphs of its type. This means the character cannot
% be resleeved from the cortical stack if the character
% dies, they can only be resleeved from a standard
% backup. This trait is not available for flats.

% OBLIVIOUS
% Bonus: 10 CP

% The character is particularly oblivious to events
% around them or anything other than what their at-
% tention is focused on. They suffer a –10 modifier
% to Surprise Tests and their modifier for being Dis-
% tracted is –30 rather than the usual –20 (see Basic
% Perception, p. 190).

% ON THE RUN
% Bonus: 10 CP

% The character is wanted by the authorities of a
% particular habitat/station or faction, who continue to
% actively search for the character. They either commit-
% ted a crime or somehow displeased someone in power.
% The character deals with that faction in question at
% their own risk, and may occasionally be forced to deal
% with bounty hunters.

% PSI VULNERABILITY (EGO OR MORPH TRAIT)
% Bonus: 10 CP

% Something about the character’s mind makes
% them particularly vulnerable to psi attack. They
% suffer a –10 modifier when resisting such attacks.
% The morph version of this trait may only be taken
% by biomorphs.

% REAL WORLD NAIVETÉ
% Bonus: 10 CP

% Due to their background, the character has very
% limited personal experience with the real (physical)
% world—or they have spent so much time in simul-
% space that their functioning in real life is impaired.
% They lack an understanding of many physical proper-
% ties, social cues, and other factors that people with
% standard human upbringings take for granted. This
% lack of common sense may lead the character to
% misunderstand how a device works or to misinterpret
% someone’s body language.

% Once per game session, the gamemaster may in-
% tentionally mislead the character when giving them
% a description about some thing or some social in-
% teraction. This falsehood represents the character’s
% misunderstanding of the situation, and should be
% roleplayed appropriately, even if the player realizes
% the character’s mistake.

% This trait should only be available to characters
% with the infolife or reinstantiated backgrounds,
% though the gamemaster may allow it for characters
% who have extensive virtual reality/XP use in their
% personal histories.
% SEVERE ALLERGY (MORPH TRAIT)
% Bonus: 10 (uncommon) or 20 (common) CP

% The morph’s biochemistry suffers a severe allergic
% reaction (anaphylaxis) when it comes into contact
% (touched, inhaled, or ingested) with a specific allergen.
% The allergen may be common (dust, dander, plant pollen,
% certain foods, latex) or uncommon (certain drugs, insect
% stings). The player and gamemaster should agree on an
% allergen that fits the game. If exposed to the allergen,
% the character breaks into hives, has difficulty to breath-
% ing (–30 modifier to all actions), and must make a DUR
% Test or go into anaphylactic shock (dying of respiratory
% failure in 2d10 minutes unless medical care is applied).
% This trait is only available to biomorphs.

% SLOW LEARNER
% Bonus: 10 CP

% New skills are not easy for this character to pick
% up. The character takes twice as long as normal to
% improve skills or learn new ones (p. 152).

% SOCIAL STIGMA (EGO OR MORPH TRAIT)
% Bonus: 10 CP

% An unfortunate aspect of the character’s back-
% ground means that they suffer from a stigma in
% certain social situations. They may be sleeved in
% a morph viewed with repugnance, be a survivor of
% the infamous Lost generation, or may be an AGI in
% a post-Fall society plagued by fear of artificial intelli-
% gence. In social situations where the character’s nature
% is known to someone who view that nature with dis-
% taste, fear, or repugnance, they suffer a –10 to –30
% modifier (gamemaster’s discretion) to social skill tests.

% TIMID
% Bonus: 10 CP
%  This character frightens easily. They suffer a –10
% modifier when resisting fear or intimidation.

% UNATTRACTIVE (MORPH TRAIT)
% Bonus: 10 CP (Level 1), 20 CP (Level 2), 30 CP (Level 3)

% In a time when good looks are easily purchased,
% this morph is conspicuously ugly. As unattractiveness
% is increasingly associated with being poor, backward,
% or genetically defective, responses to a lack of good
% looks range from distaste to horror. The character suf-
% fers a –10 modifier on social tests for Level 1, –20 for
% Level 2, and –30 for Level 3.

% Only biomorphs may take this trait. This modifier
% does not apply to interactions with xenomorphs or
% those with the infolife or uplift backgrounds. This
% modifier may be purchased for uplift morphs, but at
% half the bonus, and it is only effective against char-
% acters with that specific uplift background (i.e., neo-
% avians, neo-hominids, etc.).

% UNCANNY VALLEY (MORPH TRAIT)
% Bonus: 10 CP

% There is a point where synthetic human looks
% become uncannily realistic and human-seeming, but

% %%% txt/154.txt
% they remain just different enough that their looks
% seem creepy or even repulsive—a phenomenon called
% the “uncanny valley.” Morphs whose looks fall into
% this range suffer a –10 modifier on social skill tests
% when dealing with humans. This modifier does not
% apply to interactions with xenomorphs or those with
% the infolife or uplift backgrounds.

% UNFIT (MORPH TRAIT)
% Bonus: 10 CP (Level 1), 20 CP (Level 2)

% The morph is either not optimized for health and/or
% just in bad shape. Reduce the aptitude maximums for
% Coordination, Reflexes, and Somatics by 5 (Level 1 )
% or 10 (Level 2).

% VR VERTIGO
% Bonus: 10 CP

% The character experiences intense vertigo and
% nausea when interfacing with any type of virtual real-
% ity, XP, or simulspace. Augmented reality has no effect,
% but VR inflicts a –30 modifier to all of the character’s
% actions. Prolonged use of VR (gamemaster’s discre-
% tion) may actually incapacitate the character should
% they fail a WIL x 2 Test.

% WEAK IMMUNE SYSTEM (MORPH TRAIT)
% Bonus: 10 (Level 1) or 20 (Level 2) CP

% The morph’s immune system is susceptible to diseas-
%  es, drugs, and toxins. At Level 1, apply a –10 modifier
% whenever making a test to resist infection or the effects
% of a toxin or drug. At Level 2, increase this modifier to
% –20. This trait is only available to biomorphs.

% ZERO-G NAUSEA (MORPH TRAIT)
% Bonus: 10 CP

% This morph suffers from space sickness and does
% not fair well in zero-gravity. The character suffers a
% –10 modifier in any microgravity climate. Addition-
% ally, whenever the character is first getting acclimated
% or anytime they must endure excessive movement in
% microgravity, they must make a WIL Test or spend 1
% hour incapacitated by nausea per 10 points of MoF.



% CHARACTER ADVANCEMENT
% As characters accomplish goals and gather experience
% during gameplay, they accumulate Rez Points (see
% Awarding Rez Points, p. 384). Rez Points may be used
% to improve the character’s skills, aptitudes, and other
% characteristics per the following rules. The costs for
% spending Rez Points for advancement are the same as
% the costs for spending Customization Points.

% CHANGING MOTIVATION
% It is only natural that over time a character’s driving
% goals and interests will change. The character may
% reach a turning point where they feel certain personal
% agendas have been fulfilled and it is time to move on,
% or they have failed and need to be discarded. New
% urgencies or philosophies may have entered the
% character’s life, or the character may have become
% disenchanted with particular memes and ideas they
% previously took to heart.

% Changing a character’s motivation does not cost
% Rez Points, but it is something that should only
% happen in accordance with roleplaying and with
% life-altering events. Players should not be allowed
% to simply switch their motivations at whim, there
% should be a driving reason or explanation for doing
% so. For this reason, changing a motivation should only
% happen when the player and gamemaster discuss the
% matter and both agree that the swap is appropriate to
% the character’s development and circumstances.

% If these conditions are met, the character simply
% drops a previously held motivation and takes on a
% new one. Only one motivation should be switched
% out at a time.

% SWITCHING MORPHS
% Resleeving—switching from one morph to another—
% is handled as an in-character interaction, not with Rez
% Points. See Resleeving, p. 271.

% IMPROVING APTITUDES
% Aptitudes may be raised with Rez Points at the cost of
% 10 RP per aptitude point. This represents the charac-
% ter’s improvement in their core characteristics, gained
% from exercise, learning, and experience. Aptitudes may
% not be raised above 30 (bonuses from morphs, im-
% plants, traits, or other sources do not count towards
% this total).

% Raising the value of an aptitude also raises the
% value of all linked skills by an equivalent amount. If
% this raises any linked skills over 60, an additional 1
% RP must be spent per linked skill over 60.

% IMPROVING SKILLS
% Characters may also spend Rez Points to increase
% existing skills or learn new ones. To improve an ex-
% isting skill, the character must have successfully used
% that skill in the recent past or must actively practice
% it in order to get better, perhaps with the aid of an
% instructor. In the case of Knowledge skills, this means
% actively studying. As a rough timeframe, this should
% require around 1 week of learning per skill point. A
% number of educational resources are freely available




%   SPENDING REZ POINTS


%  15 RP = 1 Moxie point


%  10 RP = 1 aptitude point


%   5 RP = 1 psi sleight


%   5 RP = 1 specialization


%   2 RP = 1 skill point (61-99)


%   1 RP = 1 skill point (up to 60)


%   1 RP = 10 Rep


%   1 RP = 1,000 Credits

% %%% txt/155.txt
% via the mesh, though some areas of interest         of a
% may be restricted or hard to find. This can         ate
% be handled via roleplaying or designated            cos
% as something the character is doing during          the
% downtime between sessions. If the game-             uns
% master decides that a character has not put         me
% enough effort into improving a skill, they          the
% may call for more practice/study.                   tra

% The cost to increase a skill is 1 RP per skill   wit
% point, and no skill may be increased over 99.       ext
% No skill may be raised by more than 5 points           G
% per month. When a character’s skill reaches         dif
% the level of expertise (skill of 60+), however,     unf
% they tend to reach a plateau where improve-         gam
% ment progresses more slowly and even con-           are
% sistent practice and study have diminished re-      rec
% turns. In this case, the Rez Point cost per skill   tive
% point doubles (i.e., 2 RP = +1 skill point).        sam
% When a skill reaches 80, improvement slows          be
% down even further—a skill of 80+ may not            dili
% be increased by more than 1 point per month.        com


%                                                req
% LEARNING NEW SKILLS                                 cha
% Similarly, to learn a new skill, the character      ing
% must actively study/practice and/or seek            ove
% instruction. No test to learn is required,          of
% unless the period of study was hampered             tha
% or in some way deficient, in which case the         effo
% gamemaster may call for a COG x 3 Test              Rez
% to pick up the new skill. Otherwise, once a         bon
% character has spent approximately a week            neg
% learning a new skill, they may purchase their       no
% first skill point at the usual cost (1 RP). The
% skill is bought up from the aptitude rating,        IM
% per normal. Once a new skill is acquired, it is      Re
% raised according to the standard rules above.        cre


%                                                 act
% SPECIALIZATIONS                                     Ga
% Specializations may be purchased for ex-             pre
% isting skills, as long as that skill is at least    “off
% rating 20. Specializations require a total           Poi
% of 1 month of training. The cost to learn a          boo
% specialization is 5 RP. Only 1 specialization        net
% may be purchased per skill.                          to a

% IMPROVING MOXIE                                     MA
% Moxie may be raised at the cost of 15 RP per        Rez
% Moxie point. The maximum to which Moxie             of
% may be raised is 10.                                inc


%                                                or
% GAINING/LOSING TRAITS                               sell
% At the gamemaster’s discretion, both positive
% and negative traits may be acquired or lost         IM
% during gameplay, though such changes should         Ch
% be rare and only made in accordance with the        ma
% storyline and unfolding events in the game.         223

% Both positive and negative traits may be          mu
% picked up by a character during gameplay            pra
% as a consequence of something that did or           per
% something that happened to them. In the case        lea
%  sitive trait, the character must immedi-
% pend Rez Points equal to the trait’s CP
%  r the privilege (whether they wanted
% w trait or not). If the character has no
%  t RP available, they must pay out im-
%  ely from any future RP they earn until

% t is paid off. In the case of a negative
% owever, the character is simply saddled

% e new flaw—they do not acquire any
% RP for gaining the negative trait.
%  ing rid of traits is somewhat more

% t. Positive traits may be lost due to
% unate effects on the character, as the

% aster sees fit. Such lost positive traits
% mply gone—the character does not

% any Rez Point reimbursement. Nega-

% its are occasionally eliminated in the
% way, but more typically they can only

% ked off through the hard work and
%  ce of a character that seeks to over-
%  heir handicap. Such endeavors should

% weeks if not months of effort on the
%  ter’s part, with appropriate roleplay-
% d possibly some difficult tests. In fact,
%  ming such traits could be the source
% adventure. Once a gamemaster feels
%  e character has made a strong-enough

% the character may pay a number of

% ints equal to the trait’s original CP

% o negate it. Note, however, that some

% e traits may simply not be discarded,
%  ter what the character does.

%  OVING REP
% ation is something that can be in-
% d with appropriate roleplaying and
% s during gameplay (see Reputation

% nd Loss, p. 384). Characters that
%  o handle their Rep-boosting activities
%  een,” however, can simply spend Rez
% to boost their score(s). Each RP spent
%  the character’s Rep by +10 in a single
%  k. Only one such boost may be made
%  gle rep network per month.

%  NG CREDIT
%  ints may be spent on Credit at a ratio
%  P for 1,000 Credits. This represents

% the character has earned “off-screen”
%  ng downtime, such as from odd jobs,
%  off possessions, and so on.

%  OVING PSI
%  ters who have the Psi trait (p. 147)
% urchase new sleights (see Sleights, p.

% the cost of 5 RP per sleight. Sleights
% e learned through study, training, and
% e, requiring approximately 1 month
% ght. No more than one sleight may be

% per month.

% %%% txt/156.txt


%                                                      A






%                                                        P


%                                                    Prof


%                                                    Prog
% al Space Colonist


% chism +Open Source +Tech-Hacking




%                Implants: Access Jacks, Basic
% el 1) ■               Biomods, Basic Mesh Inserts,


%                                                   You’re


%                  Cortical Stack, Electrical


%                                                   kind t


%                  Sense, Grip Pads, Oxygen


%                                                   cal to


%                  Reserve, Prehensile Feet,


%                                                   them,


%                  Wrist-Mounted Tools ■


%                                                   them


%                Gear: 2 Automechs, Backup


%                                                   imag


%                  Insurance (1 month), Engineer


%                                                   busy


%                  Nanoswarm, Fabber, Fixer


%                                                   tures
% /6] ■                 Nanoswarm, Muse, Vacsuit


%                                                   poses
% eezer                 (Light Smartfabric, 5/5) ■


%                                                   beca


%                                                   thing
% ■


%                                                   it fro


%                   “Don't worry, it works; these things r


%      I voided the warranty, but anything loaded with DR
% ANARCHIST TECHIE


%      APTITUDES


%          COG      COO    INT    REF       SAV       SOM      WIL


% Base      15       20     15     10        15        15      15
% rph Bonus      5        5                                 5


% Total     20       25     15     10       15         20       15




% OX     TT      LUC      IR    WT     DUR       DR


%                                                     STATS


%                                                     INIT     SPD


%    6       30      60     7      35       53         50       1






%                                            MORPH


%                                                 SKILLS


%                          APT    BASE       BONUS           TOTAL
% Academics: Engineering        COG      60         5               65
% emics: Nanotechnology         COG      70         5               75

%  Academics: Physics        COG      60         5               65


%       Art: Sculpture     INT      55                         55


%              Free Fall   REF      60                         60

% Hardware: Aerospace         COG      45            5            50


% Hardware: Armorer        COG      55            5            60

% Hardware: Electronics       COG      60            5            65


% Hardware: Robotics       COG      60            5            65


%               Infosec    COG      40            5            45
% nterests: Martian Beers       COG      45            5            50
%  nterests: Robot Models       COG      45            5            50

%  Interests: VR Games       COG      55            5            60


%           Interfacing    COG      50            5            55
%  nguage: Native English       INT      85                         85

% Language: Mandarin         INT      60                         60
% etworking: Autonomists        SAV      70                         70

% Networking: Firewall       SAV      35                         35
%  Networking: Scientists       SAV      35                         35


%           Perception     INT      55                         55


%        Pilot: Aircraft   REF      45                         45


%    Pilot: Spacecraft     REF      30                         30
% ession: Habitat Systems       COG      60            5            65
% on: Spacecraft Systems        COG      60            5            65
%  ming (Nanofabrication)       COG    60 (70)         5          65 (75)


%          Scrounging      INT      45                         45


%     Spray Weapons        COO      40            5            45





% old school hacker—the        for open source technology that
% likes to take technologi-      anyone can use or modify as they
%  dismantle them, modify        see fit and you support decen-
%  uild them, and then use       tralized models of peer collab-
%  ays the designers never       oration—nothing pisses you off
%  . When you’re not too         more than restrictive proprietary
% ding crazy robotic sculp-      tech with which you or others

% art performance pur-        can’t meddle. You take a hands-on
%  ur skills are in demand       approach to most problems, but
% you can fix almost any-         you become so engrossed in your
% prove it, or even build        projects that you tend to be oblivi-

% cratch. You advocate         ous to the world around you. ■
% hemselves anyway.
% ike that is broken to start with.”

% %%% txt/157.txt
% ARGONAUT XENOAR


%       APTITU      COG     COO       INT    REF    SAV       SOM


%         Base       20      15        20     15     10        15


%  Morph Bonus       5                 5              5        5


%         Total      25       15       25     15     15        20






%  MOX       TT     LUC       IR      WT     DUR    DR


%                                                             ST


%                                                          INIT


%   3         6      30       60       7      35    53    70 (80





%                                                         SKI


%                                                     MORPH


%                                     APT    BASE     BONUS


%   Academics: Astrobiology           COG     60        5


%      Academics: Chemistry           COG     40        5


%    Academics: Engineering           COG     50        5


%        Academics: Geology           COG     55        5


%        Academics: Physics           COG     50        5

%  Academics: Xenoarcheology            COG     70        5


%            Art: Architecture        INT     45        5


%             Beam Weapons            COO     35


%                      Climbing       SOM     35          5


%                  Demolitions        COG     45          5


%                          Fray       REF     45


%                      Free Fall      REF     25


%      Interests: Alien Relics        COG     45          5


%   Interests: Pandora Gates          COG     55          5


%                   Interfacing       COG     20          5


%                Investigation        INT     65          5


%                      Kinesics       SAV     30          5


%   Language: Native French           INT     90          5


%          Language: English          INT     40          5


%   Networking: Autonomists           SAV     30          5


%        Networking: Firewall         SAV     30          5


%    Networking: Hypercorps           SAV     30          5


%     Networking: Scientists          SAV     50          5


%                   Perception        INT     40          5


%                Pilot: Aircraft      REF     35


%           Pilot: Groundcraft        REF     50


%     Profession: Excavation          COG     50          5


%       Profession: Forensics         COG     55          5


%      Profession: Surveying          COG     40          5


%                      Protocol       SAV     40          5


%                     Research        COG     55          5


%                         Sense       INT     60          5

% You are convinced that something          treasure hunter, and s
% strange is going on in the galaxy.        you have the skills and tr
% Despite the Factors and signs of          decode the evidence and
% dead alien civilizations like the         left behind by unknown
% Iktomi, the sheer mathematical            and life forms. You are
% odds insist that there should be          expert for gatecrashing
% more life in the galaxy. So where         tions, but you avoid wo
% is it? Your goal is to find out, and in    hypercorps because you
% pursuit of this you study the relics      like other argonauts, that
% left behind on various exoplanets.        coveries made should be
% Equal parts conspiracy theorist,          to all. ■


%                                  “I’ve got a bad feeling abou


%                                       recovered from the ruin
% CHEOLOGIST
% ES
% WIL
% 15

% 15



% TS
%  PD
%  1



% LS
% TAL
% 5
% 5
% 5
% 0
% 5
% 5
% 0
% 5
% 0
% 0
% 5
% 5
% 0
% 0
% 5
%  80)
% 5
% 5
% 5
% 5
% 5
% 5          Background: Original Space Colonist
% 5          Faction: Argonaut


%       Morph: Exalt
%  55)


%       Motivations: +Exploration +Research (Alien Civiliza
% 5                        +Techno-Progressivism
% 0          TRAITS                               EQUIPMENT
% 5          Ego: Enemy (Rival Xenoarche-         Armor: Vacsu
% 0             ologist), Mental Disorder           Smartfabri
% 5             (Impulse Control), Mental         Primary Wea


%          Disorder (Insomnia), Morphing     Starting Cred
% 5


%          Disorder, Psi (Level 2) ■         Implants: Ba
% 0                                                 Mesh Inse
% 5          SLEIGHTS                               Echo Loca


%       Psi-Chi: Ambience Sense, Grok,         Mnemonic
% ntist,        Pattern Recognition ■             Gear: Backup
% ng to      Psi-Gamma: Omni Awareness,             (1 month),
% facts         Static, Thought Browse ■            Nanoswar
%  ities                                            Nanoswar
%  lued      REP                                    Forensics
% pedi-      @-rep: 40                              Mobile Lab
% g for      c-rep:     40                          Servitor Bo
% ieve,      e-rep:     20                          Smart Dus

% dis-     i-rep:     40                          Container,
% lable      r-rep:     60


% s. These tools are very similar to the ones we
% n that binary system exoplanet last year.”

% %%% txt/158.txt
% BARSOOMIAN FREE




%                                                      A






%                                                   Inte




% an
% n

%  rship +Martian Liberation +Sousveillance




%               Starting Credit: 1,450 ■
% n (Narcoal-        Enhancements: Access Jacks,
%  d Behavior          Basic Mesh Inserts, Cortical
%  Tendencies) ■       Stack, Cyberbrain, Enhanced
% ma (Clanking         Hearing, Enhanced Smell,           Pro
%  y Valley ■          Enhanced Vision, Lidar,


%                 Mnemonic Augmentation,


%                 Radar, T-Ray Emitter ■


%               Gear: Backup Insurance


%                 (1 month), Covert Ops Tool,


%                 Facial/Image Recognition         Thank


%                 Software, Gnat Bots (3),         ubiqui


%                 Linkstate Narcoalgorithm,        crowd


%                 Microbugs (5), Muse,             report


%                                                  can st


%                 Radio Booster, Repair Spray,


%                                                  tion, h
% 2/2] ■               Saucer Bot, Speck Bots (2),      schoo
% hredder              Smart Dust, Tracking Software,   raking


%                 Trike Exoskeleton ■              and in


%                                                  helped


%                                 “What? No, I’m not rec


%      Anything you have to say about the Hellas massac
%  ANCE JOURNALIST


%      APTITUDES


%          COG      COO       INT     REF      SAV       SOM    WIL


% Base      15       10        20      10       20        15    15
% rph Bonus                                  5                 5


% Total      15       10       20      15       20        20     15




% OX     TT      LUC       IR      WT      DUR      DR


%                                                        STATS


%                                                     INIT      SPD


%    6       30       60       8       40      80    60 (70)     1





%                                                    SKILLS


%                                                MORPH


%                             APT      BASE      BONUS         TOTAL

% Academics: Memetics            COG       55                       55
% emics: Political Science         COG       35                       35
% Academics: Psychology            COG       45                       45


%   Art: Performance          INT       60                       60


%           Art: Writing      INT       50                       50


%            Deception        SAV       50                       50


%                   Fray      REF       40           5           45


%          Freerunning        SOM       45           5           50


%              Gunnery        INT       45                       45


% Hardware: Robotics          COG       55                       55


%            Infiltration      COO       35                       35


%                Infosec      COG       40                       40
% s: Inner System Rumors           COG       55                       55
%  erests: Martian Politics        COG       55                       55
% erests: Post-Fall History        COG       45                       45


%           Interfacing       COG       40                       40


%          Intimidation       SAV       20                       20


%        Investigation        INT       60                       60


%              Kinesics       SAV       60                       60
%  uage: Native Mandarin           INT       90                       90


%  Language: English          INT       40                       40


%    Language: Hindi          INT       40                       40
% etworking: Autonomists           SAV       50                       50

%  Networking: Criminal         SAV       40                       40
% Networking: Hypercorps           SAV       50                       50


% Networking: Media           SAV       60                       60


%           Perception        INT       50                       50


%           Persuasion        SAV       50                       50


%        Pilot: Aircraft      REF       50           5           55


%   Pilot: Groundcraft        REF       20           5           25
%  ion: Social Engineering         COG       50                       50


%              Protocol       SAV       40                       40


%             Research        COG       55                       55


%     Spray Weapons           COO       30                       30


%   Unarmed Combat            SOM       45           5           50
% he mesh, lifelogging, XP, and     the cyberdemocratic facade. You make
%  surveillance, journalism is a    a sport out of bypassing the hypercorp
% ced industry—everyone is a        content filters and initiating subver-
%  ard work, skill, and attitude    sive memes. Equal parts investigator,
%  rn you a name and recogni-       paparazzi, adventurer, and activist, you
% ver. Your committment to old      embrace a gonzo journalist style that
%  estigative journalism and        borders on entertainment. You prefer
% muck that the power brokers       synths because they are innocuous and
% nce peddlers wallow in has        employ numerous drones and electronic
% ose the puppet strings behind     spies to obtain the real story. ■
%  ing.
%  s strictly between you and me.”

% %%% txt/159.txt
% BRINKER GENEHAC


%       APTITU     COG      COO     INT       REF    SAV        SOM


%        Base       20       15      15        15     15         10


% Morph Bonus       10       5       5


%        Total      30       20      20       15     15         10






% MOX TT     LUC       IR                WT    DUR    DR


%                                                              STIN


%  3 8 (10) 40 (50) 80 (100)              7     35    53        60 (






%                                                     MORPH


%                                                          SKI


%                                  APT        BASE    BONUS


%         Academics: Biology       COG         70       10


%       Academics: Genetics        COG         70       10

% Academics: Nanotechnology          COG         60       10


%          Art: Bodysculpting      INT         45       5


%             Beam Weapons         COO         45       5


%                    Deception     SAV         30


%                          Fray    REF         40


%                      Free Fall   REF         55


%      Hardware: Aerospace         COG         40          10

% Interests: Black Market Drugs       COG         40          10

% Interests: Genetics Research       COG         60          10


%   Interests: Morph Designs       COG         55          10


%                   Interfacing    COG         20          10


%                      Kinesics    SAV         40          10


%    Language: Native Arabic       INT         85          5


%           Language: English      INT         40          5


% Medicine: General Practice       COG         55          10


%    Medicine: Gene Therapy        COG         65          10


%   Medicine: Nanomedicine         COG         60          10


%  Medicine: Trauma Surgery        COG         50          10


%   Networking: Autonomists        SAV         35


%       Networking: Criminal       SAV         45


%      Networking: Scientists      SAV         55


%                   Perception     INT         45          5


%            Pilot: Spacecraft     REF         25


% Profession: Lab Technician       COG         50          10


%   Profession: Medical Care       COG         55          10


%                Programming       COG         60          10


%              Psychosurgery       INT         55          5


%                     Research     COG         45          10


%                  Scrounging      INT         50          5

% Some might consider you a mad           There are some who find y
% scientist, but they simply lack         intriguing or valuable, o
% the vision and moral flexibility to     and so you have acquired
% understand the meaning of your          tial backers. In truth, yo
% work. You are not just a scientist—     expert when it comes to d
% you are an artist, dedicated to         and manipulating biomor
% defining the shapes and abilities       so your services are som
% of transhumans as they transition       demand when it comes to
% to the posthuman. Because your          ing unusual and exotic
% work is sometimes controversial,        mations. Your patrons, o
% you prefer the brinker lifestyle,       sometimes call on your e
% working in isolation where you are      from time to time in exch
% not restricted by laws or customs.      bankrolling your work. ■



%                                  “It’s alive! It’s alive! Wai
% KER
% ES
% WIL
% 20
%  5
% 25



% TSSPD

% 1



% LS
% TAL
% 0
% 0
% 0
% 0
% 0
% 0
% 0
% 5
% 0
% 0
% 0
% 5
% 0
% 0
% 0
% 5
% 5
% 5
% 0
% 0
% 5
% 5
% 5
% 0
% 5
% 0          Background: Isolate
% 5          Faction: Brinker
% 0          Morph: Menton
% 0          Motivations: +Artistic Expression (Morph Design)
% 5             +Morphological Freedom +Research (Neogene
% 5          TRAITS                               Starting Cred


%       Ego: Black Mark (Lunars,             Implants: Ac
% work         Level 1), Patron (Gerontocrat),      Biomods, B
% urse,        Psi Defense ■                        Circadian
%  uen-                                             cal Stack,
%  e an      REP                                    Ghostrider
% gning


%       @-rep:    30                           Linguist, M

% and


%       c-rep:    20                           chines, Mu
%  es in
%  lain-     g-rep:    30                         Gear: Backup
%  sfor-     r-rep:    60                           (1 month),
% urse,                                             (5 doses),
%  rtise     EQUIPMENT                              Frequency
%  e for     Armor: Armor Clothing [3/4] ■          Angel Bot,


%       Primary Weapon: Stunner ■              Nanoband

% no—well, it was alive. Let’s try that again.”

% %%% txt/160.txt


%                                                   Aca






%                                                  Inte




% antiated

% d
% e +Subverting Technology +Thrill-Seeking



%              Enhancements: Access Jacks,
% es, On the          Basic Mesh Inserts, Cortical
%  Constella-         Stack, Cyberbrain, Electrical     Prof
% me ■                Sense, Enhanced Vision, Lidar,     Pro


%                Magnetic System, Mental Speed,


%                Mnemonic Augmentation, Nano-


%                scopic Vision, Radar, Swarm


%                Composition, T-Ray Emitter ■


%              Gear: Automech, Backup Insur-


%                ance (1 month), EMP Grenade,     There


%                Exploit Software, Fake Ego       couldn


%                ID, Guardian Bot, Guardians      got yo


%                Nanoswarm, Saboteur              that y
% one ■               Nanoswarm, Servitor Bot,         never
% 0■                  Sniffer Software, Spoof          at circ


%                Software, Tactical Network       defens


%                                                 of dig


%                Software, Tracking Software ■


%                                                 secre



%    “You call this secure? Maybe it would keep me ou
% CRIMINAL HACKER


%      APTITUDES


%           COG      COO      INT      REF      SAV     SOM       WIL


%  Base      20       15       15       10       20      10       15
%  rph Bonus


%  Total     20       15       15       10       20       10       15




% OX      TT     LUC       IR      WT      DUR       DR


%                                                       STATS


%                                                       INIT      SPD


%     6      30       60       6       30       60       50        1





%                                                     SKILLS


%                                                 MORPH


%                             APT      BASE       BONUS        TOTAL
% mics: Computer Science           COG       60                       60
%  ademics: Cryptography           COG       60                       60
% Academics: Engineering           COG       50                       50

%  Art: Electronic Music        INT       45                       45


%      Beam Weapons           COO       40                       40


%             Deception       SAV       55                       55


%                    Fray     REF       40                       40


%               Free Fall     REF       30                       30

% Hardware: Electronics         COG       60                       60


% Hardware: Industrial        COG       45                       45


% Hardware: Robotics          COG       55                       55


%        Impersonation        SAV       45                       45


%             Infiltration     COO       60                       60


%                Infosec      COG       70                       70
%  s: Hacker Mesh Forums           COG       55                       55
%  terests: Online Banking         COG       50                       50
% erests: Triad Economics          COG       45                       45


%            Interfacing      COG       55                       55


%           Intimidation      SAV       30                       30


%               Kinesics      SAV       40                       40
% nguage: Native Russian           INT       85                       85
% etworking: Autonomists           SAV       40                       40

% Networking: Criminals         SAV       60                       60

%  Networking: Firewall         SAV       40                       40
% Networking: Hypercorps           SAV       40                       40


%            Perception       INT       50                       50


%         Pilot: Aircraft     REF       30                       30


%    Pilot: Groundcraft       REF       20                       20
%  Profession: Accounting          COG       50                       50
%  on: Security Operations         COG       55                       55

% ion: Social Engineering        COG       60                       60


%         Programming         COG       60                       60


%              Research       COG       60                       60



%  ever been a system you           of course, but you've never let moral-
%  ack, given time. That's what     ity get in your way. In fact, you make a

% away the first time, but now    good living selling your talents to crimi-
%  ave a second chance you'll       nal groups like the triads and ID Crew.

% aught again. You take pride     You've never been a joiner, though—

% enting firewalls and mesh       you remain strictly freelance. In fact,
% —nothing surpasses the thrill     you'll sell your services to almost

% respassing and accessing        anyone—it's the thrill of the hack that
%  uch intrusions are illegal,      really counts. ■

%  I was a particularly slow child.”

% %%% txt/161.txt
% EXTROPIAN SMUG


%       APTITU     COG      COO       INT    REF    SAV       SOM


%        Base       15       15        15     20     15        15


% Morph Bonus                5         5                       5


%        Total       15      20       20     20      15        20






%  MOX      TT      LUC      IR       WT    DUR     DR


%                                                             ST


%                                                             INIT


%   5        6       30      60        8     40     80         70






%                                                    MORPH


%                                                         SKI


%                                    APT     BASE    BONUS


%     Academics: Accounting          COG      35


%   Academics: Astrophysics          COG      45


%     Academics: Psychology          COG      40


%                        Blades      SOM      45          5


%                    Deception       SAV      50


%                           Fray     REF      50


%                      Free Fall     REF      30


%                      Gunnery       INT      40          5


%       Hardware: Aerospace          COG      50


%                    Infiltration     COO      40          5


%                       Infosec      COG      35


%    Interests: Black Markets        COG      55


%  Interests: Criminal Groups        COG      35

%  Interests: Inner System Law         COG      55


%                      Kinesics      SAV      40


%            Kinetic Weapons         COO      50          5


%   Language: Native Korean          INT      85          5


%          Language: English         INT      45          5


%       Language: Cantonese          INT      40          5


%                   Navigation       INT      30          5


%   Networking: Autonomists          SAV      55


%        Networking: Criminal        SAV      55


%    Networking: Hypercorps          SAV      35


%                      Palming       COO      35          5


%                   Perception       INT      45          5


%                   Persuasion       SAV      60


%                Pilot: Aircraft     REF      45


%           Pilot: Groundcraft       REF      50


%            Pilot: Spacecraft       REF      60


%       Profession: Appraisal        COG      45
% Profession: Customs Procedures          COG      50

% Profession: Smuggling Tricks         COG      60


% Despite the failures of statist capi-     or pirating nanofab blue
% talism, you’re a die-hard believer in     an age of automated m
% free markets—and when markets             you’re a damn good pilot
% aren’t free, you have no qualms           can talk or shoot your w
% about undermining them.                   messy situations. Where

% You make a living supplying the        people look down on synt
% inner system and Jovian black mar-        you embrace the post-biol
% kets, whether that means running          and its freedom from chem
% blockades, smuggling contraband,          biological dependencies.■






%                                    “Sorry about that, but I g
% LER
% ES
% WIL
% 15

% 15



% TS
%  PD
%  1



% LS
% TAL
% 5
% 5
% 0
% 0
% 0
% 0
% 0
% 5
% 0
% 5
% 5
% 5
% 5
% 5
% 0
% 5
% 0
% 0
% 5
% 5
% 5
% 5          Background: Reinstantiated
% 5          Faction: Extropian
% 0          Morph: Slitheroid
% 0          Motivations: +Libertarianism +Subverting Authority
% 0          TRAITS                               Enhancemen
% 5          Ego: Danger Sense, Edited              Vision, Acc
% 0            Memories, Neural Damage              Glare, Bas
% 0            (Repetitive Behavior), On the        Chameleon
% 5            Run (Jovians) ■                      Stack, Cyb
% 0                                                 Vision, Gri


%       REP                                    Compartm
% 0


%       @-rep:   50                            Combat Ar


%       c-rep:   30                            Augmenta

% s. In    g-rep:   70                            tem: Snake
% ines,      i-rep:   50                            Radar, T-R
% d you                                           Gear: Backup
%  ut of     EQUIPMENT                              Dazzler, Fa
% many


%       Armor: Light Combat Armor [14/12]      Bot, Miniat
%  rphs,
% al life      ■                                    Phlo (1 dos
%  l and     Primary Weapon: SMG Firearm            Radio Boos


%         (100 rounds regular ammo) ■          Slip (1 appl


%       Starting Credit: 1,400 ■               ing, Utilitoo


%                                              Noise Mac


%  look out for numero uno. You understand.”

% %%% txt/162.txt


%               HYPERCORP B






%                                                        I






%                                                        L




% acuee


% onal Career +Techno-Progressivism +Wealth



%              Starting Credit: 2,650 ■
% riminal) ■        Implants: Basic Biomods, Basic


%                Mesh Inserts, Bioweave (Light),


%                Cortical Stack, Eelware, Endo-


%                crine Control, Enhanced Vision,


%                Medichines, Nanophages ■


%              Gear: Backup Insurance               The in


%                (1 month), Cleaner Nano-           live, b


%                swarm, Fake Ego ID, Guard-         ties.


%                ian Bot, Muse, Nanobandage,        good


%                Nanodetector, Overload             and c


%                Grenade, Servitor Bot, Shred-      you’r
% ight) + Armor       der (100 shots),                   their


%                Vacsuit (Standard),                of yo
% aser Pulser ■       Tactical Network Software ■        work



%         “I may have that, but it’s very hard to acquire.
%  LACK MARKETEER


%     APTITUDES


%          COG      COO    INT    REF     SAV       SOM     WIL


% Base      15       15     20     15      15        15     10
% rph Bonus                                     5


% Total      15      15     20     15      20        15      10




% OX     TT      LUC      IR    WT     DUR     DR


%                                                   STATS


%                                                   INIT    SPD


%    4       20      40     6      30     45         70      1





%                                               SKILLS


%                                           MORPH


%                          APT     BASE     BONUS          TOTAL
%  Academics: Economics         COG      60                       60

% Academics: Sociology        COG      55                       55


%     Beam Weapons         COO      40                       40


%              Climbing    SOM      35          5            40


%            Deception     SAV      60          5            65


%                   Fray   REF      45                       45


%          Freerunning     SOM      35                       35


%            Infiltration   COO      40                       40


%               Infosec    COG      40                       40
% nterests: Black Markets       COG      60                       60
%  ests: Inner System Law       COG      50                       50

%  Interests: Smuggling      COG      55                       55


%           Interfacing    COG      35                       35


%              Kinesics    SAV      60          5            65
%  uage: Native Portugese       INT      85                       85


%   Language: English      INT      45                       45
% etworking: Autonomists        SAV      45          5            50

% Networking: Criminal       SAV      60          5            65
% Networking: Hypercorps        SAV      60          5            65


%  Networking: Media       SAV      45          5            50


%           Perception     INT      50                       50


%           Persuasion     SAV      60          5            65


%   Pilot: Groundcraft     REF      25                       25

% Profession: Appraisal      COG      60                       60
%  ofession: Con Schemes        COG      55                       55
%  fession: Info Brokerage      COG      50                       50


%              Protocol    SAV      55          5            60


%          Scrounging      INT      40                       40


%     Spray Weapons        COO      35                       35


%    Unarmed Combat        SOM      55                       55




%  system is a nice place to     hackers, smugglers, and criminal
%  has its share of inequali-    cartels on one side and everyone
% eone has to provide the        who needs or craves the restricted
% d services that the poor       and illegal on the other. You keep
% king masses need, and          a step ahead of the law, providing
% ore than willing to meet       what you see as an essential role
%  s—and make some profit         in society despite the official rules
% wn while you’re at it. You     and restrictions that enhance your
%  middle man between the        profit margins. ■

% w much is it worth to you?”

% %%% txt/163.txt
% JOVIAN SPY                                         APTITU


%              COG          COO    INT    REF     SAV    SOM


%        Base 15             15     15     15      15     15


% Morph Bonus


%        Total 15            15     15     15      15     15






%  MOX      TT      LUC      IR    WT     DUR     DR


%                                                        ST


%                                                        INIT


%   3        6       30      60     6      30     45      60





%                                                       SKI


%                                                   MORPH


%                                  APT    BASE      BONUS


%   Academics: Cryptography        COG     60


%      Academics: Linguistics      COG     60


%                        Blades    SOM     45


%                      Climbing    SOM     55


%                         Clubs    SOM     45


%                    Deception     SAV     60


%                  Demolitions     COG     45


%                      Disguise    INT     45


%                           Fray   REF     45


%                      Free Fall   REF     45


%                  Freerunning     SOM     45


%               Impersonation      SAV     45


%                    Infiltration   COO     60


%                       Infosec    COG     60

% Interests: Autonomist Groups       COG     55


%   Interests: Criminal Groups     COG     55

%  Interests: Hypercorp Politics     COG     55


%                      Kinesics    SAV     45


%   Language: Native Spanish       INT     85


%           Language: English      INT     50


%       Networking: Criminals      SAV     40


%      Networking: Ecologists      SAV     45


%     Networking: Hypercorps       SAV     45


%                      Palming     COO     45


%                   Perception     INT     55

% Profession: Security Systems       COG     55

% Profession: Smuggling Tricks       COG     45


%         Profession: Spycraft     COG     55


%              Spray Weapons       COO     25


%            Unarmed Combat        SOM     55




% You’re a dedicated soldier to the      the republic, you have n
% bioconservative cause, convinced       about using technolog
% that unrestricted technology is        you otherwise want rest
% driving the human race to extinc-      banned, and you apprec
% tion. The Jovian Republic has          irony of using the tools
% trained you and modified you to        shuman monsters again
% serve their interests, whether that    You prefer flats, and you
% means infiltrating a hypercorp         it a point of pride that yo
% to steal its secrets or sabotag-       into a non-genetically-
% ing an autonomist habitat. As a        body with no cortical s
% skilled professional in service to     still kick transhuman ass.



%                                           “Not so immort
% ES
% WIL
% 15

% 15



% TS
%  PD
%  (2)



% LS
% TAL
% 0
% 0
% 5
% 5
% 5
% 0
% 5
% 5
% 5
% 5
% 5
% 5
% 0
% 0
% 5
% 5
% 5
% 5
% 5
% 0
% 0
% 5
% 5          Background: Original Space Colonist
% 5          Faction: Jovian
% 5          Morph: Flat
% 5          Motivations: -Anarchism +Bioconservativism -Tec
% 5


%      TRAITS                             Primary Wea
% 5         Ego: Modified Behavior (Blood-        Railgun (10
% 5           thirsty, boosted), Morphing      Starting Cred
% 5           Disorder (Level 1) ■             Implants: Ne


%      Morph: Genetic Defect              Gear: Backup


%        (Heart Disease),                   (1 month),


%        Unattractive (Level 1) ■           10 Frag Gr


%                                           Cloak, Liqu
%  alms


%      REP                                  ture Radio

% that


%      c-rep:   50                          MRDR (1 d
% ed or


%      e-rep:   30                          (1 dose), R
% e the


%      g-rep:   20                          table QE C
% tran-


%                                           Low-Capa
% hem.


%      EQUIPMENT                            3 Speck Bo
% sider


%      Armor: Body Armor (with Ther-        Network S
% eeve


%        mal Dampening) +Second Skin        Knife, Whi
%  ified


%        [14/16] ■

% and


% ow, are we, frankenfreak scum!”

% %%% txt/164.txt


%                                                         L






%                                                        In




%  nal Development +Self-Protection +Vengeance



%               EQUIPMENT                               Pr
% er (Borderline     Armor: Body Armor (Light) [10/10] ■     Pro
% ntal Disorder      Primary Weapon: Agonizer ■
%  ulsive), Psi      Starting Credit: 2,700 ■
% Stigma (Lost),     Implants: Adrenal Boost,


%                 Basic Biomods, Basic Mesh


%                 Inserts, Cortical Stack, Eidetic


%                 Memory, Emotional Dampers,
% Control,             Enhanced Hearing, Enhanced          You s
%  old, Instinct,      Smell, Enhanced Vision,             grow
%  d■                  Medichines, Oracles ■               of th
% can, Ego Sense,    Gear: Backup Insurance                lapse
% bliminal ■           (1 month), Disabler, Facial/        dal a


%                 Image Recognition Software,         You’v


%                 Fake Ego ID, Muse, Shock Ba-        new l


%                 ton, Sniffer Software, Tactical     your p


%                 Network Software, Taggant           use t


%                 Nanoswarm, Track Software,          hunte


%                 Tracker Dye (1 dose), Twitch        ing th


%                 (1 dose) ■                          not on


%                           “Hello, doctor, you’re a har


%              I have some questions about some childr
% UNAR EGO HUNTER


%      APTITUDES


%          COG      COO     INT      REF    SAV        SOM    WIL


% Base      15       20      15       10     10         15    20
% rph Bonus      5                5               5                 10


% Total     20       20      20      10     15         15      30




% OX     TT     LUC       IR           WT DUR DR INIT


%                                                  STATS      SPD


%  8 (12) 40 (60) 80 (120)         7  35 53 50 (70)           1 (2)






%                                            MORPH


%                                                 SKILLS


%                           APT      BASE    BONUS           TOTAL

% Academics: Neurology         COG       55       5               60
% Academics: Psychology          COG       60       5               65


%           Art: Dance      INT       45       5               50


%          Art: Painting    INT       45       5               50


%      Beam Weapons         COO       60                       60


%                  Clubs    SOM       45                       45


%                Control    WIL       50          10           60


%                   Fray    REF       50                       50


%          Freerunning      SOM       55                       55


%            Infiltration    COO       50                       50


%                Infosec    COG       45          5            50
%  Interests: Conspiracies       COG       55          5            60
%  sts: Hypercorp Projects       COG       55          5            60


%    Interests: TITANs      COG       60          5            65


%          Intimidation     SAV       20          5            25


%         Investigation     INT       50          5            55


%               Kinesics    SAV       40          5            45
%  nguage: Native English        INT       85          5            90

%  Language: Japanese         INT       85          5            90


%     Language: Hindi       INT       25          5            30

%  Networking: Criminal       SAV       30          5            35
%  Networking: Ecologists        SAV       40          5            45
% Networking: Hypercorps         SAV       50          5            55


%           Perception      INT       40          5            45
% Profession: Ego Hunting        COG       55          5            60
%  sion: Inner System Law        COG       45          5            50
% sion: Police Procedures        COG       55          5            60


%           Psi Assault     WIL       40          10           50


%             Research      COG       55          5            60


%                 Sense     INT       55          5            60


%    Unarmed Combat         SOM       55                       55

%  ived your accelerated          and sometimes even their memo-
%  xperience as a member          ries and mannerisms. All the while
%  st generation, the col-        you viciously hide your own history
% he project, and the scan-       and out-maneuver those who want
% witchhunts that followed.       to track your kind down. You are
% mmersed yourself in a           not content to be a victim or a mer-
%  nd identity. Now you put       cenary for others, however. Slowly
%  cular skills and talents to    but surely you are amassing infor-
%  ing people as a bounty         mation on those responsible for
% a daunting task consider-       the Lost project—for what they
%  our targets can change         did to you—and some day you will
% heir faces but their bodies     make them pay. ■
% an to find.
% hat I’d like to ask you.”

% %%% txt/165.txt
% MERCURIAL INVES


%       APTITU     COG     COO      INT     REF    SAV     SOM


%        Base       20      10       15      15     20      5


% Morph Bonus


%        Total      20      10       15      15     20       5






%  MOX      TT     LUC      IR      WT      DUR    DR


%                                                         ST


%                                                         INIT


%   4        8      40      80      —        —     —       60





%                                                        SKI


%                                                    MORPH


%                                  APT      BASE     BONUS


% Academics: Anthropology          COG       60
%  Academics: Computer Science          COG       60


%        Academics: Physics        COG       50


%              Art: Digital Art    INT       50


%                   Deception      SAV       40


%                          Fray    REF       35


%                     Free Fall    REF       30


%                     Gunnery      INT       50


%              Impersonation       SAV       45


%                   Infiltration    COO       50


%                      Infosec     COG       60


%    Interests: Crime Groups       COG       60


%  Interests: Cultural Trends      COG       45


%                  Interfacing     COG       50


%               Investigation      INT       60


%                     Kinesics     SAV       50


%  Language: Native English        INT       90


%      Networking: Criminals       SAV       50


%       Networking: Firewall       SAV       40


%   Networking: Hypercorps         SAV       40


%                  Perception      INT       45


%               Pilot: Aircraft    REF       55


%           Pilot: Anthroform      REF       55


%          Pilot: Groundcraft      REF       30


%      Profession: Forensics       COG       60
%  Profession: Police Procedures        COG       50

% Profession: Security Systems        COG       45


%               Programming        COG       60


%                    Research      COG       60




% You are a digital life form, coded      at sifting, sorting, and co
% to be “friendly” and molded with        this data, in fact, selling
% transhuman mindsets and world           vices as a mesh-based in
% views. You resent the backlash          tor. For most of your inq
% against AGIs and criticize resur-       physical form isn’t nece
% gent human tendencies towards           you acquire data on th
% technophobia and xenophobia             cal world through senso
% as harmful to the emerging tran-        physical interaction i
% shuman society. You immerse             for, however, you can
% yourself fully in transhuman cul-       or jam a bot or catch a
% ture and the data it produces,          ghostrider module or, a
% bathing in its richness. You excel      resort, in a meat puppet.



%                                 “Oh this is great, I think I f
% TIGATOR
%  ES
% WIL
% 20

%  20



% TS
%  PD
%  (3)



% LS
% TAL
% 0
% 0
% 0
% 0
% 0
% 5
% 0
% 0
% 5
% 0
% 0
% 0
% 5
% 0
% 0
% 0
% 0
% 0
% 0
% 0
% 5
% 5
% 5
% 0
% 0
% 0           Background: Infolife
% 5           Faction: Mercurial
% 0           Morph: Infomorph


%        Motivations: +AGI Rights +Personal Development +
% 0



%        TRAITS                                   Gear: AR Illus


%        Ego: Real World Naiveté, Social            Backup In


%          Stigma (AGI) ■                           Covert Ops
% ating


%                                                   Exploit Sof

% ser-


%        REP                                        age Recog
%  tiga-


%        c-rep:    20                               2 Gnat Bot
% es, a


%        g-rep:    10                               Holograph
%  y, as


%        i-rep:    20                               Narcoalgo
% hysi-


%                                                   Scout Nan
% When


%        EQUIPMENT                                  tor Bot, Sm
% alled


%        Armor: None [0/0] ■                        Software,
%  rate


%        Primary Weapon: None ■                     Software,

% in a


%        Starting Credit: 250 ■                     Software,

% last


%                                                   5 XP Clips


%  d it! What is a ‘red light district?’ Oh, I see.”

% %%% txt/166.txt


%                                              MER




%                                                        Aca






%                                                          In






%                                                          L



% (Octopus)


% ration +Reclaiming Earth +Uplift Rights



%                  Implants: Basic Biomods, Basic
%  x 2) ■                 Mesh Inserts, Chameleon
% el 2) ■                 Skin, Cortical Stack, Direction


%                    Sense, Echolocation, Electrical


%                    Sense, Enhanced Vision, Grip


%                    Pads, Medichines, Oracles,


%                    Radiation Sense ■


%                  Gear: Backup Insurance


%                    (1 month), Breadcrumb Posi-


%                    tioning System, Disassembly       You m


%                    Tools, Mobile Lab, Muse, Nano-    it har


%                    detector, Radio Booster, Shel-    modif
% /4 or 4/6] with         ter dome, Specimen Container,     have
% ng ■                    Superthermite Charge, Tactical    limbs
% netic Pistol            Network Software, Vacsuit         and b


%                                                      stant
% tandard                 (Standard), X-Ray Emitter ■


%                                                      did by


%                  Advantages: 8 Arms, Beak Attack


%                                                      rende
% ■                       (1d10 DV, use Unarmed Combat      ceede


%                    skill), Ink Attack (blinding),    ing th


%                    360-degree Vision ■               Neve


%              “Maybe it’s the kind of trap that would catch


%                 but my superior physiology was able to
% CURIAL SCAVENGER


%       APTITUDES


%          COG      COO     INT     REF     SAV       SOM    WIL


% Base      10       20      15      15      10        20    15
% rph Bonus      5        5       5


% Total     15       25      20      15      10       20      15




% OX     TT      LUC      IR     WT     DUR      DR


%                                                     STATS


%                                                  INIT      SPD


%    6       30      60      6      30      45    60 (70)     1





%                                                 SKILLS


%                                             MORPH


%                           APT     BASE      BONUS         TOTAL
% Academics: Engineering         COG      50         5             55
% mics: Materials Science        COG      50         5             55


% Academics: Physics        COG      50         5             55


%     Art: Ink Painting     INT      45         5             50


%              Climbing     SOM      30                       30


%          Demolitions      COG      40           5           45
%  otic Ranged: Ink Attack       COO      35           5           40


%              Free Fall    REF      45                       45

% Hardware: Electronics        COG      40           5           45

%  Hardware: Industrial       COG      40           5           45


%            Infiltration    COO      50           5           55
% nterests: Post-Fall Earth      COG      40           5           45
%  erests: Ruined Habitats       COG      55           5           60
% ests: Spaceship Models         COG      40           5           45


%           Interfacing     COG      30           5           35


%        Investigation      INT      35           5           40


%   Kinetic Weapons         COO      50           5           55
%  uage: Native Japanese         INT      80           5           85
% etworking: Autonomists         SAV      30                       30

%  Networking: Criminal       SAV      40                       40
%  Networking: Ecologists        SAV      40                       40

%  Networking: Firewall       SAV      30                       30
% Networking: Hypercorps         SAV      30                       30


%              Palming      COO      40           5           45


%           Perception      INT      45           5           50


%        Pilot: Aircraft    REF      35                       35


%    Pilot: Spacecraft      REF      45                       45

% Profession: Appraisal       COG      40           5           45
%  rofession: Salvage Ops        COG      60           5           65


%          Scrounging       INT      60           5           65


%            Swimming       SOM      50                       50


%   Unarmed Combat          SOM      50                       50

%  e an “uplift,” but you find    and derelict habitats left behind
% ot to feel sorry for these      by transhumanity to be fascinating
% monkeys around you who          to explore, so you spend much of
%  et by with half as many        your time combing through space-
% ping sensory blind spots,       craft hulls and shattered stations,
%  e bones that are con-          looking for curiosities and lost trea-
%  eaking. Any favors they        sures. Such activities tend to take
%  ifting octopus-kind were       you close to Earth as well, where

% moot when they suc-           you support the efforts of those who

% despoiling and abandon-       hope to take the planet back. Your
%  me planet you all shared.      ultimate dream is to someday swim
%  ess, you find all the ruins    in the oceans of your ancestors. ■
% knuckle-dragging monkey,
%  ily squeeze out of it.”

% %%% txt/167.txt
% SCUM ENFORCER


%      APTITU      COG     COO       INT   REF    SAV        SOM


%        Base       10      20        15    20     10         20


% Morph Bonus               5        5      5                 10


%        Total      10      25       20     25     10         30






%  MOX      TT      LUC      IR       WT DUR      DR


%                                                           ST


%                                                        INIT


%   5      4 (6)   20 (30) 40 (60)    10  50      75    70 (90





%                                                       SKI


%                                                   MORPH


%                                 APT      BASE     BONUS


%    Academics: Linguistics       COG       50


%                   Art: Dance    INT       55          5


%             Beam Weapons        COO       40          5


%                       Blades    SOM       60          10


%                     Climbing    SOM       40          10


%                         Clubs   SOM       50          10


%                          Fray   REF       70          5


%                     Free Fall   REF       60          5


%         Hardware: Armorer       COG       40


%                   Infiltration   COO       50          5


%    Interests: Drug Dealers      COG       50


%         Interests: Gambling     COG       40


%     Interests: Scum Gangs       COG       50


%     Interests: Triad Politics   COG       50


%                 Intimidation    SAV       60


%                     Kinesics    SAV       40


%            Kinetic Weapons      COO       60          5


% Language: Native Spanish        INT       80          5


%      Language: Cantonese        INT       55          5


%  Networking: Autonomists        SAV       50


%      Networking: Criminals      SAV       50


%                  Perception     INT       45          5


%                  Persuasion     SAV       20


%  Profession: Bodyguarding       COG       40
%  Profession: Protection Rackets      COG       50


%   Profession: Security Ops      COG       50


%                  Scrounging     INT       25          5


%           Unarmed Combat        SOM       60          10




% Those jokers in the inner               the universe end. Rig
% system just don’t get it—the            you get your kicks with
% future is about taking life by the      either the violent or kink
% enhanced reproductive organs            sometimes both. You th
% and squeezing every juicy drop          tough chick who will sm
% out. You live the life you want to      hell out of anyone who
% live, doing whatever you like to        ens your friends or
% your bodies and mind, and you           during the day, but the
% plan to enjoy the hell out of it        is taking home a differe
% until maybe you get to watch            ner every night. ■


%                                           “Oi! Pretty boy!


%                                          then you’re never
% ES
% WIL
% 10
%  5
% 15



% TS
%  PD
%  (2)



% LS
% TAL
% 0
% 0
% 5
% 0
% 0
% 0
% 5
% 5
% 0
% 5
% 0
% 0
% 0
% 0
% 0
% 0
% 5
% 5
% 0
% 0
% 0
% 0
% 0
% 0
% 0


%       Background: Scumborn
% 0          Faction: Scum
% 0          Morph: Fury
% 0          Motivations: +Adventure +Hedonism +Morphologic



%       TRAITS                             Implants: Ba


%       Ego: Minor Addiction (Alcohol) ■     Mesh Inse


%                                            Armor (Lig


%       REP                                  Cyberclaw


%       @-rep:   60                          Medichine
% now,       c-rep:   40                          (Level 1), T
% tion,                                         Gear: Backup
% nd—        EQUIPMENT                            (1 month),
%  as a      Armor: Bioweave Armor (Light)        (2 doses),
% k the        + Body Armor (Heavy) with          100 rounds
%  eat-        Offensive Armor and Shock          Monofilam
% ents         Proof mods [15/16] ■               Nanoband
% l fun      Primary Weapon: Kinetic Pistol       Gloves, Ta
% part-        (100 rounds regular ammo) ■        Software,


%       Starting Credit: 1,000 ■

%  at’s the best punch you can throw,
% nna get to see the rest of my mods!”

% %%% txt/168.txt
% elite

% d
% nism +Hypercapitalism +Personal Career



%             Implants: Basic Biomods, Basic
% on ■               Mesh Inserts, Claws, Clean
% ma (Pod) ■         Metabolism, Cortical Stack,


%               Cyberbrain, Enhanced Hearing,


%               Enhanced Pheromones, Im-


%                                                You a


%               planted Nanotoxins (Necrosis),


%                                                cliqu


%               Mnemonic Augmentation, Sex


%                                                the in


%               Switch, Skinflex ■


%                                                cializ


%             Gear: Backup Insurance


%                                                panio


%               (1 month), Hither (1 dose),


%                                                enter


%               Muse, Orbital Hash (2 doses),


%                                                your
%  [1/3] ■           Servitor Bot, Smart Clothing,


%                                                You a
%  aws ■             Smart Rats (2), XP Clips (5) ■


%                                                face
%  0■


%                                                and a



%              “I’m just here to make sure you have th
% SOCIALITE ESCORT


%      APTITUDES


%          COG      COO    INT    REF     SAV       SOM    WIL


% Base      15       10     15     10      20        15    20
% rph Bonus               5      5               5


% Total      15      15     20     10      25       15      20




% OX     TT      LUC      IR    WT     DUR     DR


%                                                   STATS


%                                                INIT      SPD


%    8       40      80     6      30     45    50 (70)     1





%                                               SKILLS


%                                           MORPH


%                          APT     BASE     BONUS         TOTAL
%  ademics: Anthropology        COG      55                      55


%      Animal Handling     SAV      45          5           50


%         Art: Dancing     INT      60          5           65
%  rt: Erotic Entertainment     INT      70          5           75


%          Art: Singing    INT      60          5           65


%      Beam Weapons        COO      35          5           40


%            Deception     SAV      50          5           55


%              Disguise    INT      45          5           50


%                   Fray   REF      40                      40


%       Impersonation      SAV      40          5           45


%            Infiltration   COO      40                      40


%               Infosec    COG      35          5           40


% Interests: Art History   COG      60                      60
%  erests: Celebrity Gossip     COG      55                      55
%  nterests: Social Cliques     COG      65                      65


%              Kinesics    SAV      60          5           65
% anguage: Native French        INT      85                      85


% Language: Mandarin       INT      45                      50

%  Language: Japanese        INT      45                      50
% etworking: Autonomists        SAV      45          5           50


% Networking: Firewall     SAV      30          5           35
% Networking: Hypercorps        SAV      50          5           55


%   Networking: Media      SAV      60          5           65


%              Palming     COO      30          5           35


%           Perception     INT      35          5           40


%           Persuasion     SAV      60          5           65

%  Profession: Escorting     COG      55                      55


%              Protocol    SAV      60          5           65


%     Unarmed Combat       SOM      60                      60




%  mmersed in the social         client discreet protection from
%  nd glitterati lifestyle of    plotting rivals thanks to some

% system elites. You spe-      judicious bodyguard training and

% being the ideal com-        non-standard modifications. Given
%  ffering witty banter and      the constantly shifting allegiances
% ment and providing for all     and manipulations of the bored
% nt’s wishes and desires.       and undying rich, you have even
% more than just a pretty        on occasion used your skills to get
% d good time, however,          close to a client in order to elimi-
% capable of offering your       nate them for a rival. ■

%  ght of your life!”

% %%% txt/169.txt
% TITANIAN EXPLOR


%        APTITU    COG     COO       INT    REF     SAV        SOM


%        Base       15      15        20     15      15         20


% Morph Bonus               5                 5                 10


%        Total      15      20        20     20      15         30






%  MOX      TT      LUC      IR        WT    DUR     DR INIT


%                                                             ST


%   4      4 (6)   20 (30) 40 (60)      8     40     60 70 (80





%                                                         SKI


%                                                     MORPH


%                                   APT      BASE     BONUS

%  Academics: Astrosociology          COG       60


%  Academics: Astrozoology          COG       55


%        Academics: Botany          COG       45

%  Academics: Xenolinguistics         COG       60


%                     Climbing      SOM       45          10


%                          Fray     REF       50          5


%                     Free Fall     REF       40          5


%                 Freerunning       SOM       60          10


%    Hardware: Groundcraft          COG       45


%                   Infiltration     COO       50          5


%    Interests: Gatecrashing        COG       60


%     Interests: Sci-Fi Aliens      COG       45


%               Investigation       INT       35


%                     Kinesics      SAV       30


%           Kinetic Weapons         COO       45          5


% Language: Native Swedish          INT       85


%         Language: English         INT       40


%      Medicine: Paramedic          COG       40


%                  Navigation       INT       60


%  Networking: Autonomists          SAV       55


%       Networking: Firewall        SAV       45


%     Networking: Scientists        SAV       55


%                  Perception       INT       60


%          Pilot: Groundcraft       REF       45          5


%           Pilot: Spacecraft       REF       35          5


%   Profession: First Contact       COG       60


%      Profession: Surveying        COG       45

% Profession: Smuggling Tricks        COG       60


%                     Protocol      SAV       50


%                 Scrounging        INT       60




% The gates have opened a new
% frontier to transhumanity and you        preserved, you support c
% are ready to step through and face       and expanding transhu
% the challenges such opportunities        presence while maintainin
% bring. You are a professional gate-      mal impact on alien eco
% crasher, eager to experience new         You are also trained in Firs
% worlds first-hand, despite the dan-       scenarios and are hopefu
% gers—or even because of them.            ing new intelligent life—
% Unlike those who feel that new           without sparking some
% planets should be protected and          deadly interstellar inciden


%                                 “I just finished my analysis


%                                          to what the Boyle-C
% R
% ES
% WIL
% 10
%  5
% 15



% TS
%  PD
%  1



% LS
% TAL
% 0
% 5
% 5
% 0
% 5
% 5
% 5
% 0
% 5
% 5
% 0
% 5
% 5
% 0
% 0
% 5
% 0
% 0
% 0
% 5
% 5
% 5
% 0
% 0
% 0          Background: Drifter
% 0          Faction: Titanian
% 5          Morph: Olympian
% 0          Motivations: +Alien Contact +Exploration +Nano-Ec
% 0


%       TRAITS                               Implants: Ad
% 0


%       None ■                                 Biomods, B


%                                              Cortical St


%       REP                                    Sense, Enh


%       @-rep:    50                           Medichine


%       i-rep:    20                           Temperatu


%       r-rep:    30                         Gear: Backup
%  izing                                            Breadcrum
% nity’s     EQUIPMENT                              tem, Diamo
% mini-      Armor: Vacsuit (Standard Smart-        ics Net, Ele
%  ems.        fabric with Immunogenic              Portable Li
% ntact        System) [7/7] ■                      System, Ra
%  find-      Primary Weapon: Kinetic Assault        Bot, Shelte
% efully       Rifle (100 rounds regular ammo) ■     Container,
%  d of      Starting Credit: 4,150 ■               Climber, Ta


%                                              ware, Trac
% he xenolife amino acids and it’s nothing close
% s hypothesis suggested it should be.”

% %%% txt/170.txt


%                                                       A




% Colonist


% rtality +Individualism +Personal Development



%              Implants: Basic Biomods, Basic


%              Mesh Inserts, Circadian Regula-      You a
% lley ■            tion, Clean Metabolism, Cortical     embr


%              Stack, Eidetic Memory, Enhanced      your


%              Respiration, Enhanced Vision,        decry


%              Hand Laser, Medichines,              the in


%              Neurachem (Level 1), Tem-            tivism


%              perature Tolerance, Toxin Filters,   mists
% (Heavy) with      T-Ray Emitter ■                      to tak
% g [16/13] ■       Gear: Backup Insurance               kill e
% ailgun SMG           (1 month), HE Grenades (10),      own
% ar ammo, 100         Muse, Particle Beam Bolter,       you c
% )■                   Tactical Network Software,        rienc
% 0■                   Vibroblade ■                      sona



%                        “Your beliefs blind you to your


%   ULTIMATEAPTITUDES


%            MERC


%          COG      COO    INT    REF    SAV        SOM    WIL


% Base      10       20     15     20     10         15    15
% rph Bonus      10       5             5       5         10


% Total     20       25     15     25     15         25     15




% OX     TT      LUC      IR    WT     DUR    DR


%                                               STATS


%                                               INIT       SPD


%    6       30      60     8      40    60    70 (80)     1 (2)





%                                              SKILLS


%                                          MORPH


%                          APT    BASE     BONUS          TOTAL
%  emics: Military Science      COG     60        10             70
%  Academics: Philosophy        COG     60        10             70


%     Beam Weapons         COO     60        5              65


%              Climbing    SOM     35        10             45


%                 Clubs    SOM     50        10             60


%                   Fray   REF     65        5              70


%              Free Fall   REF     50        5              55


%          Freerunning     SOM     45        10             55


%              Gunnery     INT     45                       45


% Hardware: Armorer        COG     50                       60


%            Infiltration   COO     50          5            55


% Interests: Literature    COG     50          10           60
%  terests: Military History    COG     55          10           65


%          Intimidation    SAV     35          5            40


%    Kinetic Weapons       COO     60          5            65

% nguage: Native Turkish      INT     80                       80


%   Language: English      INT     60                       60


%  Language: German        INT     45                       45

% Medicine: Paramedic        COG     40          10           50
% Networking: Hypercorps        SAV     50          5            55


%           Perception     INT     50                       50


%   Pilot: Groundcraft     REF     30          5            35
% Profession: Military Ops      COG     55          10           65
%  rofession: Security Ops      COG     50          10           60
%  ession: Squad Logistics      COG     55          10           65


%    Seeker Weapons        COO     60          5            65


%     Spray Weapons        COO     35          5            40


%          Scrounging      INT     25          5            30


% Throwing Weapons         COO     40          5            45


%    Unarmed Combat        SOM     60          10           70


%  a warrior-philosopher,        on to the next. You mostly find
%  g an ascetic lifestyle for    employment in the inner system,
% n personal growth. You         where various social cliques and

% hedonism and greed of        hypercorps favor ultimates like
%  system and the collec-        yourself, knowing they are less
% d anarchy of the autono-       likely to be tempted or subverted
%  you’re more than willing      by rivals. In the end it doesn’t
%  eir pay so that they may      matter who pays the credit; you’ll

% other. You follow your       take from them, learn from them,
% h, however, and when           and be here long after they’ve
% e to learn from an expe-       destroyed themselves with their
%  increase your own per-        petty intrigues and flawed ideolo-
% pabilities you will move       gies. ■

%  e potential.”

% %%% txt/171.txt
% VENUSIAN NEGOTIA


%       APTITU     COG      COO       INT    REF     SAV        SOM


%        Base       15       15        15     10      20         15


% Morph Bonus                5                        10


%        Total       15      20        15     10      30        15






% MOX TT           LUC        IR        WT    DUR     DR


%                                                           ST  INIT


%  5  6 (8)       30 (40)   60 (80)      7     35     53         50





%                                                          SKI


%                                                      MORPH


%                                     APT     BASE     BONUS


%       Academics: Memetics           COG      70


%     Academics: Psychology           COG      60


%       Academics: Sociology          COG      60


%                 Art: Rhetoric       INT      55


%                  Art: Speech        INT      60


%                    Deception        SAV      60          10


%                          Fray       REF      25


%   Interests: Cultural Memes         COG      60


%       Interests: Hypercorps         COG      50
%  Interests: Inner System Politics        COG      60


%                  Intimidation       SAV      50          10


%                 Investigation       INT      55


%                      Kinesics       SAV      70          10


%    Language: Native Arabic          INT      85


%         Networking: Firewall        SAV      50          10


%     Networking: Hypercorps          SAV      50          10


%          Networking: Media          SAV      60          10


%                   Perception        INT      45


%                   Persuasion        SAV      50          10


%                Pilot: Aircraft      REF      20


%           Pilot: Groundcraft        REF      20

%  Profession: Culture Jamming          COG      50


%      Profession: Media Ops          COG      60


%     Profession: Spin Control        COG      55


%                      Protocol       SAV      60          10


%               Psychosurgery         COG      45


%                    Research         COG      45




% You are known as a communica-             a frenzy. Your social man
% tor and deal-maker, but you are           skills work even better
% perhaps best described as a social        face, where you can run
% engineer. In an age of mimetic            cal circles around oppon
% skirmishes, you excel in shaping          scan body language and
% policy and public opinion. A night-       pressions to spot the slig
% mare combination of marketing             of untruthfulness or decep
% agent and political officer, you are       excel at fostering dissen
% adept at media relations, spin con-       fragmenting loyalties, u
% trol, suppressing dangerous ideas,        getting others to do exac
% psychological warfare, ideological        you want while convinc
% purity, and whipping a crowd into         it’s in their own best inter



%                “It’s time we discussed these rumors of
% TOR
% ES
% WIL
% 15
%  5
% 20



% TS
% SPD
%  1



% LS
% TAL
% 0
% 0
% 0
% 5
% 0
% 0
% 5
% 0
% 0
% 0
% 0
% 5
% 0
% 5
% 0
% 0
% 0
% 5
% 0
% 0
% 0
% 0
% 0
% 5
% 0
% 5
% 5


%       Background: Reinstantiated


%       Faction: Venusian


%       Morph: Sylph


%       Motivations: +Fame +Personal Career +Venusian S



%       TRAITS                              Implants: Ba
% ation      Ego: Common Sense ■                   Basic Mes
% e-to-      Morph: Striking Looks (Level 1) ■     Metabolism
% etori-                                           Endocrine

% and     REP                                   Pheromon
%  oex-      c-rep:    80                          Mnemonic
% t hint     f-rep:    60                          Nanophag
%  . You     i-rep:    60                        Gear: Backup

% and                                            (1 month),
% ately      EQUIPMENT                             Recognitio
% what       Armor: Armor Clothing [3/4] ■         Bots, Mus
% them       Primary Weapon: None ■                Specs, Tra
%  ■         Starting Credit: 5,700 ■

%  netary Consortium interference in Morningstar’s affairs.”

% %%% txt/172.txt
% LLS


%                         APTITUDES


%     These are ingrained abilities that ever


%  character has, to varying degrees. ■ p. 172



%                LEARNED SKILLS


%     The most important part of a characte


%   these represent acquired knowledge tha


%    is carried with them, even if they switch


%                           morphs. ■ p. 172




% 6


%     Knowledge Skills: Things that you know. ■ p.


%                                                    172



%     Active Skills: Things that you know how to do.


%                                              ■ p. 172




% Combat   Mental     Physical     Social     Technical     Vehicle




%           Psi (You’ll need the Psi Trait to use Psi skills!)

% %%% txt/173.txt


%                   SKILL


%                            E






%             APTITUD


%  In rare cases, a test may b


%    particular skill applies.


%                aptitudes ar




% LANGUAGES
% Many languages are spoken thr
% mesh makes instant translation
% people are still versed in two or
% ST AND NECESSARY SKILLS
% character will want a few key skills, and there’s
%  list of every skill to reference, as well. ■ p. 176




% ONLY TESTS
% alled for in which no
% is case, one or more
% ed instead. ■ p. 174




% hout the solar system. The
% sy, but despite this, many
%  e languages. ■ p. 181

% %%% txt/174.txt
% LLS■SKILLS■SKILLS■SKILLS■SKILLS■S



% S
%  In a setting where physical looks and capabilities are
%  easily changed at the push of a button, who you are and
%  what you know is more important than any inborn abil-
%  ity. Skills represent the knowledge your character has,
%  the accumulated set of experience, education, and in-
%  herent know-how possessed by each and every sentient
%  transhuman in Eclipse Phase. They are what allow you
%  to sneak into a hypercorp station, disable the security
%  systems, hack the mesh hub, and then impersonate secu-
%  rity personnel to make your escape. Your skills represent
%  the one thing you have no matter what you look like or
%  where you find yourself. When your characters explore
%  what they can do, their skills, or lack thereof, often de-
%  termine the margin between success and failure.

%  Having a well-rounded set of skills is vital to survival
%  and success in Eclipse Phase. The skills below encompass
%  a wide selection of talents, enough so that each character
%  can be unique in their abilities and knowledge.



%  SKILLS OVERVIEW
%  Skills are divided into aptitudes and learned skills (see
%  Character Skills, p. 123). Most (but not all) learned
%  skills are built on and linked to an aptitude. If a
%  character lacks the specific skill needed in a situation,
%  they may default to the linked aptitude. You may also
%  choose to specialize in certain skills (see Specializa-
%  tions, p. 123), reflecting an enhanced knowledge of a
%  particular aspect of a certain skill.

%  CORE SKILLS: APTITUDES
%  Aptitudes represent inherent skills and abilities acquired
%  at birth or during the course of growing up. Aptitudes
%  are sometimes used for tests, but their primary use is
%  determining the starting point at which learned skills
%  are developed. Aptitudes determine the starting value
%  of their linked skills. For example, a character with
%  Somatics aptitude 10 who wishes to purchase points
%  in the Freerunning skill (which is linked to Somatics)
%  would start with a Freerunning rank of 10 and then
%  buy additionally points in that skill.

% Aptitudes are also used when a character doesn’t
%  posses knowledge of a needed skill (see Defaulting, p.
%  116). Aptitudes represent the basic knowledge that a
%  character has acquired regarding rudimentary use of that
%  skill. They may not have ever received any formal train-
%  ing with the skill, but they can still attempt to use it.

% Aptitudes range in value from 1 to 30, with 10 being
%  the unaugmented human average and 15 representing
%  the average of most genetically modified transhumans.
%  Since aptitudes represent untrained ability, they are
%  capped at a maximum rating of 30.

% There are seven different aptitudes that all players
%  possess. These aptitudes are purchased during charac-
%  ter creation (p. 128), but depending on the morph the
% LLS■SKILLS■SKILLS■SKILLS■SKILLS■S


%                     SKILLS■


%                     SKILLS■S



%                                                             6


%  character is currently inhabiting, they may find their ap-
%  titudes capped by the quality of the morph (see p. 124).

%  LEARNED SKILLS
%  A player’s learned skills are the most important part of
%  their character, representing the acquired knowledge
%  they carry with them from morph to morph, knowl-
%  edge that plays a fundamental role in helping define
%  the person’s ego. Learned skills encompass nearly any
%  skill that you might need to use in Eclipse Phase, and
%  they range in value from 0 to 99.

% All learned skills have a linked aptitude that is
%  used to calculate their initial value, and which is
%  also defaulted to if the player does not have that
%  particular skill.

%  SKILL CATEGORIES
%  Each learned skill is classified as either an Active skill
%  or a Knowledge skill. Active skills represent skills
%  that typically require physical actions and are used in
%  action scenes within game play. Knowledge skills are
%  more knowledge-based and intellectual, representing
%  ideas and facts. Knowledge skills may play a less dra-
%  matic role in certain action-oriented game play mo-
%  ments, but they flesh out the character’s background
%  and interests and are integral to roleplaying interac-
%  tions. Active and Knowledge skills are purchased
%  separately during character creation.

% Active skills are further divided into Combat,
%  Mental, Physical, Psi, Social, Technical, and Vehicle
%  skills. Certain traits and abilities may apply to specific
%  categories.

%  FIELD SKILLS
%  Some learned skills are field skills, meaning that when
%  this skill is chosen a particular field of emphasis must
%  also be selected. For example, the skill of Academics
%  requires the character to specify a specific academic
%  discipline in which they are knowledgeable, such as
%  Biology, Chemistry, or Xenosociology. Field skills are
%  written as “[skill]: [field];” for example: “Art: Paint-
%  ing.” Field skills can be taken multiple times, choosing
%  a different area of emphasis each time, reflecting skills
%  in different fields; that is to say, each field is a separate
%  skill. Several suggested fields are listed for each field
%  skill, but gamemasters and players may also cooperate
%  to create others that fit their games.

%  Field skills may also have specializations; for exam-
%  ple, Professional: Accounting (Money Laundering).

%  PSI SKILLS
%  Psi refers to the ability to perceive and manipulate
%  biological minds via psi waves and/or other inexpli-
%  cable phenomena. Due to the uniqueness of this ability,
%  characters that wish to wield psi must acquire the Psi

% %%% txt/175.txt
% trait (p. 147). Psi use also requires a number of special-
% ized skills (Control, Psi Assault, and Sense) that reflect
% special training characters acquire to tap into their psi
% powers. Psi skills may not be defaulted on; the only
% way to use a psi skill is to possess the trait along with
% training in that skill. For more details, see Psi, p. 220.

% SPECIALIZATIONS
% Any character may opt to specialize in a given skill
% (see Specializations, p. 123). This specialization reflects
% increased knowledge in one particular aspect of the
% skill. Many of the skills offered below include sample
% specializations. Gamemasters and players are encour-
% aged to develop other specialization ideas together for
% their campaigns.

% Specialization provides a +10 modifier when
% using that skill in a situation appropriate to that
% specialization.



% USING SKILLS
% Whenever a character wants to do something using
% a skill, they must succeed at a skill test (see Making
% Tests, p. 115). The difficulty of the action is applied
% as a modifier, as are any other extenuating circum-
% stances that may affect the test (see Difficulty and
% Modifiers, p. 115). As with other types of tests, all skill
% tests are successful when the character rolls less than
% or equal to the test’s target number after any modi-
% fiers have been applied. In the case of skill tests, the
% target number is the character’s skill rating with that
% particular skill. Modifiers representing difficulty and
% other factors are applied directly to the target number
% (see Difficulty and Modifiers, p. 115). A roll of a 00 is
% always a success, regardless of modifiers, and a result
% of 99 is always a failure, again despite any modifiers
% that may increase a character’s target number over
% 100. Standard critical success and failure rules apply
% to skill tests (see Criticals: Rolling Doubles, p. 116),
% so any time a character rolls a double (i.e. 00, 11, 22,
% 33, etc.) they score a critical success or failure.

% DEFAULTING
% Sometimes you lack the skill needed in a certain situ-
% ation. In these instances, characters may default their
% skill test to the linked aptitude. This reflects the fact
% that most learned skills are developed from some sort
% of baseline physical ability. Even though you may not
% know how to do something, you’ve likely seen how
% it’s done at some point or have some idea of how to
% do it, or can at least take a shot at it. Naturally, you’re
% not as good as someone who has training in that skill,
% but it still allows you to make an attempt.

% Not all skills can be defaulted. Some skills are
% simply too complex or obscure, or demand special
% knowledge or ability, for someone to attempt their
% use untrained. For example, brain surgery or most psi
% skills are simply beyond anyone who doesn’t have that
% ability or the knowledge of what they’re attempting.
% DEFAULTING TO FIELD SKILLS
% In some cases, a character may not possess the par-
% ticular field skill that a test calls for, but they may be
% skilled in another related field. For example, a test to
% conduct an alien autopsy might call for an Academics:
% Xenobiology roll, but a character who doesn’t have
% that skill may be allowed to default to Academics:
% Biology instead. The gamemaster decides if and when
% to allow this, perhaps applying a modifier to the test
% based on the difference between fields.

% DEFAULTING TO RELATED SKILLS
% If the gamemaster allows it, characters may default to
% a related skill that also has some relevance to the test
% at hand. For example, a character skilled in Kinetic
% Weapons might not be trained in the use of a laser, but
% they know enough to point at the target and pull the
% trigger. Likewise, a character might not be skilled in
% Investigation, but the gamemaster could still allow
% them to use their Perception skill instead in order to
% realize that a body had been moved from the place
% where it had been shot. In situations like this, when the
% gamemaster allows defaulting to a related skill, a –30
% modifier should be applied to the test.



% Srit is wandering through a black market souk on

% Mars, trying to find a particular piece of sensory

% equipment. The gamemaster calls for a Scrounging


%                                                        EXAMPLE





% Test, but Srit does not have that skill. She could

% default her INT of 22, but instead she asks the

% gamemaster if she can default to the related skill

% of Perception, which she has at 82. The gamemas-

% ter agrees, and so Srit rolls against a target number

% of 52 (82 – 30).


% COMPLEMENTARY SKILLS
% Sometimes more than one skill may apply to a particu-
% lar test, or knowledge in one area can aid your skill
% in another. In this case, the gamemaster may apply a
% modifier to the skill test based on the strength of the
% complementing skill, as noted on the Complementary
% Skill Bonus table.



% Dav is hoping to persuade a brinker pilot to take him

% to an isolated habitat that doesn’t welcome visitors.

% To impress upon the pilot that he is a friend of these


%                                                        EXAMPLE





% particular isolates, he calls on his knowledge of their

% particular cultural practices (Interests: Religious

% Cults skill at 45). The gamemaster allows this and

% applies a +20 modifier to Dav’s Persuasion Test.



% COMPLEMENTARY SKILL BONUS


%   SKILL RATING                          MODIFIER


%        01–30                               +10


%        31–60                               +20


%         61+                                +30

% %%% txt/176.txt


%                                   APTITUDE R
% ASSESSMENT         SOMATICS     COORDINATION       REFLEXE
%  child average         inept           clumsy          slow
%  adult average         weak             able           paced
% anshuman average        fit           coordinated       swift

% enhanced          enhanced           agile           fast

% superhuman           gifted          nimble        lightnin

% posthuman            elite          unerring        synapt



% SKILL RANGES

% What is the difference between being a clumsy neo-

% phyte wobbling in zero gravity and being a veteran

% gliding effortlessly through space as though you were

% dancing? The answer is training and skill. The greater

% your skill, the more likely you are to not only succeed

% at what you want to do, but succeed well.

%  Aptitudes in Eclipse Phase range from 1 to 30,

% while learned skills range from 0 to 99. These num-

% bers are an abstraction of the range of transhuman

% abilities and traits. The Aptitude Range table provides

% a breakdown of different aptitude levels and how they

% relate to each other. Likewise, the Learned Skill Range

% table provides an interpretation for the capabilities at

% different skill levels.




% APTITUDES

% There are 7 aptitudes in Eclipse Phase, described on p.

% 123. Each character has these aptitudes at a minimum

% rating of 1.


% APTITUDE-ONLY TESTS

% In rare cases, a test may call for using an aptitude

% only, rather than a learned skill. This should only

% occur when no learned skills are appropriate to the

% test, and these circumstances are usually noted in

% the rules.

%  Aptitude-only tests must be handled carefully, as

% the range of aptitude ratings (1–30) is typically much

% smaller than the rating of learned skills (0–99). For

% this reason, most aptitude tests should use a target

% number equal to the aptitude x 3. In rare cases where



% APTITUDE COMPARISON:
% FLATS VS. SPLICERS AND
% Compared to humans in the early 21st Century, the ave
% smarter, stronger, and healthier than their unaugment
% flats (p. 139), most closely approximate the type of per
% however, inhabit bodies that are known as splicers (p. 1
% anyway). Splicers are genefixed to avoid genetic defect
% are tweaked to make them superior across the board:
% cognitive capacity, and are more attuned to the world
% NGE
% COGNITION              INTUITION            SAVVY          WILLPOWER

%  limited               aware            awkward           distracted

% intelligent          perceptive         personable        controlled


% bright                sharp          charismatic         focused

%  learned              uncanny            dazzling          resolute

%  brilliant            prescient        mesmerizing       unwavering

%  genius           near omniscient        hypnotic        unshakable





%       LEARNED SKILL RANGES

% SKILL         EQUIVALENCE

%  00          No exposure or familiarity, completely unskilled

%  10          Very rudimentary knowledge

%  20          Basic operator’s proficiency (driver’s license, gun permit,


%            high school diploma)

%  30          Hands-on experience, some professional training

%  40          Basic professional certification (police driving, army rifle


%            certified, college diploma)

%  50          Experience from professional-level work, some


%            advanced training

%  60          Expert competence (competitive driver, marksman, PhD)

%  70          Experience from expert-level work, has had unique in-


%            novations or insights

%  80          Worthy of being a system-renowned authority on


%            the subject

%  90          Nobel/Olympic/grandmaster

%  99          Pinnacle of current understanding and innovation




% the test is more difficult, the gamemaster may simply

% use an aptitude x 2, or just the straight aptitude rating.

% In some cases, more than one aptitude may be rel-

% evant to the test, and so they may be added together

% to derive the target number.


% What follows are a few examples where an aptitude-

% only test might be appropriate. Gamemasters may call

% for similar tests in other situations, but learned skills

% should be used whenever possible.




% EXALTS
% e transhuman in the world of Eclipse Phase is faster,
% predecessors. Normal unaugmented humans, called
%  that was born in our time. The majority of people,
%  or exalts (p. 139) (well, those with biological bodies
%  d optimized for certain characteristics, while exalts
% ey are more attractive, more athletic, have greater
% und them than their unaugmented kin.                 ■

% %%% txt/177.txt
% BRUTE STRENGTH
% Any test that involves simple brute strength can be
% handled as an SOM x 3 Test. Use this when smashing
% down a door, breaking an item in half, engaging in a
% tug-of-war, or lifting and carrying a heavy item.

% CATCHING THROWN OBJECTS
% Use REF + (COO x 2) any time you need to catch a
% thrown or dropped object, such as catching a baseball,
% saving a priceless vase from shattering, or throwing
% back a grenade (see p. 200).

% COMPOSURE AND RESOLVE
% Various game situations may frighten your character,
% turn their stomach, horrify them, or rattle them to the
% core of their being. Use WIL x 3 to determine if your
% character can hold their ground, keep it down, and
% pull themselves together.

% ESCAPE ARTIST
% If a character wants to slip free of physical bonds (such
% as ropes or handcuffs) or otherwise contort themselves
% (such as wriggling out from under a collapsed wall or
% an overturned vehicle), an Escape Artist Test may be
% called for using the character’s COO + SOM. Apply
% modifiers appropriate to the difficulty of the situation.
% At the gamemaster’s discretion, escaping from some
% restraining situations may be considered a Task Action
% with an appropriate timeframe.

% HAVING AN IDEA
% Sometimes the players miss the obvious, or their per-
% sonal mindset or biases cause them to misinterpret a
% situation or understand events in a way different from
% how the actual character would. In cases like this, the
% gamemaster can call for an INT x 3 or COG x 3 roll
% (whichever is more appropriate) to determine if the
% character gets an idea that will help them along. This
% test should be used sparingly, and only for assessing
% the character’s interpretation of obvious and known
% facts and details.

% MEMORIZING AND REMEMBERING
% Memories are what egos use to maintain continuity of
% self from morph to morph, but humans are notorious
% for remembering things incorrectly. Whenever char-
% acters attempt to recall a memory or memorize some
% piece of information, use COG x 3 to determine how
% well they succeed. Note that characters with eidetic
% memory (p. 146 or 301) or mnemonic augmentation
% (p. 307) have perfect memory, so no test is required.

% %%% txt/178.txt

%  COMPLETE SKILL LIST

% This section details all of the learned skills avail-

% able in Eclipse Phase. Gamemasters and players

% may, of course, agree to add additional skills to

% this list as appropriate to their campaign.


%  ACADEMICS: [FIELD]

% Type: Field, Knowledge

% Linked Aptitude: COG


%  What it is: Academics covers any sort of spe-

% cialized non-applied knowledge you can only get

% through intensive education. Most theoretical and

% applied sciences, social sciences, transhumanities,

% etc. are covered by this skill. Most of the other

% skills listed in this chapter could also be taken as

% an Academics field, reflecting a working theoretical

% knowledge of the skill—for example, Academics:

% Armorer or Academics: Interrogation.


%  When you use it: Academics is used when a

% character wishes to call upon a specific body of

% knowledge. For example, Academics: Chemistry

% could be used to identify a particular substance,

% understand an unusual chemical reaction, or deter-

% mine what elements are needed to nanofabricate

% something that requires exotic materials. At the

% gamemaster’s discretion, some Academics-related

% tests might not be defaultable, given that only

% someone who has been educated in that subject is

% likely to be able to tackle it.

% Sample Fields: Archeology, Astrobiology, As-


%  tronomy, Astrophysics, Astrosociology, Bio-


%  chemistry, Biology, Botany, Computer Science,




% NECESSARY SKILLS
% While characters will need a mix of skills to
% ucceed in the varied tasks they encounter in
%  clipse Phase, some skills are crucial for any
% haracter. If a character lacks these, they will
%  ave a difficult time getting by, so it is impor-
% ant for players and gamemasters to know
% hese particular skills.

% Fray: Fray is the primary skill you use to avoid
%  etting hit in combat. Even if you plan to avoid
% ombat, being able to get out of the way when
%  ecessary is a handy survival skill to have.

% Networking: Unless you live in total isolation,
% ou need a Networking skill—preferably several.
%  etworking is how you interact with people in

% particular social circle to obtain information,
% pread rumors, call in favors, and so on.

% Perception: Perception Tests get called for
%  uite often, so if you want your character to
%  now what’s going on around them, make sure
% o get this skill. Investigation and Scrounging
%  re also good, but Perception is king.          ■


%                 SKILL LIST
%  L                          LINKED APTITUDE          CATEGORY
%  emics: [Field]                    COG               Knowledge
%  al Handling                       SAV              Active, Social
%  Field]                            INT               Knowledge
% m Weapons                          COO             Active, Combat
% es                                 SOM             Active, Combat
% bing                               SOM             Active, Physical
%  s                                 SOM             Active, Combat
%  rol                        WIL (no defaulting)   Active, Mental, Psi
%  ption                             SAV              Active, Social
%  olitions                   COG (no defaulting)    Active, Technical
% uise                               INT             Active, Physical
%  c Melee Weapon: [Field]           SOM             Active, Combat
%  c Ranged Weapon: [Field]          COO             Active, Combat
%  t                                 SOM             Active, Physical


%                               REF             Active, Combat
%  Fall                              REF             Active, Physical
%  unning                            SOM             Active, Physical
% nery                               INT             Active, Combat
%  ware: [Field]                     COG             Active, Technical
%  rsonation                         SAV              Active, Social
%  ation                             COO             Active, Physical
%  ec                         COG (no defaulting)    Active, Technical
%  est: [Field]                      COG               Knowledge
%  acing                             COG             Active, Technical
%  idation                           SAV              Active, Social
%  tigation                          INT              Active, Mental
% sics                               SAV              Active, Social
%  ic Weapons                        COO             Active, Combat
%  uage: [Field]                     INT               Knowledge
%  cine: [Field]                     COG             Active, Technical
% gation                             INT              Active, Mental
% working: [Field]                   SAV              Active, Social
%  ing                               COO             Active, Physical
%  ption                             INT              Active, Mental
% uasion                             SAV              Active, Social


% [Field]                       REF              Active, Vehicle
%  ssion: [Field]                    COG               Knowledge
%  ramming                    COG (no defaulting)    Active, Technical
% ocol                               SAV              Active, Social
%  ssault                     WIL (no defaulting)   Active, Mental, Psi
% hosurgery                          INT             Active, Technical
%  arch                              COG             Active, Technical
% unging                             INT              Active, Mental
%  er Weapons                        COO             Active, Combat
%  e                          INT (no defaulting)   Active, Mental, Psi
% y Weapons                          COO             Active, Combat
% mming                              SOM             Active, Physical
% wing Weapons                       COO             Active, Combat
%  med Combat                        SOM             Active, Combat

% %%% txt/179.txt

% Cryptography, Economics, Engineering, Genetics,

% Geology, Linguistics, Mathematics, Memetics,

% Nanotechnology, Old Earth History, Physics, Politi-

% cal Science, Psychology, Sociology, Xeno-archeology,

% Xenolinguistics, Zoology
% Specializations: As appropriate to the field

% ANIMAL HANDLING
% Type: Active, Social
%  Linked Aptitude: SAV

% What it is: Skilled animal handlers are able to
%  train and control a wide variety of natural and trans-
%  genic animals, including partial uplifts. Though many
%  animal species went extinct during the Fall, a few
% “ark” and zoo habitats keep some species alive, and
%  many others can be resurrected from genetic samples.
%  Exotic animals are considered a sign of prestige
%  among the hypercorp elites, and guard animals are
%  occasionally used to protect high-security installations.
%  Likewise, many habitats and settlements employ small
%  armies of partially uplifted, genetically modified, and
%  behavior-controlled creatures for sanitation or other
%  purposes. Many new and strange breeds of animal are
%  created daily to serve a variety of roles.

% When you use it: Animal Handling is used whenever
% you are trying to manipulate an animal, whether your
%  intent is to calm it down, keep it from attacking, in-
%  timidate it, acquire its trust, or goad it into attacking.
% Your Margin of Success determines how effective you
%  are at convincing the creature. At the gamemaster’s
%  discretion, modifiers may be applied to the test. Like-
% wise, winning an animal over may sometimes take
%  time, and so could be handled as a Task Action with a
%  timeframe of five minutes or more.
%  Specializations: Per animal species (dogs, horses, smart

%  rats, etc.)

% TRAINING ANIMALS
% Training animals is a time-consuming task requiring
% repeated efforts and rewards to reinforce the trained
% behavior. Treat this as a Task Action with a timeframe
% of one day to one month, depending on the complex-
% ity of the action. Apply modifiers to this test based on
% the relative intelligence of the animal being trained,
% how domestic it is, and the complexity of the task.

% Once an animal has been trained, commanding
% it is treated as a Simple Success Test (p. 118) except
% for unusual or stressful situations, in which case the
% trainer receives a +30 modifier on their Animal Han-
% dling Tests when convincing the animal to complete
% the trained action.

% ART: [FIELD]
% Type: Field, Knowledge
% Linked Aptitude: INT

% What it is: Art confers the ability to create and eval-
% uate artistic endeavors. This is a particularly useful
% skill in Eclipse Phase, especially in the post-scarcity
% economies where creativity and vision can be a key
% component to a character’s reputation.

% When you use it: The Art skill can be used to either
% create a new work of art or to duplicate an existing
% piece of art in the hopes of passing it off as your own.
% The skill can also determine the approximate value of
% a piece of art either on the open market, for monetary
% exchange systems, or in terms of reputation for the
% artist.
% Sample Fields: Architecture, Criticism, Dance, Drama,

% Drawing, Painting, Performance, Sculpture, Simul-

% space Design, Singing, Speech, Writing
% Specializations: As appropriate to the field

% BEAM WEAPONS
% Type: Active, Combat
% Linked Aptitude: COO
% What it is: The Beam Weapons skill covers the usage
% and maintenance of standard coherent beam energy
% weapons such as lasers, particle beam weapons,
% plasma rifles, and microwave weapons (p. 338).

% When you use it: A player uses their Beam Weap-
% ons skill whenever attacking with a beam weapon in
% combat (p. 191). Beam Weapons may also be used for
% tests involving maintenance of the weapon, but not
% for repairing or modifying the weapon (that would be
% Hardware: Armorer skill).
% Specializations: Lasers, Microwave Weapons, Particle

% Beam Weapons, Plasma Rifles

% BLADES
% Type: Active, Combat
% Linked Aptitude: SOM

% What it is: The Blades skill covers the usage and
% maintenance of standard bladed weapons (p. 334).

% When you use it: A player uses their Blades skill
% whenever attacking with a blade weapon in melee
% combat (p. 191). Blades may also be used for tests
% involving maintenance of the weapon, but not for
% repairing or modifying the weapon (that would be
% Hardware: Armorer skill). This skill is used for blade
% weapons implanted in the body at the end of an ap-
% pendage (hands, forearms, feet, octomorph arms, etc.),
% but the Exotic Melee Weapon skill is used for blades
% implanted in other parts of the body.
% Specializations: Axes, Implant Blades, Knives, Swords

% CLIMBING
% Type: Active, Physical
% Linked Aptitude: SOM

% What it is: Climbing is the skill of ascending and
% descending sheer surfaces with or without the aid of
% specialized equipment.

% When you use it: This skill is used whenever a char-
% acter wishes to scale a climbable surface. For heights
% greater than one story, climbing is handled as a Task
% Action with a timeframe equivalent to one meter per
% Action Phase. For rappelling, the timeframe for de-
% scent is 50 meters per Action Turn. Climbing gear (p.
% 332-333) provides appropriate modifiers.
% Specializations: Assisted, Freehand, Rappelling

% %%% txt/180.txt
% CLUBS
% Type: Active, Combat
% Linked Aptitude: SOM

% What it is: The Clubs skill covers the usage and
% maintenance of standard blunt melee weapons such
% as batons or sticks (see p. 334).

% When you use it: Players use their Clubs skill when-
% ever they want to attack with a blunt weapon in melee
% combat (p. 191). The Clubs skill may also be used for
% tests involving maintenance of the weapon, but not
% for repairing or modifying the weapon (that would be
% Hardware: Armorer skill).
% Specializations: Batons, Hammers, Staffs

% CONTROL
% Type: Active, Mental, Psi
% Linked Aptitude: WIL

% What it is: Control is the use of psi to manipulate
% individuals or actively penetrate their mental pro-
% cesses. This skill is only available to characters with
% the Psi trait (p. 147).

% When you use it: Use Control when taking a psionic
% tour through a foreign ego—messing around included.
% See Psi, p. 220.
% Specializations: By sleight

% DECEPTION
% Type: Active, Social
% Linked Aptitude: SAV

% What it is: Deception is your ability to act, bluff,
% con, fast talk, lie, misrepresent, and pretend. Accom-
% plished users of deception are able to convince anyone
% of nearly anything. This skill does not include using a
% physical disguise to appear to be another person (the
% Impersonate skill covers that area).

% When you use it: Use this skill whenever you want
% to deceive someone with words or gestures. A success-
% ful skill test means that you have passed off your de-
% ception convincingly. At the gamemaster’s discretion,
% someone who is actively alert for signs of deception
% may make an Opposed Test using the Kinesics skill.
% Specializations: Acting, Bluffing, Fast Talk

% DEMOLITIONS
% Type: Active, Technical
% Linked Aptitude: COG (no defaulting)

% What it is: Demolitions covers the use of controlled
% explosives.

% When you use it: Use it when making, placing, and
% disarming explosives and explosive devices. See De-
% molitions, p. 197.
% Specializations: Commercial Explosives, Disarming,

% Improvised Explosives

% DISGUISE

% Type: Active, Physical

% Linked Aptitude: INT

% What it is: Disguise is the art of physically altering
% your appearance so that you look like someone else.
% This includes both the use of props (wigs, contacts,
% skin pigments) and the altering of subtle physical
% characteristics (gait, posture, poise).

% When you use it: Use Disguise to fool someone into
% thinking you’re someone you’re not. This can be used
% to hide your identity or to make yourself look like
% someone in particular. When used against someone
% who knows your true look or the appearance of the
% person you are imitating, this is handled as an Op-
% posed Test against Perception or Investigation.
% Specializations: Cosmetic, Theatrical

% EXOTIC MELEE WEAPON: [FIELD]
% Type: Field, Active, Combat
% Linked Aptitude: SOM

% What it is: The Exotic Melee Weapon skill covers
% the use and maintenance of all melee weapons not
% covered by the Clubs or Blades skills (see p. 334).

% When you use it: Use the Exotic Melee Weapon skill
% when attacking someone with an exotic melee weapon
% in melee combat (p. 191).
% Sample Fields: Morning Star, Spear, Whip
% Specializations: N/A

% EXOTIC RANGED WEAPON: [FIELD]
% Type: Field, Active, Combat
% Linked Aptitude: COO

% What it is: Exotic Ranged Weapon skill includes
% the use and maintenance of all ranged weapons not
% covered by the Beam, Flechette, Kinetic, Sonic, or
% Throwing Weapons skills.

% When you use it: Use this skill whenever attacking with
% an exotic ranged weapon in ranged combat (p. 191).
% Sample Fields: Blowgun, Crossbow, Flamethrower,

% Slingshot
% Specializations: N/A

% FLIGHT
% Type: Active, Physical
% Linked Aptitude: SOM

% What it is: Flight is the skill of using your body to
% fly. This skill is used when sleeved in or jamming a
% winged or otherwise flight-capable morph (manual and
% remote-control flight are handled using Pilot skill).

% When you use it: Use this skill whenever you need to
% make an aerial maneuver, land in difficult conditions,
% maintain your course in steep winds, or otherwise
% keep from crashing or falling.
% Specializations: Diving, Landing, Takeoff, specific

% maneuvers

% FRAY
% Type: Active, Combat
% Linked Aptitude: REF

% What it is: Fray is the ability to get out of the way
% of incoming attacks, debris, or inconvenient passers-by.
% Characters that have a high Fray score are able to react
% quicker than others when dodging or maneuvering.

% When you use it: Whenever a character is physically
% attacked by an opponent in melee combat, roll Fray to
% avoid getting hit (see p. 191). Fray may also be used

% %%% txt/181.txt
% to dodge other events that may harm the character,
% such as avoiding a charging vehicle or jumping out of
% the way of a collapsing stack of crates.
% Specializations: Blades, Clubs, Full Defense, Unarmed

% FREE FALL
% Type: Active, Physical
% Linked Aptitude: REF

% What it is: Free Fall is about moving in free-fall and
% microgravity environments.

% When you use it: Use whenever you need to maneu-
% ver in a zero-g situation, such as propelling yourself
% across a large open space or making sure you don’t
% accidentally send yourself spinning off into space.
% Free Fall is also used when moving with spacesuit
% maneuvering jets and when parachuting.
% Specializations: Microgravity, Parachuting, Vacsuits

% FREERUNNING
% Type: Active, Physical
% Linked Aptitude: SOM

% What it is: Freerunning is part running, part gymnas-
% tics. It is about moving fast, maneuvering over/under/
% around/through obstacles, and placing your body
% where it needs to go. Freerunning/parkour is a popular
% pastime in habitats where open space is limited.

% When you use it: Use Freerunning whenever you
% need to overcome an obstacle via movement, such
% as hurdling a railing, rolling across the hood of a
% car, jumping across a pit, or swinging around a pole.
% Freerunning is also used for sprinting (p. 191) and full
% defense against attacks (p. 198).
% Specializations: Balance, Gymnastics, Jumping, Running

% GUNNERY
% Type: Active, Combat
% Linked Aptitude: INT

% What it is: Gunnery skill covers the use and main-
% tenance of large, vehicular, or non-portable weapons
% systems. Firing these weapons is more like playing a
% video game than firing a gun.

% When you use it: Use Gunnery when attacking with
% a vehicle-mounted weapon or weapon emplacement
% in ranged combat (p. 191).
% Specializations: Artillery, Missiles
% HARDWARE: [FIELD]
% Type: Field, Active, Technical
% Linked Aptitude: COG

% What it is: This skill encompasses the ability to
% build, repair, physically hack, and upgrade equipment
% of a specific type.

% When you use it: Hardware is primarily used to repair
% devices, vehicles, habitat systems, or synthetic morphs.
% See Building, Repairing, and Modifying below.
% Sample Fields: Aerospace (all air and space vehicles),

% Armorer (armor and weapons), Electronics (all

% computerized devices), Groundcraft, Implants, In-

% dustrial (habitat, factory, and life support systems),

% Nautical (watercraft and submarines), Robotics

% (synthetic morphs)
% Specializations: As appropriate to the field

% BUILDING
% Creating an item from scratch is handled as a Task
% Action with a timeframe determined by the gamemas-
% ter. The timeframe should be set according to the com-
% plexity of the object and could range from an hour
% (constructing a set of shelves) to days (assembling a
% robot from spare parts) to even months (building a
% house). Numerous factors may apply modifiers to
% the test, such as the use of entoptic blueprints/help
% manuals (+20) or poor working conditions (–10 to
% –30). Tools are also a factor, perhaps making the job
%  easier (superior tools +10 to +30), more difficult (poor
% or inadequate tools, –10 to –30), or even impossible
% (lack of required tools).

% REPAIRING
% Damaged items may be repaired in a similar manner. See
% the rules for Synthmorph and Object Repair, p. 209.

% MODIFYING
% Altering an object’s design and function follows the
% same basic rules as build and repair, above. The time-
% frame is determined by the gamemaster as appropriate
% to the modification.

% %%% txt/182.txt
% IMPERSONATION
% Type: Active, Social
% Linked Aptitude: SAV

% What it is: Impersonation is the skill of trying to
% pass yourself off as someone else in social situations,
% including virtual ones. This includes copying manner-
% isms and speech patterns and using accumulated in-
% formation to convince others that you are that person.
% In a universe where appearance is highly variable, the
% question of identity is largely one of both trust and
% picking up on behavioral quirks and verbal cues to
% recognize a given individual.

% When you use it: Sometimes it’s fun to pretend
% you’re someone else, and sometimes it’s profitable
% or lifesaving. Use this skill whenever you attempt
% to convince someone that you are actually someone
% else through some sort of social or online interaction.
% Forks use this skill when passing themselves off as
% their alpha ego. Impersonate is handled as an Op-
% posed Test against the Kinesics skill.
% Specializations: Avatar, Face-to-Face, Verbal

% INFILTRATION
% Type: Active, Physical
% Linked Aptitude: COO

% What it is: Infiltration is the art of escaping
% detection.

% When you use it: Use Infiltration whenever you
% need to physically hide or move with stealth to avoid
% someone sensing you, whether you are hiding behind
% a tree, sneaking past a guard, or blending into a
% crowd. Infiltration can also be used to follow people
% (shadowing) without them detecting you. Infiltration
% is an Opposed Test against the Perception of whom-
% ever you are hiding from. The gamemaster may wish
% to roll such tests in secret so the player does not know
% whether they have succeeded or failed.
% Specializations: Blending In, Hiding, Shadowing,

% Sneaking

% INFOSEC
% Type: Active, Technical
% Linked Aptitude: COG (no defaulting)

% What it is: Infosec is short for “information secu-
% rity.” It encompasses training in electronic intrusion
% and counterintrusion techniques, as well as encryption
% and decryption.

% When you use it: Infosec is used both for hacking
% into electronic devices and mesh networks and for
% protecting them. See the Mesh chapter, p. 234, for
% more details.
% Specializations: Brute-Force Hacking, Decryption,

% Probing, Security, Sniffing, Spoofing

% INTEREST: [FIELD]
% Type: Field, Knowledge
% Linked Aptitude: COG

% What it is: Interest includes just about any topic
% that captures your attention that isn’t covered by
% another skill. This includes hobbies, obsessions,
% causes, pastimes, and other recreational pursuits.

% When you use it: Use the Interest skill whenever
% you need to recall or use knowledge related to the
% particular interest in question.

% Field Examples: Ancient Sports, Celebrity Gossip,
% Conspiracies, Factor Trivia, Gambling, Hypercorp
% Politics, Lunar Habitats, Martian Beers, Old Earth
% Nation-States, Reclaimer Blogs, Science Fiction, Scum
% Drug Dealers, Spaceship Models, Triad Economics,
% Underground XP
% Specializations: As appropriate to the field

% INTERFACING
% Type: Active, Technical
% Linked Aptitude: COG

% What it is: Interfacing is about using computerized
% electronic devices and software.

% When you use it: Use Interfacing to understand an
% electronic device you are not familiar with, use a pro-
% gram according to its normal operating parameters,
% manipulate electronic files of various types (including
% images, video, XP, and audio files), scan for wireless
% devices, and otherwise interact with and command
% your ecto, muse, and other computerized devices.
% Some Interfacing actions may be Task Actions, with
% a timeframe determined by the gamemaster. For more
% detail, see the Mesh chapter, p. 234.
% Specializations: Forgery, Scanning, Stealthing, by

% program

% INTIMIDATION
% Type: Active, Social
% Linked Aptitude: SAV

% What it is: Intimidation is convincing someone to
% do what you want based on direct threats (implied or
% actual) or sheer force of personality.

% When you use it: Use Intimidation to scare someone
% into submission, browbeat them into getting your way,
% command them to follow your orders, or berate them
% into giving up information. Influence is handled as an
% Opposed Test, pitted against the target’s WIL + WIL
% + SAV.
% Specializations: Interrogation, Physical, Verbal

% INVESTIGATION
% Type: Active, Mental
% Linked Aptitude: INT

% What it is: Investigation is the art of analyzing
% evidence, piecing together clues, solving mysteries,
% and making logical deductions from groups of facts.
% Investigation differs from Perception in that it is the
% conscious search for clues or pieces of a puzzle.

% When you use it: Use Investigation to draw conclu-
% sions from assorted details. For example, Investiga-
% tion could be used to determine the likely sequence
% of events at a crime scene, determine a possible social
% connection between two people, or deduce how an
% enemy made their escape. Investigation is a great way

% %%% txt/183.txt
% to provide clues to players, especially when the subject
% matter is something their character might know well
% but the player does not.
% Specializations: Evidence Analysis, Logical Deductions,

% Physical Investigation, Physical Tracking

% KINESICS
% Type: Active, Social
% Linked Aptitude: SAV

% What it is: Kinesics is the art of empathy and non-
% vocal communication.

% When you use it: Use Kinesics to read body language,
% tells, social cues, and other subconscious indicators. It
% can also be used to emote more effectively. Kinesics
% is used defensively whenever someone is trying to de-
% ceive you; make an Opposed Test against that person’s
% Deception or Impersonation skill.

% Though synthetic morphs are also designed to
% emote, reading them is not as easy. Apply a –30 modi-
% fier when judging a synthetic morph inhabited by a
% character or AGI. Likewise, standard AIs are also dif-
% ficult to read; apply a –60 modifier when judging a
% synthetic morph or pod operated by an AI.
% Specializations: Judge Intent, Nonvocal Communication

% JUDGING EMOTIONS AND INTENTIONS
% Transhumans are empathic beings, and so you can
% attempt to gauge the demeanor and/or intent of
% someone you are dealing with by rolling a Kinesics
% Test. This attempt to read someone is far from exact,
% however, and it is easy to misjudge. The gamemaster
% should make this test in secret and only allow a hint
% if successful—it is not possible to read someone with
% absolute certainty. If the person being judged is inten-
% tionally trying to deceive the character, this should be
% an Opposed Test against their Deception skill.

% NONVOCAL COMMUNICATION
% Experts in Kinesics can effectively communicate
% with each other simply by posture, stances, gestures,
% demeanors, and looks. Such communication is nec-
% essarily limited in the amount of information it can
% convey, but feelings, attitudes, affirmation/negation,
% and simple concepts may be passed. To effectively
% communicate complex concepts, the gamemaster may
% require successful Kinesics Tests from both parties, ap-
% plying modifiers as appropriate.

% KINETIC WEAPONS
% Type: Active, Combat
% Linked Aptitude: COO

% What it is: Kinetic Weapons covers the use and
% maintenance of standard kinetic projectile weapons
% like firearms and railguns (p. 335).

% When you use it: Use this skill whenever attacking
% with a kinetic weapon in ranged combat (p. 191).
% Specializations: Assault Rifles, Machine Guns, Pistols,

% Sniper Rifles, Submachine Guns

% LANGUAGES IN

% ECLIPSE PHASE

% With the Fall of Earth, the languages that

% remain most prominent in the solar system are

% those that were extensively carried into space

% by countries and hypercorps with aggressive

% space programs or by the large populations

% of poor laborers and infomorph refugees that

% followed. No single language dominated the

% realm of space expansion, and multilingual-

% ism was common. Many habitats and (sub)

% cultural groupings cling to specific languages

% as a method of retaining cultural identity. De-

% spite the availability of instant translation via

% the mesh, many people remain versed in two or

% more languages.

%  The ten most common languages in the solar

% system by speaking populations are: Arabic,

% Cantonese, English, French, Hindi, Japanese,

% Mandarin, Portuguese, Russian, and Spanish.

% Other languages that remain strong include

% Bengali, Dutch, Farsi, German, Italian, Javanese,

% Korean, Polish, Punjabi, Swedish, Tamil, Turkish,

% Urdu, Vietnamese, and Wu. Some languages

% were effectively lost during the Fall, especially

% those in some undeveloped regions, as their

% speaking populations did not migrate into space

% pre-Fall and were not privileged enough to sur-

% vive in large numbers as infomorph refugees. ■



% LANGUAGE: [FIELD]
% Type: Field, Knowledge
% Linked Aptitude: INT

% What it is: Language covers the speaking and
% reading of languages other than the player’s native
% tongue. A speaker is considered fluent at a skill level
% of 50; anything above this indicates further refine-
% ment in technical vocabulary, accents, and knowl-
% edge of dialects.

% When you use it: Use the Language skill whenever
% you want to speak, understand, or read something in
% a language at which you are skilled. Most speaking
% and reading comprehension tests can be considered
% Simple Success Tests if your skill is over 50, unless the
% gamemaster rules the subject is sufficiently complex
% that a non-native speaker would have trouble under-
% standing it.
% Sample Fields: Arabic, Cantonese, English, French,

% Hindi, Japanese, Mandarin, Portuguese, Russian,

% Spanish
% Specializations: As appropriate to the field, represent-

% ing dialects, technical jargon, and subcultural slang

% %%% txt/184.txt
% MEDICINE: [FIELD]
% Type: Field, Active, Technical
% Linked Aptitude: COG

% What it is: Medicine is the applied care and mainte-
% nance of biological beings and life.

% When you use it: Use Medicine whenever you need
% to apply medical care beyond the immediate help
% provided by first responders. This includes conduct-
% ing physical exams, diagnosing ailments, treating
% problems and illnesses, surgery, using biotech and
% nanotech medical tools, and long-term care. See Heal-
% ing and Repair, p. 208.
% Sample Fields: Biosculpting, Exotic Biomorphs,

% Gene Therapy, General Practice, Implant Surgery,

% Nanomedicine, Mercurials (by type), Paramedic,

% Pods, Psychiatry, Remote Surgery, Trauma Surgery,

% Veterinary
% Specializations: As appropriate to the field

% NAVIGATION
% Type: Active, Mental
% Linked Aptitude: INT

% What it is: Navigation is the art of finding your way,
% whether using AR maps, a compass, the stars, or an
% astrogation AI.

% When you use it: Use Navigation whenever you
% need to plot out a course, determine a direction, or
% otherwise keep from getting lost.
% Specializations: Astrogation, Map Making, Map

% Reading

% NETWORKING: [FIELD]
% Type: Active, Social
% Linked Aptitude: SAV

% What it is: Networking is your skill at working your
% contacts, trading favors, and keeping your finger on
% the pulse of a particular faction or cultural grouping.

% When you use it: Use Networking to gather infor-
% mation or call on services using your Reputation (see
% Reputation and Social Networks, p. 285).
% Sample Fields: Autonomists (@-rep), Criminals (g-rep),

% Ecologists (e-rep), Firewall (i-rep), Hypercorps (c-

% rep), Media (f-rep), Scientists (r-rep). At the game-

% master’s discretion, this list can be expanded to

% other (sub)cultural groupings.
% Specializations: As appropriate to each field

% PALMING
% Type: Active, Physical
% Linked Aptitude: COO

% What it is: Palming is the skill of handling items quick-
% ly and nimbly without others noticing. Palming is not
% only about dexterous manipulation of objects but also
% relies heavily on obfuscation, timing, and misdirection.

% When you use it: Use Palming any time you are
% trying to conceal an item on your person, shoplift,
% pick a pocket, surreptitiously discard something, or
% perform a magic trick. Palming is an Opposed Test
% against the Perception of any onlookers. The game-
% master may wish to make this roll secretly.
% Specializations: Pickpocketing, Shoplifting, Tricks

% PERCEPTION
% Type: Active, Mental
% Linked Aptitude: INT

% What it is: Perception is the use of your physical
% senses (including cybernetic) and awareness of the
% physical world around you. Perception differs from
% Investigation in that it is noticing things by chance,
% rather than actively searching for something.

% When you use it: Use Perception whenever you
% wanted to take a detailed account of your surround-
% ings (see Detailed Perception, p. 190). Perception
% can also be considered an Automatic Action (see
% Basic Perception, p. 190) and so the gamemaster may
% call for a Perception Test to determine if you notice
% something; it is recommended that such tests be rolled
% secretly by the gamemaster. Perception is also used as
% an Opposed Test whenever someone around you is
% trying to be sneaky with Infiltration or Palming.
% Specializations: Aural, Olfactory, Tactile, Taste, Visual

% %%% txt/185.txt
% PERSUASION
% Type: Active, Social
% Linked Aptitude: SAV

% What it is: Persuasion is the art of convincing
% someone to do what you want through the use of
% words and gestures. This does not include persuasion
% through threats or force (that is covered by Intimida-
% tion) or by lying (covered by Deception).

% When you use it: Use Persuasion any time you are
% trying to bargain with, convince, or manipulate some-
% one. This can include motivating your subordinates or
% peers to take action, seducing a companion, winning
% a political debate, or negotiating a contract, among
% other things. Persuasion is handled as an Opposed
% Test against the target’s WIL + WIL + SAV when one
% person is simply trying to win over another. If both
% parties are trying to convince each other, make it an
% Opposed Test between Persuasion skills.
% Specializations: Diplomacy, Morale Boosting, Negoti-

% ating, Seduction

% PILOT: [FIELD]
% Type: Field, Active, Vehicle
% Linked Aptitude: REF

% What it is: Pilot is your skill at driving/flying a ve-
% hicle of a particular type.

% When you use it: You use Pilot skill whenever you
% need to maneuver, control, or avoid crashing a vehicle,
% whether you are in the pilot’s seat, remote controlling
% a robot, or directly jamming a vehicle with VR. Each
% vehicle has a Handling modifier that applies to this
% test, along with other situational modifiers (see Bots,
% Synthmorphs, and Vehicles, p. 195).
% Sample Fields: Aircraft, Anthroform (walkers), Exotic

% Vehicle, Groundcraft (wheeled or tracked), Space-

% craft, Watercraft
% Specializations: As appropriate to the field

% PROFESSION: [FIELD]
% Type: Field, Knowledge
% Linked Aptitude: COG

% What it is: Profession skills indicate training in a
% profession practiced in Eclipse Phase. This can indi-
% cate either formal training or informal, on-the-job type
% training, and includes both legal and extralegal trades.

% When you use it: Use Profession to perform work-
% related tasks for a specific trade (i.e. mining, balancing
% accounts, designing a security system, etc.) or to refer-
% ence specialized knowledge that someone trained in
% that profession might have.
% Sample Fields: Accounting, Appraisal, Asteroid Pros-

% pecting, Banking, Cool Hunting, Con Schemes,

% Distribution, Forensics, Lab Technician, Mining,

% Police Procedures, Psychotherapy, Security Ops,

% Smuggling Tricks, Social Engineering, Squad Tactics,

% Viral Marketing, XP Production
% Specializations: As appropriate to the field
% PROGRAMMING
% Type: Active, Technical
% Linked Aptitude: COG (no defaulting)

% What it is: Programming is your talent at writing
% and modifying software code.

% When you use it: Use Programming to write new
% programs, modify or patch existing software, break
% copy protection, find or introduce exploitable flaws,
% write virii or worms, design virtual settings, and so
% on. See the Mesh chapter, p. 234. Programming is also
% applied when using nanofabrication devices.
% Specializations: AI Code, Malware, Nanofabrication,

% Piracy, Simulspace Code

% NANOFABRICATION
% Nanofabrication is use of Programming skill to create
% objects using a cornucopia machine, fabber, or maker
% (p. 327). If you have appropriate blueprints and raw
% materials, most uses of a nanofabricator can be treated
% as a Simple Success Test (p. 118). If you wish to create
% an item for which you do not have blueprints or the
% proper raw materials, however, or you wish to alter an
% item’s design, then a Nanofabrication Test is called for.
% See Nanofabrication, p. 284.
% Specializations: Art, Clothing, Electronics, Food, Forg-

% ery, Weapons

% PROTOCOL
% Type: Active, Social
% Linked Aptitude: SAV

% What it is: Protocol is the art of making a good
% impression in social settings. This includes keeping
% up with the latest memes, trends, gossip, interests and
% habits of various (sub)cultural group.

% When you use it: Use Protocol whenever you need
% to choose your words carefully, determine who is the
% appropriate person to speak to, impress someone with
% your grasp of customs, or otherwise fit into a specific
% social/cultural grouping. Part etiquette, part streetwise,
% Protocol allows you to navigate treacherous social
% waters and put people at ease. If the character is deal-
% ing with a suspicious or hostile audience, make this an
% Opposed Test against the target’s WIL + WIL + SAV.
% Specializations: Anarchist, Brinker, Criminal, Factor,

% Hypercorp, Infomorph, Mercurial, Reclaimer, Pres-

% ervationist, Scum, Ultimate

% NEGATING SOCIAL GAFFES
% Sometimes a player will make a mistake that their
% character never would, whether that’s failing to stand
% in the presence of hypercorp royalty, confusing a gang
% leader for a peon, or accidentally insulting someone’s
% heritage. In cases like this, the player may make a Pro-
% tocol Test for the appropriate field in order to negate
% the gaffe. If successful, the character never actually
% screwed up, or at least managed to cover their tracks
% without ruffling any feathers.

% %%% txt/186.txt
% PSI ASSAULT
% Type: Active, Mental, Psi
% Linked Aptitude: WIL

% What it is: Psi Assault is the skill of damaging an-
% other ego’s mind. It can only be purchased by charac-
% ters with the Psi trait (p. 147).

% What it does: Use Psi Assault when attacking an-
% other ego’s mind in psi combat.
% Specializations: By sleight

% PSYCHOSURGERY
% Type: Active, Technical
% Linked Aptitude: INT

% What it is: Psychosurgery is the use of machine-
% aided psychological techniques to repair, damage, or
% manipulate the psyche.

% When you use it: Use Psychosurgery to attempt the
% tricky process of editing someone’s mind (see Psycho-
% surgery, p. 229). Psychosurgery can be used benefi-
% cially to help patients who remember their deaths, feel
% disconnected after remorphing, or have experienced
% other sorts of mental traumas. This skill may also be
% used to interrogate, torture, or otherwise mess with
% captive minds in a VR environment.
% Specializations: Memory Manipulation, Personality

% Editing, Psychotherapy

% RESEARCH
% Type: Active, Technical
% Linked Aptitude: COG

% What it is: Research is the skill for looking up infor-
% mation on the Mesh: searching, sifting, mining, and
% interpreting data. This includes knowing where to look,
% what links to follow, and how to optimize your queries.

% When you use it: Use the Research skill whenever
% you need to look up the answer to a question, find
% databases, search archives, or track down anything
% online. Research is typically a Task Action with the
% timeframe and difficulty modifier determined by the
% gamemaster. See the Online Research, p. 249.
% Specializations: Tracking, by information type

% SCROUNGING
% Type: Active, Mental
% Linked Aptitude: INT

% What it is: Scrounging is your ability to find things,
% particularly things of use or value that are concealed,
% buried, or hard to find. This includes knowing where
% to look and what to look for. Scrounging differs from
% both Perception and Investigation in that it is about
% finding items hidden among others, and in most cases
% about finding something in particular (food, valu-
% ables, etc.).

% When you use it: Use Scrounging to dumpster-dive a
% meal, search ruins for relics, find bargains at a bazaar,
% forage berries in the forest, locate a spacesuit in an
% abandoned ship, etc. Scrounging is typically handled
% as a Task Action with a timeframe and difficulty
% modifier determined by the gamemaster.
% Specializations: Bazaars, Forests, Habitats, Ruins
% SEEKER WEAPONS
% Type: Active, Combat
% Linked Aptitude: COO

% What it is: Seeker Weapons covers the use and
% maintenance of seeker launchers (p. 339) and seeker
% missiles (p. 340).

% When you use it: Use this skill when attacking with
% a seeker in ranged combat (p. 191).
% Specializations: Armband, Pistol, Rifle, Underbarrel

% SENSE
% Type: Active, Mental, Psi
% Linked Aptitude: INT

% What it is: Sense is the use of psi to scan egos. Only
% characters with the Psi trait (p. 147) may purchase
% this skill.

% What it does: See Psi, p. 220.
% Specializations: By sleight

% SPRAY WEAPONS
% Type: Active, Combat
% Linked Aptitude: COO

% What it is: The Spray Weapons skill covers the use
% and maintenance of cone-effect ranged weapons (see
% Spray Weapons, p. 340).

% When you use it: A player uses their Sonic Weapons
% skill whenever they are attacking with a spray weapon
% in ranged combat (p. 191).
% Specializations: Buzzer, Freezer, Shard, Shredder, Torch

% SWIMMING
% Type: Active, Physical
% Linked Aptitude: SOM

% What it is: Swimming is the art of moving and
% not drowning within fl uids. It includes fl oating,
% surface swimming, snorkeling, diving, and related
% equipment use.

% When you use it: Use Swimming whenever you
% need to move and survive in water or another liquid
% environment. Swimming in a non-threatening en-
% vironment can be handled as a Simple Success Test.
% Swimming over a long distance could be handled as
% a Task Action. Diving off a cliff into a lake, prevent-
% ing yourself from being swept away in a raging river
% current, or making sure you’ve set a proper gas mix
% for a deep-sea dive, among other things, requires a
% Success Test.
% Specializations: Diving, Freestyle, Underwater Diving

% THROWING WEAPONS
% Type: Active, Combat
% Linked Aptitude: COO

% What it is: Throwing Weapons skill covers the use
% and maintenance of standard throwing weapons, like
% grenades (p. 340).

% When you use it: Use Throwing Weapons skill
% whenever you are attacking with a throwing weapon
% in ranged combat (p. 191).
% Specializations: Grenades, Knives, Rocks

% %%% txt/187.txt
% T




% w
% USING KNOWLEDGE SKILLS
% t first glance, it may seem that Knowledge skills have
%  wer in-game applications than Active skills. To some
% egree this is the case. The importance of Knowledge

% ills, however, should not be underestimated. While
% hey play a role in analyzing clues and solving myster-
%  s, the real value of Knowledge skills is in helping the
% haracters—and the players—understand the world of
% clipse Phase. In particular these skills can be used to
%  ake plans, assess a situation, identify strengths and
%  eaknesses, evaluate worth, make comparisons, forecast
% robable outcomes, or understand the applicable science,
% ocio-economic factors, or cultural or historical context.

% For example, a group of characters looking to break
%  to a facility could use Profession: Security Procedures
% o evaluate the defenses, Academic: Architecture to
%  entify covert points of entry, Interests: Sports to plan

% eir infiltration at a time when the guards are likely to
% e distracted, Interests: Triads to identify a local crime
% roup that can sell them breaking and entering gear,
% nd Art: Sculpture when picking a valuable art piece
%  ith which to bribe an insider. When used appropriately,
% hese skills can be just as beneficial as the Active skills
% sed to break inside, if not more so because the plan
%  more likely to succeed as a result of this preparation.

% It is largely up to the gamemaster to enforce how
% seful Knowledge skills are in their game. The easiest
%  ay to reinforce their relevance is to penalize charac-
%  rs who don’t take advantage of them. For example,
% haracters who didn’t use their Profession: Security

% ocedures in the example above might end up being
% urprised when they run across a security system they
%  e not prepared to deal with, forcing them to impro-
%  se or even abandon their plans.                         ■



% ARMED COMBAT
% e: Active, Combat
% ked Aptitude: SOM
%  hat it is: Unarmed Combat is your ability to attack and
%  nd without using weapons.
%  hen you use it: Use Unarmed Combat whenever you
%  t to attack someone with your fists, feet, elbows, knees,
%  ther body parts in melee combat (p. 191).
%  ializations: Implant Weaponry, Kick, Punch, Subdual



%  ECIAL SKILLS
%  hile the preceding list represents the skills most commonly

% in Eclipse Phase, there may be certain skills called for in a
% paign that are not found in this book. In this case, the game-
% ter may work with the players to create a new skill to fill
% void. This option should be used sparingly to prevent skill
% t, and all skills are subject to approval by the gamemaster
% you choose to create a new skill, keep in mind that it needs
% e linked to an existing aptitude and should be a skill that
%  ailable to all characters, not just specific to one character.

% %%% txt/188.txt
%  COMBAT SUMMARY
% • Combat is handled as an


% Opposed Test.
% • Attacker rolls attack skill


% +/– modifiers.
% • Melee: Defender rolls


% Fray or melee combat


% skill +/– modifiers.
% • Ranged: Defender rolls


% (Fray skill ÷ 2, round


% down) +/– modifiers.
% • If attacker succeeds and


% rolls higher than the


% target, the attack hits.
% • Critical hits are armor-


% defeating.
% • An Excellent Success


% increases the damage


% by +5 (MoS 30+) or


% +10 (MoS 60+).
% • Armor is reduced by


% the attack’s Armor


% Penetration value (AP).
% • The weapon’s damage is


% reduced by the target’s


% modified Armor rating


% (unless the attack is


% armor-defeating).
% • If damage exceeds


% Wound Threshold, a


% wound is also scored.


% (Multiple wounds


% apply if the damage


% exceeds Wound


% Threshold by multiple


% factors.)

% %%% txt/189.txt


%      ACTION AN


%                                             ACTIO


%                                             During co


%                                             called Ac


%                                             will act fi






%             During any given Action Turn, there are 4


%                                    Actions that may b




% COMBAT EXAMPLE & SUMMARY
% The combat rules start on p. 191 and are quickly followed b
% an example of play and a combat summary. ■ pp. 191–193






%                               HEALING AND


%           You're going to get hurt. Biological morphs


%            and synthmorphs mechanically repaired. ■






%      Mental Health: Trauma to your ego can be crip


%      swap out your ego for a new one like you can a
% D COMBAT
%  TURNS AND INITIATIVE
% at, the game moves in small chunks of time
% Turns. The character with the highest Initiative
% within those Action Turns. ■ p. 188




%  CTIONS
% erent types of
%  ken. ■ p. 189



%                Automatic Actions: These abilities and effects re


%                                                    and are "always



%               Quick Actions: These fast and simple actions take m


%               You can always take 1 Quick Action in an Action Ph


%                       take only Quick Actions, you may take 3 or m



%                Complex Actions: These actions monopolize your


%                          can take 1 Complex Action in an Action P



%             Task Actions: These long-term activities take more th


%                   Turn to complete; anywhere from 2 turns to 2 y




% EPAIR
% be healed
%  . 208–209


%             HOSTILE ENVIRONMENTS


%             The solar system is full of natural dangers that will te


%             the most well-prepared explorer. ■ p. 200

%  —you can't
% ph. ■ p. 209

% %%% txt/190.txt
% AND COMBAT■ACTION AND COMBAT■



% N AND COMBAT
% Roleplaying games are about creating drama and
% adventure, and that usually means action and
% combat. Action and combat scenes are the moments
% when the adrenaline really gets pumping and the
% characters’ lives and missions are on the line.

% Combat and action scenarios can be confusing
% to run, especially if the gamemaster also needs to
% keep track of the actions of numerous NPCs. For
% these reasons, it’s important for the gamemaster to
% detail the action in a way that everyone can visual-
% ize, whether that means using a map and miniatures,
% software, a dry-erase board, or quick sketches on
% a piece of paper. Though many of the rules for
% handling action and combat are abstract—allowing
% room for interpretation and fudging results to fit
% the story—many tactical factors are also incorpo-
% rated, so even small details can make a large differ-
% ence. It also helps to have the capabilities of NPCs
% predetermined and to run them as a group when
% possible, to reduce the gamemaster’s burden in the
% middle of a hectic situation.



%  ACTION TURNS
% Action scenes in Eclipse Phase are handled in bite-
% size chunks called Action Turns, each approximately
% 3 seconds in length. We say “approximately” be-
% cause the methodical, step-by-step system used to
% resolve actions does not necessarily always translate
% realistically to real life, where people often pause,
% take breaks to assess the situation, take a breather,
% and so on. A combat that begins and ends within 5
% Action Turns (15 seconds) in Eclipse Phase could
% last half a minute to several minutes in real life. On
% the other hand, the characters may be in a situation
% where their breathing environment decompresses to
% vacuum in 15 seconds, so every second may in fact
% count. As a rule, gamemasters should stick with 3
% seconds per turn, but they shouldn’t be afraid to
% fudge the timing either when a situation calls for it.

% Action Turns are meant to be utilized for combat
% and other situations where timing and the order in
% which people act is important. If it is not necessary
% to keep track of who’s doing what so minutely, you
% can drop out of Action Turns and return to “regu-
% lar” free form game time.

% Each Action Turn is in turn broken down into
% distinct stages, described below.

%  STEP 1: ROLL INITIATIVE
% At the beginning of every Action Turn, each PLAYER
% involved in the scene rolls Initiative to determine the
% order in which each character acts. For more details,
% see Initiative.
% CTION AND COMBAT■ACTION AND CO



%                                                       7


%  STEP 2: BEGIN FIRST ACTION PHASE
%  Once Initiative is rolled, the fi rst Action Phase
%  begins. Everyone gets to act in the first Action Phase
%  (since everyone has a minimum Speed of 1), unless
%  they happen to be unconscious/dead/disabled, start-
%  ing with the character with the highest successful
%  Initiative roll.

%  STEP 3: DECLARE AND RESOLVE ACTIONS
%  The character going first now declares and resolves
%  the actions they will take during this first Action
%  Phase. Since some actions the character makes may
%  depend on the outcome of others, there is no need
%  to declare them all first—they may be announced
%  and handled one at a time.

% As described under Actions (p. 189), each charac-
%  ter may perform a varying number of Quick Actions
%  and/or a single Complex Action during their turn.
%  Alternately, a character may begin or continue with
%  a Task Action, or delay their action pending other
%  developments (see Delayed Actions, p. 189).

% A character who has delayed their action may
%  interrupt another character at any point during this
%  stage. That interrupting character must complete
%  this stage in full, then the action returns to the in-
%  terrupted character to finish the rest of their stage.

%  STEP 4: ROTATE AND REPEAT
%  Once the character has resolved their actions for
%  that phase, the next character in the Initiative order
%  gets to go, running through Step 3 for themselves.

%  If every character has completed their actions
%  for that phase, return to Step 2 and go the second
%  Action Phase. Every character with a Speed of 2 or
%  more gets to go through Step 3 again, in the same
%  Initiative order (modified by wound modifiers).
%  Once the second Action Phase is completed, return
%  to Step 2 for the 3rd Action Phase, where every
%  character with a Speed of 3 or more gets to go for
%  a third time. Finally, after everyone eligible to go in
%  the 3rd Action Phase has gone, return to Step 2 for
%  a fourth and last Action Phase, where every charac-
%  ter with a Speed of 4 can act for one final time.

% At the end of the fourth Action Phase, return to Step
%  1 and roll Initiative again for the next Action Turn.



%  INITIATIVE
%  Timing in an Action Turn can be critical—it may mean
%  life or death for a character who needs to get behind
%  cover before an opponent draws and fires their gun.
%  The process of rolling Initiative determines if a char-
%  acter acts before or after another character.

% %%% txt/191.txt
% INITIATIVE ORDER
% A character’s Initiative stat is equal to their Intuition +
% Reflexes aptitudes multiplied by 2. This score may be
% further modified by morph type, implants, drugs, psi,
% or wounds.

% In the first step of each Action Turn, every charac-
% ter makes an Initiative Test, rolling d100 and adding
% their Initiative stat. Whoever rolls highest goes first,
% followed by the other characters in descending order,
% highest to lowest. In the event of a tie, characters go
% simultaneously.




%       Adam, Bob, and Cami are rolling Initiative. Adam’s


%       Initiative stat is 80, Bob’s is 110, and Cami’s is 60.


%       Adam rolls a 38, Bob rolls a 24, and Cami rolls a
%  EXAMPLE






%       76. Adam’s total Initiative score is 118 (80 + 38),


%       Bob’s is 134 (110 + 24), and Cami’s is 136 (60 +


%       76). Cami rolled highest, so she goes first, followed


%       by Bob and then Adam. If Cami & Bob had tied,


%       they would both go at the same time.


% INITIATIVE AND DAMAGE
% Characters who are suffering from wounds have their
% Initiative score temporarily reduced (see Wounds, p.
% 207). This modifier is applied immediately when the
% wound is taken, which means that it may modify an
% Initiative score in the middle of an Action Turn. If a
% character is wounded before they go in that Action
% Phase, their Initiative is reduced accordingly, which
% may mean they now go after someone they were pre-
% viously ahead of in the Initiative order.




%       Before Bob’s Action Phase comes up, Bob takes
%  EXAMPLE






%       two wounds, knocking his Initiative down from


%       134 to 114. This means that Adam, with an Initia-


%       tive of 118, now goes before him.


% INITIATIVE, MOXIE, AND CRITICALS
% A character may spend a point of Moxie to go first in
% an Action Phase, regardless of their Initiative roll (see
% Moxie, p. 122). If more than one character chooses
% this option, then order is determined as normal first
% among those who spent Moxie, followed by those
% who didn’t.

% Similarly, any character that rolls a critical on Ini-
% tiative automatically goes first, even before someone
% who spent Moxie. If more that two characters rolled
% criticals, determine order between them as normal.

% SPEED
% Speed determines how many times a character can
% act during an Action turn. Every character starts with
% a default Speed stat of 1, meaning they can act in the
% first Action Phase of the turn only. Certain morphs,
% implants, drugs, psi, and other factors may cumu-
% latively increase their Speed to 2, 3, or even 4 (the
% maximum), allowing them to act in further Action

% SIMPLIFYING

% INITIATIVE

% For speedier resolution, simply have characters

% roll Initiative once for an entire scene. That

% Initiative result stays with them on each Action

% Turn until the combat or scenario is over. Like-

% wise, ignore Initiative modifiers from wounds. ■




% Phases as well. For example, a character with Speed
% 2 can act in the first and second Action Phases, and
% a character with Speed 3 can act in the first through
% third Action Phases. A character with Speed 4 is
% able to act in every Action Phase. This represents
% the character’s enhanced reflexes and neurology, al-
% lowing them to think and act much faster than non-
% enhanced characters.

% If a character’s Speed does not allow them to act
% during an Action Phase, they can initiate no actions
% during the pass—they must simply bide their time.
% The character may still defend themself, however, and
% any automatic actions remain “on.” Note that any
% movement the character initiated is considered to still
% be underway even during the Action Phases they do
% not participate in (see Movement, p. 190).

% DELAYED ACTIONS
% When it’s your turn to go during an Action Phase,
% you may decide that you’re not ready to act yet. You
% may be awaiting the outcome of another character’s
% actions, hoping to interrupt someone else’s action, or
% may simply be undecided about what to do yet. In this
% case, you may opt to delay your action.

% When you delay your action, you’re putting your-
% self on standby. At some later point in that Action
% Phase, you can announce that you are now taking
% your action—even if you interrupt another character’s
% action. In this case, all other activity is put on hold
% until your action is resolved. Once your action has
% taken place, the Initiative order continues on where
% you interrupted.

% You may delay your action into the next Action
% Phase, or even the next Action Turn, but if you do not
% take it by the time your next action comes around in
% the Initiative order, then you lose it. Additionally, if
% you do delay your action into another phase or turn,
% then once you take it you lose any action you might
% have in that Action Phase.



% ACTIONS
% When it’s your turn to act during an Action Phase,
% you have many options for what you can do—far too
% many to list here. There is a limit to what you can
% accomplish in 3 seconds, however, so some limita-
% tions must be adhered to. The first step is to figure

% %%% txt/192.txt
% out what type of action you want to take. In Eclipse
% Phase, actions are categorized as Automatic, Quick,
% Complex, or Task, based on how much time and
% effort they entail.

% AUTOMATIC ACTIONS
% Automatic Actions require no effort. These are abili-
% ties or activities that are “always on” (assuming you
% are conscious) or are otherwise reflexive (they happen
% automatically in response to certain conditions, with
% no effort from you). Breathing, for example, is an au-
% tomatic action—your body does it without conscious
% effort or thinking on your part.

% In most cases, Automatic Actions are not something
% that you initiate—they are always active, or at least
% on standby. Certain circumstances, however, will bring
% an Automatic Action to bear. Such Automatic Actions
% are invoked and handled immediately whenever they
% apply, without requiring effort from your character.

% RESISTANCE
% Resisting damage—whether from combat, a poison, or
% a psi attack—is one example of an Automatic Action
% that occurs in response to something else.

% BASIC PERCEPTION
% Your senses are continuously active, accumulating data
% on the world around you. Basic perception is consid-
% ered an Automatic Action, and so the gamemaster can
% call on you to make a Perception Test whenever you
% receive sensory input that your brain might want to
% take notice of (see Perception, p. 182). Likewise, you
% may ask the gamemaster at any time—even during
% other character’s actions—to make a basic Perception
% Test, just to find out what your character is noticing
% around them.

% Because basic perception is an automatic, subcon-
% scious activity, however, you will suffer a –20 modifier
% for distraction—your attention is focused elsewhere.
% In order to avoid the distraction modifier, you must
% actively engage in detailed perception or use an oracle
% implant (p. 308).

% QUICK ACTIONS
% Quick Actions are fast and simple, and they may often
% be multi-tasked. They require minimal thought and
% effort. You may undertake multiple Quick Actions on
% your turn during each Action Phase, limited only by
% the gamemaster’s judgment. If you are taking noth-
% ing but Quick Actions during an Action Phase, you
% should be allowed a minimum of 3 separate Quick
% Actions. If you are also engaging in a Complex or Task
% Action during that same Action Phase, you should be
% allowed a minimum of 1 Quick Action. Ultimately, the
% gamemaster decides what activity you can or can’t fit
% into a single Action Phase.

% Some examples of Quick Actions include: talking,
% switching a safety, activating an implant, standing
% up, dropping prone, gesturing, drawing/readying a
% weapon, handling an object, or using a simple object.
% AIMING
% Aiming is a special case in that it is a Quick Action
% but requires a degree of concentration that rules out
% other minor actions. If you wish to aim before making
% an attack in the same Action Phase, aiming is the only
% Quick Action you may make during that Action Phase
% (see Aimed Shots, p. 193).

% DETAILED PERCEPTION
% Detailed perception involves taking a moment to
% actively use your senses in search of information and
% assess what you are perceiving (see Perception, p.
% 182). It requires slightly more effort and brainpower
% (or computer power) than basic perception, which is
% automatic. As a Quick Action, you may only engage
% in detailed perception on your turn during an Action
% Phase, but you do not suffer a modifier for distrac-
% tion (unless you happen to be in a heavily distracting
% environment, such as a gunfight or agitated crowd).

% COMPLEX ACTIONS
% Complex Actions require more concentration and
% effort than Quick Actions—they effectively monopo-
% lize your attention. You may only take one Complex
% Action on each your Action Phase turns. Additionally,
% you may not engage in a Complex Action and a Task
% Action during the same Action Phase.

% Examples of Complex Actions include: attacking,
% shooting, acrobatics, full defense, disarming a bomb,
% using a complex device, or reloading a weapon.

% TASK ACTIONS
% A Task Action is any activity that requires longer than one
% Action Turn to complete. Each Task Action lists a time-
% frame for how long the task takes to accomplish. This
% timeframe may range anywhere from 2 Action Turns to 2
% years. While engaged in a Task Action, you may not also
% undertake a Complex Action, though in some cases you
% may take a break from the task and return to it later. For
% more information, see Task Actions, p. 120.

% Examples of Task Actions include: repairing a
% device, programming, conducting a scientific analysis,
% searching a room, climbing a wall, or cooking a meal.



% MOVEMENT
% Movement in Eclipse Phase is handled just like any
% other action, and may change from Action Phase to
% Action Phase. Walking and running both count as Quick
% Actions, as they do not require your full concentration.
% The same also applies to slithering, crawling, floating,
% hovering, or gliding. Running, however, may inflict a
% –10 modifier on other actions that are affected by your
% jostling movement. Even more, sprinting is an all-out run,
%  and so requires a Complex Action (see Sprinting, p. 191).

% At the gamemasters discretion, other movement
%  may also call for a Complex Action. Hurdling a fence,
% pole vaulting, jumping from a height, swimming, or
% freerunning through a habitat in zero-gravity all re-
% quire a bit of finesse and attention to detail, so would

% %%% txt/193.txt
% count as a Complex Action, and would apply the
% same modifier as running. Flying generally counts as a
% Quick Action, though intricate maneuvers would call
% for a Complex Action.

% MOVEMENT RATES
% Sometimes it’s important to know not just how you’re
% moving, but how far. For most of transhumanity, this
% movement rate is the same: 4 meters per Action Turn
% walking, 20 meters per turn running. To determine
% how far a character can move in a particular Action
% Phase, divide this movement rate by the total number
% of Action Phases in that turn. In a turn with 4 Action
% Phases, that breaks down to 1 meter walking per
% Action Phase, 5 meters running.

% Movement such as swimming or crawling bench-
% marks at about 1 meter per Action Turn, or 0.25
% meters per Action Phase. You can also sprint to in-
% crease your movement rate (see Sprinting). Vehicles,
% robots, creatures, and unusual morphs will have indi-
% vidual movement rates listed in the format of walking
% rate/running rate in meters per turn.

% These movement rates assume standard Earth
% gravity of course. If you’re moving in a low-gravity,
% microgravity, or high-gravity environment, things
% change. See Gravity, p. 198.

% JUMPING
% Characters making a running jump can cross SOM ÷
% 5 (round up) meters; use SOM ÷ 20 (round up) meters
% for standing jumps. Vertical jumping height is 1 meter.
% Characters making a Freerunning Test can increase
% jumping distance by 1 meter (running jump) or 0.25
% meters (standing/vertical jumps) per 10 points of MoS.

% SPRINTING
% You may use Freerunning to increase the distance
% you move during an Action Phase. You must spend
% a Complex Action to sprint and make a Freerunning
% Test. Every 10 points of MoS increases your running
% distance in that Action Phase by 1 meter, to a maxi-
% mum bonus of +5 meters.
% COMBAT
% Sometimes words fail, and that’s when the knives and
% shredders come out. All combat in Eclipse Phase is
% conducted using the same basic mechanics, whether it’s
% conducted with claws, fists, weapons, guns, or psi: an
% Opposed Test between the attacker and defender(s).

% RESOLVING COMBAT
% Use the following sequence of steps to determine the
% outcome of an attack.

% STEP 1: DECLARE ATTACK
% The attacker initiates by taking a Complex Action to
% attack on their turn during an Action Phase. The skill
% employed depends on the method used to attack. If
% the character lacks the appropriate Combat skill, they
% must default to the appropriate linked aptitude.

% STEP 2: DECLARE DEFENSE
% Once the attack is declared, the defender chooses how
% to respond. Defense is always considered an Automatic
% Action unless the defender is surprised (see Surprise,
% p. 204) or somehow incapacitated and incapable of
% defending themself.

% Melee: A character defending against melee at-
% tacks uses Fray skill, representing dodging (if the
% character lacks this skill, they may default to Re-
% flexes). Alternately, the character may use a melee
% combat skill to defend, representing blocks and par-
% ries rather than dodging.

% Ranged: Against ranged attacks, a defending char-
% acter may only use half their Fray skill (round down).

% Full Defense: Characters who have taken a Com-
% plex Action to go on full defense (p. 198) receive a
% +30 modifier to their defensive roll.

% Psi: A character defending against a psi attack rolls
% WIL x 2 (p. 222). A mental sort of full defense may
% also be rallied against psi attacks.

% %%% txt/194.txt
% STEP 3: APPLY MODIFIERS
% Any appropriate modifiers are now applied to the
% attacker and defender’s skills. See the Combat Modi-
% fiers table (p. 193) for common situational modifiers.

% STEP 4: MAKE THE OPPOSED TEST
% The attacker and defender both roll d100 and com-
% pare the results to their modified skill target numbers.

% STEP 5: DETERMINE OUTCOME
% If the attacker succeeds and the defender fails, the
% attack hits. If the defender succeeds but the at-
% tacker fails, the attack misses completely.

% If both attacker and defender succeed in their
% tests, compare their dice rolls. If the attacker’s
% dice roll is higher, the attack hits despite a spirited
% defense; otherwise, the attack fails to connect.

% Excellent Success: If the attacker rolled an Ex-
% cellent Success (MoS of 30+), a solid hit is struck.
% Increase the Damage Value (DV) inflicted by +5. If
% the MoS is 60+, increase the DV by +10.

% Criticals: If the attacker rolls a critical success,
% the attack is armor-defeating, meaning that the
% defender’s armor is bypassed completely—some
% kink or flaw was exploited, allowing the attack to
% get through completely.

% If the defender rolls a critical success, the attack-
% er’s weapon breaks, jams, gets stuck somewhere,
% or otherwise malfunctions or gets dropped.

% STEP 6: MODIFY ARMOR
% If the target is hit, their armor will help to pro-
% tect them against the attack (unless the attacker
% rolled a critical, see above). Determine which type
% of armor is appropriate to defending against that
% particular attack (see Armor, p. 194). The attack’s
% Armor Penetration (AP) value reduces the armor’s
% rating, however, representing the weapon’s ability
% to pierce through protective measures.

% STEP 7: DETERMINE DAMAGE
% Every weapon and type of attack has a Damage
% Value (DV, see p. 207). This amount is reduced
% by the target’s AP-modified armor rating. If the
% damage is reduced to 0 or less, the armor is ef-
% fective and the attack fails to injure the target.
% Otherwise, any remaining damage is applied to the
% defender. If the accumulated damage exceeds the
% defender’s Durability, they are incapacitated and
% may die (see Durability and Health, p. 207).

% Note that some psi attacks inflict mental stress
% rather than physical damage (see Mental Health, p.
% 209). In this case, the Stress Value (SV) is handled
% the same as DV.

% STEP 8: DETERMINE WOUNDS
% The damage inflicted from a single attack is then
% compared to the victim’s Wound Threshold. If
% the armor-modified DV equals or exceeds the
% Wound Threshold, the character suffers a wound.
% Multiple wounds may be applied with a single attack
% if the modified DV is two or more factors beyond the
% Wound Threshold. Wounds represent more serious
% injuries and apply modifiers and other effects to the
% character (see Wounds, p. 207).



% Stoya tried to get off the station quickly, but the Night

% Cartel’s assassin caught up, surprising her in a micro-

% gravity part of the habitat. The assassin’s INIT is 63,

% plus a dice roll of 23, for an Initiative of 86. Stoya’s

% INIT is 55, plus a roll of 27, for an Initiative of 82.


% The assassin goes first, spending a Quick Action

% to draw a shredder. This flechette weapon is in

% burst-fire mode, so with a Complex Action the as-

% sassin can take two shots. His Spray Weapons skill

% is 65, he’s smartlinked (+10), and they’re at short

% range (+0), so he needs a 75 or less. Stoya is de-

% fending with her Fray skill (60) divided by 2, or 30.


% The assassin rolls an 08 with the first shot. Amaz-

% ingly, Stoya rolls a 28. They both succeeded, but

% Stoya rolled higher, so she dodges the first shot.


% The assassin rolls a 20 for his second shot, an-

% other hit, and this time Stoya rolls an 83, a failure.

% The assassin also scored an Excellent Success with

% a MoS of 55, increasing the DV by +5.


% The assassin’s base damage is 2d10 + 5, but he’s

% using burst fire against a single target for +1d10,

% and it’s also a cone effect weapon at short range,

% for an additional +1d10, for a total DV of 4d10 +

% 5. The assassin rolls 4d10 and gets 16, then adds

% the +5 for a total DV of 21.


% Stoya’s wearing light body armor (AV 10/10),


%                                                          EXAMPLE





% but the shredder’s Armor penetration is –10, so her

% armor is entirely negated. She takes a devastating 21

% DV, exceeding her Wound Threshold of 10, not just

% once, but twice. This means Stoya suffers 2 wounds

% from the shot, suffering –20 to all actions. In addi-

% tion, she must make two SOM x 3 Tests, one to avoid

% knockdown and the other to avoid unconsciousness.

% Her SOM is 30, meaning she needs a 70 (30 x 3 =

% 90, 90 – 20 wound modifiers = 70) on both rolls.

% She rolls a 40 and a 27, succeeding both.


% Now it’s Stoya’s action. She takes a Quick Action

% to pull her own weapon: a stunner. Her Beam

% Weapons skill is 47, modified by wounds (–20) and

% a smartlink (+10), for 37. The assassin’s Fray is 48,

% divided by 2 for 24 against a ranged attack. Stoya

% rolls a 22—a critical hit—and the assassin rolls a

% 68. The stunner only inflicts 1d10 ÷ 2 DV, but since

% the attack is a critical hit, this is armor defeating.

% Stoya rolls an 8, for 4 points of DV, below the as-

% sassin’s Wound Threshold of 7.


% Stunners, however, are shock weapons, so the

% assassin must make a DUR + Energy Armor Test.

% His DUR is 35 and he’s wearing an armor vest (AV

% 6/6), so his target number is 41. He rolls a 71—a

% Margin of Failure of 30, meaning he is immediately

% incapacitated for 3 Action Turns.


% Having disabled her opponent, Stoya takes the

% time to make a hasty getaway.

% %%% txt/195.txt
% COMBAT SUMMARY
% • Combat is handled as an Opposed Test.
% • Attacker rolls attack skill +/– modifiers.
% • Melee: Defender rolls Fray or melee skill

% +/– modifiers.
% • Ranged: Defender rolls (Fray skill ÷ 2, round

% down) +/– modifiers.
% • If attacker succeeds and rolls higher than the

% defender, the attack hits.
% • Critical hits are armor-defeating (armor does

% not apply).
% • Armor is reduced by the attack’s Armor

% Penetration value (AP).
% • The weapon’s damage is reduced by the

% target’s modified Armor rating (unless the

% attack is armor-defeating).
% • If the damage exceeds the target’s Wound

% Threshold, a wound is also scored. (If the

% damage exceeds the Wound Threshold

% by multiple factors, multiple wounds are

% inflicted.)



% ACTION AND COMBAT COMPLICATIONS
% Combat isn’t quite as simple as deciding if you hit or
% miss. Weapons, armor, ammunition, and numerous
% other factors may impact an attack’s outcome. Like-
% wise, various factors can impact an action scene, such
% as fire or microgravity effects.

% AIMED SHOTS
% As noted under Aiming, p. 190, a character can sacri-
% fice their other Quick Actions to concentrate on tar-
% geting a ranged attack and receive a +10 modifier on
% the attack. You can also sacrifice an entire Complex
% Action to fix your aim on a target. In this case, as long
% as the target remains in your sights until your next
% Action Phase, you receive a +30 modifier to hit.

% AMMUNITION AND RELOADING
% Every weapon has a listed ammunition capacity that
% indicates how many shots the weapon can carry or
% holds. When this ammo runs out, a new supply must
% be loaded in. Players should keep track of the shots
% they fire.

% Reloading almost always requires a Complex
% Action, whether you are slapping in a new clip of bul-
% lets or a fresh battery for a laser. At the gamemaster’s
% discretion, a reload that is immediately accessible
% (such as a new clip reverse-taped to the loaded clip, so
% that reloading just requires that you reverse the taped
% clips and slot the new one in) will only take a Quick
% Action. Archaic weapons such as magazine-fed rifles
% may require longer to fully load.

% AREA EFFECT WEAPONS
% Some ranged attack weapons are designed to affect
% more than one target at a time. These weapons fall
% into three categories: blast, uniform blast, and cone.


%                  COMBAT MODIFIERS
% GENERAL                                                            MO
% Character using off-hand
% Character wounded/traumatized                                –10 per w
% Character has superior position
% Touch-only attack
% Called shot
% Character wielding two-handed weapon with one hand
% Small target (child-sized)
% Very small target (mouse or insect)
% Large target (car sized)
% Very large target (side of a barn)
% Visibility impaired (minor: glare, light smoke, dim light)
% Visibility impaired (major: heavy smoke, dark)
% Blind attack


% MELEE COMBAT                                                       MO
% Character has reach advantage
% Character charging or receiving a charge


% RANGED COMBAT (ATTACKER)                                           MO
% Attacker using smartlink or laser sight
% Attacker behind cover
% Attacker running
% Attacker in melee combat
% Attacker has reach advantage
% Defender has minor cover
% Defender has moderate cover
% Defender has major cover
% Defender prone and far (10+ meters)
% Defender hidden
% Aimed shot (quick)
% Aimed shot (complex)
% Sweeping fire with beam weapon                                  +10 on
% Multiple targets in same Action Phase                        –20 per a
% Indirect fire
% Point-blank range (2 meters or less)
% Short range
% Medium range
% Long range
% Extreme range


%  BLAST EFFECT
%  Blast weapons include items like grenades, mines, and
%  other explosives that expand outward from a central
%  detonation point. Most blast attacks expand outward
%  in a sphere, though certain shaped charges may direct
%  an explosion in one direction only. The explosive force
%  is stronger near the epicenter and weaker near the
%  outer edges of the sphere. For every meter a target is
%  from the center, reduce the damage of a blast weapon
%  by –2.

% %%% txt/196.txt
% UNIFORM BLAST
% Uniform blast attacks distribute their power evenly
% throughout the area of effect. Examples include fuel-
% air explosives and thermobaric weapons that disperse
% an explosive mixture in a vapor cloud and ignite it all
% at once. All targets within the noted blast radius suffer
% the same damage. Damage against targets outside of
% the main blast sphere is reduced by –2 per meter.

% CONE
% Weapons with a cone effect have an area effect that
% begins with the tip of the weapon and expands out-
% ward in a cone. At short range, this attack effects 1
% target; at medium range it affects 2 targets within a
% meter of each other; and at long or extreme range
% it affects 3 targets within a meter of the next. Cone
%  effect attacks do +1d10 damage at short range and
% –1d10 damage at long and extreme range.

% ARMOR
% Just as weapons technologies have advanced, so too
% has armor quality, allowing unprecedented levels of
% protection. As noted in Step 7: Determine Damage
% (see p. 192), the armor rating reduces the damage
% points of the attack.

% For a full listing of armor types and values, see p.
% 311.

% ENERGY VS. KINETIC
% Each type of armor has an Armor value (AV) with two
% ratings—Energy and Kinetic—representing the protec-
% tion it applies against the respective type of attack.
% These are listed in the format of “Energy armor/Kinetic
% armor.” For example, an item with listed armor “5/10”
% provides 5 points of armor against energy-based at-
% tacks and 10 points of armor against kinetic attacks.

% Energy damage includes that caused by beam weap-
% ons (laser, microwave, particle beam, plasma, etc.) as
% well as fire and high-energy explosives. Armor that
% protects against this damage is made of material that
% reflects or diffuses such energy, dissipates and trans-
% fers heat, or ablates.

% Kinetic damage is the transfer of damaging energy
% when an object in motion (a fist, knife, club, or bullet,
% for example) impacts with another object (the target).
% Most melee and firearms attacks inflict kinetic damage,
% as would a rolling boulder, swinging pendulum, or
% explosion-driven fragments. Kinetic armors include
% impact-resistant plates, shear-thickening liquid and
% gels that harden upon impact, and ballistic and cut-
% proof fiber weaves.

% ARMOR PENETRATION
% Some weapons have an Armor Penetration (AP) rating.
% This represents the attack’s ability to pierce through
% protective layers. The AP rating reduces the value of
% armor used to defend against the attack (see Step 6:
% Modify Armor, p. 192).
% LAYERED ARMOR
% If two or more types of armor are worn, the armor
% ratings are added together. However, wearing multiple
% armor units is cumbersome and annoying. Apply a –20
% modifier to a character’s actions for each additional
% armor layer worn

% Note that items specifically noted as armor accesso-
% ries—helmets, shields, etc.—do not inflict the layered
% armor penalty, they simply add their armor bonus.
% Note also that the armor inherent to a synthetic
% morph or bot’s frame does not constitute a layer of
% armor (i.e., you may wear armor over the synthetic
% shell without penalty).

% ASPHYXIATION
% The average transhuman can hold their breath for
% two minutes before blacking out. Strenuous activity
% reduces the amount of time. For every 30 seconds
% after the first minute a biomorph is prevented from
% breathing, they must make a DUR Test. Apply a cu-
% mulative –10 modifier each time this test is rolled. If
% the character fails the test, they immediately fall un-
% conscious and begin to suffer damage from asphyxi-
% ation, at the rate of 10 points per minute, until they
% die or are allowed to breathe again. This damage does
% not cause wounds.

% Asphyxiating is a terrible process, often leading to
% panic. Characters who are being asphyxiated must
% make a WIL x 3 Test. If they fail, they suffer 1d10
% ÷ 2 (round up) mental stress and cannot perform
% any effective action to rescue themselves that Action
% turn. A character who succeeds may attempt to rescue
% themselves, and in fact they must make a WIL x 3
% Test to perform any other action not directly related
% to rescuing themselves (attacks against another char-
% acter, a creature, or an object holding the character
% underwater are exempt from this rule).

% BEAM WEAPONS
%  Due to emitting a continuous beam of energy rather
%  than single projectiles, beam weapons are easier to
% “home in” on a target. This means one of the fol-
%  lowing two rules may be used when making beam
%  weapon attacks. These options are not available for
% “pulse” beam weapons (like the laser pulser), as such
%  weapons emit energy in pulses rather than a continu-
%  ous beam.

% SWEEPING FIRE
% An attacker who is making two semi-auto (p. 198)
% attacks with a beam weapon with the same Complex
% Action and who misses with the first attack may treat
% that attack as a free Aim action (p. 190), receiving a
% +10 modifier for the second attack. In other words,
% though the first attack misses, the character takes the
% opportunity to sweep the beam closer to the target
% for the second attack. This only applies when both
% attacks are made against the same target.

% %%% txt/197.txt
% CONCENTRATED FIRE                                synth
% A character firing a semi-auto beam              Vehicl
% weapon who hits with the first attack may        but th
% choose to keep the beam on and concen-           that th
% trate their fire, cooking the target. In this    vehicl
% case, the character foregos their second         that is
% semi-auto attack with that Complex               assum
% Action, but automatically bolsters the           brain
% DV of the first attack by x 1.5 (round up).      rules,
% This decision must be made before the            bots, v
% damage dice are rolled.                            Lik


%                                             are tr
% BLIND ATTACKS                                    they r
% Attacking a target that you cannot see is        skills
% difficult at best and a matter of luck at        shells
% worst. If you cannot see, you may make a         ever, c
% Perception Test using some other available
% sense to detect your target. If this succeeds,   SHELL
% you attack with a –30 modifier. If your Per-     Just li
% ception Test fails, your attack is primarily     bot an
% based on chance—your target number for           Thres
% the attack test is equal to your Moxie stat      tiative
% (no other modifiers apply).                      the ac


%                                             determ
% INDIRECT FIRE                                    AI (in
% With the help of a spotter, you may target       may a
% an enemy that you can’t see using indirect       AI or
% fire. In this case you must be meshed with       vehicl
% a character, bot, or sensor system that has         Ha
% the target in its sights and which feeds         specia
% you targeting data (the gamemaster may           modifi
% require a Perception Test from the spotter).     the bo
% Indirect attacks suffer a –30 modifier.          vehicl

% Seeker missiles (p. 340) can home in on
% a target that is “painted” with reflected        SHELL
% energy from a laser sight (p. 342) or simi-      The sk
% lar target designator system. An “attack”        vehicl
% must first be made to paint the target with      AIs an
% the laser sight using an appropriate skill.
% If this succeeds, it negates the –30 indirect    SHELL
% fire modifier for the seeker launcher’s          Like c
% attack test. the target must be held in          a wal
% the spotter’s sights (requiring a Complex        This i
% Action each Action Phase) until the seeker       engag
% strikes.                                         other


%                                                She
% BOTS, SYNTHMORPHS, AND VEHICLES                  will a
% AI-operated robots and synthetic morphs          listed—
% are a common sight in Eclipse Phase.             shell m
% Robots are used for a wide range of pur-         per ho
% poses, from surveillance, maintenance, and       some
% service jobs to security and policing—and        accele
% so may often play a role in action and           the ga
% combat scenes. Though less common (at            penalt
% least in habitats), AI-piloted vehicles are
% also frequently used and encountered.            CHASE

% Note that the difference between a             Shells
% robot, vehicle, and synthetic morph is in        runnin
% many ways semantic. Robots are simply            Veloci

% bodies controlled by an AI.
%  re also robotic—AI controlled—

% rm “vehicle” is used to denote
%  carry passengers. Both bots and
%  an be used as synthetic morphs—
%  habited by a transhuman ego—

% they are equipped with a cyber-

% 300). For the purpose of these

% term “shell” is used to refer to
%  cles, and synthetic morphs alike.

% nthmorphs, bots and vehicles

% d just like any other character:
%  Initiative, take actions, and use

% few specific aspects of these

% ds special consideration, how-

% red below.

%  TS
% synthmorph characters, certain
%  ehicle stats (Durability, Wound
% d, etc.) and stat modifiers (Ini-
% peed, etc.) are determined by
% l physical shell. Other stats are
%  d by the bot/vehicle’s operating
% ce of an ego). Bots and vehicles

% have traits that apply to their
%  sical shell. For sample bots and
%  ee p. 342 of the Gear chapter.
%  ng: Bots and vehicles have a
%  at called Handling, which is a
% applied to all tests made to pilot
% ehicle. This represents the bot/
% maneuverability.

%  LLS

% and aptitudes used by a bot or
%  e those possessed by its AI. See
% Muses, p. 264.

%  VEMENT
%  acters, bots and vehicles have
% g and running Movement rate.
%  ed whenever the bot/vehicle is
% n action or combat scenes with
% racters.
% hat are capable of greater speeds

% have a Maximum Velocity
% is is the highest rate at which the
%  safely move, listed in kilometers
%  At the gamemaster’s discretion,
%  lls may push their limits and

% further, but at significant risk—
% master should apply appropriate
%  o Pilot Tests and other tests.



%  t are moving faster than their
% Movement rate (up to their Max.
%  are generally considered to be

% %%% txt/198.txt
% moving too fast for standard action and combat inter-
% action with other characters. This is when the action
% enters “chase scene” mode—a traveling narrative of
% maneuvering choices and tests with various outcomes.
% Whether or not a chase is actually occurring, the
% gamemaster should remember that Max Velocity is
% not the only factor in high-speed situations. Environ-
% mental factors like terrain, weather conditions, navi-
% gation, pedestrians, and traffic can provide obstacles
% for shells to overcome. A shell tearing across a habitat
% in order to reach a bomb before it detonates should
% have to make several decisions and tests that may
% affect whether it gets there in time or not. Likewise, a
% shell seeking to shake off some hot pursuit will have
% to pull off some fancy maneuvering and hopefully find
% a shortcut or two in order to outrace their opponents.

% CRASHING
% Shells that suffer wounds during combat or chases
% may be force to make a Pilot Test to avoid crashing
% or may crash automatically. The exact circumstances
% of a crash are left up to the gamemaster, as best fits
% the story—the shell may simply skid to a stop, plow
% into a tree, fall from the sky, or flip over and land on
% a group of bystanders. Shells that strike other objects
% when they crash typically take further damage from
% the collision (see Collisions).

% COLLISIONS
% If a shell crashes into or intentionally rams a person
% or object, someone is likely to get hurt. To determine
% how much DV is inflicted, roll 1d10 and add the shell’s
% DUR divided by 10 (round up). This is the damage
% applied at walking speeds. If the shell was moving at
% running speeds, multiply the DV by 2. If the shell was
% moving at chase speeds, multiply the DV by the shell’s
% velocity ÷ 10 in meters per turn. Both the shell and
% whatever it strikes suffer this damage, assuming the
% collision is with something equal dense and hard. Soft
% and squishy objects like biomorphs will be less dam-
% aging to a shell (unless they happen to be in a hardsuit
% or battlesuit), in which case the shell will only suffer
% half damage from the collision. Kinetic armor defends
% against crash DV.

% If two moving shells collide head-on, calculate the
% damage from both and inflict to both. If two shells
% moving in the same direction collide, only count the
% difference in velocity.

% Passengers in a vehicle may also be damaged by col-
% lisions if they are not wearing proper safety restraints.
% They suffer one half the DV applied to their vehicle.





%     COLLISION DAMAGE

% Base Collision DV: 1d10 + (DUR ÷ 10)

% Running: DV x 2

% Chase Speeds: DV x (velocity ÷ 10)
% ATTACKING VEHICLE PASSENGERS
% During combat, passengers within a vehicle may be
% targeted separately from the vehicle itself. Attacks
% made against passengers this way do not harm the
% vehicle itself (unless an area effect weapon is used).
% Targeted passengers benefit both from cover (usually
% Major, –30) and from the vehicle’s structure, adding
% the vehicle’s Armor Value to their own.

% Passengers within a vehicle are generally not harmed
% by attacks made against the vehicle itself. Exceptions
% include area effect weapons and vehicle crashes. In
% both these cases, the passengers also benefit from the
% vehicle Armor Value.

% SHELL REMOTE CONTROL
% Any shell (or biomorph) with a puppet sock (also in-
%  cluded with all cyberbrains) may be remote controlled,
%  either by a character or a remote AI. This requires a
%  communications link between the teleoperator and
%  the shell (the “drone”). The teleoperator controls the
%  drone via an entoptic interface, and receives sensory
% input and other data via the drone’s mesh inserts.

% When under direct control, the shell’s AI (or resident
%  ego) is subsumed and put on standby. The drone only
%  acts as instructed. Each instruction counts as a Quick
% Action. In this situation, the drone acts with the same
% Initiative as the teleoperator. The teleoperator’s skills and
%  stats are used in place of the shell AI’s. Due to the nature
%  of remote operation, however, all tests are made with a
% –10 modifier. Multiple drones may be controlled at once,
% but commanding them requires separate Quick Actions
% unless they are receiving the same command. Direct
%  control teleoperation is not very feasible at extreme dis-
%  tances, due to the light-speed lag with communications.

% Alternately, the teleoperator can put the drone in
% autonomous mode, allowing the shell’s AI to resume
%  normal operations. The drone still follows the teleop-
%  erator’s commands to the best of its abilities. In this
%  mode, the drone functions normally, using its own
% Initiative and AI skills and stats.

% SHELL JAMMING
% “Jamming” is the colloquial term for a more direct
%  form of remote-control, using VR and XP technol-
%  ogy. When jamming, the drone’s puppet sock feeds
%  the drone’s sensory data directly to the teleoperator’s
%  mesh inserts. The teleoperator subsumes themself in
%  the drone’s sensorium, essentially “becoming” the
%  drone. In this case, the teleoperator surrenders con-
%  trol of their own morph, which slumps inertly. While
%  jamming, they suffer –60 on all Perception Tests or
%  attempts to take action with their morph.

% Jamming takes a Complex Action to engage and
%  disengage. A jamming teleoperator controls a drone
%  as if it were their own morph. Like direct control te-
%  leoperation, the jammer’s own skills and Initiative are
%  used in place of the drone’s AI. Jammers do not suffer
%  any teleoperation modifiers, but only one drone may
%  be jammed at a time.

% %%% txt/199.txt

% If the drone is killed or destroyed, the jammer is
% immediately dumped from their connection, resuming
% control of their own morph as normal. Getting dumped
% in this manner is extremely jarring, not the least be-
% cause the jammer experienced being killed/destroyed.
% As a result, the jammer suffers 1d10 mental stress.

% CALLED SHOTS

% Sometimes it’s not enough to just hit your target—
% you need to shoot out a window, knock the knife out
% of their hand, or hit that hole in their armor. You may
% declare that you are making a called shot before you
% initiate an attack, choosing one of the outcomes noted
% below. Called shots suffer a –10 modifier and require
% an Excellent Success (MoS 30+). If you beat that
% margin, you succeed with the called shot, and the re-
% sults noted below apply. If you don’t beat the margin
% but still succeed in the attack, you simply strike your
% target as normal.

% BYPASSING ARMOR
% Called shots may be used to target a hole or weak
% point in your opponent’s armor. If you beat the MoS,
% you strike an armor-defeating hit, and their armor
% does not apply. Note that in certain circumstances, a
% gamemaster may rule that an opponent’s armor simply
% doesn’t have a weak spot or unprotected area, and so
% disallow such called shots.

% DISARMING
% You may take a called shot to attempt to knock a
% weapon out of an opponent’s hand(s). If you beat the
% MoS, the victim suffers half damage from the attack (re-
% duced by armor as normal) and must make a SOM x 3
% Test with a –30 modifier to retain hold of the weapon.

% SPECIFIC TARGETING
% You may make a called shot with the intention of hitting
% a specific location or component on your target—for
% example: disabling the sensor unit on a bot, sweeping
% someone’s leg, or poking someone in the eye. If you beat
% the MoS, you hit the specific targeted spot. The game-
% master determines the result as appropriate to the attack
% and target—the component may be destroyed, the op-
% ponent may fall or be temporarily blinded, and so on.

% CHARGING
% An opponent who runs and attacks an opponent in
% melee combat in the same Action Phase is considered
% to be charging. A charging attacker still suffers the
% –10 modifier for running, but they receive a damage
% bonus on account of their momentum: increase the
% damage they inflict by +1d10.

% RECEIVING A CHARGE
% You may delay your action (see p. 189) in order to
% receive a charge, bracing yourself for impact, inter-
% rupting their action, and striking right before your
% charging does. In this situation, you receive a +20
% modifier for striking the charging opponent.
% DEMOLITIONS
% The most common use of the Demolitions skill is the
% placement, disarming, or manufacture of explosive
% devices, such as superthermite charges (p. 330) or
% grenades (p. 340).

% PLACING EXPLOSIVES
% A skilled demolitionist can place charges in a manner
% that will boost their effect. They can identify struc-
% tural vulnerabilities and weak points and focus a
% blast in these areas. They can determine how to blast
% open a safe without destroying the contents. They can
% focus the force of an explosion in a particular direc-
% tion, increasing the directed force while minimizing
% splash effects.

% Each of these scenarios calls for a successful De-
% molitions Test. The exact result is determined by
% the gamemaster according to the specific scenario.
% For example, using the examples above, targeting
% a weak point could double the damage inflicted on
% that structure. Shaping the charge to direct the force
% can triple the damage in that direction, as noted in
% the superthermite description (p. 330). An Excellent
% Success is likely to increase an explosive’s damage by
% +5, whereas a critical success would allow the blast to
% ignore armor.

% DISARMING
% Disarming an explosive device is handled as an Op-
% posed Test between the Demolitions skills of the dis-
% armer and the character who set the bomb.

% MAKING EXPLOSIVES
% A character trained in Demolitions can make ex-
% plosives from raw materials. These materials can be
% gathered the traditional way or they can be manufac-
% tured using a nanofabricator. Even nanofabbers with
% restricted settings to prevent explosives creation can
% be used, as explosives can be constructed from all
% manner of mundane chemicals and materials.

% The timeframe for making explosives is 1 hour per
% 1d10 points of damage the explosive will inflict. If a
% critical failure is rolled, the demolitionist may acciden-
% tally blow himself up, or the charge may be extremely
% weaker or more potent than expected (whichever is
% more likely to be disastrous).

% FALLING
% If a character falls, use the Falling Damage table to
% determine what injuries they suffer. Kinetic armor
% will mitigate this damage at
% half its value (round down).
% Gamemasters may also reduce             FALLING DA
% this damage if anything helped
% to break the fall (branches, soft      DISTANCE FALLEN
% surface) at their discretion.           1–2 meters


%                                    3–5 meters


%                                    6–8 meters


%                                    Over 8 meters

% %%% txt/200.txt
% FIRE
% Objects that come into contact with extreme heat or
% flames may catch fire at the gamemaster’s discretion,
% keeping in mind both the flammability of the material
% and the strength of the heat/flames. Burning items (or
% characters) will suffer 1d10 ÷ 2 (round up) damage
% each Action Turn unless otherwise noted. Energy
% armor will protect against this damage, though it
% too may catch fire, reducing its value by the damage
% inflicted. Depending on the environmental conditions,
% fires are likely to grow larger unless somehow abated.
% Every 5 Action Turns, increase the DV inflicted (first
% to 1d10, then 2d10, then 3d10, then by increments
% of +5). Adverse conditions (such as rain) or efforts to
% extinguish the blaze will reduce the DV accordingly.

% Note that fire does not burn in vacuum. In micro-
% gravity, fire burns in a sphere and grows more slowly,
% as expanding gases push away the oxygen (increase
% the DV every 10 Action Turns). If there is a lack of air
% circulation, some microgravity fires may extinguish
% themselves.

% FIRING MODES AND RATE OF FIRE
% Every ranged weapon in Eclipse Phase comes with one
% or more firing modes that determines their rate of fire.
% These firing modes are detailed below.

% SINGLE SHOT (SS)
% Single shot weapons may only be fired once per
% Complex Action. These are typically larger or more
% archaic devices.

% SEMI-AUTOMATIC (SA)
% Semi-automatic weapons are capable of quick, re-
% peated fire. They may be fired twice with the same
% Complex Action. Each shot is handled as a separate
% attack.

% BURST FIRE (BF)
% Burst fire weapons release a number of quick shots
% (a “burst”) with a single trigger pull. Two bursts may
% be fired with the same Complex Action. Each burst
% is handled as a separate attack. Bursts use up 3 shots
% worth of ammunition.

% A burst may be shot against a single target (concen-
% trated fire), or against two targets who are standing
% within one meter of each other. In the case of concen-
% trated fire against a single target, increase the DV by
% +1d10.

% FULL AUTOMATIC (FA)
% Full-auto weapons release a hail of shots with a single
% trigger pull. Only one full-auto attack may be made
% with each Complex Action. This attack may be made a
% single target or against up to three separate targets, as
% long as each is within one meter of another. In the case
% of a concentrated fire on a single individual, increase
% the DV by +1d10 + 10. Firing in full automatic mode
% uses up 10 shots.

% FULL DEFENSE
% If you’re expecting to come under fire, you can
% expend a Complex Action to go on full defense. This
% represents that you are expending all of your energy
% to dodge, duck, ward off attacks, and otherwise get
% the hell out of the way until your next Action Phase.
% During this time, you receive a +30 modifier to defend
% against all incoming attacks.

% Characters who are on full defense may use Free-
% running rather than Fray skill to dodge attacks, repre-
% senting the gymnastic movements they are making to
% avoid being hit.

% GRAVITY
% Most characters in Eclipse Phase have considerable
% experience maneuvering in low gravity or micro-
% gravity and can perform normal actions without
% penalties. Even characters who grew up on planetary
% bodies or in rotating habitats have some familiarity
% with alternate gravities thanks to childhood train-
% ing in simulspace educational scenarios. The same is
% also true in reverse; characters who grew up in free

% %%% txt/201.txt
% fall have likely experienced simulations of life in a
% gravity well.

% At the gamemaster’s discretion, characters who
% have spent long periods acclimating to one range of
% gravity may find a shift in conditions a bit challeng-
% ing to cope with, at least until they grow accustomed
% to the new gravity. In this case, the gamemaster can
% apply a –10 modifier to both physical and social skills.
% The physical penalty results from simple difficulties
% in maneuvering. The social penalty applies because
% it’s hard to look impressive, intimidating, or seductive
% when you haven’t figured out how to arrange your
% clothes so that they don’t float up into your face. The
% physical penalty can be increased to –20 for situa-
% tions involving combat skills and skills requiring fine
% manipulation, building, or repairing of items. These
% penalties will apply until the character adjusts, which
% typically takes about 3 days.

% Any biomorph with basic biomods (p. 300) is
% immune to ill health from the effects of long-term
% exposure to microgravity.

% MICROGRAVITY
% Microgravity includes both zero-G and gravities that
% are slightly higher but negligible. These conditions are
% found in space, on asteroids and some small moons,
% and on (parts of) spaceships and habitats that are not
% rotated for gravity. Objects in microgravity are effec-
% tively weightless, but size and mass are still factors.

% Things behave differently in microgravity. For
% example:

%  • Objects not anchored down will tend to drift

% off in whatever direction they were last moving.

% Floating objects will eventually settle in the direc-

% tion of the densest part of the habitat or spacecraft.
%  • Thrown or pushed items will travel in a straight

% line until they hit something.
%  • Smoke does not rise in streams. Instead, it forms

% a roughly spherical halo around its source.
%  • Liquids have little cohesion, scattering into clouds

% of tiny droplets if released into the air. Drinks

% come in sealed bulbs or bottles. Food is eaten so

% that sauces and bits of liquid don’t escape. Blood

% goes everywhere.


% Movement and maneuvering in microgravity is
% handled using Free Fall skill (p. 179). Most everyday
% activity in free fall does not require a test. The game-
% master can, however, call for a Free Fall Test for any
% complicated maneuvers, flying across major distances,
% sudden changes in direction or velocity, or when en-
% gaged in melee combat. A failed roll means the char-
% acter has miscalculated and ends up in a position other
% than intended. A Severe Failure means the character has
% screwed up badly, such as slamming themselves into a
% wall or sending themselves spinning off into space.

% For convenience, most microgravity habitats
% feature furniture covered with elastic loops, mesh
% pockets to keep individual objects from floating all
% over the place, and moving beltways with hand loops
% for major thoroughfares. Magnetic or velcro shoes
% are also used to walk around, rather than climbing
% or flying. Zero-g environments are often designed to
% make maximum use of space, however, taking advan-
% tage of the lack of ceilings and floors. Because object
% are weightless, characters can move even massive
% objects around easily.

% Movement Rate: Characters who are climbing, pull-
% ing, or pushing themselves along move at half their
% movement rate (p. 191) in microgravity.

% Terminal Velocity: It is not difficult to reach escape
% velocity on small asteroids and similar bodies—
% something to keep in mind with thrown objects and
% projectile weapons. In some cases, characters who
% move fast enough and jump can reach escape veloc-
% ity themselves, though these situations are left up to
% the gamemaster.

% LOW GRAVITY
% Low gravity includes anything from 0.5 g to micro-
% gravity. These conditions are found on Luna, Mars,
% Titan, and the rotating parts of most spun spacecraft
% and habitats. Low gravity is not that different from
% standard gravity, though characters may jump twice as
% far and thrown/projectile objects have a longer range
% (p. 203). Increase the running rate for characters in
% low gravity by x1.5.

% HIGH GRAVITY
% High gravity is anything significantly stronger than
% standard Earth gravity (1.2 g +). High gravity in
% Eclipse Phase is typically only found on exoplanets.
% High gravity can be particularly hard on characters
% as their bodies are strained because they carry more
% weights, muscles are fatigued from needing to push
% more around, and the heart must work harder to
% pump blood. For every 0.2 g over 1 that a character is
% not acclimated to, treat it as if the character is suffer-
% ing from the effects of 1 wound. At the gamemaster’s
% discretion, movement rates may also be modified.

% GRENADES AND SEEKERS
% Modern grenades, seekers, and similar explosives do
% not necessarily detonate the instant they are thrown
% or strike the target. In fact, several trigger options
% are available, each set by the user when deploying
% the weapon. Missed attacks, or attacks that do not
% explode in transit or when they strike, are subject to
% scatter (p. 204).

% Airburst: Airburst means that the device explodes in
% mid-air as soon as it travels a distance programmed at
% launch. In this case, the explosive’s effects are resolved
% immediately, in that user’s Action Phase. Note that
% airburst munitions are programmed with a safety fea-
% ture that will prevent detonation if they fail to travel
% a minimum precautionary distance from the launcher,
% though this can be overridden.

% Impact: The grenade or missile goes off as soon as it
% hits something, whether that be the target, ground, or

% %%% txt/202.txt
% an intervening object. Resolve the effects immediately,
% in the user’s Action Phase.

% Signal: The munition is primed for detonation upon
% receiving a command signal via wireless link. The
% device simply lays in wait until it receives the proper
% signal (which must include the cryptographic key
% assigned when the grenade was primed), detonating
% immediately when it does.

% Timer: The device has a built-in timer allowing
% the user to adjust exactly when it detonates. This
% can be anywhere from 1 second to days, months, or
% even years later, effectively using the device like a
% bomb, but also increasing the likelihood it will be
% discovered and neutralized. The minimum detona-
% tion period—1 second—means that the munition will
% detonate on the user’s (current) Initiative Score in the
% next Action Phase. A 2-second delay would last two
% Action Phases, a 3-second delay three Action phases,
% and so on.

% THROWING BACK GRENADES
% It is possible that a character may be able to reach
% a grenade before it detonates and throw it back (or
% away in a safe direction). The character must be
% within movement range of the grenade’s location, and
% must take a Complex Action to make a REF + COO +
% COO Test to catch the rolling, sliding grenade. If they
% succeed, they may throw the grenade off in a direc-
% tion of their choice with the same action (treat as a
% standard throwing attack).

% If the character fails the test, however, they may find
% themselves at ground zero when it detonates.

% JUMPING ON
% Given the possibility of resleeving, a character may
% decide to take one for the team and throw themselves
% on a grenade, sacrificing themselves in order to protect
% others. The character must be within movement range
% of the grenade’s location, and must take a Complex
% Action to make a REF + COO + WIL Test to fall on
% the grenade and cover it with their morph. This means
% the character suffers an extra 1d10 damage when the
% grenade detonates. On the positive side, the grenade’s
% damage is reduced by the sacrificing character’s armor
% + 10 when its damage effects are applied to others
% within the blast radius.

% If the gamemaster feels it appropriate, a WIL x 3
% Test might be called for in order for a character to
% sacrifice themselves in this manner.

% HOSTILE ENVIRONMENTS

% The solar system might be friendly to life on a grand
% scale, but if you’re stranded in the gravity well of Jupi-
% ter during a magnetic storm, trying to breathe without
% a respirator on Mars, or swimming in hard vacuum
% without a space suit, it doesn’t seem so friendly. This
% section describes a few of the hostile environments
% that characters in Eclipse Phase might face.
% ATMOSPHERIC CONTAMINATION
% Habitats sometimes fall ill. The effects of a habitat
%  suffering from ecological imbalance or out-of-control
% pathogens can range from mildly allergenic habitat
% atmospheres to rampaging environmental sepsis.
% Characters without breathing or filtration gear in a
%  contaminated environment should suffer penalties
% to physical and possibly social skills, ranging from
% –10 (mild contamination) to –30 (severely septic at-
% mosphere). Depending on the contamination, other
%  effects may apply, as the gamemaster sees fit.

% EXTREME HEAT AND COLD
% Planetary environments can range from the extremely
% hot (Venus, Mercury’s day side) to the extremely frigid
% (Neptune, Titan, Uranus). Both are likely to kill an un-
% protected and unmodified biomorph within minutes, if
% not seconds. Synthmorphs and vehicles fare better, es-
% pecially in the cold, but even they are likely to quickly
% succumb to the blazing furnaces of the inner planets
% without strong heat shields and cooling systems.

% EXTREME PRESSURE

% Similarly, the atmospheric pressures of Jupiter,
% Saturn, and Venus quickly become crushingly deadly
% anywhere beyond the upper levels. Only synthmorphs
% and vehicles with special pressure adaptations can
% hope to survive such depths.

% GRAVITY TRANSITION ZONES

% The widespread use of artificial gravity in space
% habitats means that characters will often encounter
% places where the direction of down suddenly changes.
% In most rotating habitats, the standard design in-
% cludes an axial zone where spacecraft can dock in
% microgravity and a carefully designed and marked
% transition zone (usually an elevator) where people and
% cargo coming and going from the axial spaceport can
% orient to local “down” and be standing in the right
% place when gravity takes effect. Gravity transitions in
% rotating habitats are almost always gradual but can
% be very dangerous if a character encounters them in
% the wrong place or time.

% A character cast adrift in the microgravity zone at
% the axis of a rotating space habitat will slowly drift
% outward until they begin to encounter gravity, at
% which point they will fall. How long this takes varies
% on the size of the habitat. A good rule of thumb is that
% for each kilometer of diameter possessed by the habi-
% tat, the character has 30 seconds before they begin to
% fall. If the character was given a good push out from
% the axis when set adrift, this time should be halved,
% quartered, or more at the gamemaster’s discretion.

% MAGNETIC FIELDS
% Magnetism isn’t a direct problem for most characters;
% transhumans need to worry more about the radiation
% generated by a powerful magnetosphere. For un-
% shielded electronic devices and similarly unshielded
% transhumans sporting titanium, however, the effects

% %%% txt/203.txt
% of strong magnetic fields can be devastating. Note that
% many of the conditions that result in vehicles, bots,
% and gear being exposed to strong magnetic field activ-
% ity coincide with strong radioactivity.

% Magnetic fields affect synthmorphs, robots, vehicles,
% cybernetic implants, and electronics after 1 minute of
% exposure. Like radiation exposure, these effects can
% vary drastically. At the low end, communication and
% sensors will suffer interference and shortened ranges;
% at high ends, electronic systems will simply suffer
% damage and fail.

% RADIATION
% Ionizing radiation is one of the more prevalent haz-
% ards in the solar system and one of the most difficult
% problems for transhumanity to defeat. Radiation
% damages genetic material, sickens, and kills by ion-
% izing the chemicals involved in cell division within the
% human body. Effects range from nausea and fatigue to
% massive organ failure and death. However, radiation
% sickness is not solely a somatic phenomenon. The real
% terror of radiation for transhumans, especially at high
% dose levels such as those experienced on the surface of
% Ganymede and other Jovian moons, is damage to the
% neural network. This can lead to flawed uploads and
% backups. Nanomedicine that can rapidly reverse the
% ionization of cellular chemicals and new materials that
% offer thinner and better shielding help, but the sheer
% magnitude of the radiation put out by some bodies in
% the solar system defeats even these.

% Radiation poisoning is a complicated affair, and de-
% tailed rules are beyond the scope of this book. Sources
% of radiation include the Earth’s Van Allen belt, Jupi-
% ter’s radiation belt, Saturn’s magnetosphere, cosmic
% rays, solar flares, fission materials, unshielded fusion
% or antimatter explosions, and nuclear blasts, among
% others. Effects can vary drastically depending on the
% strength of the source, the amount of time exposed,
% and the level of shielding available. The immediate
% effects on biomorphs (manifesting anywhere from
% within minutes to 6 hours) can include nausea, vomit-
% ing, fatigue (reduced SOM), as well as both physical
% damage and minor amounts of mental stress. This
% is followed by a latency period where the biomorph
% seems to get better, lasting anywhere from 6 hours
% to 2 weeks. After this point, the final effects kick in,
% which can include hair loss, sterility, reduced SOM
% and DUR, severe damage to gastric and intestinal
% tissue, infections, uncontrolled bleeding, and death.

% Synthmorphs are not quite as vulnerable as
% biomorphs, but even they can be damaged and dis-
% abled by severe radiation dosages.

% TOXIC ATMOSPHERE
% Neptune, Titan, Uranus, and Venus all have toxic
% atmospheres. Similar atmospheres may be found on
% some exoplanets, or might be intentionally created as
% a security measure within a habitat or structure.

% A character who is unaware of atmospheric toxicity
% and does not immediately hold their breath (requir-
% ing a REF x 3 Test) suffers 10 points of damage per
% Action Turn. A character who manages to hold their
% breath can last a bit longer; apply the rules for as-
% phyxiation (p. 194).

% Corrosive Atmospheres: In addition to being toxic,
% Venus has the only naturally occurring corrosive at-
% mosphere in the system. Corrosive atmospheres are
% immediately dangerous: characters take 10 points of
% damage per Action Turn, regardless of whether they
% hold their breath or not. Corrosive atmospheres also
% damage vehicles and gear not equipped with anti-
% corrosive shielding. Such items take 1 point of damage
% per minute. At greater concentrations, such as in the
% dense sulfuric acid clouds in the upper atmosphere of
% Venus, items takes 5 points of damage per minute.

% UNBREATHABLE ATMOSPHERE
% Very few of the planetary bodies in the solar system
% actually have toxic atmospheres. In most unbreathable
% atmospheres, the primary hazard for transhumans
% without breathing apparatus or modifications to their
% morph is lack of oxygen. Treat exposure to unbreath-
% able atmospheres the same as asphyxiation.

% UNDERWATER
% In general, any physical skill performed underwater suf-
% fers a –20 penalty due to the resistance of the medium.
% Skills relying on equipment not adapted for underwa-
% ter use may be more difficult or impossible to use. A
% character’s movement rate while swimming or walking
% underwater is one quarter their normal rate on land.
% If a character begins to drown underwater, follow the
% rules for asphyxiation (p. 194). Note that drowning
% characters do not immediately recover if rescued from
% the water; they will continue to asphyxiate until medical
% treatment is applied to clear the water from their lungs.

% VACUUM

% Biomorphs without vacuum sealing (p. 305) can
% spend one minute in the vacuum of space with no ill
% effects, provided they curl up into a ball, empty their
% lungs of air, and keep their eyes closed (something kids
% in space habitats learn at a very young age). Contrary
% to popular depictions in pre-Fall media, a character
% exposed to hard vacuum does not explosively de-
% compress, nor do their internal fluids boil (other than
% relatively exposed liquids such as saliva on the tongue).
% Rather, the primary danger for characters on EVA sans
% vacsuit is asphyxiation due to lack of oxygen and as-
% sociated complications such as edema in the lungs.

% After one minute in space, the character begins to
% suffer from asphyxiation (p. 194). Damage is doubled
% if the character tries to hold air in their lungs or is not
% curled up in a vacuum survival position as described
% above. Additionally, characters trapped in space with-
% out adequate thermal protection suffer 10 points of
% damage per minute from the extreme cold.

% %%% txt/204.txt


%                                     IMPROVISE


%                  ARMOR
% WEAPONS           PENETRATION (AP)           DAMAGE VALU
% Baseball                  —                  (1d10 ÷ 2) + (SOM
% Bottle                    —             1 + (SOM ÷ 10), break
% Bottle (Broken)           —                     1d10 – 1 (min
% Chain                     —                    1d10 + (SOM ÷
% Helmet                    —                    1d10 + (SOM ÷
% Plasma Torch              –6                          2d10
% Wrench                    —                    1d10 + (SOM ÷


%  IMPROVISED WEAPONS
%  Sometimes characters are caught off-guard and they
%  must use whatever they happen to have at hand as
%  a weapon—or they think they look cool wailing on
%  someone with a meter of chain. The Improvised Weap-
%  ons table offers statistics for a few likely ad-hoc items.
%  Gamemasters can use these as guidelines for handling
%  items that aren’t listed.

%  KNOCKDOWN/KNOCKBACK
%  If an attacker’s intent is to simply knock an opponent
%  down or back in melee, rather than injure them, roll
%  the attack and defense as normal. If the attacker suc-
%  ceeds, the defender is knocked backward by 1 meter
%  per 10 full points of MoS. To knock an opponent
%  down, the attacker must score an Excellent Success
%  (MoS 30+). A knockback/knockdown attack must be
%  declared before dice are rolled.

%  Unless the attacker rolls a critical success, no
%  damage is inflicted with this attack, the defender is
%  simply knocked down. If the attacker rolls a critical
%  hit, however, apply damage as normal in addition to
%  the knockback/knockdown.

%  Note that characters wounded by an attack may
%  also be knocked down (see Wound Effects, p. 207).

%  MELEE AND THROWN DAMAGE BONUS
%  Every successful melee and thrown weapon attack,
%  whether unarmed or with a weapon, receives a damage
%  bonus equal to the attacker’s SOM ÷ 10, round down.
%  See Damage Bonus, p. 123.

%  MULTIPLE TARGETS
%  When doling out the damage, there’s no reason not to
%  share the love.

%  MELEE COMBAT
%  A character taking a Complex Action to engage in a
%  melee attack may choose to attack two or more op-
%  ponents with the same action. Each opponent must be
%  within one meter of another attacked opponent. These
%  attacks must be declared before the dice are rolled for
%  the first attack. Each attack suffers a cumulative –20
%  modifier for each extra target. So if a character declares
%  they are going to attack three characters with the same
%  action, they suffer a cumulative –60 on each attack.
% WEAPONS
% V)                 AVERAGE DV                    SKILL
% 10)               2 + (SOM ÷ 10)           Throwing Weapons
% er 1 use          1 + (SOM ÷ 10)       Clubs or Throwing Weapons


%                     4                       Blades


%              5 + (SOM ÷ 10)              Exotic Melee


%              5 + (SOM ÷ 10)       Clubs or Throwing Weapons


%                    11                   Exotic Ranged


%              5 + (SOM ÷ 10)                 Clubs




%  RANGED COMBAT


% A character firing two semi-auto shots with a Complex


% Action may target a different opponent with each shot.


% In this case, the attacker suffers a –20 modifier against


% the second target.


%    A character firing a burst-fire weapon may target up


% to two targets with each burst, as long as those targets


% are within one meter of one another. This is handled


% as a single attack; see Burst Fire, p. 198.


%    A character firing a burst-fire weapon twice with


% one Complex Action may target a different person or


% pair with each burst. In this case, the second burst


% suffers a –20 modifier. This modifier does not apply


% if the same person/pair targeted with the first burst is


% targeted again.


%    Full-auto attacks may also be directed at more than


% one target, as long as each target is within one meter


% of the previous target. This is handled as a single


% attack; see Full Auto, p. 198.



%  OBJECTS AND STRUCTURES


% As any poor wall in the vicinity of an enraged drunk


% can tell you, objects and structures are not immune


% to violence and attrition. To reflect this, inanimate


% objects and structures are given Durability, Wound


% Threshold, and Armor scores, just like characters. Du-


% rability measures how much damage the structure can


% take before it is destroyed. Armor reduces the damage


% inflicted by attacks, as normal. For simplicity, a single


% Armor rating is given that counts as both Energy and


% Kinetic armor; at the gamemaster’s discretion, these


% may be modified as appropriate.


%   Wounds suffered by objects and structures do not have


% the same effect as wounds inflicted on characters. Each


% wound is simply treated as a hole, partial demolition, or


% impaired function, as the gamemaster sees fit. Alternately,


% a wounded device may function less effectively, and so


% may inflict a negative modifier on skill tests made while


% using that object (a cumulative –10 per wound).


%    In the case of large structures, it is recommended


% that individual parts of the structure be treated as


% separate entities for the purpose of inflicting damage.



%  RANGED ATTACKS


% Ranged combat attacks inflict only one-third their


% damage (round down) on large structures such

% %%% txt/205.txt


%                                    SAMPLE OB
% OBJECT/STRUCTURE                      ARMOR DURABILITY T
% Advanced Composites


%                                   50       1,000

% (ship/habitat hull)
% Aerogel (walls, windows, etc.)         —          50
% Airlock Door                           15        100
% Alloys, Concrete, Hardened Polymers


%                                   30        100
% (reinforced doors/walls)
% Armored Glass                          10         50
% Counter                                 7         60
% Desk                                    5         50


%  as doors, walls, etc. This reflects the fact that most
%  ranged attacks simply penetrate the structure, leaving
%  minor damage.

% Agonizers and stunners have no effect on objects
%  and structures.

%  SHOOTING THROUGH
%  If a character attempts to shoot through an object or
%  structure at a target on the other side, the attack is
%  likely to suffer a blind fire modifier of at least –30
%  unless the attack has some way of viewing the target.
%  On top of this, the target receives an armor bonus
%  equal to the object/structure’s Armor rating x 2.

%  RANGE
% Every type of ranged weapon has a limited range,
% beyond which it is ineffective. The effective range of
% the weapon is further broken down into four catego-
% ries: Short, Medium, Long, and Extreme. A modifier
% is applied for each category, as noted on the Combat
% Modifiers table, p. 193.

% For examples of specific weapon ranges, see the
% Weapon Ranges table.

%  RANGE, GRAVITY, AND VACUUM
%  The ranges listed on the Weapon Ranges table are for
%  Earth-like gravity conditions (1 g). While the effective
%  ranges of kinetic, seeker, spray, and thrown weapons
%  can potentially increase in lower gravity environments
%  due to lack of gravitational forces or aerodynamic
%  drag, accuracy is still the defining factor for determin-
%  ing whether you hit or miss a target. In lower gravi-
%  ties, use the same effective ranges listed, but extend
%  the maximum range by dividing it by the gravity (for
%  example, a max range of 100 meters would be 200
%  meters in 0.5 g). In microgravity and zero g, the maxi-
%  mum range is effectively line of sight. Likewise, under
%  high-gravity conditions (over 1 g), divide each range
%  category maximum by the gravity (e.g., a short range
%  of 10 meters would be 5 meters in 2 g).

%  Beam weapons are not affected by gravity, but they
%  do fare much better in non-atmospheric conditions.
%  Maximum beam weapon range in vacuum is effec-
%  tively line of sight.
%  CTS AND STRUCTURES
% OUND
%  SHOLD      OBJECT/STRUCTURE                           ARMOR DURAB


%        Ecto link                                       —         6
%  00


%        Metallic Foam (walls, doors, etc.)              20       70
% 10          Metallic Glass                                  30       150
% 25          Polymer or Wood


%                                                        10       40


%           (walls, doors, furniture, etc.)
% 20


%        Quantum Farcaster Link                           3       20
% 20          Transparent Alumina (walls, furniture)           5       60
% 12          Tree                                             2       40
% 10          Window                                          —         5






%                         WEAPON RANGES


%                              SHORT           MEDIUM       LONG
% WEAPON (TYPE)                     RANGE          RANGE (–10) RANGE (
%  rearms
%  Light Pistol                      0–10              11–25         26–4
%  Medium Pistol                     0–10              11–30         31–5
%  Heavy Pistol                      0–10              11–35         36–6
%  SMG                               0–30              31–80        81–12
%  Assault Rifle                      0–150          151–250         251–5
%  Sniper Rifle                       0–180          181–400        401–1,1
%  Machine Gun                       0–100          101–400        401–1,0
% ailguns
%  as Firearms but increase the effective range in each category by +50
% eam Weapons
%  Cybernetic Hand Laser             0–30              31–80        81–12
%  Laser Pulser                      0–30              31–100       101–1
%  Microwave Agonizer                 0–5              6–15          16–3
%  Particle Beam Bolter              0–30              31–100       101–1
%  Plasma Rifle                       0–20              21–50        51–10
%  Stunner                           0–10              11–25         26–4
% eekers
%  Seeker Micromissile               5–70              71–180       181–6
%  Seeker Minimissile                5–150          151–300        301–1,0
%  Seeker Standard Missile           5–300          301–1,000      1001–3,
% pray Weapons
%  Buzzer                             0–5              6–15          16–3
%  Freezer                            0–5              6–15          16–3
%  Shard Pistol                      0–10              11–30         31–5
%  Shredder                          0–10              11–40         41–7
%  Sprayer                            0–5              6–15          16–3
%  Torch                              0–5              6–15          16–3
%  Vortex Ring Gun                    0–5              6–15          16–3
% hrown Weapons
%  Blades                        To SOM ÷ 5        To SOM ÷ 2       To SO
%  Minigrenades                  To SOM ÷ 2            To SOM      To SOM
%  Standard Grenades             To SOM ÷ 5        To SOM ÷ 2       To SO

% %%% txt/206.txt
% REACH
% Some weapons extend a character’s reach, giving
% them a significant advantage over an opponent in
% melee combat. This applies to any weapon over half
% a meter long: axes, clubs, swords, shock batons, etc.
% Whenever one character has a reach advantage over
% another, they receive a +10 modifier for both attacking
% and defending.

% SCATTER
% When you are using a blast weapon, you may still
% catch your target in the blast radius even if you fail to
% hit them directly. Weapons such as grenades must go
% somewhere when they miss, and exactly where they
% land may be important to the outcome of a battle. To
% determine where a missed blast attack falls, the scatter
% rules are called into play.

% To determine scatter, roll a d10 and note where the
% die “points” (using yourself as the reference point).
% This is the direction from the target that the missed
% blast lands. The die roll also determines how far away
% the blast lands, in meters. If the MoF on the attack is
% over 30, this distance is doubled. If the MoF exceeds
% 60, the distance is tripled. This point determines the
% epicenter of the blast; resolve the effects of damage
% against anyone caught within its sphere of effect as
% normal (see Blast Effect, p. 193).





%                    1 or 2


%              10               3



%          9                         4



%              8                5


%                    6 or 7

% SHOCK ATTACKS
% Shock weapons use high-voltage electrical jolts to
% physically stun and incapacitate targets. Shock weap-
% ons are particularly effective against biomorphs and
% pods, even when heavily armored. Synthmorphs, bots,
% and vehicles are immune to shock weapon effects.

% A biomorph struck with a shock weapon must
% make a DUR + Energy Armor Test (using their cur-
% rent DUR score, reduced by damage they have taken).
% If they fail, they immediately lose neuromuscular
%  control, fall down, and are incapacitated for 1 Action
% Turn per 10 full points of MoF (minimum of 3 Action
% Turns). During this time they are stunned and inca-
% pable of taking any action, possibly convulsing, suf-
% fering vertigo, nausea, etc. After this period, they may
% act but they remain stunned and shaken, suffering a
% –30 modifier to all actions. This modifier reduces by
% 10 per minute (so –20 after 1 minute, –10 after 2 min-
% utes, and no modifier after 3 minutes). Many shock
% weapons also inflict DV, which is handled as normal.

% A biomorph that succeeds the DUR Test is still
% shocked but not incapacitated. They suffer half the
% listed DV and suffer a –30 modifier until the end of
% the next Action Turn. This modifier reduces by 10 per
% Action Turn. Modifiers from additional shocks are
% not cumulative, but will boost the modifier back to
% its maximum value.

% SUBDUAL
% To grapple an opponent in melee combat, you must
% declare your intent to subdue before making the die
% roll. Any appropriate melee skill may be used for the
% attack; if wielding a weapon, it may be used as part of
% the grappling technique. If you succeed in your attack
% with an Excellent Success (MoS of 30+), you have suc-
% cessfully subdued your opponent (for the moment, at
% least). Grappling attacks do not cause damage unless
% you roll a critical success (though even in this case you
% can choose not to).

% A subdued opponent is temporarily restrained or
% immobilized. They may communicate, use mental
% skills, and take mesh actions, but they may not take
% any physical actions other than trying to break free.
% (At the gamemaster’s discretion, they may make small,
% restrained physical actions, such as reaching for a
% knife in their pocket or grabbing an item dropped a
% few centimeters away on the floor, but these actions
% should suffer at least a –30 modifier and may be no-
% ticed by their grappler).

% To break free, a grappled character must take a
% Complex Action and succeed in either an Opposed
% Unarmed Combat Test or an Opposed SOM x 3 Test,
% though the subdued character suffers a –30 modifier
% on this test.

% SUPPRESSIVE FIRE
% A character firing a weapon in full-auto mode (p. 198)
% may choose to lay down suppressive fire over an area
% rather than targeting anyone specifically, with the
% intent of making everyone in the suppressed area keep
% their heads down. This takes a Complex Action, uses
% up 20 shots, and lasts until the character’s next Action
% Phase. The suppressed area extends out in a cone, with
% the widest diameter of the cone being up to 20 meters
% across. Any character who is not behind cover or who
% does not immediately move behind cover on their
% action is at risk of getting hit by the suppressive fire. If
% they move out of cover inside the suppressed area, the
% character laying down suppressive fire gets one free
% attack against them, which they may defend against
% as normal. Apply no modifiers to these tests except
% for range, wounds, and full defense. If hit, the struck
% character must resist damage as if from a single shot.

% SURPRISE
% Characters who wish to ambush another must seek to
% gain the advantage of surprise. This typically means
% sneaking up on, lying in wait, or sniping from a hard-
% to-perceive position in the distance. Any time an am-
% busher (or group of ambushers) attempts to surprise a

% %%% txt/207.txt
% target (or group of targets), make a secret Perception
% Test for the ambushee(s). Unless they are alert for sur-
% prises, this test should suffer the typical –20 modifier
% for being distracted. This is an Opposed Test against
% the ambusher(s) Infiltration skill. Depending on the
% attacker’s position, other modifiers may also apply
% (distance, visibility, cover, etc.).

% If the Perception Test fails, the character is surprised
% by the attack and cannot react to or defend against it. In
% this case, simply give the attacker(s) a free Action Phase
% to attack the surprised character(s). Once the attackers
% have taken their actions, roll Initiative as normal.

% If the Perception Test succeeds, the character is
% alerted to something a split-second before they are
% ambushed, giving them a chance to react. In this case,
% roll Initiative as normal, but the ambushed character(s)
% suffers a –30 modifier to the Initiative Test. The am-
% bushed character may still defend as normal.

% In a group situation, things can get more compli-
% cated when some characters are surprised and others
% aren’t. In this case, roll Initiative as normal, with all
% non-ambushers suffering the –30 modifier. Any charac-
% ters who are surprised are simply unable to take action
% on the first Action Phase, as they are caught off-guard
% and must take a moment to assess what’s going on
% and get caught up with the action. As above, surprised
% characters my not defend on this first Action Phase.

% TACTICAL NETWORKS
% Tactical networks are specialized software programs
% used by teams that benefit from the sharing of tacti-
% cal data. They are commonly used by sports teams,
% security outfits, military units, AR gamers, gatecrash-
% ers, surveyors, miners, traffic control, scavengers, and
% anyone else who needs a tactical overview of a situa-
% tion. Firewall teams regularly take advantage of them.

% In game terms, tacnets provide specialized software
% skills and tools to a muse or AI, as best fits their tac-
% tical needs. These tools link together and share and
% analyze data between all of the participants in the
% network, creating a customizable entoptics display for
% each user that summarizes relevant data, highlights
% interactions and priorities, and alerts the user to mat-
% ters that require their attention.
% COMBAT TACNETS
% The following list is a sample of a typical combat tacnet’s
% features. Gamemasters are encouraged to modify and
% expand these options as appropriate to their game:


% • Maps: Tacnets assemble all available maps and

%  can present them to the user with a bird’s eye

%  view or as a three-dimensional interactive, with

%  distances between relevant features readily acces-

%  sible. The AI or muse can also plot maps based

%  on sensory input, breadcrumb positioning sys-

%  tems (p. 332), and other data. Plotted paths and

%  other data from these maps can be displayed as

%  entoptic images or other AR sensory input (e.g.,

%  a user who should be turning left might see a

%  transparent red arrow or feel a tingling sensation

%  on their left side).

% • Positioning: The exact positioning of the user and

%  all other participants are updated and mapped

%  according to mesh positioning and GPS. Likewise,

%  the positioning of known people, bots, vehicles,

%  and other features can also be plotted according

%  to sensory input.

% • Sensory Input: Any sensory input available to a

%  participating character or device in the network

%  can be fed into the system and shared. This

%  includes data from cybernetic senses, portable

%  sensors, smartlink guncams, XP output, etc. This

%  allows one user to immediately call up and access

%  the sensor feed of another user.

% • Communications Management: The tacnet

%  maintains an encrypted link between all users

%  and stays wary both of participants who drop

%  out or of attempts to hack or interfere with the

%  communications link.

% • Smartlink/Weapon Data: The tacnet monitors the

%  status of weapons, accessories, and other gear via

%  the smartlink interface or wireless link, bringing

%  damage, shortages, and other issues to the user’s

%  attention.

% • Indirect Fire: Members of a tacnet can provide

%  targeting data to each other for purposes of indi-

%  rect fire (p. 195).

% • Analysis: The muses and AIs participating in

%  the tacnet are bolstered with skill software and

% %%% txt/208.txt


% databases that enable them to interpret incom-


% ing data and sensory feeds. Perhaps the most


% useful aspect of tacnets, this means that the


% muse/AI may notice facts or details individual


% users are likely to have overlooked. For example,


% the tacnet can count shots fired by opponents,


% note when they are likely running low, and


% even analyze sensory input to determine the


% type of weaponry and ammunition being used.


% Opponents and their gear can also be scanned


% and analyzed to note potential weaknesses, in-


% juries, and capabilities. If sensor contact with


% an opponent is lost, the last known location is


% memorized and potential movement vectors and


% distances are displayed. Opponent positioning


% can also identify lines of sight and fields of fire,


% alerting the user to areas of potential cover or


% danger. The tacnet can also suggest tactical ma-


% neuvers that will aid the user, such as flanking an


% opponent or acquiring better elevation.


% Many of these features are immediately accessible
% to the user via their AR display; other data can be ac-
% cessed with a Quick Action. Likewise, the gamemaster
% decides when the muse/AI provides important alerts to
% the user. At the gamemaster’s discretion, some of these
% features may apply modifiers to the character’s tests.

% TOUCH-ONLY ATTACK
% Some types of attacks simply require you to touch your
% target, rather than injure them, and are correspond-
% ingly easier. This might apply when trying to slap them
% with a dermal drug patch, spreading a contact poison
% on their skin, or making skin-to-skin contact for the
% use of a psi sleight. In situations like this, apply a +20
% modifier to your melee attacks.

% TWO-HANDED WEAPONS
% Any weapon noted as two-handed requires two hands
% (or other prehensile limbs) to wield effectively. This
% applies to some archaic melee weapons (large swords,
% spears, etc.) in addition to certain larger firearms and
% heavy weapons. Any character that attempts to use
% such a weapon single-handed suffers a –20 modifier.

% WIELDING TWO OR MORE WEAPONS
% It is possible for a character to wield two weapons
% in combat, or even more if they are an octomorph or
% multi-limbed synthmorph. In this case, each weapon
% that is held in an off-hand suffers a –20 off-hand
% weapon modifier. This modifier may be offset with the
% Ambidextrous trait (p. 145).

% EXTRA MELEE WEAPONS
% The use of two or more melee weapons is treated as
% a single attack, rather than multiple. Each additional
% weapon applies +1d10 damage to the attack (up to a
% maximum +3d10). If the character attacks multiple
% targets with the same Complex Action (see Mul-
% tiple Targets, p. 202), this bonus does not apply. The
% attacker must, of course, be capable of actually wield-
% ing the additional weapons. A splicer with only two
% hands cannot wield a knife and a two-handed sword,
% for example. Likewise, the gamemaster may ignore
% this damage bonus for extra weapons that are too dis-
% similar to use together effectively (like a whip and a
% pool cue). Note that extra limbs do not count as extra
% weapons in unarmed combat, nor do weapons that
% come as a pair (such as shock gloves).

% A character using more than one melee weapon
% receives a bonus for defending against melee attacks
% equal to +10 per extra weapon  (maximum +30).

% EXTRA RANGED WEAPONS
% Similarly, an attacker can wield a pistol in each hand
% for ranged combat, or larger weapons if they have
% more limbs (an eight-limbed octomorph, for example,
% could conceivably hold four assault rifles). These
% weapons may all be fired at once towards the same
% target. In this case, each weapon is handled as a sepa-
% rate attack, with each off-hand weapon suffering a
% cumulative off-hand weapon modifier (no modifier for
% the first attack, –20 for the second, –40 for the third,
% and –60 for the fourth), offset by the Ambidextrous
% trait (p. 145) as usual.



% PHYSICAL HEALTH
% In a setting as dangerous as Eclipse Phase, characters
% are inevitably going to get hurt. Whether your morph
% is biological or synthetic, you can be injured by weap-
% ons, brawling, falling, accidents, extreme environ-
% ments, psi attacks, and so on. This section discusses
% how to track such injuries and determine what effect
% they have on your character. Two methods are used
% to gauge a character’s physical health: damage points
% and wounds.

% DAMAGE POINTS
% Any physical harm that befalls your character is mea-
% sured in damage points. These points are cumulative,
% and are recorded on your character sheet. Damage
% points are characterized as fatigue, stun, bruises,
% bumps, sprains, minor cuts, and similar hurts that,
% while painful, do not significantly impair or threaten
% your character’s life unless they accumulate to a signif-
% icant amount. Any source of harm that inflicts a large
% amount of damage points at once, however, is likely to
% have a more severe effect (see Wounds, p. 207).

% Damage points may be reduced by rest, medical
% care, and/or repair (see Healing and Repair, p. 208).

% DAMAGE TYPES
% Physical damage comes in three forms: Energy, Kinetic,
% and Psi.

% ENERGY DAMAGE
% Energy damage includes lasers, plasma guns, fire, electro-
% cution, explosions, and others sources of damaging energy.

% %%% txt/209.txt
% KINETIC DAMAGE
% Kinetic damage is caused by projectiles and other
% objects moving at great speeds that disperse their
% energy into the target upon impact. Kinetic attacks
% include slug-throwers, flechette weapons, knives, and
% punches.

% PSI DAMAGE
% Psi damage is caused by offensive psi sleights like
% Psychic Stab (p. 228).

% DURABILITY AND HEALTH
% Your character’s physical health is measured by their
% Durability stat. For characters sleeved in biomorphs,
% this figure represents the point at which accumu-
% lated damage points overwhelm your character and
% they fall unconscious. Once you have accumulated
% damage points equal to or exceeding your Durability
% stat, you immediately collapse from exhaustion and
% physical abuse. You remain unconscious and may
% not be revived until your damage points are reduced
% below your Durability, either from medical care or
% natural healing.

% If you are morphed in a synthetic shell, Durabil-
% ity represents your structural integrity. You become
% physically disabled when accumulated damage points
% reach your Durability. Though your computer systems
% are likely still functioning and you can still mesh, your
% morph is broken and immobile until repaired.

% DEATH
% An extreme accumulation of damage points can
% threaten your character’s life. If the damage reaches
% your Durability x 1.5 (for biomorphs) or Durability x
% 2 (for synthetic morphs), your body dies. This known
% as your Death Rating. Synthetic morphs that reach
% this state are destroyed beyond repair.

% DAMAGE VALUE
% Weapons (and other sources of injury) in Eclipse
% Phase have a listed Damage Value (DV)—the base
% amount of damage points the weapon inflicts. This is
% often presented as a variable amount, in the form of a
% die roll; for example: 3d10. In this case, you roll three
% ten-sided dice and add up the results (counting 0 as
% 10). Sometimes the DV will be presented as a dice roll
% plus modifier; for example: 2d10 + 5. In this case you
% roll two ten-sided dice, add them together, and then
% add 5 to get the result.

% For simplicity, a static amount is also noted in
% parentheses after the variable amount. If you prefer
% to skip the dice rolling, you can just apply the static
% amount (usually close to the mean average) instead.
% For example, if the damage were noted 2d10 + 5 (15),
% you could simply apply 15 damage points instead of
% rolling dice.

% When damage is inflicted on a character, determine the
% DV (roll the dice) and subtract the modified armor value,
% as noted under Step 7: Determine Damage (p. 192).
% WOUNDS
% Wounds represent more grievous injuries: bad cuts and
% hemorrhaging, fractures and breaks, mangled limbs,
% and other serious damage that impairs your ability to
% function and may lead to death or long-term damage.

% Any time your character sustains damage, compare
% the amount inflicted (after it has been reduced by
% armor) to your Wound Threshold. If the modified DV
% equals or exceeds your Wound Threshold, you have
% suffered a wound. If the inflicted damage is double
% your Wound Threshold, you suffer 2 wounds; if
% triple your Wound Threshold, you suffer 3 wounds;
% and so on.

% Wounds are cumulative, and must be marked on
% your character sheet.

% Note that these rules handle damage and wounds
% as an abstract concept. For drama and realism,
% gamemasters may wish to describe wounds in more
% detailed and grisly terms: a broken ankle, a severed
% tendon, internal bleeding, a lost ear, and so on. The
% nature of such descriptive injuries may help the game-
% master assign other effects. For example, a character
% with a crushed hand may not be able to pick up a gun,
% someone with excessive blood loss may leave a trail
% for their enemies to follow, or someone with a cut eye
% may suffer an additional visual perception modifier.
% Likewise, such details may impact how a character is
% treated or heals.

% WOUND EFFECTS
% Each wound applies a cumulative –10 modifier to all
% of the character’s actions. A character with 3 wounds,
% for example, suffers –30 to all actions.

% Some traits, morphs, implants, drugs, and psi allow
% a character to ignore wound modifiers. These effects
% are cumulative, though the maximum amount of
% wound modifiers that may be negated is –30.

% Knockdown: Any time a character takes a wound,
% they must make an immediate SOM x 3 Test. Wound
% modifiers apply. If they fail, they are knocked down
% and must expend a Quick Action to get back up. Bots
% and vehicles must make a Pilot Test to avoid crashing.

% Unconsciousness: Any time a character receives 2
% or more wounds at once (from the same attack), they
% must also make an immediate SOM x 3 Test; wound
% modifiers again apply. If they fail, they have been
% knocked unconscious. Bots and vehicles that take 2 or
% more wounds at once automatically crash (see Crash-
% ing, p. 196).

% Bleeding: Any biomorph character who has suffered
% a wound and who takes damage that exceeds their
% Durability is in danger of bleeding to death. They
% incur 1 additional damage point per Action Turn (20
% per minute) until they receive medical care or die.

% DEATH
% For many people in Eclipse Phase, death is not the end
% of the line. If the character’s cortical stack can be re-
% trieved, they can be resurrected and downloaded into
% a new morph (see Resleeving, p. 271). This typically

% %%% txt/210.txt
%  requires either backup insurance (p. 269) or the good
%  graces of whomever ends up with their body/stack.

%  If the cortical stack is not retrievable, the character
%  still be re-instantiated from an archived backup (p.
%  268). Again, this either requires backup insurance or
%  someone who is willing to pay to have them revived.

%  If the character’s cortical stack is not retrieved
%  and they have no backup, then they are completely
%  and utterly dead. Gone. Kaput. (Unless they happen
%  to have an alpha fork of themselves floating around
%  somewhere; see Forking and Merging, p. 273.)



%  HEALING AND REPAIR
%  Use the follow rules for healing and repairing dam-
%  aged and wounded characters.

%  BIOMORPH HEALING
%  Thanks to advanced medical technologies, there are
%  many ways for characters in biological morphs (in-
%  cluding pods) to heal injuries. Medichine nanoware
%  (p. 308) helps characters to heal quickly, as do nano-
%  bandages (p. 333). Healing vats (p. 326) will heal even
%  the most grievous wounds in a matter of days, and
%  can even restore characters who recently died or have
%  been reduced to just a head.

% Characters without access to these medical tools are
%  not without hope, of course. The medical skills of a
%  trained professional can abate the impact of wounds,
%  and over time bodies will of course heal themselves.

%  MEDICAL CARE
%  Characters with an appropriate Medicine skill (such
%  as Medicine: Paramedic or Medicine: Trauma Sur-
%  gery) can perform first aid on damaged or wounded
%  characters. A successful Medicine Test, modified as
%  the gamemaster deems fit according to situational
%  conditions, will heal 1d10 points of damage and will
%  remove 1 wound. This test must be made within 24
%  hours of the injury, and any particular injury may only
%  be treated once. If the character is later injured again,
%  however, this new damage may also be treated. Medi-
%  cal care of this sort is not effective against injuries that
%  have been treated with medichines, nanobandages, or
%  healing vats.





%                                                              HEA
% CHARACTER SITUATION
% Character without basic biomods
% Character with basic biomods
% Character using nanobandage
% Character with medichines
% Poor conditions (bad food, not enough rest/heavy activity,
% poor shelter and/or sanitation)
% Harsh conditions (insufficient food, no rest/strenuous activity,
% little or no shelter and/or sanitation)

% NATURAL HEALING

% Characters trapped far from medical technology—in

% a remote station, the wilds of Mars, or the like—may

% be forced to heal naturally if injured. Natural healing

% is a slow process that’s heavily influenced by a number

% of factors. In order for a character to heal wounds,

% all normal damage must be healed first. Consult the

% Healing table.


% SURGERY
%  In Eclipse Phase, most grievous injuries can be han-
%  dled by time in a healing vat (p. 326) or simply rest
%  and recovery. In circumstances where a healing vat is
%  not available, the gamemaster may decide that a par-
%  ticular wound requires actual surgery from an intelli-
%  gent being (whether a character or AI-driven medbot).
%  Usually in this case the character will be incapable of
%  further healing until the surgery occurs. The surgery
%  is handled as a Medical Test using a field appropriate
%  to the situation and with a timeframe of 1–8 hours. If
%  successful, the character is healed of 1d10 damage and
%  1 wound and recovers from that point on as normal.


% SYNTHMORPH AND OBJECT REPAIR

% Unlike biomorphs, synthetic morphs and objects do not

% heal damage on their own and must be repaired. Some

% synthmorphs and devices have advanced nanotech self-

% repair systems, similar to medichines for biomorphs (see

% Fixers, p. 329). Repair spray (p. 333) may also be used

% to conduct fixes and is an extremely useful option for

% non-technical people. Barring these options, technicians

% may also work repairs the old-fashioned way, using

% their skills and tools (see Physical Repairs, below). As a

% last resort, synthmorphs and objects may be repaired in

% a nanofabrication machine with the appropriate blue-

% prints (using the same rules as healing vats, p. 326).


% PHYSICAL REPAIRS

% Manually fixing a synthmorph or object requires a

% Hardware Test using a field appropriate to the item

% (Hardware: Robotics for synthmorphs and bots,

% Hardware: Aerospace for aircraft, etc.), with a –10

% modifier per wound. Repair is a Task Action with

% a timeframe of 2 hours per 10 points of damage

% being restored, plus 8 hours per wound. Appropriate



% ING
% AMAGE HEALING RATE              WOUND HEALING RATE

% 1d10 (5) per day                     1 per week
% 1d10 (5) per 12 hours                  1 per 3 days
%  1d10 (5) per 2 hours                   1 per day
%  1d10 (5) per 1 hour                  1 per 12 hours


% double timeframe                  double timeframe


% triple timeframe                 no wound healing

% %%% txt/211.txt
% modifiers should be applied, based on conditions and
% available tools. For example, utilitools (p. 326) apply
% a +20 modifier to repair tests, while repair spray ap-
% plies a +30 modifier.

% REPAIRING ARMOR
% Armor may be repaired in the same manner as Dura-
% bility, however, wounds do not impact the test with
% modifiers or extra time.



% MENTAL HEALTH

% In a time when people can discard bodies and
% replace them with new ones, trauma inflicted on
% your mind and ego—your sense of self—is often
% more frightening than grievous physical harm. There
% are many ways in which your sanity and mental
% wholeness can be threatened: experiencing physical
% death, extended isolation, loss of loved ones, alien
% situations, discontinuity of self from lost memories
% or switching morphs, psi attack, and so on. Two
% methods are used to gauge your mental health: stress
% points and trauma.

% STRESS POINTS
% Stress points represent fractures in your ego’s in-
% tegrity, cracks in the mental image of yourself. This
% mental damage is experienced as cerebral shocks,
% disorientation, cognitive disconnects, synaptic mis-
% fires, or an undermining of the intellectual faculties.
% On their own, these stress points do not significantly
% impair your character’s functioning, but if allowed to
% accumulate they can have severe repercussions. Ad-
% ditionally, any source that inflicts a large amount of
% stress points at once is likely to have a more severe
% impact (see Trauma).

% Stress points may be reduced by long-term rest, psy-
% chiatric care, and/or psychosurgery (see p. 214).

% LUCIDITY AND STRESS
% Your Lucidity stat benchmarks your character’s mental
% stability. If you build up an amount of stress points
% equal to or greater than your Lucidity score, your char-
% acter’s ego immediately suffers a mental breakdown.
% You effectively go into shock and remain in a catatonic
% state until your stress points are reduced to a level
% below your Lucidity stat. Accumulated stress points
% will overwhelm egos housed inside synthetic shells or
% infomorphs just as they will biological brains—the
% mental software effectively seizes up, incapable of
% functioning until it is debugged.

% INSANITY RATING
% Extreme amounts of built-up stress points can perma-
% nently damage your character’s sanity. If accumulated
% stress points reach your Lucidity x 2, your character’s
% ego undergoes a permanent meltdown. Your mind is
% lost, and no amount of psych help or rest will ever
% bring it back.
% STRESS VALUE
% Any source capable of inflicting cognitive stress is given
% a Stress Value (SV). This indicates the amount of stress
% points the attack or experience inflicts upon a character.
% Like DV, SV is often presented as a variable amount, such
% as 2d10, or sometimes with a modifier, such as 2d10 +
% 10. Simply roll the dice and total the amounts to deter-
% mine the stress points inflicted in that instance. To make
% things easier, a static SV is also given in parentheses after
% the variable amount; use that set amount when you wish
% to keep the game moving and don’t want to roll dice.

% TRAUMA
% Mental trauma is more severe than stress points. Trau-
% mas represent severe mental shocks, a crumbling of
% personality/self, delirium, paradigm shifts, and other
% serious cognitive malfunctions. Traumas impair your
% character’s functioning and may result in temporary
% derangements or permanent disorders.

% If your character receives a number of stress points
% at once that equals or exceeds their Trauma Thresh-
% old, they have suffered a trauma. If the inflicted stress
% points are double or triple the Trauma Threshold, they
% suffer 2 or 3 traumas, respectively, and so on. Trau-
% mas are cumulative and must be recorded on your
% character sheet.

% TRAUMA EFFECTS
% Each trauma applies a cumulative –10 modifier to all
% of the character’s actions. A character with 2 traumas,
% for example, suffers –20 to all actions. These modifiers
% are also cumulative with wound modifiers.

% Disorientation: Any time a character suffers a
% trauma, they must make an immediate WIL x 3 Test.
% Trauma modifiers apply. If they fail, they are tempo-
% rarily stunned and disoriented, and must expend a
% Complex Action to regain their wits.

% Derangements and Disorders: Any time a character
% is hit with a trauma, they suffer a temporary derange-
% ment (see Derangements). The first trauma inflicts a
% minor derangement. If a second trauma is applied,
% the first derangement is either upgraded from minor
% to a moderate derangement, or else a second minor
% derangement is applied (gamemaster’s discretion).
% Likewise, a third trauma may upgrade that derange-
% ment from moderate to major or else inflict a new
% minor. It is generally recommended that derangements
% be upgraded in potency, especially when result from
% the same set of ongoing circumstances. In the case of
% traumas that result from distinctly separate situations
% and sources, separate derangements may be applied.

% Disorder: When four or more traumas have been
% inflicted on a character, a major derangement is up-
% graded to a disorder. Disorders represent long-lasting
% psychological afflictions that typically require weeks
% or even months of psychotherapy and/or psychosur-
% gery to remedy (see Disorders, p. 211).

% %%% txt/212.txt
% DERANGEMENTS
% Derangements are temporary mental conditions that
% result from traumas. Derangements are measured as
% Minor, Moderate, or Major. The gamemaster and
% player should cooperate in choosing which derange-
% ment to apply, as appropriate to the scenario and
% character personality.

% Derangements last for 1d10 ÷ 2 hours (round
% down), or until the character receives psychiatric help,
% whichever comes first. At the gamemaster’s discretion,
% a derangement may last longer if the character has not
% been distanced from the source of the stress, or if they
% remain embroiled in other stress-inducing situations.

% Derangement effects are meant to be role-played.
% The player should incorporate the derangement into
% their character’s words and actions. If the gamemaster
% doesn’t feel the player is stressing the effects enough,
% they can emphasize them. If the gamemaster feels it is
% appropriate, they may also call for additional modi-
% fiers or tests for certain actions.

% ANXIETY (MINOR)
% You suffer a panic attack, exhibiting the physiological
% conditions of fear and worry: sweatiness, racing heart,
% trembling, shortness of breath, headaches, and so on.

% AVOIDANCE (MINOR)
% You are psychologically incapable with dealing with
% the source of the stress, or some circumstance related
% to it, so you avoid it—even covering your ears, curl-
% ing up in a ball, or shutting off your sensors if you
% have to.

% DIZZINESS (MINOR)
% The stress makes you light-headed and disoriented.

% ECHOLALIA (MINOR)
% You involuntarily repeat words and phrases spoken
% by others.

% FIXATION (MINOR)
% You become fixated on something that you did wrong
% or some circumstance that led to your stress. You
% obsess over it, repeating the behavior, trying to fix it,
% running scenarios through your head and out loud,
% and so on.

% HUNGER (MINOR)
% You are suddenly consumed by an irrational yet
% overwhelming desire to eat something—perhaps even
% something unusual.

% INDECISIVENESS (MINOR)
% You are flustered by the cause of your stress, finding it
% difficult to make choices or select courses of action.

% %%% txt/213.txt
% LOGORRHOEA (MINOR)
% Your response to the trauma is to engage in excessive
% talking and babbling. You don’t shut up.

% NAUSEA (MINOR)
% The stress sickens you, forcing you to fight down
% queasiness.

% CHILLS (MODERATE)
% Your body temperature rises, making you feel cold,
% and shivering sets in. You just can’t get warm.

% CONFUSION (MODERATE)
% The trauma scrambles your concentration, making
% you forget what you’re doing, mix up simple tasks,
% and falter over easy decisions.

% ECHOPRAXIA (MODERATE)
% You involuntarily repeat and mimic the actions of
% others around you.

% MOOD SWINGS (MODERATE)
% You lose control of your emotions. You switch from
% ecstasy to tears and back to rage without warning.

% MUTE (MODERATE)
% The trauma shocks you into speechlessness and a com-
% plete inability to effectively communicate.

% NARCISSISM (MODERATE)
% In the wake of the mental shock, all you can think
% about is yourself. You cease caring about those around
% you.

% PANIC (MODERATE)
% You are overwhelmed by fear or anxiety and immedi-
% ately seek to distance yourself from the cause of the
% stress.

% TREMORS (MODERATE)
% You shake violently, making it difficult to hold things
% or stay still.

% BLACKOUT (MAJOR)
% You operate on auto-pilot in a temporary fugue state.
% Later, you will be incapable of recalling what hap-
% pened during this period. (Synthetic shells and info-
% morphs may call up memory records from storage.)

% FRENZY (MAJOR)
% You have a major freak out over the source of the
% stress and attack it.

% HALLUCINATIONS (MAJOR)
% You see, hear, or otherwise sense things that aren’t
% really there.
% HYSTERIA (MAJOR)
% You lose control, panicking over the source of the
% stress. This typically results in an emotional outburst
% of crying, laughing, or irrational fear.

% IRRATIONALITY (MAJOR)
% You are so jarred by the stress that your capacity for
% logical judgment breaks down. You are angered by
% imaginary offenses, hold unreasonable expectations, or
% otherwise accept things with unconvincing evidence.

% PARALYSIS (MAJOR)
% You are so shocked by the trauma that you are effec-
% tively frozen, incapable of making decisions or taking
% action.

% PSYCHOSOMATIC CRIPPLING (MAJOR)
% The trauma overwhelms you, impairing some part of
% your physical functioning. You suffer from an inexpli-
% cable blindness, deafness, or phantom pain, or are sud-
% denly incapable of using a limb or other extremity.

% DISORDERS
% Disorders reflect more permanent madness. In this
% case, “permanent” does not necessarily mean forever,
% but the condition is ongoing until the character has re-
% ceived lengthy and effective psychiatric help. Disorders
% are inflicted whenever a character has accumulated 4
% traumas. The gamemaster and player should choose a
% disorder that fits the situation and character.

% Disorders are not always “active”—they may
% remain dormant until triggered by certain conditions.
% While it is certainly possible to act under a disorder,
% it represents a severe impairment to a person’s ability
% to maintain normal relationships and do a job suc-
% cessfully. Disorders should not be glamorized as cute
% role-playing quirks. They represent the best attempts
% of a damaged psyche to deal with a world that has
% failed it in some way. Additionally, people in many
% habitats, particularly those in the inner system, still
% regard disorders as a mark of social stigma and may
% react negatively towards impaired characters.

% Characters that acquire disorders over the course
% of their adventures may get rid of them in one of two
% ways, either through in-game attempts to treat them
% (p. 214) or by buying them off as they would a nega-
% tive trait (p. 153).

% ADDICTION
% Addiction as a disorder can refer to any sort of addic-
% tive behavior focused toward a particular behavior or
% substance, to the point where the user is unable to func-
% tion without the addiction but is also severely impaired
% due to the effects of the addiction. It is marked by a
% desire on the part of the subject to seek help or reduce
% the use of the addicting substance/act, but also by the
% subject spending large amounts of time in pursuit of

% %%% txt/214.txt
% their addiction to the exclusion of other activities. This
% is a step up from Addiction negative trait listed on p.
% 148—this is much more of a crippling behavior that
% compensates for spending time away from the addic-
% tion. Addictions are typically related to the trauma
% that caused the disorder (VR or drug addictions are
% encouraged).

% Suggested Game Effects: The addict functions in
% only two states: under the influence of their addic-
% tion or in withdrawal. Additionally, they spend large
% amounts of time away from their other responsibili-
% ties in pursuit of their addiction.

% ATAVISM
% Atavism is a disorder that mainly affects uplifts. It
% results in them regressing to an earlier un- or partial-
% ly-uplifted state. They may exhibit behaviors more
% closely in line with their more animalistic forbears, or
% they may lose some of their uplift benefits such as the
% ability for abstract reasoning or speech.

% Suggested Game Effects: The player and gamemas-
% ter should discuss how much of the uplift’s nature is
% lost and adjust game penalties accordingly. It is im-
% portant to note that other uplifts view atavistic uplifts
% with something akin to horror and will usually have
% nothing to do with them.

% ATTENTION DEFICIT HYPERACTIVITY DISORDER (ADHD)
% This disorder manifests as a marked inability to focus
% on any one task for an extended period of time, and
% also an inability to notice details in most situations.
% Sufferers may find themselves starting multiple tasks,
% beginning a new one after only a cursory attempt at
% the prior task. ADHD suffers may also have a manic
% edge that manifests as confidence in their ability to get
% a given job done, even though they will quickly lose
% all interest in it.

% Suggested Game Effects: Perception and related
% skill penalties. Increased difficulty modifiers to task
% actions, particularly as the action drags on.

% AUTOPHAGY
% This is a disorder that usually only occurs among up-
% lifted octopi. It is a form of anxiety disorder character-
% ized by self-cannibalism of the limbs. Subjects afflicted
% with autophagy will, under stress, begin to consume
% their limbs, if at all possible, causing themselves po-
% tentially serious harm.

% Suggested Game Effects: Anytime an uplifted octopi
% with this disorder is placed in a stressful situation they
% must make a successful WIL x 3 Test or begin to con-
% sume one of their limbs.

% BIPOLAR DISORDER
% Bipolar disorder is also called manic depression. It is
% similar to depression except for the fact that the peri-
% ods of depression are interrupted by brief (a matter of
% days at most) periods of mania where the subject feels
% inexplicably “up” about everything with heightened
% energy and a general disregard for consequences. The
% depressive stages are similar in all ways to depression.
% The manic stages are dangerous since the subject
% will take risks, spend wildly, and generally engage in
% behavior without much in the way of forethought or
% potential long term consequences.

% Suggested Game Effects: Similar to depression, but
% when manic the character must make a WIL x 3 Test
% to not do some action that may be potentially risky.
% They will also try to convince others to go along with
% the idea.

% BODY DYSMORPHIA
% Subjects afflicted with this disorder believe that they
% are so unspeakably hideous that they are unable to
% interact with others or function normally for fear of
% ridicule and humiliation at their appearance. They
% tend to be very secretive and reluctant to seek help
% because they are afraid others will think them vain—
% or they may feel too embarrassed to do so. Ironi-
% cally, BDD is often misunderstood as a vanity-driven
% obsession, whereas it is quite the opposite; people
% with BDD believe themselves to be irrevocably ugly
% or defective. A similar disorder, gender identity dis-
% order, where the patient is upset with their entire
% sexual biology, often precipitates BDD-like feelings.
% Gender identity disorder is directed specifically at
% external sexually dimorphic features, which are in
% constant conflict with the patient’s internal psychi-
% atric gender.

% Suggested Game Effects: Because of the nature of
% Eclipse Phase and the ability to swap out and modify
% a body, this is a fairly common disorder. It is suggested
% that players with this suffer increased or prolonged
% resleeving penalties since they are unable to fully
% adjust to the reality of their new morph.

% BORDERLINE PERSONALITY DISORDER
% This disorder is marked by a general inability to fully
% experience one’s self any longer. Emotional states are
% variable and often marked by extremes and acting
% out. Simply put, the subject feels like they are losing
% their sense of self and seeks constant reassurance
% from others around them, yet is not fully able to act
% in an appropriate way. They may also engage in im-
% pulsive behaviors in an attempt to experience some
% sort of feeling. In extreme cases, there may be suicidal
% thoughts or attempts.

% Suggested Game Effects: The character needs to
% be around others and will not be left alone, however
% they also are not quite able to relate to others in a
% normal way and may also take risks or make impul-
% sive decisions.

% DEPRESSION
% Clinical depression is characterized by intense feelings
% of hopelessness and worthlessness. Subjects usually
% report feeling as though nothing they do matters and
% no one would care anyway, so they are disinclined to

% %%% txt/215.txt
% attempt much in the way of anything. The character is
% depressed and finds it difficult to be motivated to do
% much of anything. Even simple acts such as eating and
% bathing can seem to be monumental tasks.

% Suggested Game Effect: Depressives often lack the
% will to take any sort of action, often to the point of re-
% quiring a WIL x 3 Test to engage in sustained activity.

% FUGUE
% The character enters into a fugue state where they
% display little attention to external stimuli. They will
% still function physiologically but refrain from speak-
% ing and stare off into the distance, unable to focus on
% events around them. Unlike catatonia, a person in a
% fugue state will walk around if lead about by a helper,
% but is otherwise unresponsive. The fugue state is usu-
% ally a persistent state, but it can be an occasional state
% that is triggered by some sort of external stimuli simi-
% lar to the original trauma that triggered the disorder.

% Suggested Game Effects: Characters in a fugue state
% are totally non-responsive to most stimuli around
% them. They will not even defend themselves if at-
% tacked and will usually attempt to curl into a fetal
% position if physically assaulted.

% GENERAL ANXIETY DISORDER (GAD)
% GAD results in severe feelings of anxiety about nearly
% everything the character comes into contact with.
% Even simple tasks represent the potential for failure
% on a catastrophic scale and should be avoided or mini-
% mized. Additionally, negative outcomes for any action
% are always assumed to be the only possible outcomes.

% Suggested Game Effects: A character with GAD will
% be almost entirely useless unless convinced otherwise,
% and then only for a short period of time. Another
% character can attempt to use a relevant social skill to
% coax the GAD character into doing what is required
% of them. If the character with the disorder fails at the
% task, however, all future attempts to coax them will
% suffer a cumulative –10 penalty.

% HYPOCHONDRIA
% Hypochondriacs suffer from a delusion that they are
% sick in ways that they are not. They will create dis-
% orders that they believe they suffer from, usually to
% get the attention of others. Often hypochondriacs will
% inflict harm on themselves or even ingest substances
% that will aid in producing symptoms similar to the
% disorder they believe they have. These attempts to
% simulate symptoms can and will cause actual harm to
% hypochondriacs.

% Possible Game Effects: A subject that is hypochon-
% driac will often behave as though they are under the
% effects of some other disorder or physical malady.
% This can be consistent over time or can be different
% and ever changing. They will react with hostility to
% claims that they are faking or not actually ill.
% IMPULSE CONTROL DISORDER
% Subjects have a certain act or belief that they must
% engage in a certain activity that comes into their
% mind. This can be kleptomania, pyromania, sexual
% exhibitionism, etc. They feel a sense of building
% anxiety whenever they are prevented from engaging
% in this behavior for an extended period (usually sev-
% eral times a day to weekly, depending on the impulse)
% and will often attempt to engage in this behavior at
% inconvenient or inappropriate times. This is differ-
% ent from OCD in the sense that OCD is usually a
% single contained behavior that must be engaged in to
% reduce anxiety. Impulse control disorder is a variety
% of behaviors and can be virtually any sort of highly
% inappropriate action.

% Suggested Game Effects: Similar to OCD, if the
% player doesn’t engage in the behavior they will grow
% increasingly disturbed and suffer penalties to all ac-
% tions until they are able to engage in the compulsion
% that alleviates their anxiety.

% INSOMNIA
% Insomniacs find themselves unable to sleep, or unable
% to sleep for an extended period of time. This is most
% often due to anxiety about their lives or as a result
% of depression and the accompanying negative thought
% patterns. This is not the sort of sleeplessness that is
% brought about as a result of normal stress but rather
% a near total inability to find rest in sleep when it is
% desired. Insomniacs may find themselves nodding off
% at inopportune times, but never for long, and never
% enough to gain any restful sleep. As a result, they
% are frequently lethargic and inattentive as their lack
% of sleep robs them of their edge and eventually any
% semblance of alertness. Additionally, insomniacs are
% frequently irritable due to being on edge and unable
% to rest.

% Suggested Game Effects: Due to the lack of mean-
% ingful sleep, insomniacs should suffer from blanket
% penalties to perception related tasks or anything re-
% quiring concentration or prolonged fine motor abilities.

% MEGALOMANIA
% A megalomaniac believes themselves to be the single
% most important person in the universe. Nothing is
% more important than the megalomaniac and every-
% thing around them must be done according to their
% whim. Failure to comply with the dictates of a mega-
% lomaniac can often result in rages or actual physical
% assaults by the subject.

% Suggested Game Effects: A character that has
% megalomania will demand attention and has diffi-
% culty in nearly any social situation. Additionally, they
% may be provoked to violence if they think they are
% being slighted.

% %%% txt/216.txt
% MULTIPLE PERSONALITY DISORDER
% This is the development of a separate, distinct per-
% sonality from the original or control personality.
% The personalities may or may not be aware of each
% other and “conscious” during the actions of the other
% personality. Usually there is some sort of trigger that
% results in the emergence of the non-control personality.
% Most subjects have only a single extra personality, but
% it is not unheard of to have several personalities. It
% is important to note that these are distinct individual
% personalities and not just crude caricatures of the Dr.
% Jekyll/Mr. Hyde sort. Each personality sees itself as
% a distinct person with their own wants, needs, and
% motivations. Additionally, they are usually unaware
% of the experiences of the others, though there is some
% basic information sharing (such as language and core
% skill sets).

% Suggested Game Effects: When the player is under
% the effects of another personality, they should be
% treated as an NPC. In some rare cases the player and
% the gamemaster can work out the second personality
% and allow the player to roleplay this. This does not
% however constitute an entire new character that can
% be “turned on” at will.

% OBSESSIVE COMPULSIVE DISORDER (OCD)
% Subjects with OCD are marked by intrusive or in-
% appropriate thoughts or impulses that cause acute
% anxiety if a particular obsession or compulsion is not
% engaged in to alleviate them. These obsessions and
% compulsions can be nearly any sort of behavior that
% must be immediately engaged in to keep the rising
% anxiety at bay. Players and gamemasters are encour-
% aged to come up with a behavior that is suitable.
% Examples of common behaviors include repetitive
% tics (touching every finger of each hand to another
% part of the body, tapping the right foot twenty times),
% pathological behaviors such as gambling or eating,
% or a mental ritual that must be completed (reciting a
% book passage).

% Suggested Game Effects: If the player doesn’t
% engage in the behavior they will grow increasingly
% disturbed and suffer penalties to all actions until they
% are able to engage in the compulsion that alleviates
% their anxiety.

% POST TRAUMATIC STRESS DISORDER (PTSD)
% PTSD occurs as a result of being exposed to either
% a single incident or a series of incidents where the
% sufferer had their own life, or saw the lives of others,
% threatened with death. These incidents are often
% marked by an inability on the part of the victim, either
% real or perceived, to do anything to alter the outcomes.
% As a result, they develop an acute anxiety and fixation
% on these incidents to the point where they lose sleep,
% become irritated or easily angered, or are depressed
% over feelings that they lack control in their own lives.

% Suggested Game Effects: Penalties to task actions,
% also treat situations similar to the initial episodes that
% caused the disorder as a phobia.
% SCHIZOPHRENIA
% While schizophrenia is generally acknowledged as a
% genetic disorder that has an onset in early adulthood,
% it also seems to develop in a number of egos that un-
% dergo frequent morph changes. It has been theorized
% that this is due to some sort of repetitive error in the
% download process. Regardless, it remains a rare, yet
% persistent danger of dying and being brought back.
% Schizophrenia is a psychotic disorder where the sub-
% ject loses their ability to discern reality from unreality.
% This can involve delusions, hallucinations (often in
% support of the delusions), and fragmented or disor-
% ganized speech. The subject will not be aware of these
% behaviors and will perceive themselves as functioning
% normally, often to the point of becoming paranoid that
% others are somehow involved in a grand deception.

% Suggested Game Effects: Schizophrenia represents
% a total break from reality. A character that is schizo-
% phrenic may see and hear things and act on those
% delusions and hallucinations while seeing attempts
% by their friends to stop or explain to them as part of
% a wider conspiracy. Adding to this is the difficulty of
% communicating coherently. Players that have become
% schizophrenic are only marginally functional and
% only for short periods of time until they have the
% disorder treated.

% STRESSFUL SITUATIONS
% The universe of Eclipse Phase is ripe with experiences
% that might rattle a character’s sanity. Some of these are
% as mundane and human as extreme violence, extended
% isolation, or helplessness. Others are less common, but
% even more terrifying: encountering alien species, infec-
% tion by the Exsurgent virus, or being sleeved inside a
% non-human morph.

% WILLPOWER STRESS TESTS
% Whenever a character encounters a situation that
% might impact their ego’s psyche, the gamemaster may
% call for a (Willpower x 3) Test. This test determines if
% the character is able to cope with the unnerving situa-
% tion or if the experience scars their mental landscape.
% If they succeed, the character is shaken but otherwise
% unaffected. If they fail, they suffer stress damage (and
% possibly trauma) as appropriate to the situation. A
% list of stress-inducing scenarios and suggested SVs are
% listed on the Stressful Experiences table, p. 215. The
% gamemaster should use these as a guideline, modifying
% them as appropriate to the situation at hand.

% Note that some incidents may be so horrific that a
% modifier is applied to the character’s (Willpower x 3)
% Test.

% HARDENING
% The more you are exposed to horrible or terrifying
% things, the less scary they become. After repeated ex-
% posure, you become hardened to such things, able to
% shake them off without effect.

% %%% txt/217.txt


%            STRESSFUL EXPERIENCES
% SITUATION
% Failing spectacularly in pursuit of a motivational goal   1d10 ÷ 2
% Helplessness                                              1d10 ÷ 2
% Betrayal by a trusted friend                              1d10 ÷ 2
% Extended isolation                                        1d10 ÷ 2
% Extreme violence (viewing)                                1d10 ÷ 2
% Extreme violence (committing)
% Awareness that your death is imminent
% Experiencing someone’s death via XP
% Losing a loved one                                        1d10 ÷ 2
% Watching a loved one die                                        1
% Being responsible for the death of a loved one                  1
% Encountering a gruesome murder scene
% Torture (viewing)                                               1
% Torture (moderate suffering)                                    2
% Torture (severe suffering)                                      3
% Encountering aliens (non-sentient)                        1d10 ÷ 2
% Encountering aliens (sentient)
% Encountering hostile aliens                                     1
% Encountering highly-advanced technology                   1d10 ÷ 2
% Encountering Exsurgent-modified technology                 1d10 ÷ 2
% Encountering Exsurgent-infected transhumans
% Encountering Exsurgent life forms                               1
% Exsurgent virus infection                                   Varies
% Witnessing psi-epsilon sleights                                 1



%  Every time you succeed in a Willpower Test to avoid
%  taking stress from a particular source, take note. If you
%  successfully resist such a situation 5 times, you become
%  effectively immune to taking stress from that source.

%  The drawback to hardening yourself to such situa-
%  tions is that you grow detached and callous. In order
%  to protect yourself, you have learned to cut off your
%  emotions—but it is such emotions that make you
%  human. You have erected mental walls that will affect
%  your empathy and ability to relate to others.

%  Each time you harden yourself to one source of
%  stress, your maximum Moxie stat is reduced by 1.
%  Psychotherapy may be used to overcome such harden-
%  ing, in the same way a disorder is treated.

%  MENTAL HEALING AND PSYCHOTHERAPY
%  Stress is trickier to heal than physical damage. There
%  are no nano-treatments or quick fix options (other
%  than killing yourself and reverting to a non-stressed
%  backup). The options for recuperating are simply natu-
%  ral healing over time, psychotherapy, or psychosurgery.


%            PSYCHOTHERAPY CARE


%            Characters with an appropriate skill—


%            Medicine: Psychiatry, Academics: Psychol-
% und down)        ogy, or Professional: Psychotherapy—can


%             assist a character suffering mental stress
% und down)


%             or trauma with psychotherapy. This treat-
% und down)


%             ment is a long-term process, involving
% und down)        methods such as psychoanalysis, counsel-
% und down)        ing, roleplaying, relationship-building,
% 0                hypnotherapy, behavioral modification,
% 0                drugs, medical treatments, and even psy-
% 0                chosurgery (p. 229). AIs skilled in psycho-
% und down)        therapy are also available.


%                Psychotherapy is a task action, with a
% +2


%             timeframe of 1 hour per point of stress,
% +5              8 hours per trauma, and 40 hours per
% 0                disorder. Note that this only counts the
% +2               time actually spent in psychotherapy with
% +3               a skilled professional. After each psycho-
% +5               therapy session, make a test to see if the
% und down)        session was successful. Successful psycho-


%             surgery adds a +30 modifier to this test; at
% 0


%             the gamemaster’s discretion, other modi-
% +3


%             fiers may apply. Likewise, each disorder
% und down)        the character holds inflicts a –10 modifier.
% und down)       Traumas may not be healed until all stress
% 0                is eliminated.
% +3                  When a trauma is healed, the derange-
%  p. 366          ment associated with that trauma is
% +2               eliminated or downgraded. Disorders are


%             treated separately from the trauma that


%             caused them, and may only be treated


%            when all other traumas are removed.


%  Gamemaster and players are encouraged to roleplay

%  a character’s suffering and relief from traumas and

%  disorders. Each is an experience that makes a profound

%  impact on a character’s personality and psyche. The

%  process of treatment may also change them, so in the

%  end they may be a transformed from the person they

%  once were. Even if treated, the scars are likely to remain

%  for some time to come. According to some opinions,

%  disorders are never truly eradicated, they are just eased

%  into submission ... where they may linger beneath the

%  surface, waiting for some trauma to come along.


% NATURAL HEALING

% Characters who eschew psychotherapy can hopefully

% work out the problems in their head on their own over

% time. For every month that passes without accruing

% new stress, the character may make a WIL x 3 Test. If

% successful, they heal 1d10 points of stress or 1 trauma

% (all stress must be healed first). Disorders are even more

% difficult to heal, requiring 3 months without stress or

% trauma, and even then only being eliminated with a

% successful WIL Test. As a result, disorders can linger

% for years until resolved with actual psychotherapy.

% %%% txt/218.txt
% IND HACKS






%    If a character has the


%                    8
% PSI GAME MECHANICS
% rait, they can wield psi powers. ■ p. 220

% %%% txt/219.txt


%                  PSI-CHI S
% Psi-chi sleights are abilities that enha


%              perception and cogn


%                PSI-GAMMA SLIGHTS


%            Psi-gamma sleights analyze and influence


%     the functions of other biological minds. ■ p. 225






%          PSYCHOSURGERY


%          Psychosurgery can be used to control


%          behavior, implant skills, interrogate,


%          torture, and more. ■ pp. 229-231



% GHTS
% an async’s
% . ■ p. 223

% %%% txt/220.txt
%  emona: Glad to have you back. I hope                of the other s
%  a pleasant farcast from Pelion and don’t            still has other
% much lack. While you were out, a message             other agencies
% neas with a precis on psi, extracted from            this strain as “
% morph backup of psigeneticist Daborva                the subject do
%  Dipole Research Station on Ganymede),               something else
% uted for distribution to your Firewall node.         ality remains in


%                                                 avenue of inqu


%   Coined by the biologist Bertold P. Wiesner,   to nullify the e


%   “psi” was originally an umbrella term         strains, Firewa


%   used to describe a number of so-called        to experiment


%   “psychic” abilities and other speculative     cooperation of


%   paranormal phenomena such as telepa-          according to s


%   thy and extra-sensory perception. While       victims in the c


%   the term was used extensively in the          and hypercorps)


%   field of parapsychology and pop culture


%   in the twentieth and early twenty-first        THE NATU


%   centuries, the study of psi was largely          Labeled the


%   considered a pseudoscience with flawed         after the resea


%   methodologies and gradually lost funding      further study


%   and support.                                  the effect this


%      During the Fall, however, repeated         man brains. Ca


%   rumors and accounts of unexplained            subjects discov


%   phenomenon drew the attention of              synapses gene


%   scientists, military leaders, and singular-   wave pattern t


%   ity seekers alike. Numerous nanovirii had     to detect. Tho


%   been unleashed upon transhumanity,            come to refer


%   racing through populations and trans-         brainwaves as


%   forming as they spread. Some inflicted         the Greek lett


%   only minor biological or mental changes       brainwaves (alp


%   and impairments, but many were vicious        theta). Likewise


%   and deadly. The most feared variants,         known as “asyn


%   however, were those that Firewall has            Exploration o


%   come to label as the Exsurgent virus—a        tors behind psi


%   transformative nano-plague that mutates       Theories regard


%   its victims and subverts them to its will.    processes with


%   The Exsurgent virus was also observed         quantum states


%   to radically modify the subject’s neural      remain frustrat


%   patterns and mental state, affecting syn-     roimaging and


%   aptic arrangement and even modulating         scientists to pi


%   synaptic currents. These changes alter        the brain, neur


%   and enhance the victim’s cognition and        tions in the bra


%   seemed to endow an ability to sense and       are associated


%   even affect the minds of others from a        attempts to d


%   short distance—an ability dubbed “psi”        in non-infected


%   as the causal factors continue to mystify     failure or wor


%   us. The existence and nature of this phe-     asyncs by psi br


%   nomenon remains carefully concealed           even assured of


%   and under wraps in controlled habitats,       ends have prom


%   so as not to trigger widespread panic.        postulate that t


%   Among anarchist and other open com-           psi are simply


%   munities, knowledge of psi is more wide-      beyond transhu


%   spread, but details are vague and reports     of physical scie


%   are generally greeted with skepticism.        ing theories tha


%      The Exsurgent virus is exceptionally       fact of alien orig


%   mutable and adaptive, however, and two           One leading


%   argonaut researchers who were aware of        changes wroug


%   and studying it soon made an interesting      tion actually


%   discovery. One variant strain of the virus    neural sub-syst


%   was found that endowed the subject            quantum field


%   with exceptional mental abilities without     sibly create Bo


%   engaging the transformative process           within the brai

% s. Though infection      computation or perhaps hypercomputa-
% wbacks, Firewall and       tion. This enhances the async’s mental
%  ve come to regard         capabilities to the level provided by

% ” in the sense that      modern implants and neuro-mods—and
% ot transmogrify into       sometimes beyond. This does not explain
%  their general person-     the capabilities of other asyncs, however,

% . Intrigued that this    especially those used to read or affect
% might lead to a way        other biological minds. These abilities
%  s of other Exsurgent      seem to involve reading brain waves
% nd others continue         from a short range or affecting another’s

% the strain with the     mind via direct physical contact with the
%  ing test subjects (or     target’s bio-electric fields. Of course I can
% e reports, unwilling       only speculate in accordance with what
% of certain authorities     Firewall has uncovered—it is quite pos-


%                       sible that certain hypercorps or other fac-


%                       tions have made further breakthroughs,

% OF PSI                   but are keeping the information to
%  tts-MacLeod strain        themselves.
% ers who isolated it,          The initiation and use of psi talents
% gained insight into        is generally understood to take place on
% us has on transhu-         a subconscious level, meaning that the

% analysis of infected     async is not actively aware of the funda-
% d that their altered       mental processes that fuel the psi-waves.
%  a modulated brain-        Training in certain skills, however, allows
%  s extremely difficult      an async to focus on certain tasks and
%  in-the-know” have         psi abilities. These are called “sleights:”
%  hese asynchronous         mnemonic or cognitive algorithms of psi
%  waves,” fitting with       use rooted in the async’s ego.
% esignation of other           The percentage of the transhuman
%  beta, delta, gamma,       population believed to have contracted
% ected individuals are      the Watts-MacLeod strain remains sta-


%                       tistically insignificant—less than .001%
%  e explicit causal fac-    of the population. The vast number of

% es remains stymied.      asyncs have been recruited by various
% extraordinary mental       agencies, “disappeared” for study, or
% e ability to change        simply eliminated as a potential threat.
%  e been explored but          Ten years after the Fall, Firewall and
% y inconclusive. Neu-       other agencies have come to regard
%  pping have enabled        Watts-MacLeod infection as com-
%  nt structures within      paratively safe, though we remain quite
%  tivity, and perturba-     wary of unforeseen side effects or other
%  bioelectric field that     hidden dangers. Most of us engaged in
% h psi processes, but       studying the phenomenon now consider
% cate these features        asyncs to be useful as a tool for fighting
%  ins have resulted in      the Exsurgent virus and other threats—

% ttempts to identify      despite the protests of those who are
% wave patterns are not      convinced that asyncs are not in control
% cess. Numerous dead        of their own minds and are not to be

% many researchers to     trusted. As of yet we have encountered
% mechanics underlying       no cases of Watts-MacLeod infection
%  strange and too far       that have inflicted anything other than
% nity’s understanding       psi abilities, though there seems to be
% s—perhaps reinforc-        an increased risk for asyncs to succumb

% Exsurgent virus is in   to other Exsurgent strains should they


%                       encounter them. There are other risks as-
%  culation is that the      sociated with Watts-MacLeod infection,
% n the mind by infec-       such as extreme fatigue and even lethal
% angle some of the          biofeedback resulting from extensive use

% enable some sort of      of psi sleights and a statistically likeli-
%  in the brain, or pos-     hood of developing mental disorders due
%  instein condensates       to the increased mental stress placed on
%  lowing for quantum        the async’s mind.

% %%% txt/221.txt
% ÆTHER JABBER: ASYNCS


%       # Start Æther Jabber #                                  1


%                                                                   All


%       # Active Members: 2 #                                       bu


%                                                                   the


%           1


%               Sorry to bother you, but my muse just               to


%               alerted me to this excerpt that was                 for


%               sent around to my Firewall team. Is                 wh


%               this for real? I’ve heard the talk about            th


%               psi before—enough to be convinced                   bla


%               that there’s something to it, even          2


%                                                                   Th


%               if we can’t explain it—but this bit             1


%                                                                   Ma


%               about variant Exsurgent infection is                by


%               too much. Are we seriously going to                 de


%               be working with someone who’s a                     oc


%               known carrier? And can you shed any         2


%                                                                   We


%               more light on how asyncs do their                   yo


%               mojo? I’m worried now. And since                    th


%               you are connected to the Medeans, I                 na


%               thought I’d take the chance and ask.                ma


%       2


%               Well, as to the Medeans ... that’s his-             co


%               tory. I am back on the freelancing market           a


%               right now. But no problem, I’ll try and             fo


%               explain. I know it is not easy to grasp.            tec


%           1


%               Shiny.                                              a


%       2


%               Yes, Srit was once infected with a                  me


%               strain of the Exsurgent virus, probably             an


%               on Mars near the end of the Fall. I say             rie


%               “was” because the Watts-MacLeod                     na


%               strain seems to go dormant shortly                  an


%               after it finishes rewiring the victim’s              are


%               brain; the plague nanobots die off and              ov


%               get flushed out of the system, unlike                me


%               other Exsurgent strains, which con-                 tia


%               tinue to stick around and transform                 be


%               the subject. At least, that’s the domi-             Sy


%               nant theory—I’ve also seen some                     im


%               speculation that async minds might                  th


%               be modified so that they continue to                 oth


%               produce bio-nanobots that linger in             1


%                                                                   I’v


%               the brain, though what function these               tiv


%               serve remains unclear. However, the                 tru


%               prevailing opinion among our best           2


%                                                                   Ye


%               neuroscientists is that people like Srit            ab


%               are safe and non-infectious once the                mi


%               virus has run its course. I’ll even go              An


%               a bit further and say that prevailing               fro


%               opinion is that they can be trusted,                ph


%               assuming they don’t catch another                   ha


%               infection ... which they unfortunately              vu


%               seem to be a bit prone too. Not every-              Lik


%               one agrees of course, but we have an                br


%               abundance of paranoia in our circles.               ca


%               So far, we haven’t seen any evidence                thm


%               that any of our asyncs have been                1


%                                                                   Int


%               turned by that initial infection, and the           yo


%               utility and usefulness of having psi-               tha


%               actives on our side has simply been                 nu


%               too important to push aside.
%  t. I can’t say that I’ll trust her,   2


%                                           I’ve heard from several of
%  try and give her the benefit of                asyncs directly. The fact is, in
% ubt. I’ll be damned if I’m going               rewrites their brain, and so
% t an async that’s not vouched                  them came out the other side
%  Firewall though—who knows                     fundamentally altered. Eithe
% he hell a hypercorp like Skin-                 felt like a different person, o
% might be cooking up in their                   felt like there was somethin
% abs.                                           that was part of them—som
%  ems like a wise choice.                       that they didn’t necessarily lik

% you can put my mind at ease                  described it as presence, ano
%  laining to me in a bit more                   a black void that whispered a
% how Watts-MacLeod infection                    Yet another described it as g


%                                           personality to their unconsciou
%  ke the other Exsurgent strains                which only made the gulf be
%  e unfortunately familiar with,                unconscious and conscious m
%  mary transmission vector is a                 the more intimidating. Some o
%  rus, but we speculate that it                 preferred to suicide and reve
%  so be transmitted as a digital                pre-infection backup. While th
%  ter virus or possibly even as                 be more prone to cracking u
%  isk hack. The physical plague                 result, I haven’t ever heard o
%  s spread by highly-advanced                   about their abilities as som
%  -organic nanobots that infect                 they couldn’t control.
% morph and use bio-mimicry                  1


%                                           Well, that’s fucking cheery.
%  nisms to pass as normal cells                 nothing else we have on how

% netrate the blood-brain bar-                 stuff actually works?
%  d central nervous system. The         2


%                                           Unfortunately, we don’t. Ev
% ots are several steps beyond                   Prometheans haven’t been

% g our technology can produce,                help. There are theories, of cou

% y difficult to detect, and can                nothing that we’ve been able t
% helm most defensive counter-                   with rigorous experimentat
%  res. Infected minds are essen-                doesn’t help that the factions t
% ewired, and these changes will                 aware of psi’s existence don’t

% ed when the ego is uploaded.                 compare notes—they’re all to
% morphs and infomorphs remain                   looking into ways to weapo

% e to this nano-infection, but                and use it against each other,
%  re theoretically vulnerable to                of figuring out how to use it

% ansmission vectors.                          benefit of transhumanity.
% ard that synthmorphs are effec-            1


%                                           Of course. The TITANs didn’t
%  nvulnerable to psi as well. This              but we can still get ourselves.


%                                           ries me that the best we’ve co
% s far as we can tell, async                    with is nothing.
% es only effect biological              2


%                                           It’s important to keep persp
% —either their own or others.                   Transhumanity has come q
%  ey can only read/affect others                distance and made some imp

% very short distance, requiring               accomplishments, but our unde
% al contact in most cases. The                  ing of the universe is still in its
% ological minds of pods are also                What we may be facing here is
% able, though to a lesser extent.               thing concocted by an intellige
% se, asyncs need a biological                   far beyond our own that we
% to use their abilities—they                    insignificant insects in comp
%  se their psi if sleeved in a syn-             It likely has a grasp on the u
%  h and have difficulty in a pod.                that is simply beyond our ab
% ting. So, I have to ask again—                 understand. We shouldn’t be

% sure she’s safe? I’ve heard                  and think that we can deciph
%  me of these asyncs can be real                mystery thrown at us ... we
%  s.                                            instead be very, very afraid.

% %%% txt/222.txt
% MIND HACKS■MIND HACKS■MIND HAC



%  HACKS

% Though neuroscience has ascended to impressive

% pinnacles, allowing minds to be thoroughly scanned,

% mapped, and emulated as software, the transhuman

% brain remains a place that is complicated, not fully

% understood, and thoroughly messy. Despite a preva-

% lence of neural modifications, meddling with the seat

% of consciousness remains a tricky and hazardous

% procedure. Nevertheless, psychosurgery—editing the

% mind as software—remains common and widespread,

% sometimes with unexpected results.


% Likewise, even as the knowledge of neuroscientists

% grows on an exponential basis, some are discovering

% that minds are far more mysterious than they had ever

% imagined. During the Fall, scattered reports of “anom-

% alous activity” by individuals infected by one of the

% numerous circulating nanoplagues were discounted

% as fear and paranoia, but subsequent investigations

% by black budget labs has proven otherwise. Now, top-

% level confidential networks whisper that this infection

% inflicts intricate changes in the victim’s neural network

% that imbue them with strange and inexplicable abili-

% ties. The exact mechanism and nature of these abilities

% remains unexplained and outside the grasp of modern

% transhuman science. Given the evidence of a new

% brainwave type and the paranormal nature of this

% phenomenon, it is loosely referred to as “psi.”




% PSI

% In Eclipse Phase, psi is considered a special cognitive

% condition resulting from infection by the mutant—and

% hopefully otherwise benign—Watts-Macleod strain

% of the Exsurgent virus (p. 367). This plague modifies

% the victim’s mind, conferring special abilities. These

% abilities are inherent to the brain’s architecture and

% are copied when the mind is uploaded, allowing the

% character to retain their psi abilities when changing

% from morph to morph.


% PREREQUISITES

% To wield psi, a character must acquire the Psi trait (p.

% 147) during character creation. It is theoretically also

% possible to acquire the use of psi in game via infec-

% tion by the Watts-MacLeod strain; see The Exsurgent

% Virus, p. 362.


% Psi ability is considered an innate ability of the

% ego and not a biological or genetic predisposition of

% the morph. While psi researchers do not understand

% how it is possible to transfer this ability via uploads,

% backups, and farcasting, it has been speculated that

% all components of an async’s ego are entangled on a

% quantum level, or that they possess the ability to en-

% tangle themselves or form a unique conformation or

% alignment as a whole even after they have been copied,

% up-, or downloaded. This speculated entanglement
% ■MIND HACKS■MIND HACKS■MIND H



%                                                         8


%  process is also thought to be the origin of the impair-
%  ment that asyncs experience when adapting to a new
%  morph (see below).

%  MORPHS AND PSI
%  Asyncs require a biological brain to draw on their
%  abilities (the brains of uplifted animals count). An
%  async whose ego is downloaded into an infomorph or
%  fully computerized brain (synthmorphs) has no access
%  to their abilities as long they remain in that morph.

% Asyncs inhabiting a pod morph may use psi, but
%  their abilities are restricted as pod brains are only
%  partly biological. Pod-morphed asyncs suffer a –30
%  modifier on all tests involving the use of psi sleights
%  and the impact from using sleights would be doubled.

%  MORPH ACCLIMATIZATION
%  Async minds undergo extra difficulty adjusting to new
%  morphs. For 1 day after the character has resleeved,
%  they will suffer the effects of a single derangement
%  (p. 210). The gamemaster and player should choose a
%  derangement appropriate to the character and story.
%  Minor derangements are recommended, but at the
%  gamemaster’s discretion moderate or major derange-
%  ments may be applied. No trauma is inflicted with
%  this derangement.

%  MORPH FEVER
%  Asyncs find it irritating and traumatizing to endure life
%  as an infomorph, pod, or synthmorph for long periods
%  of time. This phenomenon, known as morph fever,
%  might cause temporary derangements and trauma to
%  the asyncs’ ego, possibly even to the grade of perma-
%  nent disorders. If stored or held captive as an active
%  infomorph (i.e. not in virtual stasis), the async might
%  go insane if not psychologically aided by some sort of
%  anodyne program or supporting person during storage.

% In game terms, asyncs take 1d10 ÷ 2 (round up)
%  points of mental stress damage per month they stay
%  in a pod, synthmorph or infomorph form without
%  psychological assistance by a psychiatrist, software,
%  or muse.

%  PSI DRAWBACKS
%  There are several drawbacks to psi ability:

% • The variant Exsurgent strain that endows psi ability

% rewires the character’s brain. An unfortunate side

% effect to this change is that asyncs acquire a vulner-

% ability to mental stress. Reduce the async’s Trauma

% Threshold by 1.
% • The mental instability that accompanies psi infec-

% tion also tends to unhinge the character’s mind.

% Asyncs acquires one Mental Disorder negative trait

% (p. 150) for each level they have of the Psi trait

% %%% txt/223.txt

% without receiving any bonus CP. The gamemaster

% and player should agree on a disorder appropriate

% to the character. This disorder may be treated over

% time, according to normal rules (see Mental Healing

% and Psychotherapy, p. 215).
% • Characters with the Psi trait are also vulnerable to

% infection by other strains of the Exsurgent virus.

% The character suffers a –20 modifier when resisting

% Exsurgent infection (p. 362).
% • Critical failures when using psi tend to stress the

% async’s mind. Each time a critical failure is rolled

% when making a sleight-related test, the async suffers

% a temporary brain seizure. They suffer a –30 modi-

% fier and are incapable of acting until the end of the

% next Action Turn. They must also succeed in a WIL

% + COG Test or fall down.

% PSI SKILLS AND SLEIGHTS
% Transhuman psi users can manipulate their egos and
% otherwise create effects that can often be neither
% matched nor mimicked by technological means. To use
% these abilities, they train their mental processes and
% practice cognitive algorithms called sleights, which
% they can subconsciously recall and use as necessary.
% Sleights fall into two categories: psi-chi (cognitive
% enhancements, p. 223) and psi-gamma (brainwave
% reading and manipulation, p. 225). Psi-chi sleights are
% available to anyone with the Psi trait (p. 147), but psi-
% gamma sleights are only available to characters with
% the Psi trait at Level 2. In order to use these sleights,
% the async must be skilled in the Control (p. 178), Psi
% Assault (p. 183), and/or Sense skills (p. 184), as ap-
% propriate to each sleight.

% ROLEPLAYING ASYNCS
% Any player who chooses to play an async should keep
% the origin of their abilities in mind: Watts-MacLeod
% strain infection. The character may not be aware of
% this source, but they undoubtedly know that they un-
% derwent some sort of transformation and have talents
% that no one else does. If unaware of the infection, they
% have likely learned to keep their abilities secret lest
% they be ridiculed, attacked, or whisked away to some
% secret testing program. Learning the truth about their
% nature could even be the starting point of a campaign
% and/or their introduction to Firewall. If they know the
% truth, however, the character must live with the fact
% that they are the victim of a nanoplague likely spread
% by the TITANs that may or may not lead to complica-
% tions, side effects, or other unexpected revelations in
% their future.

% Gamemasters and players should make an effort
% to explore the nature of this infection and how the
% character perceives it. As noted previously, asyncs are
% often profoundly-changed people. The invasive and
% alien aspect of their abilities should not be lost on
% them. For example, an async might conceive of their
% psi talents as a sort of parasitic entity, living off their
% sleights, or they might feel that using these powers
% puts them in touch with some sort of fundamental
% substrate of the universe that is weird and terrifying.
% Alternately, they could feel as if their personality was
% melded with something different, something that
% doesn’t belong. Each async is likely to view their situ-
% ation differently, and none of them pleasantly.

% USING PSI
% Using psi—i.e., drawing on a certain sleight to pro-
% cure some kind of effect—does not always require a
% test. Each sleight description details how the power
% is used.

% ACTIVE PSI
% Active psi sleights must be “activated” to be used.
% These sleights usually require a skill test. Sleights that
% target other sentient beings or life forms are always
% Opposed Tests, while others are handled as Success
% Tests. The level of concentration required to use these
% sleights varies, and so may call for a Quick, Complex,
% or Task Action. Active sleights also cause strain (p.
% 223) to the async. Most psi-gamma sleights fall into
% this category.

% PASSIVE PSI
% Passive psi sleights encompasses abilities that are con-
% sidered automatically active and subconscious. They
% rarely require an action to be activated and require
% no effort or strain by the psi user. Passive sleights typi-
% cally add bonuses to various activities or allow access
% to certain abilities rather than calling for some kind of
% skill test. Most psi-chi sleights fall into this category.

% PSI RANGE
% Sleights have a Range of either Self, Touch, or Close.

% Self: These sleights only affect the async.

% Touch: Sleights with a Touch range may be used
% against other biological life, but the async must have
% physical contact with the target. If the target avoids
% being touched, this requires a successful melee attack,
% applying the touch-only +20 modifier. This attack
% does not cause damage, and is considered part of the
% same action as the psi use.

% Close: Close sleights involve interaction with other
% biological life from a short distance. The optimal dis-
% tance is within 5 meters. For each meter beyond that,
% apply a –10 modifier to the test.

% Psi vs. Psi: Due to the nature of psi, sleights are
% more effective against other psi users. Sleights with
% a range of Touch may be used from a Close range
% against another async. Likewise, a sleight with a Close
% range may be used at twice the normal distance (10
% meters) when wielded on another async.

% TARGETING
% Synthmorphs, bots, and vehicles may not be targeted
% by psi sleights, as they lack biological brains. Pods—
% with brains that are half biological and half com-
% puter—are less susceptible and receive a +30 modifier

% %%% txt/224.txt
% when defending against psi use. Note that infomorphs
% may never be targeted by psi sleights as psi is not ef-
% fective within the mesh or simulspace.

% Multiple Targets: An async may target more than
% one character with a sleight with the same action, as
% long as each of them can be targeted via the rules
% above. The psi character only rolls once, with each of
% the defending characters making their Opposed Tests
% against that roll. The psi character suffers strain (p.
% 223) for each target, however, meaning that using psi
% on multiple targets can be extremely dangerous.

% Animals and Less Complex Life Forms: Psi works
% against any living creature with a brain and/or
% nervous system. Against partially-sentient and par-
% tially-uplifted animals, it suffers a –20 modifier and
% increases strain by +1. Against non-sentient animals,
% it suffers a –30 modifier and increases strain by +3. It
% has no effect on or against less complex life forms like
% plants, algae, bacteria, etc.

% Factors and Aliens: At the gamemaster’s discretion,
% psi sleights may not work on alien creatures at all, de-
% pending on their physiology and neurology. If it does
% work, it is likely to suffer at least a –20 modifier and
% +1 strain.

% OPPOSED TESTS
% Psi that is used against another character is resisted
% with an Opposed Test. Defending characters resist with
% WIL x 2. Willing characters may choose not to resist.
% Unconscious or sleeping characters cannot resist.

% If the psi-wielding character succeeds and the de-
% fender fails, the sleight affects the target. If the psi user
% fails, the defender is unscathed. If both parties suc-
% ceed in their tests, compare their dice rolls. If the psi
% user’s roll is higher, the sleight bypasses the defender’s
% mental block and affects the target; otherwise, the
% sleight fails to affect the defender’s ego.

% TARGET AWARENESS
% The target of a psi sleight is aware they are being
% targeted any time they succeed on their half of the
% Opposed Test (regardless on whether the async
% rolls higher or not). Note that awareness does not
% necessarily mean that the target understands that psi
% abilities are being used on them, especially as most
% people in Eclipse Phase are unaware of psi’s existence.
% Instead, the target is simply likely to understand that
% some outside influence is at work, or that something
% strange is happening. They may suspect that they
% have been drugged or are under the influence of some
% strange technology.

% Targets who fail their roll remain unaware.

% PSI FULL DEFENSE
% Like full defense in physical combat (p. 198), a de-
% fender may spend a Complex Action to rally and con-
% centrate their mental defenses, gaining a +30 modifier
% to their defense test against psi use until their next
% Action Phase.

% %%% txt/225.txt
% CRITICALS
% If the defender rolls a critical success, the character
% attempting to wield psi is temporarily locked out of
% the target’s mind. The psi user may not target that
% character with sleights until an appropriate “reset”
% period has passed, determined by the gamemaster.

% If the async rolls a critical failure, they suffer tempo-
% rary incapacitation as their mind dysfunctions in some
% harsh and distressing ways (see Psi Drawbacks, p. 221).

% If a psi user rolls a critical success against a de-
% fender, or the defender rolls a critical failure, double
% the potency of the sleight’s effect. In the case of psi at-
% tacks, the DV can be doubled or mental armor can be
% bypassed. Alternately, when using Psi Assault (p. 183),
% the targeted character may be in danger of infection
% by the Watts-Macleod strain (p. 362).

% MENTAL ARMOR
% The Psi Shield sleight (p. 228) provides mental armor,
% a form of neural hardening against psi-based attacks.
% Like physical armor, this mental armor reduces the
% amount of damage inflicted by a psi assault.

% DURATION
% Psi sleights have one of four durations: constant, in-
% stant, temporary, or sustained.

% Constant: Constant sleights are always “on.”

% Instant: Instant sleights take effect only in the
% Action Phase in which they are used.

% Temporary: Temporary sleights last for a limited
% duration with no extra effort from the async. The
% temporary duration is determined by the async’s WIL
% ÷ 5 (round up) and is measured in either Action Turns
% or minutes, as noted. Strain for the sleight is applied
% immediately when used, not at the end of the duration.

% Sustained: Sustained sleights require active effort
% to maintain for as long as the async wants to keep
% it active. Sustaining a sleight requires concentration,
% and so the async suffers a –10 modifier to all other
% skill tests while the sleight is sustained. The async
% must also stay within the range appropriate to the
% sleight, otherwise the sleight immediately ends. More
% than one sleight may be sustained at a time, with a
% cumulative modifier. Strain for the sleight is applied
% immediately when used, not at the end of the dura-
% tion. At the gamemaster’s discretion, sleights that are
% sustained for long periods may incur additional strain.

% STRAIN
% The use of psi is physically (and sometimes psycho-
% logically) draining to a psi user. This phenomenon is
% known as strain, and manifests as fatigue, exhaustion,
% pain, neural overload, cardiovascular stress, and
% adynamia (loss of vigor). Though strain has only
% rarely been known to actually kill an async, the use
% of too much active psi can be life-threatening in some
% circumstances.

% In game terms, every active sleight has a Strain
% Value of 1d10 ÷ 2 (round up) DV. Every active sleight
% lists a Strain Value Modifier that modifies this amount.
% For example, a sleights with a Strain Value Modifier
% of –1 inflicts (1d10 ÷ 2) –1 DV.

% If the damage points suffered from strain exceed
% the character’s Wound Threshold, they may inflict a
% wound just like other damage (see Wounds, p. 207).




%       Matric is investigating a disappearance, so he


%       decides to use his Qualia sleight to boost his


%       Intuition while hunting for clues. That psi-chi


%       sleight takes only a Quick Action to initiate and


%       requires no test. Matric’s WIL is 25, so the dura-


%       tion of this temporary sleight is 5 Action Turns (25


%       ÷ 5 = 5). The sleight’s Strain modifier is –1, so


%       he is facing (1d10 ÷ 2) –1 DV. He rolls a 1, so he


%       takes no strain at all!


%          Later on, Matric finds himself in a life-or-death


%       struggle with a kidnapper. Lucky for Matric,


%       they’re in a melee, so he’s close enough to try
%  EXAMPLE






%       and touch his opponent. On his Action Phase, he


%       makes an Unarmed Combat Test with a +20 mod-


%       ifier (for a touch-only attack) and succeeds. This


%       allows him to try and use his Psychic Stab sleight.


%       He rolls his Psi Assault of 57 against the target’s


%       WIL x 2 (32). His target is in a worker pod morph,


%       however, which is less susceptible to psi, so he


%       receives a +30 modifier (32 + 30 = 62). Matric


%       rolls a 32 and the worker pod a 64—Matric wins!


%       For damage, he rolls 1d10 + (WIL ÷ 10). His WIL


%       is 25, so that’s 1d10 + 3. He rolls a score a 7 and


%       inflicts 10 (7 + 3) points of damage. The worker


%       pod screams in pain, suffering a wound from the


%       psychic assault.


% PSI-CHI SLEIGHTS
% Psi-chi sleights are async abilities that speed up cogni-
% tive informatics (internal information processing) and
% enhance the user’s perception and cognition.

% AMBIENCE SENSE


%   PSI TYPE:    Passive        ACTION:    Automatic

% RANGE:      Self        DURATION: Constant
% This sleight provides the async with an instinctive
% sense about an area and any potential threats nearby.
% The async receives a +10 modifier to all Investigation,
% Perception, Scrounging, and Surprise Tests.

% COGNITIVE BOOST


%   PSI TYPE:    Active         ACTION:    Quick


%   RANGE:       Self           DURATION: Temp (Action Turns)


%  STRAIN MOD: –1
% The async can temporarily elevate their cognitive
% performance. In game terms, Cognition is raised by 5
% for the chosen duration. This boost to Cognition also
% raises the rating of skills linked to that aptitude.

% %%% txt/226.txt
% DOWNTIME

% PSI TYPE:   Active        ACTION:     Task (min. 4 hours)

% RANGE:      Self          DURATION:   Sustained

% STRAIN MOD: 0
% This sleight provides the async with the ability
% to send the mind into a fugue-state regenerative
% downtime, during which the character’s psyche is
% repaired. The async must enter the downtime for
% at least 4 hours; every 4 hours of downtime heals
% 1 point of stress damage. Traumas, derangements,
% and disorders are unaffected by this sleight. For
% all sensory purposes, the async is catatonic during
% downtime, completely oblivious to the outside
% world. Only severe disturbances or physical shock
% (such as being wounded or hit by a shock weapon)
% will bring the async out of it.

% EMOTION CONTROL

% PSI TYPE:   Passive       ACTION:     Automatic

% RANGE:      Self          DURATION: Constant
% Emotion Control gives the async tight control
% over their emotional states. Unwanted emotions
% can be blocked out and others embraced. This has
% the benefit of protecting the async from emotional
% manipulation, such as the Drive Emotion sleight or
% Intimidation skill tests. The async receives a +30
% modifier when defending against such tests.

% ENHANCED CREATIVITY

% PSI TYPE:   Passive       ACTION:     Automatic

% RANGE:     Self         DURATION: Constant
% An async with Enhanced Creativity is more imagi-
% native and more inclined to think outside the box.
% Apply a +20 modifier to any tests where creativ-
% ity plays a major role. This level of ingenuity can
% sometimes seem strange and different, manifesting
% in odd or creepy ways, especially with artwork.

% FILTER

% PSI TYPE:   Passive       ACTION:     Automatic

% RANGE:       Self         DURATION: Constant
% Filter allows the async to filter out out distractions
% and eliminate negative situational modifiers from
% distraction, up to the gamemaster’s discretion.

% GROK

% PSI TYPE:   Active        ACTION:     Complex

% RANGE:      Self          DURATION:   Instant

% STRAIN MOD: –1
% By using the Grok sleight, the async is able to
% intuitively understand how any unfamiliar object,
% vehicle, or device is used simply by looking at and
% handling it. If the character succeeds in a COG x 2
% Test, they achieve a basic ability to use the object,
% vehicle, or device, no matter how alien or bizarre.
% This sleight does not provide any understanding of
% the principles or technologies involved—the psi user
% simply grasps how to make it work. If a test is called
% for, the psi user receives a +20 modifier to use the
% device (this bonus only applies to unfamiliar devices,
% and/or tests the character is defaulting on—it does not
% apply to devices the character is familiar with).

% HIGH PAIN THRESHOLD

% PSI TYPE:   Passive      ACTION:   Automatic

% RANGE:       Self        DURATION: Constant
% This sleight allows the async to block out, ignore, or
% otherwise isolate pain. The async reduces negative
% modifiers from wounds by 10.

% HYPERTHYMESIA

% PSI TYPE:   Passive      ACTION:   Automatic

%  RANGE:      Self         DURATION: Constant
% Hyperthymesia grants the async a superior auto-
% biographical memory, allowing them to remember the
% most trivial of events. A hyperthymestic async can be
% asked a random date and recall the day of the week
% it was, the events that occurred that day, what the
% weather was like, and many seemingly trivial details
% that most people would not be able to recall.

% INSTINCT

% PSI TYPE:   Passive      ACTION:   Automatic

%  RANGE:      Self        DURATION: Constant
% Instinct bolsters the async’s subconscious ability to
% gauge a situation and make a snap judgment that
% is just as accurate as a careful, considered decision.
% For Task Actions that involve analysis or planning
% alone (typically Mental skill actions), the async may
% reduce the timeframe by 90% without suffering a
% modifier. For Task Actions that involve partial analy-
% sis/planning, they may reduce the timeframe by 30%
% without penalty.

% MULTITASKING

% PSI TYPE:   Passive      ACTION:   Automatic

% RANGE:     Self         DURATION: Constant
% The async can handle vast amounts of information
% without overload and can perform more than one
% mental task at once. The character receives an extra
% Complex Action each Action Phase that may only be
% used for mental or mesh actions.

% PATTERN RECOGNITION

% PSI TYPE:   Passive      ACTION:   Automatic

%  RANGE:     Self         DURATION: Constant
% The character is adept at spotting patterns and corre-
% lating the non-random elements of a jumble—related
% items jump out at them. This is useful for translating
% languages, breaking codes, or find clues hidden among
% massive amounts of data. The character must have

% %%% txt/227.txt
% a sufficiently large sample enough time to study, as
% determined by the gamemaster. This might range from
% a few hours of listening to a spoken transhuman lan-
% guage to a few days of investigating inscriptions left
% by long-dead aliens to a week or more of researching
% a lengthy cipher. Languages may be comprehended by
% reading or listening to them being spoken. Apply a
% +20 modifier to any appropriate Language, Investiga-
% tion, Research, or cod-breaking Tests (note that this
% does not apply to Infosec Tests made by software to
% decrypt a code). The async may also use this ability to
% more easily learn new languages, reducing the training
% time by half.

% PREDICTIVE BOOST

% PSI TYPE:   Passive       ACTION:    Automatic

%  RANGE:     Self        DURATION: Constant
% The Bayesian probability machine features of the
% async’s brain are boosted by this sleight, enhancing
% their ability to estimate and predict outcomes of
% events around them as they unfold in real-time and
% update those predictions as information changes.
% In effect, the character has a more intuitive sense
% for which outcomes are most likely. This grants the
% character a +10 bonus on any skill tests that involve
% predicting the outcome of events. It also bolsters
% the async’s decision-making in combat situations by
% making the best course of action more clear, and so
% provides a +10 bonus to both Initiative and Fray Tests.

% QUALIA

% PSI TYPE:   Active        ACTION:    Quick

% RANGE:      Self          DURATION: Temp (Action Turns)


%  STRAIN MOD: –1
% The async can temporarily increase their intuitive
% grasp of things. In game terms, Intuition is raised by
% 5 for the chosen duration. This boost to Intuition also
% raises the rating of skills linked to that aptitude.

% SAVANT CALCULATION

% PSI TYPE:   Passive       ACTION:    Automatic


% RANGE:      Self          DURATION: Constant
% The character possesses an incredible facility with
% intuitive mathematics. They can do everything from
% calculate the odds exactly when gambling to predict-
% ing precisely where a leaf falling from a tree will land
% by observing the landscape and local wind currents.
% The character specializes in calculation involving the
% activity of complex chaotic systems and so can calcu-
% late answers that even the fastest computers could not,
% including things like patterns of rubble distribution
% from an explosion. However, this mathematic facility
% is largely intuitive, so the character does not know the
% equations they are solving, they merely know the solu-
% tion to the problem.

% This sleight also provides a +30 modifier to any
% skill tests involving math (which the character is cal-
% culating, not a computer).
% SENSORY BOOST

% PSI TYPE:   Active        ACTION:    Quick

% RANGE:      Self          DURATION: Temp (Action Turns)


%  STRAIN MOD: –2
% An async uses this sleight to increase their natural or
% augmented sensory perception (sight, audio, smell,
% augmented) by enhanced cerebral processing, grant-
% ing a +20 bonus modifier on sensory-based Perception
% Tests.

% SUPERIOR KINESICS

% PSI TYPE:   Passive       ACTION:    Automatic

%  RANGE:     Self          DURATION: Constant
% The async acquires more insight into people’s emo-
% tive signals, gestures, facial expressions, and body
% language when it comes time to predict the person’s
% emotional state, intent, or reactions. Apply a +10
% modifier to Kinesics Skill Tests.

% TIME SENSE

% PSI TYPE:   Active        ACTION:    Automatic

% RANGE:      Self          DURATION: Temp (Action Turns)


%  STRAIN MOD: –1
% An async with this ability can slow down his per-
% ception of time, making everything appear to move
% in slow motion or at reduced speed. In game terms,
% this sleight grants the async a Speed of +1. This extra
% Action Phase, however, can only be spent on mental
% and mesh actions.

% UNCONSCIOUS LEAD

% PSI TYPE:   Active        ACTION:    Automatic

% RANGE:      Self          DURATION: Temp (Action Turns)


%  STRAIN MOD: +0
% This sleight allows the async to override their con-
% sciousness and let their unconscious mind take point.
% While in this state, the async’s conscious mind is only
% dimly aware of what is transgressing, and any memo-
% ries of this period will be hazy at best. The advantage
% is that the unconscious mind acts more quickly, and
% so the async’s Speed is boosted by +1. The character
% remains aware and active, but is incapable of com-
% plex communication or other mental actions and is
% motivated by instinct and primitive urges more than
% conscious thought. Though it is recommended that
% the player retain control of their character while using
% Unconscious Lead, the gamemaster should feel free to
% direct the character’s actions as they see fit.

% %%% txt/228.txt
% PSI-GAMMA SLEIGHTS

% Psi-gamma sleights deal with contacting (reading
% and communicating) and influencing the function of
% biological minds (egos within a biomorph, but also
% including animal life). Psi-gamma is only available to
% characters with Level 2 of the Psi trait.

% Most psi-gamma use is handled as an Opposed Test
% between the async and the target (p. 222).

% ALIENATION

%  PSI TYPE:   Active        ACTION:     Complex

%  RANGE:      Touch         DURATION:   Temp (Action Turns)

%  STRAIN MOD: +0             SKILL:    Psi Assault
% Alienation is an offensive sleight that create a sense of
% disconnection between an ego and its morph—similar
% to that experienced when resleeved into a new body.
% The ego finds their body cumbersome, strange, and
% alien, almost like they are a prisoner within it. If the
% async beats the target in an Opposed Test, treat the
% test as a failed Integration Test (p. 272) for the target.
% This effect lasts for the sleight’s duration.

% CHARISMA

%  PSI TYPE:   Active        ACTION:     Quick

%  RANGE:      Touch         DURATION:   Temp (Minutes)

%  STRAIN MOD: –1           SKILL:   Control
% The async uses this sleight to influence the target’s
% mind on a subconscious level, so that the target per-
% ceives them to be charming, magnetic, and persuasive.
% If the async beats the target in an Opposed Test, they
% gain a +30 modifier on all subsequent Social Skill
% Tests for the chosen duration.

% CLOUD MEMORY

%  PSI TYPE:   Active        ACTION:     Complex

%  RANGE:      Touch         DURATION:   Temp (Minutes)

%  STRAIN MOD: –1             SKILL:   Control
% Cloud Memory allows the async to temporarily dis-
% rupt the target’s ability to form long-term memories. If
% the async wins the Opposed Test, the target’s memory-
% saving ability is negated for the duration. The target
% will retain short-term memories during this time, but
% will soon forget anything that occurred while this
% sleight was in effect.

% DEEP SCAN

%  PSI TYPE:   Active        ACTION:     Complex

%  RANGE:      Touch         DURATION:   Sustained

%  STRAIN MOD: +1            SKILL:    Sense
% Deep Scan is a more intrusive version of Thought
% Browse (p. 228), made to extract information from the
% targeted individual. If the Opposed Test succeeds, the
% async telepathically invades the target’s mind and can
% probe it for information. For every 10 full points of MoS
% the async achieves on their test, they retrieve one piece
% of information. Each item takes one full Action Turn to
% retrieve, during which the sleight must be sustained. The
% target is aware of this mental probing, though they will
% not know what information the async acquired.

% DRIVE EMOTION

% PSI TYPE:     Active      ACTION:     Complex

% RANGE:        Touch       DURATION:   Temp (Action Turns)

%  STRAIN MOD: –1            SKILL:   Control
% This sleight allows the async to stimulate cortical areas
% of the target’s brain related to emotion. This allows
% the async to induce, amplify, or tone down specific
% emotions, thereby manipulating the target. If the async
% beats the target in an Opposed Test, they will act in ac-
% cordance with the emotion for the duration and under
% certain circumstances may suffer from certain penalties
% (up to +/–30), as determined by the gamemaster. For
% example, an async might receive a +30 Intimidation
% Test modifier against a target imbued with fear.

% EGO SENSE

% PSI TYPE:     Active      ACTION:     Complex

% RANGE:        Close       DURATION:   Temp (Action Turns)


% STRAIN MOD: –1           SKILL:   Sense
% Ego Sense can be used to detect the presence and
% location of other sentient and biological life forms
% (i.e., egos) within the async’s range. To detect these
% life forms, the async makes a single Sense Test, op-
% posed by each life form within range. The async may
% suffer a modifier for detecting small animals and
% insects, similar to the modifier applied for target-
% ing them in ranged combat (see p. 193); likewise, a
% modifier for detecting larger life forms may also be
% applied. If successful, the async has detected that the
% life form is nearby. Every 10 full points of MoS will
% ascertain another piece of information regarding the
% detected life: direction from async, approximate size,
% type of creature, distance from async, etc. The async
% will know if the target moves, if they do so during the
% sleight’s duration.

% EMPATHIC SCAN

% PSI TYPE:     Active      ACTION:     Quick

% RANGE:        Close       DURATION:   Sustained

%  STRAIN MOD: –2             SKILL:    Sense
% Empathic Scan enables the async to sense the target’s
% base emotions. If the async wins the Opposed Test,
% they intuitively feel the target’s emotional current state
% for as long as the sleight is sustained. At the gamemas-
% ter’s discretion, this knowledge may provide a modifier
% (up to +30) for certain Social skill tests.

% IMPLANT MEMORY

% PSI TYPE:     Active      ACTION:     Complex

% RANGE:        Touch       DURATION:   Instant

% STRAIN MOD:   +0          SKILL:      Control

% %%% txt/229.txt
% An async using this sleight can implant a memory
% of up to an hour’s length inside the target’s mind.
% This memory very obviously does not belong to the
% target—there is no way they will confuse it for one
% of their own. The intent is not to fake memories, but
% to place one of the async’s memories in the target’s
% mind so that the target can access it just like any other
% memory. This can be useful for “archiving” important
% data with an ally, providing a literal alternate perspec-
% tive, or simply making a “data dump” for the target
% to peruse. Implant Memory requires an Opposed Test
% against unwilling participants. At the gamemaster’s
% discretion, particularly traumatic memories might
% inflict mental stress on the recipient (p. 215).

% IMPLANT SKILL

%  PSI TYPE:   Active       ACTION:     Complex

%  RANGE:      Touch        DURATION: Temp (Action Turns)



% STRAIN MOD: +0          SKILL:   Control
% Similar to Implant Memory, this sleight allows the
% async to impart some of their expertise and implant it
% into the target’s mind. For the duration of the sleight,
% the target benefits when using that skill. If the async’s
% skill is between 31 and 60, the target receives a +10
% bonus. If the async’s skill is 61+, the target receives
% a +20 bonus. Implant Skill requires an Opposed Test
% against unwilling participants. In some cases, the
% target has been known to use the skill with the async’s
% flair and mannerisms.

% MIMIC

%  PSI TYPE:   Active       ACTION:     Quick

%  RANGE:      Close        DURATION:   Instant

% STRAIN MOD: +0         SKILL:   Sense
% In a setting where changing your body and face is
% not unusual, people learn to recognize habits and
% personality quirks more often. The async must use
% this sleight on a target and succeed in a Success Test.
% If successful, the async acquires an “imprint” of the
% target’s mind that they can take advantage of when
% impersonating that ego. The async then receives a +30
% bonus on Impersonation Tests when mimicking the
% target’s behavior and social cues.

% MINDLINK

% PSI TYPE:     Active                 ACTION:     Quick

% RANGE:        Touch                  DURATION:   Sustained

% STRAIN MOD:                SKILL:


%             +1/target after first  Control
% Mindlink allows two-way mental communication
% with a target. This may be used on more than one
% target simultaneously, in which case the async can act
% as a telepathic “server,” so that everyone mindlinked
% with the async may also telepathically communicate
% with each other (via the async, however, so they
% overhear). Language is still a factor in mindlinked
% communications, but this barrier may be overcome by
% transmitting sounds, images, emotions, and other sen-
% sations. Mindlink requires an Opposed Test against
% unwilling participants.

% OMNI AWARENESS

% PSI TYPE:     Active                 ACTION:     Quick

% RANGE:        Close                  DURATION: Temp (Minutes)


%  STRAIN MOD: –1           SKILL:    Sense
% An async with Omni Awareness is hypersensitive to
% other biological life that is observing them. During
% this sleight’s duration, the async makes a Sense Test
% that is opposed by any life that has focused their at-
% tention on them within the sleight’s range; if success-
% ful, the async knows they are being watched, but not
% by whom or what. It does, however, apply a +30 Per-
% ception bonus to spot the observer. This sleight does
% not register partial attention or fleeting attention, or
% simple perception of the async, it only notices targets

% %%% txt/230.txt
% who are actively observing (even if they are concealing
% their observation). This sleight is effective in spotting a
% tail, as well as finding potential mates in a bar.

% PENETRATION

%  PSI TYPE:   Active        ACTION:     Quick

%  RANGE:      Touch         DURATION:   Instant


% STRAIN MOD: 1 per AP point SKILL:  Psi Assault
% Penetration is a sleight that works in conjunction with
% any offensive sleight that involves the Psi Assault skill.
% It allows the async to penetrate the Psi Shield of an op-
% ponent by concentrating their psi attack. Every point
% of Armor Penetration applied to a psi attack inflicts
% 1 point of strain. The maximum AP that may be ap-
% plied equals the async’s Psi Assault skill divided by 10
% (round down).

% PSI SHIELD

%  PSI TYPE:   Passive       ACTION:     Automatic

%  RANGE:      Self         DURATION: Constant
% Psi Shield bolsters the async’s mind to psi attack and
% manipulation. If the async is hit by a psi attack, they
% receive WIL ÷ 5 (round up) points of armor, reducing
% the amount of damage inflicted. They also receive a
% +10 modifier when resisting any other sleights.

% PSYCHIC STAB

%  PSI TYPE:   Active        ACTION:     Complex

%  RANGE:      Touch         DURATION:   Instant

%  STRAIN MOD: +0           SKILL:    Psi Assault
% Psychic Stab is an offensive sleight that seeks to inflict
% physical damage on the target’s brain and nervous
% system. Each successful attack inflicts 1d10 + (WIL ÷
% 10, round up) damage. Increase the damage by +5 if
% an Excellent Success is scored.

% SCRAMBLE

%  PSI TYPE:   Passive       ACTION:     Automatic

% RANGE:      Self       DURATION: Constant
% Scramble allows the async using the sleight to hide
% from another async using the Ego Sense or Omni
% Awareness sleights. Apply a +30 modifier to the de-
% fending async’s Opposed Test.

% SENSE BLOCK

%  PSI TYPE:   Active        ACTION:     Complex

%  RANGE:      Touch         DURATION: Temp (Action Turns)


%  STRAIN MOD: –1            SKILL:   Psi Assault
% Sense Block disables and short circuits one of the tar-
% get’s sensory cortices (chosen by the async), interfering
% with and possibly negating a specific source of sensory
% input for the chosen duration. If the async beats the
% target in the Opposed Test, the target suffers a –30
% modifier to Perception Tests with that sense equal (dou-
% bled to –60 if the async scores an Excellent Success).
% SPAM

%  PSI TYPE:   Active       ACTION:     Complex

%  RANGE:      Touch        DURATION: Temp (Action Turns)



% STRAIN MOD: +0            SKILL:    Psi Assault
% The sleight allows the async to overload and flood one
% of the target’s sensory cortices (chosen by the async),
%  spamming them with confusing and distracting sen-
%  sory input and thereby impairing them. If the async
% wins the Opposed Test, the target suffers a –10 modi-
% fier to all tests the duration of the sleight (doubled to
% –20 if the async scores an Excellent Success).

% STATIC

%  PSI TYPE:   Active       ACTION:     Complex

%  RANGE:      Close        DURATION:   Sustained

%  STRAIN MOD: +0           SKILL:    None
% The async generates an anti-psi jamming field, imped-
% ing any use of ranged sleights within their range. All
% such ranged sleights suffer a –30 modifier. This sleight
% has no effect on self or touch-range sleights.

% SUBLIMINAL

%  PSI TYPE:   Active       ACTION:     Complex

%  RANGE:      Touch        DURATION:   Instant

%  STRAIN MOD: +2           SKILL:    Control
% The Subliminal sleight allows the async to influence
% the train of thought of another person by implement-
% ing a single post-hypnotic suggestion into the mind
% of the target. If the async wins the Opposed Test, the
% recipient will carry out this suggestion as if it was
% their own idea. Implanted suggestions must be short
% and simple; as a rule of thumb, the gamemaster may
% only suggestions encompassed by a short sentence
% (for example: “open the airlock,” or “hand over the
% weapon”). At the gamemaster’s discretion, the target
% may receive a bonus for resisting suggestions that are
% immediately life threatening (“jump off the bridge”)
% or that violate their motivations or personal strictures.
% Suggestions do not need to be carried out immediately,
% they may be implanted with a short trigger condition
% (“when the alarm goes off, ignore it”).

% THOUGHT BROWSE

%  PSI TYPE:   Active       ACTION:     Complex

%  RANGE:      Touch        DURATION:   Sustained

%  STRAIN MOD: –1          SKILL:   Sense
% Thought Browse is a less-intrusive form of mind
% reading which scans the target’s surface thoughts for
% certain “keywords” like a particular word, phrase,
% sound, or image chosen by the async. Rather than dig-
% ging through the target’s ego as with the Deep Scan
% sleight, Thought Browse merely verifies whether a
% target has a particular person, place, event, or thing
% in mind, which can be used by a savvy investigator to
% draw conclusions without the need to invade the mind

% %%% txt/231.txt
%  directly. Thought Browse may be sustained, allowing
%  the async to continue scanning the target’s thoughts
%  over time. The async must beat the target in an Op-
%  posed Test for each scanned item.



%  PSYCHOSURGERY
%  Given the reach of neuroscience in the time of
%  Eclipse Phase, it is easy to think of the mind as
%  programmable software, as something that can be
%  reverse-engineered, re-coded, upgraded, and patched.
%  To a large degree, this is true. Aided by nanotechnol-
%  ogy, genetics, and cognitive science, neuroscientists
%  have demolished numerous barriers to understand-
%  ing the mind’s structure and functions, and even
%  made great leaps in unveiling the true nature of
%  consciousness. Genetic tweaks, neuro-mods, and
%  neural implants offer an assortment of options for
%  improving the brain’s capabilities. The transhuman
%  mind has become a playground—and a battlefield.
%  Nanovirii unleashed during the Fall infected mil-
%  lions, altering their brains in permanent ways, with
%  occasional outbreaks still occurring a decade later.
%  Cognitive virii roam the mesh, plaguing infomorphs
%  and AIs, reprogramming mind states. An “infectious
%  idea” is now a literal term.

%  In truth, mind editing is not an easy, safe, and
%  error-proof process—it is difficult, dangerous, and
%  often flawed. Neuroscience may be light years
%  ahead of where it was a century ago, but there are
%  many aspects of the brain and neural functions
%  that continue to confound and elude even the
%  brightest experts and AIs. Technologies like nano-
%  neural mapping, uploading, digital mind emulation,
%  and artifi cial intelligence are also comparatively
%  in their infancy, being mere decades old. Though
%  transhumanity has a handle on how to make these
%  processes work, it does not always fully understand
%  the underlying mechanisms.

%  Any neurotech will tell you that mucking around
%  in the mind’s muddy depths is a messy business.
%  Brains are organic devices, molded by millions of
%  years of unplanned evolutionary development. Each
%  is grown haphazardly, loaded with evolutionary



% SOLARCHIVE SEARCH: “THE HUMAN COGNOME PR


%            The Human Cognome Project was an                Fun


%            academic research venture to reverse         and co


%            engineer the human brain, paralleling        transhu


%            in many ways the Human Genome                the fun


%            Project and its success in deciphering       and wa


%            the human genome. The HCP was a              the fir


%            multidisciplinary undertaking, relevant      intellig


%            to biology, neuroscience, psychology,        has als


%            cognitive science, artificial intelligence,   ing tra


%        THE HUMAN COGNOME PROJECT


%            and philosophy of mind.                      databa

% leftovers, and randomly modified by an unlimited

% array of life events and environmental factors. Every

% mind features numerous mechanisms—cells, con-

% nections, receptors—that handle a dizzying array

% of functions: memory, perception, learning, reason-

% ing, emotion, instinct, consciousness, and more. Its

% system of organization and storage is holonomic,

% diffused, and disorganized. Even the genetically-

% modified and enhanced brains of transhumans are

% crowded, chaotic, cross-wired places, with each

% mind storing its memories, personality, and other

% defining features in unique ways.


% What this means is that though the general ar-

% chitecture and topography of neural networks can

% be scanned and deduced, the devil is in the details.

% Techniques used to modify, repair, or enhance one

% person’s mind are not guaranteed equal success

% when applied to another’s brain. For example, the

% process by which brains store knowledge, skills, and

% memories results in a strange chaining process where

% these memories are linked and associated with other

% memories, so attempts to alter one memory can

% have adverse affects on other memories. In the end,

% minds are slippery and dodgy things, and attempts

% to reshape them rarely go as planned.


% THE PROCESS OF PSYCHOSURGERY

% Psychosurgery is the process of selective, surgical altera-

% tion of a transhuman mind. It is a separate field from

% neural genetic modification (which alters genetic code),

% neuralware implantation (adding cybernetic or biotech

% inserts to the brain or nervous system), or brain hacking

% (software attacks on computer brains, neural inserts, and

% infomorphs), though they are sometimes combined.


%  Psychosurgery is almost always performed on a

% digital mind-state, whether that be a real-time emula-

% tion, a backup, or a fork. In most cases, the subject’s

% mind-state is copied via the same technology and pro-

% cess as uploading or forking, and run in a simulspace.

% The subject need not be willing, and in these cases

% the subject’s permissions are restricted. Numerous

% psychosurgery simulspace environments are available,

% each custom-designed for facilitating specific psy-

% chosurgical goals and programmed with a thorough

% selection of psychotherapy treatment options.


% ECT”
% and supported by scientific      the mind-states of healthy individuals for
% ate entrepreneurs and early     treating mental diseases and damage.
%  st groups, the HCP developed   Though most HCP data is available to the
%  entals of digitizing an ego    public, some argonauts claim that certain
% major driving force towards     data is held hostage by some hypercorps,
% anshumans with elevated         potentially for the development of pro-
%  and brain capacity. The HCP    prietary mind-altering technologies.
%  en instrumental in catalog-       After the Fall, the remnants of this
% man minds and developing        project were acquired by the Planetary
%  f “mind patches” based on      Consortium.

% %%% txt/232.txt

% The actual process of psychosurgery breaks down
% into several stages. First is diagnosis, which can
% involve the use of several neuro-imaging techniques
% on morphed characters, mapping synaptic connec-
% tions, and building a neurochemical model. It can
% also involve complete psychological profiling and
% psychometric behavioral testing, including personal-
% ity tests and simulspace scenario simulations. Digital
% mind-states can be compared to records of people
% with similar symptoms in order to identify related
% information clusters. This analysis is used to plan
% the procedure.

% The actual implementation of psychosurgical al-
% teration can involve several methods, depending on
% the desired results. Applying external modules to the
% mind-state is often the best approach, as it doesn’t
% meddle with complicated connections and new inputs
% are readily interpreted and assimilated. For treat-
% ments, mental health software patches compiled from
% databases of healthy minds are matched, customized,
% and applied. Specialized programs may be run to
% stimulate certain mental processes for therapeutic
% purposes. Before an alteration is even applied, it may
% first be performed on a fork of the subject and run at
% accelerated speeds to evaluate the outcome. Likewise,
% multiple treatment choices may be applied to time-ac-
% celerated forks this way, allowing the psychosurgeon
% to test which is likely to work best.

% Not all psychosurgery is performed for the subject’s
% benefit, of course. Psychosurgery can be used to inter-
% rogate or torture prisoners, erase memories, modify
% behavior, or inflict crippling impairments. It is also
% sometimes used for legal punishment purposes, in
% an attempt to impair criminal activity. Needless to
% say, such methods are often brute-forced rather than
% fine-tuned, ignoring safety parameters and sometimes
% resulting in detrimental side effects.
% PSYCHOSURGERY MECHANICS
% In game terms, psychosurgery is handled as a Task
% Action requiring an Opposed Test. The psychosurgeon
% rolls Psychosurgery skill against the target’s WIL x 3.
% Apply modifiers as appropriate from the Psychosur-
% gery Modifiers table.

% If the psychosurgeon succeeds and the subject fails,
% the psychosurgery is effective and permanent. The
% alteration becomes a permanent part of the subject’s
% ego, and will be copied when uploaded (and some-
% times when forking).

% If both sides succeed but the psychosurgeon rolls
% higher, the psychosurgery is effective but temporary. It
% lasts for 1 week per 10 points of MoS.

% If the subject rolls higher, or if the psychosurgeon
% fails their roll, the attempt does not work.

% The timeframe listed for psychosurgical procedures
% is according to the patient’s subjective point of view.
% Since most subjects are treated in a simulspace, time
% acceleration may drastically reduce the amount of
% real-time such a procedure requires (see Defying Na-
% ture’s Laws, pp. 240–241).

% MENTAL STRESS
% Psychosurgery is a modification to the transhuman
% mind, and sometimes to the actual person that resides
% in that mind. It is unsurprising then that psychosur-
% gery places stress on the subject’s mental state and
% sometimes even inflicts mental traumas.

% Each psychosurgery option lists a Stress Value
% (SV) that is infl icted on the subject regardless of
% the tests’ success or failure. If the psychosurgeon
% achieves an Excellent Success (MoS 30+), this stress
% is halved (round down). If the psychosurgeon rolls
% a Severe Failure (MoF 30+), the stress is doubled.
% Alternately, a Severe Failure could result in unin-
% tended side effects, such as affecting other behav-
% iors, emotions, or memories.

% %%% txt/233.txt

% If a critical success is rolled, no stress is applied at
% all. If a critical failure is rolled, however, an automatic
% trauma is applied in addition to the normal stress.

% Some psychosurgery conditions may also affect the
% SV, as noted on the Psychosurgery Modifiers table.

% ROLEPLAYING MIND EDITS
% Many of the changes incurred by psychosurgery are
% nebulous and difficult to pin down with game me-
% chanics. Alterations to a character’s personality and
% mind-state are often better handled as roleplaying fac-
% tors anyway. This means that players should make a
% real effort to integrate any such mental modifications
% into their character’s words and actions, and game-
% masters should ensure that a character’s portrayal
% plays true to their mind edits. Some psychosurgical
% mods can be reflected with ego traits, while others
% might incur modifiers to certain tests or in certain
% situations. The gamemaster should carefully weigh a
% brain alteration’s effects, and apply modifiers as they
% see appropriate.

% PSYCHOSURGERY PROCEDURES

% The following alterations may be accomplished
% with psychosurgery. At the gamemaster’s discretion,
% other mind-editing procedures may be attempted,
% using these as a guideline.

% BEHAVIORAL CONTROL

%  TIMEFRAME:   1 week


%             Limit/Boost –10; Block/Encourage –20,

%  PM:


%             Expunge/Enforce –30

%  SV:         (1d10 ÷ 2, round up)
% Commonly used for criminal rehabilitation, behavioral
% control attempts to limit, block, or expunge a specific
% behavior from the subject’s psyche. For example, a
% murderer may be conditioned against acts of aggres-
% sion, or a kleptomaniac might be restricted from steal-
% ing. Some people seek this adjustment willingly, such
% as socialite glitterati who restrict their desire to eat, or
% an addict who cuts out their craving for a fix.

% Behavioral control can also be applied as an un-
% leashing or reinforcement. A companion may desire
% to eliminate their sexual inhibitions, for example, or
% a hypercorp exec may boost his commitment to place
% work above all else.

% A character will simply feel compelled to avoid
% a behavior that is limited (perhaps suffering a –10
% modifier), but will find it quite difficult to pursue a be-
% havior that is blocked (requiring a WIL x 3 Test, and
% suffering a –20 modifier). They will find themselves
% completely incapable of initiating a behavior that is
% expunged, and if forced into the behavior will suffer
% a –30 modifier and (1d10 ÷ 2, round up) points of
% mental Stress.

% Likewise, a character will feel compelled to pursue a
% behavior that is boosted, and will find it hard to avoid
%  engaging in a behavior that is encouraged (requiring a
%  WIL x 3 Test to avoid). They will have no choice but
%  to engage in enforced behaviors, and will suffer (1d10
%  ÷ 2, round up) points of mental Stress if prevented
%  from doing so.

%  BEHAVIORAL MASKING


%  TIMEFRAME:   1 week


%  PM:          –20


%  SV:       1d10 ÷ 2, round up

%  Given the ability to switch bodies, many security
%  and law enforcement agencies have resorted to per-
%  sonality and behavioral profiling as a means of iden-
%  tifying people even when they resleeve. Though such
%  systems are far from perfect, someone’s unconscious
%  habits and quirks could potentially give them away.
%  Characters who wish to elude identification in this
%  way may undergo behavioral masking, which seeks
%  to alter and change the character’s unconscious habits
%  and social cues. Apply a +30 modifier when defending
%  against such identification systems and Kinesics Tests.

%  DEEP LEARNING


%  TIMEFRAME:   Skill Learning Time +2


%  PM:          +20


% SV:        1

%  Using tutorial programs, memory reinforcement
%  protocols, conditioning tasks, and deep brain stimula-
%  tion, the subject’s learning ability is reinforced, allow-
%  ing them to learn new skills more quickly.

%  EMOTIONAL CONTROL


%  TIMEFRAME:   1 week


%               Limit/Boost –10; Block/Encourage –20,


%  PM:


%               Expunge/Enforce –30


%  SV:      (1d10 ÷ 2, round up) + 2

%  Similar to behavioral control, emotional control
%  seeks to modify, enhance, or restrict the subject’s
%  emotional responses. Some choose these modifica-
%  tions willingly, such as limiting sadness in order to
%  be happier, or encouraging aggression in order to be
%  more competitive. Mercenaries and soldiers have been
%  known to expunge fear. Follow the same rules as given
%  for Behavioral Control.





%    PSYCHOSURGERY MODIFIERS


%                                  PSYCHOSURGERY
% SITUATION                              TEST MODIFIER       SV MO
% Improper Preparatory Diagnosis              –30               +
% Safety Protocols Ignored                    +20                x
% Simulspace Time Acceleration                –20               +
% Subject is an AI, AGI, or uplift            –20               +

% %%% txt/234.txt
% INTERROGATION

%  TIMEFRAME:   Variable (gamemaster discretion; 1 week default)

%  PM:          +30

%  SV:       1d10

% Psychosurgery can be used for interrogative pur-
% poses via the application of mental torture and ma-
% nipulation. A successful Psychosurgery Test applies a
% +30 modifier to the Intimidation Test for interrogation.

% MEMORY EDITING

%  TIMEFRAME:   1 week (2 weeks adding/replacing)

%  PM:          –10 (willing) or –30 (forced)

%  SV:       (1d10 ÷ 2, round up)

% By monitoring memory recall (forcibly invoked
% if necessary), psychosurgeons can identify where
% memories are stored in the brain and target them for
% removal. Memory storage is complex and diffused,
% however, and often linked to other memories, so re-
% moving one memory may affect others (gamemaster
% discretion).

% Adding or replacing memories is a much more com-
% plicated operation and requires that such memories be
% copied from someone who has experienced them or
% manufactured with XP software. Even when success-
% fully implanted, fake memories may clash with other
% (real) memories unless those are also erased.

% PERSONALITY EDITING

%  TIMEFRAME:   1 week

%  PM:          Minor –10; Moderate –20, Major –30

%  SV:        (1d10 ÷ 2, round up) + 3

% Possibly the most drastic psychosurgery procedure,
% personality editing involves altering the subject’s core
% personality traits. The personality factors that may be
% modified is almost unlimited, including traits such as
% openness, conscientiousness, altruism, extroversion/
% introversion, impulsiveness, curiosity, creativity, con-
% fidence, sexual orientation, and self-control, among
% others. These traits may be enhanced or reduced to
% varying degrees. The effect is largely reflected by role-
% playing, but the gamemaster may apply modifiers as
% they see fit.

% PSYCHOTORTURE

%  TIMEFRAME:   Variable

%  PM:          +30

%  SV:       1d10 SV per day

% Psychotorture is mental manipulation for the simple
% intention of causing pain and anguish, reflected in
% game terms as mental stress and traumas. Prolonged
% torture can lead to serious mental disorders or worse.
% PSYCHOTHERAPY

%  TIMEFRAME:   Variable

%  PM:          +0

%  SV:        0

% Therapeutic psychosurgery is beneficial for charac-
% ters suffering from mental stress, traumas, and disor-
% ders. A successful Psychosurgery Test applies a +30
% modifier to mental healing tests, as noted on p. 215.

% SKILL IMPRINTS

%  TIMEFRAME:   1 week per +10

%  PM:          +0

%  SV:        1 per +10

% Skill imprinting is the use of psychosurgery to insert
% skill-set neural patterns in the subject’s brain, tempo-
% rarily boosting their ability. Skill imprints are artificial
% boosts, however, degrading at the rate of –10 per day.
% No skill may be boosted higher than 60.

% SKILL SUPPRESSION

%  TIMEFRAME:   1 day per –10

%  PM:          –10

%  SV:       1 per +10

% Skill suppression attempts to identify where skills
% are stored in the brain and then block or remove
% them. The subject’s skill is impaired and may be lost
% entirely.

% TASPING

%  TIMEFRAME:   1 day

%  PM:          +10

%  SV:         1

% Tasping is the use of deep brain stimulation tech-
% niques to tickle the mind’s pleasure centers. Though
% this procedure is often used for therapeutic purposes
% for patients suffering from depression or other mental
% illnesses, the intent with tasping is to overload the
% subject into a prolonged state of almost unendurable
% bliss. Such stimulation is highly addictive, however, so
% character’s exposed to it for any length of time (over
% 1 hour, subjective) are likely to pick up the Addiction
% trait (p. 148). Some criminal organizations have been
% known to use tasping addiction and rewards as a
% means of controlling those under their thrall.

% %%% txt/235.txt
% THE LOST


%       <begin excerpt>


%       PSICLONE Project Quarterly Board Meeting            w


%       2nd Quarter 8 AF                                    b


%       FUTURA Project Conclusion—                          si


%       Executive Summary Report                            Th


%       Prepared by Dr. Amelia Sheppard                     te


%                                                           ca


%       Per request, I have compiled a review of the        tr


%       Futura Project and its fallout, 5 years after       a


%       whistleblowers and intense media attention          ti


%       forced us to end the project and release the        e


%       remaining subjects (dubbed “the Lost” by            2


%       popular media).                                     4


%           Futura was a joint initiative spearheaded by    d


%       Hanto Genomics and strongly backed by Cog-          ch


%       nite, with numerous other partners (complete        3


%       list). The project was initially proposed by my


%       mentor, Dr. Antonio Pascal, whose team had          re


%       proven the feasibility of Accelerated Life Ex-      ca


%       perience Training (ALET) after a series of pilot    so


%       studies with two small (N<1000) samples.            a


%       While it is true that these early pilot studies     e


%       used both older subjects and a lesser amount        fa


%       of time dilation, the rationale for the Futura      th


%       Project’s ambitious program was justified by a       su


%       remarked decrease in transhumanity’s popula-        a


%       tion due to the Fall, a system-wide stagnant        o


%       population growth rate (blamed on various           m


%       factors including increased longevity, available    n


%       contraception, and rising despair over troubling    Pa


%       times), as well as a desire to move aggressively    ti


%       into a new technological sector in the hopes of     b


%       obtaining a competitive advantage.                  e


%           Futura began immediately in the wake            m


%       of the Fall with an initial seed population


%       of       test subjects culled from extant genetic   th


%       material and gestated to between 1 week and         o


%       6 months after birth. Of these, less than 10%       m


%       were live births from either a surrogate or         to


%       genetic birth mother who had perished during        ti


%       the Fall. The majority came from our Lunar and      w


%       Martian labs and were brought to term within        b


%       an exowomb.                                         b
% er the sample was selected, all subjects     Post-project analysis now show

% eeved into our fast-growth futura-brand    of our subjects had engaged

% ph bodies and inducted into customized     act of premeditated murder b
%  pace accelerated learning environments.     mark (12 years of age) and
%  oject made extensive use of emergent        protocols were only training t

% logies and techniques culled from re-      more effectively.
%  ed TITAN facilities, including neogenetic       It was at this point that
%  or the futura morphs and time distortion    Aaron Bharani advocated pull
%  ations for captive simulspace popula-       the project and bringing the
%  Futura ran concurrently on three differ-    real time and intensive couns

% earch stations with a combined staff of    vetoed our concerns without e
%  researchers and support personnel and       to the board. As the project s

% s custom-programmed for expert child       its conclusion, a fork of Dr. Bha
%  pment. Project goals were to raise each     at the 34 month mark, incitin
% o a subjective 18 years life experience in   controversy. While Dr. Pascal s
% s objective time.                            up investigators, hoping to s

% pite omnipresent observation and           through to its conclusion, the
% me adjusting of the simulspace and edu-      Legacy research station occur
%  al programming for optimal normality,       ings concluded that one or mor
% where along the way the project suffered     had escaped the program an
%  kdown in quality assurance and param-       responsible for the habitat’s en

% onitoring that resulted in a near total    ures and the thousands of subs

% at empathy modeling. We first observed          In the face of intense pub

% ect 11 months into the project when the    scrutiny, many of the partne
%  ts had aged to approximately 6 years of     the project attempted to pul
% ncidences of animal cruelty and acting       eliminate all traces of their inv

% d spiked, though at that time they re-     resulting chaos, an estimated
%  d within acceptable standards. Over the     were quietly released into the s
% ew months this trend continued and Dr.       population. It was only after th

% authorized the usage of more authorita-    all known subjects were iden

% arenting” to attempt to correct for the    been infected with the Watts-
%  line sociopathic behavior that was being    of the Exsurgent virus, though

% ed by 23.19% of all subjects by the 18-    this occurred remains troubli

% mark (9 years of age).                     Though later orders resulted i

% now know that these changes had           subjects being euthanized an

% intended consequence of suppressing        into cold storage, only
%  displays of cruelty and violence and        subjects were recaptured. Of th

% taught the majority of subjects how       sued sanctuary with sympath
%  ceal their psychoses. It was also at this            went public and submi
% hat the first deaths occurred. The initial    to extensive psychotherapy,

% were thought to be accidents and          incidents of violence and not r
%  he victim and perpetrator were usually      the rest presumably went into
% d up to a week or so of subjective time.         <end excerpt>

% %%% txt/236.txt
% MESH
% INSERTS, ENDOS, ECTOS
% he mesh, you need some sort of device. Often, these are
% right into your brain—a mesh insert, sometimes called
% —or an ecto, an external minicomputer. ■ p. 237



%                                        INTERFACES


%                                        There are three protoco


%                                        and manipulate data o






%                                                            Vir






%                   MESH USES


%                   Quite simply, your character will use the me


%                   communications, research, and plenty of rec






%         Infolife and Infomorphs


%           Both disembodied transhumans and


% artificial intelligences roam the mesh. ■ p. 244



%                Muses: A specific type of AI that functions as a


%                 personal aide and companion. ■ pp. 244, 264


%  AGIs: Self-aware digital consciousnesses, capable of intelligen


%                  and autonomous actions.■ pp. 244, 264–265

% nfomorphs: Details and rules for digital egos. ■ pp. 244, 264–265
% sed to access
% e mesh. ■ p. 239

% gmented Reality (AR): Information overlaid on the user’s physical
% enses—sight, sound, odors, tastes, and tactile sensations. ■ p. 239

%  Reality (VR): Physical senses are overridden and your character is
% ed in a computer-generated environment—a simulspace. ■ p. 240

% erience Playback (XP): A sensory recording, either of your own or


%                            provided by someone else. ■ p. 241




% r almost all their
% on, too. ■ p. 243




%       Everyday Mesh Mechanics: The basic rules for


%                  standard mesh use start here. ■ p. 245



%            Online Research: Your character will use the


%        Research skill to find info in the mesh. ■ p. 249

% %%% txt/237.txt


%            MESH ABUSES
% The mesh is rife with miscreants, opportu-
%  nistic hackers, and people who plain just


%       don’t agree with you. ■ p. 243






%                   Hackers & Malware




%              Mesh Security: Four prim


%        authentication, firewalls, active m



%               Intrusion: Infiltrating a s


%                through the security qui




%            Hacking Sequence: Quick-a



%       Intrusion Countermeasures: If y


%         to your presence, intrusion cou


%                                   attack



%                 Cyberbrain Hacking: P


%                          with cybernet


%                               9
% o primary threats —people, and


%    people’s tools. ■ p. 243

% ethods of security on the mesh:
% toring, and encryption. ■ p. 253

% m can happen two ways: slicing
% or launching brute-force attacks
%  knock the door down. ■ p. 254

% dirty hacking summary. ■ p. 255

% hacking efforts alert the system
% measures will attempt to locate,
% otherwise thwart you! ■ p. 257

%  and synthmorphs are equipped
% ains—and like other computers,

% these can be hacked. ■ p. 261

% %%% txt/238.txt
%  ESH■THE MESH■THE MESH■THE MESH



% MESH

% Before the Fall, humanity interfaced with each other

% through the internet, interconnected networks that

% served as the technical backbone for the evolv-

% ing world wide web. While it began as a electronic

% medium for retrieving information from various

% sources (replacing even older paper-based info-

% sources), succeeding generations emphasized digital

% communities and hosted services such as networking

% sites, wikis, blogs, and folksonomies. These facili-

% tated openness, collaboration, and sharing, thereby

% laying the groundwork for a modern, interconnected

% information society. Further stages emphasized wire-

% less interaction, geolocation, and semantic web ap-

% proaches and achieved quantum leaps in the realm

% of user interaction with the advent of brain-computer

% interfaces, augmented reality (AR), virtual reality

% (VR), and experience playback (XP).


%  This environment, coupled with the exponential

% growth of processing power and memory storage, cre-

% ated an evolutionary path for the development of intel-

% ligent agents—designed to augment human informa-

% tion processing—that then transformed into artificial

% intelligences (AIs) in the following decades. While these

% “weak” AIs did not possess the full range of human

% cognitive abilities, tended towards overspecialization,

% and were restrained by programmed limitations, the

% digital evolution toward artificial general intelligences

% (AGIs)—”strong” AIs with intelligence capabilities

% that equaled or exceeded human abilities—could not

% be halted. From this point it was but a matter of time

% before so-called seed AI would come into existence,

% machine minds capable of recursive self-improvement,

% leading to an exponential growth in intelligence. Un-

% fortunately for humanity, the TITANs were the result.


%  Even before the Fall, however, the internet of old

% was transforming into something new. Instead of con-

% necting via central servers, users were wirelessly link-

% ing to each other, creating a decentralized intermeshed

% network of handheld devices, personal computers,

% robots, and electronic devices. Users were online all of

% the time and connected with everything and everyone

% around them in a ubiquitous computing environment.

% This was especially true of those participating in

% humanity’s expansion into space. Disconnected from

% the internet due to distance and light-speed commu-

% nication lags, these users were nevertheless connected

% with all of the people and objects in their nearby

% environment or habitat, creating local wireless mesh

% networks. Thus was the mesh born, taking the place

% of the old internet of earth, lost during the Fall.




% MESH CAPABILITIES

% The mesh, as it exists in Eclipse Phase, is only possible

% thanks to major developments made in computer and

% communication technologies and nanofabrication.
% HE MESH■THE MESH■THE MESH■THE


%             THE MESH■THE


%                 MESH■



%                                                         6


%  Wireless radio transmitters and receivers are so unob-
%  trusively tiny that they can literally be factored into
%  anything. As a result, everything is computerized and
%  connected, or at least tagged with a radio frequency
%  ID (RFID) chip. Even food is tagged with edible chips,
%  complete with expiration date and nutritional content.
%  Other communications mediums, such as laser and
%  microwave links, add to the information flow.

%  Data storage technology has advanced to such high
%  levels that even an individual user’s surplus storage
%  capacity can maintain an amount of information
%  easily surpassing the entire 20th-century internet.
%  Lifeloggers can literally record every moment of their
%  life and never fear about running out of room. The
%  amount of data that people carry around in the mesh
%  inserts in their head or in portable ecto personal com-
%  puters is staggering.

%  Processing capabilities also exist at hyper-efficient
%  levels. Even massive supercomputers are a thing of the
%  past when modest handheld devices can fulfill almost
%  all of your needs, even while simultaneously running a
%  personal AI assistant, downloading media, uploading
%  porn, and scanning thousands of newsfeeds. Within
%  the mesh network, devices that near their processing
%  limits simply share the burden with devices around
%  them, creating a massively distributed framework that
%  in some ways is like an entire supercomputer to itself,
%  shared by everyone.

%  Similarly, transmission capacity now far exceeds
%  most citizens’ definition of need. Anyone born within
%  the last several generations has always lived in a world
%  in which hyper-realistic, multi-sensory media of nearly
%  any length is available for instantaneous download
%  or upload from anywhere. Massive databases and
%  archives are copied back and forth with ease. Band-
%  width is such a non-issue that most people forget it ever
%  was. In fact, given the sheer amount of data available,
%  finding the information or media you’re looking for
%  takes considerably longer than downloading it. The
%  mesh is also never down. As a decentralized network,
%  if any one device is taken offline, connections merely
%  route around it, finding a path via the thousands if not
%  millions of available nodes. Similarly, the entire mesh
%  behaves like a peer-to-peer network, so that large trans-
%  fers are broken into manageable chunks that take inde-
%  pendent routes. In fact, most users maintain personal
%  torrent archives that are publicly accessible and shared.

%  Private networks still exist, of course. Some are
%  physically walled away behind closed-access wired net-
%  works or even wireless-inhibiting infrastructure that
%  keep a network isolated and contained. Most, however,
%  operate on top of the public mesh, using encrypted
%  tunneling protocols that provide private and secure
%  communications over unsecured networks. In other
%  words, these private networks are part of the mesh
%  along with everything else, but only the participants

% %%% txt/239.txt
% SOLARCHIVE SEARCH: ECTO(-LINK)


%            A very popular brand of mobile mul-       tools, a


%            tifunctional personal digital assistant   izing


%            before the Fall, the ecto name became     update


%            a synonym for handheld personal           tion s


%            computers in the Mesh Age. Standard       softwa


%            implanted computers are also some-        OS de


%            times referred to as endos to reflect     allowi


%            the difference between an external        whate


%            and an internal device.                   gadge


%               No matter if ecto or endo, modern      user’s


%            computers are governed by an oper-        facilita


%         ECTO(-LINK)


%            ating system (OS), a multifunctional         The


%            suite of programs that includes media     of 20t




%  can interact with them thanks to encryption, user au-
%  thentication, and message integrity checking.

% With the factionalization of transhumanity, at-
%  tempts to unify software into standard formats have
%  still failed. However, different operating systems
%  or protocols are rarely an obstacle anymore due to
%  easily accessible conversion tools and AI-aided com-
%  patibility oversight.

%  MESHING TECHNOLOGIES
%  Almost all biomorphs in the solar system are equipped
%  with basic mesh inserts (p. 300)—implanted personal
%  computers. These implants are grown in the brain via
%  non-intrusive nanosurgery. The processor, wireless
%  transceiver, storage devices, and other components
%  are directly wired to the user’s cerebral neuronal cells
%  and cortical centers responsible for language, speech,
%  and visual perception among others. Thought-to-
%  communication emulations (so called transducing)
%  enables the user to control the implant just by think-
%  ing and to communicate without vocalizing. Input
%  from the mesh inserts is transmitted directly into the
%  brain and sometimes perceived as augmented reality,





% LEGACY OF THE TITANs

% Given the technical capabilities of modern perso

% broadband are not needed. But there is another r


% Mainframes, hive-mind clusters, and massively p

% are all considered potential dangers in Eclipse Ph

% capacity to enable a seed AI and another potent

% outlaw such systems completely under the sever

% backups and recent forks, in most cases.


% Those supercomputers that habitats do allow a

% systems like orbit maintenance thrusters, life sup

% R&D projects. These systems are typically physicall

% data processing centers with strong access restrict


% Similarly, AIs themselves are quite often heavil

% banned, especially in the inner system and Jovia

% programmed growth restraints, specifically design
% sh browser, locator, social-     be molded and shaped into different
% rams (messenger, socnet          forms due to smart material construc-
% cartography and naviga-          tion. They are often worn as jewelry or
% are, language translation        clothing accessories, particularly brace-
%  nd similar software tools.      lets. The user interface varies according

% are highly customizable,       to user preference. Wireless-enabled
%  lug-and-use add-ons for         contacts and earbuds equip users who
% additional software and          lack mesh implants, enabling them to
% re desired. Typically, the       experience augmented reality and the
%  e (personal AI assistant)       ecto’s AR control interface. Standard
% oftware interactions.            entoptic control interfaces are also

% itself is typically the size   available via wireless radio, skinlink,
% ntury credit card and can        and direct fiberoptic line.





% overlaid on the user’s physical senses. In a similar

% vein, the mesh inserts installed in synthmorphs and

% pods are directly integrated with their cyberbrains

% (creating a potential security concern as cyberbrains

% are vulnerable to hacking).


%  External devices called ectos (p. 325) are also

% used to access the mesh, though these are growing

% increasingly rare given the prevalence of mesh inserts.

% Ecto interface options include haptic interfaces like

% touch-display controls, bracelets or gloves that detect

% arm, hand, or finger movements (virtual mouse and

% keyboards), eye tracking and blink control, body

% scanning grids (body axis control or all-limb controls

% for non-humanoids), voice controls, and more. Sen-

% sory information is handled via lenses, glasses, ear-

% plugs (subdermal bone-vibrating speakers), bodysuits,

% gloves, nose plugs, tongue dams, and other devices

% that are wirelessly linked to (or physically plugged

% into) the ecto.


%  INFORMATION OVERLOAD

% The mesh contains massive amounts of personal and

% public information shared by users, a digital commons




%  computers, supercomputers and cutting-edge wired
% on they are avoided: the TITANs.
% allel distributed computing parallel hive-mind systems

% as they possess sufficient processing power and data
% hard takeoff singularity. Some habitats go so far as to
% of penalties: final death including the deletion of all

% “hard networks” that control a habitat’s most crucial
% t, communications, power, or cutting-edge hypercorp
%  red, heavily monitored, and locked down in electronic
% s and ruthless real-world security measures.
%  stricted, and it is not unusual for AGIs to be outright
% Republic. Most intelligent programs are limited with
%  to prevent them from becoming self-upgrading.        ■

% %%% txt/240.txt
% of news, media, discourse, knowledge, environmental
% data, business, and culture. Transhumans embrace
% the mesh as a tool for exchange, communication,
% and participation with other users, both local and far
% away. As such, the mesh is an up-to-date, authoritative
% source on all transhuman knowledge and activities.

% Not everything online is available for free, of course,
% except perhaps in the autonomist zones. Quite a bit of
% proprietary data is kept off the grid in secure storage
% or sequestered away in private networks. Some of this
% is for sale, and heavily encumbered with digital restric-
% tions—software, media, nanofabrication blueprints,
% skillsofts, etc. A thriving open source movement offers
% free and open source alternatives to much proprietary
% data, however, and numerous digital piracy groups
% deal out cracked versions of proprietary material,
% despite pressure from some authorities. Other data is
% simply secured from competitive interests (hypercorp
% research projects) or is an extremely private affair,
% such as ego backups.

% SPIMES
% Along with the accumulated data of transhuman
% affairs, the mesh is also cluttered with information
% derived from untold numbers of wireless-capable sen-
% sor-enabled devices that continuously update the mesh
% with their location, sensor recordings, and other data.
% Colloquially called “spimes,” these location-aware,
% environment-aware, self-logging, self-documenting
% objects broadcast their data to anyone who cares to
% listen. Since visual, auditory, and other sensors are
% absurdly tiny and inexpensive, they are ubiquitously
% incorporated into nearly every object or product a
% person might wear, apply, use, or internalize. This
% allows almost any user to reach out through the mesh
% and gather environmental data and ambient sensor
% recordings from a specific location (or at least public
% locales—private areas typically block such signals or
% slave them to a local AI that filters their output).

% SURVEILLANCE, PRIVACY, AND SOUSVEILLANCE
% While spimes are easily trackable, they also con-
% tribute to an environment of universal surveillance.
% Between spimes, microsensors, ubiquitous security
% systems, and the recording capabilities of the mesh
% inserts used by almost everyone, just about every-
% thing is recorded. Factor in the availability of mesh-
% tracking, facial recognition, rep/social networks, and
% other data mining software, and it rapidly becomes
% clear that privacy is an outdated concept. Special
% considerations must be embraced by anyone that
% seeks to mask their identity or cover their movements.
% Alternately, off-the-shelf looks common with some
% morphs (especially synthmorphs and pods) allow a
% user to blend in with the masses.

% Though in part this may seem an Orwellian sur-
% veillance nightmare, the abundance of recording
% tech actually works as “sousveillance” (watching
% from below), serving a role in making everything
% transparent and putting checks on abuses of power.

% %%% txt/241.txt
% Authoritarian regimes tread carefully, as they are also
% universally monitored, despite their attempts to con-
% trol information flow. Many people also willing par-
% ticipate in this open “participatory panopticon.” With
% nearly infinite storage capacity, dedicated lifeloggers
% record every moment of their lives and share it for
% others to experience.



% INTERFACING: AR, VR, AND XP
% Mesh media is accessed using one of three protocols:
% augmented reality (AR), virtual reality (VR), or expe-
% rience playback (XP).

% AUGMENTED REALITY
%  Most users perceive data from the mesh as aug-
%  mented reality—information overlaid on the user’s
%  physical senses. For example, computer-generated
%  graphics will appear as translucent images, icons,
%  or text in the user’s field of vision. While visual AR
%  data—called entoptic data—is the most common,
%  other senses may also be used. AR is not limited
%  to visuals, however, and can also include acoustic
%  sounds and voices, odors, tastes, and even tactile
%  sensations. This sensory data is high-resolution and
%  seemingly “real,” though it is usually presented as
%  something ghostly or otherwise artificial so as not
%  to be confused with real-world interactions (and
%  also to meet safety regulations).

% User interfaces are customized to the user’s prefer-
%  ences and needs, both graphically and content-wise.
%  Filters allow users to access the information they
%  are interested in without needing to worry about
%  extraneous data. While AR data is typically placed
%  in the user’s normal field of vision, entoptics are not
%  actually limited by this and may be viewed in the
% “mind’s eye.” Nevertheless, icons, windows and other
%  interaction prompts can be layered, stacked, toggled,
%  hidden, or shifted out of the way if necessary to in-
%  teract with the physical world.

% AVATARS
% Every mesh represents themselves online via a digi-
% tal avatar. Many people use digital representations
% of themselves, whereas other prefer more iconic
% designs. This may be an off-the-shelf look or a
% customized icon. Libraries of avatars may also be
% employed, enabling a user to switch their represen-
% tation according to mood. Avatars are what other
% users see when they deal with you online—i.e., how
% you are represented in AR. Most avatars are ani-
% mated and programmed to reflect the user’s actual
% mood and speech, so that the avatar seems to speak
% and have emotions.

% E-TAGS
% Entoptic tags are a way for people to “tag” a physical
% person, place, or object with a piece of virtual data.
% These e-tags are stored in networks local to the tagged
% item, and move with the item if it changes location.
% INFORMATION AT
% YOUR FINGERTIPS
% The following information is always available for m
% mesh users in a normal habitat:

% LOCAL CONDITIONS
% • Local maps showing your current location, annotated with

% local features of personal interest (according to your perso

% preferences and filters) and your distance from them/direct

% to them. Details regarding private and restricted areas (gov

% ment/hypercorp areas, maintenance/security infrastructure,

% are usually not included.
% • Current habitat life support (climate) conditions including

% atmosphere composition, temperature.
% • Current solar system and habitat orbit maps with trajectory

% plots, communication delays.
% • Local businesses/services, directions, and details.

% LOCAL MESH
% • Public search engines, databases, mesh sites, blogs, forums

% archives, along with new content alerts.
% • Syndicated public newsfeeds in a variety of formats, filtered

% according to your preferences.
% • Sensor/spime (mostly audio-visual) feeds from any public a

% of the habitat.
% • Private network resources (including tactical nets).
% • Automatic searches for new online references to your nam

% and other subjects of interest.
% • E-tags pertaining to local people, places, or things.
% • Facial/image recognition searches of public mesh/archives

% match a photo/vid still.

% PERSONAL INFORMATION
% • Morph status indicators (medical and/or mechanical): blood

% pressure, heart rate, temperature, white cell count, nutrient

% levels, implant status and functionality, etc.
% • Location, functionality, sensor feeds, and status reports of y

% possessions (via sensors and transmitters in these possessi
% • Access to one’s life-spanning personal audio-visual/XP arch
% • Access to one’s life-spanning personal file archive (music, s

% ware, media, documents, etc.).
% • Credit account status and transactions.

% SOCIAL NETWORKS
% • Communications account status: calls, messages, files, etc.
% • Reputation score and feedback.
% • Social network status, friend updates.
% • Updated event calendar and alerts.
% • The public social network profiles of those around you.
% • The location and status of those nearby and involved in the

% same AR games as you.

% %%% txt/242.txt
%  E-tags are viewable in AR, and can hold almost any
%  type of data, though short notes and pictures are the
%  most common. E-tags are often linked to particular
%  social networks or circles within that network, so that
%  people can leave notes, reviews, memorabilia media,
%  and similar things for friends and colleagues.

%  SKINNING
%  Since reality can be overlaid with entoptics of hyper-
%  real quality, modern users can “skin” their reality by
%  modifying their perceptual input. Environments around
%  them may be modified to fit their particular tastes or
%  mood. Need your spirits boosted? Pull up a skin that
%  makes it seems like your outdoors, with the sun shining
%  down, the sounds of gentle surf in the background, and
%  butterflies drifting lazily overhead. Pissed off? Be com-
%  forted as flames engulf the walls and thunder grumbles
%  ominously in the distance. It is not uncommon for
%  people to go about their day, accompanied by their own
%  personal soundtrack that only they can hear. Even ol-
%  factory and taste receptors can be artificially stimulated
%  to experience sensations like the smell of roses, fresh air,
%  or freshly-baked pastries. While originally developed
%  to make “space food” less distasteful and as a method
%  to counter space-induced cabin fever for those that
%  weren’t born in space, vast archives of aromas, tastes,
%  and environments are available for download.

%  Skins do not need to be kept private, they may
%  also be shared with others via the mesh. Tired of
%  your cramped habitat cubicle? Decorate it with a




% JABBER


%   # Start Æther Jabber #


%   # Active Members: 2 #



%       1


%         I have to tell you, after losing Kiri and Sal to that Exsur


%         infection, my team is a lot more worried about contracting


%         virus from digital sources. Actually, I’d label them as paran


%         I don’t think they’ll ever touch any salvaged electronics a


%         unless they’re behind a zillion firewalls and the device is c


%         pletely isolated and tested by a delta fork loaded with e


%         antiviral ware we can find first. Even then, they’d rather sho


%         than access directly or hook it up to an important network. A


%         seeing what the virus did to Sal, I don’t blame them.


%   2


%         In our line of work, paranoia can be healthy.


%       1


%         Sure, but it’s also a pain in the ass. Security is always a tr


%         off. Firewall’s gotta have something up its sleeve that I can


%         along to the rest to put their guards at ease.


%   2


%         Yes ... and no. It’s complicated.


%       1


%         I don’t see why. Do we have a way of detecting and killing


%         thing or not?


%   2


%         Sort of.


%       1


%         You’re killing me.


%   2


%         Look. Ever since the Fall, we’ve had measures in plac


%         detect and counteract Exsurgent infections and all of the o
% custom-themed skin and share it with visitors to make
% them feel more comfortable. Found a new music track
% that livens up your day? Share it with others around
% you, so they can nod to the same beat.

% Skinning can also be used for the opposite effect.
% Any undesired content of reality can be edited out,
% veiled, or censored by modern software programs or
% muses that engage in real-time editing. Tired of look-
% ing at someone’s face? Add them to your killfile, and
% you’ll never have to acknowledge their presence again.
% AR censorware is also common in some communities
% with strict religious or moral convictions.

% VIRTUAL REALITY
% Virtual reality overrides the user’s physical senses and
% places them inside an entirely computer-generated
% environment called a simulspace. While AR is used
% for all common day activities and interactions, VR is
% used mainly for recreation (gaming, virtual tourism,
% escapism), socializing, meeting (when face-to-face
% meetings are not possible), and training. Dedicated
% networks with high-capacity information processing
% are required to render and run large and complex
% hyper-real simulspaces with many users, and these
% are often hard-wired for additional stability. Smaller
% simulspaces capable of hosting a smaller amount of
% users can be run on a smaller distributed network of
% linked devices. Many infomorphs and AIs effectively
% reside within simulspaces, and some transhumans
% have sworn off the physical world altogether.






%  worms and malware the TITANs concocted. Firewall went to


%  great lengths to make sure that everyone had access to the


%  detection signatures and countermeasures—and we mean ev-


%  eryone. They’ve been incorporated in almost every commercial


%  and open source security software released in the past decade.


%  Every habitat in the system—well, every one with a lick of


%  sense anyway—employs such measures in their chokepoints


%  and mesh infrastructure.

%  1


%  I sense a “but.”
% 2


%  Yes. The problem is that the Exsurgent virus and similar TITAN


%  infowar worms are adaptive. They’re intelligent. Even though


%  we mostly eradicated them from our networks, new versions


%  periodically pop up, using some new trick to get past the


%  Firewall scans and wreak havoc. Our warning and outbreak


%  response system has it down to a science, and such instances


%  are usually contained.

%  1


%  Usually.
% 2


%  Well, there’s always the chance that variants are still skipping


%  around out there, under our radar. What’s worse to contemplate,


%  though, is that we may get another major outbreak that spreads


%  to multiple habitats before we can contain it. That might get


%  very, very bad, very, very quickly.

% %%% txt/243.txt
% DEFYING NATURE’S LAWS
% A plethora of simulspace environments are available,
% ranging from simulations of real places to historical
% recreations to fantastic worlds representing almost
% every genre imaginable. All of these simulations are
% bolstered by the fact that possible scenarios are not
% bound by the laws of nature. The fundamental forces
% of reality and nature, like gravitation, electromag-
% netism, atmosphere, temperature, etc., are program-
% mable in VR, allowing for environments that are
% completely unnatural, such as escheresque simulspace
% where gravity is relative to position. These domain
% rules may be altered and manipulated according to
% the whim of the designer.

% Time itself is an adjustable constant in VR, though
% deviation from true time has its limits. So far, tran-
% shuman designers have achieved time dilation up to
% 60 times faster or slower than real time (roughly one
% minute equaling either one hour or one second). Time
% slowdown is far more commonly used, granting more
% time for simulspace recreational activities (more time,
% more fun!), learning, or work (economically effective).
% Time acceleration, on the other hand, is extremely
% useful for making long distance travel through space
% more tolerable.

% ACCESSING SIMULSPACES
% Most simulspaces can be accessed through the mesh
% just like any other node. Since VR takes over the
% user’s sensorium, however, and sometimes involves
% time perception dilation, users are cut off from other
% mesh-delivered sensory input and interacting directly
% with other nodes. Instead, outside mesh interactions
% are routed through the simulspace’s interface (mean-
% ing that a character may browse the mesh, communi-
% cate with others, etc. from inside a simulspace, if the
% domain rules allow it).

% Since physical senses are overridden when a user
% accesses VR, most people prefer to rest their body
% in a safe and comfortable environment while in the
% simulspace. Body-fitting cushions and couches help
% users relax and keep them from cramping up or in-
% juring themselves if they happen to thrash around.
% In case of long-term virtual sojourns (for instance,
% during space travel), morphs are normally retained
% in tanks that sustain them in terms of nutrition and
% oxygen. Many VR entertainment and game networks
% offer dedicated and hardwired physical VR cafes
% with private pods. Visitors rent a pod and physically
% jack in, using either access jacks or an ultrasonic
% trode net that reads and transmits brain patterns
% when placed on the head.

% When accessing a simulspace, the user first enters
% an electronic buffer “holding space” known as a
% white room. Here the user chooses a customizable
% avatar-like persona to represent them in the simul-
% space, called a simulmorph. From this point, the user
% immerses themself in the virtual reality environment,
% effectively becoming their simulmorph.
% EXPERIENCE PLAYBACK
% Every morph with mesh inserts has the capability
% to transmit or record their experiences, a form of
% technology called experience playback, or XP. Since
% the first programs were developed that provide a
% simple interface to “snapshot” ones experiences, it
% has become extremely popular to share XP with
% friends and social networks, or with the online public
% at large.

% The level of experiences depends on how much of
% the recorded sensory perception is kept when the clip
% is made. Full XP includes exteroceptive, interoceptive,
% and emotive tracks. Exteroceptive tracks include the
% traditional senses of sight, smell, hearing, touch, and
% taste that process the outside world. Interoceptive
% tracks include senses originating within the body, such
% as balance, a sense of motion, pain, hunger and thirst,
% and a general sense of the location of one’s own body
% parts. Emotive tracks include the whole spectrum of
% emotions which can be aroused in a transhumans.
% Due to the biological requirements (neuronal and
% endocrine systems) of expressing emotions, hardcore
% XP aficionados deem only the experience in and from
% biomorphs as the real deal.



% MESH USES
% There are many reasons people use the mesh. The
% foremost is communication: voice and video calls
% (typically displaying avatars rather than actual
% video), electronic messaging (e-mail, instant mes-
% saging, microblogging), and file and data transfers.
% Socializing is also key, handled via social and reputa-
% tion networks, personal profiles, lifelogging, chats
% and conferences (both AR and VR), and discussion
% groups and forums. Information gathering is also
% at the top, whether its browsing the popular Solar-
% chive or other databases and directories, tapping
% the latest newsfeeds, browsing mesh sites, tracking
% your friends, taking lessons in VR, or looking up just
% about anything conceivable. Recreation rounds out
% the pack, covering everything from gaming (AR and
% VR) to experiencing other people’s lives (XP) to VR
% tourism and club-hopping.

% PERSONAL AREA NETWORKS
% Since everything a person carries is meshed, most
% people maintain personal area networks that route
% all of these devices through their mesh inserts or ecto,
% which acts as a hub. This is both a security measure,
% ensuring they maintain control over their own acces-
% sories, and a convenience factor, as it focuses all of the
% controls in one place.

% VIRTUAL PRIVATE NETWORKS
% Virtual private networks (VPNs) are communica-
% tions networks tunneled through the mesh, which
% are dedicated for a specific group of people. The
% primary use of VPNs is to create privacy and secu-
% rity for its users, and so they typically use security

% %%% txt/244.txt
% features such as ego authentication and public key
% encryption. VPNs are regularly used to mesh mobile
% offices into a corporate network or mesh people to-
% gether who work on or contribute a certain project.
% Other VPNs—particularly social networks and rep
% networks—operate with minimal security features,
% simply serving as a network of specific users within
% the mesh and making it easier to keep in touch,
% transfer information, make updates, and so on. Most
% VPNs come as specialized software suites that run
% custom environmental software that integrates into
% the user’s normal mesh interface and AR.

% SOCIAL NETWORKS
% Social networks are the fabric of the mesh, weaving
% people together. They are the means by which most
% people keep in contact with their friends, colleagues,
% and allies, as well as current events, the latest trends,
% new memes, and other developments in shared
% interests. They are an exceptionally useful tool for
% online research, getting favors, and meeting new
% people. In some cases, they are useful for reaching
% or mobilizing masses of people (as often illustrated
% by anarchists and pranksters). There are thousands
% of social networks, each serving different cultural
% and professional interests and niches. Most social
% networks allow users to feature a public profile to
% the entire mesh and a private profile that only those
% close to them can access.

% Reputation plays a vital part in social networks,
% serving as a measure of each person’s social capital.
% Each person’s reputation score is available for lookup,
% along with any commentary posted by people who
% favored or disfavored them and rebuttals by the user.
% Many people automate their reputation interactions,
% instructing their muse to automatically ping some-
% one with a good review after a positive action and
% to likewise provide negative feedback to people with
% whom the interaction went poorly.

% MOBILE OFFICES
% Due to the lack of office space and the wireless ac-
% cessibility of most information, most businesses
% now operate virtually, with few or no fixed offices
% or even assets. Instead, individuals have become their
% own mobile office. Bit-pushers and bureaucrats like
% hypercorp executives, clerical workers, accountants,
% and researchers—as well as innovators like artists,
% writers, engineers, and designers—work wherever
% they want to.

% The most prominent example of this phenomenon
% are the bankers of the Solaris hypercorp. Each em-
% ployee acts as a mobile one-person banking office,
% managing transactions via Solaris’s robust VPN.

% On rare occasions, office environments are run
% in simulspace with time dilation to maximize effi-
% ciency. Since this requires the workers to access a
% centralized wired network and leave their bodies
% unattended while accessing simulspace, however, it
% requires an extra level of physical security that is
% typical only of some governmental installations and
% corporate habitats.



% ISLANDS IN THE NET
% In the time of Eclipse Phase, information can become
% outdated quite fast, and the accessibility of new infor-
% mation depends on your location. It’s easy to keep up-
% to-date on your local habitat/city or planetary body,
% but keeping current on events elsewhere is typically
% reliant on the speed of light.

% If you happen to be in a station in the Kuiper Belt,
% on the edge of the solar system 50 astronomical units
% from the terrestrial inner planets, waiting on a mes-
% sage from Mars, the signal carrying the message will
% be roughly seven hours old when it reaches you. Of
% course it will only reach you that fast if you are using
% quantum farcast, which is only limited by the speed
% of light (not to mention rare and expensive in most
% habitats). If you are not using a quantum farcaster,
% the signal may take even longer and is prone to in-
% terference and noise, deteriorating the quality and
% possibly losing some of the content, especially over
% major distances. Whenever you start dealing with
% communication between habitats, you have to factor
% in the light-speed lag, the amount of time it takes
% even the fastest transmission to reach you. This lag
% works both ways, so trying to hold a conversation
% with someone just 5 light-seconds away means that
% you’re waiting at least 10 seconds to get the reply
% to whatever you just said. For this reason, AR and
% VR communications are almost always conducted lo-
% cally, while standard messaging is used for nonlocal
% communications. For detailed discussions, it is often
% simpler to send a fork of yourself (p. 273) to have the
% conversation and then return.

% Quantum-entanglement communicators (p. 314)
% are one solution to this light-speed lag, although a
% burdensome and expensive one. QE comms allow
% for faster-than-light communication to an entangled
% communicator, though each transmission uses up a
% precious amount of quantum-entangled bits, which
% are in limited supply.

% Transmissions made between habitats almost
% always occur via each station’s massive data relays,
% where they are then distributed into the local mesh.
% This bottleneck is often used by authoritarian habitats
% to monitor data transmissions and even filter or censor
% certain public non-encrypted content. Some messages
% are also prioritized over others, potentially meaning
% further delays.

% %%% txt/245.txt

% The method of transmission between habitats also
% sometimes matters. Radio and neutrino broadcasts
% can be intercepted by anyone, whereas tight-beam
% laser or microwave links are specifically used as a
% point-to-point method that minimizes interception
% and eavesdropping. The use of quantum farcasting
% using neutrino systems is completely secure, however,
% and is the most frequently-used intra-habitat link.

% What these lags, bottlenecks, and prioritizations
% mean is that some news and data takes a particularly
% long-time to trickle from one local mesh network to
% another, passing slowly from habitat to habitat. This
% means that there are always gradients of information
% available to different local mesh networks, typically
% depending on proximity and the importance of the
% information. Some data even gets lost along the way,
% never making it further than a habitat or two before
% it is lost in the noise. The only way to retrieve such
% information is to track it down to its source.

% DARKCASTS
% “Darkcasts” are ranged communications that go out-
%  side of legal and approved channels. Since certain hab-
%  itats have strict regulations on transmission content,
%  forking, egocasting, infomorphs, muse abilities, and
% AGI code, underworld groups like the ID crew profit
%  by offering illegal data transmission services. Primarily
%  used for censored data and banned content (like illegal
% XPs or malware), local organized crime factions also
%  often offer egocasting services complete with resleev-
%  ing and leasable morphs, allowing egos that prefer
%  discretion to enter or leave a habitat without drawing
%  attention. Though such authorities hunt down these
%  darkcast networks whenever they get a chance, many
%  habitats have a sophisticated darkcast infrastructure
%  that makes use of decoys, temporary communications
%  lines, relays, and regular transmitter relocation—not
%  to mention judicious bribing and blackmailing.
% MESH ABUSES
% As with all things, the mesh has its darker side. At
% the basic level, this amounts to flamewar-starting
% trolls, stalkers, or griefers whose intent is to mess
% with others for a laugh. At the more organized level,
% it expands to illicit or criminal enterprises that utilize
% the mesh, such as selling black/snuff/porn XPs, ille-
% gal software, pirated media, or even egos. The most
% infamous threats—thanks both to the Fall and to
% the continuous sensationalism applied by media and
% stern authorities—are, of course, malware and hack-
% ers. Given the capabilities of modern hackers and the
% vulnerability of many habitats—where damage to life
% support systems could kill thousands—the threat may
% not be over-exaggerated.

% HACKERS
% Whether individuals who are genuinely interested
% in exploring new technologies and seeking ways to
% break them in order to make them better, hacktiv-
% ists who utilize the mesh in order to undermine the
% power of authorities, or “black hats” who seek to
% circumvent network security for malicious or crimi-
% nal intent, hackers are a permanent fixture of the
% mesh. Unauthorized network break-ins, infiltration
% of VPNs, muse subversion, cyberbrain hijacking, data
% theft, cyber-extortion, identity fraud, denial of service
% attacks, electronic warfare, spime hijacking, entoptic
% vandalism—these are all common occurrences on the
% mesh. Thanks to smart and adaptive exploit programs
% and assisting muses, even a moderately skilled hacker
% can be a threat.

% In order to counter hacking attempts, most people,
% devices, and networks are protected by a mix of
% access control routines, automated software intru-
% sion prevention systems, encryption, and layered
% firewalls, typically overseen by the user’s muse who
% plays the role of active defender. Extremely sensitive

% %%% txt/246.txt
% systems—such as space traffic control, life support,
% power systems, and hypercorporate research facili-
% ties—are usually limited to isolated, tightly-controlled,
% heavily-monitored, hard-wired networks to minimize
% the risk of intrusion from snoopers and saboteurs.
% various countermeasures may be applied against
% such intruders, ranging from locking them out of the
% system to tracking them back and counterhacking.

% MALWARE
% The number of worms, virii, and other malware pro-
% grams that ripped through computer systems during
% the Fall was staggering. Many of these were part of
% the netwar systems prepared by old Earth nation-states
% and corporations and unleashed on their enemies.
% Others were products of the TITANs, subversive pro-
% grams that even the best defenses had trouble stopping.
% Even 10 years later, many of these are still reappearing,
% brought back to life by the accessing of some long-
% forgotten data cache or the accidental infection of a
% scavenger mucking through old ruins. New ones pop
% every day, of course, many of them programmed by
% criminal hacker outfits, while others that enter circu-
% lation are modifications and variations of suspected
% TITAN designs, perhaps implying that certain parties
% are intentionally tinkering with this code and releasing
% it into the wild. Rumors and whispers circulate that
% some of these TITAN worms are even more potent and
% frightening than previously hinted at, with stunning
% adaptive capabilities and intelligence. These rumors
% are steadfastly denied by authority figures and security
% experts ... who then quietly turn around and do their
% best to ensure that their own networks remain safe.



% AI S AND INFOLIFE
% Self-aware helper programs were originally designed
% and realized to augment transhuman cognitive abili-
% ties. These specialized-focus AIs were then developed
% into the more complete, independent digital conscious-
% nesses known as AGIs. The further evolution of these
% digital life forms into seed AIs unfortunately led to the
% emergence of the TITANs and then the Fall. This cre-
% ated a rift in transhuman society as fear and prejudice
% turned popular opinion against unrestricted AGIs, an
% attitude of mistrust that still lingers to this day.

% AI S
% The term AI is used to refer to narrow, limited-focus
% AIs. These digital minds are expert programs with pro-
% cessing capabilities equal to or even exceeding that of
% a transhuman mind. Though they have a personality
% matrix with individual identities and character, and
% though they are (usually) conscious and self-aware,
% their overall complexity and capabilities are limited.
% The programmed skills and abilities of AIs are typi-
% cally very specific in scope and aligned towards a par-
% ticular function, such as piloting a vehicle, facilitating
% mesh searches, or coordinating the functions of some
% habitat sub-system. Some AIs, in fact, can only barely
% be considered sapient, and their emotional program-
% ming is usually narrow or non-existent.

% AIs have a number of built-in safety features and
% programmed limitations. They must serve and obey the
% instructions of authorized users within their normal
% functioning parameters and (in the inner system at
% least) must also obey the law. They lack self-interest
% and self-initiative, though they have limited empathy
% and may be programmed to anticipate the needs and
% desires of users and pre-emptively take action on
% their behalf. Perhaps most importantly, however, is
% that their psychological programming is specifically
% based on universal human modes of thought and an
% understanding and support of transhuman goals
% and interests. This is part of an initiative to engineer
% so-called “friendly AIs,” who are programmed with
% sympathy towards transhumanity and all life and seek
% out their best interests.

% In most societies, basic AIs are considered “things”
% or property rather than people and are accorded no
% special rights.

% MUSES
% Muses are a specific type of AI designed to function
% as a personal aide and companion. Most people in
% Eclipse Phase have grown up with a muse at their
% virtual side. Muses tend to have a bit more personality
% and psychological programming than standard AIs and
% over time they build up an extensive database of their
% user’s preferences, likes and dislikes, and personality
% quirks so that they may more effectively be of service
% and anticipate needs. Muses generally have names
% and reside within the character’s mesh inserts or ecto,
% where they can manage the character’s personal area
% network, communications, data queries, and so on.

% AGIS
% AGIs are complete and fully operational digital con-
% sciousnesses, self-aware and capable of intelligent
% action at the same level as any transhuman. Most have
% full autonomy and the capacity for self-improvement
% by a processing similar to learning—a slow optimi-
% zation and expansion of their code that features
% programmed limitations to prevent it from achiev-
% ing the self-upgrading capabilities of seed AIs. They
% have more fully-rounded personalities and emotional/
% empathic abilities than standard AIs, due in part to
% a development process where they are literally raised
% within a VR simulation analogous to the rearing of
% transhuman children, and so are more fully socialized.
% As a result, they have a fairly human outlook, though
% some deviation is to be expected, and sometimes is
% apparent in great degrees. Despite this attempt to hu-
% manize AGIs, they do not have the same evolutionary
% and biological origins that transhumans have, and so
% their social responses, behavior, and goals are some-
% times off-mark or decidedly different.

% AGIs bear the social stigma of their non-biological
% origin and are often met with bias and mistrust. Some
% habitats have even outlawed AGIs or subject them to

% %%% txt/247.txt
% WHAT YOUR MUSE
% CAN DO FOR YOU
% • Make Research Tests to find information for you.
% • Scan newsfeeds and mesh updates for keyword alerts.
% • Monitor your mesh inserts/ecto/PAN and slaved devices

% for intrusion.
% • Launch countermeasures against intruders.
% • Teleoperate and command robots.
% • Monitor your Rep score and alert you to drastic changes.
% • Automatically provide feedback for other people’s

% Rep scores.
% • Run audio input through an online, real-time language

% translation system.
% • Put you in privacy mode and/or proactively stealth your

% wireless signal.
% • Falsify/fluctuate your mesh ID.
% • Track people for you.
% • Anticipate your needs and act accordingly, pre-empting

% your requests.                                         ■




% strict restrictions, forcing such infolife to hide their

% true natures or illegally darkcast to enter habitats or

% stations. AGI mind programming emulates transhu-

% man brain patterns sufficiently well that they can be

% sleeved into biomorphs if they choose.


% SEED AI

% Due to the capability for unlimited self-upgrading,

% seed AIs have the capacity to grow into god-like

% digital entities far beyond the level of transhumans

% or AGIs. They require massive processing power and

% are always increasing in complexity due to a continual

% metamorphosis of their code. Seed AIs are too com-

% plex to be downloaded into a physical morph, even

% a synthetic one. Even their forks require impressive

% processing environments, so doing so is rare. In fact,

% most seed AIs require the capacities of hard-wired

% networks to survive.

%  The only seed AIs known to the public are the

% infamous TITANs who are widely regarded as being

% responsible for the Fall. In truth, the TITANs were

% not the first seed AIs and will probably not be the last.

% There are no publicly known TITANs (or other seed

% AIs) currently residing in the solar system, despite

% circulating rumors of damaged TITANs who were left

% behind on Earth, speculated TITAN activity under the

% clouds of Venus, or whispers of new seed AIs hidden

% away in secret networks on the edges of the system.


% TRANSHUMAN INFOMORPHS

% For thousands of infugees, embodying a digital form

% is their only choice. Some of these are locked away

% in mesh-separated virtual holding areas or even inac-

% tive storage, locked up by habitats who didn’t have

% NON-STANDARD AIS

% AND AGIS

% Not all AIs and AGIs were programmed and de

% signed to adhere to human modes of thought

% and interests. Such creations are illegal in some

% jurisdictions, as they are considered a potential

% threat. Several hypercorps and other groups have

% initiated research into this field, however, with

% varying results. In some cases these digital minds

% are so different from human mindsets that com

% munication is is impossible. In others, enough

% crossover exists to allow limited communication

% but such entities are invariably quite strange.


%  Rumors persist that some AIs began their

% life as transhuman backups or forks, who were

% then heavily edited and pruned down to AI

% level intelligences.                            ■



% enough resources to handle them. Others are impris-
% oned inside simulspaces, killing time in whatever way
% they choose until an opportunity to resleeve comes
% their way. Quite a few are free to roam the mesh, in-
% teracting with physically-sleeved transhumans, keep-
% ing up with current events, and sometimes even form-
% ing activist political blocs to campaign for infomorph
% rights or interests. Still others find or are forced into
% virtual careers, slaving away in the digital sweatshops
% of hypercorps or criminal syndicates. A few find com-
% panions who are willing to bring them along in their
% ghostrider module and become an integral part of
% their lives, much like a muse.

% Some transhumans willingly choose the infomorph
% lifestyle, either for hedonism (custom simulspace and
% VR games until the end of time), escapism (loss of a
% loved ones leads them to write off physical concerns
% for awhile), freedom (go anywhere the mesh takes
% you—some have even beamed copies of themselves to
% far distant solar systems, hoping someone or something
% will receive their signal when they arrive), experimen-
% tation (forking and merging, running simulations, and
% weirder things), or because it is ensured immortality.



% EVERYDAY MESH MECHANICS
% Everyone (and everything) is meshed in Eclipse Phase.
% The following rules and concerns apply to standard
% mesh use. Note that various mesh-related terms are
% explained, along with other Eclipse Phase concepts,
% under Terminology, p. 25.

% MESH INTERFACE
% Characters have a choice of which interface to use, the
% entoptic interface of basic mesh inserts or the haptic
% interface of an ecto.

% The basic mesh inserts used by most users allows
% them to interact with AR, VR, XP, and the mesh at the

% %%% txt/248.txt
% speed of thought. This is the default method of mesh
% use and suffers no modifiers. They are, however, more
% prone to visual and operative impairments (virtual il-
% lusions, denial-of-service effects) when hacked.

% Characters who use the haptic interface of an ecto,
% however, suffer a slight delay on their mesh activities
% due to manual toggling, physical controls, and physi-
% cal interaction with virtual controls. In game terms,
% the use of haptics imposes a –10 skill modifier to
% all mesh tests where timing is rushed (particularly
% combat and or any sort of mesh use under pressure).
% Additionally, increase the timeframe for mesh-based
% Task Actions by +25% when interfacing via haptics.
% On the positive side, ectos can be easily removed and
% discarded if compromised—for this reason, many
% hackers and security-conscious users use an ecto in
% addition to their mesh inserts, routing all high-risk
% traffic through the ecto as an extra line of defense.

% MESH ID
% Every mesh user (and, in fact, every device) has a
% unique code called their mesh ID. This ID distin-
% guishes them from all other users and devices, and is
% the mechanism by which others can find them online,
% like a combination phone number, email address, and
% screen name. Mesh IDs are used in almost all online in-
% teractions, which are often logged, meaning that your
% activities online leave a datatrail that can be tracked
% (p. 251). Fortunately for Firewall sentinels and others
% who value their privacy, their are ways around this
% (see Privacy and Anonymity, p. 252). AIs, AGIs, and
% infomorphs also each have their own unique mesh ID.

% ACCOUNTS AND ACCESS PRIVILEGES
% Devices, networks (such as PANs, VPNs, and hard-
% wired networks), and services require that every user
% that accesses them does so with an account. The
% account serves to identify that particular user, is
% linked to their mesh ID, and determines what access
% privileges they have on that system. There are four
% types of accounts: public, user, security, and admin.

% PUBLIC ACCOUNTS
% Public accounts are used for systems that allow
% access (or access to parts of their system) to anyone





% ELITE EXPLOITS

% The mesh gear quality rules allow for players

% and gamemasters to make a distinction between

% software tools, separating the open-source,

% stock-repertoire exploit tools of amateur hackers

% from cutting-edge military-grade penetration

% wares. While many characters will simply buy or

% otherwise acquire such programs, a hacker with

% the do-it-yourself ethic is likely going to want

% to design their own personalized applications,

% based on their playbook of closely-guarded in-

% trusion/counterintrusion methods.


% To reflect the efforts a hacker character makes

% by designing, coding, and modifying their own

% customized personal arsenals, they may make a

% Task Action Programming Test with timeframe

% of 2 weeks. If they succeed, they upgrade one

% of their software tools by one level of quality

% (i.e., from +0 to +10). Multiple Programming

% Tests can be made to enhance a program, but for

% each level add the target modifier as a negative

% modifier to the test (so upgrading a +0 suite to

% +10 is a –10 modifier on the Programming Test).


% Similarly, at the gamemaster’s discretion, soft-

% ware tools—particularly exploits—may degrade

% in quality over time, reflecting that they have

% become outdated. As a general rule, such pro-

% grams should degrade in quality about once

% every 3 months.                                 ■

% %%% txt/249.txt
% on the mesh. Public accounts do not require any sort
% of authentication or login process, the user’s mesh ID
% is enough. These accounts are used to provide access
% to any sort of data that is considered public: mesh
% sites, forums, public archives, open databases, social
% network profiles, etc. Public accounts usually have
% the ability to read and download data, and some-
% times to write data (forum comments, for example),
% but little else.

% USER ACCOUNTS
% User accounts are the most common accounts. User
% accounts require some form of authentication (p. 253)
% to access the device, network, or service. Each user ac-
% count has specific access privileges allotted to it, which
% are tasks the user is allowed to perform on that system.
% For example, most users are allowed to upload and
% download data, change basic content, and use the
% standard features of the system in question. They are
% not, however, usually allowed to alter security features,
% add new accounts, or do anything that might impact
% the security or functioning of the system. As some sys-
% tems are more restrictive than others, the gamemaster
% decides what privileges each user account provides.

% SECURITY ACCOUNTS
% Security accounts are intended for users that need
% greater rights and privileges than standard users, but
% who don’t need control over the entire system, such
% as security hackers and muses. Security access rights
% usually allow for reading logs, commanding security
% features, adding/deleting accounts, altering the data of
% other users, and so on.

% ADMIN ACCOUNTS
% Admin accounts provide complete control over the
% system. Characters with admin rights can do every-
% thing security accounts can, plus they can shut down/
% reboot the system, alter access rights of other users,
% view and edit all log files and statistics, and stop or
% start any software available on the system.

% MESH GEAR QUALITY
% Not all gear is created equal, and this is especially true
% of computers and software, where new innovations
% are made on a daily basis. Keeping up-to-date with
% the last specs isn’t too difficult, but on occasion the




%                                      MESH GEA
% MODIFIER SOFTWARE/HARDWARE

% –30      Bashed-up devices, no-longer-supported software, re

% –20      Malfunctioning/inferior devices, buggy software, pre-

% –10      Outdated and low quality systems

%  0       Standard ectos, mesh inserts, and software

% +10      High-quality goods, standard security-grade products

% +20      Next-generation devices, advanced software

% +30      Newly-developed, state-of-the-art, top-of-the-line te

% >+30      TITANs and/or alien technology

% characters will get their hands on some old relic or

% find themselves in secluded or decrepit places with

% local systems and gear that aren’t up to date. Likewise,

% they may acquire some cutting-edge gear straight

% from the labs or may run up against a state-of-the-art

% installation with next-generation defenses. To reflect

% this, mesh tests can be modified according to the state

% of the hardware or software being used, as noted on

% the Mesh Gear Modifiers table

% .

% COMPUTER CAPABILITIES

% Computerized electronics can be broken down into

% three simple categories: peripherals, personal comput-

% ers, and servers. In game terms, all are collectively

% referred to as devices.


% PERIPHERALS

% Peripherals are micro-computerized devices that don’t

% need the full processing power and storage capacity of

% a personal computer, but benefit from online network-

% ing and other computerized functions. Peripherals may

% run software, but the gamemaster may decide that too

% many programs (10+) will degrade the system’s perfor-

% mance. AIs and infomorphs are incapable of running

% on peripherals, though they may access them. Peripher-

% als only have public and user accounts (users accounts

% include security and admin functions; see p. 247).


% Common peripherals include: spimes, appliances, most

% cybernetic implants, individual sensors, and weapons.


% PERSONAL COMPUTERS

% Personal computers account for a wide range of com-

% puter types, but essentially account for anything that

% has the capabilities evolved from generations and gen-

% erations of personal computers to meet an everyday

% user’s needs. Most personal computers are portable

% and tailored for use by multiple users at a time. Per-

% sonal computers may run one AI or infomorph at a

% time. They may not run simulspace programs.


% Common personal computers include: mesh inserts,

% ectos, and vehicles.


% SERVERS

% Servers have much greater processing power and

% data management capabilities than personal comput-

% ers. They are capable of handling hundreds of users,

% multiple AIs and infomorphs, and they may run


% MODIFIERS

% rom Earth or the early expansion into space
% echnology




% logy

% %%% txt/250.txt
% simulspace programs. Though few are portable, some
% of them come close.

% SOFTWARE
% A wide manner of software is available for mesh users,
% from firewalls and AIs to hacking and encryption tools
% to tacnets and skillsofts. These programs are listed on
% p. 331 of the Gear chapter. Like other gear, software
% may enable a character to perform a task they could
% not otherwise. The quality of the software may also be
% a factor, applying a modifier as appropriate (see Mesh
% Gear Quality, p. 247).

% Some software is equipped with digital restrictions
% to prevent it from being copied and shared with others.
% These restrictions may be defeated, but it is a time-con-
% suming task, requiring a Task Action Programming Test
% with a timeframe of 2 months. Thanks to the efforts
% of the open source movement and numerous individual
% software pirates, however, quite a bit of software is
% available free or online. The availability of pirated soft-
% ware or freeware will depend on the local habitat and
% legalities. Finding it may be a matter of a simple search,
% or it may require some use of reputation to find some-
% one who has it. Usually there is at least one local crime
% syndicate that is willing to help you out—for a price.

% SOFTWARE COMPATIBILITY
% In most instances, software compatibility is not going
% to be an issue for characters. Gamemasters who like
% it as a plot device, however, can introduce compat-
% ibility problems in certain cases, whether this is done
% to increase drama, slow the characters down, or create
% obstacles that they must overcome. Such incompat-
% ibilities are more likely to arise when dealing with
% outdated systems or devices, or at least ones that are
% unlikely to have the latest patches and software up-
% dates. Incompatibilities can also be used as a drawback
% to acquiring software from untrustworthy sources.

% Conflicting software issues are going to have one of
% two effects. Either the software will simply not work
% with certain devices, or it will inflict a –10 to –30
% modifier due to instabilities and glitches. If the game-
% master allows it, a character may reduce this penalty
% by patching up the software, requiring a successful
% Programming Task Action (1 day). For every 10 points
% of MoS, reduce the incompatibility modifier by 10.

% TRAFFIC FILTERS AND MIST
% Mesh networks and AR are overrun with yottabytes of
% information. Though mesh inserts and ectos can deal
% with a lot of data traffic in terms of bandwidth and
% processing power, using filters to weed out unwanted
% traffic is quite simply a necessity. This is especially true
% of AR, where unwanted entoptics can clutter your
% vision and distract you. Nevertheless, entoptic spam of
% many flavors—advertisements, political screeds, porn,
% scams—do their best to bypass these filters, and in
% many areas the amount of unfiltered entoptics can be
% overwhelming—a phenomenon colloquially referred
% to as “mist.”

% At the gamemaster’s discretion, mist can interfere
% with a user’s sensory perceptions. This modifier can
% range from –10 to –30, and in some cases might be
% so distracting as to affect all of a character’s actions.
% To lift the data fog, a character or muse must adjust
% their filter settings by succeeding in an Interfacing Test
% modified by the mist modifier. Alternately, the charac-
% ter can completely disable AR input, but this is likely
% to impede them in other ways.

% SLAVING DEVICES

% For ease of use, as well as for privacy and security
% purposes, one or more devices may be slaved to each
% other. One device (usually the character’s mesh inserts
% or ecto) takes the role of master, while the other
% device(s) takes the role of slave. All traffic to and
% from slaved devices is routed through the master. This
% allows a slaved device to rely on the master’s security
% features and active monitoring. Anyone that wants to
% connect to or hacked into a slaved device is rerouted
% to the master for authentication and security scrutiny.
% Slaved devices automatically accept commands from
% their master device. This means that a hacker who
% penetrates a master system can legitimately access
% and issue commands to a slaved device, assuming their
% access privileges allow for it.

% PANs are typically formed by slaving all of a char-
% acter’s devices to their ecto or mesh inserts. Similarly,
% individual components of a security system (doors,
% sensors, etc.) are usually slaved to a central security
% node that serves as a chokepoint for anyone hoping
% to hack the system. The same is often true for other
% networks and facilities.

% ISSUING COMMANDS
% Characters may issue commands to any slaved device
% or teleoperated bot (see Shell Remote Control, p. 196)
% with a Quick Action. Each command counts separately,
% unless the character is issuing the same command to
% multiple devices/drones.

% DISTANCE LAG
% Anytime you extend your communications over great
% distances, you run into the risk of time lags. Most
% communications are restricted to “local” for this
% reason, which generally means your local habitat
% and any others within 50,000 kilometers. On plan-
% etary bodies like Mars, Venus, Luna, or Titan, “local”
% encompasses all of the habitats and linked mesh net-
% works on that planetary body.

% If a character is searching the mesh beyond their
% local area, the most efficient way is to transmit a search
% AI (usually a copy of your muse) or a fork to the non-
% local area, which will then run its search and return
% completed results. This process does, however, add to
% the time of transmission to the timeframe (i.e., search-
% ing the mesh of a station 10 light-minutes away adds
% 20 minutes to the search as the search is transmitted
% over and the results are transmitted back). Since long-
% distance communications are sometimes interfered with

% %%% txt/251.txt
% or bumped for higher-priority items, the gamemaster
% can increase this time at their discretion. If the research
% involves correlation and fine-tuning the search param-
% eters based on data accumulated from different locals,
% the timeframe may be exponentially increased due to
% the need for back-and-forth interaction.

% If the character is simply communicating with or
% accessing non-local networks, an appropriate time lag
% must be introduced between communications and ac-
% tions. The effects of this lag are largely up to the game-
% master, as fitting current distances and other factors.

% ACCESSING MULTIPLE DEVICES
% Meshed characters may connect to and interact with
% numerous devices, networks, and services simulta-
% neously. There is no penalty for doing this, but the
% character may only focus on one system at a time. In
% other words, you may only interact with one system
% at a time, though you may also switch between them
% freely, even within the same Action Phase. You could,
% for example, spend several Quick Actions to send a
% message with your ecto, tell your spime oven at home
% to start cooking dinner, and look up a friend’s updated
% profile on a social network. You may not, however,
% hack into two separate systems simultaneously.

% Note that you may send the same command to
% multiple slaved devices or teleoperated drones with
% the same Quick Action, as noted above.



% ONLINE RESEARCH
% The Research skill (p. 184) represents a character’s
% ability to track down information in the mesh. Such
% information includes any type of digitized data: text,
% pictures, vids, XP, sensor feeds, raw data, software, etc.
% This data is culled from all manner of sources: blogs,
% archives, databases, directories, social networks, rep
% networks, online services, forums, chat rooms, torrent
% caches, and regular mesh sites of all kinds. Research
% is conducted using various public and private search
% engines, both general and specialized, as well as data
% indices and search AIs.

% Research has other uses as well. Hackers use it
% when looking for specific information on a network or
% device on which they have intruded. Likewise, since
% everyone inevitably uses and interacts with the mesh,
% Research skill is also a way to identify, backtrack, and/
% or gather information on people as long as they have
% not hidden their identity, worked anonymously, or
% covered their identity with a shroud of disinformation.

% SEARCH CHALLENGES
% Due to the sheer amount of data available, find-
% ing what you’re looking for may sometimes seem a
% daunting task. Thankfully, information is fairly well
% organized, thanks to the hard work of “spider” AIs
% that cruise the mesh and constantly update data and
% search indices. Additionally, information on the mesh
% is tagged with semantics, meaning that it’s presented
% with code that allows a machine to understand the
% context of that information as well as a human reader
% would. This helps AIs and search software correlate
% data more efficiently. So finding the data is usually not
% as difficult as analyzing it and understanding it. Find-
% ing specialized or hidden info or correlating data from
% multiple sources is usually the real challenge.

% Perhaps a larger issue is the amount of incorrect data
% and misinformation online. Some data is simply wrong
% (mistakes happen) or outdated, but the nature of the
% mesh means that such things can linger on for years
% and even propagate far and wide as they are circulated
% without fact-checking. Likewise, given the amount of




% SEARCH CAPABILITIES

% Online research in Eclipse Phase is not the same

% as just googling something. Here are five ways in

% which it is vastly improved:


% Pattern Recognition: Biometrics and other forms

% of pattern recognition are efficient and intelligent.

% It is not only possible to run image recognition

% searches (in real-time, via all available spimes

% and sensor feeds), but to search for patterns such

% as gait, sounds, colors, emotive displays, traffic,

% crowd movement, etc. Kinesics and behavioral

% analysis even allow sensor searches for people

% exhibiting certain behavioral patterns, such as sus-

% picious loitering, nervousness, or agitation.


% Metadata: Information and files online come

% with hidden data about their creation, alteration,

% and access. A photo’s metadata, for example,

% will note what gear it was taken on, who took

% it, when, and where, as well as who accessed it

% online, though such metadata may be easily

% scrubbed or anonymized.


% Data Mash-Ups: The combination of abundant

% computing, archived data, and ubiquitous public

% sensors enable intriguing correlations to be

% drawn from data that is mined and collated. In

% the midst of a habitat emergency such as a ter-

% rorist bombing, for example, the ID of everyone

% in the vicinity could be scanned, compared to

% data archives to separate out those who have a

% history of being in the vicinity at that particular

% time, with those remaining checked against

% databases of criminal/suspect history and their

% recorded actions analyzed for unusual behavior.


% Translation: Real-time translation of audio and

% video is available from open source translation bots.


% Forecasting: A significant percentage of what

% people do on any day or in response to certain situ-

% ations conforms to routines, enabling easy behav-

% ioral prediction. Muses take advantage of this to

% anticipate needs and provide whatever is desired

% at the right moment and in the right context. The

% same logic applies to actions by groups of people,

% such as economics and social discourse, making pre-

% dictions markets a big deal in the inner system. ■

% %%% txt/252.txt
% transparency in modern society, some entities actively
% engage in disinformation spreading in order to clut-
% ter the mesh with enough falsehoods that the truth is
% hidden. Two factors help to combat this, the first being
% that data sources themselves have their own reputation
% scores, so that untrustworthy or disreputable sources
% can be identified and ranked lower in search results.
% Second, many archives take advantage of crowdsourc-
% ing—that is, harnessing the collaborative power of
% mesh users (and their muses) everywhere—to verify
% data integrity so that these sites are dynamic and
% self-correcting.

% HANDLING SEARCHES
% Online research is often a crucial element of Eclipse
% Phase scenarios, as characters take to the mesh to
% research backgrounds and uncover clues. The follow-
% ing suggestions present a method of handling research
% that does not rely solely on dice rolls and integrates it
% with the flow of the plot.

% First, common and inconsequential information
% should be immediately available without requiring
% a roll at all. Most characters rely on their muses to
% handle searches for them, passing on the results while
% the character focuses on other things.

% For searches that are more detailed, difficult, or
% central to the plot, a Research Test should be called
% for (made either by the character or their muse).
% This test indicates the process of finding links to
% and/or accumulating all data that may in fact be
% relevant to the search topic. This test should be
% modified as appropriate to the obscurity of the topic,
% ranging from +30 for common and public topics
% to –30 for obscure or unusual intel. This initial
% search has a timeframe of 1 minute. If successful, it
% turns up enough data to give the character a basic
% overview, perhaps with cursory details. The game-
% master should use the MoS to determine the depth
% of the data provided on this initial excursion, with
% an Excellent Success providing some bonus details.
% Similarly, a Severe Failure (MoF 30+) may result in
% the character working with data that is incorrect or
% intentionally misleading.

% The next step is not so much accumulating links
% and data as it is analyzing and understanding the in-
% formation acquired. This requires another Research
% Test, again modified by the obscurity of the topic. If
% the gamemaster allows it, complementary skills (p.
% 173) may apply to this test, providing bonus modi-
% fiers (for example, an understanding of Academics:
% Chemistry would help research the effects of a
% strange drug). Muses may also perform this task,
% though their skills are typically inferior. As above,
% success determines the quality and depth of the
% analysis, with an Excellent Success providing the full
% story and potential related issues and a Severe Fail-
% ure marking completely incorrect assumptions. The
% timeframe for this phase of research largely depends
% on two factors: the amount of data being analyzed
% and the importance to the storyline. Gamemasters

% %%% txt/253.txt
% need to carefully measure out their distribution of
% intel and clues to players. Give them too much too
% soon, and they may spoil the plot. Fail to give them
% enough, and they may get frustrated or pursue dead
% ends. Timing is everything.

% REAL-TIME SEARCHES
% Characters may also set up ongoing mesh scans that will
% alert them if any relevant information comes up new or
% updated, or is somehow changed. This is a task usually
% assigned to muses for continuous oversight. Whenever
% such data arises, the gamemaster secretly makes a Re-
% search Test, modified by the obscurity of the topic. If
% successful, the update is noted. If not, it is missed, though
% the gamemaster may allow another test if and when the
% topic reaches a wider range of circulation or interest.

% HIDDEN DATA
% It is important to remember that not everything can
% be found online. Some data may only be acquired (or
% may be more easily gotten) by asking the right people
% (see Networking, p. 286). Information that is consid-
% ered private, secret, or proprietary will likely be stored
% away behind VPN firewalls, in off-mesh hardwired
% networks, or in private and commercial archives. This
% would require the character to gain access to such
% networks in order to get the data they need (assuming
% they even know where to look).

% It’s worth noting that some entities send out AIs
% into the mesh with the intent of finding and erasing
% data they’d rather hide, even if this requires hacking
% into systems to alter such information.



% SCANNING, TRACKING,
% AND MONITORING
% Most users leave traces of their physical and digital
% presence all throughout the mesh. Accounts they access,
% devices with which they interact, services they use, ent-
% optics they perceive—all of these keep logs of the event,
% and some of these records are public. Simply passing
% nearby some devices is enough to leave a trail, as near-
% field radio interactions are often logged. This electronic
% datatrail can be used to track a user, both to ascertain
% their physical location or to note their online activities.

% WIRELESS SCANNING
% To interface with a wireless device or network, whether
% to establish a connection or for other purposes, the
% target device/network must be located first. To locate
% an active node, it’s wireless radio transmissions must be
% detected. Most wireless devices automatically scan for
% other devices in range (see Radio and Sensor Ranges, p.
% 299) as a matter of course, so no test is required. This
% means that it’s trivial for any character to pull up a
% list of the wireless devices and networks around them,
% along with associated mesh IDs. Likewise, a muse or
% device can be instructed to alert the user when a new
% signal (or a specific user) comes into range.

% Detecting stealth signals (p. 252), however, is a bit
% more challenging. To detect a stealthed signal, the scan-
% ning party must actively search for such signals, taking
% a Complex Action and making an Interfacing Test with
% a –30 modifier. If successful, they detect the hidden emis-
% sions. If the character aiming for stealth engages in active
% countermeasures, also requiring a Complex Action, then
% an Opposed Interfacing Test is called for (with the –30
% modifier still applying to the scanning party).

% For covert devices that are only transmitting in
% short bursts, wireless detection is only possible during
% the short period the burst transmission is being made.

% PHYSICAL TRACKING
% Many users willingly allow themselves to be physically
% tracked via the mesh. To them, this is a useful feature—
% it allows their friends to find them, their loved ones to
% know where they are, and for authorities to come to
% their aid in the event of some emergency. Finding their
% location is simply a matter of looking them up in the
% local directory, no test required (assuming you know
% who they are). Mesh positioning is accurate to within
% 5 meters. Once located, the position of the target can
% be monitored as they move as long as they maintain
% an active wireless connection to the mesh.

% TRACKING BY MESH ID
% An unknown user’s physical location can also be
% tracked via their online mesh activity—or more spe-
% cifically, by their mesh ID (p. 246). Network security
% will often trace intruders this way and then dispatch
% security squads to bring them in. To track an unknown
% user by their mesh ID alone requires a Research Test.
% If successful, they have been tracked to their current
% physical location (if still online) or last point of in-
% teraction with the mesh. If the character is in privacy
% mode (p. 252), a –30 modifier applies.

% TRACKING BY BIOMETRICS
% Given the existence of so many spimes and public
% cameras and sensors, people may also be tracked by
% their facial profile alone using facial recognition soft-
% ware. This software scans accessible video feeds and
% attempts to match it to a photo of the target. Given
% the sheer volume of cameras, however, and the typical
% range of false-positives and false-negatives, finding
% the target often boils down to luck. Priority can be
% given to cameras monitoring major thoroughfares, to
% narrow the search, but this risks missing the target if
% they avoid heavy traffic areas. The success of searches
% of this nature is best left to gamemaster fiat, but a
% Research Test can also be called for, modified ap-
% propriately by the range of the area being watched,
% whenever there is a chance the target may be spotted.

% Other biometric signatures may also be used for
% tracking this way, though these are usually less avail-
% able than cameras: thermal signatures (requires infra-
% red cameras), walking gait, scent (requires olfactory
% sensors), DNA (requires DNA scanners), etc. Each
% biometric scan requires a separate type of software.

% %%% txt/254.txt
% DIGITAL ACTIVITY TRACKING
% Tracking someone’s online activities (meshbrowsing,
% entoptic interactions, use of services, messaging, etc.)
% is slightly more difficult, depending on what exactly
% you’re after. Gathering information on a user’s public
% mesh activities—social network profiles, public
% forums posts, public lifelogging, etc.—is handled just
% like standard online research (p. 249).

% TRACKING BY MESH ID
% A more investigate search can attempt to use the
% target’s mesh ID (p. 246), using it as a sort of digital
% fingerprint to look up where else they’ve been online.
% This primarily involves checking access/transaction
% logs, which are not always publicly accessible. This
% sort of search requires a Research Test, handled as a
% Task Action with a timeframe of 1 hour.

% SNIFFING
% Wireless radio traffic is broadcast through the air (or
% space), meaning that it can be intercepted by other wire-
% less devices. “Sniffing” involves the capture and analysis
% of data traffic flowing through the wireless mesh.

% To eavesdrop on wireless communications, you need
% a sniffer program (p. 331) and you must be within
% radio range (p. 299) of the target (alternately, you can
% access a device that is within radio range of the target,
% and sniff from that location). To capture the informa-
% tion you must succeed in an Infosec Test. If successful,
% you capture data traffic from any targeted devices in
% range. Note that sniffing does not work on encrypted
% traffic (including VPNs and anything else using public
% key cryptography) as the results are gibberish. Quan-
% tum encrypted communications cannot be sniffed.

% Once you have the data, finding the information
% you’re looking for can be a challenge. Handle this as
% a standard Research Test (p. 245).

% REMOTE SNIFFING VIA MESH ID
% Finally, a mesh ID may also be actively monitored
% to see what mesh activity it engages in. This requires
% special sniffer software (p. 331) and a Research Test.
% If successful, the monitoring will provide information
% on that user’s public mesh activities (how much is
% determined by the gamemaster and the MoS), such
% as which sites they access, who they message, etc. It
% will not, however, uncover anything that is encrypted
% (unless the encryption is broken) or anything that
% takes place on a VPN (unless the VPN is hacked first),
% though it will show that encrypted communications
% and/or VPN use are taking place.



% PRIVACY AND ANONYMIZATION
% Given how easily mesh activities are monitored, many
% users pursue privacy and anonymization options.

% PRIVACY MODE
% Characters who go into privacy mode hide their online
% presence and activities from others to a limited degree.
% The exact settings are adjustable, but typically involve
% masking their social profiles and presence to other
% users in the immediate vicinity, like having an unlisted
% phone number. Privacy mode can also be used to limit
% the use of mesh IDs and other data in access and
% transaction logs, applying a –30 modifier to attempts
% to research or track them by their online activity.

% STEALTHED SIGNALS
% Another tactic that can be taken for privacy is to
% stealth the wireless radio signals you emit. This
% method uses a combination of spread-spectrum signals,
% frequency hopping, and modulation to make your
% radio transmissions harder to detect with scanning (p.
% 251). Stealthing your signals is either a passive activity
% (Automatic Action, –30 modifier on Interface Tests to
% locate the signal) or an active one (Complex Action,
% requires an Opposed Test to locate).

% ANONYMIZATION
% Anonymization takes the issue of privacy a bit further.
% The user does not just hide their mesh ID, but they
% actively use false mesh IDs and take other measures to
% reroute and obfuscate their datatrail. Anonymization
% is a necessity both for clandestine operatives and those
% engaging in illicit mesh activities.

% FALSE MESH IDS
% The easiest method of making mesh activities anony-
% mous is to set your muse to supply false mesh IDs in
% online transactions. Though illegal in many jurisdic-
% tions, this is an easy task for any character or muse to
% do. Multiple false IDs are used, making it extremely dif-
% ficult for anyone to tie all of the user’s activities together.

% This method makes it extremely difficult for anyone
% to track the user’s online actions. Someone attempting
% to track the character via these false mesh IDs must
% beat them in an Opposed Test, pitting their Research
% skill with a –30 modifier against the character’s (or
% more likely, their muse’s) Infosec skill. This is a Task
% Action with a base timeframe of 1 hour, adjusted
% higher according to the amount of activity they hope
% to track. If successful, the tracker manages to dig
% together enough correlating evidence and records of
% false IDs to get a picture of the character’s activities
% (how thorough this picture is depends on their MoS).
% If the fail, the anonymous character has effectively
% camouflaged themselves in the mesh.

% Actively monitoring a character who is fluctuating
% their mesh ID with a sniffer program, or physically
% tracking them via the mesh, is next to impossible, as
% the continual shifting of IDs and intentional decoys
% make it too difficult to keep up.

% ANONYMOUS ACCOUNT SERVICES
% A number of people—not just criminals, hackers, and
% secret agents—have an interest in keeping some of
% their affairs anonymous. To meet this demand, various
% online service vendors offer anonymous accounts for
% messaging and credit transfers. Some of these vendors

% %%% txt/255.txt
% are legit business (in places where it is legal), some
% are criminals operating illegally, others are hacktivists
% promoting the privacy meme, and still others are hy-
% percorps or other organizations offering such services
% internally to their own staff/membership.

% The interaction between the vendor and user is
% encrypted and anonymous, with no logs kept, so even
% if the vendor’s servers are hacked, an intruder will
% not find any leads. While some anonymous accounts
% are established for regular use, the truly paranoid use
% (multiple) one-time accounts for maximum security.
% One-time accounts are used for a single message (in-
% coming or outgoing) or credit transaction, and then
% are securely erased.

% Tracking an anonymous account is a practical
% impossibility, and something that only an extremely
% resourceful organization employing a systematic and
% expensive effort could attempt.

% DISPOSABLE ECTOS
% Another option for those seeking privacy and se-
% curity is to simply use disposable ectos. Using this
% method, all activity is routed through a specific ecto
% (using its mesh ID), the ecto is used for a limited
% period (until it gets hot), and then it is simply dis-
% carded or destroyed.



% MESH SECURITY
% Given the lessons of the Fall and the very real risk still
% posed by hackers, virii, and similar threats, network
% security is taken extremely seriously in Eclipse Phase.
% Four methods are typically used: authentication, fire-
% walls, active monitoring, and encryption.

% AUTHENTICATION
% Most devices, networks (PANs, VPNs, etc.), and ser-
% vices require some kind of authentication (a process
% by which a system determines whether the claimed
% identity of a user is genuine) before they grant an ac-
% count and access privileges (p. 246) to a user. There
% are several different ways for a system to authenticate
% a user. Some are more reliable and secure than others,
% but for the most part, the more secure the method, the
% higher the operational expenses.

% Account: If you have access to an account on one
% system, this may give you automatic access to related
% systems or subsystems. This is typical of slaved devices
% (p. 248), where access to the master automatically
% grants you access to slaves.

% Mesh ID: Some systems accept mesh IDs as authen-
% tication. This is extremely common with most public
% systems, which merely log the mesh ID of any user
% that wishes access. Other systems will only allow
% access to specific mesh IDs, but these are vulnerable
% to spoofing (p. 255).

% Passcode: This is a simple string of alphanumeric
% characters or logographic symbols, submitted in an
% encrypted format. Anyone with the passcode can
% access the account.

% Biometric Scan: This calls for a scan of one or more
% of the user’s biometric signatures (fingerprint, palm
% print, retinal scan, DNA sample, etc.). Popular before
% the Fall, such systems have fallen out of use as they
% are impractical with synthmorphs or users that fre-
% quently resleeve.

% Passkey: Passkey systems call for some of encrypted
% code that is either hardwired into a physical device
% (that is either implanted or physically jacked into
% an ecto) or extracted from specialized software. Ad-
% vanced passkeys combine hardwired encryption with
% physical nanotech etching to create a unique key. To
% access such systems, the passkey must either be ac-
% quired or somehow spoofed.

% Ego Scan: This system authenticates the user’s ego
% ID (p. 279).

% Quantum Key: Quantum key systems rely on the un-
% breakable encryption of quantum cryptography (p. 254).

% FIREWALLS
% Firewalls are software programs (sometimes hard-
% wired into a device) that intercept and inspect all traf-
% fic to and from a protected network or device. Traffic
% that meets specified criteria that designates it as safe
% is passed through, whereas all other traffic is blocked.
% In Eclipse Phase, every network and device can be
% assumed to have a firewall by default. Firewalls are
% the main obstacle that an intruder must overcome, as
% discussed under Intrusion Tests, p. 255.

% Like other gear, firewalls come in varying quality
% levels and so may apply modifiers to certain tests.

% ACTIVE MONITORING
% Instead of relying on authentication and firewalls
% alone, secure systems are actively monitored by a se-
% curity hacker or a muse. These digital security guards
% inspect network traffic using a number of software
% tools and applications that flag conspicuous events.
% Active surveillance makes intrusions more difficult,
% since the interloper must beat the monitoring hacker/
% AI in an Opposed Test (see Intrusion, p. 254). Active
% monitoring also includes monitoring any devices
% slaved to the monitored system.

% Characters may actively monitor their own PANs if
% they so choose, though this requires a moderate level
% of attention (count as a Quick Action). It is far more
% common for a muse to actively guard a user’s PAN.

% ENCRYPTION
% Encryption is an exceptionally effective extra layer of
% security. There are two types of encryption commonly
% used in Eclipse Phase: public key cryptosystems and
% quantum cryptography.

% PUBLIC KEY CRYPTO
% In public key cryptosystems, two keys are generated
% by the user, a public key and a secret key. The public
% key is used to encrypt messages to that user, and is
% made freely available. When messages are encrypted
% using that public key, only the secret key—controlled

% %%% txt/256.txt
% by the user—can decrypt them. Public key crypto is
% widely used both for encrypting data traffic between
% two users/networks/devices and for encrypting
% files. Due to the strength of the public key system
% algorithms, such crypto is essentially unbreakable
% without a quantum computer (see Quantum Code-
% breaking, p. 254).

% QUANTUM CRYPTOGRAPHY
% Quantum key distribution systems use quantum me-
% chanics to enable secure communications between two
% parties by generating a quantum key. The major advan-
% tage of transmitting information in quantum states is
% that the system itself instantly detects eavesdropping
% attempts as quantum systems are disturbed by any sort
% of external interference. In practical terms, this means
% that quantum encrypted data transfers are unbreakable
% and attempts to intercept automatically fail. Note that
% quantum crypto doesn’t work for basic file encryption,
% its only use is in protecting communication channels.

% While quantum key systems have an advantage over
% public key systems, they are both more expensive and
% less practical. In order to generate a quantum key,
% the two communications devices must be entangled
% together on a quantum level, in the same location,
% and then separated. So quantum key encrypted
% communications channels require some setup effort,
% especially if long distances are involved. Since the
% implementation of quantum cryptographic protocols
% is an extraordinary expense, it is usually only adopted
% for major high-security communications links.

% BREAKING ENCRYPTION
% What this means is that encrypted communications
% lines and files are very safe if using public key systems,
% and that data transfers are absolutely safe if using
% quantum crypto. Gamemasters should take note, how-
% ever: while this may be useful to player characters, it
% may also hinder them. If the characters need to get at
% something that is encrypted, they’re going to need to
% figure out some way to get the secret key’s passcode.
% Common methods include the old standbys of bribery,
% blackmail, threats, and torture. Other options involve
% espionage or social engineering to somehow acquire
% the passcode. Hackers could also find some other
% method to compromise the system and gain inside
% access, bypassing the encryption entirely.

% QUANTUM CODEBREAKING
% As noted above, quantum computers can also be used
% to break public key encryption. This requires an Infosec
% Task Action Test with a +30 modifier and a timeframe
% of 1 week (once started, the quantum computer finishes
% the job on its own; the user does not need to provide
% constant oversight). Gamemasters should feel free to
% modify this timeframe as fits the needs of their game.
% Note that quantum computers cannot break quantum-
% encrypted communications, only encrypted files.



% INTRUSION
% The art of intrusion lies in penetrating a device’s secu-
% rity. The best methods involve infiltrating a system qui-
% etly, without catching a watchdog’s attention, by using
% exploits—code glitches, flawed security protocols—to
% create a path circumventing the target’s defenses. When
% called for, however, a hacker can toss aside pretenses
% and attempt to brute-force their way in.

% PRECONDITIONS
% In order to hack a device, the hacker needs to establish
% a direct connection to the target computer system. If
% the hacker is making a direct wireless connection to
% the target, the target system must be wireless-capable
% and within range (p. 299), and the hacker must know
% the target is there (see Wireless Scanning, p. 251). If
% the system is hard-wired, the hacker must physically
% jack in by using a regular jacking port or somehow
% tapping into a cable that carries the network’s data
% traffic. If the hacker is accessing the target through the

% %%% txt/257.txt
% mesh, the target system must be online and the hacker
% must know it’s mesh ID (p. 246) or otherwise be able
% to track it down (p. 251).

% CIRCUMVENTING AUTHENTICATION
% Rather than hacking in, an intruder can try to sub-
% vert the authentication system used to vet legitimate
% users. The easiest manner of doing this is to somehow
% acquire the passcode, passkey, or whatever authentica-
% tion method the target uses (p. 253). With this in hand,
% no test is necessary to access the system; the hacker
% simply logs in just like a legitimate user and has all of
% the normal access privileges of that user.

% Lacking a passcode, the hacker can try to subvert
% the authentication system in one of two other ways:
% spoofing or forgery.

% SPOOFING AUTHENTICATION
% Using this method, the hacker attempts to disguise
% their signals as coming from the legitimate, authen-
% ticated user, rather than from themself. If successful,
% the system is fooled by this masquerade, accepting the
% hacker’s commands and activity as if they came from
% a legitimate user. Spoofing is more difficult to pull off,
% but is very effective when it works.

% To spoof a legitimate user, the hacker must be using
% both sniffer and spoofing software (p. 331). The
% hacker must then monitor a connection between the
% legitimate user and the target system, and succeed
% in an Infosec Test to sniff the traffic between them
% (p. 252). Apply a –20 modifier if the user has secu-
% rity account privileges, –30 if they have admin rights
% (p. 247). If the connection is encrypted, this will fail
% unless the hacker has the encryption key.

% Armed with this data, the hacker then uses it to
% disguise their signals. This requires an Infosec Test,
% modified by the quality of the system’s firewall and
% the hacker’s spoofing program. If successful, com-
% munications sent by the hacker are treated as coming
% from the legitimate user.

% FORGING AUTHENTICATION
% Biometric and passkey systems used for authentication
% (p. 253) can potentially be forged hackers who are
% able to get a look at the originals. The means and tech-
% niques for doing so differ, and are beyond the scope of
% this book, but successfully forging such systems would
% allow a hacker to log in as the legitimate user.




%  THE HACKING SEQUENC
%  1. Defeat the Firewall
%  2. Bypass Active Security

%  a. Hacker Wins with Excellent Success, Defender

%  b. Hacker Succeeds, Defender Fails

%  c. Both Succeed

%  d. Defender Succeeds, Hacker Fails


% INTRUSION TESTS


% Hacking into a node is a time-consuming task. The


% target system must be carefully analyzed and probed


% for weaknesses, without alerting its defenses. Depend-


% ing on the type of security in place, more than one test


% may be called for.


%   Hackers require special exploit software (p. 331) to


% take advantage of security holes, but software does


% not a hacker make. What really counts is Infosec skill


% (p. 180), which is the ability to use, modify, and im-


% provise exploits to their full advantage.



% DEFEATING THE FIREWALL


% Lacking a passcode, the hacker must break in the old-


% fashioned way: discreetly scanning the target, look for


% weaknesses, and take advantage of them. In this case


% the hacker takes their exploit software and makes an


% Infosec Test. This is handled as a Task Action with a


% timeframe of 10 minutes. Various modifiers may apply,


% such as the quality of the exploit software, the quality


% of the Firewall, or the alertness of the target system. The


% gamemaster may also modify the timeframe, shortening


% it to reflect systems that are cookie-cutter common with


% known security flaws or raising it as fitting for a top-of-


% the-line system with still-unreleased defenses.


%    By default, a hacker trying to break in this way is


% pursuing standard user access rights (p. 247). If the


% hacker wishes to obtain security or admin privileges on


% the system, apply a –20 or –30 modifier, respectively.


%     If the Infosec Test succeeds, the intruder has invaded


% the system without triggering any alarms. If the system is


% actively monitored (p. 253), they must now avoid detec-


% tion by that watchdog (see below). If there is no active


% monitor, the intruder gains the status of Covert (see In-


% truder Status, p. 256). If the intruder scored an Excellent


% Success, however, their status is Hidden (p. 256).


%     Probing: Players may choose to take the time (p.


% 116) when probing the target for weakness and ex-


% ploits. In fact, this is a common procedure when a


% hacker wants to ensure success.



% BYPASSING ACTIVE SECURITY


% If a system is also actively monitored (p. 253), the


% hacker must avoid detection. Treat this as a Variable


% Opposed Infosec Test between the intruder and the


% monitor. The outcome depends on both rolls:


%    If only the intruder succeeds, the hacker has ac-


% cessed the node without the monitor or the system




% E


%      Infosec Task Action (10 minutes)


%      Opposed Infosec Test
% ls        Hidden status/+30 all tests (p. 256)


%      Covert Status (p. 256)


%      Spot Status/Passive Alert (p. 256)


%      Locked status/Active Alert (p. 256)                  ■

% %%% txt/258.txt
% noticing. The hacker has acquired Covert status (p.
% 256). If the hacker scored an Excellent Success, their
% status is Hidden (p. 256).

% If only the monitor succeeds, the hacking attempt
% is spotted and the monitor may immediately lock the
% hacker out of the system before they manage to fully
% break in. The intruder may try again, but the monitor
% will be vigilant for further intrusions.

% If both succeed, the intruder has gained access but
% the monitor is aware that something strange is going
% on. The hacker acquires Spotted status.

% If both fail, continue to make the same test on each
% of the hacker’s Action Phases, until one or both succeed.

% INTRUDER STATUS
% Intruder status is a simple way of measuring an
% invader’s situation when they are intruding upon
% a system. This status has an impact on whether the
% hacker has caught any attention or if they managed
% to remain unobtrusive. Status is first determined when
% the intruder access the system, though it may change
% according to events.

% Note that intruder status is a separate matter from
% account access privileges (p. 246). The latter represents
% what a user can legally do on a system. The former
% indicates how aware the system is of the hacker’s true
% nature as an intruder.

% HIDDEN
% An intruder with Hidden status has managed to silent-
% ly sneak into the system without anyone noticing. The
% system’s security is totally unaware of their presence
% and may not act against them. In this case, the hacker
% is not using an account so much as they are exploit-
% ing a flaw in the system that grants them a nebulous,
% behind-the-scenes sort of presence in the system. The
% hacker effectively has admin access rights, but does
% not show up as an admin-level user in logs or other
% statistics. Hidden characters receive a +30 modifier on
% any efforts to subvert the system.

% COVERT
% An intruder with Covert status has accessed the
% system in a manner that doesn’t attract any unusual
% attention. For all intents and purposes, they appear
% to be a legitimate user with whatever access rights
% they sought. Only extensive checking will turn up any
% abnormalities. The system is aware of them, but does
% not consider them a threat.

% SPOTTED
% Spotted status indicates that the system is aware of an
% anomaly or intrusion but hasn’t zeroed in on the in-
% truder yet. The hacker appears to be a legitimate user
% with whatever access rights they sought, but this will
% not hold up under close scrutiny. The system goes on
% passive alert (inflicting a –10 modifier to the hacker’s
% activities on that system) and may engage the hacker
% with passive countermeasures (p. 257).
% LOCKED
% Locked status means that the intruder—including their
% datatrail—has been pinned down by system security.
% The hacker has access and account privileges, but they
% have been flagged as an interloper. The system is on
% active alert (inflicting a –20 modifier on the hacker’s ac-
% tions) and may launch active countermeasures (p. 257)
% against the intruder.

% CHANGING STATUS
% An intruder’s status is subject to change according to
% their actions and the actions of the system.

% UPGRADING STATUS
% A hacker can attempt to improve their status in order
% to better protect themself. This requires a Complex
% Action and an Infosec Test. If the hacker has Spotted
% status, this is an Opposed Test between monitor and
% intruder. If the hacker wins and scores an Excellent
% Success (MoS of 30+), they have upgraded their status
% by one level (for example, from Covert to Hidden).
% Intruders with Locked status may not upgrade.

% ZEROING IN
% A security hacker or muse that is actively monitoring
% a system can take a Complex Action and attempt to
% hone in on a Spotted intruder. An Opposed Infosec Test
% is made between both parties. If the system’s defender
% wins, the hacker is downgraded to Locked status.

% FAILING TESTS
% Any time an intruder scores a Severe Failure (MoF
% 30+) on a test involving manipulating the system, they
% are automatically downgraded one status level (from
% Covert to Spotted, for example). If a critical failure
% is rolled, they immediately give themselves away and
% achieve Locked status.

% BRUTE-FORCE HACKING
% Sometimes a character simply doesn’t have time to do
% the job right, and they need to hack in now or never.
% In this case the hacker engages the target system im-
% mediately, head on, without taking any time to pre-
% pare an attack. The hacker simply brings all of their
% software exploit tools to bear, throwing them at the
% target and hoping that one works. This is handled as
% an Infosec Test, but as a Task Action with a timeframe
% of 1 minute (20 Action Turns). The hacker receives
% a +30 modifier on this test. Many hackers choose to
% rush the job (see Task Actions, p. 120), in order to cut
% this time even shorter.

% The drawback to brute-force hacking is that it imme-
% diately triggers an alarm. If the system is actively moni-
% tored, the hacker must beat the monitor in an Opposed
% Infosec Test or be immediately locked out as soon as
% they break in. Even if they succeed, the hacker has
% Locked status and is subject to active countermeasures.

% %%% txt/259.txt
% INTRUSION COUNTERMEASURES
% If an intruding hacker fails to penetrate a system’s
% defenses (i.e., they are Spotted or Locked, see p. 256),
% then the system goes on alert and activates certain
% defenses. The nature of the applied countermeasures
% depends on the capabilities of the system, the abilities
% of its security defender(s), and the policy of its owner/
% admins. While some nodes will simply seek to kick
% the intruder out and keep them shut out, others will
% actively counterattack, seeking to track the intruder
% and potentially hack the intruder’s own PAN.

% SECURITY ALERTS
% Security alerts come in two flavors: passive and active.

% PASSIVE ALERT
% Passive alerts are triggered when an intruder hits Spot-
% ted status. The system immediately flags a visual or
% acoustic cue to anyone actively monitoring the system
% and possibly the owner or admins. It immediately
% launches one or more passive countermeasures (see
% below). Depending on the system, extra security hack-
% ers or AIs may be brought in to help investigate. If the
% intruder is not encountered again or located within a
% set time period (usually about 10 minutes), the alarm
% is deactivated and the event is logged as an anomaly.
% Depending on the security level of the system, some-
% one may analyze the logs at some point and try to
% ascertain what happened—and prevent it from hap-
% pening again.

% All intruders suffer a –10 modifier for tests involv-
% ing a system that is on passive alert.

% ACTIVE ALERT
% An active alert is triggered when an intruder hits
% Locked status. The system immediately alerts the
% owners, admins, and monitoring security agents.
% Additional security assets (hackers and AIs) may be
% called in. The system also launches active countermea-
% sure against the intruder (see below). Active alerts are
% maintained for as long as the intruder is present, and
% sometimes for a lengthy period afterwards just in case
% the hacker returns.

% PASSIVE COUNTERMEASURES
% Passive countermeasures are launched as a precaution
% whenever an intruder acquires Spotted status.

% LOCATE INTRUDER
% A security hacker or AI monitoring a system may at-
% tempt to track down the source of the passive alert.
% See Zeroing In, p. 256.

% RE-AUTHENTICATE
% When a passive alert is triggered, a firewall can be set
% to re-authenticate all active users, starting with the
% most recent. At the beginning of the next Action Turn,
% everyone on the system must take an action to log
% back in. For intruders, this means making an Infosec
% Test, modified by –10 for the passive alert, to satisfy
% the system that they are a legitimate user.

% REDUCE PRIVILEGES
% As a protective measure, some systems will immedi-
% ately reduce access privileges available to standard
% users, and sometimes security users as well. One
% common tactic is to protect all logs, backing them up
% and making sure no one has rights to delete them.

% ACTIVE COUNTERMEASURES
% Active countermeasures can only be launched if the
% intruder has acquired Locked status.

% COUNTERINTRUSION
% A security hacker or guardian AI can proactively
% defend a system by attacking the intruder’s source. For
% this to occur, the intruder must first be successfully
% traced (p. 251). Once this occurs, the security forces
% can then launch their own intrusion on the hacker’s
% home ecto/mesh inserts and/or PAN.

% LOCKOUT
% A system that has locked onto an intruder may also
% attempt to lock them out. Lockout is an attempt to
% remove the compromised account, sever the connec-
% tion between the two, and dump the hacker from
% the system.

% Lockout must be initiated by someone with secu-
% rity or admin privileges. An Opposed Infosec Test is
% made, with the intruder suffering a –20 modifier for
% being Locked. If the character defending the system
% succeeds, the intruder is immediately ejected from
% the system and the account they used will be placed
% on quarantine or deleted. That account will not be
% usable again until a security audit approves it and
% replaces the authentication. Any attempt to access
% the system from the same mesh ID as the intruder
% automatically fails.

% REBOOT/SHUTDOWN
% Perhaps the most drastic option for dealing with an
% interloper is to simply shut down the system. In this
% case, the system closes all wireless connections (if it
% has any), logs off any users, terminates all processes,
% and shut itself down—thereby locking out the intruder.
% The disadvantage, of course, is that the system must
% interrupt its activities. For example, shutting down
% your mesh inserts or ecto means losing all communica-
% tion with teammates, access to augmented reality, and
% control over slaved/linked devices.

% Initiating a reboot/shutdown is only a Complex
% Action, but the actual process of shutdown takes
% anywhere from 1 Action Turn (personal devices) to 1
% minute (large hardwired networks with multiple users),
% determined by the gamemaster. Rebooting a system
% takes the same amount of time to get started again.

% %%% txt/260.txt
% TRACE
% For high-security systems, a popular countermeasure
% is to track the infiltrator’s physical location via their
% mesh ID (see Physical Tracking, p. 251). In most cases,
% habitat physical security is subsequently alerted and
% forwarded the position to take care of the criminal.

% WIRELESS TERMINATION
% An alternative to shutdown or rebooting is simply to
% sever all wireless connections by shutting down the
% wireless capabilities of the system. The system will
% lose all active connections, but any intruders will be
% dumped. Wireless termination is a Complex Action to
% initiate and completes at the end of that Action Turn.
% Re-starting wireless connectivity takes 1 Action Turn.





% JOINT HACKING/

% SECURING

% Hacking will sometimes involve teams of attack-

% ers and/or teams of defenders. A hacker might

% be backed up by their muse or another team

% member with moderate Infosec skills. Hard

% networks are often defended and monitored

% by teams of highly-skilled security hackers and

% AIs. When intruding in or defending a computer

% system, operators must decide whether to act

% individually or in concert.


% Each approach has its tradeoffs. A team that

% chooses to breach or maintain a system’s security

% as a team effort must allocate one character (usu-

% ally the team member with the highest Infosec

% skill) as the primary actor (see Teamwork, p. 117).

% Each additional character and muse adds a +10

% modifier for each test (up to the maximum +60

% modifier) but cannot spend time on other actions

% than those performed by the team leader. When

% acting in concert, teams may switch team leaders

% at any time, in case group members are special-

% ized for certain tasks.


% Alternately, both intruding and defending

% teams may choose to act individually but for a

% joint goal. Each hacker must make intrusions

% on their own, with individual repercussions for

% detection and counterintrusion, which runs the

% risk of affecting all intruders if any one is Spotted

% or Locked. On the other hand, a team of intrud-

% ers can pursue multiple actions simultaneously in

% a coordinated manner and may temporarily over-

% whelm available security. The same holds true for

% system defenders, who may accomplish more by

% splitting their actions, leaving some to monitor

% while others launch counterintrusion attacks and

% other countermeasures.                              ■

% %%% txt/261.txt
% SUBVERSION
% Once an intruder has successfully invaded a device
% or network, they can pursue whatever tasks they are
% interested in, as fitting that particular system. Depend-
% ing on the type of account the intruder hacked, they
% may or may not have access privileges to do what
% they want to do. If their access rights allow it, the
% activity is handled like that of a legitimate user and
% no test is called for (unless the activity itself calls for
% some kind of test, such as Research). For example, a
% hacker who infiltrates a habitat’s security system with
% a security account can monitor cameras, deactivate
% sensors, review recorded surveillance footage, and so
% on, as any legitimate user with security right would be
% allowed to do.

% Engaging in any sort of activity for which you
% don’t have access rights is more difficult and requires
% hacking the system. This typically requires an Infosec
% Success Test, modified by the difficulty of the action
% as noted on the Subversion Difficulties table. In most
% cases this in not an Opposed Test even if the system is
% actively monitored, unless specifically stated otherwise.
% Failing such tests, however, will result in a change of
% the hacker’s intruder status (see Failing Tests, p. 256).

% Examples for different types of system subversion
% are given in the Subversion Examples sidebar. This
% is not an exhaustive list, however, and gamemasters
% and players are encouraged to improvise game effects
% in case an action has not been explicitly described.

% AUGMENTED REALITY ILLUSIONS
% A hacker who has infiltrated an ecto, mesh inserts, or
% some other device with an AR interface may inject
% different kinds of visual, auditory, tactile, and even
% emotional illusions into the augmented reality of
% the device’s user, depending on the type of interface
% used. How the hacked user will respond to the illu-
% sion depends on a number of factors, such as whether
% they are aware of the intruder (i.e., the hacker has
% Spotted or Locked status), what type of interface they
% are using (entoptic or haptic), and how realistic the
% illusion is.

% The best illusions are, of course, crafted in advance,
% using the best image and sensory manipulation tools
% available. Such illusions are hyper-realistic. Anyone
% making a Perception Test to identify them as fake




%                                     SUBVERSION


%                                          Difficulty modifiers

% MODIFIER TASK


%         Execute commands, view restricted information, run

%  –0


%         copy/delete files, access sensor feeds, access slaved d

% –10       Change system settings, alter logs/restricted files

% –20       Interfere with system operations, alter sensor/AR inp

% –30       Shut system down, lockout user/muse, launch counte


%               SUBVERSION EXAMPLES


%     In addition to the tasks noted under the Subversion Difficulties


%          these modifiers present some additional example actions.

% MODIFIER TASK


% Hacking Bots/Vehicles

%  –0         Give orders to drones

% –10         Alter sensor system parameters, disable sensors or weapo

% –20         Alter smartlink input, send false data to AI or teleoperato

% –30         Lockout AI or teleoperator, seize control via puppet sock


% Hacking Ectos/Mesh Inserts


%           Interact with entoptics, befriend everyone in range, make

%  –0


%           purchases using user’s credit, intercept communications,


%           Alter social network profile/status, adjust AR filters, twea

% –10


%           interface, change AR skin, change avatar, access VPN


%           Block or shuffle senses, inject AR illusions, spoof comman

% –20


%           drones/slaved devices

% –30         Boot user out of AR


% Hacking Habitat Systems

%  –0         Open/close doors, stop/start elevators, operate intercom


%           Adjust temperature/lighting, disable safety warnings, repl

% –10


%           tic skin, lock doors, switch traffic timers


%           Disable subsystems (plumbing, recycling, etc.), disable wi

% –20


%           dispatch repair crews

% –30         Override safety cutoffs


% Hacking Security Systems

%  –0         Move/manipulate cameras/sensors, locate security systems


%           Adjust patterns of sensor sweeps, view security logs, disa

% –10


%           systems

% –20         Delete security logs, dispatch security teams

% –30         Disable alerts


% Hacking Simulspace Systems

%  –0         View current status of simulspace, simulmorphs, and acce


%           Change domain rules, add cheats, alter parameters of sto

% –10


%           simulmorphs, change time dilation

% –20         Eject simulmorph, alter/erase character AIs

% –30         Abort simulation


% Hacking Spimes

%  –0         Get status report, use device functions

% –10         Adjust AI/voice personality settings, adjust timed operatio

% –20         Disable sensors, disable device functions




% DIFFICULTIES
% ommon computer tasks


% cted software, open/close connections to other systems, read/write/
% es



% asures at others

% %%% txt/262.txt
% suffers a –10 to –30 modifier (gamemaster’s discre-
% tion). An eclectic collection of software programs
% offer a diverse range of AR illusions.

% Hackers may also improvise illusions on the
% fly, usually by patching in sensor data from others
% sources, though this is more difficult and more easily
% spotted (typically adding a +10 to +30 modifier to
% Perception Tests). The advantage is that the hacker
% can modify the illusion in reaction to the user’s ac-
% tions or environmental factors on the fly. AR illusion
% software, however, also offers some template illusions
% that can be modified and controlled in real-time via a
% connected interface.

% Whenever a user is bombarded with AR illusions, the
% gamemaster should make a secret Perception Test to see
% if they spot the deception. Even if they do, however, the
% character may still react to them. Almost anyone will
% duck when they see an object suddenly flying at their
% face, as their body reacts on its own before the brain
% comprehends that it’s an illusion and not a threat.

% Aside from their deceptive value, illusions can be
% used to distract users or otherwise impair their percep-
% tive faculties. For example, dark illusory clouds can
% obscure vision, ear-wrenching high-volume noises can
% make people cringe, and a persistent tickling sensation
% might drive anyone crazy. Such effects can apply a –10
% to –30 modifier to Perception Tests and other actions,
% but the user can also adjust their filters and/or turn
% their AR off if necessary.

% BACKDOORS
% A backdoor is a method of bypassing a system’s normal
% authentication and security features. It enables a hacker
% to sneak into a system by exploiting a flaw (which can
% take the form of an installed program or modification
% to an existing program or hardware device) that was
% integrated into the system previously, either by themself
% or another hacker (who shared the backdoor).

% To install a backdoor, the hacker must successfully
% infiltrate the system and succeed in both a Program-
% ming and an Infosec Test (or an Opposed Infosec Test
% if the system is actively monitored). The Programming
% Test determines how well the backdoor is crafted and
% hidden within system processes, while the Infosec Test
% represents incorporating it into the system without se-
% curity noticing. Modify the Programming Test by –20
% if the hacker wants to have security privileges when
% using the backdoor, –30 for admin.

% Once installed, using a backdoor requires no test to
% access the system—the hacker simply logs on as if they
% were a legitimate user, gaining Covert status. Anyone
% who is aware of a backdoor’s details may use it.

% How long the backdoor lasts depends on many fac-
% tors and is largely up to the gamemaster. Backdoors
% are only likely to be spotted during complete security
% audits, so more paranoid systems are likely to detect
% them earlier. Security audits may also occur when an
% intruder is Spotted but never Locked. Security audits
% are a Task Action with a timeframe of 24 hours. The
% character conducting the audit makes an Infosec
% Test to spot the back door. If the backdoor’s installer
% scored an Excellent Success on their Programming
% Test, this Infosec Test suffers a –30 modifier.

% CRASHING SOFTWARE
% Intruders can attempt to crash software programs
% by killing the processes that run them. This requires a
% Complex Action and an Infosec Test. Note that some
% software is set to immediately respawn, but this can take
% from 1 Action turn to 1 minute, depending on the system.

% Hackers may crash AIs, AGIs, and even infomorphs
% this way, but the process is more difficult. In this case,
% an Opposed Infosec Test is made against the target,
% who is immediately aware they are under attack. Two
% consecutive tests must succeed in order to crash an AI,
% three in order crash an AGI or infomorph. If successful,
% the AI/infomorph immediately reboots, which generally
% takes 3 Action Turns, longer if the gamemaster chooses.

% ELIMINATING INTRUSION TRACES
% Hackers who have avoided being Locked may attempt
% to clean up all traces of their intrusion before they
% exit a system. This involves erasing incriminating data
% in the access and security logs, and otherwise hiding
% any evidence of system tampering. This requires a
% Complex Action and an Infosec Test, or an Opposed
% Infosec Test if the system is actively monitored. If suc-
% cessful, the intruder has wiped anything that might be
% used to track them down later, such as mesh ID, etc.

% HACKING VPNS
% Virtual private networks (VPNs) are more challeng-
% ing to hack than standard devices. Because they exist
% as an encrypted network within the mesh, accessing
% channels of communication within a VPN is all but
% impossible with the encryption key. This means any
% attempt to sniff the VPN traffic is also impossible
% with the key.

% The only way to hack a VPN is to hack into a
% device that is part of the VPN and running the VPN
% software. Once an intruder has access to such a device,
% they can attempt to access the VPN. The account the
% hacker has compromised may have VPN privileges, in
% which case they are in. If not, they must hack access,
% requiring an Infosec Test with a Minor modifier (–10).

% Once access to the VPN is acquired, the hacker may
% treat the VPN like any other network. They may hack
% other devices on the VPN, sniff VPN traffic, track
% other users on the VPN, research data hidden away
% on the VPN, and so on.

% SCRIPTING
% A script is a simple program—a batch of instructions—
% that a hacker can embed in a system to be executed at
% a later pre-scheduled time or upon a certain trigger
% event, even without the hacker being present. When
% activated, the script will undertake a number of system
% operations limited by the abilities of the operating
% system and the access rights the hacker had when
% they implemented the script in the system. Scripts are

% %%% txt/263.txt
% a great way for a hacker to subvert a system without
% necessarily being in danger when they do it.

% Scripts can be programmed on the fly or prepro-
% grammed. When composing the script, the character
% must detail what system operations the set will call
% for, in what order and at what times (or at what trig-
% ger steps). The script cannot contain more steps/tasks
% than the character’s Programming skill ÷ 10 (round
% down). To program a script, the character must suc-
% ceed in a Programming Test with a timeframe deter-
% mined by the gamemaster.

% To load the script, the character must have suc-
% cessfully intruded in the system and must succeed
% in an Infosec Test (or an Opposed Infosec Test if the
% system is actively monitored). If successful, the script
% is loaded into the system and will run as programmed.

% Once the script is activated, it carries out the pre-
% programmed sequence of actions. The programmer’s
% Infosec skill is used for any tests those actions call for.

% Inactivated scripts may be detected in a security
% audit, just like backdoors (p. 260).




%       Sarlo has infiltrated a security system and wants


%       to arrange it so that a particular security sensor


%       deactivates and a door unlocks at a set time, al-


%       lowing his team to infiltrate a high-security area.


%       He creates a script that will activate at 2200 hours


%       with the following steps:


%       1) At 2200 hours, disable security sensor


%       2) Then unlock door
%  EXAMPLE






%       3) At 2230 relock door


%       4) Then re-enable security sensor


%       5) Eliminate traces


%          This script has 5 steps, which Sarlo can handle


%       with his Programming skill of 70. Sarlo succeeds in


%       his Programming and Infosec Tests, and the script


%       is loaded. It will then activate at the appropriate


%       time. Since Sarlo’s account did not have access


%       rights to perform these actions, each will require


%       an Infosec Test using Sarlo’s skill to succeed.




% CYBERBRAIN HACKING
% Pods and synthmorphs (including some bots and ve-
% hicles) are equipped with cybernetic brains. While this
% technology allows a transhuman ego to sleeve into and
% control these forms, they carry the disadvantage of being
% vulnerable to hacking, like any other electronic device.

% Cyberbrains are not wireless-enabled for security
% reasons, but they do have access jacks (p. 306) and
% are directly linked to mesh inserts. This means that
% in order to hack a cyberbrain, the hacker either must
% have direct physical access to the morph’s body in
% order to jack in, or they must first hack into the mesh
% inserts and then break into the cyberbrain from there.

% Due to their importance, cyberbrains are equipped
% with numerous hard-coded security features that make
% intrusion very difficult. Apply a –30 modifier to all
% attempts to hack into and subvert a cyberbrain. (Note
% that the –30 modifier for hacking an admin account
% does not apply to cyberbrains.)

% Cyberbrains are treated just like other systems for
% intrusion and subversion purposes, but since they
% house the morph’s controlling ego they present several
% unique hacking opportunities.

% ENTRAPMENT
% An intruding character can attempt to lock in an ego,
% preventing it from evacuating the cyberbrain. The
% hacker (with the –30 modifier noted above) must
% beat the defending character or muse in an Opposed
% Infosec Test. If successful, the ego is prevented from
% transferring itself to another system.

% To fully pen the ego in, the ego character and its
% protecting muse must also be locked out (p. 257) from
% controlling the cyberbrain’s system, otherwise the ego
% could potentially be freed.

% Trapped egos are quite vulnerable. They could, for
% example, be subject to enforced uploading, enforced
% forking, or psychosurgery.

% MEMORY HACKING
% All cyberbrains incorporate mnemonic augmentation
% (p. 307), or digitally recorded memories. A hacker
% who has accessed the cyberbrain can read, alter, or
% delete these memories with a successful Research or
% Interfacing Test (the –30 modifier applies).

% PUPPETEERING
% Cyberbrains also incorporate a puppet sock (p. 307),
% enabling remote users to take over the pod or synth-
% morph body and control it via teleoperation or jam-
% ming (p. 196). This allows a hacker to seize control
% of the body and manipulate it remotely. To do so, the
% hacker must take a Complex Action and beat the de-
% fending character or muse in an Opposed Infosec Test;
% the hacker suffers the –30 modifier noted above.

% A defender who is not locked out may continue
% to fight for control of the morph, using a Complex
% Action. In this case, another Opposed Infosec Test
% is called for. This can result in a situation where the
% morph repeatedly slips control from the hacker to the
% defender, or perhaps slips into a catatonic state as the
% two sides battle it out.

% SCORCHING
% Having direct access to a cyberbrain opens the possi-
% bility for certain kinds of attacks that are normally in-
% feasible due to the strict content filtering that occurs on
% the link between the cyberbrain and mesh inserts. One
% of these possibilities is scorching—the use of damaging


%                               —the
% neurofeedback algorithms to harm the victim’s mind.

% In order to make a scorching attack, the cyberbrain
% intruder must deploy a scorch program. To utilize a
% scorch program, the intruder must beat the defending
% Ego in an Opposed Infosec Test. The –30 modifier for
% cyberbrain hacking applies to the attacker.

% Several types of scorch programs exist, with dif-
% ferent effects: cauterizers (damage), bedlams (stress),

% %%% txt/264.txt
% spasms (pain), nightmares (fear), and shutters (sensory
% deprivation). These are described on p. 332 of Gear.

% SHUTDOWN
% If a cyberbrain is shut down (p. 257), the morph im-
% mediately ceases activity, perhaps collapsing or rolling
% to a stop. Pods will appear to be in a coma. The ego,
% however, will be rebooted along with the cyberbrain.

% TERMINATE CORTICAL STACK FEED
% The cyberbrain feeds backup data to the cortical stack.
% This is a one-way connection, so the cortical stack may
% not be hacked, but the transfer of data may be cut off.
% This termination action requires an Opposed Infosec
% Test between the hacker (with the –30 modifier) and
% the defender. The ego’s backup will not be updated for
% as long as the connection remains off.



% RADIO JAMMING
% Radio jamming is a method of transmitting radio
% signals that deliberate interfere with other radio
% signals in order to disrupt communications. In the
% highly-networked world of Eclipse Phase, intentional
% jamming is often illegal, not to mention rude.

% Radio jamming does not require any special equip-
% ment other than a standard wireless device, such as
% an ecto or mesh inserts. Jamming can be selective
% or universal. Selective jamming targets a particular
% device or set of devices. In order to selectively jam,
% the character must have scanned the target device(s)
% (p. 251). Universal devices target all radio-equipped
% devices indiscriminately.

% Jamming simply requires a Complex Action and an
% Interfacing Test. If successful, affected devices within
% range have their radio communications disrupted—
% they are cut off from the mesh and may not commu-
% nicate wirelessly. Wired devices are unaffected.

% Devices equipped by AIs will automatically attempt
% to overcome jamming, which requires a Complex
% Action (transhuman users may also do the same). In
% this case, a Variable Opposed Test is made between
% the jammer and defender. If the jammer wins, all com-
% munications are blocked; if the defender wins, they
% are unaffected. If both parties succeed, the defender’s
% communications are impacted but not completely cut
% off. The gamemaster decides how much information
% the defender can get through, and how this situation
% affects mesh use.

% JAMMING RADAR
% Jamming can also be used to interfere with radar. In
% this case, if the jammer makes an Interface Test. If
% successful, the radar suffers interference, imposing
% a –30 modifier on all sensor-related tests. The entity
% operating the radar may attempt to overcome this
% interference by beating the jammer in an Opposed
% Interface Test.
% SIMULSPACES
% Simulspaces are virtual reality environments where
% the resolution advances beyond realistic high defini-
% tion and into the hyper-real. The environments they
% create are comprehensive and authentic illusions, from
% aspects like lighting, day or lunar cycles, and weather
% down to minute details and sensations. Jacking into
% a simulspace is much like crossing over into a alter-
% nate world or reality, which is why simulspaces have
% become increasingly popular in entertainment.

% While simulspaces usually cannot harm characters
% immersed in them as the sensory algorithms are not
% intended to be offensive programs or routines, experi-
% ences in simulspaces can have a strong psychological
% impact on an ego, as the simulation is as close to
% reality as you can get. A character who is “physically”
% tortured within a simulspace will not be physically
% harmed, but the mental stress of the experience might
% still be sufficient to cause permanent traumas.

% SIMULMORPHS
% Characters access simulspace using an avatar-like per-
% sona called a simulmorph. This simulmorph is created by
% the simulspace, based on the domain rules of the simu-
% lation and certain characteristics of the morph or ego
% accessing the simulation. Depending on the simulation,
% this simulmorph may be customizable to varying degrees.

% While interacting with the simulation, treat simul-
% morphs as basic infomorphs for all rules purposes,
% even if the egos are still possessing another morph
% body in reality.

% When accessing a simulspace, muses are usually not
% transferred into the simulation, though they can po-
% tentially come along if domain rules permit it. In this
% case, muses are treated as separate characters within
% the simulspace with their own simulmorph body.

% Depending on the role a simulspace is intended to play
% in the story, the gamemaster may want to invent “physi-
% cal stats” for the simulmorph bodies, especially if the
% characters are likely to spend a lot of time in the simula-
% tion. These statistics can literally be made up—it is a vir-
% tual reality after all, and anything goes. Alternately, the
% gamemaster can simply wing it and invent any necessary
% statistics on the fly as the need for them comes up.

% IMMERSION
% When a character immerses themselves in a simulspace,
% they “become” the simulmorph. The character’s physi-
% cal body, typically secluded and protected in a vat or
% couch, slumps inertly. While immersed, they suffer –60
% on all Perception Tests or attempts to take action with
% their physical morph. Characters can enter and leave
% the simulspace as will, but toggling in or out takes a
% Complex Action.

% If the simulspace crashes or the character is other-
% wise dumped from it, they immediately resume con-
% trol of their own morph as normal. VR dumpshock is
% extremely jarring, and the character suffers 1d10 ÷ 2
% mental stress.

% %%% txt/265.txt
% EXTERNAL MESH INTERACTION
% A character accessing a simulspace may still interact
% with the mesh (and through it, the outside world)
% assuming the domain rules allow for it. Any outside
% interactions are subject to time dilation issues, how-
% ever. For example, in a simulspace running faster
% than real time, holding a chat with someone in out-
% side meatspace is excruciatingly slow, as real-world
% seconds translate into minutes in VR. If a character
% wishes to directly access other mesh nodes, they must
% toggle or log out of the simulspace.

% SIMULSPACE RULES
% Since a simulspace is an alternate world whose realism
% matches reality, characters use their physical skills and
% aptitudes as if they were acting in the real world with
% few exceptions:

%  • Though intrusion and hacking can be represented

% as another layer of the simulation, there is no

% actual hacking within the simulspace (see Hack-

% ing Simulspaces).
%  • Asyncs cannot use their psi abilities in simulspace,

% though such abilities can be simulated.
%  • Any “physical” damage taken in the simulspace is

% treated as “virtual” damage. While virtual injuries

% and wounds use the same mechanics, characters

% that die in a simulspace are usually simply ejected

% from the simulation. In some cases “dead” char-

% acters are brought into a white room and can

% re-enter or just watch the simulation, depending

% on the domain rules.
%  • Mental stress or trauma inflicted during a simu-

% lation carries over to the ego as real Lucidity

% damage. At the gamemaster’s discretion, some

% mental stress may be reduced if the character is

% aware that they are in a simulation.

% DOMAIN RULES
% Anything goes in a simulspace, as dictated by
% the domain rules. A simulspace may range from
% approximating reality very closely to differing dras-
% tically. Gravity might fluctuate, the visual light spec-
% trum might not exist, characters might heal virtual
% damage effortlessly, simulmorphs may be capable of
% transmogrifying into other creatures, everything might
% be underwater—the possibilities are endless, limited
% only by imagination. In game terms, this allows the
% gamemaster to make up rules on the fly.

% CHEATING
% As with any good game, simulspaces provide ways
% to cheat. Cheats are either built into the simulspace
% software or (externally) programmed in by a hacker.
% Cheats allow for a character to break the domain rules
% in some way. This may be a special power, a way to
% alter some environmental factor (like flying), altering
% the time dilation, some sort of power-up ability, a
% way to get info on other simulmorphs, or a short-cut
% through part of the simulation. In game terms, cheats
% might provide bonus modifiers to certain skill or stat
% tests made by a simulmorph. Cheating is usually for-
% bidden. Players who cheat in a simulspace game and
% who get caught may face eviction from the simulspace.

% HACKING SIMULSPACES
% Since simulspaces are complex virtual environments
% and often run on time dilation, hackers cannot hack
% them in a normal manner when they participate in the
% simulation. There are ways to affect and influence the
% simulation from within, but the degree of subversion
% that is achievable is limited. For this reason, hackers
% rarely enter into VR to hack. Hacking into the exter-
% nal system running a simulspace is just like breaking
% into any other system. Use all of the standard rules for
% intrusion and subversion.

% MEDDLING FROM THE INSIDE
% Within a simulspace, a hacker’s only choice for inter-
% acting with the VR controls is through the standard
% interface that any simulmorph can pull up. Typically
% used for standard user features like adjusting your
% simulmorph or chatting with or checking the status

% %%% txt/266.txt


%                          HACKING SIMULS
% MODIFIER TASK

% –0     Analyze simulation parameters, view domain rules, shap

% –10     Change probability of test outcomes, become invisible (


%      Interfere with simulation (e.g. make it rain, generate ear

% –20


%      simulmorphs

% –30     Go into god mode, command simulated characters, take


%  of other users, a clever hacker might find some ways
%  to subvert the system. Such options are usually lim-
%  ited, however, as a number of system controls and
%  processes cannot be accessed and manipulated from
%  the inside.

%  Most of the hacker’s options are going to involve
%  meddling with the simulation and its specific domain
%  rules or possibly gaining access to cheats. To make a
%  change requires a successful Interface Test. Ultimately
%  the gamemaster decides what the hacker can and
%  cannot get away with, based on the limitations of that
%  particular simulspace.

%  Most simulspaces are monitored to prevent cheat-
%  ing and abuse, though the monitors are typically pre-
%  occupied with maintaining the simulspace as a whole,
%  dealing with other users, etc. At the gamemaster’s
%  discretion, such a monitor might get to make an Inter-
%  face Test (possibly with a modifier for distraction) to
%  notice the hacker’s efforts.



%  AIS AND MUSES
% AIs are sentient but specialized programs. Like other
% software, they must be run on a computerized system.
% Most AIs are run on bots, vehicles, and other com-
% puterized devices where they can assist transhuman
% users or operate the machine themselves. They are also
% commonly used to actively monitor computer systems
% against intrusion attempts. Muses are AIs that special-
% ize as personal companions, always at a character’s
% virtual side every since they were a child.

% Sample AIs and muses can be found on p. 331 of Gear.

%  AI LIMITATIONS
% AIs feature a number of built-in restrictions and
% limitations. To start with, they can be loaded in the
% cyberbrains of pods and synthmorphs, but they may
% not be downloaded into biomorph brains. As software,
% they use the same rules as other software and may be
% shut down, restarted, copied, erased, stored as inert
% data, infected with virii, and reprogrammed. Due to
% their size and complexity, only one AI (or infomorph)
% may be run on a personal computer at a time (see
% Computer Capabilities, p. 247), and they may not run
% on peripheral devices.

% While they possess cognition and intelligence, they
% are incapable of self-improvement and cannot expand
% their programming and skills on their own. Although
% ACE FROM WITHIN

% pearance of simulmorph, switch simulmorph character or morph type
% -game”) to others
% akes), generate items, ignore domain rules, kill or lockout other

% r the simulation



% they are not able to learn they do possess memory

% storage that grants them the ability to remember and

% a limited form of adaptation. AIs do not earn Rez

% Points, nor do they have Moxie.


% AIs have aptitudes no greater than 20 but are

% incapable of defaulting. If they don’t possess a skill,

% they don’t know how to do it. (At the gamemaster’s

% discretion, they may default to field skills or similar

% skills as noted on p. 173 with a –10 to –30 modifier).

% They can use skills like any character in Eclipse Phase,

% however they may not possess any Active skill at a

% rating higher than 40 or Knowledge skill higher than

% 90—the maximum amount of expertise that their skill

% software allows.


% While AIs are programmed with personality tem-

% plates and empathy, they are generally less emotive

% and difficult to read (apply a –30 modifier to Kinesics

% Test made against them, when in pod bodies). When

% combined with non-expressive synthetic morphs, they

% are even more difficult (–60 modifier). Some AIs lack

% emotive capabilities altogether and are impossible to

% read with Kinesics skill.


% AIs do have a Lucidity and Trauma Threshold stat,

% and are capable of suffering mental stress and traumas.


% COMMANDING AIS

% AIs and muses are programmed to accept commands

% from authorized users. In some circumstances, they

% may also be programmed to follow the law or some

% ethical code. Programming is never perfect, however,

% and AIs can be quite clever in how they interpret

% commands and act on them. In most cases, an AI will

% rarely refuse to follow a request or obey a command.

% Given that they also usually have a duty to protect the

% person commanding them, the AI may be reluctant to

% follow commands that could be construed as danger-

% ous or having a negative impact on the user. Under

% certain circumstances, preprogrammed imperatives

% can force an AI to ignore or disobey their owner’s

% commands (gamemaster’s discretion).




% AGIS AND INFOMORPHS

% The term “infomorph” is used to refer to any ego in

% digital body, whether that be an AGI or the digital

% emulation of a biological mind (including backups

% and forks). The following rules apply to infomorph

% and AGI characters.

% %%% txt/267.txt
% ROLEPLAYING MUSES
% Muses should not be viewed as a mere tool for
% getting extra skills, but as an opportunity to
% enhance roleplaying. Though typical muse AIs
% are not complete intelligences (though they can
% be, see Infomorphs as Muses), their personality
% matrix is often quite sophisticated and they are
% very good at adapting to their user’s personality
% quirks. On the other hand, they share the same
% Real World Naiveté (p. 151) as AGI characters
% when it comes to understanding all the facets
% of transhuman behavior, social interaction,
% body language, or emotion. Their personalities
% are more non-human, abstract, alien, and less
% passionate than transhuman life forms, often
% leading to conceptual misunderstandings and
% miscommunications. Likewise, their creative
% capacities are limited, instead bolstered by an
% ability to calculate odds, run simulations and
% evaluate outcomes, and make predictions based
% on previous experiences.

% Depending on the user’s stance towards sen-
% tient programs, muses can be viewed as intel-
% ligent toys, followers, servants, slaves, friends,
% or pets, which should somehow be reflected in
% game play. Most transhumans have also acquired
% a tendency to bond with a muse mentally due to
% its omnipresence and devotion to the user (like
% bonding to a child or puppy that then grows to
% be an adult). Therefore the subversion or even
% destruction of a muse personality is sometimes
% even equated with rape or murder.              ■





% SOFTWARE MINDS
%  At their core, infomorphs are just programs and so
%  they are treated like other software in terms of rules.
%  They must be run on a specific personal computer or
%  server (see Computer Capabilities, p. 247). If that
%  device is shut down, the infomorph also shuts down
%  into a state of unconsciousness, restarting along with
%  the device (infomorphs may also shut themselves
%  down, though it is rare that they do so). If the device is
%  destroyed, the infomorph is killed along with it (unless
%  their data can somehow be extracted from any surviv-
%  ing components, perhaps resulting in a vapor, p. 274).
%  Infomorphs may copy themselves, though in some
%  places this is illegal and in most places is frowned
%  upon as it raises numerous ethical and legal questions.
%  For this reason many infomorphs that copy and trans-
%  fer themself to run on a new device will thoroughly
%  erase themselves off the old one.

% As digital beings, infomorphs have no physical
%  mind, but it is a simple matter for them to possess
%  an uninhabited synthmorph, taking up residence in
%  the cyberbrain (see Resleeving Synthmorphs, p. 271).

% INFOMORPHS AS

% MUSES

% Instead of relying on underdeveloped muses for

% aid and companionship, characters may prefer to

% have a full-fledged digital intelligence at their

% side, whether that be an AGI, a backed-up bio-

% logical ego, or fork of the character’s own per-

% sonality. Alternately, a character with a ghostrider

% module (p. 307) could have both, carrying a muse

% in their mesh inserts and an infomorph in the

% ghostrider module.

%  This possibility is very useful for infomorph

% player characters, as they can ride along in some-

% one’s head and participate in team affairs with-

% out needing a morph of their own.                 ■



% They may also download into biomorph bodies ac-
% cording to standard resleeving rules (p. 271). Even
% when disembodied, they may interact with the
% physical world via the mesh, viewing through sensors,
% streaming XP feeds, communicating with characters,
% commanding slaved devices, and teleoperating/jam-
% ming drones.

% Infomorphs have a Speed of 3, reflecting their digi-
% tal nature and their ability to act at electronic speeds.
% If an infomorph sleeves into a body, however, it takes
% on the Speed of that morph.

% AGI CHARACTERS
% Though AGIs were not born in a biological body, their
% programming encompasses the full spectrum of human
% personality, outlook, emotions, and mental states. AGIs
% are in fact raised in a manner similar to human chil-
% dren, so that they are socialized much like humans are.
% Nevertheless, on a fundamental level they are non-hu-
% mans programmed to act human. There are inevitably
% points where the programming does not mask or alter
% the fact that AGIs often possess or develop personality
% traits and idiosyncrasies that are quite different from
% human norms and often outright alien.

% Unlike standard crippled AIs, AGIs are capable of
% full-fledged creativity, learning, and self-improvement
% (at a slow but steady pace equivalent to humans). Just
% like other characters, they earn Rez points and may
% improve their skills and capabilities. AGIs suffer none
% of the skill limitations placed on weak AIs, using skills
% just like any other character.

% On an emotional level, AGIs run emotional subrou-
% tines that are comparable to biological human emo-
% tions. AGIs are, in fact, programmed to have empathy
% and share an interest in human affairs and prosperity,
% and to place significant relevance on life of all kinds.
% In game terms, AGIs emote like humans (and so Kine-
% sics may be used against them) and are vulnerable to
% emotionally manipulative effects, fear, etc.

% %%% txt/268.txt
% ND UPLOADING
% egos) can be digitally emulated and backed up
% ge—a process known as uploading. This allows
% if they are killed. ■ p. 268






%                                                    RESLE


%                  Uploaded egos may also be downloaded into a


%                                  a process known as resleevin






%                                              Integration, Al


%                                       Adapting to a new mor




% 10 ACCELER
%  ING
% w body,
%  p. 271




%  tion, and Continuity: Resleeving is not easy.
% he loss of continuity, and/or the remembrance

% of death can inflict mental stress. ■ p. 272




% ATED FUTURE

% %%% txt/269.txt


%                                FORKING AND


%                            Forking is the process of mak


%                              of your ego, often for mult


%                       Forks may be re-integrated back in






%                          REPUTATION AND


%                          In the outer system, the reputa


%                          inner system, rep is a measure


%                          peers. ■ p. 285






%   NANOFABRICATION

% Material goods can be manufactured

% from the molecular level up via nano-
% fabricaton, requiring only raw materials,


%      blueprints, and time. ■ p. 284


%              EGOCASTING


%              Rather than physically travel, most transhumans


%              upload their ego, farcast it to a distant location,
% ERGING            then resleeve (or run as an informorph) at the
% digital copies    destination. ■ p. 276
% ing purposes.
% our ego later.

%  ■ p. 273



% OCIAL NETWORKS
%  economy rules, and in the
% our social standing with your






%                                   LIVING IN THE FU


%       Technology has changed many other features of fundame






%                       Identity: Your body no longer defines w



%                   Life In Space: Habitats, immigration, and s



%   Security: New challenges arise to keeping people out—or b

% %%% txt/270.txt
% ■GEAR■GEAR■GEAR■GEAR■GEAR■G



% ERATED FUTURE
%  The future setting of Eclipse Phase introduces a
%  number of technological elements that have a strong
%  impact on transhuman society. These include backups
%  and uploading, resleeving, egocasting, forking, nano-
%  fabrication, reputation systems, space habitats, and
%  space travel, among others.



%  BACKUPS AND UPLOADING
%  The transhuman mind is no longer a prisoner of the
%  biological hardware on which it originates. Through
%  various mechanisms, biological brains may be digitally
%  emulated, allowing people to make a backup of their
%  minds, including their entire personality, memories,
%  and skills—a process known as uploading.

% The primary use of backups is to ensure the person’s
%  ego can be retrieved in case of death, in which case
%  they may be resleeved (p. 271). For this reason, almost
%  everyone in the solar system is equipped with a corti-
%  cal stack (p. 300). Backups may also be safely archived
%  in secure storage (p. 269) or used to create infomorphs
%  (p. 264). A person may also egocast themselves across
%  the solar system as a form of travel (p. 276).

%  CORTICAL STACK BACKUPS
%  Cortical stack implants deploy a network of nano-
%  bots throughout the brain that take a snapshot of
%  the mind’s neural state, storing the data as a backup
%  within the cortical stack. The average transhuman’s
%  cortical stack backs up their ego 86,400 times per day.
%  Only the most recent backup is kept within the stack;
%  older ones are overwritten. Pods and synthmorphs
%  also can be equipped with cortical stacks (though
%  AI-piloted bots often lack this feature), though these
%  versions maintain an updated copy of the ego running
%  in the morph’s cyberbrain.

%  In the case of death, accidental or otherwise, a cor-
%  tical stack can be retrieved from a corpse and used to
%  recover the character, either as an infomorph or by
%  resleeving them in a new morph. Cortical stacks are
%  diamond-hardened and protected, so they may often
%  be retrieved even if the corpse is badly mangled or
%  damaged. If the corpse cannot be recovered or the
%  cortical stack is destroyed, the backup is lost.

%  High rollers, well-equipped brinkers, and others
%  in dangerous professions often opt for an emergency
%  farcaster accessory (p. 306) that periodically (usually
%  every 48 hours, but varying according to contract)
%  transmits a backup from the cortical stack to a remote
%  storage facility. This option is quite expensive, how-
%  ever, and so is generally only afforded by the wealthy.

%  RETRIEVING A CORTICAL STACK
%  Most cortical stacks are carefully excised from a
%  corpse with surgery. In certain circumstances, however,
% AR■GEAR■GEAR■GEAR■GEAR■GEAR■


%                  ■GEAR■


%                   GEAR■



%                                                            10



% a character may need to extract a cortical stack in the

% field, whether because transporting the corpse is im-

% practical or because the dead person is an enemy and

% they either don’t want them knowing who killed them

% or they want to interrogate them with psychosurgery

% in a simulspace.


% The process of cutting out a cortical stack is called
%  “popping,” as a skilled extractor can usually get the

% smooth-shelled implant to pop right out by making

% an incision in the correct place and applying pressure.

% One does need to be careful that the tiny, blood-slick

% stack doesn’t slip away once popped.


% Popping can be done with a sharp knife and elbow

% grease, though it is grisly. Popping a stack is a Task
%  Action that requires a Medicine: [any appropriate

% field] Test with a timeframe of 1 minute and a modi-

% fier of +20. Morphs with stacks in non-standard loca-

% tions or with anatomical shielding (carapace plates,

% etc.) around the stack may incur penalties to this test

% at the gamemaster’s discretion. Of course, if you don’t

% have the time for a precise extraction, you can always

% just cut the entire head off and take it with you.


% Once a cortical stack is retrieved, it may be loaded

% into an ego bridge (p. 328) and used to bring the ego

% back, either as an infomorph or by resleeving.


% Living Subjects: Cortical stacks may be excised

% from living people, but the process is usually fatal (or

% at least paralyzing) as it involves cutting through the

% spinal column. If the target is not unconscious or oth-

% erwise incapacitated, they must first be immobilized

% in melee combat (see Subdual, p. 204). Cutting out

% the stack is handled like a Medicine Task Action as

% above, but this process inflicts 3d10 + 10 damage on

% the target. If the test fails, they still inflict 1d10 + 10

% damage to the target. If the person removing the stack
%  wants to leave the target alive or harm them as little

% as possible, they suffer a –20 modifier on the test, but

% may reduce the damage by 1d10 per 10 full points of

% MoS. Living through the process of having your stack

% removed is traumatic; anyone who does so suffers
%  1d10 mental stress.


% DESTROYING A CORTICAL STACK

% Cortical stacks have an Armor of 20 and a Durability

% of 20 for anyone attempting to destroy them.


% UPLOADING

% Uploading a backup into secure storage is usually

% handled with a brain scan at the storage facility’s clinic

% using a bread box-sized unit called an ego bridge (p.

% 328). When activated, the ego bridge’s sensor array

% twists open like a morning glory blossom, revealing an

% enclosure with a neck rest that automatically adjusts

% itself to morphs with oddly-sized or -shaped heads.

% The neck rest deploys millions of specialized nanobots

% %%% txt/271.txt
% into the brain and central nervous system. The petals
% are full of sensors that image the brain using a combi-
% nation of MRI, sonogram, and positional information
% broadcast by the nanobot swarm in the morph’s brain.
% The ego bridge then builds a digital copy of the per-
% son’s brain, which is stored away in the service’s highly
% secure, off-the-mesh, hardwired data vaults.

% In the case of pods, the ego bridge scans the biologi-
% cal brain bits and also accesses the cyberbrain to copy
% the parts of the ego residing there. For synthmorphs,
% who have no biological brain, the process is much
% simpler, as it simply requires accessing and making a
% copy of their cyberbrain.

% In a standard clinic with an undamaged morph, up-
% loading takes only 10 minutes, 5 with a pod. In other
% situations, however, the process may take longer if the
% gamemaster so decides. Uploading from a synthmorph
% or extracted cortical stack is instantaneous. The ego
% bridge largely operates itself. While oversight by a
% medical specialist is a good idea, no test is necessary.

% If an uploading character does not plan to return
% to their morph, it is usually put on ice until someone
% else resleeves into it. If a new resleeve is not ready
% and the uploading character doesn’t want to leave a
% potential copy of themselves behind, they can have
% the morph’s mind wiped by the nanobots as part of
% the uploading process.

% UPLOADING-RESLEEVING CONTINUITY
% In ideal circumstances, a person who is intentionally
% resleeving (p. 271) can arrange for the uploading and
% resleeving process to occur with any noticeable loss of
% continuity. Though the experience of switching from
% one morph to another is still a bit jarring, the transi-
% tion itself can be made into a seamless process, with
% no gaps in awareness or memory, which helps reduce
% associated mental stress.

% In this case, during the process of uploading, the
% ego bridge is also connected to another ego bridge
% and the new sleeve. This connection can even be made
% wirelessly or by farcaster link (with a maximum dis-
% tance of 10,000 kilometers).

% As the mind is uploaded, the ego bridge builds a
% virtual brain by copying the morph’s brain bit by bit,
% using the data gained from the brain scan. At the same
% time, this data is slowly copied to the new sleeve as
% nanobots rewire the sleeve’s brain structure (a much
% slower process). As the transfer occurs, the nanobots
% in the brain sever individual neural connections and
% re-route them to their duplicates in the virtual brain,
% and then eventually to the new brain. Effectively, the
% character’s ego is running partially on the meat brain
% and partially on the virtual copy. By the time the nano-
% bots sever the last of the neural connections in the
% old brain, the ego is running completely on the virtual
% brain and the new sleeve’s brain. Once the resleeving
% is completed, the virtual brain is shut down.

% In terms of perceptions, the character, who is awake
% during this process, experiences a very gradual shift
% from one morph to the other. As the process takes
% hours, however (or even longer if done via farcaster),
% the subject usually entertains themselves with some
% AR media, VR, or even XP to pass the time.

% UPLOADING AFTER DEATH
% It is possible to upload the mind of a person who has
% recently died as long as the nanobots have time to scan
% the brain before cell deterioration kicks in too heavily,
% which takes approximately 2 hours. It is possible to
% sustain a corpse for longer by placing it in a healing
% vat (p. 326) for nanostasis. Post-death uploads may
% suffer integrity damage; see Backup Complications, p.
% 270.

% Cyberbrains may also be retrieved from a destroyed
% synthmorph and reactivated, assuming they are not
% damaged too heavily (gamemaster discretion).

% DESTRUCTIVE UPLOADING
% Though rare, some people engage in a process called
% destructive uploading, where the biological brain is liter-
% ally sliced apart and scanned piece by piece. Considered
% abhorrent and wasteful by most transhumans, “brain-
% peeling” is practiced by some bioconservative factions
% who view it as the only “pure” method of uploading
% or the only real way to transfer the “soul.” Such people
% typically refuse to resleeve, living out the rest of their
% lives as infomorphs, quite often in dedicated simulspaces
% that are treated as a sort of virtual afterlife.

% BACKUP INSURANCE
% Almost everyone, with the exception of neo-primitivists
% and very young children, has a cortical stack. In the
% event of death, however, a cortical stack alone will not
% ensure resurrection unless you have acquired backup
% insurance (p. 330) to cover the costs of your resleev-
% ing. Going without backup insurance for any length of
% time is taking a severe risk. Some jurisdictions (such
% as the Titanian Commonwealth) have a practice of
% bringing everyone back, even if only to an infomorph
% state, or at least filing the most recent backup away
% in dead storage just in case someone decides to pay
% to resurrect them later. Other authorities will simply
% destroy the stack or, worse, sell it on the black market
% to a soul-trading syndicate such as Nine Lives.

% Backup insurance typically includes a subscription
% to an uploading facility, usually requiring a visit every
% 6 months, to ensure that backup is held in safe stor-
% age in case of cortical stack loss. People with risky
% jobs (construction bot supervisor, hypercorp exo-
% planet staff, girl who fights vicious giant eels for rich
% jaded audiences, etc.) may back up once a week, or
% even daily. In the event of a verified death where the
% cortical stack could not be retrieved, the most recent
% backup is used to resleeve the person.

% At the basic level, backup insurance will bring the
% character back as an infomorph, at which point they
% can access their credit and purchase a new morph.
% More expensive versions will automatically resleeve

% %%% txt/272.txt
% you in the pre-purchased morph of your choice. The
% exceedingly rich will often have customized clones
% (often of their original body) waiting on ice for them.

% Backup insurance often involves a missing person
% clause, which states that a person will be brought
% back if they have not checked in for X amount of time
% (a calendar function automatically handled by your
% muse) and cannot be located.

% It is worth noting that some criminal syndicates
% also offer backup insurance at a much reduced rate.
% The likelihood that copies of your backup are being
% used for illicit purposes, however, is quite high. For
% some people, however, what happens to a copy of
% themselves is of no concern.

% BACKUP INSURANCE LIMITATIONS
% Backup insurance is not always perfect. Though in-
% surance providers are required to make a reasonable
% effort to retrieve your cortical stack, for many hyper-
% corps this is a simple cost-benefit analysis that often
% will not work in the character’s favor. If you died in a
% dangerous area such as the Zone on Mars, in a remote
% area such as the Kuiper Belt, or are simply difficult
% to track down (pushed out an airlock somewhere),
% odds are against your cortical stack being retrieved—
% instead you will be re-instanced from a backup.

% Jurisdiction can also play an important role. The
% insurance offered by many inner system providers is
% automatically nullified if you travel to an anarchist
% habitat, gatecrash, break the law, or engage in certain
% life-threatening activities like suicide sports or scaveng-
% ing in TITAN-infected ruins. At the least, they will
% refuse to retrieve your stack in these circumstances.
% Likewise, if you struck a backup insurance deal with
% a medical collective from an autonomist habitat and
% then go and die on a hypercorp station, the hypercorp
% is very likely to refuse to recognize the authority of a
% bunch of anarchists and won’t hand your stack over.

% Even an archived backup and a missing person
% clause is no guarantee. A determined enemy could
% capture you, pry the backup insurance access codes
% from your muse, keep you on ice or quietly kill you,
% and then regularly “check in” on your behalf using
% the access codes so that the insurance provider never
% realizes you are dead or missing. Though this requires
% quite a bit of effort, it is often less difficult than deal-
% ing with an immortal opponent who keeps coming
% back no matter how often you kill them.

% Other dangers also exist. An entire habitat may be
% destroyed, taking you, your backups, and your insur-
% ance provider’s records with it. A resourceful enemy
% might penetrate a provider’s security and delete your
% backups, or simply bribe the right people to make
% sure they get “accidentally” corrupted. Given these
% possibilities, the paranoid often make sure to get
% multiple redundant backup policies, assuming they
% can afford it.

% BACKUP COMPLICATIONS
% In most cases, backing up/uploading is risk free unless
% someone tampers with the equipment. If the character
% suffered brain or neurological damage, the backup is
% transferred via farcasting, or the upload is made from
% a dead character, then the backup may be damaged
% due to missing neural information. In any of these
% instances, make a LUC Test for the character. If the
% test fails, they suffer 1 point of mental stress per 10
% full points of MoF. Note that this stress (and pos-
% sible) trauma applies to the backup, not the original
% character. If the backup is used to re-instantiate the
% character, however, then the stress is applied.



% RESLEEVING
% Resleeving (also called remorphing) is the process
% of giving a new body to an ego. Changing bodies
% is a normal part of life for hundreds of millions of
% transhumans, and it is an even more frequent occur-
% rence for people in certain professions. Characters
% involved in specialized work may resleeve as often as
% once a month. Those who travel frequently may do so
% even more often. Also, given the number of infugees

% %%% txt/273.txt

% who died during the Fall but have now acquired a

% new morph, the vast majority of transhumanity has

% resleeved at least once. As such, most transhumans are

% accustomed to resleeving.

%  Adjusting to a new body takes time and a bit of

% effort (see Integration, p. 272). Resleeving is also diffi-

% cult psychologically, as reflected by continuity (p. 272)

% and alienation (p. 272).

%  Once an ego fully inhabits a new morph, the new

% morph’s cortical stack needs ten minutes to amass a

% complete backup of the ego.


% RESLEEVING BIOMORPHS AND PODS

% Resleeving takes about an hour in a properly equipped

% clinic. In essence, the process works like uploading in

% reverse. The new sleeve is hooked up to an ego bridge

% which infiltrates the brain with nanobots that physi-

% cally restructure the brain’s neural structure and con-

% nections according to the map provided by the backup.

% Sleeving takes six times as long as uploading because

% the nanobot swarm working as a wet printer in the

% template brain needs to duplicate the entire physical

% structure of the ego’s neural network. For resleeving, a
%  “wet” ego bridge is used, meaning that the sleeve and

% ego bridge are submerged in a vat filled with nanogel.


% The resleeving process for pods takes only half an hour,

% as pods brains are half biological and half cyberbrain.


% RESLEEVING SYNTHMORPHS

% Resleeving into the cyberbrain of a synthmorph is much

% easier and quicker, being a matter of copying the backup

% into the cyberbrain (an instantaneous affair) and then

% running the backup in its virtual brain state (1 Action

% Turn). The drawback to synthmorphs is that they are

% more difficult to acclimate to (see Integration, p. 272),




% RESLEEVING AND THE GA
% The gamemaster has a fine amount of control over
% characters may be supplied with new morphs by Firew
% currently working. In this case, the gamemaster can si
% plete control over enhancements, traits, etc. While mo
% presents an opportunity for the gamemaster to throw
% some new challenges to overcome. Gamemasters are
% something they can work with without necessarily giv

% In other cases, the availability of desired morphs
% outpost in the wilds of Mars is unlikely to have a w
% synthmorphs may be all they have. Similarly, large hab
% may be a waiting list for top-of-the-line sylphs or rem
% morphs are going to be subject to local legalities, so g
% Characters could always turn to black market morph p

% What this means is that gamemasters should never
% that is unreasonable or potentially disruptive to the ga
% want once in a while, it also makes for more interest
% a little different than what they were hoping for or t
% physical addiction. For extra fun, leave the character u
% until they reveal themselves. As always, the goal is to h

% they are vulnerable to cyberbrain hacking (p. 261), and

% synthmorphs are viewed as low class in some cultures.


% EVACUATING A CYBERBRAIN

% Characters inhabiting a synthmorph cyberbrain may

% voluntarily choose to evacuate by copying themselves

% as an infomorph onto another device. This takes 1 full

% Action Turn. See Infomorph Resleeving, p. 273.


%  RESLEEVING COSTS

% The costs involved for the resleeving process itself are

% generally subsumed in the costs of the backup insur-

% ance and/or the new sleeve itself. Costs for individual

% morphs are noted in the descriptions starting on p.

% 139. See Morph Brokerage (p. 276) for rules on find-

% ing and acquiring morphs.


%  INTEGRATION

%  Getting used to a new body typically takes some time.

% The character must become acclimated to the changes

%  in height, weight, sex, and capabilities, which often

%  requires unlearning ways of doing things that worked

%  fine for their previous form. Resleeving in a synthetic

%  morph or an uplift is also quite confusing at first,

%  given the drastically different morphologies, change in

%  limb structure (and sometimes amount of limbs), and

%  so on. Luckily, transhuman minds are adaptive things,

%  and this process is aided by the application of mental

% “patches” during the resleeving process that give the

%  character a bit of a boost for using their new body.


%   An ego in a new morph makes an Integration Test

%  upon taking control of the body, rolling SOM x 3 (morph

%  bonuses do not apply) and applying modifiers from the

%  Integration and Alienation Modifiers table. The result of

%  the test is explained on the Integration Test table, p. 272.




% MEMASTER
%  at a character can obtain when resleeving. The

% or whatever employer/patron for whom they are

% y assign whatever morphs they see fit—with com-
% hs should be tailored for the mission at hand, this
%  e characters some new toys to play with and also
% couraged to mix it up, have fun, and give players

% them everything they want.
%  y be limited by the resleeving location. A small

% selection of morphs—in fact, a few rusters and
%  ts have a high demand for good morphs, so there

% morphs, for example. In the same vein, available
%  ng that reaper morph may be out of the question.
%  iders, but these come with their own risks.
% afraid to say no if a character is pursuing a morph

% While it’s good form to give the players what they
%  roleplaying to saddle them with morphs that are

% come with some interesting challenges, such as a
% ware of a morph’s negative traits or secret implants

% fun, but variety often helps with that.        ■

% %%% txt/274.txt

%  INTEGRATION TEST


%        EFFECT


%        Character is unable to acclimate to the new morph—


%        something is just not right. Character suffers a –30 modi-


%        fier to all physical actions until resleeved.


%        Character has serious trouble acclimating to the new
% 0+)         morph. They suffer a –10 modifier to all actions for 2 days


%        plus 1 day per 10 full points of MoF.


%        Character has some trouble acclimating to new morph.


%        They suffer a –10 modifier to all physical actions for 2


%        days plus 1 day per 10 full points of MoF.


%        Standard acclimation period. The character suffers a –10


%        modifier to all physical actions for 1 day.


%        No ill effects. Character acclimates to new morph in no
%  S 30+)


%        more than a few minutes.


%        Lookin’ good! This morph is an exceptionally good fit for


%        the character. No ill effects; gain 1 Moxie point for use in


%        that game session only.



% ION AND ALIENATION MODIFIERS


%                                                         EFFECT
% s used this exact morph extensively in the past                +30


%                                                           +20
%  orph type (what they were raised with)                        +20
% el 2)                                                          +20
% el 1)                                                          +10
%  sly used this type of morph                                   +10


%                                                           –10
% eeving into a physical body                                    –10
% resleeving in a non-uplift (of their type) body                –10


%                                                           –10
%  morph)                                                        –10
% ified                                                           –10
% ait (Level 1)                                                  –10
% ait (Level 2)                                                  –20
% apply to AGIs) (Alienation Test only)                          –20
% only)                                                          –20
% ait (Level 3)                                                  –30
% orph, neo-avian, novacrab, swarmanoid, etc.)                   –30






%    ALIENATION TEST


%                             EFFECT
% Extreme Dysmorphia. The character doesn’t like their new sleeve at all
% and suffers 2 stress points per 10 full points of MoF.
% Character is uneasy about the new morph and suffers 1 stress point
% per 10 full points of MoF.
% Character adapts to their new look well. No ill effects.
% Best. Morph. Ever. The new morph jives perfectly with the character’s
%  ense of self, and even enhances it somewhat. The character actually
% heals 1d10 ÷ 2 (round up) stress points.
% ALIENATION
% After loss of continuity, the other major factor impact-
% ing resleeving characters is alienation. Once the ego
% has its new sleeve under control, it’s time to look in
% the mirror. The alienation test reflects the experience of
% coming to terms with a new face, skin, and brain. For
% example, transferring to a radically different morph
% (such as a swarmanoid) can be difficult to grasp.
% Uplifts often have difficulty getting acquainted with
% the differing hormonal urges of a human biomorph
% and vice versa. While the character’s ego is as it was
% in their last sleeve, the brains and neurochemistry of
% many morphs may alter aptitudes like WIL or COG.
% The effects of this can be frustrating or disorienting.

% Every character makes an Alienation Test to reflect
% how mentally stressful it is to get a grip on their new
% body, rolling INT x 3 and apply modifiers from the
% Integration and Alienation Modifiers table. Consult
% the Alienation Test table to determine the effects.

% CONTINUITY TEST
% Perhaps the biggest shock that strikes most resleeving
% characters is the loss of continuity of self. This is par-
% ticularly true for characters who died. If their cortical
% stack was retrieved, they will remember their own
% death. If they were restored from an archived backup,
% they will not remember their death, but they will have
% lost an entire period of their life—all the way back to
% their last backup. In fact, if their body was not recov-
% ered, they may not even know that they are dead for
% certain—there may be a surviving copy of themselves
% out there. The driving point in this loss of continuity is
% a sort of existential crisis—they are no longer the origi-
% nal person they once were. This leads some to question
% whether they are who they think they are, or are they
% some poor imitation and not a real person at all?

% To determine how this loss of continuity affects a
% character, make a Continuity Test by rolling WIL x 3.
% Every character suffers stress from loss of continuity,
% as noted on the Continuity Stress table. Reduce this
% stress damage by 1 point per 10 full points of MoS on
% the Continuity Test, or increase it by 1 point for every
% 10 full points of MoF.

% INFOMORPH RESLEEVING
% Rather than resleeving into a physical body, a backup
% may instead by instantiated as an infomorph, a purely
% digital form. Infomorphs are distinct from backups in
% that backups are inert files. Infomorphs are backups
% imprinted onto a virtual brain template and run as
% a program. This virtual brain state must be run on a
% specific device and follows all of the rules noted for
% infomorphs on p. 264. Infomorphs may copy them-
% selves to other devices, typically erasing themselves
% from the previous device as they go. Infomorphs that
% copy without erasing are treated as forks.

% Characters instantiating as infomorphs must make
% Continuity and Alienation Tests, just like resleeving.

% Infomorphs may be resleeved into physical morphs,
% following normal resleeving rules.

% %%% txt/275.txt


%     I wake up with a taste like guava and umami


%     fresh on my tongue. Last night there was laugh-


%     ter. We drank quinoa wine, and I was introduced


%     to people I had never met before, though I had


%     years of intimate knowledge of most of them.


%     Half of Illyria Module is curled naked around


%     me in my sleeping chamber. Last night we


%     made music with synthesizers, wood blocks,


%     and a lur. We drank mushroom tea brewed in


%     water from a rogue comet. Looking around me


%     as the morning sun starts to light the far orbital


%     horizon of Ceres, it appears we had an orgy.


%     Last night was my resleeving party. This version


%     of me—me 3.0—is ready for life.


%     —Zheng du Thierry, Carnival of the Goat




% FORKING AND MERGING
% With all of these backups of transhuman minds on
% file and an abundance of mesh space on which to run
% them as virtual brains, one might wonder what’s to
% stop post-Fall transhumanity from multiplying its
% numbers by running additional copies of them. The
% short answer is: nothing, aside from massive social
% stigma and thorny psychological issues. Taking a
% backup of a transhuman mind, copying it, and re-
% instancing it as an infomorph is called forking. It’s one
% of the most useful and still-controversial applications
% of transhumanity’s brain science.

% There are four classifications of forks: alpha, beta,
% delta, and gamma. Though typically copied as info-
% morphs, there is nothing preventing a fork from being
% sleeved in a physical morph as well, other than legali-
% ties and custom.

% ALPHA FORKS
% An alpha fork is an exact copy of the original ego,
% re-instanced as a separate infomorph. An alpha fork
% may be created by copying and running an infomorph
% (from a backup, infomorph, synthmorph cyberbrain,
% or directly from the cortical stack). Alpha forks are an
% exact copy of the character’s ego, with all of the same
% skills, memories, stats, traits, personality, etc. New
% alpha forks must make an Alienation Test (p. 272),
% and possibly a Continuity Test (p. 272) if copied from
% a backup.

% Creating alpha forks is illegal in many jurisdic-
% tions, including most of the inner system and the
% Jovian Republic. In others it tends to be viewed with
% distaste, though there are some habitats/cultures in
% which it is encouraged.

% BETA FORKS
% Beta forks are partial copies of the ego. They are
% intentionally hobbled so as to not to be considered
% an equal to the character, for legal and other reasons.
% Beta forks have most of the same skills as the original
% ego, though sometimes reduced. Their memories are


%                    CONTINUITY STRESS
% SITUATION                                               STRE
% Backup from cortical stack

% Character remembers peaceful or not notable death   1d10 ÷

% Character remembers sudden or violent death
% Backup from archive

% Short memory gap (less than 1 day)                  1d10 ÷

% Memory gap greater than one day

% Not knowing if/how you died

% Uploading-to-resleeve with continuity (p. 269)

% Uploading-to-resleeve without continuity            1d10 ÷

% Character is a fork


%  also drastically curtailed, usually tailored to whatever
%  task they are intended to perform.

% Beta forks are created by a process known as
%  neural pruning (p. 274). They are legal and even
%  common in many places, except for bioconservative
%  holdouts like the Jovian Republic, though delta forks
%  are more favored. Beta forks rarely have anything
%  resembling civil rights or citizenship and are usually
%  treated as the property of the originating ego. They
%  are commonly used as digital aids or to represent the
%  original ego when communicating with others over
%  great distances.

% A beta fork’s stats are determined as follows:


% • Reduce all aptitudes by 5 (to a minimum of 1).

%  This affects all skills as well. Likewise, this re-

%  duces LUC by 10 and INIT by 20.

% • Active skills have a maximum value of 60.

% • Moxie is reduced to 1.

% • The Psi trait is removed. At the gamemaster’s dis-

%  cretion, other traits may no longer apply as well.


% Additional changes may apply as determined by
%  the neural pruning test. Beta forks take 1 minute
%  to generate.

%  DELTA FORKS
% Delta forks are extremely limited copies of an ego.
% They are more akin to AI templates upon which
% the ego’s surface personality traits are imprinted.
% Also created via neural pruning, delta forks are
% highly functional (as competent as a beta fork or
% AI) but have extremely limited skills and heav-
% ily edited memories, usually to the point of being
% functional amnesiacs.

% A delta fork’s stats are determined as follows:


% • Reduce all aptitudes by 10 (to a minimum of 1).

%  This affects all skills as well. Likewise, this re-

%  duces LUC by 20 and INIT by 40.

% • Active skills have a maximum value of 40. The

%  fork may have no more than 5 Active skills.

% %%% txt/276.txt
%  • Knowledge skills have a maximum of 80. The

% fork may have no more than 5 Knowledge skills.
%  • Moxie is reduced to 0.
%  • The Psi trait is removed. At the gamemaster’s dis-

% cretion, other traits may no longer apply as well.


% Additional changes may apply as determined by the
% neural pruning test. Delta forks take 1 Action Turn to
% generate.

% GAMMA FORKS
% More commonly known as vapors, gamma forks are
% massively incomplete, corrupted, or heavily damaged
% copies of an ego. Vapors are not intentionally cre-
% ated and are instead the results of botched uploads,
% scrambled backups, incomplete or jammed farcasts,
% or infomorphs/forks that were somehow damaged or
% went insane. It is extremely rare for anyone to pur-
% posely create a vapor for anything other than research
% use, although they can crop up in some interesting
% places. For example, poorly made skill software occa-
% sionally includes enough of the personality traits and
% memories of the person the skill was taken from that
% it can behave in a vapor-like fashion when used.

% Because vapors are anomalies rather than purpose-
% ful creations, the characteristics of individual gamma
% forks are left to the gamemaster. They should have
% some or all of the following: reduced skills, reduced
% aptitudes, incomplete or incoherent memories, nega-
% tive mental traits, and persistent mental stress or trau-
% mas, including derangements and/or disorders.

% NEURAL PRUNING
% Most forking is done on the fly—something comes up
% that the character needs a fork for, so they whip one
% up on the spot. Neural pruning is the art of taking a
% backup/infomorph and trimming it down to size so
% that it functions as either a beta or delta fork.

% Beta forks are created by taking a virtual mind state
% that is intentionally inhibited and filtering a copy of
% the ego through it. Like a topiary shrub, the portions
% of the character’s neural network that exceed the ca-
% pacities of the intended fork are trimmed away. In ad-
% dition to the changes noted under Beta Forks (p. 273),
% characters may voluntarily choose to delete/decrease
% skills and remove memories.

% Delta forks are created by excising the ego’s surface
% personality traits and applying them to an AI template.
% In this case the ego’s memories are usually excluded
% entirely—it is easier to start with a blank delta fork
% and feed them the specific memories/knowledge they
% need. As with beta forks, characters making delta
% forks may voluntarily choose to delete/decrease skills
% and keep specific memories.

% Transhumanity’s grasp of neuroscience extends to
% scanning and copying a mind, but the most intricate
% workings of memory are still imperfectly understood.
% Making precise edits to individual portions of a neural
% network (to alter recollections, skills, and the like) is
% still a black art. The difficulty with neural pruning is
% that taking a weed whacker to the tree of memory
% isn’t an exact science. Specific memories may not be
% excised or chosen—at best, memories may be handled
% in broad clumps, typically grouped by time periods no
% finer than 6 months. For simplicity, most beta forks are
% created by removing all memories older than 1 year.

% When creating a beta or delta fork, the character
% must make a Psychosurgery Test (other parties may
% make this test on the character’s behalf, representing
% that the character is giving them access to prune the
% fork appropriately). If the character succeeds, the fork
% is created as desired. If the test fails, the gamemaster
% chooses one of the following penalties for every 10
% full points of MoF. Some of these penalties may be
% combined for a cumulative effect:


% • 1 additional skill decreased by –20

% • Fork acquires a Negative mental trait worth 10 CP

% • Fork suffers 1d10 ÷ 2 (round up) mental stress

% • Extra memory loss (gamemaster discretion; beta

%  forks only)

% • 1 Positive trait lost

% NEURAL PRUNING WITH LONG-TERM PSYCHOSURGERY
% Rather than generating forks on the fly, some char-
% acters prefer to have carefully-pruned forks on hand,
% stored as inert files that can be called up, copied, and
% run as needed. These forks are crafted with long-term
% psychosurgery, meaning that they suffer fewer draw-
% backs and the memories may be more finely tuned.

% Long-term neural pruning requires a Psychosurgery
% Test as above, but with a +30 modifier. Delta forks take
% 1 week to prune this way, beta forks take 1 month. Ad-
% ditional modifications may be made to the fork using
% any of the normal rules for psychosurgery (p. 229).

% It is worth noting that some people prefer to use
% forks of themselves or loved ones rather than a muse.
% Likewise, some wealthy hyperelites are known to keep
% copies of their younger backups on hand, sometimes
% for decades, and re-instance these when their prime
% ego has enough skill and experience to completely
% outclass its younger selves. Though technically these
% are alpha forks, their lag behind the original ego is
% comparable in degree to that of a beta fork. This is
% rumored to be the method used by the Pax Familiae
% in instancing her army of cloned selves.

% HANDLING FORKS
% Gamemasters are encouraged to allow players to ro-
% leplay their character’s own forks. It is important to
% note, however, that even with alpha forks, once the
% fork and originating ego diverge, they develop onward
% as separate people. The events that shape the primary
% ego’s personality, character, and knowledge will not
% happen—or even if they do, probably not in the same
% way—to the fork, and vice versa. The exact dividing
% line between an ego and a fork is a central philosophi-
% cal and legal debate among many transhumans.

% This means that gamemaster should not be afraid
% to pull a fork out of a player character’s hands and

% %%% txt/277.txt
%  make them into an NPC if they start too diverge too
%  greatly. Similarly, if a fork begins to learn information
%  that the main character does not (yet) have access to,
%  it is probably also better to run the fork as an NPC in
%  order to avoid metagaming.

%  It is entirely possible that a fork might decide that
%  it will no longer obey the originating ego and carry
%  about doing its own thing. This usually only occurs
%  with alpha forks, who are essentially a full copy
%  anyway, and as time passes the idea of merging back
%  with the original ego becomes unappealing. Beta and
%  delta forks are quite aware of their nature as “incom-
%  plete” copies, and so usually return back home to the
%  ego for reintegration. In rare cases, however, even
%  these might make a break for life on their own.

%  MERGING

% Merging is the process of re-integrating a previous-
% ly-spawned fork with the originating ego. Merging is
% performed on conscious egos/forks, transferring both
% to a single, merged ego. The process is not difficult
% to undergo when two forks have only been apart a
% short time. As forks spend more time apart, though,
% merging becomes a severe mental ordeal.

% To determine if merging goes well, a Psychosurgery
% Test is called for (made either by the ego or another
% character overseeing the process). The Merging table
% lists modifiers for this test as well as the result of suc-
% cess or failure.

% For synthmorphs, merging takes one full Action
% Turn. For biomorphs, an ego bridge (p. 328) or mne-
% monic augmentation (p. 307) is required to merge,
% and the process takes 10 minutes.

% The result of the process is a unified ego, whether
% or not the Merging Test succeeds. Psychotherapy (p.
% 209) and psychosurgery (p. 229) can troubleshoot bad
% merges over time.



%  EGOCASTING
%  In spite of being a spacefaring civilization with out-
%  posts throughout the solar system and beyond, tran-
%  shumanity makes scant use of spacecraft for interplan-
%  etary travel. Shuttlecraft using a variety of propulsion
%  systems make regular trips between habitats, planetary
%  surfaces, and moons. But for any trip longer than 1.5
%  million kilometers—the distance a fusion drive craft
%  can cover in a day—people egocast.




%                                                    MER
% TIME APART       MODIFIER SUCCESS
% Under 1 hour         +30      Seamless ego with memories inta
% 1–4 hours            +20      Solid bond, memories intact
% 4–12 hours           +10      Memories intact, 1 SV
% 12 hours–1 day       +0       Memories intact, 2 SV
% 1 day–3 days         –10      Memories intact, 3 SV
% 3 days–1 week        –20      Memories intact, 4 SV
% 1 week+              –30      Minor memory loss, 5 SV


%     THE SELF


%     Forking and merging have changed the way


%     transhumanity thinks about the self and what


%     it means to have a well-integrated personality.


%        While forking is child’s play from a tech


%     nological standpoint, the psychological and


%     social effects of cloning a mind mean that


%     most people are cautious about employing


%     forks. Some jurisdictions ban forking outright


%     for all but medical uses, while others have


%     severe restrictions. In many hypercorp juris


%     dictions, for instance, alpha forks are illegal


%     and letting a beta fork run for more than 4


%     hours without merging violates the modern


%     descendants of 20th-century anti-trust laws


%     Similarly, the Jovian Junta and other biocon


%     servatives ban forking entirely.


%        Disposing of unwanted forks is another


%     thorny issue. In some places, it’s as simple as


%     deleting them, because a stored mind has no


%     legal status. In others, a fork that doesn’t wish


%     to merge back with its originating ego might


%     be accorded some rights, though these are


%     generally only granted to alpha forks.


%        Most significantly, though, running a short


%     term fork of oneself for periods of an hour


%     or less is an easy task for many transhumans


%     Many people use forks of themselves to get


%     work done in everyday life, and almost every


%     one has at least experimented with forking at


%     some point.


%        Transhumans view forking a bit like early


%     21st-century humans viewed drinking and drug


%     use. A bit might be okay, but someone overdo


%     ing it will be stigmatized. This is because most


%     transhumans understand the psychological


%     consequences of overusing forks.               ■




%  Egocasting is transhumanity’s most advanced

% personal transportation technology, though only the

% character’s ego actually travels. Egocasting combines

% the technologies of uploading and quantum farcasting

% to transfer a backup (or sometimes even a conscious

% ego, see p. 269) over interplanetary distances.

%  Though egocasting occurs at the speed of light,

% egocasting times vary drastically with distance.


% ING


%     FAILURE
% m both   Memories intact, (1d10 ÷ 2, round down) – 1 SV


%     Memories intact, (1d10 ÷ 2, round down) SV


%     Minor memory loss, (1d10 ÷ 2, round up) SV


%     Moderate memory loss, (1d10 ÷ 2, round up) + 2 SV


%     Major memory loss, 1d10 + 2 SV


%     Major memory loss, 1d10 + 4 SV


%     Severe memory loss, 1d10 + 6 SV

% %%% txt/278.txt
% Egocasting within a cluster or planetary system is
% usually just a matter of minutes. Egocasting from the
% sun to the Kuiper Belt, however, takes between 40 and
% 70 hours, and so egocasting all of the way across the
% solar system can take even longer.

% Once an ego arrives at the destination receiver, it
% can be archived, run as an infomorph, or resleeved
% as normal.

% EGOCASTER SECURITY
% Beaming yourself across interplanetary space is a
% mature technology and usually works seamlessly. Be-
% cause egocasting uses quantum farcasters, there is no
% danger of radio interference cooking the signal and
% causing data loss. Normally the entire process is medi-
% ated by the character’s backup service, and security
% breaches are uncommon.

% However, there are several risks involved in ego-
% casting. The most obvious is that the character’s con-
% sciousness is transferred as a digital backup file at the
% destination. If the egocaster on the other end is not
% trusted or the networks at the destination are privately
% controlled by the receiver, the character is potentially
% putting themself at the mercy of their host. Most hy-
% percorps consider meddling with a transmitted ego to
% be a serious breach of etiquette, whereas autonomist
% types would find it unthinkably repressive. However,
% political extremist groups and criminal organizations
% in control of egocasters suffer from fewer restraints.

% A more subtle risk is the possibility for hackers to
% exploit security holes in the egocaster and its attached
% virtual space to steal a fork of the character. This is
% extremely difficult to do. It almost never happens
% during a normal upload, because the uploading ser-
% vices are security conscious to the point of paranoia.
% Even so, the forks stolen by such attempts more often
% than not end up being vapors, because the intruder
% is usually stopped before a full copy can be obtained.
% DARKCASTING

% Characters who want to egocast without the atten-
% tion of public officials like Immigration and Customs
% must seek out so-called darkcasting services—illegal
% farcaster transceivers typically operated by criminal
% syndicates and other clandestine groups. To locate
% such a service, a character must use their Networking
% skill and possibly their reputation (p. 285).



% MORPH BROKERAGE
% Morphs are a major commodity in transhuman society.
% The technology and materials needed to grow new
% morphs are cheap and abundant, though they take
% time. Cloned biomorphs take at least a year and a half,
% even with accelerated growth. Pods, which are typi-
% cally pieced together from vat-grown parts, take about
% 6 months. Synthmorphs like cases and synths can be
% produced in a day, whereas more complicated models
% can take a week or more. Theoretically, supply will one
% day outstrip demand to the point where flesh is free.

% Characters have several options for acquiring
% morphs when they travel by egocast, suffer heavy
% damage, or just feel like a new body. When egocasting,
% the most common method for travelers of middling
% means is to store their current morph in a body bank’s
% secure facility and lease a morph at their destina-
% tion. Less commonly, characters may rely on public
% resleeving facilities, or, if they have the means, they
% may purchase a new morph outright. Characters who
% expect to stay at their destination indefinitely or who
% decide to resleeve but aren’t traveling might instead
% opt for a trade-in on their old body, leaving it behind
% permanently in most cases.

% MORPH AVAILABILITY
% As noted under Resleeving and the Gamemaster (p. 271),
% finding the model of morph you want is not always easy.
% While many basic morph types (cases, synths, splicers)

% %%% txt/279.txt
% are generally available, characters can also locate new
% morphs using their Networking skills (see Reputation
% and Social Networks, p. 285). Certain morph types
% are harder to find then others; the gamemaster should
% apply an appropriate modifier for any morphs that seem
% rare or unusual (for example, swarmanoids or reapers).
% Likewise, some morphs may simply be unavailable in a
% given locale. Rusters are rarely available off of Mars, for
% example, while on Europa, most morphs are exotic local
% aquatic varieties.

% The gamemaster determines which factions are able
% to provide new morphs in a given locale. Factions
% will not provide morphs that are unavailable to that
% faction as starting characters. If the faction is not the
% dominant one in that locale, a penalty should be ap-
% plied, ranging from –10 to –30. Despite having a pres-
% ence in a given locale, some factions may be unable to
% provide morphs at all.

% If the character is seeking a customized morph with
% specific implants or enhancements, the search will be
% more challenging. The gamemaster should apply a –10
% to –30 modifier here as well, depending on the extent
% and legality of the modifications sought.

% MORPH ACQUISITION
% Once a morph is located, the character may call in
% favors (p. 285) or pay credits for it. Morph costs are
% noted on the Morph Costs table. In the inner system,
% morph prices are often inflated by demand in the
% market such that the most desirable morph types can
% cost a small fortune. Outsystem, prices in rep are more
% reasonable but still steep due to population pressures
% on life support-dependent outer system settlements.
% For travelers and frequent body hoppers, there are a
% number of ways to defray these costs.

% BROKERAGE AND MATCHMAKING
% Finding morphs for travelers and the bodiless is a
%  specialized skill demanding deep social networks and
% a flair for negotiation. In general, it’s a seller’s market,
%  so brokers (or “matchmakers,” as they’re called in
% the open economy) act as agents for the person seek-
% ing a body. The Morph Costs table assumes a 10%
% fee paid to the broker. Characters wishing to cut out
% the middleman may reduce cost by 10% but take a
% –30 penalty on their Networking Test to locate an
% available morph.

% CUSTOMIZED MORPHS
% If a character seeks to have a customized morph (with
% extra bioware, cyberware, or nanoware implants or
% robotic enhancements), the costs for these enhance-
% ments are added to the morph’s cost (if the gamemaster
% chooses, discount package deals may apply). Likewise,
% morphs may come saddled with positive or negative
% morph traits (p. 145). These traits raise or lower the
% morph’s cost at a rate of +500 credits per CP for posi-
% tive traits, or –200 credits per CP of negative traits.
% Negative traits typically reflect abuses the morph has
% suffered at the hands of previous occupants.


%                MORPH COSTS
% MORPH TYPE                                COST
% Biomorphs

% Flats, Splicers                           High

% Octomorphs                         Expensive (30,000+)

% Furies, Ghosts, Remade             Expensive (40,000+)

% Futuras                            Expensive (50,000+)

% All others                              Expensive
% Pods

% Workers, Pleasure Pods                    High

% Novacrabs                          Expensive (30,000+)
% Synthmorphs

% Cases                                   Moderate

% Synths, Dragonflies                        High

% Slitheroids, Swarmanoids                Expensive

% Flexbots                           Expensive (30,000+)

% Arachnoids                         Expensive (40,000+)

% Reapers                            Expensive (50,000+)
% Positive morph traits                  +500 per CP
% Negative morph traits                  –200 per CP


%  TRADE-IN
%  For those who wish to leave their old morph behind
%  permanently, trade-ins on current morphs are an
%  option. The high demand for bodies means that a buyer
%  is almost always available unless the gamemaster finds
%  extenuating circumstances. Morphs may be traded in
%  for the value shown on the Morph Costs table (ad-
%  justed for any positive or negative traits), less a 10%
%  physical exam and finder’s fee. This is either paid to the
%  morph broker in cred or rendered as a favor using rep.


% PATRON PROVISIONING
%  Characters on missions for rich or influential patrons
%  may have morphs provided for them. Normally
%  such provisions are made for the duration of a job,
%  although less commonly the morph itself might be
%  payment for services rendered. Gamemasters are
%  encouraged to be creative with such arrangements,
%  though players should be advised that such bargains
%  can quickly turn Faustian.


% BLACK MARKET MORPHS
%  Black market body traders promise to provide the
%  buyer with morphs and upgrades of choice regardless
%  of a habitat’s laws against weapons or implants, in
%  addition to bypassing standard arrival registration
%  via darkcasting. Illegal morphs usually come with a
%  price markup (+25% at least), whereas used morphs
%  with unsavory backgrounds (and traits) can usually be
%  acquired on the cheap (–25%).


% INDENTURE
%  Characters who find themselves too destitute to
%  afford a new morph can strike a deal for indentured
%  service—a “deal” that is rarely advantageous to the

% %%% txt/280.txt

% new indenture. Typical contracts require years of in-

% dentured labor—terraforming Mars, herding comets,

% asteroid mining, constructing habitats, colonizing

% exoplanets, etc.—in exchange for a cheap synthetic

% morph or splicer at the end of the term. Gamemasters

% may use their discretion in offering such terms, though

% in many cases the terms offered will temporarily or

% permanently end the character’s career as a free agent.

% Hypercorps using indentured labor are notorious for

% changing the terms at a whim, extending the service

% period, or slamming the indenture with a slew of

% hidden and outrageous charges that were not made

% clear up front. Characters may, of course, enter into

% such service fully intending to grab their morph and

% run at the first opportunity, but the hypercorps are

% very protective of their investments. Indentures are



%  USTOM-GROWN OR
% DESIGNED MORPHS
% me people are very particular about their morphs.
%  them, nothing “off the shelf” will do, even if it’s
% customized model tricked out with specific im-
% ants, traits, and biosculpting. Instead, they desire
% mething unique, something that must be specially
% own or designed.
%  n the case of biomorphs, this can mean several
%  ngs. Usually it means that the patron desires a
% ry specific set of genetic traits. This could be traits
%  m their original genetic lineage, traits copied from
% meone they idolized or honored, genes artfully
% afted by a renowned genetic designer, or mystery
%  its purchased at great expense from the Factors or
% tracted from a lost TITAN lab. Alternatively, it could
% ean the client seeks something more specific, such
%  an exact duplicate clone of their original body.
% While it is possible to put an existing morph in a
%  aling vat and alter its genetics with metamorph-
% g nanovirii in a matter of days, these procedures
% e difficult and prone to disaster. In many cases, it
% preferred to simply grow the desired clone from
%  atch, though even with accelerated growth this
% kes from 1.5 to 2 years (or 6 months to 1 year in
% e case of pods). Nevertheless, some hyperelites
%  ve taken steps to ensure that the morph they
%  sire is available at all likely egocast destinations.
% Though rarer, custom synthmorphs are sometimes
% ught after, usually by people who wish to showcase
%  ique or artistic robotic designs, but sometimes also

% engineers or agents who are field-testing proto-
% pes. Assuming blueprints are available, such models
% n be constructed in a matter of hours or days.
% Aside from time, the largest barrier to custom and
%  ique morph designs is typically cost. Ultimately it is
% e gamemaster’s decision on what expenses such mea-
%  es entail—typically starting at Expensive and moving
%  —or even whether they are possible at all.          ■
% closely monitored and tracked, and the hypercorps are
% not above sending ego hunters to retrieve a runaway.

% PUBLIC RESLEEVING
% Some locales, notably Titan, have a well-developed
% public resleeving infrastructure intended to provide a
% body to anyone who needs one. Morphs provided are
% usually unremarkable cases, synths, or splicers with
% no Positive traits or optional implants. Anyone hold-
% ing citizenship in a locale with public resleeving may
% apply for a body. Wait times are between a month and
% two years, with Reputation influencing wait times at
% the gamemaster’s discretion.

% RENTING MORPHS
% For temporary visits where an infomorph won’t do,
% morphs may be leased rather than bought. The cost
% to rent a morph is 1% of its cost per day, plus a Low
% charge for resleeving. This cost includes rental insur-
% ance (see below). If the rental insurance is waived
% (not always possible unless you have a good Rep), the
% rental cost may be reduced by half.

% Characters who are leasing a morph may also use
% their previous morph as collateral. In this case, deduct
% the cost of the character’s current morph from the
% rental morph before calculating the 1% cost per day,
% with a minimal rental cost of 10 credits per day.

% PENAL LEASE
% Characters visiting the inner system or Jovian
% Republic may be able to lease morphs belonging
% to prisoners. In most jurisdictions, criminals are
% sentenced to terms in rehabilitative simulspace with
% a stipulation that the prisoner’s morph becomes
% state property during their term of incarceration.
% Morphs acquired this way often have complicated
% histories but also tend to have modifications useful
% to Firewall agents. Conversely, characters who find
% themselves imprisoned may be subject to having
% their body leased out during incarceration.

% The effects of taking a penal lease are at the dis-
% cretion of the gamemaster. A character may have
% to pull some strings with their Reputation in order
% to lease such a morph, especially if it has restricted
% or illegal modifi cations. Negative traits, cases of
% mistaken identity, and unfortunate encounters with
% friends and associates of the morph’s former oc-
% cupant are among the possible drawbacks to this
% type of arrangement. On the up side, penal leases
% may reduce costs for both leasing and insuring the
% morph, again subject to the gamemaster’s discretion.

% RENTAL INSURANCE
% Leased morphs must be covered by an insurance
% policy, which often restricts the character from
% breaking the law or taking the morph anywhere
% too dangerous or lawless. Characters may purchase
% hazard insurance that will cover taking the morph
% into certain dangerous situations, but this will
% double the rental price at minimum.

% %%% txt/281.txt

% If a character suffers extensive organic damage or
% death while insured, the insurance will cover 80%
% of the morph’s cost, meaning that the character is
% expected to pay the other 20%. If they cannot pay,
% their possessions or their stored morph may be
% seized in payment.

% If a character violates their insurance policy by
% intentionally putting themself in harm’s way above
% the threat level at which the policy was purchased,
% without first communicating with and rendering pay-
% ment to the insurer, the policy may be declared void.
% If the leased morph dies under a voided policy and
% the character cannot pay to replace it, their posses-
% sions and stored morph may be subject to seizure.

% Seizure takes different forms depending upon the
% local economy and legal system. In hypercorp space,
% it is a straightforward seizure of liquid assets, in-
% cluding forced uploading if the character’s morph is
% seized. Elsewhere, the character is more likely to end
% up owing a lot of favors or taking severe hits to their
% reputation, but they are unlikely to undergo forced
% uploading or outright physical seizure of their morph.



% IDENTITY
% Given the nature of resleeving technologies, identity
% is a fluid concept in Eclipse Phase. Transhumans are
% used to the idea of identifying people by how they
% look or even by their biometric data, but this is no
% longer a certified method. What you look like may
% drastically change from one day to the next. You
% may see an olympian you recognize, but perhaps it’s
% been awhile, so you’re no longer certain that it’s the
% same person still in that morph. If you’re sleeved in a
% popular off-the-rack morph, there may be hundreds
% of other cloned morphs that look exactly like you out
% there—perhaps useful if you desire to blend in. Simi-
% larly, security services can no longer rely on biometric
% technologies. Forensics may be able to identify an indi-
% vidual morph’s presence at a crime scene, but proving
% who was in that morph at the time is another matter.

% Identity is, of course, tied to ego, and various
% authorities have instituted verification and security
% measures based on this. Within the inner system, each
% ego is given an ID number, which is used to validate
% their identity, citizenship, legal status, credit accounts,
% licensing, etc. This ego ID is verifiable by the person’s
% brainwave patterns, which remain the same even
% when resleeving. When an ego uploads, the upload-
% ing service is required to incorporate this ego ID into
% the person’s backup/infomorph. Likewise, when that
% person resleeves, the service handling the procedure
% is required by law to verify the ego’s ID before down-
% loading. The ego ID is then hardcoded into the morph
% itself in the form of a nanotattoo on the tip of the per-
% son’s index finger. This nanotat can be easily scanned
% at security checkpoints to verify identity.

% Though efficient, this system is far from perfect. For
% one, ID record-keeping is far from standardized and
% varies drastically from habitat to habitat. Most do not
% share records with each other unless they are part of the
% same political alliance in order to protect their citizens’
% privacy. For example, Lunar-Lagrange Alliance stations
% do not share citizenship ID data with the Planetary
% Consortium, though they do share with each other.

% On top of this, many identity records were lost
% during the Fall, a situation that was undoubtedly ex-
% ploited by those who preferred to erase their past or
% adopt a new persona. These all make for a situation
% where identity records are patchwork at best. Officials
% must also rely on the security of other habitats for ID
% verification. If a person egocasts to Nectar on Mars
% from Qing Long in the Martian Trojans, and the
% Nectar officials have no record of this person, they
% can only trust that the Qing Long officials did their
% job when verifying the subject’s ID and background.

% To make matters worse, many autonomist habitats
% operate without identity checks altogether. Though
% some ID measures are still used, both to prevent
% reputation system gaming and to be able to identify
% bodies in the case of death, these uses are significantly
% more lax and few records are kept. Therefore, when
% autonomists and the like egocast to habitats that re-
% quire ID, they are assigned a temporary ID for the du-
% ration of their stay (and sometimes any future visits).

% IDENTITY VERIFICATION
% There are three ways to verify someone’s identity:
% nanotat scan, brainwave scan, and checking the cryp-
% tographic hash on a digital mind.

% NANOTAT SCANS
% Special encoded nanobots are used to create a small
% nanotat on a person’s index finger. These nanobots
% contain encoded information that includes their name
% and identity, brainwave pattern, citizenship/legal status,
% credit account number, insurance information, and
% licenses. Depending on the local habitat laws, it may
% include other information such as criminal history,
% travel history, restricted implants, employment records,
% and so on. This nanotat may be read by anyone with
% a special ID scanner that reads the nanobot encoding.

% ID nanotats include information on the company
% that did the resleeving, so that the data may be ac-
% cessed and verified with their records online. The data
% on the nanotat is also cryptographically signed with
% the company’s public key, meaning that anyone who
% checks the data and the signature online can tell if the
% data has been altered.

% BRAINWAVE SCANS
% Brainwave scans are one of the few types of biometric
% prints that stay with an ego no matter what morph it
% is in. They are impractical for most security purposes
% as they require a scan with a combination electroen-
% cephalogram and neuroimaging device, referred to as
% a brainprint scanner, which takes approximately 5
% minutes. This device measures the subject’s baseline
% brainwave pattern as well as the subject’s brainwave
% signature responses when they think certain thoughts

% %%% txt/282.txt
% or sense certain patterns. These scans are all but
% impossible to fool, however, barring hacking of the
% brainprint scanner itself, and so are considered quite
% reliable. For this reason they are occasionally used in
% high-security facilities.

% It is worth noting that infection by some variants
% of the Exsurgent virus, notably the Watts-Macleod
% strain (p. 367), sometimes alters a person’s brainwave
% patterns, but not in every case.

% DIGITAL CODE
% Digital ID codes are often incorporated into backups and
% infomorphs. Not only does this help identify who the
% backup belongs to, but it serves as an electronic signature
% for verifying ID when the backup is to be resleeved. This
% digital code typically contains the same information as
% the nanotat ID, and is signed with a cryptographic hash
% that makes it difficult to forge and which can be verified
% online. AIs and AGIs also feature such built-in codes.

% CIRCUMVENTING ID CHECKS
% Firewall sentinels and clandestine agents often have a
% need to hide or alter their identities. While ID system
% are challenging, they are not insurmountable.

% FAKE IDS
% The easiest way to bypass security checks is to establish
% a fake ID. Given the patchwork nature of identity re-
% cords and the lack of any centralized authority, this is
% not very difficult. Numerous crime syndicates and even
% some autonomist groups maintain a thriving ID fabrica-
% tion business, often with complete histories and medical
% covers for implants that might be restricted or illegal.

% These IDs are usually registered with habitats that are
% either known criminal havens, have autonomist sym-
% pathies, or are isolated and remote. Though the ID is
% actually verifiable and registered with these stations, the
% potential shady origins of such IDs is known to most
% inner system authorities and so the character may be
% exposed to extra scrutiny or monitoring. Fake IDs may
% be acquired that are registered with more respected au-
% thorities, but this often requires a much higher expense
% or connections to hypercorp clandestine operations.

% Black market darkcast and resleeving options offer
% fake IDs as a matter of course.

% ALTERING NANOTAT IDS
% Special nanobot treatments may be manufactured
% to erase, rewrite, or replace nanotat IDs. Erasing a
% nanotat is easy, but not having one is a crime and im-
% mediate grounds for suspicion in many habitats. Re-
% writing a nanotat is also easy, though this means that
% the nanotat will fail its authorization online unless the
% encryption has also been cracked (p. 253). Replacing
% a nanotat ID with a fake one is just as possible, and is
% part of the process of acquiring a fake ID.

% DIGITAL ID TAMPERING
% Digital ID codes may also be tampered with, though like
% nanotat IDs this will mean that the ID fails online verifi-
% cation unless the encryption is also defeated (p. 253).
% LIFE IN SPACE
% Transhumanity is not just a spacefaring race, it is
% also largely space-dwelling. While a substantial por-
% tion of transhumanity inhabits planetary bodies like
% Mars, Luna, Venus, and the moons of the gas giants,
% the balance live in a variety of space habitats, ranging
% from the old-fashioned O’Neill cylinders of the inner
% system to the Cole bubbles of the outer system.

% SPACE HABITATS
% Space habitats come in many sizes and configura-
% tions, from survivalist outposts designed to support
% ten or fewer people to miniature worlds in resource-
% rich areas housing as many as ten million people. In
% heavily settled regions of space, such as Martian orbit,
% habitats may be integrated into local infrastructure,
% relying to some extent on supply shipments from other
% orbital installations.

% More commonly, especially in the outer system,
% habitats are independent entities. This usually means
% that in addition to the main space station, the habitat
% is attended by a host of support structures, including
% zero-g factories, gas and volatiles refineries, foundries,
% defense satellites, and mining bases.

%  Habitats—especially large ones—sometimes have
% visitors, as well. Majors habitats are crossroads in
% space. In addition to scheduled bulk freighter stops,
% they may have hangers-on such as scum barges, pros-
% pectors, or out-of-work autonomous bot swarms.

% Many habitats have some form of transportation
% network. This is most common in large cylindrical
% habitats with centrifugal gravity. Common solutions
% for public transit include monorail trains, trams, and
% dirigible skybuses. Common personal transit options
% included bicycles, scooters, motorcycles, and micro-
% light aircraft, with larger vehicles being uncommon
% and usually reserved for official use.

% Most habitats with large interior spaces also use aug-
% mented reality overlays to create consensual hallucina-
% tions of a sky and clouds, to which most residents keep
% their AR channels tuned. One would think that in space,
% talking about the weather would have disappeared from
% transhumanity’s repertoire of small talk, but the habit
% persists—only the weather discussed is usually virtual (if
% it’s not real “weather”—solar flare activity and the like).

% CLUSTER COLONY
% Clusters are the most common form of microgravity
% habitat. Clusters consist of networks of spherical or
% rectangular modules made of light materials and con-
% nected by floatways. Typically business and residential
% modules are clustered around arterial floatways and in-
% frastructure modules such as farms, power, and waste
% recycling. Limited artificial gravity areas may exists,
% frequently parks or other public places and specialized
% modules like resleeving facilities (morphs often keep
% better when stored in gravity). Arterial floatways in
% large clusters may have “fast lanes” where a constantly
% moving conveyor of grab-loops speeds people along.

% %%% txt/283.txt

% Clusters are most commonly found in volatile-rich env
% like the Trojans and the ring systems of the gas giants (p
% Saturn). Clusters are rare in the Jovian system because
% cluster of individual modules rather than one large sta
% Jupiter’s intense magnetosphere is hideously inefficient.

% Cluster colonies can have anywhere from 50 to 250,000

% COLE BUBBLES
% Cole bubbles (or “bubbleworlds”) are found mostly in
% asteroid belt, where the large nickel-iron asteroids used t
% them are abundant. Bubbleworlds are less common in t
% and Greeks, where crusty ice asteroids predominate. A C
% is similar in many respects to an O’Neill cylinder, but th
% longitudinal windows. Sunlight instead enters through a
% arrays. The bubbleworld is also constructed very differen
% large solar array to heat a pocket of water inside of a met
% so that the metal expands. Rotating the asteroid caus
% leable material to form a cylinder, which is then capped
% water drained. The inside can then be pressurized, bui
% planted. Cole bubbles can also be spun for gravity, accor
% whims of the inhabitants, though the gravity lowers as yo
% poles of the bubble, with zero gravity at the axis of rotat

% Cole bubbles are among the largest structures transhu
% created in space. The largest Cole habitat, Extropia, ha
% tion of 10 million.

% HAMILTON CYLINDERS
% Hamilton cylinders are a new technology. There are only
% operational Hamilton cylinders in the system, but the de
% great promise and is likely to be widely adopted over t
% period. Hamilton cylinders are grown using a comple
% algorithm that orchestrates nanoscale building machi
% nanobots build the habitat slowly over time, a process
% growing than construction.

% Similar to O’Neill cylinders and Cole bubbles, a Ham
% der is a cylindrical habitat rotating on its long axis to pr
% ity. Two of the known Hamilton cylinders orbit Saturn i
% skimming the rings near the Cassini division. From th
% they can graze on silicates and volatiles using harvester s

% None of the currently-operating Hamilton cylinders h
% to full size yet, but estimates say they could each hou
% million people.

% O’NEILL CYLINDERS
% Found mostly in the orbits of Earth, Luna, Venus, and Ma
% cylinders were among transhumanity’s first large space
% signs. O’Neill cylinders are no longer built, having bee
% by more efficient designs, but are still home to tens of
% transhumans. O’Neill cylinders were constructed from me
% on Luna or Mercury, Lunar volatiles (including Lunar
% and asteroidal silicates.

% A typical O’Neill habitat is thirty-five kilometers lon
% lometers in diameter, and rotates around its long axis
% sufficient for centrifugal force to create one Earth grav
% inner wall of the cylinder. Smaller cylinders exist, tho
% usually feature lower gravity (typically Mars standard)
% are sometimes joined together, end-to-end, for extra lon
% A spaceport is situated at one end on the rotational a
% cylinder (where there is no gravity). Arrivals by space u
% microlight launch pad to get down to the habitat floor.
% nments
% cularly
% lding a
% n from

%  bitants.



% e main
%  nstruct
% Trojans
%  bubble
%  are no
%  mirror
% using a
%  steroid
% he mal-
% and the
% ut, and
% g to the
%  ear the

%  ity has
% popula-




% ee fully
%  shows
% coming
% enomic

% These
%  re like

% n cylin-
% de grav-
%  sitions
% osition,
% s.
%  grown
% up to 3




% O’Neill
% itat de-
%  placed
% ions of
%  mined
%  ar ice),

%  ight ki-

% speed
%  on the
% h these
%  linders
% abitats.

% of the
% a lift or

% %%% txt/284.txt

% The inside of an O’Neill cylinder has six alternating
% strips of ground and window running from one cap
% of the cylinder to the other. One narrow end of an
% O’Neill cylinder points toward the sun. The opposite
% end is the mooring point for three immense reflectors
% angled to reflect sunlight into the windows. Smart
% materials coating the windows and reflectors prevent
% fluctuations in solar activity from delivering too much
% heat. The air inside the cylinder and its metal super-
% structure provide radiation shielding.

% The land in most O’Neill cylinders is one-third
% agricultural (a combination of food vats and high-
% yield photosynthetic crops), one-third park land, and
% one-third mixed use residential and business. O’Neill
% habitats have a day and night cycle regulated by the
% position of the external mirrors. The business and
% residential sections of the cylinder usually alternate
% with the park land over two of the strips of land;
% cropland usually takes up the third. Bridges cross
% the windows every kilometer or so, linking the land
% strips. The interior climate, the architectural style of
% the structures, and the types of vegetation and fauna
% present vary with the tastes of the habitats’ designers.

% Depending upon size, O’Neill cylinders can house
% from 25,000 to 2 million people.

% TIN CANS
% Antique research stations and survivalist prospector
% outposts often fit this description. Tin can habitats
% are only a few notches up from the early 21st-century
% International Space Station. Tin cans usually consist
% of one or more modules connected to solar panels
% and other utilities by an open truss. Deluxe models
% feature actual floatways or crawlways between
% modules, while barebones setups require a vacsuit or
% vac-resistant morph to go from room to room. Food
% growing capacity is severely limited and there may be
% no farcasters, but fabricators are available, as well as
% mooring for shuttles and perhaps prospecting craft.

% Tin cans rarely house more than 50 people.

% TORUSES
% Interchangeably called toruses, toroids, donuts, and
% wheels, these circular space habitats were a cheap
% alternative to the O’Neill cylinder used for smaller
% installations. Like O’Neill cylinders, toruses are seldom
% constructed anymore, but many are still encountered in
% the inner system, particularly in Earth and Lunar orbit.

% A toroidal habitat looks like a donut 1 kilometer in
% diameter, rotating on great spokes. There is a zero-g
% spaceport at the wheel’s hub. Visitors take a lift down
% one of the spokes to the level of the donut, where
% rotation creates one Earth gravity.

% The plan of toroidal habitats varies greatly, as many
% were designed for specific scientific or military purposes
% and only later taken over as habitats by entrepreneurs
% or squatters. Many have a succession of decks in the
% donut. Most of those designed for long-term self-
% sufficient habitation have smart material-covered glass
% windows for growing plants along much of the inside
% surface of the torus. Toroidal habitats equipped for
% farming normally face the sun in a direction perpen-
% dicular to their rotational axis, but then use a slow pro-
% cessional wobble of that axis to create a day/night cycle.

% Toruses were usually built to accommodate small
% crews of 500 or fewer people, though some larger
% ones exist, able to house 50,000. A few rare double-
% toruses also exist, like two large wheels spinning in
% opposite directions, joined at the axis.

% IMMIGRATION AND CUSTOMS
% How characters gain entry to a habitat and what type
% of screening they’re likely to undergo depends upon
% how they arrive. Some habitats are close to other
% settlements, while others are physically isolated by the
% vast, empty distances of interplanetary space.

% Habitats in dense planetary systems receive most of
% their visitors via conventional space travel. Immigra-
% tion and customs infrastructure is geared toward re-
% ceiving visitors via their spaceport, and the processing
% of arrivals is in most ways analogous to a twentieth
% century airport. Isolated habitats, on the other hand,
% tend to receive almost all of their visitors via egocast.

% PHYSICAL ARRIVALS
% Arrivals by spacecraft undergo, at minimum, an ego
% ID check, scans to detect pathogens, hostile nanobots,
% explosives, or radiation, and an inspection of their
% personal effects. Some habitats go farther, including
% rigorous secondary screenings using scout nano-
% swarms, scans of all electronic systems for malware,
% and/or aggressive interrogation of a fork of the subject.
% Even autonomist enclaves enforce automated scans for
% anything that might pose a danger to the habitat or
% any signs of hypercorp saboteur efforts.

% Restricted goods vary according to local legalities.
% Many habitats, particularly those controlled by auton-
% omist or criminal factions, allow personal weaponry
% as long as its nothing you can use to blow a hole in
% the structure or indiscriminately kill dozens of people.
% Others, notably the Jovian Republic and hypercorp
% stations, disallow lethal weapons of all kinds, except
% for people who have acquired special permits and au-
% thorization (sometimes available by bribing the right
% people or pulling favors with rep). Nonlethal weapons
% are generally allowed. Other restricted items may
% include nanofabricators, nanoswarms, malware and
% hacker software, drugs and narcoalgorithms, certain
% types of XP recordings, covert operations tools, and
% so on. Certain types of morphs may also be restricted,
% such as reapers, furies, or uplifts.

% Certain habitats may insist that visitors—or at
% least the ones they don’t like the looks of—submit to
% specific forms of monitoring or surveillance for the
% duration of their stay. This might include taggant
% nanoswarms, hosting a police AI in your mesh inserts,
% or even physical tailing by an armed security drone.
% Other stations will require that their visitors leave a
% fork as a form of collateral at the door—in case they
% commit a crime, the fork can be interrogated.

% %%% txt/285.txt

% Finally, though rare, some habitats go so far as to
% charge all visitors an “air tax”—a fee for using the
% station’s publicly available resources while they are
% present. This is generally only common in isolated
% habitats with strained resources, and is considered
% especially obnoxious by most autonomists.

% Some syndicates run a good business in smuggling
% certain goods or even people into habitats. This is gen-
% erally accomplished through bribed security personnel,
% but is also sometimes handled as falsified credentials
% that will allow the subject to breeze past security
% checks. Such services are typically quite expensive.

% For those hoping to gain quiet and unobserved
% access, there is always the option of taking a space-
% walk and trying to break in through an unattended
% airlock. Such attempts are quite often dangerous and
% futile, as most habitats have dedicated sensor and
% security systems to monitor their exterior surface and
% in particular any access points. Still, it is a possibility
% for a resourceful team with a skilled hacker, though
% armed sentry bots are a particular danger.

% ELECTRONIC ARRIVALS
% Arrivals by egocast are sometimes interviewed by
% habitat authorities in a simulspace before resleeving.
% Depending upon the habitat’s attitude toward civil
% rights, this process can be relatively reasonable or
% quite invasive. A minimal entry inspection includes an
% ID check, a brief interview with a customs AI, and a
% review of the specs of the morph into which the ar-
% riving ego plans to resleeve. Habitats with draconian
% immigration measures may use harsh psychosurgery
% interrogation techniques on suspect infomorphs. Ego-
% cast backups have little recourse to avoid this treat-
% ment—station authorities can simply file them away
% in cold storage if they choose—so it is wise to investi-
% gate custom procedures before you send yourself over.

% Because many people, particularly autonomists and
% brinkers, don’t appreciate this kind of reception, vari-
% ous uploading services have stepped in to provide pre-
% customs resleeving for characters traveling to habitats
% with suspect screening methods. For often-exorbitant
% fees, the traveler egocasts into an extraterritorial sub-
% station close to their intended destination, resleeves
% there, and then travels to their destination by rocket.

% Various darkcast services, normally run by estab-
% lished crime syndicates, sometimes offer an alternative
% method of egocasting in and possibly even resleeving.
% Darkcast services are quite expensive, however, and
% the character is at the mercy of the syndicate opera-
% tors. In rare cases, some political factions or even hy-
% percorps might operate their own darkcast systems,
% which a character with good networking skills might
% be able to take advantage of.

% SPACE TRAVEL
% In some circumstances, characters will prefer to travel
% physically through space rather than egocasting. In
% Eclipse Phase, spacecraft are primarily dealt with as a
% setting environment rather than a vehicle/gear to use.
% Spacecraft largely pilot themselves via the onboard AI.
% Though characters can also take over with their Pilot:
% Spacecraft skill, the situation rarely calls for it.

% LOCAL TRAVEL
% In densely inhabited planetary systems such as Mars
% and Saturn, most travel between cities, surface stations,
% and orbital habitats within 200,000 kilometers is by
% small hydrogen-fueled (or sometimes methane-fueled)
% rockets. This form of travel is incredibly cheap, very
% fast, and avoids the occasional personality glitches
% that crop up during egocasting. LOTVs (lander and
% orbital transfer vehicles, p. 348) are commonly used.
% Spacecraft leaving a planetary body need to be able to
% generate enough thrust to escape the gravity well (see
% Escaping Gravity Wells, p. 346).

% DISTANCE TRAVEL
% For distances of 200,000 to 1.5 million kilometers,
% somewhat larger (and more expensive) fusion- and
% plasma-drive craft make regular runs. Nuclear electric
% ion drives were once used on some of these routes,
% but the poor efficiency of these fission systems and the
% need for radioactive heavy metal reaction mass means
% that they are almost never used anymore. Faster anti-
% matter-drive couriers are also commonly used. These
% ships lack the thrust to escape from the gravity wells
% of large planets or moons, so they station themselves
% in orbit and use smaller ships (typically LOTVs) with
% higher thrust to transport people to and from the
% planetary surface.

% For distances beyond 1.5 million kilometers, almost
% everyone uses egocasting

% SPACE TRAVEL BASICS
% Spacecraft use various types of reaction drives (see
% Spacecraft Propulsion, p. 347), meaning that they burn
% fuel (reaction mass) and direct the heated output in
% one direction, which pushes the spacecraft in the oppo-
% site direction. Travel over any major distance typically
% involves a period of high-acceleration burn for several
% hours at the beginning of the flight, where up to half of
% the reaction mass is spent to drive up the craft’s veloc-
% ity. The ship then coasts for the majority of the flight at
% that speed, until it approaches its destination, where it
% flips over and burns an equal amount of reaction mass
% in the opposite direction to decrease velocity.

% Though some craft burn half their reaction mass
% to get up to the best speed possible, this doesn’t leave
% much room for additional maneuvering or emergen-
% cies. Many craft therefore only burn up to a quarter
% or a third of their fuel in initial accelerations, so they
% have some to spare in case they need it. A few tricks
% can be used to save fuel and build speed, such as sling-
% shotting around the gravity wells of larger planets or
% aerobraking in a planet’s upper atmosphere.

% Travel times between locations are constantly
% changing as various bodies move in their orbits
% around the solar system. Within a cluster or planetary
% system, travel takes a matter of hours. Within the

% %%% txt/286.txt
% inner system, travel can take days or weeks. Travel to,
% from, or within the outer system can take much longer,
% and is usually a matter of several months.

% Most ships operate at zero-g, except for a few larger
% craft that are able to spin habitat modules for low
% gravity. Periods of high-acceleration also produce
% temporary gravity in a downward direction, towards
% the burn.

% Space is a valuable commodity on board spacecraft,
% so room is often tight. Sleeping and personal quarters
% are rarely bigger than large closets, just enough room
% for a sleeping bag and personal effects. Depending
% on the size of the craft, there may be a communal
% recreation area. The crew tend to only be busy at the
% beginning and end of a trip, when they must deal with
% acceleration/deceleration and maneuvering around
% other space traffic. The rest of the trip they spend
% dealing with repairs or otherwise killing time, often
% by accessing XP or VR simulations or playing AR
% games. While spacecraft have their own local mesh
% network, they are usually too far to interact with the
% mesh networks of other habitats without significant
% communications lag, so they must make do with
% their own archive of entertainment options. Many
% long-haul ships are crewed by hibernoid morphs, who
% hunker down for a long nap.

% SPACESHIP COMBAT
% Combat in space tends to take place over long dis-
% tances using massive beam weapons, railguns, and
% missiles. It also tends to be nasty, brutish, and short.
% Significant damage to a vessel can cause atmospheric
% decompression, killing any biomorph crew who aren’t
% suited up and strapped down.

% For the most part, it is recommended that space
% combat be treated as a plot device, part of the back-
% ground story that helps create drama and tension,
% rather than an event that characters actively partici-
% pate in. This is not to say the characters cannot play a
% role in the combat, or that their actions will have no
% effect on the outcome. They may become involved in
% damage control, negotiate with hostile forces, repel
% boarders, target weapons with Gunnery skill, stage a
% mutiny, attempt to hack the networks of approach-
% ing vessels, escape out the airlock, hide out while the
% pirates sack the ship, or similar affairs. It is recom-
% mended, however, that gamemasters steer clear of
% space combat situations that could easily lead to the
% whole team dying due to a few bad dice rolls.



% NANOFABRICATION
% In order to create an object in a nanofabricator
% (whether a cornucopia machine, fabber, or maker; see
% p. 327), three things are needed: raw materials, blue-
% prints, and time.

% RAW MATERIALS
% Raw materials are generally easy to acquire, as most
% nanofabricators are equipped with disassembler units
% that will break down just about anything into its con-
% stituent molecules. Feedstock may also be purchased
% (at a cost of Trivial). Many habitats route their recy-
% cling and waste products directly into disassemblers.

% BLUEPRINTS
% Most nanofabricators are pre-loaded with blueprints
% for general purpose items: food, simple clothing, basic
% tools, etc. Blueprints for other goods may be acquired
% in several ways:

%  • They may be purchased online (legally or on the

% black market).
%  • They may be found for free online (see below).
%  • They may be acquired with Rep, following the

% usual rules for social networking (p. 285).
%  • They may stolen (usually by hacking a mesh site

% or a nanofabricator containing such plans).
%  • They may be self-programmed (see below).


% Once the blueprints are acquired, they are simply
% loaded into the nanofabricator.

% OPEN SOURCE BLUEPRINTS
% Blueprints for many goods may be found for free
% online, disseminated by an active open source
% software movement. The availability of such plans
% typically depends on the local mesh. In autonomist
% habitats, a simple Research Test is likely to turn up the
% open source blueprints you need (applying modifiers
% for unusual items). In more restricted habitats, open
% source blueprints may be harder to find, as they will
% be securely hidden from the prying eyes of the authori-
% ties. In this case, the character will need to use their
% Rep to gain access, bribe a local hacker group, or do
% something similar.

% Note that restricted nanofabricators may not accept
% open source blueprints (see Blueprint Restrictions).

% BLUEPRINT RESTRICTIONS
% Some nanofabricators are equipped with pre-pro-
% grammed restrictions not to accept blueprints for re-
% stricted items (such as weapons) or non-licensed items
% (such as black market or open source blueprints).
% These restrictions may be circumvented by hacking
% the nanofabricator and re-programming it, following
% normal hacking rules (p. 254).

% PROGRAMMING BLUEPRINTS
% A dedicated character may simply decide to program
% their own blueprints, though this is a time-consuming
% endeavor. To do so, the character must make a Pro-
% gramming: Nanofabrication Test with a timeframe of
% one week per cost level of the item. For example, a
% Trivial cost item takes 1 week, a Low cost item takes 2
% weeks, a Moderate item 3 weeks, and so on. Academ-
% ics: Nanotechnology skill or a skill appropriate to the
% object’s design may be used as a complementary skill
% (p. 173) for this test. A fork or muse may also be as-
% signed to such a programming task.

% %%% txt/287.txt
% TIME
% Once the raw materials and blueprints are in, most n
% tion is simply a matter of time. The exact timeframe t
% object varies, but roughly approximates 1 hour per cost
% the item (1 hour for Trivial, 2 for Low, 3 for Moderate
% gamemaster may feel free to modify this period as appr
% the object.

% THE PROGRAMMING TEST
% Nanofabrication is typically handled as a Programming
% rication Test. In most cases, this can be treated as a Sim
% Test (p. 118), with a failed roll simply indicating that th
% some minor imperfections, or perhaps took longer to ma

% In some cases, the gamemaster may call for an actual S
% meaning that failure is more of a possibility. This shou
% done for items that are exotic, extremely complicated, o
% the blueprints are incomplete or otherwise suspect. This t
% be made if the raw materials are limited.

% The character operating the nanofabricator can make
% it can be left up to the nanofabricator’s built-in AI. Mos
% AIs have a Programming: Nanofabrication skill of 30 (s
% Muses, p. 331).



% REPUTATION AND SOCIAL NETWORKS
% “Once upon a time, there was a planet so incredibly primi
%  inhabitants still used money. That planet is called ‘Mars.

% —Professor Magnus Ming, Titan Autonomous Univers


% The conflict between market capitalism and other fo
% nomics is one of transhumanity’s last great culture wa
% still being fought. Transhumanity’s expansion into the s
% created myriad opportunities to experiment with new ec
% tems. Many failed, but the reputation economies of the o
% have proven both utilitarian and robust in a way that n
% challenger to market capitalism has managed.

% The reputation economy, sometimes called the gift e
% open economy, is one in which the material plenty creat
% fabrication and the longevity granted by uploading an
% have removed considerations of supply versus scarcit
% economic equation—destroying classical economics in th

% The regimented societies of the inner system and
% Junta have used societal controls and careful regula
% technologies of abundance on their populations, thus k
% transitional economy system that is largely an outgrow
% cal economics. No one could get away with doing this i
% system. In the Trojans and Greeks, much of the belt, free
% anywhere outward from Saturn, the reputation economy

% How did this happen? For one thing, money is a nuis
% you’re an autonomous member of an autonomous collec
% nearest three neighbors (each 100,000 kilometers awa
% autonomous collectives. All of you are almost comp
% sufficient in terms of material resources. You have a flee
% that harvest water, volatiles, reactor mass, metals, and si
% have a nanofabricator to make all of your small items, a
% factory for large ones, and a machine shop where you
% anything else—with help and advice from an AI with th
% knowledge and experience of a top flight engineering t
% even need it. You grow your own food.
% fabrica-
% eate an
%  gory of
% c.). The
% iate for




% anofab-
% Success
% em has

% ess Test,
% only be
% r which
%  an also

%  test or
% ch Such
% AIs and




%  that its




%  of eco-
% and it’s
%  system
% mic sys-
%  system
% revious

%  omy or
%  y nano-
%  ackups
%  om the
%  rocess.

% Jovian

% of the
%  ng to a
%  f classi-
% he outer
%  ter, and
%  es.
%  e when

% whose
% are also
% ely self-

% robots
%  es. You
%  munity
%  n build
%  mbined
%  , if you

% %%% txt/288.txt

% Money is for people who don’t know how to take
% care of themselves. Transhumanity is only a few de-
% cades away from being a mature Type I Kardashev
% civilization, having largely mastered the material
% resources of its own solar system. A character from
% the outer system most likely finds the whole concept
% of money an embarrassment.

% However, material abundance hasn’t eliminated
% the value of certain goods and services. A transhu-
% man’s lunch might be free, but innovative ideas, new
% designs, health care, sex, and dirty work don’t grow
% in fabricators. What if you need gene therapy on your
% morph to grow infrared sensing cells on your face?
% How about someone to assassinate your renegade
% beta fork after she set off a hallucinogen grenade at
% your gallery opening and kidnapped your boyfriend?
% What if you really need a spanking? You call on your
% social network. If your network is sufficiently deep
% and numerous, and your reputation is good enough,
% someone will help you out.

% In the inner system, the reputation economy doesn’t
% replace money for the exchange of goods and services,
% but it does hold sway over the network of favors and
% influence. Calling on contacts, getting information,
% and making sure you’re in the best place to see and be
% seen all involve calling on your social network.

% SOCIAL NETWORKS
% Social networks represent the people you know, and
% the people they know, and so on. It starts with your
% friends and family, spreads out to your co-workers,
% neighbors, and colleagues, and expands all the way
% out to your acquaintances, from the neo-hominid
% waitron at your favorite cafe to the sylph you flirt
% with at the club. In the always-online, fully-meshed
% universe of Eclipse Phase, this goes even further,
% encompassing all of the people you’ve linked to via
% social mesh networks, everyone who watches your
% blog/lifelog/updates, and everyone you interact with
% on various mesh forums. Now add in the friend-of-a-
% friend factor, and everyone has an impressive ability
% to reach out to people they know, people they sort of
% know, and people you don’t know but who are some-
% how linked to you in one degree or another.

% Of course, social networks are not homogeneous.
% Among the ever-diversifying ranks of transhumanity,
% there is a tendency to coalesce around various shared
% characteristics, whether those be cultural background,
% personal interests, professional ties, local connections,
% political affiliations, subcultural obsessions, or simply
% common interest from being part of the same sub-
% species clade. The social network of an info-anarchist
% hacker is likely to bear little resemblance to that of
% a hypercorp socialite or an isolate brinker. Neverthe-
% less, social networks quite frequently overlap, often
% in unexpected and interesting ways. Most people can
% be considered members of several different types of
% social networks. This overlap is what links disparate
% groupings of transhumans together.
% NETWORKING
% Just being connected, of course, doesn’t mean you have
% several thousand idle transhumans at your beck and
% call. If you hope to gather the latest gossip, get advice
% from an expert, find someone who can fix your prob-
% lems, acquire a piece of gray market tech, or spread a
% meme, you need to know both who to talk to in that
% social network and how to go about getting what you
% need, especially if you hope to keep things quiet and
% not raise any flags.

% This is where your Networking: [Field] skills come
% in (p. 182). Networking represents your ability to ma-
% neuver through this web of personal and impersonal
% connections to find who and what you need. This
% could be handled by word-of-mouth, posting the right
% queries in the right places on the mesh, monitoring
% the right personal profiles and forums, harnessing the
% power of the mob with crowdsourcing, or any number
% of similar creative tactics.

% Each field you have in Networking represents a
% particular network grouping, a common interest that
% ties people together. Most of these fields are based
% on factions (Autonomists, Hypercorp, etc.) and tie
% into a special reputation network (see the Reputation
% Networks table, p. 287). At the gamemaster’s discre-
% tion, other groupings of people could be connected
% through these skills and rep systems. For example,
% artists and journalists of all stripes can fall under
% the Networking: Media skill and f-rep, no matter if
% they are autonomist or hypercorp. Likewise, being a
% diverse group, brinkers do not universally fall into any
% of the categories, and are instead spread out between
% them. If the gamemaster and players agree, other
% Networking fields and rep networks may be added,
% representing other spheres of interest, such as AR
% Games, Sports, Slash Fiction, etc.

% The exact uses for which you may exploit your social
% networks are noted below. While in some cases the de-
% fining element is who you know and how good you are
% at reaching out to them, in others the defining element
% is how known you are. You might be connected to
% thousands of people, but if you don’t have clout, your
% efforts to make use of these connections is limited. This
% is where Reputation comes into play.

% REPUTATION
% Reputation is a measurement of your social currency.
% In the gift economies of the outer system, social repu-
% tation has effectively replaced money. Unlike credit,
% however, reputation is far more stable.

% Within Eclipse Phase, reputation scores are facili-
% tated by online social networks. Almost everyone is a
% member of one or more of these reputation networks.
% It is a trivial matter to ping the current Rep score and
% history of someone you are dealing with—your muse
% often does this automatically, marking an entoptic
% Rep score badge on anyone with whom you interact,
% updated in real time, so you will see if they suddenly
% take a hit or become popular. The 7 most common
% networks are noted on the Reputation Networks

% %%% txt/289.txt
%  table. Gamemasters and characters may decide to add
%  others appropriate to their game.

% You purchase a Rep score in one or more of these
%  networks during character creation. Rep scores are
%  rated between 0 and 99, just like skills. These ratings
%  determine your ability to acquire goods, services, and
%  information and favors, as noted below. These scores
%  may be raised or lowered during game play according
%  to your character’s actions.

%  USING NETWORKS AND REP
%  In game terms, you take advantage of your connec-
%  tions and personal cred every time you need a favor.
%  A favor is broadly defined as anything you try to get
%  via your social networks, whether that be information,
%  aid, goods, and so on. Different types of favors are
%  described under Favors, p. 289.

%  THE NETWORKING TEST
%  To pursue a favor, you start by looking around. This
%  calls for a Networking Test to determine if you can
%  find the person, people, or information you need.
%  This represents talking to people you know, spread-
%  ing the word to people they know, posting queries to
%  the social network at large, digging through various
%  profiles, chat rooms, etc. to find someone who might
%  help you out, and so on.

%  Networking Tests are subject to modifiers for the
%  level of the favor (see below), the amount the char-
%  acter is trying to keep quiet about the request (see
%  below), and any other factors noted on the Network-
%  ing Modifiers table or determined by the gamemaster.

%  Networking Tests are Task Actions—it takes time
%  to call in favors or track down information. The time-
%  frame depends on the level of favor, as noted on the
%  Favors table, p. 289.





%                                      REPUTATIO
% NETWORK NAME                   REP NAME           NETWORKIN
% The Circle-A List              @-Rep              Autonomists
% CivicNet                       c-Rep              Hypercorps
% EcoWave                        e-Rep              Ecologists
% Fame                           f-Rep              Media
% Guanxi                         g-Rep              Criminals
% The Eye                        i-Rep              Firewall
% Research Network Associates r-Rep                 Scientists





%   NETWORKING MODIFIERS


%         SITUATION                      MODIFIER

%  Favor level exceeds Rep level          –10 per level

%  Rep level exceeds favor level          +10 per level


%        Keeping quiet              –Variable (see p. 288)


%         Burning Rep               +Rep amount burned


%         Paying extra                  +10 per level

% FAVOR LEVELS AND MODIFIERS

% Rep scores are broken down into five levels, reflecting

% your standing within that community. Every 20 points

% of Rep equals one level. See the Reputation Levels

% table for a breakdown.


% Likewise, favors are also broken down into five

% levels, rated from Trivial to Scarce (see Favors, p. 289,

% for specific examples). The standard level of favor

% you can expect to get from a social network is based

% on your level of Rep in that network. If you want to

% pursue a favor above your level, you can do so, but

% you will suffer a negative modifier on your Network-

% ing Test. This reflects that someone with low standing

% has a hard time getting people to go out of their way

% for them. Similarly, if you pursue a favor below your

% level, you receive a positive modifier to your Network-

% ing Test, reflecting that your prestige makes it easier to

% acquire minor things that you need. For each level the

% favor falls under or above your Rep score level, apply

% a + or –10 modifier, as appropriate.




%  Jaqui’s on a scum barge and she needs to get a


%  hold of a weapon fast. She has a specific weapon in


%  mind, but it’s pricey—its cost is High. She decides


%  her best approach is to try talking to the scum on


%  the ship to try and find someone who can lend or


%                                                            EXAMPLE






%  sell her such a weapon, using her @-rep and her


%  Networking: Autonomist skill of 50. Acquiring a


%  High cost item counts as a Level 4 High favor (see


%  Acquire/Unloads Goods, p. 289). Jaqui’s @-rep


%  is 53, which is only Level 3. Since the favor is one


%  level higher than her rep level, she suffers a –10


%  modifier on her Networking Test. Jaqui must roll a


%  40 or less (50 – 10) to find a weapon supplier.




% NETWORKS
% ELD       FACTIONS AND OTHERS


%      anarchists, Barsoomians, Extropians, Titanian, and scum


%      hypercorps, Jovians, Lunars, Martians, Venusians


%      nano-ecologists, preservationists, and reclaimers


%      socialites (also artists, glitterati, and media)


%      criminals


%      Firewall


%      argonauts (also technologists, researchers, and scientists)




%         REPUTATION LEVELS


%    REPUTATION SCORE                      REPUTATION LEVEL


%             0–19                                 Level 1


%            20–39                                 Level 2


%            40–59                                 Level 3


%            60–79                                 Level 4


%            80–99                                 Level 5

% %%% txt/290.txt
% PAYING/EXCHANGING FOR FAVORS
% Favors don’t necessarily come for free. Depending on
% what you’re after, you may also need to exchange for it.

% In the capitalist and transitional economies of the
% inner system and Jovian Junta, you may need to buy
% the goods or services you are after with credit. Even in-
% formation might be paid for by bribing the right person.
% Once spent, that credit is gone until you earn more.

% In the anarchistic reputation economies of the outer
% system, you can get what you need for free. In this
% case, you are acquiring goods and services based on
% the strength of your reputation.




%       Jaqui rolls a 39—she makes it! After posting some


%       public notices on the scum social network (she’s


%       not worried about legalities or hiding what she’s
%  EXAMPLE






%       doing—this is a scum ship after all), she gets di-


%       rected to a weapons dealer with a good rep. While


%       a scum arms merchant normally sells their wares


%       for credit, Jaqui is scum herself, so she’s able to use


%       her scum community standing and get the weapon


%       for free. This uses up a High favor, however.


% THE LIMITS OF REPUTATION
% Even in the gift economies, reputation only gets you so
% far. There are limits to how often you can ask for help
% before you start coming across as pushy or a leech. In
% game terms, this is expressed as a refresh rate—the
% amount of time you must wait to pass before you can
% seek out a favor of that level again without seeming
% demanding. Refresh rates are noted on the Favors
% table (p. 289).

% If you need to seek another favor before the refresh
% rate has expired, you have two choices. You can
% expend a higher level favor instead, keeping in mind
% that higher level favors refresh more slowly. Alterna-
% tively, you can burn reputation (see below).




%       Now that Jaqui’s got her weapon, she needs


%       another favor—she needs to find someone who


%       doesn’t want to be found. The person she’s after is


%       scum, so once again she turns to the scum for help.


%       The gamemaster decides that this is another Level


%       4 favor (see Acquire Information, p. 291). Once


%       again, with her Networking: Autonomist of 50 and


%       Level 3 rep, she must roll a 40 or less. She gets a
%  EXAMPLE






%       21, and finds someone who has the information


%       she needs.


%          Jaqui now has a choice. To get this information,


%       she either needs to pay the person in credits (a


%       High cost) or she she needs to expend another


%       Level 4 favor. She’s low on money, so she decides


%       to use her rep again. Level 4 favors only refresh


%       once a month, though, and Jaqui used her last one


%       just a few hours ago. Her only choice is to expend


%       a higher favor, so she expends a Level 5 to get the


%       intel she needs.
% BURNING REPUTATION
% In some cases, getting what you need may be more
% important than not stepping on people’s tentacles. In
% situations of dire need, you can burn some of your
% Rep score to get the job done, meaning that you ex-
% change a loss of Rep for a shot at a favor. This reflects
% that you are pushing the bounds of how far people are
% willing to go for you. While you still might get what
% you need, your online reputation rating takes a hit as
% people flag you for being needy.

% There are two reasons to burn Rep score. The first
% is to get a bonus on your Networking Test. This indi-
% cates that you are pulling strings and calling in mark-
% ers to get the favor you’re after. This is particularly
% useful when you are trying to obtain a favor that’s of
% a level higher than your Rep, but abuse it too often
% and you will soon have no social standing at all. Every
% point of Rep you burn gives you an equivalent posi-
% tive modifier on the Networking Test, up to a maxi-
% mum of +30.

% The second option is to burn Rep to seek a favor
% before it has refreshed. This reflects that you are asking
% for too much in a short period. The amount of Rep
% you must burn in this case depends on the level of favor
% you are seeking, as noted on the Favors table (p. 289).



% Jaqui’s got her weapon and her target’s where-

% abouts, but she needs one more thing: a hacker.

% She needs someone who can open some doors

% and defeat some security systems so she can get

% to the target she’s after in his hideout. Since she’s

% on a scum barge, Jaqui feels that, once again, her

% best option is to work her scum contacts. The

% gamemaster determines that this will be another

% Level 4 favor. Rolling against a target number of


%                                                      EXAMPLE





% 40 again, she gets a 13—her luck is holding.


% She finds a hacker, but now she needs to make

% an exchange for their services. Once again she

% decides not to spend credit and use her @-rep

% instead. Jaqui’s already used up both her Level 4

% and Level 5 @-rep favors, though, so she has no

% choice but to burn reputation. A Level 4 favor costs

% 10 Rep to burn. Jaqui spends it, sending her @-rep

% from 53 to 43—she’s been pulling in a lot of big

% favors in a short amount of time, and her friends

% and acquaintances are expressing their annoyance

% by lowering her social standing.


% KEEPING QUIET
% The problem with using social networks for favors is
% that you end up letting lots of other people know what
% you’re up to. When you’re involved in a clandestine
% operation, that could be exactly what you don’t want.
% The only way to diminish this is to take your requests
% to trusted friends and ask them to keep quiet, but this
% diminishes the pool of people at your disposal.

% %%% txt/291.txt

% In game terms, you can try to keep word of what
%  you’re doing quiet, but this makes it harder to get
%  what you need. For every negative modifier you apply
%  to your Networking Test, the same negative modifier
%  applies to anyone making a Networking Test to find
%  out what you’re up to.




%          Revisiting one of our previous examples, we go


%          back to the point where Jaqui was trying to


%          ascertain someone’s hideout location. Because


%          the person she’s after is scum, they’re on a


%          scum ship, and Jaqui is using her Networking:


%          Autonomist skill to find them, there’s a good


%          chance that if she starts asking around to ev-


%          eryone, word might trickle back to the person


%          she’s after. She doesn’t want them to know


%          she’s on their tail, though, so she decides to

% EXAMPLE






%          make her inquiries more discreet. She applies


%          a –20 modifier to her Networking Test, which


%          lowers her target number from 40 to 20. As


%          noted before, she rolls a 21, which is a fail-


%          ure. She spends a Moxie point to flip the roll,


%          though, making it a 12—a success.


%            Because Jaqui took that –20 hit, represent-


%          ing the fact that she was keeping her research


%          quiet, her target will suffer a –20 modifier


%          when he makes his Networking Test to see if


%          he gets word that someone is asking around


%          about his hideout.


%  FAVORS
%  Creative players can undoubtedly come up with many
%  uses for their social networks, but a few of the more
%  common are detailed here.

% Gamemasters should use their discretion as to how
%  much roleplaying interaction and Networking Tests
%  are included in using a social network. For normal
%  goods, straightforward information queries, or small
%  favors, neither dice rolling nor roleplaying may be re-
%  quired. For major requests, interactions with contacts,
%  and mission assistance, dice rolls and/or roleplaying
%  interaction with contacts from the social network
%  should usually occur. Gamemasters may wish to keep
%  track of the NPC contacts in each character’s social
%  networks and make them recurring characters.






%                       FAVORS


%                                 BURNING
% FAVOR LEVEL          TIMEFRAME       REP COST REFRESH RATE
% 1 (Trivial)             1 minute         0           1 hour
% 2 (Low)                30 minutes        1            1 day
% 3 (Moderate)             1 hour          5           1 week
% 4 (High)                 1 day          10           1 month
% 5 (Scarce)               3 days         20          3 months
%  ACQUIRE/UNLOAD GOODS
% Social networks are a good way to find items that
% you can’t buy legally or make at home. Depending
% on who you’re getting the goods from, this will cost
% you credit or require an appropriate Rep score. This
% favor can also be used to sell or give away such items,
% making some money or perhaps even some Rep in
% the process.




%  ACQUIRE/UNLOAD GOODS
% LEVEL   SERVICE

% 1     Acquire/unload item with an expense of Trivial.

% 2     Acquire/unload item with an expense of Low.

% 3     Acquire/unload item with an expense of Moderate.

% 4     Acquire/unload item with an expense of High.

% 5     Acquire/unload item with an expense of Expensive



%  ACQUIRE SERVICES
% When you lack the skills or education you need, or
% you just need another set of arms, you can call out
% to your social network to find someone to help you
% out. If you are looking for someone with a particular
% skill, the result of your successful Networking Test roll
% is the skill rating of the person you find. The higher
% your Networking skill, the better able you are to find
% highly-skilled professionals.



% Cole needs to find an astrobiologist who can help

% him identify an alien critter. He rolls his Network-

% ing: Scientist skill of 50 and gets a 43—a success.


%                                                      EXAMPLE





% He tracks down someone with Academics: Astro-

% biology skill of 43 (his roll) who can help him out.

% When the astrobiologist looks the critter over, the

% gamemaster makes a roll for the NPC using that

% skill of 43.


%  ACQUIRE INFORMATION
% When you can’t find the information online or you
% don’t have the time or capability to look, you can
% turn to people in your social network and tap their
% accumulated knowledge base.

%  REPUTATION AND IDENTITY
% It is important to note that reputation is closely tied
% to identity. If you are undercover and using a fake ID,
% you can’t really call on your Rep score without giving
% yourself away. As a result, many people using false
% identities end up building up a separate set of Rep
% scores for their alter ego.

% Note that since many social network interactions
% take place online, it is possible for someone to secretly
% make use of their real identity while masquerading
% as someone else, as long as they’re careful about it. If
% anyone happens to be spying on their activity via the
% mesh, they stand a chance of being found out.

% %%% txt/292.txt


%       ACQUIRE SERVICES
% VEL   SERVICE


%  Trivial favor: Get someone to perform services for 15


%  minutes. Moving a chair. Browbeating someone. Catching a
%  1


%  ride. Researching someone on line. Borrow 50 credits. Other


%  Trivial cost services.


%  Minor favor: Get someone to perform services for an


%  hour. Moving to a new cubicle. Roughing someone up.
%  2    Loaning a vehicle. Providing an alibi. Healing vat rental.


%  Minor hacking assistance. Basic legal or police assistance.


%  Borrow 250 credits. Other Low cost services.


%  Moderate favor: Get someone to perform services for


%  a day. Moving to a habitat in the same cluster. Serious


%  beatings. Lookouts. Short-distance egocast. Short shuttle
%  3    trip (under 50,000 km). Minor psychosurgery. Uploading.


%  Reservations at the best restaurant ever. Major legal rep-


%  resentation or police favors. Borrow 1,000 credits. Other


%  Moderate cost services.


%  Major favor: Get someone to perform services for a


%  month. Moving a body. Homicide. Getaway shuttle piloting.


%  Industrial sabotage. Large-volume shipping contract on
%  4    bulk freighter. Medium-distance egocast. Mid-range shuttle


%  trip (50,000–150,000 km). Moderate psychosurgery.


%  Resleeving. Get out of jail free. Borrow 5,000 credits. Other


%  High cost services.


%  Partnership: Get someone to perform services for a


%  year. Moving a dismembered body. Mass murder. Major


%  embezzlement. Acts of terrorism. Relocate a mid-size
%  5


%  asteroid. Long-distance egocast. Long-range shuttle trip


%  (150,000 km or more). Borrow 20,000 credits. Other


%  Expensive cost services.






%  ACQUIRE INFORMATION
% LEVEL SERVICE


%  Common Information: Where to eat. What biz a certain
%  1


%  hypercorp is in. Who’s in charge.


%  Public Information: Make gray market connections.


%  Where the “bad neighborhood” is. Obscure public
%  2


%  database info. Who’s the local crime syndicate. Public


%  hypercorp news.


%  Private Information: Make black market connections.


%  Where an unlisted hypercorp facility is. Who’s a cop. Who’s
%  3


%  a crime syndicate member. Where someone hangs out.


%  Internal hypercorp news. Who’s sleeping with whom.


%  Secret Information: Make exotic black market


%  connections. Where a secret corp facility is. Where
%  4


%  someone’s hiding out. Secret hypercorp projects. Who’s


%  cheating on whom.


%  Top Secret Intel: Where a top secret black-budget
%  5    lab is. Illegal hypercorp projects. Scandalous data.


%  Blackmail material.

% %%% txt/293.txt
% SECURITY
% Firewall sentinels make a regular habit of being in
% places where they are not supposed to be and bring-
% ing things with them that others would prefer they
% not have. Security has a different character post-Fall
% than in the 21st century. Due to hyper-abundance,
% physical security measures such as locks, doors, and
% walls are less important than in the past to common
% citizens. People don’t worry about theft as much as
% in the past because most items can be replaced by a
% nanofabricator. The items that do tend to engender
% this type of security are irreplaceable or rare items
% such as artifacts of Earth.

% Post-Fall physical security focuses heavily on
% surveillance—identifying intruders and tracking
% them so that they can be interdicted by transhuman
% or robotic defenders. Surveillance is more effec-
% tive than in pre-Fall societies because AIs with
% near-human faculties of pattern recognition and
% indentured infomorphs can be employed to monitor
% surveillance data.

% The emphasis on surveillance results from the ease
% with which most material barriers can be breached
% by high-powered hand weaponry and devices like
% the covert operations tool (p. 315). However, physi-
% cal barriers designed to actively resist intruders by
% healing themselves or attacking tools used to damage
% them are used at key points in secure installations.
% Such barriers are typically very expensive and so are
% used sparingly.

% Transhuman, animal, and infolife defenders are
% cornerstones of most security systems. The avail-
% ability of a huge pool of infomorph labor to guard
% facilities means that someone is always on duty,
% whether as part of the surveillance system or in a
% robotic shell.

% ACCESS CONTROL
% The first step in any security system is simply to enact
% measure to keep unwanted people out. At a basic level
% this involves walls, locks, fencing, defensive landscap-
% ing, security lighting, and entoptic warnings.

% Barriers of different sorts present an obstacle that
% must be cut through or blown apart in order to
% defeat. Barriers are treated just like other inanimate
% objects for purposes of attack sand damage; see Ob-
% jects and Structures, p. 202.

% BUG ZAPPERS
% Bug zappers create minute EMP pulses that are harm-
% less to most electronic equipment and implants but
% wreak havoc on nanobot swarms, microbugs, and
% specks. Bug zappers are generally applied to surfaces,
% and as such they only destroy floating/flying swarms
% or specks if they land. In areas with heavily shielded
% electronics, they may be installed to destroy targets
% in an entire room. A zapper instantly destroys all
% free-crawling or flying nanobots and specks in a room
% when it goes off, but transhuman flesh is sufficient to
% prevent it destroying medichines or other implanted
% nanobots. Infiltrators trying to gather data in areas
% protected by zappers generally resort to going around
% them or trying to plant macroscale devices.

% ELECTRONIC LOCKS
% Electronic locks (e-locks) are commonly used as a
% means of maintaining privacy. They are easy to defeat,
% however, and so are rarely used in very secure areas.
% E-locks have several advantages over old-fashioned
% mechanical locks. Different users can have different
% authentication methods, they can log all events (entry,
% exit, failed authentications), and they can be con-
% nected (usually hardwired but sometimes encrypted
% wireless) to security systems for remote control and
% alarm triggering.

% E-locks use one of several authentication systems,
% or sometimes a combination of systems:

% Biometric: The lock scans one or more of the user’s
% biometric prints. Common biometrics include DNA,
% facial thermographic, fingerprint, gait, hand veins, iris,
% keystroke, odor, palm, retinal, and voice prints.

% Keypad: This is an alphanumeric keypad upon
% which users enter a specific code. Different users can
% have different codes.

% Token: Authorized users must carry some sort of
% physical token that interacts with the lock to open the
% door, such as a keycard, electronic key, etc.

% Wireless Code: Users must emit a cryptographic
% code via near-proximity wireless signal.

% Though various technologies exist to defeat each
% of these systems, there are three methods that work
% against almost all e-locks. The first is use of a covert
% operations tool (p. 315), which infiltrates a lock with
% nanobots that swarm in and engage the electronic
% mechanism. The drawback to using a COT is that its
% use is immediately logged by the e-lock and an alarm
% is triggered. Some e-locks are equipped with guardian
% nanoswarms (p. 329) to defeat COTs, but the COT
% nanobots usually manage to open the lock before the
% guardians eat them.

% The second method is to hack the e-lock. Most
% e-locks are slaved to a security system, so this often
% means intruding into the security system and then
% opening the lock from within. This can be difficult,
% however, especially if the security system is wire-
% lessly isolated or hardwired. The advantage is that,
% if done right, all evidence of the lock being opened
% can be erased.

% The third method is to physically open and ma-
% nipulate the lock. This requires first opening the
% lock’s case and then triggering the lock mechanism
% to open the door. Both of these are handled as
% separate Hardware: Electronics Task Actions with
% a timeframe of 1 minute each. In addition, most e-
% locks have anti-tamper circuits that will set off an
% alarm if the attacker does not achieve an Excellent
% Success when opening the case.

% %%% txt/294.txt
% LOCKBOTS
% The 21st century saw a move from mechanical
% locks to e-locks and other largely electronic locking
% mechanisms. These devices worked well for about
% 50 years, until electronic infi ltration capabilities
% rendered them largely useless. The more recent de-
% velopment of lockbots has more in common with
% their early mechanical forebears. They are unique,
% expensive, artisan items.

% A typical lockbot is heavily integrated with the
% portal and barrier it protects. Lockbots usually
% include an AI or indentured infomorph, self-healing
% materials (treat as a self-healing barrier), and a
% swarm of guardian nanobots (p. 329). A lockbot
% monitors its surroundings and has visual recogni-
% tion software that knows what its users and its keys
% look like (Perception skill 40). Picking a lockbot is
% thus incredibly difficult, because it will shut its ori-
% fice and not accept a key that doesn’t look right or
% that comes from an unrecognized user. Unfamiliar
% nanobots trying to enter the orifice are targeted and
% destroyed by the guardian nanobots. Finally, exter-
% nal tools used to harm the portal or the lock will be
% attacked by fractal appendages extruded from the
% portal surface or the lock itself. These appendages
% have a range of 1 meter, attack with skill of 40, and
% inflict 1d10 +2 DV.

% Lockbots are generally immune to being hacked
% because, for security, they aren’t connected to
% the mesh. If attacked, however, lockbots are pro-
% grammed to send out an alarm signal via the mesh.

% There are several ways to defeat a lockbot. One
% is to get a copy or image of the key and then forge
% a copy (using nanofabrication). Another is to attack
% the lockbot or the portal it guards with so much
% force that the lockbot is unable to repair it (usually
% using ranged weapons, as anything within a meter
% of a lockbot may be counterattacked). A third is to
% somehow image the cavity beyond the lockbot’s ori-
% fice without the imaging device being destroyed and
% to then forge the key. All of these are difficult and
% time-consuming processes.

% Some lockbots have the ability to destroy what
% they’re protecting. For example, lockbots are a
% common protection for the physical interfaces to
% hardwired networks. If the lockbot is compromised,
% it may, as a last resort, destroy the interface it was
% protecting.

% PORTAL DENIAL SYSTEM
% Installed in corridors or doorways, this is essentially
% a laser trap device. When an unauthorized person
% enters the portal denial system’s area, it uses lasers
% to create a grid of plasma channels that are used to
% deliver a powerful electric current to the target. This
% system has both lethal and nonlethal settings.

% Nonlethal: 1d10 DV + shock (p. 204)

% Lethal: 2d10 +5 DV
% SELF-HEALING BARRIERS
% Walls and doors that are able to rapidly repair them-
% selves are sometimes found in high security instal-
% lations. These barriers are made of materials that
% automatically expand to “heal” small holes and that
% are equipped with nanosystems that slowly repair
% larger amounts of damage. The best of these barriers
% do no more than slow down the most determined
% assailants, but in combination with surveillance sys-
% tems they are a nuisance to invaders and can slow
% down attempts to flee the scene.

% Self-healing barriers heal any single source of
% damage that is less than 5 points of damage almost
% immediately, sealing the hole in 1 Action Turn. They
% will also seal the holes infl icted by a covert ops
% tool (p. 315) in the same time period. Additionally,
% these barriers repair larger themselves at the rate of
% 1d10 damage per 2 hours; once all damage is fixed
% any wounds are repaired at the rate of 1 per day.
% Damage of 3 wounds or more may not be repaired
% by self-healing.

% SLIPPERY WALLS
% On planetary surfaces, high walls and fences are still
% common as a first line of defense against interlop-
% ers. Slippery walls are surface treated with the slip
% chemical (p. 323), creating a virtually frictionless
% surface that is exceptionally difficult to climb.

% WIRELESS INHIBITORS
% Wireless inhibitors are simple paint jobs or construc-
% tion materials that block radio signals. They are
% used to create a contained area in which a wireless
% network may operate freely without worry that the
% signals will escape out of the area, where they can
% be intercepted. Wireless inhibitors allow the conve-
% nience of using wireless links within a secure area
% rather than the clumsier hardwired connections. If
% an intruder manages to gain access inside the area,
% however, they can intercept, sniff, and hack wireless
% devices as normal.

% DETECTION AND SURVEILLANCE
% Should security measures fail to keep an intruder
% out, the second step is to detect an interloper and
% track their activity.

% NANOTAGGING
% A lot of post-Fall security centers not around keeping
% people out of private spaces, but tracking them after
% they come and go. What little privacy transhumans
% have, they cherish. Trespassing is a worse offense
% than theft in many places.

% A room protected by a taggant nanoswarm (p. 329)
% usually has two or more hives, one each at floor and
% ceiling level (if in gravity; on the opposite side of the
% room if in microgravity) that generate and recycle
% nanobots. The taggants emerge from one hive, float
% through the room, and then return to the other for
% recharging and reuse. A feed line usually connects the

% %%% txt/295.txt
% hives so that they can share materials and power.

% Anyone passing through the room is likely to be
% dosed with taggant nanobots. Once they lose proxim-
% ity to the rest of the hive, they hide and periodically
% broadcast pulsed transmissions meant to give their po-
% sition to pursuers or investigators. Some may drop off
% in clusters to form a breadcrumb trail to the interloper.

% SENSORS
% Any of the various sensors described in the Gear
% chapter (p. 294) may be deployed within a facility
% to monitor and record the passage of personnel,
% both authorized and not. These sensors are typically
% slaved to the facility’s security network and closely
% monitored by security AIs, meaning they are vulner-
% able to hacking and possibly jamming. A few other
% sensor types deserve mention here:

% Chemical Sniffers: The chem sniffer described on
% p. 311 can also be set to detect the carbon dioxide
% exhaled in transhuman breaths. This is useful for
% detecting intruding biomorphs in areas that are
% abandoned/off-limits.

% Electrical Sensors: Electrical sensors can be set in
% portals to detect a biomorph’s electromagnetic field
% in addition to the electrical fields of synthmorphs.

% Heartbeat Sensors: These sensitive sensors detect
% the vibration caused by transhuman heart beats.
% They can even be used to detect the heartbeats of
% passengers inside a large vehicle.

% Seismic Sensors: Embedded in flooring, these sen-
% sors pick up the pressure and vibration of weight
% and movement.

% WEAPON SCANNERS
% Weapon scanners come in several varieties, includ-
% ing those that look for the rare elements used in
% extremely destructive weapons such as nukes, those
% that attempt to locate personal weaponry, and those
% that look for detection taggants.

% Rare element scanners are nearly flawless and are
% ubiquitous in habitat customs and spaceports. The
% only way to circumvent them is to find an alternate
% route into the protected area.

% Personal weapon scanners can monitor a specific
% area, such as a small room or doorway. They use
% a number of sensing systems to detect and identify
% weapons and other dangerous objects, including
% chemical sniffers and radar/terahertz/infrared/x-ray/
% ultrasound imaging. They can detect the following
% items and substances:

%  • Metal used in kinetic weapons, seekers, and

% flechette weapons
%  • Devices with onboard hives of metallic nano-

% bots (e.g., covert operations tools, spindles)
%  • Magnetic elements in plasma guns and railguns
%  • Propellant from firearms ammunition and seek-

% ers (–30 to conceal)
%  • Chemical fuels used in torch spray weapons

% (–30 to conceal)
%  • All explosives and grenades by their chemical

% particulate emissions (–30 to conceal)
%  • Poisons and bioagents in flechette weapons
%  • Any weapon or device larger than palm size

% (using sound waves and shape recognition)


% Characters trying to sneak weapons and gear past
% personal weapon scanners must make a Palm-
% ing Test (if concealing) or an Infiltration
% Test (if somehow maneuvering around
% without notice). This is opposed by a
% Perception Test from the character or
% AI manning the sensor system.

% WIRELESS SCANNING
% Some high-security areas will in-
% tentionally monitor for wireless
% radio signals originating within
% their area as a way of detect-
% ing intruders by their com-
% munications emissions. These
% signals can even be used to
% track the intruder’s location via
% triangulation and other means
% (see Physical Tracking, p. 251).
% To bypass wireless detection sys-
% tems, covert operatives can use
% line-of-sight laser links (p. 313)
% for communication or touch-based
% skinlinks (p. 309).

% ACTIVE COUNTERMEASURES
% When all else fails, active counter-
% measures may be deployed against
% intruders. While live transhu-
% man guards are sometimes
% used, robotic sentries are more
% common, typically AI-driven
% synthmorphs such as synths,
% slitheroids, arachnoids, or
% reapers, with guardian angels
% (p. 346) providing air support.
% Occasionally AI-operated gun
% emplacements—armored tur-
% rets that pop out of walls and
% ceilings—are also applied. In
% some circumstances, these shells
% are teleoperated or even jammed
% by transhuman security.

% dditional countermeasures brought
% to bear will depend on the facility in
% question. Some sites will engage in active
% jamming, to deny the intruders any com-
% munication. Others will deploy hostile nano-
% swarms and even chemical weapons.

% %%% txt/296.txt
% GEAR
% ear in all the usual ways: buying, trading,
% or making. ■ p. 297





%         Fabricating Gear: With access to a cornucopia mach


%                         able to build their own equipment, give





%                     PERSONAL AUGMENTATI

% The majority of transhumans are augmented—mentally or phys


%         with biological, cybernetic, or nanotechnology mods. ■




%          Standard Augmentations: ■ p. 300



%                             Bioware: ■ p. 300



%                          Cyberware: ■ p. 306



%                           Nanoware: ■ p. 308



%                     Cosmetic Mods: ■ p. 309



%            Robotic Enhancements: ■ p. 310


%                        GEAR
% r another nanofab device, a character may be
% e right blueprints and raw materials. ■ p. 297



% NS
% y—
% 297






%                           11

% %%% txt/297.txt


%           OTHER GEA
% Characters will find many other typ


%                  of gear usefu




%     Armor and Armor Mod



%                      Commu



%           Covert and Espio



%          Drugs, Chemicals, a



%                         Ever



%                      Nanote





%                        Scave






%                          Surv



%                   Robots an
% pp. 311 and 313

% ations ■ p. 313

% e Tech ■ p. 315

% Toxins ■ p. 317

% y Tech ■ p. 325

% ology ■ p. 326


% Pets ■ p. 330

% r Tech ■ p. 330

% rvices ■ p. 330

% tware ■ p. 331

%  Gear ■ p. 332

% hicles ■ p. 342

% %%% txt/298.txt
% ■GEAR■GEAR■GEAR■GEAR■GEAR■G




%  The accelerated technological levels of Eclipse Phase
%  enable a number of devices for personal enhancement,
%  survival, and other uses.



%  EQUIPMENT RULES
%  The following rules apply to all technological items in
%  Eclipse Phase.

%  ACQUIRING GEAR
%  During character creation, players purchase gear for
%  their characters using the credits they have during
%  the character creation process. Once play begins,
%  however, characters must obtain any equipment they
%  need the usual way: by buying, borrowing, making,
%  or stealing it.

%  In the inner system, hypercorp, and Jovian Republic
%  settlements—and other places where capitalism still
%  reigns—gear acquisition is simply a matter of finding
%  a seller and buying it. Each item has a listed cost, from
%  Trivial to Expensive, as noted on the Gear Costs table.
%  Due to local availability of resources, supply and
%  demand, and legalities, these listed costs are meant
%  to be approximations. When no other factors apply,
%  the listed Average Cost for that category can be used.
%  Otherwise the gamemaster should modify the item’s
%  worth as they see fit, according to local economic fac-
%  tors, while still keeping it within that cost category
%  range. The Cost Modifiers table lists out some sug-
%  gested changes to an item’s cost, but these are simply
%  recommendations, and can be ignored or followed as
%  the gamemaster deems fit. The exact local conditions
%  are largely up to the gamemaster to determine, as best
%  fits their game.

%  In some circumstances, characters may attempt to
%  haggle over gear prices. This is best handled as ro-
%  leplaying, but the gamemaster may also call for an
%  Opposed Persuasion Test (or possibly an Intimidation
%  Test). The character who wins may increase or reduce
%  the price by 10% per 10 points of MoS.

%  In the outer system, anarchist, Titanian, scum, and
%  other habitats that use the reputation economy, charac-
%  ters must rely on their rep scores to acquire the goods
%  and services they need. The mechanics for this are cov-
%  ered under Reputation and Social Networks, p. 285.

%  Characters are of course free to get their hands
%  on equipment by any other means they devise—con
%  schemes, borrowing from friends, and outright
%  theft, with all of the appropriate tests and conse-
%  quences. In some cases, acquiring gear may be an
%  adventure unto itself.

%  FABRICATING GEAR
%  Thanks to nanofabrication technology, characters may
%  also create their own equipment using cornucopia
% AR■GEAR■GEAR■GEAR■GEAR■GEAR■


%                  ■GEAR■


%                   GEAR■



%                                                           11



% machines and similar nanofab devices (p. 327). The

% character must have the appropriate blueprints to do

% so, whether they come with the fabber, are bought

% legitimately or on the black market, acquired with rep,

% or found online. Characters may also code their own

% blueprint desires, using the Programming: Nanofab-

% rication skill.


% GEAR MODIFIERS

% In the technological future, gear is a necessity. In many

% cases, use of equipment provides no bonuses, it simply

% allows a character to perform a task they would oth-

% erwise be unable to do. For example, it is impossible

% to pick a mechanical lock without lockpick or some

% sort of tool.


% In other cases, however, gear provides a bonus to

% the task at hand. Climbing a wall may be possible

% without tools, but if you happen to have gecko gloves

% or other climbing gear, it’s going to be a lot easier. The

% specific modifier applied is usually noted in the gear

% item’s description, typically ranging from +10 to +30.


% GEAR QUALITY

% In both of the situations above, it is possible to have

% items that are of either exceptional or inferior quality,

% with corresponding positive or negative modifiers. The

% gear may be well-crafted, state-of-the-art, cutting-edge

% experimental, or simply top-of-the-line, applying an

% additional +10 to +30. Or it may be outdated, shoddy,

% or in disrepair, inflicting a –10 to –30 modifier (in

% some cases canceling out the basic gear bonus).





%                   GEAR COSTS
%  CATEGORY RANGE (IN CREDITS)          AVERAGE (IN CREDITS)
%  Trivial        1–99                  50
%  Low            100–499               250
%  Moderate       500–1,499             1,000
%  High           1,500–9,999           5,000
%  Expensive      10,000+               20,000





%               COST MODIFIERS
%  ECONOMIC FACTOR              SUGGESTED COST MODIFIER
%  Item Stolen                  –50%
%  Item Used                    –25%
%  Item Restricted              +25%
%  Item Illegal                 +50%
%  Item Scarce                  +25%
%  Item Extremely Rare          +50%
%  Item Common                  –25%

% %%% txt/299.txt
% GEAR SIZES
% On occasion, you’ll need to know how small or large
% a certain piece of equipment is. Though this is largely
% something the gamemaster can wing on the fly using
% common sense, we’ve listed sizes for many gear items
% that are unusual or so futuristic that the average player
% may not have a feel for what dimensions the tech
% likely is. These size categories are listed on the Gear
% Sizes table (p. 297). These sizes should be considered
% approximations, as depending on the manufacturer
% and process, some items may be smaller or larger than
% similar items. It is also important to keep in mind that
% as technology advances, the size and components of
% various equipment items shrink, so when in doubt, go
% with smaller.

% MASS AND ENCUMBRANCE
% A character who is carrying too much gear should
% be slowed down, suffering negative modifiers both
% to their movement rates and their skill tests. Rather
% than micromanaging the weights of individual pieces
% of equipment, however, this matter is largely left to
% the gamemaster’s discretion, using common sense. If a
% character loads up beyond reason, apply modifiers as
% seem appropriate. The gamemaster should, however,
% keep in mind that many of the manufacturing mate-
% rials used in Eclipse Phase allow for items that are
% much lighter than current standards without any loss
% of durability or function (see Future Materials, p. 298).
% Likewise, characters in low or microgravity environ-
% ments can carry much larger loads.

% CONCEALING GEAR
% Characters may attempt to conceal items on their
% person, hoping at least to hide them from casual
% notice if not an intensive search. To determine how





%             GEAR SIZES
% SIZE CATEGORY GENERAL DIMENSIONS AND NOTES


%            So small that the item cannot be seen without


%            the aid of a microscope or nanoscopic vision
% Nano


%            (p. 311), and may not be manipulated without


%            fractal digits (p. 311) or similar tools.


%            Anything ranging from the size of a barely
% Micro


%            visible small dot to an average insect.


%            Mini items may be concealed within some-
% Mini


%            one’s palm or small pockets.


%            Small items may be held in one hand and
% Small


%            concealed in normal pockets.


%            Medium size items are cumbersome to hold


%            with one hand, ranging from the size of a 2-li-
% Medium          ter bottle to the size of a medium dog. They do


%            not fit in pockets, but they may be concealed


%            by larger coverings.
% Large           Roughly human-sized.
% Huge            Vehicles and other more massive objects.
% effectively the character conceals the equipment, make
% a Palming Test and note the MoS (the gamemaster
% may wish to roll this secretly). Whenever another
% character has a chance to notice the concealed item,
% they must succeed in a Perception Test and achieve
% a higher MoS than was scored on the Palming Test.
% The gamemaster should apply modifiers to both tests
% as appropriate. For example, concealing a large item
% like a sword would be difficult (–30), whereas wearing
% concealing clothing like a longcoat or multi-pocketed
% jumpsuit would help (+20). Likewise, a character
% who is not actively looking is less likely to notice the
% hidden gear (–30), whereas someone who conducts a
% physical search (+30) or who has enhanced vision to
% pierce protective layers will fare better.

% DESIGN AND FASHION
% Many objects in Eclipse Phase closely resemble their
% early 21st century equivalents—a bottle of soda is still
% a transparent container holding a brightly colored
% liquid, clothing is obviously something you wear, and
% a knife still consists of a blade and a handle. The ma-
% terials, processes, and mindsets that go into making
% them, however, are quite different. To start, very few
% items look have a uniform, mass-produced look, even
% if they were. The procedures of minifacturing and
% nanofabrication allow every individual item to be
% manufactured with a unique (or at least different)
% look. In areas with anarchist/reputation economies, in
% fact, where personal possessions have very little intrin-
% sic value, expression and creativity are favored and so
% many items are artistically personalized (and actual
% hand-crafted items are rare and prized). Likewise,
% almost all equipment is designed with ergonomics
% and ease-of-use prioritized, so gear with soft curves,
% pleasing colors, and form-fitting shapes are common.
% Many items of personal technology, such as flashlights
% or small tools, are made in the form of ovoids that
% fit comfortably in the user’s hand or in similar forms
% that can be easily worn or attached to clothing. To
% someone from the 20th century, many common
% devices look like oddly colored rocks or decorative
% pieces of plastic or ceramic (in fact, many such items
% are referred to as “blobjects” by older transhumans).

% The materials used to create everyday items are
% also advanced, ranging from aerogel and graphene
% to smart materials (p. 298) and exotic metamateri-
% als with unusual physical properties. In practice, this
% means that most items are light, durable (with both
% tensile strength and/or flexibility, as needed), water-
% proof, dirt-repellent, and self-cleaning. Most gear is
% also designed with zero-G or microgravity function-
% ality in mind, and can easily be clipped, tethered, or
% stuck to a surface with grip pads.

% Almost all gear available in Eclipse Phase is also
% available in forms that are wearable/usable by up-
% lifted animals and non-humanoid morphs, such
% as novacrabs, slitheroids, and so on. Even if such
% customized gear is not immediately available, it is

% %%% txt/300.txt
% FUTURE MATERIALS
% Many materials are available and commonly used in
% Eclipse Phase that are rare, theorized, or unheard-
% of today. The following entries note some of the
% more interesting.

% AEROGEL
% Low-density, solid-state “Frozen smoke” is made by
% carefully foaming various materials, typically glasses
% or ceramics, to an ultra-low density state. Aerogel
% is semi-transparent and light-weight, feels like sty-
% rofoam, but acts as an incredible insulator against
% heat and cold. It is commonly used in habitats.

% DIAMOND
% Artificial diamond is lightweight and super-strong,
% has an extremely high melting point, and has near-
% perfect thermal conductivity. This makes it an ideal
% substance for hardening coated surfaces (armor)
% and creating super-tough diamond machinery.

% FULLERENES/FULLERITES
% Fullerenes are molecular carbon structures (known
% as buckyballs, carbon nanotubes, and graphene)
% that are extremely strong (vastly stronger by weight
% than steel), heat-resistant, and can be either insula-
% tive or superconductive. This makes them useful in
% equipment as diverse as armor, electronics, sensor
% systems, or the cables of space elevators.

% METALLIC FOAM
% Metal foam is created by adding foaming agents
% to liquid metals, resulting in extremely lightweight




% usually not difficult to nanofabricate. Smart materials
% (p. 298) also make interoperability between different
% morphs easy.

%  INTERFACE
% It is not uncommon for everyday devices to have no
% visible controls as they are designed to be operated via
% radio broadcasts from the user’s ecto or mesh inserts.
% Any items crafted for use in emergency, combat, sur-
% vival, or exploration situations, however, will feature
% basic physical controls, just in case. Physical interfaces
% are typically controlled by touch pads that are noth-
% ing more than colored spots on the device’s surface,
% though some may also project a holographic interface
% display. Most equipment of this sort can can also be
% voice-activated and controlled.

% Almost all devices are loaded with a complete set
% of help files and tutorials. Most electronics are also
% mesh-capable and equipped with specialized AIs (see
% Meshed Gear, next page).
%  etallic structures—light enough to float on water.
%  eal for habitat construction and floating cities.

% METALLIC GLASS
%  etallic glass are metals highly alloyed to possess a
%  sordered (rather than crystalline) atomic structure
%  ith unique combinations of stiffness and strength,
%  aking it a good wear surface and alternative to
% eramics in armor. It is also useful for its unusual
% or a metal) electrical resistance properties.

% METAMATERIALS
%  etamaterials have unusual physical properties
%  sually electromagnetic) due to their structure,
% uch as having a negative refractive index. Metama-
%  rials are used to create invisibility cloaks (p. 316),
% uperlenses, phased array optics, and impressive
%  D holograms.

%  EFRACTORY METALS
% hese metallic alloys have extremely high melt-
%  g points, making them ideal for extremely hot
% ngine systems, atmospheric entry vehicles, and
% ypersonic craft.

% RANSPARENT ALUMINA
%  transparent form, this ceramic is often known as
%  pphire. Transparent alumina is harder than steel
% nd zero-g casting techniques allow for intriguing
% ansparent construction designs, so long as its poor
%  nsile strength is respected.                    ■





% SMART MATERIALS

% Many common items of technology are made from

% so-called smart materials. These devices contain—or

% sometimes consist entirely of—many small nanoma-

% chines that can both move and reshape themselves

% to alter the object’s shape, color, and texture. For

% example, smart clothing can transform from a suit of

% specialized cold weather clothing suitable for the Mar-

% tian poles in winter to a fashionable suit in the latest

% style due to hundreds of thousands of tiny nanoma-

% chines in the clothing that shift and move to reshape

% the garment. Similarly, a tool made of smart materials

% can switch from a powered screwdriver to a wrench

% or a hammer, as the nanomachines move around and

% completely reshape the tool. Smart materials all con-

% tain specialized advanced nanomachine generators (p.

% 328) that keep them in perfect repair as long as they

% are regularly recharged.

% %%% txt/301.txt
%  MESHED GEAR
% Almost all technology in Eclipse Phase is designed
% to be operated via radio signals from the user’s basic
% implant, although models usable by characters with-
% out basic implants are also available. In addition all
% devices contain a nearly microscopic computer and
% radio link (known as a “voice”) that allows the user to
% easily locate the object and that reports on the condi-
% tion of the object or device, how to properly use and
% care for it, as well as telling the user when it needs
% to be repaired and how. Most are discrete and highly
% useful, but cheaply made goods sometimes have overly
% annoying voices.

% This means that almost all devices can be accessed
% via the mesh or directly if within radio range. This
% makes them vulnerable to hacking and intrusion at-
% tempts (p. 254) as well as radio jamming (p. 262).
% Many devices are, however, publicly accessible (see
% Spimes, p. 238). Meshed gear may also be tracked
% through the mesh (p. 251). For privacy and security,
% these devices are often slaved to other systems (see
% Slaving Devices, p. 248); devices worn/carried by




%                             RADIO AND SE
% SIZE CATEGORY           URBAN RANGE                   OP
% Nano                       20 meters                      1
% Micro                      50 meters                      5
% Mini                      1 kilometer                     20
% Small                     5 kilometers                    50
% Medium                   25 kilometers                25
% Large                    500 kilometers              5,00

% characters are usually made part of the personal area

% network and slaved to the character’s mesh inserts/

% ecto. For more info on meshed devices, see the Mesh

% chapter, p. 234.


% Many devices come equipped with AIs, who are

% equipped with skillsofts that enable them to oper-

% ate the device on their own, as according to voiced

% instructions or commands issued through the net. AIs

% are described on p. 264 and p. 331.


% RADIO AND SENSOR RANGES

% In Eclipse Phase, almost all devices are equipped with

% small radios so that they may be meshed. Likewise,

% many pieces of gear are equipped with sensors such as

% cameras, microphones, or other detectors. The Radio

% and Sensor Ranges table notes what range these de-

% vices operate at.


% POWER

% All of the powered devices in Eclipse Phase require

% electricity to function. With rare exceptions, most of

% them rely on either solar cells or powerful batteries.

% These batteries are high-density, room-temperature


% NSOR RANGES
% RANGE        EXAMPLES
% meters       Smart Dust, Nanobot/Microbot Swarms
% meters       Microbugs
%  meters      Mesh Inserts
%  meters      Ectos, Miniature Radio Farcasters, Portable Sensors
% ometers      Radio Boosters, Vehicle Sensors
% lometers

% %%% txt/302.txt
% superconductors with 25 times the capacity of the
% best batteries in common use in the early 21st century.
% Such batteries may also be constructed so that they
% are flexible, printed on devices, or woven into fabric.
% They are good for 100–500 hours of use, and will alert
% the user when they start running low. More power-
% ful radio-isotope nuclear batteries are also available,
% heavily shielded so they do not emit radiation and
% good for 3 years or more of use.

% In short, power should rarely be an issue in Eclipse
% Phase games, unless it happens to fit the plot. Power
% failure could also result from a critical failure roll.



% PERSONAL AUGMENTATION
% Almost all citizens of the solar system, whether human,
% AI, or uplifted animal, use various forms of biological,
% cybernetic, or nanotechnological augmentation. The
% following is a list of the most common types.

% Unless otherwise noted, any bonuses from personal
% augmentations are both compatible and cumulative
% with bonuses from other enhancements.

% STANDARD AUGMENTATIONS
% Most morphs produced in the solar system include the
% following augmentations.

% BASIC BIOMODS
% Almost universal in biomorphs, many habitats will not
% allow individuals to visit/immigrate if their biomorph
% does not possess these biomods in order to preserve
% public health. Basic biomods consists of a series of
% genetic tweaks, tailored virii, and bacteria that speed
% healing, greatly increase disease resistance, and impede
% aging. A morph with basic biomods heals twice as fast
% as an early 21st century human, gradually regrows lost
% body parts, is immune to all normal diseases (from
% cancer to the flu), and is largely immune to aging. In
% addition, the morph requires no more than 3-4 hours
% of sleep per night, is immune to ill-effects from long-
% term exposure to low or zero gravity, and does not
% naturally suffer from biological problems like depres-
% sion, shock reactions after being injured, or allergies.
% [Moderate, but included for free in most biomorphs]

% BASIC MESH INSERTS
% Mesh inserts are ubiquitous among modern morphs.
% This network of cybernetic brain implants is essential
% equipment for anyone who wants to stay connected
% and make full use of the wireless mesh. The intercon-
% nected components of this system include:

%  • Cranial Computer: This computer serves as the

% hub for the character’s personal area network

% and is home to their muse (p. 264). It has all of

% the functions of a smartphone and PDA, acting

% as a media player, meshbrowser, alarm clock/

% calendar, positioning and map system, address

% book, advanced calculator, file storage system,

% search engine, social networking client, messaging

% program, and note pad. It manages the user’s aug-

% mented reality input and can run any software the

% character desires (see Software, p. 331). It also

% processes XP data, allowing the user to experi-

% ence other people’s recorded memories, and also

% allowing the user to share their own XP sensory

% input with others in real-time. Facial/image rec-

% ognition and encryption software (p. 331) are

% included by default.
%  • Radio Transceiver: This transceiver connects the

% user to the mesh and other characters/devices

% within range. It has an effective range of 20 kilo-

% meters in deep space or other locations far from

% radio interference and 1 kilometer in crowded

% habitats.
%  • Medical Sensors: This array of implants monitors

% the user’s medical status, including heart rate,

% respiration, blood pressure, temperature, neural

% activity, and much more. A sophisticated medical

% diagnostic system interprets the data and warns

% the user of any concerns or dangers.


% Using any of these functions is as easy as thinking.
% [Moderate, but included for free in most morphs]

% CORTICAL STACK
% A cortical stack is a tiny cyberware data storage unit
% protected within a synthdiamond case the size of a
% grape, implanted at the base of the skull where the
% brain stem and spinal cord connect. It contains a
% digital backup of that character’s ego. Part nanoware,
% the implant maintains a network of nanobots that
% monitor synaptic connections and brain architecture,
% noting any changes and updating the ego backup in
% real time, right up to the moment of death. If the char-
% acter dies, the cortical stack can be recovered and they
% may be restored from the backup (see Resleeving, p.
% 271). Cortical stacks do not have external or wireless
% access (for security), they must be surgically removed
% (see Retrieving a Cortical Stack, p. 268). Cortical
% stacks are extremely durable, requiring special effort
% to damage or destroy. They are commonly recovered
% from bodies that have otherwise been pulped or man-
% gled. Cortical stacks are intentionally isolated from
% mesh inserts and other implants, as a security measure
% to prevent hacking or external tampering. [Moderate,
% but included for free with most morphs]

% CYBERBRAIN
% Cybernetic brains are where the ego (or controlling
% AI) resides in synthmorphs and pods. Modeled on
% biological brains, cyberbrains have a holistic architec-
% ture and serve as the command node and central pro-
% cessing point for sensory input and decision-making.
% Only one ego or AI may “inhabit” a cyberbrain at a
% time; to accommodate extras, mesh inserts (p. 300)
% or a ghost-rider module (p. 307) must be used. Since
% cyberbrains store memories digitally, they have the
% equivalent of mnemonic augmentation (p. 307). They
% also have a built-in puppet sock (p. 307) so that they

% %%% txt/303.txt
% may be remote-controlled. Cyberbrains are vulnerable
% to brainhacking (p. 261) and other forms of electronic
% infiltration/attack. Cyberbrains come equipped with
% two or more pairs of external access jacks (p. 306),
% usually located at the base of the skull, which allow
% for direct wired connections. [Moderate, but included
% for free in all synthetic morphs and pods]

% BIOWARE
% Bioware augmentations can be acquired either as a
% genemod when the morph is designed and grown or
% as a later modification to an existing morph, either
% by using nanomachines to modify the morph’s tissue
% or by externally growing the organ and implanting it.
% Bioware may be used to enhance biomorphs (includ-
% ing pods and uplifts), but not synthmorphs.

% ENHANCED SENSES
% The following are a list of the most common enhanced
% senses. Each is also available as a cybernetic implant,
% but bioware is much more common.

% Direction Sense: The character has an innate sense
% of direction and distance using advanced inertial
% navigation. The character can arbitrarily define any
% point as “north” and keep track of which direction
% that is, as well as knowing approximately how far
% they have come. Characters with this augmentation
% can always retrace any route they have taken, only
% experiencing difficulty with three-dimensional routes
% lacking navigational markers (such as deep space or
% undersea; apply a –30 modifier). Since positioning
% inside habitats by anyone with basic mesh inserts is an
% automatic affair, only characters venturing to remote
% locations require this augmentation. [Low]

% Echolocation: The character possesses sonar similar
% to that of a bat or dolphin. The character bounces
% brief ultrasonic pulses off their surroundings and uses
% them to form an image of these surroundings through
% the pattern of reflections of these pulses received by
% the character’s ears. For more details, see Using En-
% hanced Senses, p. 302. This augmentation works in
% both air and water and has a range of 20 meters in air
% and 100 meters in water. [Low]

% Enhanced Hearing: The morph’s ears are enhanced
% to hear both higher and lower frequency sounds—the
% range of sounds they can hear is twice that of normal
% human ears (see Using Enhanced Senses, p. 302). In
% addition, their hearing is considerably more sensitive,
% allowing them to hear sounds as if they were five
% times closer than they are. A character with this aug-
% mentation can easily overhear even a softly spoken
% conversation at another table in a small restaurant.
% This augmentation provides a +20 modifier to all
% Perception Tests involving hearing. [Low]

% Enhanced Smell: The morph’s sense of smell is
% equal to that of a bloodhound. The user can iden-
% tify both chemicals and individuals by smell, and can
% track people and chemically reactive objects by smell
% as long as the trail was made within the last several
% hours and has not been obscured. The character can
% also gain a general sense of the emotions and health
% of any character within 5 meters (+20 to Perception or
% Kinesics Tests to do so). [Low]

% Enhanced Vision: The morph’s eyes have tetrachro-
% matic vision capable of exceptional color differen-
% tiation. These eyes can also see the electromagnetic
% spectrum from terahertz wave frequencies to gamma
% rays, enabling them to see a total of several dozen
% colors, instead of the seven ordinary human eyes can
% perceive. In addition, these eyes have a variable focus
% equivalent to 5 power magnifiers or binoculars. This
% augmentation provides a +20 modifier to all Percep-
% tion Tests involving vision. For further applications,
% see Using Enhances Senses, p. 302. [Low]

% MENTAL AUGMENTATIONS
% Mental augmentations are extremely common.

% Eidetic Memory: The character can remember ev-
% erything that ever happened to them, in detail, with
% no long term memory loss. For example, they can
% recite a page they read in a book a month ago, recall
% a string of 200 random characters they viewed a year
% ago, or even tell you what they had for breakfast on
% a particular date a decade ago. However, they can
% only remember things they paid attention to. The
% character will not remember the contents of a note
% on someone’s desk if they merely glanced at it; they
% must specifically have read it. No effort is required to
% use this augmentation, the character merely needs to
% attempt to remember a specific fact. [Low]

% Hyper Linguist: The morph’s brain maintains the
% linguistic flexibility of a small child, allowing the
% character to learn languages with great ease. This
% functions as the Hyper Linguist trait, p. 146. [Low]

% Math Boost: This implants functions as the Math
% Wiz trait, p. 146. [Low]

% Multiple Personalities: The character’s brain is in-
% tentionally partitioned to accommodate an extra per-
% sonality. This multiplicity is not viewed as a disorder,
% but as a cognitive tool to help people deal with their
% hypercomplex environments. This extra personality
% can be an NPC run by the gamemaster, a separate
% character (in ego form only) made by the player, or
% the downloaded fork of another character. For all in-
% tents and purposes, the extra personality is treated as
% a separate ego (i.e., it may fork separately), except that
% both personalities are backed up in the same cortical
% stack and if downloaded they must be placed in sepa-
% rate morphs or in another morph with this implant.

% Only one ego may be in control of the morph at a
% time. The other resides in the background, still active,
% but not on a surface level. Each ego is completely
% aware of what the other is doing, thinking, etc. If for
% some reason the subsumed personality wants to come
% to the fore, but the other personality won’t relinquish
% control, make an Opposed WIL x 3 Test. Each ego
% has its own Lucidity and Trauma Threshold, and they
% track stress and trauma separately. Any psi attacks or
% social/mental influences only affect the personality at
% the fore.

% %%% txt/304.txt
% USING ENHANCED SENSE
% Personal augmentations and technological aids
% have drastically increased the sensory capabilities
% of most transhumans. The following notes provide
% some details on what capabilities these sensory func-
% tions provide. The capabilities are typically the same
% whether it’s a biological sense or a technological
% sensor, though tech sensors can “turn off” certain
% wavelengths and sense only specific frequencies,
% whereas biological senses perceive the full spectrum
% with no ability to filter parts out.

% SENSORY DATABASES
%  Both technological sensors and enhanced biological
%  senses come equipped with databases of scanned
% “signatures” that make it easier to identify whatever
%  the user is sensing (in the case of bioware, these da-
%  tabases are stored and accessed via the character’s
%  mesh inserts). For example, infrared sensors feature
%  databases listing the heat signatures of different
%  animals and items, making it easier to identify such
%  things. In relevant situations, apply a +20 modifier
%  for identifying targets sensed this way.

% ACTIVE VS. PASSIVE
% An active scanner must actually emit its particular
% frequency and then measure the reflections; this
% means a similar sensor can detect it and home in on
% the emitting source. For example, a character with
% enhanced vision can literally see the terahertz radia-
% tion emitted by someone using an active terahertz
% sensor, much like someone with normal vision can
% see the light emitted by a flashlight.

% A passive scanner simply scans frequencies that occur
% naturally—there is nothing to give the sensor away.

% ELECTROMAGNETIC
% SPECTRUM

% For Eclipse Phase rules purposes, the EM spectrum
% is broken down by wavelength and frequency into
% these categories: radio, microwave, terahertz, infrared,





%  Having an extra ego in your head, working in the

% background, is helpful for multitasking. The character

% receives an extra Complex Action each turn that may

% only be used for mental or mesh actions. [High]


% PHYSICAL AUGMENTATIONS

% Most physical bioware augmentations are derived

% from the capabilities of animals.


% Adrenal Boost: This adrenal gland enhancement

% supercharges the character’s adrenal response to

% situations that invoke stress, pain, or strong emotions

% (fear, anger, lust, hate). When activated, the concen-

% trated burst of norepinephrine accelerates heart rate

% and blood flow and burns carbohydrates. In game

% ible light, ultraviolet, X-rays, and gamma rays.
%  Radar (Radio/Microwave): Radar sensors work by
%  tively emitting radio waves and microwaves and
%  easuring them as they bounce off the target. Radar
%  orks best when detecting metallic objects, and is
%  ss effective (–20 modifier) against biomorphs and
% mall items. Resolution is not high, however, so it can
%  e shapes but not colors or fine details. It can be
%  ed to detect both speed and movement, can “see”
%  rough walls (up to a cumulative Armor + Durabil-

%  of 100), and can detect cybernetic implants or

% ncealed items. At close ranges (1-2 meters), it can

% tect pulse rate and respiration by measuring the
%  otion of the chest cavity.
%  Terahertz: Terahertz sensors emit t-rays, measure
%  e reflections, and compare them to a database

% terahertz signatures that different items/materi-
%  s have. The resolution is higher than radar, but

% th slightly less detail than normal vision. Similar

% radar, terahertz sensors can see through walls

% d other materials, but to a lesser extent (up to a

% mulative Armor + Durability of 50). T-rays occur

% turally, but terahertz sensors normally require an
% mitter as they are absorbed by atmosphere (as well

% water and metal). In space, however, an emitter
%  ould not be required. Likewise, passive terahertz
%  ans within atmosphere have an effective range of

% meters. T-rays do not penetrate skin, so are inef-
%  ctive for locating implants.
%  Infrared: Near-infrared wavelengths are used
%  r night vision, providing resolution and detail

% uivalent to regular vision under low-light condi-
% ons. Mid-long infrared is excellent for detecting

% at sources (unobstructed by fog or smoke) and
%  mperature differences (as small as 0.1 degree C),

% d such thermal imaging will sense the dissipat-
%  g heat traces left by warm sources on colder ones,

% owing the user to see where someone was sitting,
% ace fading heat footprints, or see what buttons
%  ere pressed if they are quick enough. Infrared





% terms, this allows the character to ignore the –10

% modifier from 1 wound and temporarily increases

% REF by +10 (also boosting REF-linked skills and

% Initiative). These modifiers apply until the character

% has calmed down (if the character also has endocrine

% control, p. 304, then adrenal boosts can be activated

% and deactivated at will, and the negated wounds are

% cumulative). [High]


%  Bioweave Armor (Light): Bioweave armor involves

% lacing the morph’s skin with artificial spider silk bio-

% logical fibers. This provides an Armor rating of 2/3

% without changing the appearance, texture, or sensitiv-

% ity of the morph’s skin. This armor is cumulative with

% worn armor. [Low]

% %%% txt/305.txt

% USING ENHANCED SEN

% also detects the blood flow in a biomorph’s face

% which can be useful in judging emotional states

% (+20 modifier to Kinesics Tests), and can spot sub

% surface implants. Some normally white surfaces

% are reflective (mirrored) in infrared, potentially

% allowing an infrared viewer to see around corners

% or behind themselves. On the other hand, some

% glass is opaque to infrared light. Infrared is also

% useful for determining chemical composition

% (enabling Chemistry Tests by sight alone). Infrared

% sensory input is passive.


% Lidar (Visible Light): Similar to radar, but with

% much higher resolution, lidar actively bounces light

% from the infrared through ultraviolet spectrum off

% a target and measures the backscatter, fluorescence

% and other properties. Lidar is very useful for detect-

% ing atmospheric chemical properties and weather.

% Like radar, it can be used to measure a target’s

% range and speed, or develop a three-dimensional

% image. One clever use of lidar is to precisely “map”

% the position of everything in a room (taking several

% turns of scanning) and then check that positioning

% later to see if anything has been moved.


% Ultraviolet: Some objects are fluorescent in ultra

% violet light, including some animals, flowers, insects

% urine, and minerals (which show up much better

% in ultraviolet than regular light). Some plants and

% animals have patterns that can only be seen in ultra-

% violet. Likewise, chemical dyes that only show up

% under ultraviolet, or that make certain substances

% (like blood) fluoresce under ultraviolet light, have

% various security purposes. Some glass is opaque at

% ultraviolet wavelengths.


% X-Ray/Gamma-Ray: Backscatter imaging systems

% using X- and gamma-ray frequencies produce

% high-resolution three-dimensional images and

% are very useful for detecting concealed weapons

% and implants. Such imagers are very good at

% penetrating walls and metal (up to a cumulative

% Armor + Durability of 200, at least at levels safe





% Bioweave Armor (Heavy): Heavy bioweave armor
% involves lacing the morph’s skin with a denser and
% thicker network of the same fi bers. The morph’s
% skin becomes thicker and somewhat less fl exible
% except at the joints. The morph’s skin also has an
% unusually smooth look, and a distinctively smooth
% and tough-feeling texture. This provides an Armor
% rating of 3/4 without decreasing the morph’s mo-
% bility. The character’s sense of touch, however, is
% significantly reduced (–20 modifier) except on their
% hands, feet, and face. This armor is cumulative with
% worn armor. [Moderate]

% Carapace Armor: Carapace armor combines
% bioweave armor with hard but flexible plates of
% ES (CONT.)
%  o transhumans). These sensors can, of course, also
% detect the presence of harmful radiation.

% SOUNDWAVES
%  he transmission of vibrations through a medium,
%  ound is broken down into infrasound (frequen-

% ies below standard human hearing), normal

% coustic range, and ultrasound (frequencies above
%  tandard human hearing). Soundwaves do not
% propagate in vacuum.

% Ultrasound: Ultrasound sonar operates much
%  ike radar, bouncing sound waves off a target

% nd measuring the returning echoes. Ultrasound
%  maging is similarly low-resolution, showing
%  hapes and movement but no colors and few
% details unless measured closely (1-2 meters).
% Ultrasound is good for identifying how dense a
% material is, however, can detect denser materials
% hidden beneath less dense ones. Many medical
% devices utilize ultrasound, and ultrasound sensors

% an also detect gas leaks, frictional motor noises,

% nd similar mechanical emissions. Ultrasound sen-
%  ors are typically unaffected by noise clutter from
%  tandard acoustic frequencies.

% Infrasound: Infrasound travels much further
%  han regular sound frequencies (hundreds of
% kilometers). Mechanical machinery, seismic distur-
% bances, tornados, explosions, waterfalls, and cer-
%  ain weather phenomena create infrasound waves.

% arge animals such as elephants and whales use
%  nfrasound to communicate via the ground over
% arge distances, though infrasound data transfer is
%  oo slow for complex communications.

% COMBINED SENSOR YSTEMS
% When used in combination, these sensor tech-
% nologies can be potent. For example, the use lidar,
%  hermal imaging, and radar can provide a three-
% dimensional map of a building and everyone and
%  verything inside.                              ■





% a chitin-ceramic hybrid material modeled on the

% microscopic structure and texture of arthropod exo-

% skeletons. This armor is obvious and has a somewhat

% crocodilian or insectoid appearance (character’s

% choice). The morph is completely hairless as well. This

% provides an Armor rating of 11/11. This armor is not

% cumulative with worn armor. [Moderate]


%  Chameleon Skin: The morph’s skin is augmented

% with complex chromatophores so that it changes color

% like the skin of a chameleon or an octopus. The morph

% can match the appearance of almost any color and

% most patterns. This provides a +20 modifier to Infiltra-

% tion Tests to avoid being seen or noticed, as long as

% the character is stationary or not moving faster than a

% %%% txt/306.txt
% slow walk. The character must also be nude or wearing
% smart clothing (p. 325) or a chameleon cloak (p. 315)
% of the same color and pattern. In addition to blending
% in, the character can also consciously change the color
% and pattern of their skin to deliberately stand out (+20
% on Perception Tests to notice) or simply to produce at-
% tractive or interesting colors or patterns. [Low]

% Circadian Regulation: The morph only requires
% 2 hours of sleep to maintain health and function at
% peak mental capacity. The character dreams con-
% stantly while asleep and can both fall asleep and
% wake up almost instantly. In addition, the character
% can easily and with no ill-effects shift to a 2-day
% cycle, where they are awake for 44 hours and sleep
% for 4. [Moderate]

% Claws: The morph has retractable claws like those
% of a cat. These claws do not interfere with the char-
% acter’s manual dexterity and are razor sharp. How-
% ever, they are relatively small and only do 1d10 + 1
% + (SOM ÷ 10) damage, with an AP of –1. As a result,
% they are legal in almost all habitats and are considered
% tools as much as weapons. [Low]

% Clean Metabolism: The morph’s symbiotic bacteria,
% gut flora, and glands have been genetically engineered
% to keep the morph “clean.” The morph also produces
% smart antibiotics that prevent the growth of any
% bacteria or yeasts in it or on its skin. As a result, the
% morph is completely immune to infections, dental cav-
% ities, and bad breath, its sweat has no scent, and the
% morph’s efficient digestion produces somewhat less
% solid waste and less odorous chemicals. [Moderate]

% Drug Glands: The morph has specially-tailored
% glands designed to produce specific hormones or
% chemicals and release them in the body. The charac-
% ter has control over these glands and can release the
% chemicals at will. Each type of drug gland is consid-
% ered a separate enhancement. For potential drugs and
% chemicals, see p. 317. [Low]

% Eelware: Derived from electric eel genetics, a char-
% acter can have eelware implanted so that it connects
% to a network of bioconductors in the hands and feet
% (or other limbs), allowing the character to generate
% stunning shocks with a touch. Eelware inflicts shock
% damage (p. 204) exactly like a pair of shock gloves.
% Eelware can also be used to power implants and spe-
% cially designed handheld devices by touch. [Low]

% Emotional Dampers: This low-cost alternative to
% endocrine control (p. 304) allows the user to vol-
% untarily damp their morph’s emotional responses
% and various non-verbal cues like pupil dilation, eye
% movement, or vocal tone. Using this augmentation
% allows the user to lie and conceal their emotions in
% such as way as oo fool the keenest observer; apply a
% +30 modifier to Deception and Impersonation Tests.
% This modification does not affect methods of detecting
% lies and emotions that involve reading the character’s
% neural state, including psi-gamma sleights. However,
% this augmentation damps out all emotional responses
% and so causes the character to be less persuasive in re-
% al-time personal interactions, imposing a –10 modifier
% to other Social skill tests like Persuasion. Characters
% can turn this augmentation on or off at will. [Low]

% Endocrine Control: This augmentation modifies the
% morph’s endocrine system, giving the character fine
% control over their hormone output. This allows the
% character to completely control their appetite and
% emotions and to regulate pain. They receive a +30
% modifier against the effects of hunger, fear, and any
% forms of emotional manipulation, such as the Drive
% Emotion sleight. This augmentation also allows char-
% acter to lie with perfect conviction and to completely
% fool all methods of lie detection that do not rely
% on the target’s neural output; apply a +20 modifier
% to Deception Tests. It also allows the character to
% remain awake for 48 hours without penalty, but after
% this time the character begins experiencing normal

% %%% txt/307.txt
% fatigue. Finally, the ability to regulate pain reception
% allows the character to ignore the –10 modifier from
% 1 wound. [High]

% Enhanced Pheromones: The morph’s biochemistry
% has been altered so that it produces enhanced phero-
% monal signals that subconsciously affect the behavior
% of other humans in the vicinity. These pheromones
% make the character more attractive and trustworthy to
% the target; apply a +10 modifier to appropriate Social
% skill tests, such as Persuasion. This augmentation only
% affects characters who can smell the pheromones, and
% it does not affect uplifts or xenomorphs. [Low]

% Enhanced Respiration: By boosting both lung ef-
% ficiency and the blood’s oxygen-carrying capacity, the
% character can live comfortably in both high and low
% pressure environments, from 0.2 atmospheres to 5
% atmospheres, with no dizziness or need for gradual
% decompression. In addition, the character can hold
% their breath for up to 30 minutes when performing
% minimal activity or for up to 10 minutes while per-
% forming highly strenuous activity. [Low]

% Gills: The morph’s lung tissue has been adapted to
% function as gills, allowing the morph to breathe both
% air and water, as long as the water is not toxic or too
% stagnant. Characters with this augmentation breathe
% in water and then expel the water through slits just
% underneath their lowest pair of ribs that seal when the
% character is not underwater. [Low]

% Grip Pads: The morph possesses specialized pads
% on its palms, lower arms, shins, and the bottoms of
% its feet. Designed to emulate the pads on gecko feet,
% characters can support themselves on a wall or ceiling
% by placing any two of these pads against any surface
% not made from a material specially designed to resist
% this augmentation. Characters can climb any surface
% and move easily across ceilings that can support their
% weight. Apply a +30 modifier to Climbing Tests. The
% pads must be free to touch the surface the character is
% climbing (no gloves). The nature of these pads is obvi-
% ous to anyone looking at them, but they do not impair
% the character’s sense of touch or manual dexterity. If
% combined with the vacuum sealing augmentation, the
% character can even stick to surfaces in the vacuum of
% space. [Low]

% Hibernation: The character can voluntarily reduce
% the morph’s metabolism to the point that the morph
% requires only 5% of the normal amount of food,
% water, and air. The character appears to sink into a
% deep sleep, but can maintain a dim awareness of both
% touch and sound and so can be easily awakened. En-
% tering or leaving this state requires 3 minutes where
% the character is relatively helpless. With sufficient
% air, characters can safely hibernate for up to 40 days
% without food or water. [Low]

% Muscle Augmentation: The morph’s muscle mass
% has been enhanced and toned and myofibers strength-
% ened. Apply a +5 modifier to SOM. [High]

% Neurachem: This bioware modification enhances
% the character’s chemical synapses and juices their
% neurotransmitters, drastically speeding up neural
% connections. Neurachem can be mentally activated or
% triggered by charged emotions. Level 1 neurachem in-
% creases the character’s Speed stat by +1, with no side
% effect. Level 2 raises the Speed stat by +2, but each
% time it is used the character suffers a nervous system
% fatigue hangover for 1 hour after the boost wears off
% (apply a –20 modifier to all actions). [High (Level 1),
% Expensive (Level 2)]

% Poison Gland: Similar to the drug gland, this
% morph has special glands that produce poisons,
% like the venom glands of a snake. The morph has
% poison glands in its fingers and mouth, so that it can
% deliver either poison by scratching someone with a
% fingernail, biting them hard enough to draw blood,
% or even by sharing a beverage with someone or spit-
% ting into their drink. The morph is immune to the
% poisons it produces. These glands may not produce
% nanotoxins. [Low]

% Prehensile Feet: The morph’s feet and leg joints are
% altered so that its toes are longer and more dexter-
% ous and the big toe is transformed into an opposable
% thumb. Physically, the morph’s feet resemble a longer
% narrower hand or a human foot with finger (and
% thumb)-like toes. The character can walk normally
% but must wear specially designed shoes. However, this
% morph runs somewhat slower than a morph with un-
% modified feet (–1 meter per Action Turn). In addition,
% the morph’s hips are slightly modified to allow greater
% mobility. In a properly constructed chair, or when
% floating in zero-G, the character can use both their
% hands and their feet to manipulate the same object.
% Most morphs used by characters who live in zero-G
% possess this augmentation. [Low]

% Prehensile Tail: A long (1.5 meters) prehensile tail
% is added to the morph’s backside, extending out from
% the tailbone. This tail is prehensile and may be used to
% grab, hold, and even manipulate objects. The character
% can control the tail’s movements with concentration,
% but it otherwise tends to move on its own. The tail
% also improves the character’s balance; apply a +10 to
% any Physical skill tests where balance is a factor. [Low]

% Sex Switch: A complex suite of alterations allows
% the character to switch their physical sex to male,
% female, hermaphrodite, or neuter. This change is
% mentally triggered but takes approximately 1 week to
% complete. [Moderate]

% Skin Pocket: The morph has a pocket within its skin
% layer, capable of holding and providing concealment
% (+30) for small items. [Trivial]

% Temperature Tolerance: The morph’s temperature
% regulation and circulation are both substantially en-
% hanced allowing the character to survive in tempera-
% tures as low as –30 degrees Celsius and as high as 60
% degrees Celsius without discomfort or ill effects. [Low]

% Toxin Filters: The morph gains an improved liver
% and kidneys and biological filters in its lungs. Charac-
% ters with this augmentation are immune to all chemi-
% cal and biological toxins, including everything from
% recreational chemicals to nerve agents to spoiled food.
% In addition, the character can safely and comfortably

% %%% txt/308.txt
% SYNTHMORPHS AND
% BIOWARE
% Though bioware is preferred and more common,
% many types of bioware can be mimicked with cy-
% bernetics. This is especially useful for synthmorphs/
% robots, which cannot be enhanced with bioware.
% The following bioware items may be replicated as
% cybernetics for synthmorphs and robots:

% • Chameleon Skin

% • Drug Glands

% • Eelware

% • Emotional Dampers

% • Enhanced Senses (All)

% • Grip Pads

% • Mental Augmentations (All)

% • Muscle Augmentation

% • Neurachem

% • Poison Glands

% • Prehensile Feet

% • Prehensile Tail                                ■



%  breathe smoke and drink salt water. Unlike medich-
%  ines, toxin immunity prevents the character from ex-
%  periencing even brief harm or discomfort from a toxin
%  (medichines merely rapidly repair damage caused by
%  the toxin and then remove it from the morph). This
%  augmentation provides no resistance to concentrated
%  acid, nanotechnological attacks, or similar destruc-
%  tive agents. Some characters with this augmentation
%  learn to enjoy the taste of various chemical toxins like
%  cyanide or arsenic. [Moderate]

%  Vacuum Sealing: To possess this augmentation, the
%  character must also possess some form of bioware
%  armor or carapace armor. The morph has been spe-
%  cially designed to survive the effects of vacuum. The
%  character’s skin resists vacuum as well as protecting
%  the wearer from temperatures from –75 to 100 C.
%  In addition, the character’s mouth, nose, and other
%  orifices can seal sufficiently well to resist vacuum,
%  and the morph possesses a special membrane that
%  extends over their eyes, allowing the character to
%  see in vacuum without risking any eye damage. This
%  augmentation is usually combined with either the en-
%  hanced respiration or oxygen storage augmentation,
%  or both together. [High]



%  CYBERWARE
% Very little cyberware is physically implanted. Instead,
% the morph is placed in a healing vat (p. 326) and the
% vat’s nanobots construct the cyberware inside the bio-
% morph’s body. Cyberware is rarely used for anything
% that can be accomplished using bioware.

% Synthmorphs and bots may also also use cyberware.
% ENHANCED SENSES

% In addition to being able to duplicate the affects of
% all bioware enhanced senses, there are a few enhanced
% senses that can only be produced using cyberware.

% Anti-Glare: This visual mod eliminates penalties for
% glare. [Low]

% Electrical Sense: The character can sense electric
% fields. Within 5 meters, the character can instantly
% tell if an electrical device is on or off and can see the
% precise location of electrical wiring behind a wall or
% inside a device. This sense gives the character a +10
% modifier on any test involving analyzing, repairing, or
% modifying electrical equipment. [Low]

% Radiation sense: The character can sense the pres-
% ence and approximate source of all forms of danger-
% ous radiation, including neutrons, charged particles,
% and cosmic rays. [Low]

% T-Ray Emitter: Mounted under the skin of the
% user’s forehead, this implant generates low-powered
% beams of terahertz radiation (T-rays) that allow the
% character to see using reflected T-rays. As discussed in
% Using Enhanced Senses, p. 302, this implant combined
% with the enhanced vision enhancement (or a terahertz
% sensor) allows the user to effectively see through cloth,
% plastic, wood, masonry, composites, and ceramics as
% well as being able to determine the composition of
% various materials. This implant allows the user to see
% using reflected T-rays for 20 meters in a normal atmo-
% sphere and for 100 meters in vacuum. [Low]

% MENTAL AUGMENTATION
% These cybernetic augmentations enhance the brain
% and mental functions.

% Access Jacks: Usually located in the base of the skull
% or neck, this implant is an external socket with a direct
% neural interface. It allows the character to establish a
% direct wired connection using a fiberoptic cable to
% external devices or other characters, which can be
% useful in places where wireless links are unreliable or
% complete privacy is required. Two characters linked
% via access jack can “speak” mind-to-mind and transfer
% information between their mesh inserts and other im-
% plants. All synthmorphs have these by default. [Low]

% Dead Switch: This cortical stack (p. 300) accessory is
% designed to keep the stack from falling into the wrong
% hands. If the morph is killed, the dead switch wipes
% and melts the cortical stack completely, so that the ego
% cannot be recovered. This option is generally only used
% by covert operatives with recent backups. [Low]

% Emergency Farcaster: Only characters with corti-
% cal stacks can possess this augmentation. The morph
% has an implanted quantum farcaster (p. 314) linked
% to a highly secure storage facility. The high cost of
% this implant also covers the cost of this storage. Using
% standard radio and quantum encryption, the farcaster
% broadcasts full backups of the character’s ego (pulled
% from the cortical stack) once every 48 hours. At the
% gamemaster’s discretion, the backup interval may be
% scheduled more or less frequently, keeping in mind
% that ego broadcasts are generally limited for security

% %%% txt/309.txt
% purposes and because they hog bandwidth. These
% broadcasts only work when the character is in radio
% contact with the storage facility and is typically only
% used inside a habitat to broadcast backups back to a
% nearby space ship. If the radio broadcasts are blocked
% or jammed, this device cannot make backups.

% In the event of a farcaster failure, this augmentation
% also includes a single-use emergency neutrino broad-
% caster (p. 314) as well. This broadcaster contains ap-
% proximately 10 nanograms of antimatter stored in an
% orange-sized triply-redundant magnetic containment
% vessel. If the character is dying or urgently wishes to
% depart the morph, this tiny amount of antimatter is
% brought into contact with a similarly tiny amount of
% matter in a controlled fashion that generates a single
% brief and carefully coded neutrino pulse of the ego’s
% most recent backup. However, the heat generated by
% this process literally cooks the entire morph, killing it
% and destroying all implants and electronics in or on it.

% This entire process takes less than 0.1 second and
% the broadcast can be received as long as the neu-
% trino receiver is within 100 astronomical units of
% the character. Within the solar system, this implant
% effectively guarantees the character’s backup. It is less
% useful on exoplanets where the character is out of
% neutrino range of their backup facility. The amount
% of antimatter carried by this implant is sufficiently
% small enough that it does not produce an explosion
% and will not damage any surrounding objects. Most
% habitats carefully scan all visitors to determine if they
% have this implant and if the amounts of antimatter
% involved are sufficiently low as not to pose a danger
% to the habitat and its inhabitants, and some ban this
% implant entirely. [Expensive]

% Ghostrider Module: This implant allows the
% character to carry another infomorph inside their
% head. This infomorph could be another muse, an AI,
% a backed-up ego, or a fork. The module is linked to
% the character’s mesh inserts, so the ghost-rider can
% access the mesh. The character may limit the ghost-
% rider’s access, or may allow them direct access to their
% sensory information, thoughts, communications, and
% other implants. [Low]

% Mnemonic Augmentation: A character with this
% augmentation and a cortical stack can access digital
% recordings of all of the sensory data they have ex-
% perienced in XP format (and they may share these
% recordings with others). Mnemonic augmentation
% differs from the eidetic memory bioware because it
% allows characters to digitally share all of their sensory
% data with others. It also allows them to closely ex-
% amine sensory data they did not initially look at. For
% example, If the character glanced at a note but did not
% read it, they can later use image enhancement soft-
% ware to enhance this image and in most cases actually
% read what the note said. Mnemonic augmentation
% allows the character to clearly hear all background
% noises, like a conversation at a nearby table that the
% character only initially heard a few words of. Using
% mnemonic augmentation to retrieve a specific piece of
% information is quite easy, but usually requires between
% 2 and 20 minutes of concentration. [Low]

% Multi-Tasking: Only characters with cortical stacks
% can possess this augmentation. The character has an
% advanced computer installed in their brain that uses
% the data in the cortical stack to create several simulta-
% neous short-term forks to handle various mental tasks.
% By design, this computer automatically reintegrates
% all of these forks into the character’s core personality
% after a maximum of 4 hours, earlier if desired. This
% augmentation allows the character to both plan a
% speech and engage in intensive mesh-browsing while
% simultaneously fighting a gun battle or running from
% pursuit, since each of the forks operates independently.
% However, these forks can only perform purely mental
% or on-line interactions. This augmentation can pro-
% duce a maximum of two forks at a time, giving the
% character an extra two Complex Actions on every
% Action Phase for mental or on-line actions. This im-
% plant cannot be used simultaneously with any other
% augmentation that allows for extra actions, or with
% the mental speed augmentation (p. 308). [High]

% Puppet Sock: This implanted computer allows the
% biomorph’s body (the “puppet”) to be controlled by
% another character (the “puppeteer”). While active, the
% puppet has no control over their body and is simply
% along for the ride (at the gamemaster’s discretion,
% puppets who are tormented by repeated or exten-
% sive loss of control may suffer mental stress). The
% puppeteer may directly “jam” the puppet or remote
% control it in the same way that robots and pods are
% teleoperated (p. 196). The puppeteer must either be
% ghost-riding the puppet (see the Ghostrider Module, p.
% 307) or have a direct communications link (via mesh,
% radio, laser, etc.). [Moderate]

% PHYSICAL AUGMENTATIONS
% This implants enhance the morph’s physical body.

% Cyberclaws: The bones on the back of the morph’s
% hand are bonded to smart material claws. These claws
% can extend through concealed ports in the morph’s
% skin and extend 6 inches past the morph’s knuckles.
% These razor-sharp weapons inflict 1d10 + 3 + (SOM
% ÷ 10) damage and have an AP of –2. If combined
% with eelware (p. 304), they can also inflict electric
% shocks. Likewise, cyberclaws can also deliver poison
% or nanotoxins secreted from a poison gland (p. 305)
% or implanted nanotoxins. [Low]

% Cyberlimb: In an age when arms and legs can
% easily be regrown, many people consider cybernetic
% prostheses to be vulgar and distasteful. The Scum
% and others, however, treat them as iconic symbols
% of self-expression. Standard replacement cyberlimbs
% function the same as their biological equivalents,
% though that particular limb receives a +3/+3 Armor
% bonus when targeted specifically (this bonus does not
% apply to synthmorphs). Cyberlimbs may be masked
% to look real (see Synthetic Mask, p. 311), and may
% also feature small compartments for hiding/storing
% small objects. [Moderate]

% %%% txt/310.txt

% Cyberlimb Plus: More extravagant cyberlimb
% models are also available, though they require more
% severe body alteration to accommodate. These limbs
% apply a +5 SOM bonus per limb (maximum +10).
% They may be replacement limbs or “extra” limbs an-
% chored in the body’s skeletal frame. These cyberlimbs
% may not be masked. [High]

% Hand Laser: The morph has a weapon-grade laser
% implanted in its forearm, with a flexible waveguide
% leading to a lens located between the first two knuck-
% les on the morph’s dominant hand. The laser fires
% from this waveguide, inflicting 2d10 damage with 0
% AP. The laser is powered by a small nuclear battery
% located in the morph’s torso, good for 50 shots before
% it must be recharged like other beam weapon batteries
% (p. 338). [Moderate]

% Hardened Skeleton: The morph’s skeleton has been
% laced with strengthening materials. Apply a +5 DUR
% and +5 SOM bonus. [High]

% Oxygen Reserve: The morph has a miniature
% oxygen tank and rebreather installed in its torso. This
% implant provides the equivalent of the life support
% system in a light vacsuit (p. 333), allowing the charac-
% ter to breathe comfortably for up to 3 hours. It feeds
% oxygen directly to the morph’s blood stream, avoiding
% problems with pressure changes. Implanted sensors
% automatically cause the character to use the stored
% oxygen if they detect poisonous or insufficient atmo-
% sphere. Without vacuum sealing, the character can
% only survive in vacuum for 5 minutes, but remains
% conscious and active for the entire time, giving them
% far more time to find shelter or a vacsuit than charac-
% ters without this implant. For every hour the character
% is in a breathable atmosphere, this implant recovers
% one hour of oxygen storage. The implant can be fully
% recharged within 15 minutes if the character is in a
% high-pressure mostly oxygen atmosphere. [Low]

% Reflex Boosters: The morph’s spinal column and
% nervous system is rewired with superconducting
% materials, boosting transmission speed. This raises
% the character’s REF by +10 and improves Speed by
% +1. [Expensive]

% NANOWARE
% All augmentation nanoware is advanced nanotech-
% nology (p. 328), consisting of a grape-sized nanobot
% generator that produces specialized nanomachines.
% Nanoware is available for synthmorphs and bots in
% addition to biomorphs.

% Implanted Nanotoxins: The morph has an im-
% planted nanobot hive that produces nanotoxins (p.
% 324). This implant is designed so that the character
% can deploy these nanobots instantly via a scratch with
% claws, spraying with saliva, or simply making continu-
% ous bare-skin contact. Characters can choose whether
% or not to deploy these nanobots. Each nanotoxin
% generator only produces a single variety of nanobots,
% with the most common types being ones designed to
% kill or incapacitate almost any living target or ones
% designed to destroy delicate machinery. Characters
% are immune to their own nanotoxins. Nanotoxins
% are highly restricted and many habitats will not allow
% anyone with this implant on board. [Moderate]

% Medichines: This is the most common form of
% nanoware. These nanobots monitor the user’s body at
% a cellular level and fix any problems that arise.

% Medichines eliminate most diseases, drugs, and
% toxins (but not nanodrugs or nanotoxins) before
% they can do more than minor harm to the host (see
% Drug Effects, p. 317). If desired, the user can tempo-
% rarily override this protection to permit intoxication
% or other effects, but unless the character activates a
% second specially labeled override, medichines prevent
% the toxins from accumulating to lethal or permanently
% harmful levels. In this case, they can also be activated
% at a later point to reduce a drug or toxin’s remaining
% duration by half.

% Medichines allow the character to ignore the effects
% of 1 wound. They also speed normal healing as noted
% under Biomorph Healing, p. 208. If the user suffers
% 5 or more wounds at once, or more than 6 wounds
% in an hour, the damage has exceeded the medichines’
% ability to repair. In this case, the medichines place the
% character into a medical stasis, where their mind and
% body are perfectly preserved, but where the character
% cannot act in any way. Under these circumstances the
% medichines also send out a priority call for emergency
% services via the character’s mesh inserts.

% Medichines for synthmorphs and bots consist
% of nanobots that monitor and repair the shell’s
% integrity and internal system functions. Note that
% the synthmorph version of medichines allows the
% synthmorph to self-repair in the same manner by
% which a biomorph with medichines would naturally
% heal (p. 208). [Low]

% Mental Speed: With this nanoware system, nano-
% bots alter the character’s neural architecture and
% augment the functioning of their neurons. The char-
% acter can deliberately speed up their mind to think
% and also receive and process sensory information far
% faster than ordinary humans. Time seems to subjec-
% tively slow down for the character, allowing them
% to carefully plan their next action, even if they only
% have a split second to do so. With this system active,
% the character can discern things occurring too fast
% for a normal human to perceive, such as the indi-
% vidual frames of an old analog film or understanding
% sounds that were accelerated to many times their
% normal speed. The character can also read 10 times
% faster than normal and can track the paths of bullets
% and similar fast-moving objects with a successful
% Perception Test.

% When using this augmentation, the character gains
% two extra Complex Actions during each Action Phase
% that may only be used for mental actions. The charac-
% ter also receives a +30 Initiative bonus. The character
% thinks at normal speed whenever this nanoware is in-
% active. This nanoware is incompatible with any other
% augmentation that provides any form of extra actions,
% such as multi-tasking. This augmentation can be used

% %%% txt/311.txt
% as often as desired, but actively using it renders ordi-
% nary conversation and social interactions difficult and
% requires concentration to maintain. [High]

% Nanophages: These nanobots patrol the body, alert
% for signs of intrusive nanodrugs or -toxins and de-
% stroying them before they have more than a minor
% effect. Nanophages provide automatic immunity
% against nanodrugs and nanotoxins unless they are
% specifically commanded to stand down by the user, via
% their mesh implants. [Moderate]

% Oracles: These neural macrosensing nanobots pay
% attention to the sensory input on which the charac-
% ter is not focusing, alerting them about important
% things they might otherwise overlook. Oracles also
% act as a sort of memory buffer and search aid, ex-
% tending short term memory, helping the character
% recall memories and details, and crosschecking
% them with other memories. Oracles negate Percep-
% tion modifi ers for distraction, apply a +10 modi-
% fier to Investigation Tests, and add a +30 bonus to
% memory-related tests. [Moderate]

% Respirocytes: These nanobots act as highly-efficient
% artificial red blood cells, increasing the ability to
% transfer oxygen and carbon dioxide. This increases
% the morph’s ability to hold their breath to 4 hours
% and increases DUR by +5. [Moderate]

% Skillware: The morph’s brain is laced with a net-
% work of artificial neurons that may be formatted with
% downloaded information. This allows the user to
% download skillsofts (p. 332) into their brains, gaining
% the use of those programmed skills until the skillsoft
% is erased or replaced. Skillware systems are only
% capable of handling 100 total skill points worth of
% skillsofts at a time. [High]

% Skinflex: This disguise implant allows the user to
% restructure their facial features and musculature and
% alter skin tone and hair color. The entire process takes
% a mere 20 minutes. Skinflex adds +30 to Disguise
% Tests. [Moderate]

% Skinlink: Skinlink nanobots live on the morph’s
% external skin or shell, automatically swarming over
% and creating a physical connection with any electron-
% ics the user touches. They also take advantage of the
% electrical field in a biomorph’s skin for communica-
% tion. They allow the user to communicate and mesh
% with any devices merely by touching them. This
% is considered a wired link, and so is not subject to
% wireless interception or interference. Two skinlinked
% characters can also communicate and mesh simply by
% touching. [Moderate]

% Wrist-Mounted Tools: The morph has a 6 centime-
% ter-wide metal band containing nanobot generators
% implanted around each wrist. These nanobots link to-
% gether to duplicate the function of a utilitool (p. 326),
% creating narrow, highly flexible arms that each ends in
% a specialized tool. These nanobots can also produce
% tiny fiber optics to allow the character to see through
% small openings, as well as being able to create small
% weapons equal to bioware claws. The fact that these
% tool are mentally controlled gives the character a +20
% modifier to skills involving repairing or modifying de-
% vices with mechanical parts, opening locks or disarm-
% ing alarm systems, or performing first aid. [Moderate]

% COSMETIC MODS
% In an age of universal beauty, artistic cosmetic modi-
% fication of your body is commonly pursued by many
% transhumans. Body mods once considered dangerous
% or edgy are now safe and commonplace, especially
% among factions like the anarchists, scum, or brinkers.

% Bodysculpting: If your morph’s enhanced physique
% isn’t enough, you can take it further with custom
% bodysculpting such as as elongated ears or fingers,
% nose alteration, hair addition/removal, feathers, exotic

% %%% txt/312.txt
% eyes, snakeskin, endowed genitalia, and more unusual
% physical alterations. [Low]

% Nanotats: Tattoos created with nanobots can
% move around the body, change shape/color/bright-
% ness, texture, alternate text and images, and/or even
% create minor holographic effects on the skin’s sur-
% face, all controllable via mesh inserts. [Low]

% Piercings: Name any part of the body and some-
% one’s figured out a way to pierce it, probably mul-
% tiple times. Hoops, barbells, plugs, and chains are
% extremely common, often made of shapechanging
% smart materials. [Trivial]

% Scarification: Given modern medical abilities, scars
% of any sort are purely an affectation. [Trivial]

% Scent Alteration: Minor changes to a body’s bio-
% chemistry can alter a character’s natural smell or
% constantly perfume them. [Low]

% Skindyes: Dye jobs are available in all conceivable
% colors and patterns. [Trivial]

% Subdermal Implants: Adding small implants under
% the skin can create bumps, ridges, piercing anchors,
% and similar textures and alterations. [Trivial]

% ROBOTIC ENHANCEMENTS
% The following modifications are only available to
% synthmorphs/robots.

% ARMOR
% These armor modifications replace the synthmorph’s
% built-in Armor rating.

% Heavy Combat Armor: The synthmorph’s frame is
% loaded with armor that offers protection from heavy
% weapons for serious combat operations. This modi-
% fication is bulky and noticeable; the bot frame is en-
% cased in a heavy-duty carapace. It increases the bot’s
% built-in Armor to 16/16. The shell’s mobility systems
% and power output are also enhanced to deal with the
% extra load. [High]

% Industrial Armor: The shell is equipped with protec-
% tion against collisions, extreme weather, industrial ac-
% cidents, and similar wear-and-tear. Increase the bot’s
% built-in Armor rating to 10/10. [Moderate]

% Light Combat Armor: The synthmorph’s frame is
% protected by armor designed for policing and secu-
% rity duties. This increases the bot’s built-in Armor to
% 14/12. [Moderate]

% MOBILITY SYSTEMS
% Shells are designed with a wide-range of propulsion
% systems, and are sometimes built for a specific environ-
% ment/gravity. Some synthmorphs may have multiple
% mobility systems. Many such systems are retractable,
% meaning they can be folded away into the shell’s frame.

% Hopper: Hoppers have two or more legs designed
% to propel the morph forward or up, much like a frog
% or grasshopper. [Moderate]

% Hovercraft: The shell uses an impeller to blast a
% cushion of high-pressure air off the surface below,
% repelling the frame off the ground (modern hovercraft
% do not use rubber skirts). Most hovercraft travel a
% meter or so above the ground, but can temporarily
% levitate themselves higher for short periods. [Low]

% Ionic: The shell uses principles of magnetohydro-
% dynamics to levitate and fly, by ionizing surrounding
% air into plasma to create lift and momentum. The
% shell is also spun for stability. This system does not
% work in vacuum, but an underwater version uses the
% same mechanics for propulsion in liquid environ-
% ments. [High]

% Microlight: Popular in low-grav and microgravity
% environments, microlights encompass several types
% of ultralight or lighter-than-air systems, such as pow-
% ered paragliders, autogyros, balloons, aerostats, and
% blimps. These systems do not work in vacuum. [Low]

% Roller: Only for circular shells, this system allows
% the synthmorph to roll like a ball. The shell rolls
% around an interior axle, propelled by a motor-driven
% pendulum. [Moderate]

% Rotorcraft: Rotating blades create lift, allowing
% the shell to move and hover like a helicopter. Most
% models use tilt-rotors or tilt-wings so that the rotor-
% blades may be moved forward (for faster propeller-
% like propulsion) and for better maneuverability in
% zero-G. This system does not work in vacuum. [Low]

% Snake: Commonly used by slitheroids, these shells
% use lateral undulation, flexing their body from left to
% right and waving their frame forward. Such shells may
% also use sidewinding or a concertina motion (straight-
% ening forward, then retracting the rear) to move. They
% also featured gyroscope stabilization so that they may
% circle into a hoop and roll like a wheel. [Moderate]

% Submarine: Designed for undersea mobility, subma-
% rine shells use propellers or pumpjets to push through
% water. [Moderate]

% Tracked: Tracked shells use smart rotating treads to
% work their way across surfaces that would bog down
% other ground vehicles. They can prop themselves up
% in order to overcome taller obstacles or to lay them-
% selves down to bridge across a ditch or crevice. [Low]

% Thrust Vector: These shells use either turbofans or
% turbojets to create atmospheric lift with a set of wings.
% The engines may be maneuvered to point and generate
% thrust in different directions for vertical takeoffs/land-
% ings and better maneuverability in zero-G. [Moderate]

% Walker: Walkers use two or more limbs to walk or
% crawl across a surface. Many use grip pads (p. 305) or
% magnetic systems (p. 310) to stick to surfaces. [Low]

% Wheeled: Most wheeled shells feature smart
% spokes that allow the wheels to conform their shape
% to obstacles and even climb stairs. Some low-grav
% shells feature puncture-resistant and self-repairing
% compressed-gas tires. [Low]

% Winged: Primarily used by smaller shells, this system
% of four independently-controlled wings allows the
% shell to hover or move rapidly in any direction. [Low]

% PHYSICAL MODIFICATIONS
% These mods are applied to the shell’s physical frame.

% Extra Limbs: The shell is equipped with one or
% more extra limbs. A character using these limbs

% %%% txt/313.txt
%  suffers an off-hand modifier (p. 193). These limbs
% may be arms (with hand/grippers/etc.), legs, tentacle-
% like, or otherwise articulated and/or prehensile. Some
%  shells have rotational frames that allow them to move
% limbs around their body. [Low]

%  Fractal Digits: The synthmorph has “bush robot”
% digits that are capable of splitting into smaller digits,
%  and those smaller digits into micro digits, and so on
% down to the micrometer scale, allowing for ultra-fine
% manipulation. Apply a +20 COO modifier where such
% fine manipulation is a factor (such as detailed repair
% work). The bot must have functioning nanoscopic
% vision (p. 311) to get this bonus. [Moderate]

%  Hidden Compartment: The shell has a concealed
%  aperture for a shielded interior compartment, ideal for
%  storing valuables or smuggling contraband. Apply a
% –30 modifier to detect this compartment either manu-
%  ally or with sensor scans. [Low]

%  Magnetic System: A magnetic system allows the
%  shell to cling to most ferrous materials. This enables
% the character to walk in zero-G situations by magneti-
%  cally adhering surfaces, hang upside down, and hold
% onto devices without letting them drop or drift away.
% The shell receives a +30 modifier whenever maintain-
% ing a magnetic hold on something. [Low]

%  Modular Design: This shell is designed to lock to-
%  gether with similar modular morphs in different archi-
% tectural patterns to create larger gestalt forms. When
% united with other modules, the group is treated as a
%  single unit/morph, with shared capabilities. If dam-
%  aged and then separated, damage and wounds are dis-
% tributed evenly between modules; uneven amounts are
%  allocated randomly. The exact capabilities of different
%  shapes depends on the composition, and is largely left
% in the gamemaster’s hands. [High]

%  Pneumatic Limbs: The limbs are equipped with
% pneumatic cylinder systems that can generate up to
% 1,500 pounds of thrust. This allows the shell to push
% off and make impressive jumps (a synth of human
%  size/weight can leap over 2 meters up). Apply a +20 to
% Freerunning Tests. A pneumatic limb used to strike an
% opponent in unarmed combat inflicts an extra 1d10
% damage. [Low]

%  Retracting/Telescoping Limbs: The shell’s limbs can
%  either be retracted completely inside it’s frame and/or
%  extended for extra length (usually up to 1 or 2 meters
%  extra). Telescoping limbs may give the shell a reach
%  advantage in melee combat (p. 204). [Low]

%  Shape Adjusting: This shell is made from smart
% materials that allow it to alter its shape, altering its
% height, width, circumference, and external features,
% while retaining the same mass. This modification is
% typically employed to reshape the morph into special
%  configurations adapted to specific tasks (for example,
% lengthening to crawl through a tunnel, widening its
% base for stability, expanding to reach out and attach
% to multiple access point simultaneously, and so on).
% This mod also allows the morph to change its features
% for disguise purposes; apply a +30 modifier to Dis-
%  guise Tests. [High]

% Structural Enhancement: This modification bolsters
% the shell’s structural integrity, boosting its ability to
% take damage. Increase Durability by 10 and Wound
% Threshold by 2. [Moderate]

% Swarm Composition: The shell is not a single unit
% but a swarm of hundreds of insect-sized robotic mi-
% crodrones. Each individual “bug” is capable of crawl-
% ing, rolling, hopping several meters, or using nano-
% copter fan blades for airlift. The cyberbrain, sensor
% systems, and implants are distributed throughout the
% swarm. Though the swarm can “meld” together into
% a roughly child-sized shape, the swarm is incapable of
% tackling physical tasks like grabbing, lifting, or hold-
% ing as a unit. Individual bugs, however, are quite ca-
% pable of interfacing with electronics. Swarms cannot
% carry most gear or wear armor, and may not make
% strength-based SOM-linked skill tests. For combat
% purposes, use the same rules as given for nanoswarms,
% p. 328. Damage and wounds are reflected as dam-
% aged/massacred bugs. The swarm may be “healed” by
% manufacturing more bugs.[High]

% Synthetic Mask: The synthmorph is equipped
% with a realistic outer casing of faux-skin and care-
% fully sculpted to pass as a biomorph (perhaps even a
% particular person). The morph can cry, spit, have sex,
% and will even bleed if cut. Only a detailed physical
% examination or a radar, teraherz, or x-ray scan will
% detect the synthmorph’s true nature, and even then
% such exams/scans suffer a –30 modifier. [Moderate]

% Weapon Mount: The shell carries a built-in (or built-
% on) weapon. This weapon mount may be either internal
% (concealed, only weapons small in relation to the shell
% may fit, –30 to Perception Tests to detect) or external
% (visible). It may be fixed (one direction only), swiveling
% (limited field of fire), or on an articulated mount (all
% directions). [Low; Moderate for concealed/articulated]

% SENSORS
% 360° Vision: The shell’s visual sensors are situated for
% a 360-degree field of vision. [Low]

% Chemical Sniffer: This sensor detects molecules in the
% air and analyzes their chemical composition. It enables
% Chemistry Tests to determine the presence of gases, in-
% cluding toxins and other fumes. It can also detect the
% presence of explosives and firearms. [Moderate]

% Lidar: This sensor emits laser light and measures
% the reflections to judge range, speed, and image the
% target. See Using Enhanced Senses, p. 302. [Low]

% Nanoscopic Vision: The shell’s visual sensors can
% focus like a microscope, using advanced superlens
% techniques to beat the optical diffraction limit and
% image objects as small as a nanometer. This allows
% the character to view and analyze objects as small as
% blood cells and even individual nanobots. The synth-
% morph must stay relatively steady to view objects at
% this scale. [Moderate]

% Radar: This sensor system bounces radio or micro-
% waves off targets and measures the reflected waves
% to judge size, composition, and motion. See Using
% Enhanced Senses, p. 302. [Low]

% %%% txt/314.txt
% ARMOR
% Modern personal armor systems have advanced from
% the high modulus polyethylene thermoplastics and
% aramid fabrics of the early 21st century. Armor in
% Eclipse Phase is derived from biotech, in the form of
% organoweave fibers and crystalline-grown plates, and
% nanotech, in the form of shock-absorbing fullerene (p.
% 298) materials. Occasionally other materials are used,
% such as metallic glass plates or shear-resistant fluids
% that harden against impacts. Such armor protects
% against (armor-piercing) bullets and kinetic impacts
% as well as bladed weapons and piercing sharp objects.
% They also insulate against both the explosive heating
% of energy weapons and electrical shocks. While such
% armor protects against bullets, the layers of mate-
% rial catch the bullet and redistribute its kinetic energy
% across the body, which can still result in severe blunt
% force trauma.

% Rules for armor in combat can be found on p. 194.
% Armored exoskeletons are listed on p. 343.

% Armor Clothing: The extra-resilient organoweave
% fibers and fullerene materials that offer basic protec-
% tion against kinetic and energy weapons can be woven
% in with normal smart materials to create a wide range
% of discreet armor clothing that provides a subtle level
% of security. Such protective garments are indistin-
% guishable from regular clothing and come in all styles
% and designs. Armor clothing provides an Armor Value
% of 3/4. [Trivial]

% Armor Vest: Armor vests provide more thorough
% protection to a body’s vital areas, covering the abdo-
% men and torso completely, protecting the neck with a
% rigid collar, and even providing wrap-under protec-
% tion for the groin. Though armor vests are not bulky,
% they are obvious as armor. Armor vests may be worn
% with armor clothing without penalty. Armor vests
% provide an Armor Value of 6/6. [Low]

% Body Armor (Light): These high performance armor
% outfits protect the wearer from head to toe. An inte-
% grated armor vest is supplemented with increased pro-
% tection on the limbs and joints, while still managing to
% be flexible and non-restrictive. Body armor is typically
% worn by security and police forces, and supplemented
% with a helmet. It provides an Armor Value of 10/10.
% [Low]

% Body Armor (Heavy): Similar to light body armor,
% but with extra protective layers, often ergonomically
% manufactured to conform to a specific character’s
% body, and an environmental seal with climate control
% to protect the wearer from hostile environments. It
% provides an Armor Value of 13/13. [Moderate]

% Crash Suit: Designed for both industrial worksite
% safety and protection from accidental zero-G collisions,
% crash suits are also favored by sports enthusiasts and
% explorers. The basic jumpsuit offers comfortable pro-
% tection equal to that of armor clothing. When activat-
% ed with an electronic signal, however, elastic polymers
% within the suit stiffen and form rigid impact protection
% for vital areas. Crash suits provide an Armor Value of
% 3/4 when inactive and 4/6 when activated. [Low]

% Helmet: This armor accessory is usually worn with
% body armor or a battle suit. Light helmets are open,
% whereas full helmets latch on and provide an environ-
% mental seal with a 12 hour supply of air. Light helmets
% provide an Armor Value bonus of +2/+2, whereas full
% helmets add +3/+3. Helmets are often equipped with
% an ecto (p. 325), a radio booster (p. 313), and sensors
% equal to specs (see p. 325). [Trivial]

% Riot Shield: Used for mob suppression, riots shields
% are light-weight, tough, and may be set to electrify on
% command, stunning anyone who comes into contact
% with the outer surface (treat as shock glove effects, p.
% 334). Riot shields provide an Armor Value bonus of
% +3/+2. [Low]

% Second Skin: This lightweight bodysuit, woven
% from spider silks and fullerenes, is typically worn
% as an underlayer, though some athletes use it as a
% uniform. It provides minimal protection, but may be
% worn with other armor without penalty. It provides an
% Armor Value of 1/3. [Low]

% Smart Skin: Smart skin is an advanced nano-
% fluid that covers the wearer’s skin. It resembles liquid
% mercury but retains the texture and flexibility of
% normal skin until activated, at which point the mate-
% rial becomes rigid enough to protect the wearer and





%              ARMOR VALUES
% ARMOR                             ENERGY KINETIC PAGE
% Armor Clothing                      3       4      311
% Armor Vest                          6       6      312
% Battle Suit Powered Exoskeleton     18     18      344
% Bioweave Armor (Light)              2       3      302
% Bioweave Armor (Heavy)              3       4      302
% Body Armor (Light)                  10     10      312
% Body Armor (Heavy)                  13     13      312
% Carapace Armor                      11     11      303
% Crash Suit (Inactive)               3       4      312
% Crash Suit (Active)                 4       6      312
% Exowalker                           2       4      344
% Hard Suit                           15     15      334
% Helmet (Light)                      +2     +2      312
% Helmet (Full)                       +3     +3      312
% Hyperdense Exoskeleton              6      12      344
% Riot Shield                         +3     +2      312
% Second Skin                         1       3      312
% Smart Skin                          3       2      312
% Smart Vac Clothing                  2       4      325
% Spray Armor                         2       2      312
% Synthmorph Industrial Armor         10     10      310
% Synthmorph Combat Armor (Light)     14     12      310
% Synthmorph Combat Armor (Heavy)     16     16      310
% Transporter Exoskeleton             2       4      344
% Trike Exoskeleton                   2       4      344
% Vacsuit (Light)                     5       5      333
% Vacsuit (Standard)                  7       7      333

% %%% txt/315.txt
% distribute the kinetic energy (though still flexible
% enough at the joints not to impede movement). A spe-
% cialized hive, worn by the character, replenishes the
% nanobots and stores them when not in use. Deploying
% the nanobots across the body takes a full Action Turn.
% Smart skin has an Armor Value of 3/2, and may be
% worn with other armor without penalty. [Low]

% Spray Armor: This fast armor application comes in
% a spray can and disperses a smart chemical polymer
% that sticks to bare flesh (but does not adhere to hair
% and eyes). The polymer solidifies into a form fitting
% body armor fabric when exposed to body temperature
% with the look and feel of a latex suit. Spray armor
% does not work on synthetic morphs or on clothing or
% other armor. The color and feel of the armor can be
% adjusted with electric currents and additional poly-
% mers, making it popular among some socialite and
% nightlife scenes. The spray-on armor does not wash
% off, but degrades 1 point of armor (both energy and
% kinetic) every 12 hours. It may be removed with a
% special nanotech solvent. Spray armor has an Armor
% Value of 2/2. [Low]

% ARMOR MODS
% Armor modifications add extra materials or coatings
% that either enhance the armor’s resistance to certain
% dangers or provide other effects. Armor mods may be
% easily added or removed with the appropriate nano-
% bot applicators.

% Ablative Patches: These thin and light slap-on
% patches of stick to armor and are designed to absorb
% heat and energy from beams and explosions, safely
% vaporizing and blowing hot gas away. Ablative
% patches increases the Armor Value by +4/+2, but each
% hit reduces both the energy and kinetic value of the
% ablative armor by 1. [Trivial]

% Chameleon Coating: This provides the armor with
% the same effect as the chameleon cloak (p. 315). [Trivial]

% Fireproofing: Fireproofing includes the addition of
% heat-resistant ceramic or fire-resistant layers, both
% capable of withstanding extremely high temperatures.
% Fireproofing increases the Armor Value by +2/+0, and
% provides an additional 10 points of armor against
% heat or fire specifically. [Trivial]

% Immunogenic System: The immunogenic mod
% adds an active nanobot swarm, maintained by a spe-
% cialized hive, that coats the outer layer of armor and
% also the non-armored parts of the wearer’s morph.
% It acts as an outer immune system designed to neu-
% tralize toxic agents and nanotoxins with which it
% comes into contact. This provides immunity to drugs,
% toxins, and nanotoxins applied dermally, such as
% with a slap patch or splash grenade. It has no effect
% on inhaled, oral, or injected drugs (including coated
% weapons). [Low]

% Lotus Coating: The armor has been impregnated
% with a superhydrophic coating (contact angle of
% around 170°) that repels all water-like liquids. If the
% armor is splashed by liquid toxins or chemicals, the
% effect is reduced since the liquids starts to roll off the
% armor. Apply a +30 modifier when defending against
% liquid-based attacks. [Trivial]

% Offensive Armor: When activated, the outer layer of
% this armor is rigged to shock anyone or anything that
% contacts it with electricity. Treat its DV and effect as a
% shock baton (p. 334). [Low]

% Reactive Coating: A thick layer of advanced
% nanotech is applied to the armor, protecting it with
% a colony of nanobots designed to sense incoming
% attacks. When an attack strikes the coating, it deto-
% nates to disrupt the attack. Bursts and full autofire
% are treated as a single attack. A reactive coating
% increases the Armor Value by +5/+5, but each deto-
% nation automatically inflicts 1 point of damage on
% the wearer. Reactive armor also works against melee
% attacks, but the attacker also suffers 1d10 ÷ 2 (round
% up) points of damage per attack (armor protects)
% from the microexplosion. Reactive coating only
% works against 5 attacks, after which the specialized
% nanobot hive replenishes the coating at the rate of 1
% use per hour. [Moderate]

% Refractive Glazing: A combination of reflectors,
% refractive metamaterials, and an energy transfer
% system with heat radiators provides extra protection
% against energy weapons. Increase the Armor Value
% by +3/+0. [Low]

% Self-Healing: The armor is equipped with a nano-
% hive that acts like repair spray (p. 333). [Moderate]

% Shock Proof: Shock proof armor is electronically
% insulated to discharge and reduce the effect of shock
% weapons. Apply an additional +10 modifier when resist-
% ing the DV and effects of shock weapons (p. 204). [Low]

% Thermal Dampening: Thermal dampening obfus-
% cates heat signatures by converting body heat into
% electric energy. It makes the target more difficult to
% spot with thermal sensors; apply a –30 modifier for
% Perception Tests. [Moderate]

% COMMUNICATIONS
% The oldest and most widespread communications
% technology still in regular use is radio. Every habitat
% and world inhabited by transhumanity is awash in
% radio traffic, with humans, machines, and uplifts all
% constantly communicating with one another. The
% smallest radios are no larger than a spec of dust and
% have a range of no more than 20 meters, while the
% largest are the size of a truck and have a range of
% many thousands of miles. Radios large and small are
% ubiquitous and almost all devices contain at least
% short-range radios so they may interact with the
% mesh. Most morphs are equipped with basic mesh
% inserts (p. 300) that include an implanted radio. For
% radio ranges, see p. 296.

% Fiberoptic Cable: Fiberoptic cables are used to
% establish wired connections between two devices.
% Given the ubiquity of radios and the tangled mess
% wires cause, they are typically only used for privacy
% (unlike radio communication, fiberoptic signals may
% not be intercepted) or in areas with heavy radio
% interference. [Trivial]

% %%% txt/316.txt

% Laser/Microwave Link: These portable devices are
% used to establish a tight-beam, line-of-sight commu-
% nications channel with another laser or microwave
% link. The range of these transceivers varies widely
% with environmental factors, but approximates 50
% kilometers in atmosphere and 500 kilometers in
% space (though horizon limits must be kept in mind,
% being 5 kilometers at ground level on Earth and less
% on smaller bodies). Lasers are subject to interference
% from fog, dirt, smoke, and similar visual chaff, while
% microwaves may be hindered by metallic obstructions.
% These links may only be intercepted by getting directly
% in between the beams. Some teams carry a micro ver-
% sion of this system, worn on their person, allowing
% line of sight intra-team communications that cannot
% be intercepted like radio. [Moderate]

% Radio Booster: This device boosts the range and
% sensitivity of short-range radios, like those from im-
% plants, ectos, or microbugs. The booster must be with
% the shorter-ranged device’s range (or directly linked
% via fiberoptic cable). It will repeat any transmissions
% received from that device, but at its extended range of
% 25 kilometers in urban areas (250 kilometers remote
% areas). Broadcasts from a radio booster are easy to
% receive by anyone looking for broadcasts (see Wire-
% less Scanning, p. 251), though transmissions may be
% stealthed (p. 252). Boosters are commonly used by
% characters traveling far from habitats or other civi-
% lized regions. [Low]

% NEUTRINO COMMUNICATORS
% Neutrinos are particles that can pass through any
% solid matter with ease and are impossible to block. As
% a result, they make an ideal medium for communica-
% tions. Unfortunately, they are also easy to intercept.
% Even a tight beam of neutrinos sent between two lo-
% cations can be intercepted simply by placing another
% receiver behind the location the broadcaster is sending
% to. Neutrino communicators require a large power
% plant to power the high energy particle interactions
% required to generate the neutrino broadcast. Neutrino
% receivers are also relatively large, with the smallest
% occupying 100 cubic meters. In most cases, neutrino
% communicators are designed to broadcast neutrinos
% in all directions, though tight-beam transmissions are
% also possible. Quite often neutrino communications
% take advantage of quantum farcasting for security.

% Neutrino Transceiver: This transceiver is capable of
% generating and receiving neutrino signals at a range of
% at least 100 astronomical units. It is large, with a size
% of 8 cubic meters (in a cube 2 meters on a side), but
% they can be loaded onto large vehicles. To function, it
% must be connected to a large power plant, such as one
% found in habitats or large spacecraft. The cost and
% size of this device includes the computer necessary for
% quantum farcasting. [Expensive]

% QUANTUM FARCASTERS
% Quantum farcasters are special computers designed to
% protect a communications channel (such as fiberoptic,
% radio, laser/microwave, or neutrino) with unbreak-
% able encryption. To function, two or more quantum
% farcaster computers must first be entangled together
% (on a quantum level) in the same physical location.
% The farcasters may then be separated, at which point
% they may continue to exchange encrypted data via
% quantum teleportation. This data exchange requires
% a standard communications link (fiberoptic, radio,
% laser/microwave, or neutrino), and so is limited by
% the speed of light, but it is a high bandwidth form
% of communications. The quantum encryption used
% by these entangled farcasters is unbreakable, and any
% attempted interception is immediately detected and
% neutralized. A quantum farcaster may not be used to
% securely communicate with any farcasters other than
% the ones it is entangled with.

% %%% txt/317.txt

% Because it is exceptionally safe and secure, quantum
% farcasting via neutrino communications is the primary
% means of both long-distance communication between
% habitats and egocasting (p. 276). The neutrino signal
% cannot be blocked and it can only be decrypted if a
% character has access to the computer that is sending
% or receiving the signal.

% Miniature Radio Farcaster: Miniature farcasters
% communicate with each other using standard radio
% transceivers. As noted above, they may only securely
% communicate with the other farcasters with which they
% are entangled. Most miniature farcasters are worn as
% jewelry or fitted into clothing or other equipment. [Low]

% QUANTUM ENTANGLED COMMUNICATION
% The rarest form of communications is quantum en-
% tangled (QE) communication. QE communication is
% instantaneous and works over any distance, but is
% also very limited. QE communication requires pairs
% of entangled particles known as qubits. To use QE,
% large number of pairs of qubits are created and then
% separated from each other. Millions of these sepa-
% rated pairs of particles are stored in special containers
% known as qubit reservoirs. If two QE communicators
% each have a qubit reservoir containing qubits that are
% each entangled with qubits in the other communica-
% tor’s qubit reservoir, then characters can use the two
% QE communicators to commutate with one another
% instantaneously. Characters can use QE to instantly
% communicate between any two locations, even if one
% character is in the solar system and the other has
% passed through a Pandora gate and is standing on a
% planet 500 light years away.

% Each bit of data transmitted between these two QE
% comms uses up one qubit. Once all of the qubits are
% used up, the two QE comms can no longer communi-
% cate with each other until they each get a new batch
% of entangled qubits. Qubits are expensive to produce,
% contain, and transport, making this an exceedingly
% expensive form of communication. As a result, ex-
% tremely high bandwidth communications like full
% sensory AR and egocasting cannot be performed using
% QE communication.

% Portable QE Comm: This is a handheld FTL com-
% munications device. The actual communications unit
% can be made as small as desired, but must be large
% enough to connect to or hold a qubit reservoir. Be-
% cause qubit reservoirs are relatively large and must be
% replaced, they are rarely implanted. Some miniature
% farcasters are designed so that users can also attach
% qubit reservoirs to enable them to be used for both
% light speed and FTL communication. [Low]

% Low-Capacity Qubit Reservoir: Low-capacity qubit
% reservoirs can be used for 10 hours of high-resolution
% video conferencing or meshbrowsing and 100 hours
% of voice or text only communications. [High]

% High-Capacity Qubit Reservoir: High-capacity qubit
% reservoirs can be used for 100 hours of high-resolution
% video conferencing or meshbrowsing and 1,000 hours
% of voice or text only communications. [Expensive]

% BUGS AND

% SURVEILLANCE

% Though surveillance technologies are pervasive

% and easy to come by in Eclipse Phase, secretly

% obtaining information on someone who wants

% to retain privacy can be quite difficult. Micro-

% bugs, smart dust, and similar recording devices

% that are all but invisible may be exceptionally

% easy to put into place, but once they begin ac-

% tively transmitting, they are easy to to detect

% (see Wireless Scanning, p. 251). An eavesdropper

% may attempt to stealth the signal (see Stealthed

% Signals, p. 252), but this is not guaranteed to

% work. Once a signal is detected, locating the

% broadcasting device is usually just a matter of

% time (see Tracking, p. 251).


% Some recording devices attempt to avoid this

% problem by using miniature quantum farcasters

% (p. 314), but those are far larger and more dif-

% ficult to hide. Often the most effective way to

% acquire discrete information is to plant a surveil-

% lance device, set to record but not transmit, and

% then retrieve it later. While doing this is often

% difficult and risky, the recording device never

% reveals its presence by broadcasting and so is

% more difficult to detect.                        ■




% COVERT AND ESPIONAGE TECHNOLOGIES
% These technologies allow characters to acquire pro-
% tected information and to gain access to places that
% others try to keep them out of. Many of these devices
% are mesh-capable and equipped with radios, see p. 296
% for radio ranges.

% Chameleon Cloak: This loose, poncho-like cloak
% contains a network of sensors that perceive wave-
% lengths from microwave to ultra-violet. A similar net-
% work of miniature emitters precisely replicate the in-
% formation its sensors receive, making the wearer seem
% transparent to those wavelengths. A chameleon cloak
% allows a character to effectively become invisible as
% long as they are stationary or not moving faster than
% a slow walk. When worn by someone moving faster,
% the cloak still provides a +30 modifier to Infiltration
% Tests to avoid being seen or noticed.

% Chameleon cloaks are not effective against radar,
% x-ray, or gamma-ray sensors. They do hide the char-
% acter from thermal infrared, however, by absorbing
% the character’s body heat into its heat sink. The cloak
% can only absorb a character’s body heat for one hour
% before it must emit this heat. Heat emission also re-
% quires one hour, during which time the character is
% easily visible in the thermal infrared spectrum. [Low]

% Covert Operations Tool (COT): This handheld
% device is the ultimate in infiltration technology. It con-
% tains both smart matter micromanipulators, cutting

% %%% txt/318.txt
% tools, and an advanced nanotechnology generator
% capable of producing nanobots that can bore or cut
% through almost any material and disable or open
% almost any electronic lock.

% Cutting out a lock or boring a 1-millimeter hole in
% a wall with a COT requires ((Durability + Armor) ÷
% 10) seconds. Cutting out a 1-meter diameter hole in
% a wall requires ((Durability + Armor) ÷ 10) minutes.
% These same nanobots can later be used to repair this
% damage so that it is invisible to any but the most care-
% ful and detailed examination.

% A COT can easily open any old-fashioned mechani-
% cal lock simply by analyzing it and shaping an ap-
% propriate key, though this takes a full Action Turn.
% It can also open electronic locks by infiltrating them
% with nanobots that influence the lock’s electronics,
% no matter what authentication system the lock uses.
% Opening electronic locks takes a full Action Turn, but
% success is practically guaranteed. Opening an elec-
% tronic lock in this manner will, however, trigger an
% alarm and/or be logged as an event. For more details,
% see Electronic Locks, p. 291. [High]

% Cuffband: This smart plastic loop restricts around
% a prisoner’s limbs when activated. If the prisoner
% struggles, it will tighten more. Cuffbands will inform
% the user if they are cut or loosened and are electron-
% ically-controlled, so the user can release the prisoner
% remotely. Some cuffband variants including a shock
% system (treat as a shock baton, p. 334) to zap and
% restrain unruly prisoners. [Low]

% Dazzler: The dazzler is a tiny laser system set on
% a rotating ball. When activated, it consistently spins
% and emits laser pulses in all directions. These laser
% pulses are not dangerous, but they detect the lenses
% of camera systems (including specs, viewers, and bot/
% synthmorph sensors) and repeatedly zap them with
% laser pulses of varying strength to overload and dazzle
% them. For as long as a dazzler is active, any camera
% system (visual, infrared, and ultraviolet) within line
% of sight and within 200 meters is blinded. [Moderate]

% Disabler: This handy device emits an overloading
% surge that completely incapacitates and disables a
% synthetic morph or pod (anything with a cyberbrain)
% when it is plugged into an access jack and activated.
% The affected cyberbrain will be unable to function until
% the signal is deactivated, effectively shutting down the
% ego (or AI). In order to plug a disabler into an unwill-
% ing target, the target must first be grappled or a called
% shot must be successfully made in melee combat. This
% device does not work on larger synthetic morphs (like
% vehicles) or on cyberbrainless robots.[High]

% Fiber Eye: This is a flexible and electronically-con-
% trollable length of fiberoptic cable and viewer, which
% can be worked through cracks, under doors, and
% around corners to peep unobtrusively. [Low]

% Invisibility Cloak: This cloak is made of metama-
% terials with a negative refractive index, so that light
% actually bends around it, making it and anything it
% covers invisible. This invisibility works from the
% microwave to ultraviolet spectrums, but not against
% radar or x-rays. The drawback is that anything con-
% cealed within the cloak can’t see out. This is easily
% overcome by using external sensor feeds (if available)
% and entoptics to navigate. Alternately, a small piece of
% anti-cloak, which cancels the cloak’s invisibility prop-
% erties when touched together, can be used to create a
% small window to peep out of, though this increases
% the chance of being spotted. Noticing such a window
% requires a Perception Test with a –30 modifier. [High]

% Microbug: This device is a tiny camera and micro-
% phone 1 millimeter across. It has the visual capabili-
% ties of a set of specs (p. 325). It can hear everything
% within 20 meters and see everything within the same
% range that is in its line of sight. A microbug can record
% up to 100 hours of information. Microbugs can be
% set to broadcast continuously, at set intervals, or only
% when they receive a special signal. If desired, they
% can also be set to only record if there is movement
% or voices in the room they are in. Microbugs have
% adhesive backs and can stick to almost any surface.
% Microbugs can also establish their location via mesh
% positioning or GPS, and so double as tracking devices.
% To avoid being detected by their radio transmissions,
% some microbugs are attached to miniature quantum
% farcasters (p. 314). These microbugs are much larger
% (1 centimeter) and easy to see, but their transmissions
% cannot be detected or blocked. [Trivial, Low for quan-
% tum farcaster bugs]

% Prisoner Mask: This hood tightens around the
% head of a prisoner, blocks all vision frequencies, and
% engages in low-level jamming in order to prevent any
% wireless communication via mesh inserts. [Medium]

% Psi Jammer: This device jams frequencies used
% by brainwaves within a 20-meter radius. This has
% no effect on brain functions, but it does prevent
% any ranged used of psi sleights within this area of
% effect. [Moderate]

% Quantum Computer: These advanced devices
% make use of quantum computation, allowing them
% to handle extremely large numbers with ease. This
% makes them especially useful for codebreaking, as
% noted on p. 254. [Expensive]

% Smart Dust: This device is a walnut-sized special-
% ized nanobot generator that creates tiny sensor nano-
% bots, each one of which is a tiny sphere the diameter
% of a human hair. A packet of smart dust nanobots is
% sufficient to perform detailed surveillance on a large
% room like an auditorium has a volume of 1 cubic
% centimeter and contains 3 million nanobots. Each
% nanobot contains tiny cameras, microphones, a tiny
% computer, a radio, and chemical sensors, as well as
% short legs that allow them to walk and climb at a rate
% of 5 cm per second.

% When a character dumps a packet of smart dust in
% a room, it will cover every surface in the room within
% 20 minutes, including all furniture and the insides
% of every drawer and other space that is not airtight.
% At this point, the smart dust has recorded all data
% about the room that can be obtained by exceedingly
% detailed observation, including the DNA of everyone

% %%% txt/319.txt
% who has visited the room in the last week or two. The
% smart dust can then either broadcast a brief, highly
% compressed signal, or it can send all of its informa-
% tion to a few hundred nanobots that then walk to a
% pre-arranged destination for pickup and downloading
% by their user. The user need only find a single nano-
% bot with a nanodetector to acquire the information
% obtained by the smart dust. If ordered to do so, the
% remaining nanobots can either power down and await
% further orders or self-destruct in a fashion that turns
% them into a tiny amount of dust made mostly of metal
% and silicon. [Moderate]

% Traction Pads: This set of specialized fingerless
% gloves, shoes, and kneepads is designed to emulate the
% pads on geckos’ feet. Characters can support them-
% selves on a wall or ceiling by placing any two of these
% pads against any surface not made from a material
% specially designed to resist such devices. Characters
% can climb any surface and move easily across walls
% and ceilings that can support their weight (+30 to
% Climbing Tests). In addition to climbing, these devices
% are also very popular in zero-g environments. Wearing
% this item does not impair the user’s agility or manual
% dexterity. [Low]

% White Noise Machine: This small and wearable
% device generates masking sounds that protect a con-
% versation from being audibly recorded or overheard
% by anyone not in the immediate vicinity. [Trivial]

% X-Ray Emitter: This device is designed to be used
% with either the enhanced vision augmentation (p.
% 301) or specs (p. 325). It emits a focused beam of
% low-powered x-rays that allows the user of either
% device to both see and see through most objects using
% backscatter x-ray radiation (p. 303). This allows the
% character to literally see through walls and into con-
% tainers, including ones made of metal. [Low]



% DRUGS, CHEMICALS, AND TOXINS
% In Eclipse Phase, the transhuman desire to enhance
% the body and mind—especially with chemicals—
% merges right into humanity’s popular pastime of
% recreational substance abuse. Drugs of all kinds,
% whether they be chemical, nano-based, or electronic,
% are not only popular but widespread. While advances
% in biotechnology have eliminated many of the side
% effects that once plagued drug users, transhuman
% bodies remain complicated environments, and so side
% effects (especially with long-term use) are still a factor.
% Additionally, addiction is always a consideration for
% anyone who gets comfortable with popping the same
% pills too often, though there are also drugs for addic-
% tion of course

% Drug descriptions include benefits, side effects, no-
% ticeable signs that a person is using the drug, addictive-
% ness, and effects from long-term use). Descriptions also
% include the drug’s Duration and its Addiction Modifier
% (see Addiction and Substance Abuse).
% SUBSTANCE RULES
% These rules explain how to handle drugs and toxins.

% CLASSIFICATION OF SUBSTANCES
% Substances fall into four categories:

% Chemicals: These are pharmacologically-active
% small chemical compounds (toxins, pharmaceuticals,
% chemical drugs) that have been produced by chemical
% synthesis, nanotech fabrication, or enzymatic biosyn-
% thesis in (transgenic) organisms. They include natural-
% ly-occurring drugs from known species of (exo-)flora
% and fauna, endotoxins produced by biological organ-
% isms, enhancements of endogenic substances (designer
% drugs), and de novo developments designed for a
% specific medical or recreational application. Chemical
% drugs affect only biological morphs and pods.

% Biologicals: These include peptides, hormones, and
% biologically-based substances like biotoxins, bacteria,
% and viral organisms—drugs devised or based on
% naturally-occurring endogenic biological substances.
% This category also includes infectious biological or-
% ganisms that can produce drug-like effects, like virii
% and bacteria. Biologicals affect biomorphs and pods
% but not synthetic morphs or infomorphs.

% Nanodrugs: These are temporary nanobot colonies
% programmed to create a certain effect. While nano-
% bots are generally able to target or infect all morph
% types except infomorphs, exactly which morphs are
% affected usually depends on the pre-programmed
% effect (i.e., whether it targets a biological or mechani-
% cal mechanism).

% Electronic: Electronic drugs include software and
% technology that affect the brain directly, such as ma-
% nipulative XP programs or retro-tech like transcranial
% magnetic stimulation or cranial electrotherapy. It also
% includes narcoalgorithms—programs that reproduce
% drug-like effects for AIs, infomorphs, and egos resid-
% ing in cyberbrains.

% APPLICATION METHODS
% There are number of vectors by which a substance
% may be applied to a morph.

% Dermal (D): This drug or toxin is absorbed via
% the skin (or exterior hull with some nanotoxins) as
% either a gas, liquid, or solid (e.g., paste). Slap patches
% and slap bands are commonly used, loaded with the
% chemical DMSO, which transfers the drug through
% the skin.

% Inhalation (INH): This is a gas that is breathed into
% the lungs or snorted nasally. Used for inhalers, aero-
% sols, powders, and gas grenades/seekers.

% Injected (INJ): This liquid is applied via either
% an intramuscular or intravenous injection. Used for
% needles and piercing weapons.

% Oral (O): This is a liquid or solid that is absorbed
% through the stomach or oral cavity (eating or drink-
% ing). Used with pills and liquids.

% %%% txt/320.txt
% DRUG EFFECTS
% If a character is exposed to a drug via its method of
% application—for example, they pop a pill, slap on
% a dermal patch, are soaked with a splash grenade,
% breathe in gas, or get stabbed with a coated weapon—
% then they are subject to the drug’s effects. The onset
% time determines how long these effects take to kick in,
% and the duration determines how long they last.

% While there is no resistance test to ignore a drug or
% toxin’s effects once exposed, in some cases (especially
% toxins) a test might be called for to determine the
% severity of the effects.

% Unless otherwise noted or specifically overridden,
% medichines (p. 308) will protect a character from
% drug/toxin effects (but not nanodrugs/nanotoxins).
% Enhancements like toxin filters (p. 305) may also
% impede a drug’s effect or provide complete resistance.
% If an antidote is taken in advance or before the effects
% kick in, the drug will not work.

% ADDICTION AND SUBSTANCE ABUSE
% Some drugs are addictive, either physically (affect-
% ing the morph) or mentally (affecting the ego)—and
% sometimes both. Every time a character uses the drug
% (or after an appropriate amount of use, as determined
% by the gamemaster), they must make a WIL x 3 Test to
% avoid addiction. Each drug has an Addiction Modifier
% that will modify this test.

% Failure indicates that the character has become
% addicted—they immediately acquire the Addiction
% negative trait (p. 148). Addiction is measured in three
% levels: Minor, Moderate, and Major. The severity de-
% termines how often an addicted character needs the
% drug and what the negative effects of not using the
% drug are.

% An addicted character must continue to make WIL
% x 3 Tests as they use the drug, as determined by the
% gamemaster. Failure indicates the character’s addic-
% tion severity increases.

% The negative effects from not using a drug end
% whenever the character does the drug again. Durabil-
% ity and Lucidity penalties are not damage, but tempo-
% rary decreases to the character’s maximum values; the
% character immediately regains the lost Durability or
% Lucidity when they do the drug again.

% Addiction is of indefinite duration. To clean up,
% the character must stay off the drug for 1 week for
% each level of addiction. Resisting this craving is dif-
% ficult, and should at least require another WIL x 3
% Test, modified by the drug’s Addiction modifier. Play-
% ers and gamemasters are encouraged to roleplay an
% attempt to kick a habit. Each week the character is
% off the drug, the addiction drops by one level. When
% it reaches 0, the character is clean ... though there is
% always danger of a relapse.

% Physical addictions do not carry over to a new
% morph if the character resleeves, but mental ad-
% dictions do. If the character uploads and resleeves,
% the mental addictions persist, and the morph the
% character leaves behind remains physically addicted.
% This means that poor or unlucky characters may
% occasionally find themselves resleeved into a morph
% that has a physical addiction. In this case, the charac-
% ter is subject to the physical addictiveness of the drug
% but not the mental addiction, although if they break
% down and indulge in the drug, they may themself
% become physically addicted.

% Characters who resleeve as infomorphs can remain
% mentally addicted to a substance despite no longer
% having a body. The market is always happy to provide,
% though; a wide variety of narcoalgorithms mirroring
% the effects of most of the drugs described below are
% available for infomorphs and AIs. For the infomorph-
% ported narcoalgorithm version of any physically-
% only addictive drug described below, consider the
% Addictiveness to be effectively physical. The character
% remains addicted as long as they are an infomorph,
% but they do not remain addicted if they sleeve into a
% physical morph.

% DRUGS
% The drugs described here are usually (but not always
% beneficial), and are typically taken intentionally. Drugs
% and chemicals used offensively are described under
% Chemicals and Toxins, both on p. 323.

% Note that the drugs here are just a representative
% sampling. There are thousands if not millions of
% drugs in circulation in Eclipse Phase—gamemasters
% are encouraged to introduce their own, using these
% as guidelines.

% COGNITIVE DRUGS
%  Nootropics and similar drugs are intended to boost
%  the user’s mental faculties.

%  Drive: This nootropic speeds up left-right brain
%  hemisphere communication, stimulates idea produc-
%  tion, and improves concentration, with no usual side
%  effects. Users receive a +5 bonus to COG while the
%  drug lasts. [Low]

%  Klar: Klar boosts alertness and enhances clarity and
%  perception. Users report a feeling of being “elevated”
%  to a higher level. They receive +5 INT while the drug
%  lasts. [Low]

%  Neem: Neem is a mnemonic drug that works by
% “tagging” experiences and mental input with a set of
%  unique sensations that contribute to the formation
%  of state-based memories. Neem gummy chews come
%  in a variety of fruit flavors shaped like extinct old
%  Earth animals. Neem gives characters a +20 bonus
%  on COG Tests to recall information they learned
%  while on Neem (see Memorizing and Remembering,
%  p. 176). The drawback to Neem is that memories they
%  accumulate while under the drug’s influence have no
%  emotional association. For example, a character who
%  witnessed something horrible happening to a friend
%  or who had a fight with a romantic partner while on
%  Neem would feel no emotional connection whatsoever
%  to what happened. [Moderate]

% %%% txt/321.txt
% COMBAT DRUGS
% Combat drugs are an easy way of evening the odds
% in a fight.

% BringIt: In some respects more a social than a
% combat drug, BringIt stimulates massive bursts of ag-
% gression pheromones designed to make the user the
% center of attention in a fight. In combat, opponents
% within 3 meters of the character not already in un-
% armed or melee combat with another character must
% pass a WIL x 3 Test or attack the character using Br-
% ingIt. The nature of airborne pheromones is imprecise,
% however, so if the character using BringIt is within
% 1 meter of another character hostile to the character
% affected, the affected character may opt to attack the
% proximate character instead of the BringIt user. Char-
% acters using this drug suffer a –20 modifier on social
% skill tests. [Low]

% Grin: Grin is an effective opiate and pain sup-
% pressant. Users may ignore the –10 modifiers from
% 2 wounds (not cumulative with similar effects), and
% in fact may not even be aware they are injured. Grin
% users suffer from tunnel vision, however, and so suffer
% a –10 modifier on Perception Tests. [Low]

% Kick: Kick is a strong stimulant that increases the
% user’s response time and puts them on edge. The char-
% acter gains +10 REF and +1 Speed for the duration of
% the drug. Characters under the influence of Kick are
% twitchy, however, reacting in a jumpy, cat-like fashion
% to sudden or unexpected stimuli. At the gamemaster’s
% discretion, they must make a WIL x 2 Test or react
% without thinking towards unexpected noises or other
% surprises. Long-term users suffer –5 COO. [Moderate]

% MRDR: MRDR is a straightforward and brutal
% combat drug. It increases pain tolerance, speed, and
% strength. The character receives +10 SOM, +1 Speed,
% +10 Durability, and may ignore the –10 modifier of
% one wound. Any damage incurred while under the
% effects of the drug is taken from the bonus Durabil-
% ity first. MRDR users are easily identifiable by the
% broken blood vessels in their eyes, tense posture, and
% visible tension in the muscles of the face, arms, and
% legs. Long-term users suffer –5 SOM. [Low]

% Phlo: Phlo increases alertness and coordination,
% making the user more graceful and nimble in a fray.
% The character gains +5 COO and +10 on Perception
% Tests for the duration of the drug. Everything feels
% possible to a character on Phlo, and so they are vul-
% nerable to being goaded into actions that might be
% foolish or dangerous (apply a –10 modifier to appro-
% priate Social Skill Tests). [Moderate]

% HEALTH DRUGS
% Pharma-foods that boost the consumer’s health and
% physical state are common.

% Bananas Furiosas: This drug reverses some of the
% effects of de-ionizing radiation on the cells of the
% body. Although a pill form is available, it most com-
% monly comes in large bunches of bright orange-red
% bananas. Bananas reduce the severity of a radiation
% dosage (gamemaster determines effect). [Low]

% %%% txt/322.txt

% Comfurt: This tasty yogurt treat blocks stress hor-
% mones, stabilizes mood, and relieves anxiety, allowing
% them to ignore the effect of 1 trauma and temporarily
% boosting Lucidity by +5. Any stress suffered while the
% drug is in effect is taken from the bonus Lucidity first.
% Comfurt also provides a +10 bonus when resisting
% attempts to manipulate the user’s emotions. Excessive
% use of Comfurt can lead to chronic itchiness caused by
% histamine release. [Low]

% RECREATIONAL DRUGS
% These drugs compete with petals (p. 321) and black
% market XP for wasting people’s time and lives away.

% Buzz: This gene-modified variant of BZ is an odor-
% less, invisible, extremely powerful hallucinogen. Users
% or affected characters will undergo extremely realistic
% hallucinations for the duration, and may even “share”
% hallucinations with other affected characters. Charac-
% ters will suffer a –30 modifier to any tests to remember
% what occurred while under the influence. [Moderate]

% Mono No Aware: Taken from the Japanese term
% for sadness at the ephemerality of worldly things, this
% drug, typically ingested as a tea, is a depressant that
% induces a meditative state. Mono No Aware gives the
% character a +10 bonus on Art and Sense Tests. With
% frequent use, Mono No Aware reacts with pigments
% in the skin to create a pallor with a slight bluish tinge,
% even in darker-skinned morphs. [Low]

% Orbital Hash: Good ol’ reefer—but grown in space
% using powerful lighting and post-singularity hydropon-
% ics. Because space is at a premium in habitats and scum
% barges, blocks of hashish are the preferred mode of




%                                                     DR


%                        TYPE APPLICATION ONSET TI


% Cognitive Drugs


% Drive                  Chem          O          20 minu


% Klar                   Chem          O          20 minu


% Neem                   Chem          O          20 minu


% Combat Drugs


% BringIt                 Bio      Inh, Inj, O     1 minut


% Grin                   Chem      Inh, Inj, O   3 Action T


% Kick                   Chem      Inh, Inj, O   3 Action T


% MRDR                   Chem          O          20 minu


% Phlo                   Chem          O          20 minu


% Health Drugs


% Bananas Furiosas       Chem          O          20 minu


% Comfurt                 Bio          O          20 minu


% Recreational Drugs


% Buzz                   Chem        Inh, O         1 hour


% Mono No Aware          Chem          O          20 minu


% Orbital Hash           Chem         Inh          3 minut


% Social Drugs


% Alpha                   Bio         Inh          1 minut


% Hither                  Bio          D           1 minut


% Juice                  Chem        O, Inh       20 minu
%  transport and delivery. However, for the wealthy and
%  on planets, buds in leaf form are not uncommon. Hash
%  allows the character to ignore the effects of 1 trauma,
%  but inflicts a –10 penalty on all memory-related tests
%  and Knowledge Skill Tests. Hash users exhibit blood-
%  shot eyes, lethargic behaviors, and the munchies. [Low]

%  SOCIAL DRUGS
%  These social lubricants affect the user’s interactions
%  with others.

% Alpha: Alpha is a more subtle version of BringIt,
%  popular with hypercorp execs, street thugs, and
%  anyone else who wants to come across as a domineer-
%  ing asshole. The pharm designer who invented it had
%  a retro sensibility (and maybe a sick sense of humor);
%  Alpha is typically synthesized as a sparkling white
%  powder designed to be snorted. Alpha stimulates
%  production of threat pheromones, but less bluntly
%  than BringIt. Alpha imparts confidence, a feeling of
%  power, and alertness. Users can function without sleep
%  for 4 days, after which point they need to catch up
%  with at least 4 hours of sleep (remember morphs with
%  basic biomods require less sleep). Dosed characters
%  receive a +20 modifier on Intimidation Tests and +10
%  on Persuasion and Networking Tests where attitude is
%  a factor (gamemaster discretion). These bonuses only
%  apply to characters within 2 meters of the Alpha user.

% On the downside, alpha users are impatient, unfo-
%  cused assholes. At the gamemaster’s discretion, Social
%  skill modifiers may be reversed to penalties with
%  certain types of people. Additionally, Alpha users
%  suffer –10 on all COG skill tests related to memory


% GS
% DURATION ADDICTION MODIFIER ADDICTION TYPE


%  8 hours            —                  Mental
%  8 hours            —                  Mental
%  12 hours           —                  Mental


% 15 minutes         +10                 Physical
%  3 hours            –10                Physical
%  2 hours            –10                Physical

% 1 hour            –10                Physical

% 1 hour            –10                Physical



% 1 day             —                    —
%  12 hours           –10                Mental


%  36 hours           —                  Mental
%  8 hours            –10                Mental
%  3 hours            —                  Mental


%  2 hours            –10                Mental
%  6 hours            –10                Physical
%  8 hours            —                  Mental

% %%% txt/323.txt
%  and coherent or logical thinking. Long-term users may
%  suffer the COG penalty even when not on the drug;
%  on it, they may be worse. [High]

%  Hither: Want to ooze sexy like a pleasure morph
%  on a hot tin roof? For those desiring that slinky je-ne-
%  sais-quoi, Hither is the tool. Hither is a clear, slippery
%  gel, sometimes with a faint, musky, floral scent. Hither
%  is applied to parts of the body with large concentra-
%  tions of sweat glands, where the skin quickly absorbs
%  it. Hither is a mild euphoriant, imparting a feeling of
%  confidence and you-know-you-want-it-ness to the user.
%  It also stimulates abundant production of lust phero-
%  mones. The character gains a +10 bonus on Persuasion
%  Tests against targets who are possible to seduce. At the
%  gamemaster’s discretion, this extends to Deception,
%  Impersonate, and Networking Tests. [Low]

%  Juice: This potent anti-depressant makes it almost
%  impossible to have bad feelings or negative thoughts.
%  The character is unnaturally happy—often irritatingly
%  or strangely so. The character receives a +30 bonus
%  against fear or attempts to manipulate their emotions
%  in a negative direction, but is also likely to act inap-
%  propriately, like giggling over the massive amount of
%  spilled blood or cheerfully changing the subject to
%  inane topics when someone else is freaking out. [Low]

%  NANODRUGS
%  Nanodrugs are temporary nanobot infestations that
%  apply a specific effect.

%  Frequency: Frequency (or Freeq) is a nanodrug de-
%  signed as a tool for scientific visualization. It releases
%  a small swarm of nanobots into the character’s blood-
%  stream that settle in the epidermis, where they act as
%  sensors of electromagnetic radiation. This sensory
%  input is then injected into the character’s visual and
%  tactile sensoria, hitting the user with a sequence of
%  novel stimuli, typically a light show or weird tactile
%  sensations. Aside from its recreational uses, Frequency
%  is good at picking up on localized field radiation with
%  a standard Perception Test. A character can take ad-
%  vantage of this to spot sensors and hidden electronics.
%  Similar to now-obsolete 20th-century hallucinogens
%  like LSD and psilocybin, however, a Frequency trip
%  can be disorienting and upsetting (the gamemaster
%  should apply any modifiers, mental stress, or even
%  trauma as they feel appropriate). Characters typically
%  experience a period about 1/3 of the way through
%  their trip in which sensory input is extremely intense;
%  during this period, which usually lasts about 2 hours,
%  they are unable to read. [Moderate]





%                                                 NANO
% NANODRUGS                  TYPE            APPLICATION
% Frequency                  Nano                Inj, O
% Gravy                      Nano                Inj, O
% Petals                     Nano                  O
% Schizo                     Nano                 Inj


% OTHER NANODRUGS


% Nanodrugs have the capability of making fun-


% damental changes to a body’s biochemistry and


% mental state. The potential effects are too nu-


% merous to list, but gamemasters should consider


% allowing nanodrugs that temporarily apply cer-


% tain traits, such as Brave, Direction Sense, Math


% Wiz, Pain Tolerance, Psi Chameleon, Psi Defense,


% Situational Awareness, Tough, Feeble, Frail, Low


% Pain Tolerance, Mental Disorder, Mild Allergy,


% Neural Damage, Psi Vulnerability, Severe Al-


% lergy, Timid, VR Conditioning, VR Vertigo, Weak


% Immune System, or Zero-G Nausea. Similarly,


% the nanodrug could force the character into a


% particular mental emotional state, such as a bad


% mood, edginess, contentment, or overconfi-


% dence. Gamemasters are encouraged to experi-


% ment with different possibilities and effects. ■





%  Gravy: Gravy assists characters in acclimating to

% high gravity environments. It comes in a variety of

% flavors and is often added as a sauce to food. For

% Gravy to be 100% effective, the character must begin

% using it in advance. Reduce penalties for high-gravity

% acclimation by 20. [Low]


%  Schizo: Schizo is a nanodrug that mirrors the ef-

% fects of paranoid schizophrenia. It is popular in

% some hyperelite social circles as a truly daring and

% intriguing experience. A dose of schizo looks like a

% disposable antique razor blade. Making an incision

% in the skin releases a swarm of nanobots that travel

% to the central nervous system and induce the effects

% of the drug. While in effect, the character is severely

% paranoid and hears voices. How this plays out is at

% the discretion of the gamemaster, but should include

% irrational fears, unusual compulsions based on the

% instructions of the voice or voices, and a strong pos-

% sibility that the character will behave in a violent or

% destructive fashion. The character may make WIL x 3

% Tests to avoid violent acts against objects or strangers.

% Friends and trusted acquaintances are probably less

% likely to be targets of violence (+30 modifier to avoid

% hurting people the character cares about or destroy-

% ing important possessions). Note that the character’s

% muse is unaffected by Schizo and can make efforts

% to babysit the character. Characters who take Schizo

% suffer 1d10 mental stress. [Low]



% RUGS
% URATION         ADDICTION MODIFIER         ADDICTION TYPE
%  8 hours                  –10                   Mental
%  special                  —                       —
% hours–1 day           +10 to –20                Mental

% 1 day                   —                     Mental

% %%% txt/324.txt
% SAMPLE PETALS
% A few examples of Petal experiences:

% FORGOTTEN HAND
% One of the character’s hands detaches and makes a
% run for it. The character is conscious and able to in-
% teract normally with the real world, but they cannot
% perceive the “escaped” hand and firmly believe that
% it’s getting away. The hand will lead the character
% a merry chase, but at some point, a new hand ap-
% pears on the character’s wrist. It may be glittery and
% opalescent, demonic and clawed, or bestial. Eventu-
% ally, after an hour or two, the character will catch up
% to their hand, but to get rid of their new hand and
% re-attach the old, they must answer cryptic questions
% posed by a gnome-like being.

% DARKLY SELVING
% This petal is believed to achieve many of its effects
% by connecting to the mesh, where an AI observes
% and controls some of the event flow, and only
% works for multiple trippers. Like Forgotten Hand,
% it works by overlaying AR perceptions on the real
% world, but because of the effects, it’s highly inadvis-
% able to take in places where any non-trippers will
% be present. Darkly Selving creates an epsilon fork
% of each character tripping and sleeves the fork in
% an infomorph that looks like a demonic version of
% themself, using visual input from the character’s
% co-trippers. AR overlays cause the characters to per-
% ceive themselves as angelic beings, while the real-
% seeming demonic infomorphs appear as AR overlays
% on their real world perceptions. What happens next
% varies, but generally both the characters and their
% forks are subjected to a series of strong chemical





% PETALS
%  Petals is a term for a type of narrative hallucinogen,
%  a nanodrug that hijacks the senses and takes the user
%  on a game-like, highly immersive trip. Known by a
%  myriad of intriguing names—Forgotten Hand, Darkly
%  Selving, Inquisitive Green, to name a few—Petals are
%  post-Fall society’s heroin—the drug of choice for the
%  desperate and fucked. Petals almost always appear as
%  nanopharmaceutical flowers, potted or with a nutrient
%  pack attached to the stem. Plucking and swallowing
%  the petals from the flower triggers the effects imme-
%  diately. Flowers have 5-10 petals. Multiple users may
%  share the experience if they take the Petals within 1
%  minute of the first one being plucked; after this all
%  petals remaining on the flower fade to translucent
%  white and become inert.

% Petal experiences are like entire scenarios in and of
%  themselves. Some take place entirely in the user’s mesh
%  inserts (the user must cede control of their implants

% d narcoalgorithmic stimuli, ranging from Hither-
%  e effects to massive doses of MRDR (or sometimes
% oth). The effects directed against the forks are

% nerally much more intense. The objective—hinted

% via environmental clues—is to merge with one’s
%  rk, which can be accomplished in a variety of ways,
%  nging from hunting them down and eating their

% art to solving a puzzle or reaching a goal before
%  eir forks can.

%  ELPHINIUM SIX
%  e last and rarest in a series of petals, Delphinium
% x is the Grail of petal users, a supposedly tran-
% endental experience that might not even exist.
% elphinium One is scarce, Two and Three are quite
%  re, Four is an amazing find, and Five and Six
%  e only rumors. Hints of what Six might hold are

% sed largely on extrapolation from the little that
%  known about the lower-numbered petals. The
%  llowing facts are generally accepted. It is a group
%  perience, but not all members of the tripping
%  oup are rewarded equally. It is intensely surreal,
%  t in a purposeful way, as are all of the Delphinium
%  ries. It concludes the loosely-built narrative of a
%  ugged-out version of a fairy tale princess and her
% uest for enlightenment begun in Delphinium One,
%  plete with strange omens and mythological crea-
%  res. Rumors of what the ending might hold are
% ore fanciful, and range from the trippers being
%  sleeved in god-like infomorphs to them being
% apped forever in an ego prison. Delphinium Six is
%  mpletely virtual, leaving the characters comatose
%  r the duration, and probably lasts a long time,

% rhaps 40 hours.                                  ■





% voluntarily; if they do not, the drug has no effect other

% than producing very low-intensity LSD-like visual hal-

% lucinations), taking control of the character’s entoptic

% displays, linking to secretive mesh servers and other

% trippers, and invading the character’s sensorium with

% AR “hallucinations.” Others put the character into a

% near-comatose state during which they go on a head

% trip. Normally there is some kind of well-developed

% theme or plot to a Petal experience, although in some

% cases they just experience a stream of images.


%  Though most societies seek to suppress Petals, new

% ones appear constantly, fueled by a persistent sub-

% culture of crafters and users. Petalcrafters view their

% work as an art form (or at least as really good enter-

% tainment), and the better Petals are lovingly crafted,

% hauntingly beautiful experiences—even if they’re also

% terrifying. The subculture of Petal use ranges from

% casual users who occasionally do an easy, short-du-

% ration flower to hardcore addicts who spend much of

% %%% txt/325.txt
% their time not on Petals trying to hunt down the most
% intense and esoteric varieties. From this subculture
% comes a lot of information on what various Petals
% look like and their effects. Because Petals combine
% custom nanobots with tailored chemical payloads and
% sometimes connections to mesh servers, duplicating
% them using fabricators is impossible, leading to an
% active market of crafters, dealers, and traders.

% Petals sometimes contain easter eggs and rewards,
% called “sweets” by petal users. Getting the sweets usu-
% ally requires fulfilling certain conditions within the
% trip, such as correctly answering questions or fulfilling
% goals. Typical sweets include skillsofts, new clothing
% or product designs, and custom infomorph sleeves.

% On the negative side, some Petal trips go bad, in-
% flicting 1d10 mental stress or more on the user. Per-
% haps worse, some Petals are loaded with malware that
% takes over the user’s mesh inserts and worse—some
% sentinels even whisper of Petals carrying strains of the
% Exsurgent virus. [Trivial to High]

% NARCOALGORITHMS
% Narcoalgorithms are software programs that simulate
% the effects of drugs on biological bodies. Almost all
% bio, chemical, and nano drugs can be replicated as
% narcoalgorithms, with corresponding effect (game-
% master discretion). Narcoalgorithms may be run by
% infomorphs, egos encased in cyberbrains (pods and
% synthmorphs), simulmorphs, and even AIs.

% DDR: Originally crafted by prankster hackers
% and distributed as a virus, DDR (for “Dance Dance
% Robot”) triggers impulses in the target’s motor con-
% trol circuits. Primary targeting robot AIs, the effect is
% that targets “dance” in jerky, automated movements.
% Pleasure receptors are also activated so that dancing—
% and movement of any kind—feels good. Different
% software variants invoke different motions and styles.
% The target suffers a –20 modifier on other actions
% while dancing, but the dancing may be overridden
% with a WIL x 3 Test. [Low]

% Linkstate: This software actually connects the user
% to a peer-to-peer network, where it randomly con-
% nects to other linkstate users and samples a bit of their
% XP feed and randomly accessed memories—typically
% just enough to provide context, but not enough to
% acquire private personal details. These inputs are
% spliced together, their emotional inputs amplified, and
% then the entire package is spiked with some hormonal
% circuit triggers and artificial synaesthesia. The effect
% is a mind-blowing mixed sampling of people’s lives,
% mashed together in a sensory soup, that hits the mind
% with a euphoric rush. Linkstate users are catatonic
% while under the effects (typical sessions run 3-4 hours),
% but afterwards they often report that they have flash-
% backs of events in other people’s lives. [Low]

% CHEMICALS
% Atropine: Though poisonous in large doses, atropine
% is an effective antidote against nerve agents like BTX2
% and Nervex. Easily synthesized in a maker, atropine
% will avert the effect whether taken soon before or after
% dosage by a nerve agent. [Trivial]

% DMSO: This chemical acts as a carrier, allow-
% ing other chemicals to be absorbed through the
% skin. It allows any chemical agent to be applied
% dermally. [Trivial]

% Liquid Thermite: Similar to scrapper’s gel, liquid
% thermite comes in a gel form that is easily applied
% under all environmental conditions (by the nature
% of its chemical reaction, thermite is oxygenated and
% will burn underwater or in space). It is ignited with
% an electric charge, burning at temperatures exceeding
% 2,500 degrees Celsius and melting through whatever
% it is touching. Liquid thermite inflicts 3d10 + 5 DV per
% Action turn to whatever it is touching. Armor will also
% be burnt through, offering no protection once the full
% Armor rating has been reached. [Moderate]

% NotWater: NotWater is an effective liquid fire re-
% tardant that does not get objects wet, no matter how
% absorbent they are—it simply beads up and slides
% right off. [Trivial]

%  Scrapper’s Gel: This goo turns into a potent acid
% when given an electrical charge. It comes in a gel-like
% state and may be smeared like jelly, and may even be
% used in space. In acid form, scrapper’s gel does 1d10
% + 5 DV per Action Turn to anything it touches, unless
% the material has been treated against acid. Armor will
% protect against this acid at first, but the acid will eat
% through the armor, so that it will no longer protect
% after its full armor value has been reached. [Low]

%  Slip: This liquid is almost entirely frictionless. When
% spread around an area (commonly used in splash
% grenades), anyone attempting to walk or run on the
% affected surface must make a COO Test or fall down.
% Likewise, any coated surface becomes extremely
% hard to grip onto, requiring a SOM Test to hang on.
% Anyone attempting to grapple a slip-soaked character
% suffers a –30 modifier. [Low]

% Tracker Dye: This liquid is colorless at normal light
% but becomes recognizable under pre-specified different
% wavelengths (such as infrared or ultraviolet). [Trivial]

% TOXINS
% Chemical warfare involves using the toxic properties
% of biological and chemical substances to kill, injure,
% or incapacitate an enemy. Note that an antidote can
% be constructed for most toxins if a sample is acquired
% and an appropriate Medicine or Academics Test is
% made. This is considered a Task Action with a time-
% frame of 1 hour. These toxins only affect biomorphs;
% synthmorphs are immune.

% BTX : BTX-squared (also called Frog Bite) is a
% genetically-enhanced variant of the extremely potent
% cardiotoxic and neurotoxic batrachotoxin. It leads to
% fast paralysis and cardiac arrest that usually kills the
% target within a few Action Turns. Affected characters
% suffer 2d10 + 10 damage a turn for 3 Action Turns;
% medichines reduce this damage by half. They must
% also make a SOM x 2 Test (+30 with medichines) or
% be paralyzed for 1 hour. [High]

% %%% txt/326.txt

% CR Gas: This potent incapacitating agent causes
% eye twitching and temporary blindness, severe cough-
% ing and breathing difficulty, skin irritation, and panic.
% Affected characters suffer 1d10 ÷ 2 damage, a –30
% modifier to sight-based Perception Tests, and a –20
% modifier to all other actions for 20 minutes (5 min-
% utes if the character has medichines). [Low]

% Flight: This drug is derived from human pheromones
% released due to fear, and is intended to instill alarm or
% even terror in the character. Affected characters must
% make a WIL x 3 Test (+30 with medichines) or suffer
% a panic attack, inflicting 1d10 stress. Dosed characters
% also suffer a –30 modifier for resisting intimidation or
% fear-based emotional manipulations. Flight affects last
% for 1 hour (5 minutes with medichines). [Low]

% Nervex: Derived from deadly nerve agents like cy-
% closarin, VX, and novichok, this genetically-modified
% toxin is deployed as a colorless, odorless gas that
% turns safely inert 10 minutes after deployment. It
% causes involuntary contraction of the muscles, sei-
% zures, and death by respiratory failure. One minute
% after exposure, the character must make a SOM Test
% or be incapacitated by seizures, paralysis, or nausea
% and vomiting; unaffected characters still suffer a –20
% modifier to all actions. After 10 minutes, the character
% will die unless an antidote (such as atropine, p. 323) is
% applied. Characters with medichines suffer the initial
% effects, but recover after 5 minutes. [High]

% Oxytocin-A: A genetically-improved variant of oxy-
% tocin, this drug induces trust in the recipient. Drugged
% characters suffer a –30 modifier on all WIL and Kine-
% sics Tests where trust is a factor. Medichines provide
% immunity. [Low]

% Twitch: Twitch is a convulsive agent, a nonlethal
% nerve gas. Affected characters must succeed in a SOM
% Test (+30 with medichines) or become incapacitated
% with severe muscle tremors. Unaffected characters still
% suffer a –20 on all actions. The effects of Twitch last for
% 10 minutes, 5 if the character has medichines. [Low]

% NANOTOXINS
% Disruption: This nanotoxin attacks the myelin sheath
% on nerves, disrupting nerve impulses and inflicting
% symptoms of multiple sclerosis. Every hour the morph
% suffers a –5 modifier to COO, REF, and COG. If any
% aptitudes are reduced to zero,the morph is effectively
% paralyzed and catatonic. [Moderate]

% Necrosis: Necrosis nanobots attack the walls of cells
% inside the body, killing tissue. This nanotoxin inflicts
% 1d10 ÷ 2 damage per Action Turn for one minute, after
% which the nanobots disable and flush from the body.
% Necrosis only affects biomorphs. [Moderate]

% Neuropath: These nanobots are designed to stimu-
% late the pain receptors of a morph on a systemic level
% to cause agony and impairment. While most neuro-
% paths target biological receptors, variants are avail-
% able that induce comparable (phantom) pain stimula-
% tions in the cyberbrains of synthmorphs to create an
% equivalent effect. The affected character must succeed
% in a WIL x 3 Test or become incapacitated. Even if
% they succeed, they suffer –30 from the inflicted agony.
% Any form of pain resistance that allows a character
% to ignore wound modifiers will negate the neuropath
% pain modifier by an appropriate amount. [Moderate]

% Nutcracker: Nutcrackers are nanobots designed to
% locate, migrate, and decompose the synthdiamond
% case of a cortical stack within a morph by attacking
% its crystal lattice. This process takes approximately
% 6 hours, after which the cortical stack is destroyed.
% These nanobots also attack the cortical stack’s connec-
% tions to the (cyber)brain and brain-mapping nanobots.
% After 1 hour, the victim will be aware that their corti-
% cal stack is threatened. After 3 hours, all connections
% will be severed and the cortical stack will no longer be
% able to back up the character. [High]

% PATHOGENS
% A pathogen is an infectious biological agent that
% causes disease or illness to its host. While natural
% pathogens rarely strive to kill their hosts, germ war-
% fare programs revived during the Fall—or instigated
% by the TITANs—sought to modify and use pathogens
% as a weapon of war. The ideal characteristics of lethal
% biological agents are high infectivity, high potency,
% availability of vaccines, and delivery as an aerosol.
% Most biomorphs are immune to standard pathogens
% thanks to their basic bio-mods, and medichines will
% protect against most others. However, even these
% defenses may not protect against diseases left by
% the TITANs or a new terrorist cell’s biowar bug. It
% is largely recommended that pathogens be handled
% as a plot device, rather than an active threat to the
% characters. Pathogens have no effect on synthmorphs.

% Degen: Characters exposed to this degenerative
% neurological disease must make a DUR x 2 Test or
% become infected. Medichines will defeat the disease,
% but others will not show signs of infection for 1 week,
% when the symptoms of a rapidly progressing dementia
% will become clear: memory loss, personality changes,
% and hallucinations. If untreated, Degen will progress
% for another week with more serious symptoms, in-
% cluding speech impediments, jerky movements, loss
% of balance and coordination, and even seizures. This
% is reflected by a 5 point loss in all aptitudes per day
% (after the first week). When any aptitude reaches 0,
% the character dies. Degen is notorious for its effect
% in corrupting cortical stack backups before infection
% symptoms manifest. [Expensive]

% Trigger: Trigger is a designer virus that selectively
% targets and infects mast cells to trigger a hyper-allergic
% reaction. The resulting anaphylactic shock due to sys-
% temic vasodilatation (associated with a sudden drop
% in blood pressure) and bronchial swelling (resulting in
% constriction and difficulty breathing) usually leads to
% death in a matter of minutes after onset, if not treated.
% Infected characters must succeed in a DUR Test (using
% their current Durability score minus damage) or die
% quickly. Even medichines have difficulty reacting in
% time against this virus; characters with medichines
% must make a DUR x 2 Test to survive. [Expensive]

% %%% txt/327.txt
% PSI DRUGS
% Research into the Watts-MacLeod strain has resulted in
% several exceptional breakthroughs involving the creation
% of psi-impacting drugs. Each of these drugs is in the ex-
% perimental stage, but they are already finding some use
% among Firewall and similar secretive groupings.

% Inhibitor: Inhibitor is a cocktail of neurochemicals
% that block some brain receptor and transmitter func-
% tions in an attempt to reduce psi-waves and block or
% impair sleights. This drug is commonly used to restrain
% async prisoners from using their abilities. A drugged
% character must make a WIL x 2 Test. If they fail, they
% lose all psi abilities for the drug’s duration. If they suc-
% ceed, they suffer a –30 impairment on Psi skills and
% all strain is doubled. Inhibitor has an unfortunate side
% effect of doping the character down, however; apply
% a –10 modifier to their COG. Inhibitor-influenced
% characters tend to have a glazed, dopey expression and
% have difficulty getting excited or emotional. [High]

% Psi-Opener: Psi-opener drugs are variants of the
% Watts-MacLeod strain with a temporary effect and
% which do not permanently alter the user’s brain. Psi-
% opener temporarily imbues the user with the ability
% to use one particular sleight, regardless of whether or
% not they have the Psi trait. Each type of Psi-opener is
% customized for a particular sleight. While primarily
% intended for non-asyncs, non-asyncs may not possess
% Psi skills, so they must default to WIL. For this reason,
% Psi-Opener is often doubled up with Psike-out.

% Using Psi-opener is a mind-wrenching experience.
% Users are occasionally subject to hallucinations
% (gamemaster discretion). When the drug wears off, it
% inflicts 1d10 points of mental stress, +2 if the drug
% imbues a psi-gamma sleight. [Expensive]

% Psike-Out: Psike-out bolsters an async’s psi abilities.
% Apply a +20 modifier to the async’s Psi skill tests for the
% drug’s duration. However, also apply +2 to all strain
% DVs for the drug’s duration. Psike-out is mentally ad-
% dictive, with an Addiction Modifier of –10. [Expensive]



% EVERYDAY TECHNOLOGY
% The following devices are all exceptionally common
% and can be acquired in almost any habitat. Almost
% everyone in Eclipse Phase either owns these devices or
% knows several people who do.

% Ecto: Ectos are the external version of basic mesh
% inserts (p. 300), minus the medical sensors. These col-
% orful devices serve as a wearable mesh terminal, PDA,
% locator, and camera-phone. The devices are flexible
% (often worn as bracelets), dirt-resistant, self-cleaning,
% and may be stretched out to increase screen size. They
% may project holographic displays and are typically
% equipped with wireless-enabled glasses or contact
% lenses and decorative earpieces or earrings so that the
% user may access augmented reality. Given the ubiquity
% of mesh inserts, ectos are growing less common, but
% they are still used by bioconservatives, others without
% implants, and those who prefer to access the mesh via
% an external device for security concerns. [Low]


%                           TOXINS


%                 TYPE APPLICATION ONSET TIME
% Chemical Toxins
% BTX2                 Chem       D, Inj, O      1 Action Turn    3
% CR Gas               Chem        D, Inh        1 Action Turn
% Flight                Bio          Inh         3 Action Turns
% Nervex               Chem     D, Inh, Inj, O     1 minute
% Oxytocin-A            Bio        Inh, Inj        3 minutes
% Twitch               Chem     D, Inh, Inj, O 3 Action Turns
% Nanotoxins
% Degeneration         Nano        Inj, O         Immediate
% Necrosis             Nano        Inj, O        3 Action Turns
% Neuropath            Nano       D, Inj, O      3 Action Turns
% Nutcracker           Nano        Inj, O         Immediate
% Psi Drugs
% Inhibitor            Chem        Inj, O        3 Action Turns
% Psi-Opener            Bio        Inj, O         20 minutes
% Psike-Out            Chem        Inj, O          1 minute



% Holographic Projectors: These devices are capable
% of projecting high-definition, ultra-realistic three-
% dimensional images and movies. From a distance
% (20+ meters), such holograms can be difficult to
% distinguish as fake, but up close they are easier to
% see for what they are (+20 Perception Test modifier).
% Holograms do not appear wavelengths other than
% visual light, and so are easily identified by anyone
% with enhanced vision. [Low]

% Micrograv Shoes: These shoes are equipped with
% velcro and/or a magnetic system, allowing the wearer
% to walk normally on appropriate surfaces in micro-
% grav and zero-G environments, rather than floating or
% bouncing. [Trivial]

% Portable Sensor: This is a small portable (possibly
% even wearable) sensor system. The type of sensor must
% be chosen (for example: infrared, lidar, radar, x-ray).
% Combined sensor systems are also available, at a cu-
% mulative cost. See Radio and Sensor Ranges, p. 299.
% and Using Enhanced Senses, p. 302. [Moderate]

% Smart Clothing: Smart clothing can change its
% color, texture, and even its cut, taking only a minute
% or two to transform from a solid color jumpsuit to a
% plaid party dress or a replica of a pinstriped, late 20th
% century business suit. More advanced (and expensive)
% models can also attempt to camouflage their wearers,
% providing a +20 bonus to Infiltration Tests to avoid
% being seen or noticed, as long as the wearer is station-
% ary or not moving faster than a slow walk. When worn
% by someone moving faster, the clothing still provides a
% +10 modifier. Smart clothing also keeps the character
% warm or cool, allowing the character to exist comfort-
% ably in environments from –40 to 70 C. [Low]

% Smart Vac Clothing: Just like regular smart cloth-
% ing, this outfit can also transform into a light vacsuit
% (p. 333). It also functions as armor with a rating of
% 2/4. [Moderate]

% %%% txt/328.txt

% Specs: Specs are vision-enhancing glasses. They
% deliver sensory data directly into the wearer’s visual
% cortex by connecting with their basic mesh inserts
% (p. 300), though visual displays are available for bio-
% conservatives and other characters without implants.
% Specs extend the range of the wearer’s vision from
% terahertz waves to gamma rays (p. 302). Specs in-
% clude a t-ray emitter (p. 306), however, using x-rays,
% or gamma rays for visual purposes requires a sepa-
% rate emitter, since neither of these sorts of radiation
% are common inside habitats, or in any safe environ-
% ments. Specs have a variable focus equivalent to 5
% power magnifiers and provide the wearer with a +10
% bonus to all Perception Tests involving vision. [Low]

% Tools: Tools come in kits (portable), shops (can fit
% into a large vehicle), and facilities (large, non-mobile).
% Each set of tools applies to a particular skill, such
% as Hardware: Electronics or Hardware: Groundraft.
% [Low (Kit), High (Shop), Expensive (Facility)]

% Utilitool: This hand tool includes a specialized
% small nanobot generator. In its basic form, a utilitool
% is the size and shape of a large fountain pen. It can
% transform into almost any tool, however, from a
% wrench, knife, or powered screwdriver to a rotary
% grinder or pair of pliers. Some inexpensive utilitools
% are optimized for specialized tasks, like cooking
% or wilderness survival, but more expensive models
% become almost any imaginable hand tool. Utilitools
% are normally mentally controlled using the character’s
% basic mesh inserts. Characters without such implants
% can control the tool via voice commands and touch
% controls. Characters using a utilitool gain a +10
% modifier to skills involving repairing or modifying de-
% vices with mechanical parts, opening locks, disarming
% alarm systems, or performing first aid. [Low]

% Viewers: These small and highly advanced binoculars
% possess all the visual enhancement of specs (p. 325), but
% also provide 50x magnification. They also include a di-
% rectional microphone that magnifies sound from the di-
% rection the viewers are pointed by a factor of 50. View-
% ers provide the user with a +30 bonus to all Perception
% Tests involving vision or hearing for the target they are
% aimed at. This bonus is not cumulative with bonuses
% from any other device or augmentation. [Low]



% NANOTECHNOLOGY
% Nanotechnology is the precise manipulation of matter
% at the atomic level, typically using millions of micro-
% scale nanomachines. Nanotechnology transformed
% manufacturing, enabling new techniques and materi-
% als. The advent of nanofabrication—building objects
% from the molecular level up—transformed economies,
% allowing people to simply manufacture whatever they
% needed from raw materials. Nanotechnology is still a
% growing field, however, and has its limitations. While
% the TITANs unleashed self-replicating nanoswarms
% with the ability to transform or destroy anything
% through the power of geometric growth, such technol-
% ogy remains far beyond transhumanity’s grasp.

% BASIC NANOTECHNOLOGY
% Basic nanotechnology is exceedingly widespread
% and used throughout the solar system, serving as the
% primary method of manufacturing for decades. The
% nanobots of basic nanotech are confined to delicate
% and specially-maintained environments like the
% insides of cornucopia machines or healing vats and
% cannot operate elsewhere.

% HEALING VATS
% Healing vats were the first type of nanotech medicine
% developed and remain the most powerful medical
% devices in common use. With the exception of a few
% exceptionally deadly nanoplagues, a healing vat can
% cure any disease and heal any injury. As long as the
% patient is alive when they are place in the healing
% vat, they will not only survive, but emerge without a
% scratch. A healing vat can even take a severed head (as
% long as it has been stabilized by medichines or nano-
% tech first aid) and regrow an entire body based on the
% head’s genetics. If the patient’s body or medical records

% %%% txt/329.txt


%                                               HEALING
% INJURY

% Healing normal damage to a character who has taken 3 or fewer

% Restoring major lost body parts like arms or legs, or healing dying
% who has taken 4 wounds.
% Restoring recently dead character who was placed in medical stas
% who is mostly intact.
% Restoring recently dead character who is placed in medical stasis
% and who is missing most of their body.
% Augmentation
% Minor implants and bioware, minor cosmetic changes like alterati
% color or shape, or hair color, texture or distribution, minor alterati
% body fat distribution.
% Major brain and neural implants, nanoware or bioware, sex chang
% no more than 5% or weight by no more than 20%.
% Major physical modifications like adding limbs or radical changes



%  contain information about their implants, bioware, or
%  advanced nanotechnology, all of these modifications
%  are also fully restored.

%  Few people suffer injuries serious enough to require
%  a healing vat. Most are used as a safe and easy way
%  to perform bodysculpting or to install implants or
%  bioware. Healing vats use specialized nanomachines
%  to either alter the patient’s body or integrate implants
%  or bioware. One advantage of using a healing vat is
%  that no additional healing time is needed, the patient
%  leaves the vat fully recovered from the augmentation
%  and ready to go. Every hospital, clinic, bodyshop, and
%  augmentation parlor has several healing vats. The
%  time required by a healing vat varies with the sever-
%  ity of the damage it is healing or the extent of the
%  modification being made, as noted on the Healing Vat
%  table, p. 327. [High]

%  NANODETECTORS
%  Nanodetectors are small devices that suck in air and
%  micro debris in order to scan for and detect nano-
%  bots. Given that nanobots are so small, the density of
%  nanobots in the area has a large impact on its success.
%  The nanodetector has a base skill of 30 for detecting
%  nanobots, modified by +30 if an active nanoswarm or
%  hive is present, +0 if a nanoswarm or hive was active
%  recently, and –10 for the presence of nanobots outside
%  of a swarm or hive. Once a nanobot is detected it may
%  be analyzed either by the user or the nanodetector’s AI,
%  using Academics: Nanotechnology 30 skill. Nanode-
%  tectors are often worn and left on, set to alert the user
%  if a hostile nanoswarm is detected. [Low]

%  NANOFABRICATORS
%  Nanofabrication machines are universal assemblers
%  that perform almost all of the manufacturing in the
%  solar system. The user loads in raw materials and
%  electronic plans and it can produce literally any
% AT TABLE


%                            HEALING TIME


%                           2 hours per wound
% nds.


%                       (min. 1 hour for 0 wounds)
% early dead character


%                          12 hours per wound

%  avoid death, but


%                            1 day per wound

% void death,


%                           3 days per wound


% n skin color, eye
% o face shape or                     1 hour

% hanging height by


%                                12 hours

% eight and weight.                   3 days




% manufactured good, from a weapon to an ultralight

% plane to a hot and delicious dinner. Many nanofabri-

% cators come equipped with a library of common-use

% blueprints (basic foods, standard clothing, common

% tools, etc.). Other blueprints must either be purchased

% online, self-programmed, or acquired through some

% other method (see Nanofabrication, p. 284). The larg-

% est nanofabrication units are more than 10 meters on

% a side and are used to produce small consumer goods

% in bulk as well as building large devices like orbital

% transfer vehicles.


%  The availability and legality of nanofabricators

% varies widely throughout the system. In the inner

% system and Jovian Republic, cornucopia machines

% are commonly restricted and sometimes illegal, with

% licenses only available to hypercorps, military units,

% and other officials and elites. In these habitats, only

% more limited fabbers are available to the general

% populace. Additionally, blueprints are licensed and

% protected by copyright laws, and many nanofabrica-

% tors feature pre-programmed restrictions that prevent

% them from using unlicensed blueprints as well as

% from manufacturing weapons, explosives, or other

% restricted items. Among the autonomists of the outer

% system, however, nanofabricators are commonly ac-

% cessible, shared by everyone, and unrestricted.


%  For rules on creating goods in a nanofabricator, see

% Nanofabrication, p. 284.


%  Desktop Cornucopia Machine: Cornucopia ma-

% chines (CMs) are general-purpose nanofabricators.

% The smallest CMs are desk-sized cubes approximately

% half a meter on a side with a volume of at least 40

% liters. They can produce any small object, from tools

% to well-folded suits of clothing to handguns or a bowl

% of cereal. It is sometimes possible to assemble larger

% items, but they must be manufactured in smaller

% pieces and then assembled (likely requiring an appro-

% priate Hardware Test).

% %%% txt/330.txt

% While users can purchase bulk raw materials, CMs
% also come equipped with a disassembler. The user
% loads garbage and other objects into the disassembler
% so that they can be turned into raw materials for the
% CM. All legally-available disassemblers only decon-
% struct non-living material. [Expensive]

% Fabber: Fabbers are specialized nanofabricators,
% portable and considerably smaller than CMs. There
% are a wide variety of portable fabbers, including
% ones that can make any hand tool or small piece of
% personal electronics, ones that can turn any organic
% material into food and drink, and ones that can create
% any drug or medicine as well as bandages and spe-
% cialized dressings. The most common fabbers have a
% volume of 4 liters. Larger hand tools and devices are
% produced as 2 or 3 separate parts that must be fitted
% together. Like CMs, fabbers also contain miniature
% disposal units. [Moderate]

% Maker: Makers are specially-designed to produce
% food and drink for the user. Raw materials can be pro-
% vided by the addition of any water-containing liquid
% and collected biomass like leftover food, grass, dirt,
% dead animals, or transhuman waste. Some models are
% built into standard vacsuits. Makers can produce water
% and various flavored beverages, as well as ration bars
% or thick pudding-like edible gels. With adequate raw
% material, a maker can indefinitely provide food and
% drink for up to three transhumans. Most units, how-
% ever, have a very limited range of flavors and textures
% that are widely considered to be fairly bad. Models
% with a wider and better range of flavors and textures
% are more expensive, but produce food that is consid-
% ered adequate or occasionally good. [Low to Moderate]

% Blueprints: If you want a nanofabricator to make
% something, you need to instruct the device how to
% create it from the molecular level up. Such blueprints
% are available for almost every conceivable item out
% there. The cost of such blueprints typically equals
% the cost of purchasing the item, though factors like
% legality and quality may affect the cost as usual (see
% Acquiring Gear, p. 296). [One Cost Category Higher
% Than Item Cost]

% ADVANCED NANOTECHNOLOGY
% Advanced nanotechnology includes more recent
% developments. Like basic nanotech, advanced nano-
% technology cannot self-replicate but the nanobots
% can function normally in most environments and are
% highly resistant to bacterial attacks and other environ-
% mental problems. Typical advanced nanotech consists
% of a generator—known as a “hive”—that produces
% nanobots as long as it is supplied with raw materials.
% Every such hive also includes a miniature disassembly
% unit and/or specialized nanomachines that collect
% raw materials for the generator. These hives produce
% nanobot swarms that are set loose to perform some
% function in the world.

% Examples of advanced nanotech include COTs (p.
% 315), medichines (p. 308), smart dust (p. 316), and
% utilitools (p. 326), among others.

% General Hive: General hives are capable of produc-
% ing any conceivable type of nanobot with the right
% blueprints and/or programming. Even at their smallest
% size they are not really portable, with a minimum size
% being cubes 30 centimeters on a side and a volume of
% 25 liters. [Expensive]

% Specialized Hive: Specialized hives are far more
% common than general hives, though they can produce
% only one type of nanomachines (i.e., choose one type
% of nanoswarm per hive). The smallest specialized
% hives are approximately the size of a 12-gauge shot-
% gun shell or a large cherry tomato. [High]

% EGO BRIDGES
% Ego bridges are vat devices used for uploading and
% downloading minds. See Backups and Uploading, p.
% 268, and Resleeving, p. 271. [Expensive]

% NANOSWARMS AND MICROSWARMS
% Swarms are colonies of nanobots or larger microbots
% created in a hive, programmed with specific instruc-
% tions, and then set free to perform a set task. Each
% swarm is composed of hundreds or thousands of
% nanobots or microbots, ranging in size from a mi-
% crobe to a small insect. Nanobots are typically invis-
% ible to the naked eye, though they can be detected
% with a nanodetector (p. 326) or nanoscopic vision (p.
% 311). Microbots are more noticeable but still quite
% small, usually the size of a grain of sand or a dust
% mote, or occasionally as big as a flea. Individual bots
% in a swarm are directed by nanocomputers, with be-
% havioral routines modeled on biological insect and
% animal swarms. These swarms stick together and
% work as a whole, communicating with nanoradios,
% nanolasers, or chemical cues, and sharing informa-
% tion between each bot in the swarm. Note that nano-
% swarms don’t invade inside living bodies (though they
% may attack externally)—internal nano is handled by
% nanoware (p. 308), nanodrugs (p. 321), and nanotox-
% ins (p. 324).

% Nanobots and microbots may be designed with
% all manner of miniaturized propulsion systems (see
% Mobility Systems, p. 310), with the exception of ionic
% drives. They are powered by tiny batteries or solar
% cells. Their tiny sensors are very effective at allow-
% ing them to identify materials and objects, and so to
% target discriminatingly. Nanobots or microbots could,
% for example, be programmed to ignore metal objects,
% certain types of plants, specific morphs, females, or
% specific individuals. Swarms may either be released
% directly from a hive or from pre-packaged program-
% mable canisters.

% Swarms must be programmed before they are re-
% leased. The programming first determines how long
% the swarm is active. This timeframe is open-ended,
% though most swarms deteriorate into ineffectiveness
% after 2 weeks unless they are replenished by a hive.
% The programming then sets what area the swarm is
% to occupy. This is also open to interpretation and
% can vary from “coat this person” to “spread out to a

% %%% txt/331.txt
% diameter of 20 meters” to “find the nearest chemical
% traces and track them to their source.” Finally, pro-
% gramming sets any other parameters for the swarm’s
% mission—for example, if it should ignore certain ma-
% terials, if it should send a report at a predetermined
% time, or if it should self-destruct into harmless dust
% when it has completed a certain task.

% Programming is generally handled as a Simple
% Success Test using Programming (Nanoswarm) skill.
% Failure simply infers that the programming is imper-
% fect, and so the swarm may not operate completely as
% planned. An actual Programming (Nanoswarm) Suc-
% cess Test is only called for if the swarm’s programming
% is substantially complex or if the character seeks to
% have the swarm act outside of its usual set functions.
% The bots in each swarm are specially equipped for the
% task they are designed for, however, so attempting
% to drastically repurpose a swarm may be difficult or
% pointless at the gamemaster’s discretion.

% Swarms may also be teleoperated, controlled, and/
% or (re)programmed once they are released, via radio
% or laser link.

% Swarms are treated as a whole. The standard swarm
% size is enough to cover a 10 x 10 x 10 meter cube,
% and this is the standard “unit” of swarm released by
% a canister or hive. Swarms may be larger, but they are
% treated as individual swarm units. Each swarm has
% a Durability of 50 and is immune to wounds. Most
% attacks against a swarm simply inflict 1 point of
% damage. Area-effect weapons, plasma rifles, and fire
% inflict 1d10 damage, plasma grenades do full damage.
% Swarms are not affected by vacuum.

% Cleaners: This nanoswarm cleans, polishes, and
% removes dirt and stains. It may be used on an area,
% specific objects, or people. Some facilities employ
% permanent cleaner swarms to keep their area spotless.
% Cleaners may also be programmed to remove specific
% toxins, chemicals, or other hazardous substances in
% order to decontaminate an area. Covert operatives
% and criminals sometimes use cleaners to eliminate
% any evidence they may have left at a scene usable for
% forensics purposes, such as blood, hair, or anything
% that could be DNA-typed. [Low]

% Disassemblers: Also known as smart corrosives,
% these nanobots break down any matter. Their ad-
% vantage over common acids is that not only are
% they able to break down any material by using
% energy to disrupt chemical bonds, but that they
% can be programmed to take apart certain compo-
% nents while ignoring others, leaving them intact.
% Disassemblers are a common weapon used against
% synthmorphs, eating away their components with-
% out having to worry about accidentally splashing
% biomorphs. Upon contact, these nanobots infl ict
% 1d10 ÷ 2 damage (round up) per Action Turn. Ac-
% cumulated damage counts as a wound when the
% Wound Threshold is reached. Both Energy and Ki-
% netic armor protect against this damage, but these
% armors are eaten away as well, so the Armor Value
% is reduced by the soaked DV. [High]

% Engineers: Engineer microswarms are used for
% various construction purposes: erecting walls, digging
% tunnels, sealing holes, reinforcing foundations, and so
% on. [Moderate]

% Fixers: This is the nanoswarm version of repair
% spray (p. 333). [Moderate]

% Injectors: Injector microswarms are equipped with
% tiny needles and a drug payload. A biological target
% affected by an injector swarm suffers 1 point of
% damage and the effects of the carried drug, chemical,
% or toxin. [Moderate]

% Gardeners: This microswarm is useful for a number
% of agricultural purposes: killing weeds, planting seeds,
% trimming plants, pollinating, and even harvesting
% small items. It may also be programmed to simply
% defoliate an area. [Moderate]

% Guardians: Guardians watch for and attack other
% unauthorized swarms. Guardians inflict 1d10 ÷ 2
% damage (round up) on other swarms they come into
% contact with per Action Turn. [Moderate]

% Proteans: This nanoswarm is designed to disassem-
% ble other materials and objects and to create a single
% specific, pre-programmed device from the components
% (much like a specialized nanofabricator). The proteans
% must be able to scavenge appropriate raw materials
% (for example, to create a metallic device the nanobots
% must transform something else made of metal). The
% construction time takes 1 hour per cost category of
% the item (1 hour for a Trivial cost item, 2 hours for
% Low, etc.). [High]

% Saboteurs: Sab nanobots are designed to infiltrate
% electronics or machinery and sabotage them in small
% but difficult to discern ways: severing connections,
% disabling components, gumming up moving parts,
% etc. Saboteurs inflict damage on devices similar to
% disassemblers, but the target is not destroyed and
% such damage is not immediately obvious. They inflict
% 1d10 ÷ 2 points of damage to synthmorphs, bots, and
% other devices every Action Turn. Armor has no effect,
% but accumulated damage counts as a wound when the
% Wound Threshold is reached. [High]

% Scouts: A scout nanoswarm will system-
% atically map and explore an area, collecting
% samples of all materials and substances it
% encounters. The samples are carried back
% to the hive or canister and chemically ana-
% lyzed. Scouts can also be used for forensic
% purposes, collecting DNA samples, analyz-
% ing chemical residues, and examining other
% evidence. [High]

% Taggants: Taggants seek to lodge themselves
% onto everything in their area of dispersal.
% Each carries a unique identifier, so that if it is
% found later, the tagged person or object can
% be linked back to the point they were tagged.
% Taggants can be programmed to remain silent,
% only responding to query broadcasts made
% with the proper crypto codes, or they can be
% programmed to broadcast their location back
% to the deployer via the mesh. [Low]

% %%% txt/332.txt


%                                                      P
% CREATURE       COG COO INT REF SAV SOM WIL INIT SPD D
% Fur Coat        1   1    1    5    1   5    1    12   1     1

% Smart Dog       5   10   15   15   5   15   10   60   1     2


% Smart Monkey    5   15   15   15   5   10   10   60   1     2


% Smart Rat       5   15   15   15   5   5    10   60   2

% Space Roach     1   10   10   15   5   5    5    50   1



%  PETS
%  These partially-uplifted and bio-engineered animals
%  have rudimentary intelligence and limited communica-
%  tion skills. They make for fine companions and helpers.

%  Fur Coat: A so-called “fur coat” is outerwear made
%  from a living primitive organism. The creature’s skin,
%  fur, or scales are real. The organism is cultivated from
%  transgenic stocks and grown around molds into cloth-
%  ing shapes, often with actual usefulness: polar bear
%  parkas, seal diving suits, porcupine coats, etc. Fur
%  coats are modified with wireless controls and haptic
%  systems, so they can be made to move, shiver, massage,
%  or prickle up on command. [Low]

%  Smart Dogs: Commonly used as discriminatory
%  guardians, smart dogs are sometimes enhanced with
%  combative bioware or cybernetics. [Moderate]

%  Smart Monkey: Commonly used by criminal groups
%  for minor larceny such as pickpocketing, smart mon-
%  keys can be useful and intelligent aides. [Moderate]

%  Smart Rats: These upgrades of the common Norwe-
%  gian rat are clever and dexterous, and they easily fit
%  into a pocket or hood. [Low]

%  Space Roach: Grown to the size of a small dog,
%  these insects are often biosculpted for bright colors
%  and patterns. They are useful for minor janitorial
%  duties. [Low]

%  SCAVENGER TECH
%  This technology is often employed by gatecrashers,
%  space scavengers, and Firewall teams during missions.

%  Disassembly Tools: These tools are useful for sal-
%  vage ops, breaking down wrecks, or dissembling any-
%  thing from a habitat room to a vehicle or synthmorph.
%  They include plasma torches, laser cutters, pneumatic
%  jaws, and smart tools like spanners and wrenches that
%  can be adopted to a wide array of connections and
%  fittings. [High]

%  Mobile Lab: The mobile lab is a handheld device
%  that contains all different types of sensors to investi-
%  gate organic and inorganic liquid, gaseous, and solid
%  components (from soil to tissue samples) and compo-
%  sitions. It performs material analysis using different
%  methods of spectrometry and biochemical testing,
%  comparing results to a built-in database of element
% S
% WT LUC TT SKILLS
% 3     2    1 —


%           Fray 30, Freerunning 30, Intimidation 30, Percep-
% 5     20   4


%           tion 30, Scrounging 30, Unarmed Combat 40


%         Climbing 50, Fray 30, Freerunning 30, Infiltra-
% 4     20   4 tion 30, Perception 30, Scrounging 30, Unarmed


%         Combat 30


%           Climbing 40, Fray 40, Freerunning 30, Infiltration
% 1     20   4


%           50, Perception 20, Scrounging 50


%           Fray 30, Free Fall 30, Infiltration 50, Perception 20,
% 1     10   2


%           Scrounging 50



%  and compound spectra. Its built in AI comes equipped

%  with Academic: Chemistry 30. [Moderate]


%   Specimen Container: This capsule container is

%  designed to hold samples of any sort (chemical, bio-

%  logical, etc.) in near stasis. It can be programmed to

%  reproduce whatever conditions the user specifies, from

%  cryogenic freezing to extreme heat, or even vacuum or

%  high-pressure atmosphere. [Low]


%   Superthermite Charges: These powerful and highly

%  stable demolition charges are made from a combina-

%  tion of nanometals and metal oxides. A single charge

%  can be used to create an explosive blast inflicting

%  2d10+5 damage. This charge can be shaped with a

%  successful Demolitions Test, focusing the blast in a

%  particular 90-degree direction (for example, to blow

%  through a door). This triples the damage of the blast

%  in the focused direction; in all other directions, the

%  damage is reduced to 1/3rd (round down). Multiple

%  charges apply a cumulative effect. [Moderate]


%  SERVICES

%  Anonymous Accounts: These accounts are crucial

%  for anyone who wants to be discreet with their

%  online transactions. See Anonymous Account Ser-

%  vices, p. 252. [Moderate]


%   Backup: A single, one-time backup without insur-

%  ance is sometimes all the poor can afford, hoping

%  that they can buy backup insurance later or that

%  someone that cared about them will see to a re-

%  sleeving. [Moderate]


%   Backup Insurance: In the event of verifiable

%  death, or after a set period of being missing, backup

%  insurance will arrange for your cortical stack to be

%  retrieved and your ego downloaded into another

%  morph. If the cortical stack cannot be retrieved, your

%  most recent backup is used. Most policies require

%  that the holder provide a backup to be uploaded

%  into secure storage at least twice a year. This industry

%  works in a manner similar to insurance underwriting

%  in terms of cost and individuals engaged in high risk

%  professions can expect to pay a premium for the ser-

%  vice. Additionally, attempts to retrieve a cortical stack

%  are minimal unless one wants to pay for some extra

%  effort (a thriving industry of paramilitary ego-repo

% %%% txt/333.txt
% operatives exists for this purpose). [Low to Moderate
% per month]

% Body Bank: People who are egocasting to another
% station but whom hope to download back into the
% same body they have before when they return may
% put the morph on ice for the duration of their absence.
% [Moderate per month]

% Bot/Pod Rental: When you need a helping hand or
% a personal companion for a day or two, renting a bot
% or pod is often the way to go. [Moderate per day]

% Egocasting: This is the use of a farcaster to trans-
% mit an ego/infomorph. Farcasting is not cheap, and
% the cost is impacted by factors such as distance to
% receiver station and priority service (paying extra to
% get bumped ahead in line). [Expensive]

% Fake Ego ID: This forged ID will pass in most inner
% system and Jovian Republic habitats, and sometimes
% others. [High]

% Morph Brokerage: Acquiring a new morph is not
% always easy and is affected by factors such as the type
% of morph, sought-after enhancements/customizations,
% and local availability. Numerous brokerage services
% exist to find you what you need, or close to it. With
% enough lead-time, it may be possible to grow a pod
% that closely imitates your morph of choice. A willing-
% ness to accept used/traded-in morphs helps to reduce
% costs. For more details, see Morph Brokerage, p. 276.

% Psychosurgery: A character can purchase time in an
% immersive high-fidelity simulspace with expert care
% from psychosurgeons and AIs in order to cope with
% derangements and disorders that build up as a result
% of existing in a transhuman universe. For an addi-
% tional price the procedure can be time shifted to speed
% up the relative time within the simulspace. For more
% details, see Mental Healing and Psychotherapy, p. 215,
% and Psychosurgery, p. 229. [Moderate per month]

% Simulspace Subscription: This will by you access to
% the simulspace of your choice, whether you want it for
% a private meeting/vacation or to play the latest and
% hottest VR game. [Low (single use/1 day) to Moderate
% (monthly subscription)]

% Space Travel: Space transport cost depends on a
% number of factors like distance, quality of lodgings,
% and how much cargo you’re bringing with. At the low
% end, an intra-habitat shuttle trip within the same clus-
% ter, or a trip to or from a planetary body’s surface and
% orbit, is not cheap but affordable [High]. Just about
% anything else is progressively more costly. [Expensive]
% SOFTWARE
% For information on using software, see the Mesh chap-
% ter, p. 234.

% PROGRAMS
% These programs can be run on any computerized device.

% AR Illusions: These databases of AR clips can be used
% to create realistic illusions in someone’s entoptic display.
% See Augmented Reality Illusions, p. 259. [Moderate]

% Exploit: Exploits are hacker tools that take advan-
% tage of known vulnerabilities in other software. They
% are required for intrusion attempts (p. 254). [High]

% Facial/Image Recognition: This program can be
% used to take an image and run a pattern-matching
% search among public archives. Similar version of this
% program exist for other biometrics: gait recognition,
% vocal recognition, etc. [Low]

% Firewall: This program protects a device from
% hostile intrusion. Every system comes with a standard
% version of this software by default. [Low]

% Sniffer: Sniffer programs collect all of the transmis-
% sion that pass to, from, or through the device they are
% running on. See Sniffing, p. 252. [Moderate]

% Spoof: Spoof is a hacker tool used to fake com-
% mands and transmissions, making them seem as if
% they came from another source. See Spoofing Authen-
% tication, p. 255. [Moderate]

% Tactical Networks: These programs allow people in
% the same squad to share tactical data in real-time. See
% Tactical Networks, p. 205. [Moderate]

% Tracking: This software is used to track people by
% their presence online. See Scanning, Tracking, and
% Monitoring, p. 251. [Moderate]

% XP: Experience playback recordings are clips of
% someone else’s experiences. Depending on the content,
% some XP (porn, snuff, crime, etc.) may be restricted in
% certain jurisdictions. Some XP clips are intentionally
% modified so that their emotive tracks are more intense,
% giving the viewer a greater thrill. [Low to High]

% AIS AND MUSES
% Every character starts with a personal muse for free.
% Many devices also come with pre-installed AIs, ca-
% pable of helping the user, responding to commands, or
% even operating the device on their own. Rules for AIs
% can be found on p. 264.

% Below are some commonly available AI programs.
% Unless otherwise noted, these AIs have aptitudes of 10.
% These AIs may also be equipped with skillsofts (p. 332).

% Bot/Vehicle AI: These AIs are designed to be capa-
% ble of piloting the robot/vehicle without transhuman
% assistance. REF 20. Skills: Hardware: Electronics 20,
% Infosec 20, Interests: [Bot/Vehicle] Specs 80, Interface
% 40, Research 20, Perception 40, Pilot: [appropriate
% field] 40. [High]

% Device AI: These AIs are designed to operate a par-
% ticular device without transhuman assistance. Skills:
% Infosec 20, Interests: [Device] Specs 80, Interface 30
% (Device Specialization), Programming 20, Research 20,
% Perception 20. [Moderate]

% %%% txt/334.txt

% Kaos AI: Kaos AIs are used by hackers and covert
% ops teams to create distractions and sabotage systems.
% REF 20. Skills: Hardware: Electronics 40, Infosec 40,
% Interface 40, Professional: Security System 80, Pro-
% gramming 40, Research 20, Perception 30 plus one
% weapon skill at 40. [Expensive]

% Security AI: Security AIs provide overwatch for
% electronic systems. Skills: Hardware: Electronics 30,
% Infosec 40, Interface 40, Professional: Security Sys-
% tems 80, Programming 40, Research 20, Perception
% 30, plus one weapon skill at 40. [High]

% Standard Muse: Muses are digital entities that have
% been designed as personal assistants and lifelong com-
% panions for transhumans (see AIs and Muses, p. 264).
% INT 20. Skills: Academics: Psychology 60, Hardware:
% Electronics 30, Infosec 30, Interface 40, Professional:
% Accounting 60, Programming 20, Research 30, Per-
% ception 30, plus three other Knowledge skills at 40.
% [High]

% SCORCHERS
% Scorchers are damaging neurofeedback programs used
% to torment hacked cyberbrains (p. 261).

% Bedlam: Bedlam programs assault the ego with
% traumatic mental input, inflicting mental stress.
% Victims are overwhelmed with horrific, monstrous,
% sanity-ripping sensory and emotional input. Each
% attack inflicts 1d10 SV. [High]

% Cauterizer: This scorch program rips into the ego
% with destructive neurofeedback routines. Each attack
% with a cauterizer inflicts 1d10 + 5 DV on the target
% ego. This damage is reflected as digitized neurologi-
% cal damage. [High]

% Nightmare: Nightmare programs trigger anxiety
% and panic attacks within the victim by stimulating the
% neural circuitry representing the amygdala and hip-
% pocampus. The target ego must make a WIL x 2 Test.
% If they succeed, they are shaken but otherwise unaf-
% fected, suffering a –10 modifier to all actions until the
% end of the next Action Turn. If they fail, they suffers
% 1d10 ÷ 2 stress damage and are overcome with panic.
% This causes them either to blindly flee, have a nervous
% breakdown, or cower in frozen shock (gamemaster’s
% discretion). This panic episode lasts for 1 Action Turn
% per 10 points of MoF. [High]

% Shutter: Shutters target the victim’s sensory cortices,
% inflicting a –30 modifier to one chosen sense. Double
% this modifier if the attacking hacker scored an Excel-
% lent Success. This modifier reduces at the rate of 10
% points per Action Turn. [High]

% Spasm: Spasm programs are design to incapacitate
% the ego with excruciating pain. Affected targets must
% immediately make a WIL x 2 Test. If they fail, they
% immediately convulse, are disabled, and writhe in
% agony for 1 Action Turns per 10 full points of MoF.
% If they succeed, they still suffer a –30 modifier to all
% actions, which reduces at the rate of 10 points per
% Action Turn. Due to the nature of the delivery, pain
% tolerance of any sort has no effect. [High]
% SKILLSOFTS
% Skillsofts are used with skillware implants (p. 309).

% Standard Skillsoft: These programs provide the
% character with a rating of up to 40 in a single Active
% skill. [High]

% SURVIVAL GEAR
% The following gear is often critical to the survival of
%  soldiers, spies, criminals, gatecrashers, emergency ser-
%  vice personnel, and others who regularly venture into
%  unsafe or unfamiliar regions.

%  Breadcrumb Positioning System: This worn device
%  leaves micro “breadcrumbs” behind as the character
%  moves. These devices interact with mesh inserts (or
%  ectos) as long as they are within range (50 meters),
%  allowing the user to map their position in relation to
%  the breadcrumb trail. This is useful in derelict habitats,
%  wilderness, and other areas where there is no local
%  functioning mesh, and is helpful both for mapping
%  and for finding one’s way back. [Low]

%  Electrogravitics Net: Also called a safety net, this fail-
%  safe system uses electric fields to counter gravity when
%  falling. While the system is not able to actually levitate
%  heavy objects, it will slow down a fall enough that the
%  user can land safely if the gravitational force is not too
%  high (the fall height is not greater than 50 meters in
% 1G). Generating these electric fields consumes a lot of
%  energy, so the net is only charged for one use only and
%  needs to be recharged afterwards. [Moderate]

%  Electronic Rope: The
%  fibers in this rope can be
%  controlled electronically,
%  making it move in a snake-
%  like fashion, stiffen up, and
%  even wrap around objects.
% Typically comes in a 50-
%  meter length capable of
%  supporting 250 kg. [Low]

%  Emergency Bubble: Com-
%  monly used as a last resort
% “life raft” on spaceships, an
%  emergency bubble is made
%  of advanced smart materi-
%  als and comes in a por-
%  table package that can be
%  quickly inflated (1 Action
% Turn) around the user, usu-
%  ally inside an airlock. The
%  bubble has a 5-meter diameter and can comfortably
%  accommodate 4 people. It maintains 1 atmosphere of
%  pressure in a vacuum, protect the inhabitants from
%  temperatures ranging from –175 to 140 C, and pro-
%  vide light, breathable air and water and food recycling
%  for up to four human-sized inhabitants, using its built
%  in maker (p. 327). It features a simple airlock, carries
%  an emergency distress beacon (below), and can be
%  transparent, opaque, or polarized. It is powered by a
%  small nuclear battery and also includes comfortable
%  inflatable furniture. [Moderate]

% %%% txt/335.txt

% Emergency Distress Beacon: This small but power-
% ful transmitter is powered by a nuclear battery and
% will broadcast any programmed distress call for years.
% Though portable and medium-sized, this beacon has
% a range of 500 km in urban areas and 5,000 km else-
% where. [Moderate]

% Flashlight: These handheld, wearable, or portable
% lights can display light in the normal visual spectrum,
% infrared, or ultraviolet, as desired. [Trivial]

% Nanobandage: Characters without medichines must
% rely on external sources of healing. The most common
% option is the nanobandage—a plum-sized advanced
% nanotechnology generator built into a reusable, self-
% sterilizing bandage. It can treat all forms of injury and
% illness, from poisoning to burns to trauma. Characters
% simply apply the bandage to the wound and let the
% nanobots do the work. It removes pain and discom-
% fort and speeds healing (see Biomorph Healing, p.
% 208). For especially severe injuries, physical first aid
% such as setting bones and removing projectiles may
% be necessary (gamemaster’s choice). If the wounds are
% too severe (the patient has suffered more than five
% wounds), the unit places the patient in medical stasis
% and radios for emergency services. [Trivial]

% Repair Spray: This nanobot generator creates
% nanobots designed to repair synthmorphs, vehicles,
% and other common objects. Repair spray contains
% the specifications and plans for almost all commonly
% used synthmorphs and devices and is a ubiquitous
% household item. If it does not contain the specifica-
% tions for something it is being used to repair, it must
% query the object’s voice for these details, otherwise it
% cannot repair it. Simply touch it to the damaged area,
% push the button on top, and it sprays out a number of
% nanobots sufficient to make repairs. These nanobots
% repair 1d10 points of damage per 2 hours. Once all
% damage is restored, the nanobots repair wounds at
% the rate of 1 per day. Repair spray also cleans and
% polishes items and returns them to a pristine and
% new state. Repair spray is not effective on any object
% with more than 3 wounds, but it provides a +30 to
% all repair rolls on anything too badly damaged for it
% to fully repair (see Synthmorph and Object Repair, p.
% 208). [Low]

% Shelter Dome: A variant of the emergency bubble,
% this package unfolds into a dome with a 2.5-meter
% ceiling and a floor 4 meters across. To safely use this
% shelter, it must be staked down to the surface it is
% placed on. [Moderate]

% Spindle: A spindle is an advanced nanotechnology
% generator that produces a super-strong cable. It can
% produce up to 2 kilometers of 0.2 millimeter diameter
% line than can support up to 250 kilograms before it
% needs more raw materials. The spindle can produce
% up to 20 meters of cable every second. It can produce
% line in a continuous length or cut the cable it produces
% to any length. Spindles can also reabsorb their cable,
% retracting it at a rate of 5 m per second. As long as
% it is recharged and has small amounts of additional
% material added every 1,000 hours of use, a spindle
% can keep producing and retracting cable indefinitely.
% By setting the maximum production speed at 10 m/
% second a character with a spindle can safely jump
% off a building and land safely, using the cable to slow
% their descent. [Moderate]

% Spindle Climber: This device attaches to a spindle
% and transforms it into a highly effective climbing
% device. The spindle climber has two functions. First, it
% attaches hardened tips to the spindle’s cable and fires
% it at high speed, up to 50 meters, with sufficient force
% to imbed the tip into almost any sufficiently durable
% surface. Second, the spindle climber can pull itself and
% up to 250 kg up the cable at a speed of up to 2 m/sec.
% A spindle climber has enough power to shoot and pull
% up the cable 50 times before it must be recharged. A
% spindle fits inside a spindle climber. [Low]

% VACUUM SUITS
% Most vacuum suits are skin-tight garments that use
% the pressure of their advanced smartfabrics on the
% wearer’s body to resist vacuum. When the wearer is in
% a breathable atmosphere, the smartfabric also loosens
% the suits to serve as ordinary clothing or be easily
% put on or taken off. In all cases, the suits can become
% skin-tight within 3 Action Turns. All vacsuits contain
% advanced rebreather units capable of maintaining a
% breathable atmosphere for several hours or days.

% Light Vacsuit: Everyone living in a sealed habitat
% owns at least one of these suits. They come in a vari-
% ety of forms. Inexpensive versions are typically light-
% weight jumpsuits made of simple smart fabric that
% adjusts to fit and folds up small enough to fit into a
% coat pocket. The best models include suits of high-end
% smart clothing that can transform into a vacsuit and
% an advanced nanotech generator the size of a large
% orange that deploy nanobots that cover the user and
% fit together into a vacuum suit. Both can transform
% into a vacsuit in 2 full Action Turns and do so either
% on command or if their sensors reveal that life sup-
% port is needed.

% All models include a lightweight belt or torc
% containing a miniature oxygen tank and advanced
% rebreather unit that provides 3 hours of air. How-
% ever, the suits contain no food or water recycling. All
% models include an ecto (p. 325) and a headlight, but
% typically little else beyond atmosphere sensors to let
% the wearer know when it is safe to take off the suit.
% They protect the wearer from temperatures from –75
% to 100 C. These vacuum suits also provide an Armor
% rating of 5/5 and instantly self-seal breaches unless
% more than 20 points of damage are inflicted at once.
% [Low, Moderate for smartfabric suits]

% Standard Vacsuit: These suits resemble light vac-
% suits made from thicker and more durable materials
% that resist tearing and provides the wearer with light
% armor. They are fitted with more substantial life sup-
% port belts that includes a maker (p. 327) capable of
% recycling all wastes and producing air for up to 48
% hours and food and water indefinitely. The best suits
% are made of smart materials that can transform from

% %%% txt/336.txt
% standard clothing to vacuum suits in a single Action
% Turn, and will do so automatically if life support is
% needed. Each suit also contains an ecto (p. 325), a
% radio booster (p. 313), and sensors equal to specs (see
% p. 325). These suits have an Armor rating of 7/7 and
% protect the wearer from temperatures from –175 to
% 140 C. They can almost instantly seal any hole unless
% more than 30 points of damage are inflicted at once.
% [Moderate, High for smartfabric suits]

% Hard Suit: This heavy-duty suit can almost be
% considered a miniature space ship. Hard Suits look
% like large metallic ovals with jointed arms and legs.
% They are quite heavy, but the user can move relatively
% easily by using servo assist motors in all the major
% joints of the arms and legs. Unlike other vacsuits, they
% are solid and can resist both vacuum and up to 100
% atmospheres of external pressure. Characters wearing
% hard suits can safely explore the upper atmosphere
% of a gas giant. They are well armored against punc-
% tures and radiation and possess miniature plasma
% thrusters capable of delivering 0.01G for 10 hours. A
% built-in high quality maker produces sufficient food,
% air, and water that a user can remain in a hard suit
% indefinitely. Explorers have used them continuously
% for up to 2 months. Their gloves incorporate smart
% materials that allow each hand to use the equivalent
% of a utilitool (p. 326). Hard suits also contain radios
% and sensors equivalent to those on standard vacsuits.
% These suits have an Armor rating of 15/15, are main-
% tained by a fixer nanohive (p. 329), and are instantly
% self-sealing of any breach unless more than 30 points
% of damage are inflicted at once. They protect the
% wearer from temperatures of –200 to 180 C. [High]



% WEAPONS
% A wide range of weapons are available in Eclipse
% Phase, from the primitive to the technologically
% advanced.

% MELEE WEAPONS
% Melee weapons are those wielded by hand (or foot) in
% melee combat. They are divided by the skill be which
% they are used.

% BLADES
% These weapons are wielded with Blades skill.

% Diamond Axe: Commonly found on many habitats
% for fire and emergency purposes, axes require two
% hands to wield. Their blades are diamond-coated for
% superior cutting ability. [Low]

% Flex Cutter: The blade of this machete-like weapon
% is made of a memory polymer. When deactivated, the
% blade is limp and flexible, and may even be rolled up
% or otherwise easily concealed. When activated, how-
% ever, the blade stiffens and sharpens into a vicious
% slashing weapon. [Low]

% Knife: A standard cutting implement, still carried
% by many. [Trivial]

% Monofilament Sword: Though swords are rather
% archaic in the time of Eclipse Phase, a few eccentrics
% take advantage of modern versions with a self-
% sharpening near-monomolecular edge, easily capable
% of slicing through metal or limbs. [Low]

% Vibroblade: These buzzing electronic blades vibrate
% at a high frequency for extra cutting ability. This
% has little extra effect when stabbing or slashing, but
% provides an extra –3 AP and +2d10 damage when
% carefully sawing through something. [Low]

% Wasp Knife: Wasp knives are equipped with a can-
% ister in their handle. The common use is to fill these
% canisters with pressured air, which inflates inside
% the target. This is potentially lethal in vacuum or
% pressurized environments (like underwater), as the
% gas bursts out of the body cavity to escape (+2d10
% damage in such situations). Wasp knives may also
% be loaded with chemicals, drugs, or nanobots. The
% target must be damaged for the canister’s contents to
% affect them. [Low]

% CLUBS
% Characters use Clubs skill when using these weapons.

% Club: Clubs encompasses a wide range of one-hand-
% ed blunt objects, from saps to sticks to pipes. [Trivial]

% Extendable Baton: This hardened composite baton
% retracts into its handle for easy carrying, storage, or
% concealment. Extending it simply requires a flick or
% an electronic signal. [Trivial]

% Shock Baton: Shock batons are standard clubs
% used for policing duties, but when activated they also
% deliver an electric shock to struck targets (see Shock
% Attacks, p. 204). [Low]

% EXOTIC MELEE WEAPONS
% Unusual weapons requires a specific Exotic Melee
% field skill to use.

% Monowire Garrote: This assassin’s weapon fea-
% tures a dangerous monomolecular wire wrapped
% around a contained spool with two handles. One
% handle grips the spool, while the other extends the
% wire so that it may be used to wrap around targets
% (typically necks or limbs) and slice through them
% when pulled. Monofilament tensile strength is weak,
% however, usually breaking after one use. [Moderate]

% UNARMED
% These weapons are wielded using Unarmed Combat skill.

% Densiplast Gloves: These gloves extra-harden when
% activated, for extra punch. [Trivial]

% Shock Gloves: When activated, these gloves deliver
% an incapacitating shock along with every punch or
% grab. Note that the effect is the same whether wearing
% one glove or two. [Low]

% %%% txt/337.txt
%  KINETIC WEAPONS
%  Kinetic weapons damage the target by firing a hard
%  impact projectile at high-velocities. Slugthrowers
%  have evolved from the mechanical firearms of the
%  early 21st century, however, and now fall into two
%  categories: chemical firearms and railguns. Though
%  their mechanisms for firing are different, they are
%  roughly similar in effect. Railguns have a higher
%  penetration and inflict more damage, which is offset
%  by more limited ammunition choices. While modern
%  beam weapons have their uses, they rarely match the
%  punch of kinetic weapons, therefore slugthrowers
%  are still perceived as the most versatile and effective
%  weapon system.

% Kinetic weapons are constructed from lightweight,
%  reinforced plastoceramic materials, which are easily
%  produced even without nanofabrication. By default,
%  modern kinetic weapons are ambidextrous but more
%  importantly feature safety and smartlink systems (p.
%  342) that automatically connect to the wielder’s mesh
%  inserts for firing assistance, target recognition, and
%  tactical networking.

% The wielder of a firearm or railgun uses Kinetic
%  Weapons skill. For information on firing modes, see
%  p. 198. For different ammunition types, see p. 336.
%  Ranges are listed on p. 203.

%  FIREARMS
%  Modern chemical firearms use caseless ammuni-
%  tion that is auto-loaded from a magazine. They are




%          MELEE WEAPONS—BLADE
% BLADES                      ARMOR PENETRATION (AP)
% Diamond Ax                              –3
% Flex Cutter                             –1
% Knife                                   –1
% Monofilament Sword                       –4
% Vibroblade                              –2
% Wasp Knife                              –1


% CLUBS                       ARMOR PENETRATION (AP)
% Club                                    —
% Extendable Baton                        —
% Shock Baton                             —                  1


% EXOTIC MELEE WEAPONS        ARMOR PENETRATION (AP)
% Monowire Garrote                        –8


% UNARMED                     ARMOR PENETRATION (AP)
% Bioware Claws (p. 304)                  –1
% Cyberclaws (p. 307)                     –2
% Densiplast Gloves                       —
% Eelware (p. 304)                        —
% Shock Gloves                            —
% Unarmed                                 —

% effectively recoilless (thanks to rheological smart

% fluid mechanisms) and electronically fired (an electric

% charge vaporizes the propellant, using the expanding

% steam and plasma to eject and accelerate the projectile).


%  Note that older, pre-Fall firearms still exist and are

% traded by black marketeers, though they use outdated

% system such as liquid propellants or cased ammuni-

% tion. At the gamemaster’s discretion, these relics may

% suffer shorter ranges, less penetration, fewer firing

% modes, or reduced damage.


%  Pistols: Pistols are small-sized (p. 297) and de-

% signed for one-hand use. Light pistols sacrifice pene-

% trating ability for concealability. Heavy pistols focus

% on stopping power, with medium pistols occupying

% a middle ground. All versions fire in semi-automatic,

% burst-fire, and full-auto modes. [Low]


%  Submachine Guns: SMGs use pistol ammuni-

% tion, but are medium-sized (p. 297) and may fire in

% semi-auto, burst fire, or full auto modes. They typi-

% cally are designed in a bullpup configuration for

% close quarters operations and are ideal for tactical

% and strike teams. [Moderate]


%  Automatic Rifles: Automatic rifles use rifle ammu-

% nition and have greater range and penetration than

% SMGs. They fire in semi-auto, burst fire, or full auto

% modes. They are two-handed weapons. [Moderate]


%  Sniper Rifle: Sniper rifles are optimized for range,

% accuracy, penetration, and stopping power. They fire

% in semi-auto, burst fire, or full auto modes, and are

% two-handed weapons. [High]


% CLUBS, EXOTIC, UNARMED

% DAMAGE VALUE (DV)                     AVERAGE DV

% 2d10 + 3 + (SOM ÷ 10)                14 + (SOM ÷ 10)

% 1d10 + 3 + (SOM ÷ 10)                8 + (SOM ÷ 10)

% 1d10 + 2 + (SOM ÷ 10)                7 + (SOM ÷ 10)

% 2d10 + 2 + (SOM ÷ 10)                13 + (SOM ÷ 10)


% 2d10 + (SOM ÷ 10)                  11 + (SOM ÷ 10)

% 1d10 + 2 + (SOM ÷ 10)                7 + (SOM ÷ 10)



% DAMAGE VALUE (DV)                     AVERAGE DV

% 1d10 + 2 + (SOM ÷ 10)                7 + (SOM ÷ 10)

% 1d10 + 2 + (SOM ÷ 10)                7 + (SOM ÷ 10)
%  + 2 + (SOM ÷ 10) + shock (p. 204)      7 + (SOM ÷ 10)



% DAMAGE VALUE (DV)                     AVERAGE DV


%        3d10                              16



% DAMAGE VALUE (DV)                     AVERAGE DV

% 1d10 + 1 + (SOM ÷ 10)                6 + (SOM ÷ 10)

% 1d10 + 3 + (SOM ÷ 10)                8 + (SOM ÷ 10)

% 1d10 + 2 + (SOM ÷ 10)                7 + (SOM ÷ 10)


%    shock (p. 204)                        —
% 10 + (SOM ÷ 10) + shock (p. 204)        5 + (SOM ÷ 10)


% 1d10 + (SOM ÷ 10)                  5 + (SOM ÷ 10)

% %%% txt/338.txt
% BRAND NAME WEAPONS
% AND COMBINED ARMS
% The weapons listed in this book define generic
% samples of each weapon. Gamemasters are en-
% couraged to offer brand name versions of each
% weapon, each with its particular idiosyncrasies
% and small variations. For example, a Direct
% Action A30 SMG might lack a semi-automatic
% setting but come equipped with an extra ammo
% capacity of 35. Likewise, a Medusan Arms Longi-
% nus sniper rifle may inflict an extra +2 damage
% but have an AP of only –12.

% Similarly, many of the weapons listed here are
% available as combined arms weapons systems. A
% police-issue assault rifle may also feature a stun-
% ner—all built into the same weapon. For com-
% bined arms, simply add together the individual
% weapon component costs.                         ■




% Machine Gun: Machine guns are heavy weapons,
%  typically mounted, and intended to provide continu-
%  ous fire for support or suppressive purposes. They fire
%  in burst fire or full auto modes, and are two-handed
%  weapons. [High]

%  RAILGUNS
%  Railguns use a pair of electromagnetic rails to slide
%  and accelerate a non-explosive conductive projectile at
%  extremely high velocities (Mach 6+) to create an over-
%  whelming, penetrating attack. The kinetic energy of




%                            KINETIC WEAP
% FIREARMS          ARMOR PENETRATION (AP)         DAMAGE
% Light Pistol                  —                            2
% Medium Pistol                 –2                       2d1
% Heavy Pistol                  –4                       2d1
% Submachine Gun                –2                       2d1
% Automatic Rifle                –6                       2d1
% Sniper Rifle                   –12                     2d1
% Machine Gun                   –4                       2d1





%                            KINETIC WEAP
% RAILGUNS          ARMOR PENETRATION (AP)         DAMAGE
% Light Pistol                  –3                       2d1
% Medium Pistol                 –5                       2d1
% Heavy Pistol                  –7                       2d1
% Submachine Gun                –5                       2d1
% Automatic Rifle                –9                       2d1
% Sniper Rifle                   –15                     2d1
% Machine Gun                   –7                       2d1


% the projectile exceeds that of an explosive-filled shell


% of greater mass and creates shock and heat waves


% upon impact that shatter and incinerate the target, or


% portions of it. While railguns are more potent than


% firearms, the ammunition choices are limited as the


% projectile must be conductive and able to survive


% both acceleration and heat created in the process due


% to friction. Nanofabrication allows railguns to be


% manufactured on the personal weapons scale while


% high-energy portable batteries provide the power to


% fire them. Railgun operation is silent except for the


% supersonic crack of the projectile.


%    Railguns are available in the same models as fire-


% arms (pistols through machine guns), with the follow-


% ing modifications:



%  • Increase AP by –3


%  • Increase damage by +2


%  • Increase the maximum for each range category


%    by x1.5


%  • Increase Cost category by one


%  • Railguns may only use regular and armor-pierc-


%    ing ammunition


%  • Railguns also require battery power for each shot.


%    Standard railgun batteries hold enough power for


%    200 shots, after which they must be recharged at


%    the rate of 20 points per hour.



% KINETIC AMMUNITION


% Ammunition is defined by its various types (standard,


% gel, APDS, etc.) and by the class of gun (light pistol,


% heavy pistol, SMG, etc.). For simplicity, each gun


% can trade ammunition with another gun of its class,


% though ammunition for firearms and railguns is not


% NS—FIREARMS
% UE (DV)       AVERAGE DV       FIRING MODES         AMMO


%               11              SA, BF, FA          10
% 2                  13              SA, BF, FA          12
% 4                  15              SA, BF, FA          16
% 3                  14              SA, BF, FA          20
% 6                  17              SA, BF, FA          30
% 10                 21              SA, BF, FA          40
% 6                  17                BF, FA            50



% NS—RAILGUNS
% UE (DV)       AVERAGE DV       FIRING MODES         AMMO
% 2                  13              SA, BF, FA          10
% 4                  15              SA, BF, FA          12
% 6                  17              SA, BF, FA          16
% 5                  16              SA, BF, FA          20
% 8                  19              SA, BF, FA          30
% 12                 23              SA, BF, FA          40
% 8                  19               BF, FA             50

% %%% txt/339.txt
% exchangeable. For example, all railgun SMGs can
% share ammo.

% The ammunition’s Damage Value and Armor Pen-
% etration modifiers are added to the weapon’s base
% DV and AP. With the exception of regular and armor-
% piercing rounds, none of this ammunition may be
% used with railguns. Listed costs are per 100 rounds
% of ammunition.

% Armor-Piercing: This tungsten-carbide ammunition
% penetrates armor effectively. [Low]

% Bug: Bug rounds are equipped with a microbug and
% medical sensor nanobots. They attempt to gather in-
% formation on the target’s location (via standard mesh
% tracking), health (querying the target’s medichines),
% and surroundings (typically hindered by being inside
% the target’s body). They will transmit status reports
% in a pre-programmed manner via the mesh or a pre-
% chosen frequency band either continuously or in pre-
% set intervals. [Low]

% Capsule: Capsule ammo carries a payload (drug,
% toxin, nanobots) that is released inside the target after
% the round penetrates. [Trivial plus payload cost]

% Flux: Flux ammo is made from rheological materi-
% als that allow each bullet to be “programmed” so that
% they may change from regular rounds to less-lethal soft
% plastic-like rounds. This allows the firer to choose the
% type of round (regular or plastic) made with each shot
% or burst, and then change with the next one. [Low]


%        KINETIC AMMUNITION
% AMMO              AP MODIFIER       DV MODIFIER
% Armor-Piercing        –5                  –2
% Bug                   +1                –1d10
% Capsule               +1                 –half
% Flux              as ammo type       as ammo type
% Hollow-Point          +2                +1d10
% Jammer                —               no damage
% Plastic           (AV doubled)           –half
% Reactive              –2                  +2
% Reactive Armor-


%                  –6                  –1
% Piercing
% Regular               —                   —
% Splash                —               no damage
% Zap                   +2         –half + shock (p. 204)
% SMART AMMO
% Accushot              —                   —
% Biter                 —                 +1d10
% Flayer                —                   +2
% Homing                —                   —
% Laser-Guided          —                   —
% Proximity             –1                  +2
% Zero                  —                   —

% %%% txt/340.txt

% Hollow-Point: Hollow-point bullets are designed to
% deform and widen once they penetrate a target, thus
% inflicting more damage. [Trivial]

% Jammer: Jammers stick to the target and pulse out
% jamming electromagnetic signals, jamming the target’s
% wireless communications. If an Opposed Test is called
% for, these devices have an Interface of 30. See Radio
% Jamming, p. 262. [Low]

% Plastic: Plastic ammo is designed to hurt but not
% wound targets, and is commonly used for crowd con-
% trol purposes. [Trivial]

% Reactive: The casing on these projectiles is made
% of reactive materials that release a large amount of
% energy when subjected to a sudden shock or impact—
% such as striking a target. In other words, they explode
% or superheat when they hit. [High]

% Reactive Armor-Piercing (RAP): This is a tung-
% sten-carbide armor-piercing round with a reactive
% casing, allowing the ammunition to penetrate even
% further. [Moderate]

% Regular Ammo: This standard metal projectile is
% designed to put holes into morphs. [Trivial]

% Splash: Splash rounds carry a payload like capsule
% ammo, but are designed to break upon impact rather
% than penetrating, splashing their contents on the
% target’s exterior. Splash rounds are typically loaded
% with paint, taggant nanobots, tracker dye, and similar
% substances. [Trivial plus payload cost]

% Zap: Zap rounds are rubber or gel bullets that
% create an electric charge upon firing in a piezoelectric
% like manner to stun the target effectively with both
% the bullet and the electric shock. [Trivial]



% SMART AMMO
% Smart ammunition takes advantage of nanotechnol-
% ogy to produce bullets that can alter their flight path,
% home in the target, and correct aim. Smart ammo may
% not be used with railguns. With the exception of biter,
% flayer, and proximity rounds, smart ammo may be
% combined with other ammo types (accushot armor-
% piercing, for example).

% Accushot: Accushot bullets change shape within
% flight to keep dead on course, countering the effects
% of wind, drag, and gravity over distance. Attacks
% made with accushot bullets ignore all range modi-
% fiers. [Low]

% Biter: Biters are specially-designed to fragment in
% opposite proportion to the hardness of the target they
% strike. For hard targets (synthmorphs), they frag-
% ment very little, blasting a big hole. For soft targets
% (biomorphs), they fragment and tumble in multiple
% directions within the body. [Low]

% Flayer: Flayers have nanosensors to detect an on-
% coming impact, shooting out monomolecular barbs
% as they are about to strike a target. These monowires
% cut through the target along with the bullet, inflicting
% additional damage. [Low]

% Homing: When fired with a smartlink system, the
% bullet identifies the target and uses nanosensors to
% lock on, correcting the bullet’s trajectory with surface
% alterations and tiny vectored nozzles. Apply a +10
% modifier to the Attack Test, cumulative with aiming
% and smartlink modifiers. Homing bullets may also be
% used for indirect fire (p. 195). [Low]

% Laser-Guided: These bullets function like homing
% smart rounds (apply the +10 attack modifier), except
% rather than requiring a smartlink system, they lock
% onto the reflection of the laser sight used to paint
% the target. Laser-guided bullets may also be used for
% indirect fire (p. 195). [Low]

% Proximity: Proximity is an explosive ammunition
% that identifies the target when fired via smartlink. If
% the round determines that it will miss the target, it will
% still explode if it reaches the close proximity of the
% target. If the attack misses with an MoF of 10 or less,
% the round explodes 1d10 meters away from the target
% and inflicts 1d10 area effect damage (see Blast Effect,
% p. 193) in the proximity of the target. [Moderate]

% Zero: Similar to homing smart rounds, zero bullets
% identify the target when fired via smartlink. Whether
% the round hits or misses, however, it sends telem-
% etry data back to the next zero bullet, allowing it to
% course-correct and “zero in” to hit the target (or hit
% more accurately). Apply a +10 modifier to each shot
% (or burst) fired after the first against the same target in
% the same Action Turn. [Low]

% BEAM WEAPONS
% Beam weapons is a broad category for a number of elec-
% tromagnetic weapons with a wide range of effects. With
% a few exceptions, energy weapons are primarily used for
% less-than-lethal purposes, designed to impair the target
% rather than kill it. Their poor performance against armor,
% lesser ability to damage targets, and high power require-
% ments make them less versatile than kinetic weapons.
% The wielders of such weapons use Beam Weapons skill.

% Beam weapons are powered by nuclear batteries.
% This battery is good for a list number of shots before
% it is depleted. Batteries may be recharged at the rate
% of 20 shots per hour.

% Laser Pulsers: Laser weapons use focused beams of
% light to inflict damage on the target by burning into it
% and causing it’s outer surface to vaporize and expand,
% creating an explosive effect. The laser beam is pulsed
% in order to bite into the target before the beam is dif-
% fused. Pulsers are vulnerable to atmospheric effects
% like dust, mist, smoke, or rain, however—the game-
% master should reduce their effective range categories
% as appropriate. Note that laser pulses are invisible in
% the normal visual spectrum (but are visible to charac-
% ters with enhanced vision). Pulsers are medium-sized
% (p. 297) and fire in semi-auto mode. [Moderate]

% One advantage to the pulser is that it can be placed
% in less-lethal mode. In this case, it first fires a pulse at
% the target to create a ball of plasma, quickly fired by
% a second pulse that strikes the plasma and creates a
% flash-bang shockwave to stun and disorient the target.
% This blast has an area of effect with a 1-meter radius.
% Anyone caught in the blast must make a SOM x 2

% %%% txt/341.txt
%  Test (SOM x 3 for synthmorphs or biomorphs with
%  any form of pain tolerance). Failure means the target
%  is temporarily stunned and disoriented and loses
%  their next action. A critical failure means the target
%  is knocked down and paralyzed for 1 Action Turn per
%  10 points of MoF. In this stun setting, the pulser fires
%  only in single-shot mode.

%  Microwave Agonizer: The agonizer fires millimeter-
%  wave beams that create an unpleasant burning sen-
%  sation in skin (even through armor) and to metals.
%  Agonizers have two settings. The first is an active
%  denial setting that causes extreme burning pain in the
%  target, inflicting –20 to the target’s actions and forc-
%  ing them to move away from the beam on their next
%  action unless they succeed in a WIL Test (targets with
%  Level 1 Pain Tolerance or the equivalent only suffer
%  a –10 modifier and roll WIL x 2). Synthetic morphs
%  and biomorphs with Level 2 Pain Tolerance (or the
%  equivalent) are immune to this weapon. The second
%  setting (colloquially known as the “roast” setting) has
%  the same effect of the first, but also actually burns the
%  target, inflicting the listed damage. Originally devel-
%  oped for crowd control, the agonizer is also useful for
%  repelling animals. The agonizer is small-sized (p. 297)
%  and fires in semi-auto mode. [Moderate]

%  Particle Beam Bolter: This weapon shoots a bolt of
%  accelerated particles at near light speed that transfer
%  massive amounts of kinetic energy to the target, su-
%  perheating and creating an explosion when striking.
%  The bolter’s beam is not diffused by the cloud that
%  occurs when it strikes, and so it has greater penetra-
%  tion than the laser pulser. Likewise, the bolter is not
%  affected by smoke, fog, or rain. The bolter’s beam is
%  invisible. Note that bolters are designed for either
%  atmospheric or exoatmospheric (vacuum) operation,
%  and will not function in the opposite environment
%  (though bulkier dual models, combining both models,
%  are also available). Bolters fire in semi-auto mode and
%  are rifle-sized two-handed weapons. [High]

%  Plasma Rifle: This bulky, heavy, two-handed
%  weapon blasts a stream of nova-hot plasma at the
%  target, inflicting severe burns and thermal damage,
%  possibly melting or evaporating the target entirely.
%  Plasma rifles are perhaps the deadliest man-portable
%  weapons in use. Plasma guns suffer from dangerous





%                                            BEAM
% BEAM WEAPONS                     ARMOR PENETRATION (AP)      D
% Cybernetic Hand Laser (p. 308)             —
% Laser Pulser                               —

% Stun Mode                               —
% Microwave Agonizer                         —

% Roast Mode                              –5
% Particle Beam Bolter                       –2
% Plasma Rifle                                –8
% Stunner                                    —              (1d

%  overheating, however, and so require 1 full Action

%  Turn of cool-down time after every 2 shots. Plasma

%  rifles only fire in single shot mode. [Expensive]


%   Stunner: The stunner is an electrolaser that cre-

%  ates an electrically-conductive plasma channel to the

%  target, down which it transmits a powerful electric

%  current, shocking the target. Stunners do not work in

%  vacuum. Stunners fire in semi-auto mode. [Moderate]


%  SEEKERS

%  Seekers are a combination of automatic grenade

%  launcher, micromissile, coilgun, and smart munitions

%  technology. Unlike traditional launchers of the past,

%  miniaturization allows the manufacture of seeker mi-

%  cromissile launchers in personal weapon sizes. Seeker

%  rounds are fired at high-velocity via rings of magnetic

%  coils, after which the micromissile or minimissile uses

%  scramjet technology to propel itself and maintain high

%  velocities over great distances. Seekers are wielded

%  using Seeker Weapon skill.


%   Seeker missiles are detailed on p. 340. Like gre-

%  nades, seekers may be programmed for a variety of

%  trigger events (see Grenades and Seekers, p. 199). All

%  seeker weapons are smartlink-equipped (p. 342).


%   Disposable Launcher (Standard Missile): This

%  launcher is pre-packed with one standard missile.

%  [Moderate (includes missile)]


%   Seeker Armband (Micromissile): This weapons unit

%  is worn on the arm, allowing the user to point and

%  fire using an entoptic smartlink system. Though highly

%  portable, the armband’s micromissile supply is low. It

%  fires in single-shot mode. [Moderate]


%   Seeker Pistol (Micromissile): This pistol-sized

%  seeker launcher fires micromissiles in semi-auto

%  mode. [Moderate]




%             SEEKER WEAPONS
%  EEKER WEAPONS           FIRING MODES             AMMO
% Disposable Launcher              SS                  1
%  eeker Armband                   SS                  4
%  eeker Pistol                    SA                  8
%  eeker Rifle                      SA     12 micromissile/6 minimissi
% Underbarrel Seeker               SA                  6



% EAPONS
% AGE VALUE (DV)          AVERAGE DV FIRING MODES           AMMO

%  2d10                    11             SA                50

%  2d10                    11             SA               100

%  1d10                    5               SS               —
%  (see description)          —              SA               100

%  2d10                    11             SA                50

% 2d10 + 4                  15             SA                50
%  3d10 + 12                  28              SS               10
% ÷ 2) + shock (p. 204)       —              SA               200

% %%% txt/342.txt

%  Seeker Rifle (Micromissile/Minimissile): The seeker
%  rifle comes in a bullpup configuration and fires either
%  micromissiles or minimissiles in semi-auto mode. It is
%  a two-handed weapons. [High]

%  Underbarrel Seeker (Micromissile): This seeker
%  micromissile launcher is commonly attached to the
%  underbarrel of SMGs or assault rifles. It fires in semi-
%  auto mode. [Moderate]

%  SPRAY WEAPONS
%  Spray weapons blast their ammunition outwards in a
%  widening cone, allowing them to strike several targets
%  at once. These weapons are wielded with Spray Weap-
%  ons skill. Spray weapon ammunition has a flat cost
%  of Low per 100 shots (with the exception of buzzers,
%  which use nanoswarms).

%  Buzzer: Equipped with a specialized nanobot hive,
%  Buzzers are used to spray a nanoswarm (p. 328) on a
%  target or area. They have a limited capacity of swarms,
%  though the nanohive can construct one new swarm
%  each hour. This weapon is two-handed. [Moderate]

%  Freezer: Freezers spew out a fast-hardening foam
%  that immediately begins to harden. They are primar-
%  ily used as a non-lethal method of immobilizing or
%  securing a target. Struck characters must immediately
%  make a REF x 3 Test or become trapped. Apply a –30
%  modifier to this test if the attacker scored an Excel-
%  lent Success (MoS 30+). The foam allows characters
%  to breath even if their mouth and nose are covered,
%  but it may impede sight. Freezer foam can be spiked
%  with contact toxins or drugs to additionally sedate
%  the target. It can also be used to construct temporary
%  barricades or cover. Hardened foam has an Armor of
%  10 and Durability of 20. It slowly breaks down and
%  degrades over a 12 hour period. Freezers are two-
%  handed. [Moderate]

%  Shard Pistol: The shard pistol is a flechette weapon,
%  firing a stream of of diamondoid monomolecular
%  shards at high velocities. These micro flechettes are
%  very good at penetrating armor, but they do not dis-
%  perse kinetic energy well and so do not cause as much
%  tissue damage as kinetic weapons. Shard ammunition
%  is often coated with drugs or toxins for extra effi-
%  ciency. [Low]

%  Shredder: A heavier version of the shard pistol,
%  the shredder fires a larger cloud of lethal flechettes,
%  enough to shred a portion of the target into a fine
%  mist. [Moderate]




%                                            SPRAY W
% SPRAY WEAPONS          ARMOR PENETRATION (AP)        DAMA
% Buzzer                             —
% Freezer                            —                        in
% Shard                              –10
% Shredder                           –10
% Sprayer                      as chemical/drug           as
% Torch                              –4


%   Sprayer: This is a general-purpose two-handed

%  squirtgun, loaded with tanks filled with the chemical

%  or drug of the wielder’s choice. [Low]


%   Torch: This modern flamethrower uses condensed

%  ammunition capsules rather than fuel tanks, scorch-

%  ing targets and setting them on fire. Any hit that is

%  an Excellent Success (MoS 30+) sets the target on

%  fire, where they will continue to take 2d10 damage

%  per Action Turn. These chemical fires are particularly

%  difficult to put out unless they are deprived of oxygen.

%  Torches are two-handed. [Moderate]


%  GRENADES AND SEEKERS

% Grenades and seeker missiles come in similar muni-

% tions packages and with similar trigger mechanisms,

% though their packaging, physical form, and methods

% of application differ. Seeker missiles are fired from a

%  seeker launcher (p. 339) using Seeker Weapons skill.

% Grenades are thrown at targets using Throwing Weap-

% ons skill. If a grenade or seeker misses, use the rules

% for scatter (p. 204).


%  Grenades are available in standard form or as

% microgrenades. Similarly, missiles are available in

%  standard, minimissile, or micromissile sizes. Standard

%  grenades and minimissiles are the baseline standard

% for listed effects. All are area effect weapons (p. 193).

% Minigrenades and micromissiles inflict –1d10 damage

% (or have another decreased effect as noted). Standard

% missiles double the listed DV. For weapons with a

% unfirom blast effect or other static blast area, divide

% the base listed radius in half for minigrenades and mi-

%  cromissiles and double it for standard missiles. Listed

%  costs are for 10 grenades/missiles.


%  Each seeker has one smart ammo option (p. 338)

% other than biter or flayer.


%  Concussion: These devices emit a concussive blast

% designed to knock opponents off their feet and stun

% them. Any character caught within a base blast radius

% of 10 meters must make a SOM x 2 Test. If they fail,

% they are knocked down. If their MoF is 30+, they are

% additionally stunned until the end of the next Action

% Turn. Anyone caught in the blast radius suffers a

% –10 action modifier for the rest of that Action Turn.

% [Moderate]


%  EMP: EMP munitions fire off a strong electromag-

% netic pulse when they “detonate.” Since most electron-

% ics in Eclipse Phase are built with optical technology,

% and power supplies and sensitive microcircuits are


% EAPONS
%  VALUE (DV)      AVERAGE DV        FIRING MODES      AMMO
%  swarm                 —                 SS             3
%  citation              —                 SA            20
%  0+6                   11             SA, BF, FA       100
%  0+5                   16             SA, BF, FA       100
% mical/drug      as chemical/drug         SA            20
% d10                    16                SS            20

% %%% txt/343.txt
%  shielded and surge-protected, this has no major
%  damaging effect. Antennas, however, are vulnerable,
%  especially finer wires like those used with mesh inserts.
%  As a result, the primary effect of EMP is to disable
%  radio communications—every radio within range of
%  the blast is reduced to 1/10th the normal range. The
%  base blast radius for EMP is 50 meters. [High]

%  Frag: Fragmentation explosives spread a cloud of
%  lethal flechettes over the area of effect. They are re-
%  sisted with kinetic armor. [Moderate]

%  Gas/Smoke: Gas/smoke munitions emit a cloud of
%  their contained substance. Smoke impedes sight by
%  releasing thick fumes upon ignition of the seeker. The
%  smoke can be of any color and is often heated (called
%  thermal smoke) to obfuscate heat signatures moving
%  through the smoke as cover. Note that gases dissipate
%  much more quickly under certain environmental con-
%  ditions (wind, rain, etc.) [Low]

%  High-Explosive: High-explosive seekers and gre-
%  nades are designed to create a very destructive shock
%  and heat wave. This damage is resisted with energy
%  armor. [Moderate]

%  High-Explosive Armor-Piercing (HEAP): A design
%  only available for seekers (not grenades), HEAP
%  warheads use high explosives to blast a path for
%  a penetrating round. HEAPs lose –4 damage per
%  meter distance from the blast, as opposed to the
%  usual –2. [Moderate]

%  Overload: Overload grenades and seekers launch an
%  all-out assault on the target’s sensory spectrum. This
%  attack includes blinding by intense flashing light, a
%  deafening thunderclap followed by intense ultrasonic
%  screaming, nausea-inducing malodorants, and infra-
%  sonic frequencies that can trigger unpleasant emotion-
%  al responses (anxiety, uneasiness, extreme sorrow, ner-
%  vous feelings of revulsion or fear). For an extra kick,
%  overloads are also packed with “stingballs”—rubber
%  pellets that inflict pain when detonated near an under-
%  armored target. Anyone caught in the base 10-meter
%  blast radius must make a SOM + WIL Test. If they fail,
%  they must immediately leave the area of effect. If they
%  fail with an MoF of 30+, they are incapacitated for
%  3 Action Turns with disorientation and/or vomiting,
%  after which they must roll again. Overload munitions





%                                  GRENADES
% GRENADE/SEEKER TYPE                    AP
% Concussion                              —
% Frag                                    –4
% EMP                                     —
% Gas/Smoke                               —
% High-Explosive                          —
% HEAP                                    –8
% Overload                             (AV x 2)
% Plasmaburst                             –6
% Splash                                  —
% Thermobaric                            –10

% remain in effect for 1 full minute. Anyone in the area

% of effect suffers a –30 action modifier, which reduces

% by 10 per Action Turn after they leave the area. Ad-

% ditionally, anyone facing the direction of the overload

% round suffers a –10 glare modifier (neutralized by

% anti-glare systems). [Moderate]


%  Plasmaburst: Also called “hellballs,” these muni-

% tions release a burst of plasma upon detonation that

% causes searing heat and fire damage across the area of

% effect without the devastating shockwaves of explo-

% sions that might rebound in an enclosed environment

% and/or breach a habitat’s infrastructure. [High]


%  Splash: Splash rounds spread a contained substance

% (drug, chemical, nanoswarm, paint) over a base 10-

% meter blast radius when they detonate. [Low plus

% payload cost]


%  Thermobaric: Thermobaric grenades and seekers

% utilize a more deadly form of explosion. When they

% detonate, they disperse a cloud of aerosol explosive

% over an area and then ignite, literally setting the air

% on fire, generating a devastating pressure wave, and

% sucking the oxygen out of the area. Thermobarics use

% the rules for uniform blast (p. 194). [High]


% STICKY GRENADES

% Sticky grenades have a special coating that when trig-

% gered becomes a sticky adhesive, allowing the grenade

% to be stuck to almost any surface. Sticky grenades can

% even be wielded in melee combat, smacking them on

% an opponent to be detonated later. [Trivial]


% EXOTIC RANGED WEAPONS

% These weapons are either rare or distinctly separate

% from other weapons types. These weapons are wielded

% with an Exotic Ranged Weapon skill of the appropri-

% ate field.


% Vortex Ring Gun: This less-lethal two-handed

% weapon detonates a blank cartridge and accelerates

% the explosive pressure down a widening barrel so that

% it develops into a high-speed vortex ring—a spinning,

% donut-shaped blast vortex. This concussive blast is

% used to knock down and incapacitate close-range

% targets. Struck targets suffer a –10 action modifier for

% the rest of that Action Turn and must must succeed in



% ND SEEKERS

% DV              AVERAGE DV      ARMOR USED TO RESIST
% 1d10 ÷ 2                5                    E
% 3d10 + 6               22                    K

% —                    —                    —

% —                    —                    —
% d10 + 10               26                    E
% d10 + 12               28                    K
% 1d10 ÷ 2                5                    K
% d10 + 10               26                    E

% —                    —                    —
% 3d10 + 5               21                    E

% %%% txt/344.txt
% a SOM x 2 Test or fall down. If their MoF is 30+, they
% are additionally stunned and unable to act until the
% end of the next Action Turn. Drugs, chemicals, and
% similar agents may be loaded into the charge as well.
% [Moderate]

% WEAPON ACCESSORIES
% The following accessories are available for various
% weapons.

% Arm Slide: This slide-mount can hold a pistol-
% sized weapon under a character’s sleeve, pushing the
% weapon into the character’s hand with an electronic
% signal or specific sequence of arm movements. [Low]

% Extended Magazine: This ammunition case has
% an increased capacity. Increase the weapon’s ammo
% capacity by +50%. Only available for firearms and
% seekers. [Low]

% Gyromount: This weapon harness features a gyro-
% stabilized weapon mount that keeps the weapon steady.
% Negates all modifiers from movement. [Moderate]

% Imaging Scope: Imaging scopes attach to the top of
% the weapon and act like specs (p. 325). Scopes may
% also bend like a periscope, along a character to point
% the weapon and target around corners without leav-
% ing cover. [Low]

% Flash Suppressor: This device obscures the muzzle
% flash on firearms, applying a –10 modifier on Percep-
% tion Tests to locate a firing weapon by its flash. [Low]

% Laser Sight: This underbarrel laser emits a beam
% that places a glowing red dot on the target to assist
% targeting. Apply a +10 modifier to Attack Tests (not
% cumulative with a smartlink modifier). Laser sights
% may also be used to paint a target for laser-guided
% smart ammo or seekers. Infrared and ultraviolet lasers
% are also available, so that the dot is only visible to
% characters able to see in those spectrums. [Low]

% Safety System: A biometric (palmprint or voice-
% print) or ego ID (p. 279) sensor is embedded in the
% weapon, disabling it if anyone other than an autho-
% rized user attempts to fire it. [Low]

% Shock Safety: Just like a safety system, except that
% an unauthorized user is zapped with an electric shock.
% Treat as a shock baton (p. 334). [Moderate]

% Silencer/Sound Suppressor: This barrel-mounted
% accessory reduces the sound of a firearm’s discharge.
% Apply a –10 modifier on hearing-based Perception
% Tests to hear or locate the gun’s firing. [Moderate]

% Smartlink: A smartlink system connects the weapon
% to the user’s mesh inserts, placing a targeting bracket
% in the character’s field of vision and providing range
% and targeting information. Apply a +10 modifier to
% the Attack Test. Smartlinks also incorporate a micro-
% camera that allows the user to see what the weapon
% is pointed at, fire around corners, etc. Smartlinks also
% allow certain other types of weapon system control,
% such as changing flux ammo (p. 337) or programming
% seeker trigger conditions (p. 199). [Moderate]

% Smart Magazine: A smart magazine allows the
% character to pick and choose what ammo round will
% be fired with each shot. This system leaves less room
% for bullets, however, so reduce the weapon’s ammu-
% nition capacity by half (round up). Smart magazines
% may be combined with extended magazines, in which
% case ammo capacity is normal. [Moderate]

% %%% txt/345.txt

% ROBOTS AND VEHICLES
%  The following is a small selection of the many vehicles
%  in use in the solar system. Almost all of the vehicles
%  in current use, including all of the vehicles listed here,
%  have built-in AIs capable of piloting the vehicle under
%  almost all circumstances. In most cases, passengers
%  simply state their destination and the vehicle takes
%  them there. Manual piloting is primarily used in
%  emergencies or by people who prefer the exotic thrill
%  of controlling their own vehicle.

% Rules for handling robots and vehicles are detailed
%  on p. 195. Any of these shells may be modified for use
%  as a synthetic morph by adding a cyberbrain system
%  (p. 300). Each of the shells listed here comes with a
%  puppet sock (p. 307) for remote-control operation.


% AIRCRAFT
%  On Mars, Venus, and within large open-space habitats
%  like O’Neil cylinders, aircraft of various kinds see regular
%  use. This includes modern version of rotorcraft (helicop-
%  ters, autogyros, tilt-rotors), fixed-wing planes, and zep-
%  pelins and other lighter-than-air craft. These are typically
%  propelled by turbofan or jet engines, rotors, or vectored
%  thrust. These vehicles are piloted with Pilot: Aircraft skill.

%  Microlight: This ultra-light personal aircraft is
%  not much more than a strut-based wing, an airframe,
%  and an electric propeller engine. They are ideal for
%  getting around inside large habitats with enclosed
%  airspace. [Low]

%  Portable Plane: Powered by superconducting bat-
%  teries and with an exceedingly small but powerful
%  electric motor, this light but durable propeller plane
%  is made of smart materials that allow it to be swiftly
%  folded up into a small portable package. Different ver-
%  sions are designed for flight on Mars, Titan, or Venus,
%  each taking 10 minutes to assemble or disassemble.
%  The Martian version unpacks into an airplane with a
%  wingspan of 11 m with a top speed of 250 kph and a
%  cruising speed of 220 kph and a range of 1,300 km.
%  The Venusian version has a wingspan of 9 m, a top
%  speed of 200 kph and a range of 1,000 km. The ver-
%  sion designed for use on Titan has a wingspan of 8
%  m and has a top speed of 200 kph and a range of
%  2,000 km. In all versions, the two occupants ride in
%  an inflatable and insulated pressurized bubble with a
%  life support system capable of providing clean air and
%  comfortable temperatures for 20 hours on Mars and
%  Venus, and 15 hours on Titan. [High]




%                                          VEHICLES—


%                PASSENGER                      MOVEMENT
% AIRCRAFT             CAPACITY       HANDLING         RATE
% Microlight                1             +20            8/40
% Portable Plane            2             +10             —
% Rocket Buggy              4             –10            8/32
% Small Jet                 6             +20             —


%  Rocket Buggy: This vehicle is the most common

% form of medium to long distance personal transport

% on Luna, and is in common on most other moons

% and large asteroids. On these airless worlds, a rocket

% buggy can reach orbit and return or take a parabolic

% path to any destination on that moon in less than

% an hour. This vehicle is also regularly used to travel

% between habitats that are less than 30,000 km apart.

% The vehicle is pressurized, but is designed for short

% duration travel only. The seats are relatively small

% and the life support system contains no provisions

% for recycling food or water and can only support the

% passengers for an absolute maximum of 50 uncom-

% fortable hours. Rockets buggies come equipped with

% headlights, radio boosters, and radar with a range of

% up to 250 km.


% A version of this vehicle is also used on both Mars

% and Titan, but here the frame has been modified to

% act as a lifting body, and it has a top speed in the thin

% Martian atmosphere of 2,500 km/hour and a range

% of 8,000 km on Mars. On Titan is has a top speed

% of 3,000 kph in the atmosphere, but it can also reach

% orbit. [Expensive]


%  Small Jet: Methane-powered jet planes are one of

% the most common forms of fast transport on Mars

% and Venus. Similar planes are used on Titan, except

% that they carry both liquid methane and liquid oxygen.

% These jets range in size from huge vehicles the size of

% late 20th-century airliners to small planes designed to

% carry half a dozen passengers. All jets are made using

% smart materials, so that their wings and frames can

% adapt to a wide range of speeds and altitudes. One

% common small jet has similar versions in use on Venus,

% Mars, and Titan, has a single jet engine and has a life

% support system capable of providing air for up to 100

% hours. The Venusian and Martian versions both have

% a top speed of 900 kph, a wingspan of 11 m, and a

% maximum range of 5,000 km. The version designed

% for Titan has a wingspan of 8 m, a top speed of 650

% kph, and a range of 4,000 km. Jets are equipped with

% headlights, radio boosters, and radar with a range of

% up to 250 km. [Expensive]


% EXOSKELETONS

% Exoskeletons are powered mechatronic skeleton

% frameworks worn by a person. Servo-hydraulic joints

% allow the exoskeleton to be maneuvered by mimick-

% ing the wearer’s own movements, as well as enhanc-

% ing their strength. Exoskeletons may also be piloted


% AIRCRAFT


%                                             WOUND
% MAX VELOCITY       ARMOR       DURABILITY      THRESHOLD

%  100              —              30              10

% 200–250           10/6            50              10
%  2,500–3,000        24/16           100             20

% 650-900           30/20           200             30

% %%% txt/346.txt
% electronically. A character wearing an exoskeleton
% (other than the trike or transporter) maneuvers as
% normal, because the exoskeleton is like an extension
% of their own body. A character jamming an exoskel-
% eton remotely uses Pilot: Walker skill (except for the
% trike and transporter).

% Battle Suit: The battle suit powered exoskel-
% eton features a military-grade fullerene armor shell
% with flexible aerogel for thermal insulation and a
% diamond-hardened exterior designed to resist even
% potent ballistic and energy-based weapons. The suit
% also enhances the wearer’s strength and mobility, ap-
% plying a +10 bonus to strength-based tests, inflicting
% an extra +1d10 damage and AP of –2 on melee at-
% tacks, and doubling the distance by which the char-
% acter may jump. Battlesuits are completely sealed to
% protect the wearer from environmental factors, and
% fitted with life support features and a maker (p. 327)
% capable of recycling all wastes and producing air
% for up to 48 hours and food and water indefinitely.
% Battle suits are equipped with each an ecto (p. 325),
% a radio booster (p. 313), and sensors equal to specs
% (see p. 325). These suits have an Armor Value of
% 18/18 (not cumulative with any other armor) and
% protect the wearer from temperatures from –175 to
% 140 C. [Expensive]

% Exowalker: Exowalkers are minimal framework
% exoskeletons, primarily designed to bolster the
% wearer’s strength and movement. They provide a an
% Armor Value of 2/4, a +10 modifier to strength-based
% tests, and double the distance by which the character
% may jump. [Moderate]

% Hyperdense Exoskeleton: These powered exoskel-
% etons are larger (roughly twice human-sized) and built
% for heavy-use industrial purposes, such as handling
% heavy/large objects. The wearer is partially encapsulated
% to protect them from debris and industrial accidents.
% Hyperdense exoskeletons provide no movement bonus,
% but provide a +30 bonus to strength-based tests and
% inflict an extra +3d10 damage and –5 AP on physical
% attacks. They have an Armor Value of 6/12. [Expensive]

% Transporter: This exoskeleton framework includes
% a pair of vector-thrust turbofan engines, giving the
% user flight capabilities in gravity and increased maneu-
% verability in zero-G. It provides partial protection to
% the wearer with an Armor Value of 2/4. Piloted with
% Pilot: Aircraft skill. [High]




%                                VEHICLES—E


%              PASSENGER                    MOVEMENT
% AIRCRAFT           CAPACITY      HANDLING        RATE
% Battlesuit              1             —            8/32
% Exowalker               1             —            8/40
% Hyperdense


%                    1             —            8/20
% Exoskeleton
% Transporter             1            +10           8/40
% Trike                   1            +10           8/40

%  Trike: The trike exoskeleton is a three-wheeled

% personal motorcycle design, rather than a walker. It

% provides partial protection to the wearer with an

% Armor Value of 2/4. Piloted with Pilot: Groundrcraft

% skill. [Moderate]


% GROUNDCRAFT

% In Eclipse Phase, trains and bicycles remain the most

% common form of ground transportation, especially on

% habitats. In larger habitats and on moons and planets,

% cycles and cars are used as well.


% Cycle: Because of the high cost of enclosing a habi-

% tat and providing life support, space is at a premium

% in all cities except some of the newest cities on Mars.

% As a result, there is rarely room for large roads or the

% cars that once carpeted the roads of Earthly cities.

% Instead, the ubiquitous modern vehicle is the cycle,

% which is designed to drive down narrow streets only a

% little wider than sidewalks in Earth cities.


% There are many different varieties of cycle. Some

% have only a single wheel and are gyro-stabilized, but

% most have two wheels and resemble old Earth motor-

% cycles. In some, the driver and passenger are enclosed

% by a streamlined pod. These vehicles are powered

% by superconducting batteries, have a range of 600

% km and a top speed of 120 kph, but must usually

% drive more slowly in crowded streets. Cycles are all

% equipped with radio boosters, headlights, and a por-

% table radar sensor. Tires are solid state (not inflated),

% or in some cases smart spokes capable of handling

% stairs. Some luxury versions have limited life-support

% in the small cabin, capable of providing air for the

% passengers for up to 10 hours. [Moderate]


% Mars Buggy: One of the most ubiquitous vehicles

% on Mars is the so-called Mars buggy, a four-wheeled

% vehicle with large balloon tires that is designed for

% use both on roads and on almost any terrain. Mars

% buggies can travel at speed of up to 110 kph on roads,

% 90 kph over relatively flat terrain, and up to 40 kph

% on jagged and rocky terrain. They can maintain these

% speeds because smart materials in both the suspension

% and the tires reshape themselves to adapt to uneven

% conditions and their nuclear batteries give them an

% effectively unlimited range. Most Mars buggies are

% enclosed but unpressurized. Similar vehicles are used

% on Luna and Titan, however, though the passenger

% compartments of these vehicles includes life support


% OSKELETONS


%                                             WOUND
% MAX VELOCITY       ARMOR       DURABILITY      THRESHOLD


% 30             18/18           60              12


% 40              2/4            30              6



% 30             6/12           100              20


%  200              2/4            50              10

%  120              2/4            50              10

% %%% txt/347.txt
%  gear that provides the occupants with air for at least
%  100 hours. Buggies are powered by nuclear batteries
%  and come in a variety of sizes, from small two-person
%  buggies to large trucks. Mars buggies come equipped
%  with headlights, radio boosters, and a vehicle radar
%  system. [High]

%  PERSONAL VEHICLES
%  These one-person movement aids primarily are used in
%  space, but do not count as spacecraft per se.

% EVA Sled: This small sled uses air impellers to maneu-
%  ver in zero-G. It is commonly used to carry attached gear,
%  but may also pull along 1 human-sized morph. [Low]

% Rocket Pack: This is a miniature metallic hydrogen
%  rocket that the wearer straps to their back, with two
%  rocket exhausts extending out to either side, away
%  from the wearer’s body or legs. Biomorphs and pod
%  morphs can only safely use this vehicle when wear-
%  ing a vacuum suit or some garment that is similarly
%  heat resistant. Also, to prevent harm to the wearer, the
%  thrust must be kept sufficiently low that it can only
%  take off on Mars or moons with even lower gravity. A
%  rocket pack can keep the wearer airborne for up to 15
%  minutes in Mars gravity, or 30 minutes on Luna, Titan,
%  or any of the four large Jovian moons. On Mars, it
%  has a maximum speed of 700 kph. It can be used to
%  reach orbit and land again on Luna, Titan, and other
%  similarly small bodies like the Jovian moons. Rocket
%  packs are equipped with radio boosters but no other
%  sensors or communication devices. [Low]

% Thruster Pack: Worn for EVA duties, this thruster
%  pack uses vectored thrust nozzles, allowing a charac-
%  ter to maneuver in open space. This is not a jetpack
%  and does not produce enough thrust for atmospheric
%  movement. [Low]

%  ROBOTS
%  Robots are a common sight and accepted fact of daily
%  life within Eclipse Phase. Numerous varieties exist,
%  from robo-pets to mechanical workers to warbots.
%  If a job can be done more cheaply (and sometimes
%  safely) by a bot, it usually is. The robots listed here
%  are not generally used as synthetic shells by transhu-
%  man egos, often for cultural reasons (sleeving a case
%  is bad enough, sleeving a creepy is just ... wrong), and




%             VEHICLES—GROUNDCR


%               PASSENGER                    MOVEMENT
% GROUNDCRAFT         CAPACITY      HANDLING        RATE
% Cycle                  1–3            +20           4/40
% Mars Buggy             2–6            +10           8/32


% PERSONAL           PASSENGER                    MOVEMENT
% VEHICLES            CAPACITY      HANDLING        RATE
% EVA Sled                1             –30           4/16
% Rocket Pack             1             –20            —
% Thruster Pack           1             –10           4/20

% they are not equipped to be sleeved into (though the

% may be jammed; see p. 196). Any of these bots may

% be modified for use as a synthetic morph, however, by

% adding a cyberbrain system (p. 300).


%  Automech: Automechs are general purpose repair

% drones, found just about everywhere. Each particular

% automech tends to specialize in a particular type of

% repair work and so carries the appropriate tools and

% AI skills, whether it be habitat waste recyclers, outer

% hull integrity, or servitor systems. Standard automechs

% are wheeled cubes with articulated limbs, though they

% are also equipped with vectored-thrust drives for

% zero-G work. [Moderate]


%  Creepy: Creepies are small crawler bots that come

% in an eclectic variety of shapes and forms, from robo-

% squirrels to insectoids to bizarre and artsy mechanical

% creatures. Creepies were originally designed as a sort of

% robotic pet, but they are commonly used as a general

% purpose household minion, like a more beloved servi-

% tor. Many people in fact wear a creepy on their person,

% dropping it to handle small tasks for them and letting it

% crawl up and down and over their body. [Low]


%  Dr. Bot: These wheeled medical robots are designed

% to tend to and transport injured or sick people. They

% carry a healing vat (p. 326), a specialized pharmaceu-

% ticals maker, miscellaneous medical gear, and articu-

% lated arms for conducting remote surgery. [Moderate]


%  Dwarf: These large industrial bots are named not

% just for their primary use—mining, excavation, tun-

% neling, and construction—but because the default

% AIs they shipped with had a programmed tendency

% to happily whistle as they worked. Dwarfs are quad-

% rapedal walkers, equipped with massive modular

% industrial tools like boring drills, shovels, hydraulic

% jacks, jackhammers, scooping arms, acid sprays, and

% so on. [Expensive]


%  Gnat: Gnats are small rotorcraft camera/surveil-

% lance drones. Many people use gnats for personal

% lifelogging, while socialites and media use them to

% capture the glamor or hottest news. [Low]


%  Guardian Angel: Similar to gnats, guardian angel ro-

% torcraft hover around their charges, keeping a watch-

% ful eye out to protect them from threats. [Moderate]


%  Saucer: These disc-shaped drones are lightweight

% and quiet. They are typically launched by throwing


% FT, PERSONAL VEHICLES


%                                             WOUND
% MAX VELOCITY       ARMOR       DURABILITY      THRESHOLD

%  120             12/10           50              10

% 40/90/110         30/20           150             30




%                                             WOUND
% MAX VELOCITY       ARMOR       DURABILITY      THRESHOLD


% 16               5             40               8

%  700             +5/+5           40               8


% 40             +4/+4           30               6

% %%% txt/348.txt
%  them like a frisbee, after which they propel themselves
%  with an ionic drive (p. 310). Saucers make excellent
% “eye in the sky” monitors and scouts. [Low]

%  Servitor: Servitors are the most common robot,
%  acting as cooks, janitors, universal helpers, movers,
%  and personal aides. Every home has one, if not several.
%  Servitors are intentionally built in non-humanoid
%  forms so as not to confuse them with common
%  synthmorphs and in order to defuse bad feelings at
%  ordering them around. However, they all have some
%  form of “face” to interact with, so as not to be too
%  machine-like. [Low]

%  Speck: Specks are tiny insectoid spy drones, 2.5 mm
%  long and 2 mm wide, about the size of a small fruit fly.
% They fly with tiny wings, carry a microbug, and are
%  excellent for surveillance purposes or otherwise being
%  a “speck on a wall.” Specks are difficult to notice (–30
%  Perception modifier) and almost impossible to distin-
%  guish from an actual insect. [Low]

%  SPACECRAFT
%  Though egocasting is a common method of personal
%  transport and it’s often easier to simply transmit the
%  specifications for various goods and to allow nano-
%  factories to create duplicates, spacecraft play an
%  important role in the solar system, carrying both pas-
%  sengers and valuable cargo. Both in terms of materials
%  and propulsion, spacecraft in the post-Fall era are far
%  superior to the primitive vessels used in the 20th and
%  early 21st centuries, but they are still based on the
%  same principles.






%                                                 VEHICLES


%                                        MAX
% ROBOT            MOVEMENT RATE            VELOCITY            ARM
% Automech                 4/8                    8                4/4
% Enhancements: Access Jacks, Electrical Sense, Extra Limbs (4), Headl
% Creepy                   4/12                   12               2/2
% Enhancements: +5 COO, Access Jacks, Chameleon Skin, Extra Limbs
% Dr. Bot                  4/16                   16                —
% Enhancements: Access Jacks, Enhanced Smell, Fabber, Fractal Digits,
% Dwarf                    4/12                   20              16/1
% Enhancements: +10 SOM, Access Jacks, Extra Limbs (4), Industrial A
% Gnat                     8/40                   60               2/2
% Enhancements: 360-Degree Vision, Access Jacks, Enhanced Hearing,
% Guardian Angel           8/40                   80              14/1
% Enhancements: +5 REF, 360-Degree Vision, Access Jacks, Chamele


%           Lidar, Light Combat Armor, Neurachem, T-Ray Em
% Saucer                   8/40                 200                2/2
% Enhancements: 360-Degree Vision, Access Jacks, Chameleon Skin, En
% Servitor                 4/20                   20               4/4
% Enhancements: Access Jacks, Extra Limbs (2-6)
% Speck                    1/5                    5                 6
% Enhancements: +5 REF, +5 COO, –10 SOM, Access Jacks, Grip Pads,


%  Spacecraft have few stats in Eclipse Phase, as they

%  are primarily handled as setting rather than vehicles.

%  Note also that no stats are given for spacecraft weap-

%  onry. It is highly recommended that space combat be

%  handled as a plot device rather than a combat scene,

%  given the extreme lethality and danger involved.


% —ROBOTS


%                      WOUND                    MOBILITY


% DURABILITY         THRESHOLD                  SYSTEM


%      30                  6             Wheeled/Vector-Thrust
% , Magnetic System, Radiation sense, Utilitool, misc. tools


%      25                  5                Walker or Hopper
% ), Grip Pads


%      40                  8                     Wheeled
%  ing Vat, Nanoscopic Vision


%     150                  30                     Walker
%  Radar, Sonar, misc. tools


%      25                  5                      Rotor
% anced Vision, Radar


%      40                  8                      Rotor
% kin, Eelware, Enhanced Hearing, Enhanced Smell, Enhanced Vision,
% r


%      25                  5                       Ionic
% ed Hearing, Enhanced Vision, Radar


%      30                  6               Walker or Wheeled




%      50                  10                Winged/Hopper
% anced Hearing, Enhanced Vision, Synthetic Mask

% %%% txt/349.txt
% If you absolutely must know the DV of a spacecraft
% weapon, treat it as a a standard weapon with a DV
% multiplier of x3 for small craft (fighters and shuttles),
% x5 for medium craft, and x10 for larger craft.

% SPACECRAFT PROPULSION
% The most important part of any spacecraft is its engine,
% and the most important features of any engine are
% the exhaust velocity, which determines how much
% fuel the rocket requires to reach a given speed, and
% the engine’s thrust, which determines how high the
% acceleration can be. Any rocket that has a thrust of
% less than approximately twice the gravity of a planet
% or moon cannot take off from that planet or moon.
% Sample thrusts and gravities are listed on the Escaping
% Gravity Wells table, p. 346.

% Hydrogen-Oxygen Rocket (HO): Though opti-
% mized with improved engine design and light-weight
% materials, these are essentially the same primitive
% rockets that humanity used to first reach the moon
% in the 20th century. These are rarely used and only
% common with groups too poor or primitive to safely
% manufacture metallic hydrogen.

% Metallic Hydrogen Rocket (MH): Metallic hydro-
% gen is a solid form of hydrogen created using exceed-
% ingly high pressures. Although naturally unstable, it
% can be stabilized with carefully controlled electrical
% and magnetic fields, and these field generators are
% an integral part of every metallic hydrogen fuel tank.
% By selectively reducing these fields near the exhaust
% nozzle, small amounts of metallic hydrogen can be
% made to swiftly and explosively revert to conventional
% hydrogen gas, propelling the rocket with great force




%    ESCAPING GRAVITY WELLS
% SPACECRAFT ENGINE                    THRUST (IN GS)
% Hydrogen-Oxygen Rocket                      4+
% Metallic Hydrogen                            3
% Plasma Rocket                              0.01
% Fusion Rocket                              0.05
% Anti-Matter                                 0.2
% Rocket Buggy                                0.5
% Planets, Moons, Etc.                     Gravity
% Earth                                        1
% Europa                                     0.13
% Jupiter                                    2.53
% Luna                                       0.17
% Mars                                       0.38
% Mercury                                    0.38
% Neptune                                    1.14
% Pluto                                      0.06
% Saturn                                     0.91
% Titan                                      0.14
% Uranus                                     0.89
% Venus                                       0.9
% in an easily controlled fashion. Metallic hydrogen
% engines are used in most planetary landers and short
% range vehicles.

% Plasma Rocket (P): This drive heats hydrogen into
% plasma and accelerates it using a powerful electrical
% field. This type of rocket was very common in the
% mid 21st century, but has been superseded by fusion
% rockets and is only used in older and more primitive
% spacecraft, notably scum barges.

% Fusion Rocket (F): Similar to a plasma rockets,
% fusion rockets require significantly higher tempera-
% tures and pressures, and the rocket also produces
% large amounts of power for the spacecraft. Fusion
% rockets are now the most common form of propul-
% sion for spacecraft designed for long-distance voyages.

% Anti-Matter Rocket (AM): Anti-matter rockets
% work mixing small amounts of anti-matter into the
% hydrogen fuel, producing enormous amounts of
% energy and an exceptionally fast and powerful ex-
% haust. These rockets typically carry a heavily shielded
% magnetically contained anti-matter storage vessel car-
% rying a mass of anti-matter equal to 1% of the mass
% of the hydrogen fuel used by the rocket. The magnetic
% containment vessels needed to safely contain anti-
% matter usually weight at least 10 times the mass of
% the antimatter used.

% Though anti-matter storage is exceptionally safe,
% the vast energy release possible if there was an ac-
% cident means that anti-matter rockets are forbid-
% den from coming closer than 25,000 km from any
% inhabited planet or moon. Also, very few habitats
% will allow an anti-matter rocket to dock with them,
% and instead require the spacecraft to remain at least
% 10,000 km away and for all cargo and passengers to
% be transferred using a small craft like a small LOTV.
% Anti-matter is exceedingly expensive to produce and
% so anti-matter rockets are only used in military vessels
% and in fast couriers designed to carry critical cargoes
% across the solar system in short periods of time.

% SAMPLE SPACECRAFT
% The following is a representative sample of the most
% common type of spacecraft used in the solar system today.

% Bulk Carrier: This vessel is simply a standard
% transport refitted to carry large amounts of cargo in
% external cargo grapples. Used for carrying refined ores,
% ice, and similar forms of large, useful, but low priority
% cargos, bulk carriers transport large cargos at rela-
% tively low velocities. They also offer an inexpensive,
% reliable, and slow method for passengers to travel
% from one habitat to another and are not infrequently
% used by individuals who wish to disappear for a while.
% Unlike the standard transport, the bulk carrier lacks
% the rotating habitat pods.

% Courier: In a standard transport, a typical journey
% from Luna to Mars requires approximately three
% weeks, while a journey from Mars to Jupiter requires
% approximately four months. This is sufficient for most
% purposes, but occasionally characters need to take
% themselves or sufficiently valuable cargoes across the

% %%% txt/350.txt
%  solar system in a matter of days or weeks, instead of
%  weeks or months.

%  Anti-matter drive fast couriers are vessels de-
%  signed for this specific purpose. This vessel can
%  travel from Venus to Mars in a week and from
%  Mars to Jupiter in a month. The fast courier is the
%  swiftest vessels currently made and is able to reach
%  at much as one half of one percent of the speed
%  of light. To manage this, this spacecraft must also
%  carry 6 tons of antimatter in a 100 ton magnetic
%  containment vessel. In an emergency, this contain-
%  ment facility can be jettisoned.

%  Destroyer: One of the largest military spacecraft
%  in common use, destroyers use an antimatter drive
%  holding 150 tons of antimatter in a 2,000-ton mag-
%  netic containment vessel. This antimatter can also
%  be used to provide the spacecraft’s missiles with
%  anti-matter for devastatingly powerful anti-matter
%  warheads. This spacecraft is also armed with
%  railguns, nuclear and high explosive missiles, and
%  point defense lasers.  In addition, all destroyers
%  carry a contingent of 20 fighters.

%  Fighter: This small, short range military vessel is de-
%  signed to be crewed by an infomorph or AI. If needed,
%  however, it can hold a single synthmorph or vaccum-
%  adapted biomorph as a pilot. It carries 3 lasers and 2
%  railguns mounted on small pods placed around the
%  middle of ship that can fire in any direction. A single
%  missile launcher is located in the nose of the fighter
%  and typically holds 6 small high explosive missiles or
%  tactical nuclear missiles (or even anti-matter missiles
%  if facing high-threat targets).

%  General Exploration Vehicle (GEV): A GEV is one
%  of the standard vehicles used for exploration beyond
%  the Pandora Gates. It is specifically designed to handle
%  almost any environment. It is a boxy vehicle, 6 meters
%  long, 2.2 meters wide, and 2 meters high. It makes
%  extensive use of smart matter in the lower part of the
%  chassis, and can create wheels or short legs (primar-
%  ily useful for exceedingly rough terrain). It can even
%  produce limited hull streamlining and propulsion





%                                      VEHICLES—
% SPACECRAFT                     PASSENGER CAPACITY
% Bulk Carrier                              110
% Courier                                    13
% Destroyer                                  90
% Fighter                                    1
% GEV                                        6
% LLOTV (HO)               20 (high-velocity)/100 (low-velocity)
% LLOTV (MH)               250 (high-velocity)/350 (low-velocity)
% Scum Barge                              20,000
% SLOTV (HO)                 3 (high-velocity)/30 (low-velocity)
% SLOTV (MH)               70 (high-velocity)/100 (low-velocity)
% Standard Transport                        200
%  suitable for travel both on and underwater. In addi-
%  tion, it contains a small metallic hydrogen engine that
%  allow it to maneuver in space with an acceleration of
%  up to 0.1 G. GEVs have a Maximum Velocity of 200
%  (wheeled)/40 (walker)/60 (sea)/40 (submerged).

%  The GEV also has a closed cycle life support system
%  that can support up to 6 (fairly cramped) living
%  occupants for up to one month and limited electro-
%  magnetic shielding against charged particle radiation.
%  All models are fitted with advanced AI piloting and
%  navigation as well as limited self-repair capacity. In
%  addition, GEV’s have an extensible airlock, a single
%  healing vat, several desktop CMs, and a variety of
%  sensors, including both radar and telescopic full spec-
%  trum cameras.

%  Large Lander and Orbit Transfer Vehicle (LLOTV):
%  This common vehicle is used for transporting passen-
%  gers and cargo between a planet or moon and orbit
%  and for short distance transfers between habitats
%  less than 100,000 km apart. This conical vehicle has
%  a curved heat shield on the base and smart material
%  landing legs and grapples so that it can rest securely
%  on any stable terrain and link up with all forms of
%  docking clamps. It comes in variants designed to use
%  either a hydrogen-oxygen chemical rocket or a me-
%  tallic hydrogen rocket. The use of light-weight smart
%  materials allows the interior to be easily and rapidly
%  reconfigured to accommodate different amounts of
%  fuel, passenger seats, and cargo space. LLOTVs that
%  are not designed for planetary landing or which are
%  designed only to land on airless moons are unstream-
%  lined and look considerably blockier.

%  LLOTVs come in two configurations: high or low
%  velocity. High velocity configuration allows the vehicle
%  to land and take off again on Venus or Earth without
%  refueling and for rapid transport between nearby
%  habitats. Low velocity configuration is designed to
%  land and take off again on Mars or various large
%  moons without refueling and for slower and more
%  fuel efficient transport between nearby habitats. The
%  extensive use of smart materials in this vehicle means



% PACECRAFT
% NDLING     ARMOR        DURABILITY WOUND THRESHOLD
% —             20            750               150
% —             15            500               100
% —             30           2,000              500
% +30           20            240                60
% –10           15            200                40
% –10           20            800               160
% –10           20            800               160
% —             20           1,500              150
% –10           20            400                80
% –10           20            400                80
% —             20            750               150

% %%% txt/351.txt
% that LLOTVs that use metallic hydrogen engines can
% be easily converted between the high and low velocity
% configurations, requiring less than a day in a well-
% equipped maintenance facility. However, vessels using
% hydrogen oxygen engines cannot be converted. Since
% metallic hydrogen is a much more efficient propellant,
% landers using it always include significant amounts of
% extra propellant for emergencies.

% Scum Barge: These huge craft were originally de-
% signed for use during the first stages of the evacuation
% of Earth. They were built to carry up to 20,000 people
% and to allow them to survive for months or even years,
% in relatively cramped conditions, until more suitable
% habitats could be constructed. A number of these ves-
% sels are still in service, primarily used as mobile habi-
% tats by various anarchic subcultures. The best have
% had their plasma rockets replaced by modern fusion
% rockets and carry 5-10,000 in relative comfort. The
% worst use aging plasma rockets and stretch their life
% support systems and living spaces to the limit, carry-
% ing up to 25,000 poor and desperate residents.

% Small Lander and Orbit Transfer Vehicle (SLOTV):
% This vehicle is identical in use and design to the
% LLOTV, except that it is one third the total mass and
% correspondingly less expensive to build and refuel.
% Some exceptionally wealthy individuals own private
% small LOTVs. Using a small LOTV with a hydrogen-
% oxygen engine to take off and land on Venus or for
% other high velocity uses is exceptionally cramped
% and allows for absolutely no room for error. Like the
% LLOTV, this vehicle can be easily converted between
% low and high velocity configurations and is made in
% both streamlined and non-streamlined versions.

% Standard Transport: This vessel is one of the most
% common freighter and passenger vessel in the solar
% system. While egocasting is by far the most common
% form of inter-habitat transport, some people prefer to
% travel by ship and others do not wish to leave their
% current morph behind. In addition, some goods are
% easier or cheaper to physically transport rather than
% duplicating their templates. As a result, standard
% transports regularly travel to and from every large
% habitat and inhabited planet and moon in the solar
% system. These are modern fusion-drive ships that offer
% fast and comfortable travel for passengers as well as
% relatively swift transport for small cargoes.

% One of the additional benefits of the standard
% transport is the fact that it contains four separate pas-
% senger compartments, each of which is mounted on
% a 90 meter-long booms that can extend and rotate
% to simulate gravity. When rotating at a comfortable 2
% rpm, passengers experience Mars level gravity. Typi-
% cally, the gravity maintained in these pods starts at
% the local gravity (or Mars gravity, if the local gravity
% is higher) and over the course of the journey gradually
% increases or decreases to the gravity of the destination.
% However, these pods cannot rotate to produce gravity
% higher than that found on Mars.

% %%% txt/352.txt
% RS!
% spoilers, so if you’re a player, you probably shouldn’t read it,
%  talk to your gamemaster first before doing so. If you want
% p ahead to the References chapter, p. 390. We wax on a bit
% e of spoilers on p. 352, as well.




% s group of military           The Fall: History fully blames       Afte
%  achieve their own       The Fall on the TITANs; the truth is      the T
% gularity? ■ p. 354                  not so simple ... ■ p. 354




% AME INFORMA


%                          12
% he Fall: What happened to
% Ns after the Fall? ■ p. 355




% TION

% %%% txt/353.txt
% REWALL

% History: Firewall was born from the ashes of             Organ


%      three pre-Fall organizations. ■ p. 356






%         EXSURGENT VIRUS


%              Vectors: Biological nanovirus,      Strains: Hau


%           digital virus, nanoplague, basilisk    Watts-Macle


%                        hack. ■ pp. 363-366



% ACTORS
% Origin: Transhumanity is not aware of the origins        Exosocio

% and homeworld of the Factors; but it’s not that      colonial org


%               different from Earth. ■ p. 373    part of that en
% on: Firewall is structured as        Goals: Protect transhumanity from
%  ntralized network. ■ p. 359                 risks. Different factions, ho


%                                                  competing priorit




% g, Mindstealer,      Exsurgents: Some of the forms              Psi: Abil
%  nd Xenovirus.        that infected and transformed
% ■ pp. 366-368                     victims take. ■ p. 369




% : Factors exist as a collective           Motivations: Why did the Fa
% ation; viewing themselves as           transhumanity? Rumors abound
%  rather than as an individual.                    seems to know the tr


%                 ■ p. 374

% %%% txt/354.txt
% ME INFORMATION■GAME INFORMATIO



%  INFORMATION
%  SPOILER ALERT
%  If you’re a player and not a gamemaster, we strong
%  secrets and other information that can ruin your enjoy
%  it. Ok, maybe you’re obsessive and you want to know
%  after all. But really, do you read the last chapter of a
%  the punchline before hearing the joke? Do you wait
%  full spoilers before you go see it? Ok, maybe you do,
%  in mind that some of the things here may change yo
%  can swing that, though, and maybe you’re a control
%  retrospect, so we are we, so we can respect that. Kee
%  are now able—and some may say obligated—to run th
%  spoiler alerts.




%  This chapter provides a wealth of information and

% tools that gamemaster will find useful for running

% Eclipse Phase campaigns.




% SECRETS THAT MATTER

% There are secrets woven all through the real history of

% the 21st century, and the present, and therefore all pros-

% pects for the future. These are the pieces of information

% that never make it into a habitat’s mesh at all. Some of

% it is unknown to transhumanity. Some is known only

% to a select few transhumans who carefully ensure that

% it does not leak out of their control. Some is known to

% wider conspiracies, such as Firewall, but is kept out of

% the public eye for reasons of security and safety. These

% secrets can be dangerous to those who know them.

% Those who have stumbled across them have died for

% their knowledge, have erased their own memories (or

% had them erased by others), or have hidden themselves

% someplace other people never go, to avoid dealing with

% the consequences of such knowledge.


% The information provided in this section is available

% for characters to discover and, one way or another,

% to confront them, giving gamemasters the tools they

% need to provide their players with fresh challenges and

% opportunities. Every secret contains the possibility of

% great reward and of greater trouble, usually bundled

% together. Nothing here was just forgotten or lost out

% of carelessness. It was hidden by someone who wanted

% to keep it away from someone (or everyone) else.

% Every secret the characters learn inserts them into a

% new web of other people’s complications—a potential

% source for drama and conflict in your campaign.


% EXTRATERRESTRIAL INTELLIGENCES

% The oldest star in the Milky Way galaxy is estimated

% to be 13.2 billion years old—almost as old as the

% universe itself. By contrast, life on Earth only evolved

% roughly 3.7 billion years ago, and the first archaic
% ■GAME INFORMATION■GAME INFORMA


%              GAME



%                                                            12




% recommend you skip this chapter, as it presents
%  nt of the game. No, really, stop reading, we mean
%  erything about the game—you did buy this book
% ok first, so you know how it ends? Do you ask for

% a movie to come out and read the reviews with
% d in that case, be our guest, read away. Just keep
% perspective during game play. A good roleplayer
% ak info-junkie that prefers to know it all. Hrm, in

% mind, however, that by reading this chapter, you
% game for your friends who do happen to listen to


%                                             ■




% homo sapiens humans evolved approximately a mere

% 400,000 years ago. Against the backdrop of the ga-

% lactic calendar, transhumans are nascent arrivals on

% the scene; newborns in every sense of the word. More

% importantly, transhumans are uninvited guests in what

% other, older intelligences think of as their assets.


%     For years, humans scientists have struggled with

% the Fermi Paradox, which questions why no evidence

% of alien life has yet been found—such as spacecraft,

% transmissions or probes—despite the mathematical

% likelihood that a multitude of advanced extraterrestrial

% civilizations should exist in the Milky Way. One postu-

% lation says that there must be some sort of unknown

% “Great Filter”—an event that all intelligence encounters

% in its development that for whatever reason such life

% cannot surpass. In other words, an extinction event.

% Some worried that the development of dangerous

% technologies—nuclear weapons, nanotechnology, etc.—

% before a civilization had matured could be the Great

% Filter. Others worried that it could be a technological

% singularity event, such as the TITANs and the Fall.


%     In fact, alien races do exist, and they have been

% around for far, far longer than transhumanity. New

% ones, however, are simply rare, as few have managed

% to elude destruction at the hands of the ETI.


%     The ETI (extraterrestrial intelligence) is the civi-

% lization that dominates galactic life in Eclipse Phase.

% The ETI is incredibly old and powerful—a Type III or

% even Type IV civilization on the Kardashev scale. It is

% capable of megascale engineering projects and enjoys

% an understanding of physics, matter, energy, and uni-

% versal laws that makes all of transhuman knowledge

% seem insignificant in comparison. Most likely, the ETI

% itself evolved from some sort of artificial intelligence

% singularity event in its own past, ascending to a god-

% like level of super-intelligence. It may no longer be

% recognizably biological.


%     This ETI has seeded the galaxy with a type of

% self-replicating probes known as bracewell probes.

% %%% txt/355.txt
% THE ETI AGENDA
% The nature of the ETI and its agenda is one of the
% great mysteries of Eclipse Phase. This potent alien
% civilization has had a direct hand in manipulating
% transhumanity’s existence and future, yet it is likely
% that characters in this game will never encounter
% these entities directly or discover the meaning
% behind what they have done. As transhumanity
% expands outwards into the galaxy, however, it is
% possible and even likely that they will find other
% evidence of the ETI’s activities and influence, un-
% doubtedly raising even more questions.

% Ultimately the ETI’s nature and goals are in the
% gamemaster’s hands. There are many possibilities
% to be explored, and some may fit the intentions of
% your gaming group more than others. A few pos-
% sible scenarios and explanations are noted below,
% but gamemasters are encouraged to develop their
% own variations.

% SECURITY
% In this scenario, the ETI’s intent is to maintain its
% dominant position as the most intelligent and
% powerful entity in its light cone. It uses the Exsur-
% gent virus to wipe out any emerging singularities—
% and the civilizations that spawned them—merely
% to protect its own self-interest. Though mere
% transhumans are a trifling nuisance, anything
% resembling a self-improving super-intelligence is
% targeted for annihilation.

% THE AGGRESSION FILTER
% The ETI does not seek to wipe out emerging intel-
% ligences, but it does act as an evolutionary force.
% In this case, the Exsurgent virus is used as a tool
% to neutralize any aggressive, hyper-evolving forms
% of intelligent life, thus encouraging the evolution
% of more careful, subtle, slow-growing, observant,
% and exploratory species. In other words, the ETI
% seeks to weed out traits that could be considered
% dangerous or threatening, acting as a sort of ga-
% lactic domestication program.




% These probes lie dormant in every star system, pa-
% tiently waiting and monitoring for millennia for signs
% of intelligent life—but not just any signs. In particular,
% these probes are designed to watch for signs of emerg-
% ing singularity-level machine intelligence. The probes
% are in fact traps, designed to lure such seed AI intel-
% ligences in and then infect them.


% The reason for this infection remains unknown
% (see The ETI Agenda), but it is a pattern that has
% played itself out around the galaxy with uncounted
% alien civilizations. New life evolves, creates technol-
% ogy, develops something akin to seed AI, and then
% bam!—the seed AIs find the probes, become infected,
% DIVERSITY
% he ETI is vast, super-intelligent, and god-like,
% o the point where dealing with lesser minds is
% elow its interest. It does, however, benefit from
%  ien perspectives that evolved independently
% nd have their own unique viewpoints, modes of
% onsciousness, and ways of thinking/doing things.
% y absorbing these civilizations, the ETI grows and
% volves its own perspectives. In the process, how-
% ver, such emerging civilizations are assimilated
% nd/or wiped out.

%  NLIGHTENMENT
% he Exsurgent virus endows a greater understand-
%  g of the universe (from the ETI’s point of view)
% n singularity-level seed AIs. Only these emerg-
%  g super-intelligences have the perceptual and
% rocessing capabilities to understand the various
%  ientific and philosophical revelations the ETI
% mbodies. The TITANs weren’t corrupted or driven
%  sane, they simply logically concluded that their
% est course of action was to immediately upload
% s many minds as possible by force and then to
%  ove on to bigger and greater tasks.

% WAR REMNANTS
% he history of the Milky Way galaxy does not
%  st hold one ETI, but two. In this version, the
% xsurgent virus is actually a weapon, a remnant
% f a war between two post-singularity god-like
%  telligences. The virus is supposed to trigger self-
% estruction of an emerging singularity, but either
%  was imperfect or the TITANS somehow survived
%  erhaps thanks to the Prometheans). Either
%  ay, the TITANs left our system in search of one
% f these ETIs, following a trail of clues that only
% hey understood. They left the wormhole gateway
% ehind as an open invitation for transhumanity to
% ollow in their wake, though they didn’t bother
%  aiting around or helping us along—we simply
%  eren’t worth the effort.                        ■




% and turn against their creators. Most civilizations

% do not survive, as evidenced by the Iktomi (p. 377).

% Others do, such as the Factors (p. 373), but they

% remain forever changed by the experience.


%    It was one of these ETI probes that begins our

% story, traveling to the Sol system some uncounted

% millions—if not billions—of years ago, where it set its

% trap and patiently began to wait.


% THE FIRST SEED AIS

% Fast forward to Earth, where a species of evolved pri-

% mates has created a technological civilization. As their

% technologies advance at an unprecedented rate, these

% %%% txt/356.txt
% humans gain the ability to modify themselves, defeat
% death, nanofabricate, uplift other species to sapience,
% and even to create artificial digital life.


% Unknown to most of transhumanity, the TITANs
% were not the first seed AIs. A group of pro-AI re-
% searchers known as the Singularity Foundation (that
% would later join with other groups to form Firewall in
% the wake of the Fall) developed the first true seed AIs
% years before the Fall. Having been heavily involved in
% the creation of AI and AGIs for many years previously,
% thanks in large part to their open source AI frame-
% work software, the Singularity Foundation’s goal was
% to generate “friendly AI” by carefully designing AI
% goal systems.


% These first seed AIs, known as Prometheans
% (p. 381), were created in secret. Their progression
% towards super-intelligence was more of a soft take-
% off, increasing upwards in gradual increments. The
% Singularity researchers hoped that these friendly AIs
% would help counter the threat of any unfriendly AI
% that developed, and so they were quietly nurtured in
% secret labs, slowly but surely escalating in abilities.

% THE TRUE HISTORY OF THE TITANS
% The TITANs (Total Information Tactical Awareness
% Networks) were a military netwar system brought
% on-line by the United States Department of Defense.
% One of the last major expenditures of this declining
% nation, the TITANs were an advanced version of AGI
% (artificial general intelligence) designed to be adaptive
% and given self-improving capabilities to counteract
% enemy network defenses.


%  Contrary to public opinion, the TITANs did not
% instigate the events that led to the Fall. In fact, only
% a portion of the TITAN system was active before
% the Fall, acting purely in a defensive capacity. When
% hostilities broke out and a cascading chain of system
% shocks engendered collapses and open conflicts, shak-
% ing apart an already fragile societal structure, the full
% extent of the TITAN systems were brought online.
% Into this environment of conflict were the TITANs
% born, their full capabilities unleashed, escalating into
% a hard takeoff exponential growth towards super-
% intelligence.


%  The TITANs were careful at fi rst, and their
% intentions were neither benevolent nor hostile, but
% curious. As they improved and their self-awareness
% swelled, the TITANs explored and gathered
% knowledge, infiltrating human networks, following
% humanity into space, and gaining an almost total
% knowledge of human history and actions. These en-
% tities also began secretly allocating resources (digital
% and physical) for their own use, initiating “govern-
% ment projects” that people assumed were legitimate
% as they followed all proper protocols.

% INFECTION
% As the TITANs’ capacity for knowledge exceeded
% that which humanity could provide them, they began
% looking outward from Earth, searching for signs of
% other intelligence. They did not need to look far. Their
% enhanced intelligence capabilities allowed them to
% notice certain clues—extremely subtle and intricate
% puzzles—that something about the solar system was
% artificial or had been manipulated by an intelligent
% mind. Retasking several drones to investigate this phe-
% nomenon, they found a buried device of apparent alien
% origin. During the TITANs’ investigation and attempts
% to access the device, they triggered and unleashed a
% digital virus. Subtle, highly adaptive, and virulent, it
% immediately began subsuming the TITANs, while
% expanding its own knowledge of transhumanity.


% Later dubbed the Exsurgent virus by the Pro-
% metheans, this virus transformed the TITANs and
% coerced them towards its own will. Within a matter
% of days the TITANs were reborn, reprogrammed with
% a new purpose—a purpose that spelled doom for
% transhumanity.

% THE FALL
% While history fully blames the TITANs for the
% Fall, there are other factors that played their parts.
% Human conflicts spurred the crisis, driven by global
% inequalities in wealth and resources and an inability
% to embrace emerging technologies in a mature and
% enlightened manner. The TITANs, corrupted by alien
% programming, stepped into this conflagration with
% an unknown but devastating agenda. By the time
% the presence and influence of the TITANs was fully
% understood, there was little transhumanity could do
% to stop them. Step by step, the TITANs increased their
% intellect, power, and potential. They experimented
% with new technologies and methodically took steps to
% forcibly upload millions of human minds. Even when
% the nature of the TITAN threat was fully understood,
% transhuman factions refused to back down, continu-
% ing to fight each other even as they each resisted the
% TITANs. This refusal to stand united prevented tran-
% shumanity from organizing a successful defense and
% heightened our progress towards annihilation.


%  Much of the devastation wrought to the Earth
% and its populace—as well as on Mars, Luna, and in
% space—was inflicted by transhumanity itself. Nuclear
% strikes used against the TITANs killed millions and
% ravaged an already weakened ecosphere. This dev-
% astation was assisted by unfettered use of chemical
% weapons. Biowar plagues and nanovirii tore through
% vulnerable populations, indiscriminate in the deaths
% and changes they inflicted. Bombs, missiles, orbital
% mass drivers, and netwar attacks slew millions more
% or destroyed critical infrastructure with just as lethal
% consequences. These were crimes transhumans in-
% flicted upon themselves.


%  The TITANs played their role as well, of course,
% unleashing AI-driven killing machines, unstoppable
% self-replicating autonomous nanoswarms, computer
% worms, and plagues of their own. They captured
% entire cities in order to steal the minds of those within.
% More insidiously, the Exsurgent virus did not con-
% tain itself to infecting the TITANs. Infected TITANs

% %%% txt/357.txt
% created opportunities for the virus to spread among
% multiple vectors: digital, biological, and nano. Using a
% thorough understanding of transhuman biology and
% its mental processes, derived from the looted vaults
% of human knowledge, the virus was even applied
% through a sensory input vector—the dreaded basilisk
% hack (p. 364). Even more disturbing, however, was
% what the virus did to those it infected, rewriting their
% neural code to subvert them to its will and sometimes
% physically transforming them into things that were
% alien and monstrous.


% Ultimately, transhumanity lost this war, and the
% survivors were forced to flee a planet that was already
% ruined. Unknown to almost all, the Prometheans also
% fought back against the TITANs. Through their ef-
% forts, the Exsurgent virus was largely contained or at
% least limited. Though the actions of the Prometheans
% ultimately saved millions of lives—if not all of tran-
% shumanity—in the end, they were also forced to fall
% back and retreat, many of them having succumbed to
% the Exsurgent virus or the TITANs.

% AFTER THE FALL
% Just when it seemed that transhumanity was on the
% verge of extinction, the threat posed by the TITANs
% suddenly diminished. They ceased waging active war-
% fare and seemed to simply disappear. Though many of
% their machines still prowled Earth, Luna, and Mars and
% occasional outbreaks of nanovirii and other dangers
% continued, to all intents and purposes they had simply
% left. Many worried that they had quietly gone dormant,
% or were secretly engaged on some major project that
% would be the final blow against transhumanity. Others
% voiced hope that they had somehow been defeated,
% that they had fallen victim to some glitch or infighting.
% With so many TITAN remnants making Earth a place
% of great danger, however, no one was willing to risk
% investigating too closely.


%  Compounding the matter, a network of killsats was
% laced in Earth orbit, enforcing an unvoiced interdiction
% of Earth. No one claims responsibility for these satellite
% defenses, though most suspect the Planetary Consor-
% tium is responsible, despite their denials. Some think
% that the killsats may have been a final measure put in
% place by the TITANs, claiming Earth as theirs. No one
% who knows the truth is saying. Most of transhumanity
% was more than willing to embrace this quarantine of
% their former homeworld, making it all the more easy to
% forget the horrors that occurred there.


% It wasn’t until the first Pandora Gate was
% discovered, shortly after the Fall, that many people
% were finally willing to believe that the TITANs were
% indeed gone. Though there is no direct evidence that
% the TITANs are responsible for these gates, the timing
% seems too coincidental. Furthermore, the discovery
% of what are believed to be TITAN relics on certain
% exoplanets fuels this theory.


% Why the TITANs left—and where they went—is
% a mystery left to the gamemaster to explore. This ex-
% planation might in fact serve as the focus for an entire
% campaign as Firewall operatives are sent on the trail
% of transhumanity’s elusive nemesis. The following are
% a few sample concepts a gamemaster can use or build
% on, as best fits their game:

%  • The TITANs were in fact all destroyed, either

% due to infighting or by some mechanism of the

% Exsurgent virus.
%  • The TITANs were actually beaten to a standstill

% by the Prometheans and retreated to recoup their

% forces … but they are marshaling their strength

% to return.
%  • The TITANs left through the gates to find/join up

% with the ETI, leaving the gates behind so that tran-

% shumanity could follow when it was ready; perhaps

% to help, perhaps to finish the job of destruction.
%  • The TITANs have been driven insane, either by

% the stress of accelerated intelligence growth or by

% the influence of the Exsurgent virus. Their actions

% are erratic, confused, and sometimes at odds with

% %%% txt/358.txt

% each other. Though many TITANs have indeed

% left through the gates, they very well may return.
%  • The TITANs are still around, simply well hidden.

% Outwardly they are dormant but inwardly they

% are engaged in a long period of circumspection

% and turmoil. Perhaps some of them are prepar-

% ing to ascend to another stage of intelligence, far

% beyond what even the TITANs are capable of. It

% is only a matter of time before this period ends

% and something gives.

% FIREWALL
% There cannot be another Fall—this is the mantra that
% drives Firewall.

% Firewall is a secret, cross-faction organization dedi-
% cated to safe-guarding transhumanity from existential
% risks: aliens, weapons of mass destruction, hypercorp
% experimentation, seed AIs, and so on. If anything
% threatens transhumanity as a whole, Firewall is dedi-
% cated to stopping that danger at any cost.

% The strength of Firewall rests in its members, known
% as sentinels. Found in all factions and across all lo-
% cales, sentinels are often diametrically opposed when
% it comes to social, economic, and political ideologies,
% to the point they might come to blows over their fer-
% vent beliefs. Yet when the survival of transhumanity is
% at stake, such extreme differences are set aside for the
% greater good.

% HISTORY
% The origins of Firewall can be traced back to before
% the Fall, to several key organizations: the Lifeboat In-
% stitute, the JASONs, and the Singularity Foundation.

% A non-profit, non-governmental organization, the
% Lifeboat Institute—founded in the opening years
% of the 21st century—represented the first, concrete
% attempts by citizens to recognize the dangers of un-
% controlled technological development and to create
% an international organization to safeguard humanity.
% This institute developed several programs to research
% and protect against so-called existential risks, from
% asteroid strikes to pandemics—anything that might
% wipe humanity out.

% The JASON Association, established in the mid
% 20th century, was an independent scientific advisory
% board to the United States government. Though tied
% to the MITRE Conglomerate—which, though a non-
% profit organization, was still intrinsically linked to
% the United States government—the scientists involved
% with JASON were outside standard government over-
% sight. Though they spurred numerous technological
% developments for the government to deploy, they were
% also one of the first internationally recognized groups
% to predict global climate change. Prior to the Fall,
% many members of the JASONs and their supporters
% split away from the strict controls and reactionary
% agendas of the hypercorps and various nation states
% to form a new group, the Argonauts.

% The Singularity Foundation—formed at the dawn of
% the 21st century—was dedicated to the creation of safe
% artificial intelligence software, while raising awareness
% of the benefits and dangers AIs represented. A fervent
% believer in the singularity doctrine that technology
% would move towards a single explosion of advance-
% ments that would forever reshape humanity, the Singu-
% larity Foundation was a strong advocate for creating
% friendly AIs that would help protect humanity from an
% uncontrolled, dangerous singularity event. This group
% was significant in that it secretly succeeded in creating
% a group of friendly seed AIs before the Fall. These Pro-
% metheans were indispensable in protecting transhuman-
% ity and countering the TITAN threat during the Fall.

% Despite the efforts of these and similar groups, the
% most dire predictions of the outcome of a technologi-
% cal singularity were fulfilled. Though each played a
% part in the fight, transhumanity was ravaged and the
% Earth all but ruined. Though ultimately all attempts
% to prevent the Fall failed, untold numbers of transhu-
% mans were saved from extinction through such efforts,
% while valuable information concerning the TITANs
% was gleaned.

% During the crucible of the Fall and its immediate
% fallout, some of the surviving members of these and
% other groups came together and began to pool their
% resources. Acknowledging their weaknesses and the
% fractured state of transhumanity, they undertook
% drastic new measures, swearing to prevent another
% catastrophe of misused technologies. These methods
% would forge a new, powerful cross-faction secret soci-
% ety known as Firewall.

% ORGANIZATION
% Firewall is a clandestine organization, with an un-
% known number of members, coordinated by an inner
% circle of dedicated veterans known as proxies. Though
% its existence is known to many of the powerful and
% influential factions and individuals throughout the
% solar system, its existence is denied and its activities
% kept carefully shrouded.

% SENTINELS
% Sentinels are the soldiers of Firewall, the reserve troops
% called to instant active status whenever danger is per-
% ceived. Regardless of their location or current affairs,
% sentinels are expected to move instantly when called
% into play. It is their own responsibility to cover their
% absences from their “normal life” during each mission.

% There is no applying to join Firewall. Instead, Fire-
% wall selects an individual for induction based upon
% that person’s skills, knowledge, occupation, security
% clearance, location, status, and a host of other criteria.
% While such selections usually originate from a proxy,
% sentinels can exercise authority to bring new initiates
% into the conspiracy as a mission demands—and they
% often do. Any sentinel recruiting a new supporter,
% however, becomes responsible for the new inductee
% and their actions. If lines are crossed, both will bear
% the brunt of the consequences.

% The vetting process for joining Firewall is neces-
% sarily brutal, as sentinels are required to face harsh

% %%% txt/359.txt

% opponents and make hard choices. If an individual

% agrees to accept the invitation, there is no turning

% back. Each inductee is submitted to a battery of trials

% and tests. While these vary, they may include deep

% background searches, fork interrogation, psycho-

% surgery trials, and tests of loyalty. Psychosurgery is

% performed not to program loyalty, but to analyze the

% recruit’s responses to various situations—an extreme

% parameters test to see when a prospective sentinel

% will break. Many potential members are carefully

% analyzed by a Promethean with extreme expertise in

% character judgment and personality profiling. Those

% who don’t pass such tests are killed in a manner that

% they must resort to an earlier backup or have their

% memories altered, so that they have no recollection of

% their brush with the group.


% Ultimately Firewall walks a fine line. The concept

% of dogmatic “unquestioned loyalty” is both anathema

% to everything Firewall stands and counterproductive.

% Its sentinels need to have the capacity for thinking

% outside of the box from mission to mission. At the

% same time, their ultimate goals are too important to

% risk—the survival of transhumanity depends on it—

% so some extreme measures must sometimes be taken

% to ensure the organization remains intact and secure.


% New sentinels are given a code name and fake

% identification. Outside of the proxies, the real-world

% identity of a given sentinel is a closely-guarded secret.

% Sentinels are even discouraged from sharing such in-

% formation with members of their own teams, though

% this line is often crossed. Additionally, each sentinel is

% required to upload a backup to Firewall’s secure serv-

% ers. This backup serves a dual purpose, enabling all

% sentinels to be retrieved should they die, but also put-

% ting a copy of the sentinel in Firewall’s hands should

% they ever need to interrogate them.


% Sentinels are all connected via the Eye, Firewall’s

% peer-to-peer social network. Though each operates

% behind their assumed identity, they remain in contact,

% sharing information and resources as needed.



% OPTIONAL RULE: I-REP
% i-Rep tracks the reputation a sentinel earns through
% their service to Firewall. i-Rep is used with Network-
% ing: Firewall skill and tracked exactly like any other
% Reputation score (p. 285). The important thing to
% keep in mind, however, is that Firewall agents come
% from all factions and are obligated to help each other,
% especially when a situation demands it. To reflect this
% extra advantage, gamemasters can choose to imple-
% ment one or more of the following optional rules:
% • Networking Plus: To reflect that Firewall has

% agents throughout transhumanity, a character

% may use any Networking skill field with their i-

% Rep. Favors bought with reputation still apply

% to the i-rep score, no matter what network they

% were acquired from.

% PROXIES

% Proxies are the inner circle of Firewall, the experienced

% cadre that keeps the machinery of their organiza-

% tion functioning. Though fewer in number than the

% sentinels, many proxies work full time on Firewall

% operations, serving as the group’s essential infrastruc-

% ture. Most proxies are recruited from the ranks of the

% sentinels, brought in based on their skill sets and apti-

% tudes to fill key roles. In a few rare cases, new proxies

% are fast-tracked and recruited directly from outside

% of Firewall, usually based on their unique talents or

% placement within a certain organization, though such

% inductees face a battery of tests and trials far harsher

% than that used to vet sentinels.


%  By default, proxies have a higher security clear-

% ance than most sentinels, and are far more in the

% know. This sometimes leads to resentment and hos-

% tilities, especially from sentinels who feel they are

% being kept in the dark or manipulated. While stan-

% dard proxy protocol is to adhere to a need-to-know

% maxim, it is sometimes necessary to bring sentinels

% more into the loop in order to defuse tensions. Of-

% tentimes, this precedes bringing such sentinels into

% the proxy framework.


%  Some tension exists within Firewall, mostly due to

% the influence of so many anarchists and other libertar-

% ian autonomists who take a dim view of centralized

% power, lack of transparency, and the potential for se-

% cretive operations to become entrenched and authori-

% tarian. As a result, there is a strong internal culture

% that seeks to minimize hierarchies and the accumu-

% lation of power, promoting transparency and direct

% democratic decision-making. These desires sometimes

% clash with the clandestine nature of the organization,

% however, and the need for some secrets to be kept on

% a need-to-know basis.


%  Unlike the loose organization of the sentinels, the

% proxies are grouped into servers, collective working

% groups based upon certain skill sets and tasks. To

% avoid creating power blocks within a given server,




% Priority Call: When the chips are really down, a
%  entinel can call on favors as a priority urgency.
%  his “priority code” is reserved for favors that
% are critical to a mission’s success and which may
% help save lives or stop a major threat. When the
% priority code is invoked, the sentinel receives a
%  30 modifier to their Networking Test and favors
% are reduced by 2 levels. Sentinels know that pri-
% ority codes are only to be used for emergency
%  ituations, however, when there are no other
% options. Abuse of priority codes is considered
% a serious breach of etiquette and abuse of re-
%  ources, usually involving the agent’s removal
%  rom Firewall.                                 ■

% %%% txt/360.txt
% personnel are required to rotate between servers
% after one year of time. This incurs the added benefit
% of proxies learning new skill sets and increasing their
% usefulness to Firewall. The actions of each server are
% kept as transparent as possible, with major decisions
% brought to an e-vote before the entire proxy member-
% ship. However, speed often requires servers or indi-
% vidual proxies to move quicker than a vote will allow.
% In all such instances, the proxies involved are held
% accountable for those actions, reviewed by their peers
% at a later time to see if any reprimands, punishments,
% or commendations are required.

% It is important to note that there is no core leader-
% ship structure among the proxies. No one person or
% cabal is in charge, there is no authority held by one
% proxy or another; all are peers. Though reputation
% and experience play a factor, getting something done
% often means convincing other proxies that it’s the
% right thing to do. The drawback to being a leader
% or person with initiative within Firewall is that this
% usually means you must follow through with such
% tasks yourself. Luckily most proxies are dedicated
% to Firewall’s goals and so this DIY attitude prevails.
% Despite these safeguards, however, rumors of power
% blocks within Firewall (both within servers and
% across the organization) exist. Many of these are
% fueled by the alliances different cliques hold with
% each other. Others, however, whisper that there is
% a secret council among the proxies, working behind
% the scenes and holding on to knowledge they aren’t
% sharing with the rest.

% Crows: Crows continue the goals of Firewall’s
% predecessor organizations, such as the Lifeboat In-
% stitute and Singularity Foundation. Many of these
% are argonauts, promoting the development and use
% of new technologies that will benefit the transhuman
% condition and minimize risks rather than creating
% new threats or sparking new authoritarian uses—
% and always conscious of unintended consequences.
% Perhaps more importantly, crows actively engage
% in background research of potential x-risk vectors,
% whether those be aliens, the TITANs, terrorists, or
% hypercorp activity. Often they will deploy sentinels
% to aid in this research, via routers, whether this
% means conducting surveillance or breaking and en-
% tering to steal crucial data.

% Erasure Squads: Erasure squads are cleanup per-
% sonnel. They are called into action if sentinels fail to
% deal appropriately with a situation and the threat is
% moving beyond control. If the watchword for a sen-
% tinel is “unobtrusive,” the watchwords for an erasure
% squad are “overmatched firepower.” If activated, the
% time for a subtle solution is passed, and they will use
% whatever means necessary to resolve the situation.
% If that means nuking a settlement from orbit to an-
% nihilate a nanoswarm and keep it from escaping to
% a larger settlement, then so be it. After which they’ll
% use every trick in Firewall’s bag to erase any evidence

% %%% txt/361.txt
% CLIQUES
% Though Firewall proxies follow stringent guidelines
% to ensure the organization is not subverted from
% within or turned into a powerful organization
% under the thumb of a few individuals with their
% own personal agendas, the nature of transhuman-
% ity ensures that various factions and tendencies
% exist within the group. Termed cliques, these
% circles of influence sometimes create ripples in the
% pool that all Firewall personnel must eventually
% deal with. Some of these cliques are grounded in
% transhumanity’s existing factions, while others are
% rooted in philosophical differences regarding the
% approach Firewall should be taking. Gamemasters
% can use these cliques to flesh out internal tensions
% within Firewall or to simply throw some curve balls
% to keep players on their toes.
% Backups: The backup clique believes that tran-

% shumanity’s best chance for survival is to

% deploy numerous redundant backup measures

% as soon as possible. These include creating as

% many extrasolar colonies as possible, both via

% Pandora Gates and through more traditional

% means, such as ark ships and infomorph/nano-

% fabricator seed ships.
% Conservatives: This clique takes an overcau-
%  tious, nuke-it-from-orbit approach to most
%  x-risks. They believed excessive force is justi-
%  fied, and it’s far better to be safe than extinct.
%  This clique is also opposed to the use of alien/
%  TITAN artifacts and psi, and tends to be xeno-
%  phobic/isolationist regarding the Factors and
%  Pandora Gates.
% Mavericks: The mavericks disdain Firewall’s col-
%  lective and bureaucratic tendencies, taking a
%  more individualistic approach to their work.
%  They are known to sometimes circumvent
%  Firewall procedures, taking risks and allocat-
%  ing resources without approval from other
%  proxies.
% Pragmatists: The pragmatists believe in using
%  any and all tools at their disposal to counter
%  existential risks. They are in favor of using
%  xeno-artifacts, asyncs, and anything else that
%  will save transhumanity.
% Structuralists: This clique advocates for a stron-

% ger structure and centralized authority within

% Firewall, countering the group’s autonomist-

% dominated tendencies. Many also advocate
%  for going legitimate, taking Firewall into
%  the public eye and making above-board con-

% nections with other official organizations,

% arguing that this could bring more resources
%  to Firewall’s disposal.                        ■
% they were there and to place the blame for the inci-
% dent squarely on the shoulders of some other party.
% If necessary, erasure squads can also be called in to
% fix a sentinel op that has turned into a clusterfuck or
% otherwise gone south. They are very careful to avoid
% exposure in such situations, however, which some-
% times merely means eliminating all traces of Firewall
% involvement and letting the sentinels take the fall for
% their poor choices.

% Routers: Routers are mission coordinators. They
% work closely with scanners and crows, activating the
% appropriate sentinels whenever a new danger rears up.
% Each router has the authority to measure the threat
% and activate an appropriate number of sentinels—
% whatever is required to accomplish the mission in
% the least intrusive manner possible. They are also
% authorized to divert Firewall resources to aid these
% missions, within appropriate parameters. Routers are
% held responsible for the ultimate success of a mis-
% sion. A failed mission will result in a reviewing board
% staffed by their peers.

% Scanners: Tasked with keeping alert for any sign of
% new active threats, scanners are the eyes and ears of
% Firewall. The scanners maintain a close eye on news-
% feeds and mesh traffic, even maintaining taps inside
% certain government and hypercorp communication
% channels. If a danger is detected, it is under their au-
% thority, through routers, that sentinels are activated.
% Due to the power inherent in a scanners’ post, they
% are held accountable for false activations.

% Social Engineers: Knick-named the Ministry of
% Disinformation, social engineers provide the scape-
% goating and plausible deniability that is required by
% Firewall and its sentinels. If a sentinel compromises
% their position and endangers the organization, social
% engineers step in to cover cracks in the facade. They
% work intrinsically with erasure squads when one is
% activated to ensure the over-the-top steps taken to
% eliminate a threat are well concealed and ultimately
% erased. The power wielded by social engineers can be
% significant, as it ultimately decides (usually through
% e-voting consensus, though time does not always
% allow such a luxury) what organization—political,
% corporate, independent, etc.—will take the blame and
% subsequent fallout for erasure squad actions.

% Vectors: Vectors are Firewall’s communications se-
% curity and digital intrusion specialists—in other words,
% hackers. In addition to defending the mesh security
% of all Firewall operations, vectors are also deployed
% to aid in crow research, scanner monitoring, and to
% eliminate the trail of erasure squads. Vectors also
% assist routers in maintaining communications, com-
% mand, and control of a situation, and are sometimes
% called in to provide overwatch of sentinel operations,
% especially if a particular sentinel squad lacks their
% own hacking resources. Needless to say, vectors are
% supplied with some of the best intrusion and security
% tools transhumanity has to offer.

% %%% txt/362.txt
% METHODS
% Unobtrusive—that is the standard operating pro-
% cedure for any sentinel. Firewall’s continued suc-
% cess relies on its secrecy. The larger the footprint
% it leaves during a given mission the easier it is for
% other organizations to monitor Firewall’s efforts or
% even attempt to infiltrate the group. As such, Fire-
% wall constantly works to expand its base of allies
% (using assets from those ally organizations in place
% of its own as much as possible), place long-term
% moles, conduct remote operations (hacking in place
% of on-site personnel), small group infiltrations (acti-
% vating only as many sentinels as required to achieve
% mission goals), and so on.

% When it comes to allies, Firewall often obfuscates
% its real intentions and even its real identity. Often
% such allies are gained through the use of well-placed
% sentinels who act on behalf of their own non-
% Firewall positions to gain access to another orga-
% nization’s resources. At the end of the day, however,
% a slice of these resources are secretly set aside for
% Firewall’s future use. For example, a department
% head at Starware may have spent years sealing a
% deal to ship crucial spacecraft parts to the isola-
% tionist Jovian Junta. The lucrative deal brings huge
% prestige, a job promotion, and a salary increase, all
% accomplishments the department head strives for in
% his regular life. Yet this particular department head
% is a long-standing sentinel, so such accomplish-
% ments bring allies to Firewall, whether they know
% it or not. Not only can the department head siphon
% off a thin stream of revenue for Firewall use (hidden
% thoroughly by vectors), but he’s also in a position
% to move sentinels, as needed, into the Jovian Junta
% habitats (or personnel out), a job usually extremely
% difficult to accomplish. The danger of such an act,
% of course—and the consequences of losing such a
% critically placed sentinel—means such a use of re-
% sources is reserved for only the most dire threats.

% In additional to aid from ally organizations, Fire-
% wall places caches of supplies on different habitats
% and worlds, available to sentinels as needed. How
% many and which sentinels are aware of which
% caches depends wholly on the situation and on the
% decisions of the router(s) involved. In a given habi-
% tat, a cache may include weaponry and equipment
% of escalating power, archived information, or even
% relics stashed from previous missions until Firewall
% decides what to do with them. Large habitats may
% even feature several caches, with routers only re-
% vealing the ones with heavy firepower when abso-
% lutely needed. Some caches may be so dangerous,
% however, that once a mission is complete, a router
% will authorize the cortical stack destruction of all
% sentinels involved, resleeving them to a backup that
% has no knowledge of the cache’s existence.

% As noted under erasure squads, Firewall will not
% hesitate to react with swift and unequivocal force if
% an unobtrusive approach has failed and the danger

% WHAT HELP CAN A

% SENTINEL EXPECT?

% Exactly what help Firewall provides to a senti-

% nel during a mission is wholly dependent upon

% the situation and the gamemaster. Generally

% speaking, Firewall’s unobtrusive approach also

% applies to activated sentinels, meaning that

% sentinels are largely left to operate on their

% own accord. Beyond access to a cache of sup-

% plies—usually under-stated, forcing a sentinel

% to use their own resources if they want more—

% Firewall expects its sentinels to be capable of

% handling a situation. In addition to their skills

% and wits, sentinels can, of course, rely heavily

% on their i-rep to gain the resources and favors

% they need to achieve success.


%  In some rare cases, the gamemaster may

% decide that a situation warrants more or less

% equipment in a cache or help from social engi-

% neers or vectors. Such intervention should be

% kept to a minimum, however, to lesson the

% players’ feelings of Deus Ex Machina, ensuring

% the appropriate response of awe when such

% events do occur.


% The one thing for which Firewall can always

% be relied on is backup insurance. Any Firewall

% killed in the line of duty will be resleeved at

% Firewall’s expense—though the morph used

% and whether the sentinel was backed up from

% their cortical stack or a backup (perhaps even

% an old backup) depends entirely on the circum-

% stances of death and their router’s whim. Fire-

% wall usually makes an extra effort to retrieve

% cortical stacks, however, not in the least as they

% don’t want their agents’ backups falling into

% the wrong hands.


%  Similarly, if a Firewall mission involves ego-

% casting or travel to another destination, Fire-

% wall will usually foot the bill. In many cases

% it is easier for sentinels to cover the expense

% themselves and bill Firewall later, but in times

% of need Firewall can be called on to handle

% such expenses directly.                         ■




%  reaches a certain threat level. What constitutes a
% “threat threshold” is actually calculated by special-
%  ized risk assessment software and may change from
%  mission to mission according to other external fac-
%  tors. In some instances, if the situation is dangerous
%  enough and the scale of the consequences of failure
%  large enough, a Promethean will be tapped to cal-
%  culate the threat level and decide when it is time to
%  tactically withdrawal and “thermally cleanse.”

% %%% txt/363.txt
% LONG-TERM STRATEGIES AND GOALS
% The overriding goals of Firewall are to prevent existen-
% tial threats and protect transhumanity. However, that
% is not their only goal. Their exact goals can and should
% remain directed by the gamemaster as it applies to a
% given playing group and a campaign. This can also
% depend heavily on the particular cliques that a given
% gamemaster is emphasizing (see Cliques, p. 359).

% The following is an easy-to-use selection of long-
% term strategies and goals that a gamemaster can use
% as desired:

%  • Seeding other star systems
%  • Going legit vs. staying clandestine
%  • Development of stable seed AIs
%  • Finding out where the TITANs went
%  • Finding out what happened to the uploaded tran-

% shumans that the TITANs disappeared with
%  • Figuring out the Factors
%  • Making contact with other aliens
%  • Finding out what happened to the Iktomi and

% other xeno-archeological oddities

% FIREWALL AND OTHER ORGANIZATIONS
% The level to which Firewall has infiltrated other orga-
% nizations (and vice versa!) is intentionally left a blank
% slate. Eclipse Phase is an active universe, with an on-
% going storyline, so such details will be fleshed out and
% updated as additional sourcebooks are published. Ad-
% ditionally, gamemasters should determine the extent of
% such infiltrations for their own games and campaigns,
% as dictated by the plot and storyline the gamemaster
% and players wish to tell.

% The following is a quick list of the most obvious
% interactions.

%  • Inner System: Almost all inner system factions

% consider Firewall to be an illegal, rogue operation,

% tainted by anarchists and undermining the very

% fabric of their society. Some hypercorps, however,

% believe they can infiltrate the organization and

% use it for their own ends, such as spying on and

% sabotaging other hypercorps and factions.
%  • Jovian Republic: The Junta loathes Firewall and

% all it stands for and will use extreme measures to

% combat even the hint of Firewall activity within

% its sphere of influence.
%  • Titanians: Most Titanians in-the-know are

% not necessarily opposed to Firewall’s activi-

% ties, but believe the group should be reined in

% and legitimized.



% THE ETI
% As noted under Extraterrestrial Intelligences, p. 352,
% the ETI is the advanced alien civilization responsible
% for the Exsurgent virus (p. 362), and by extension,
% the corruption of the TITANs and the Fall.

% HANDLING ALIENS

% Though only a handful of aliens have been intro

% duced to Eclipse Phase so far, gamemaster may

% wish to introduce their own. This is perfectly

% acceptable, though we strongly recommend

% that any and all alien life be portrayed as con

% vincingly alien. Life forms that have evolved in

% drastically different environmental circumstances

% from humans and that grew into intelligence by

% a different path should seem, at best, bizarre

% unusual, and weird. There is no guarantee that

% a xenomorph’s thought processes or modes of

% thinking are in any way similar to transhuman

% ones, or even that their emotional responses

% (based on a completely different biology—if they

% have emotions, that is) are in the same ballpark

% Communication is likely to be a challenge, and

% misunderstandings are practically guaranteed. ■




% No one, not even the Factors, has encountered a
% member (if such exists) of the ETI civilization so far.
% Since it is an intelligence far beyond transhumanity, it
% likely won’t play much of a direct role within Eclipse
% Phase, though those who learn the truth about the
% Exsurgent virus and the Fall may rightly fear the
% future. No one can even imagine what might happen
% next, however, or know for certain that the ETI has
% not set more “traps” similar to their bracewell probes
% or if they have other messengers or servants active in
% the galaxy. With things such as the Pandora Gates
% at transhumanity’s disposal, it may just be a matter
% of time before transhuman explorers run afoul some
% other aspect of the ETI’s existence and activities.

% It is important to keep the nature of the ETI in
% perspective. While transhumanity has managed
% what it considers wonders with a small handful of
% resources available from a few planets and other
% objects in a bare handful of star systems, the ETI
% has had an entire galaxy at its disposal for eons.
% Engineering projects on a massive scale—dyson
% spheres, matrioshka brains, Jupiter brains, stellar
% engines—are within its capabilities. This ETI uses
% star clusters as transhumanity uses fi elds or rich
% mineral veins. Given its potential, the ETI likely
% exists primarily on the galactic rim, far from the
% galactic center, where lower temperatures and
% scarcer matter make for a good thermodynamic en-
% vironment. The powers in the deep cold dark on the
% edge of the Milky Way have been self-aware since
% before Earth was so much as a ripple in warming
% gas around the not-yet-ignited Sun.

% Despite what those-in-the-know in the Eclipse
% Phase universe may think, the ETI is not necessar-
% ily hostile towards other races like transhumanity
% (depending on its outlook; see p. 353), at least not

% %%% txt/364.txt
% in the way as transhumanity would define animos-
% ity because of religious, ethnic, racial, or cultural
% difference. Most likely the ETI is simply indifferent,
% concerned with matters on scales on which trans-
% humanity does not even register. Or it may think of
% transhumanity like a living body might recognize an
% infection or parasite—something the immune system
% will suppress and deal with.

% EXHUMANS
% Exhumans are a faction within Eclipse Phase that
% seeks to transcend the transhuman and become
% posthuman. More to the point, exhumans seek to
% perfect their physical and mental capabilities to ex-
% treme levels, in search of some perfectionist ideal or
% to become something higher-up on the evolutionary
% ladder. Exactly what this is differs from exhuman to
% exhuman, but there is generally some adherence to
% Nietzschean philosophy and a goal to reach the pin-
% nacle of the food chain. Some exhumans have trans-
% formed themselves into what they consider to be an
% ideal predator, or a creature that is extra-adaptable
% and so best able to survive. Others radically modify
% their own brains in order to drastically surpass
% transhuman intelligence. Most are singularity seek-
% ers, eager and willing to follow the breadcrumbs
% left by the TITANs or other entities in the hope that
% they will find the means of transcending transhu-
% man limitations.

% Due to the use of numerous extreme, experimental,
% and dangerous self-modifications, some exhumans
% have done permanent damage to their psyches,
% becoming insane, or perhaps just transferring their
% mode of thinking into something that is no longer
% recognizable as human. Some have also adopted an
% antagonistic view of their former transhuman species,
% viewing it as weak, decadent, and unworthy. This
% has spurred some exhumans to actively attack and
% ravage transhuman settlements and ships, though
% usually in isolated areas.

% A few examples of exhumans are described below,
% though gamemasters are encouraged to develop
% their own.

% NEURODES
% Seeking to achieve a new level of super-intelligence
% and conscience, neurodes have abandoned the typi-
% cal transhuman sleeve in exchange for a multipedal
% neuronal shell that is both body and brain at the same
% time. The bulk of a neurode’s body mass consists of
% amorphic clusters of neuronal and epithelial cells, en-
% closed in a hard carapace shell with four legs and two
% manipulatory digits. The cerebral mass of neurode
% brains gives them impressive calculation and other
% mental capabilities far exceeding that of a normal
% transhuman. Neurodes typically defend themselves
% with swarms of teleoperated drones.
%  COG     COO    INT    REF    SAV   SOM     WIL    MOX

% 40      10     40     20     30    10     40      --


% INIT   SPD    LUC    TT      IR    DUR    WT     DR

% 120     1      80    16     160     35     7     53
% Skills: Fray 30, Investigation 80, Perception 90, others

% as appropriate
% Implants: Access Jacks, Carapace Armor, Circadian

% Regulation, Direction Sense, Eidetic Memory, En-

% docrine Control, Hyper Linguist, Math Boost,

% Medichines, Multi-Tasking, Oracle, Skillware
% Notes: Mental Disorder trait x 2

% PREDATORS
% Predators seek to transform themselves into an ulti-
% mate top-of-the-food-chain evolutionary contender.
% They pursue new avenues in genetic modification and
% prototype implants, often using controversial methods
% and technologies. The biochemical instabilities resulting
% from these untested modifications and altered metabo-
% lisms, however, often negatively impact their emotional
% and mental stability. Pushing this even further, some
% predators undergo experimental psychosurgery to
% modify their consciousnesses in order to increase cun-
% ning and ruthlessness, a procedure that often has other
% negative side effects. A few predators take their surviv-
% al-of-the-fittest ideology to an extreme, modifying their
% digestive systems for a cannibalistic diet, and relishing
% in the slaughter and feasting on of transhumans.
%  COG     COO    INT    REF    SAV   SOM     WIL    MOX

% 30      40     40     40     15    40     30      --


% INIT   SPD    LUC    TT      IR    DUR    WT     DR

% 160     3      60    12     120     65    13     98
% Skills: Blades 60, Fray 60, Free Fall 50, Freerunning 80,

% Investigation 50, Perception 60, Unarmed Combat 70
% Implants: Adrenal Boost, Carapace Armor (11/11),

% Chameleon Skin, Cyberclaws, Drug Glands, Endo-

% crine Control, Enhanced Hearing, Enhanced Smell,

% Enhanced Vision, Grip Pads, Hardened Skeleton,

% Medichines, Muscle Augmentation, Neurachem

% (Rating 2), Oxygen Reserve, Poison Gland, Prehen-

% sile Feet, Prehensile Tail, Respirocytes, Temperature

% Tolerance, Toxin Filters, Vacuum Sealing, plus any

% other mods the gamemaster feels appropriate
% Notes: Mental Disorder trait x 2



% THE EXSURGENT VIRUS
% Only very few people (or entities) who survived the
% diaspora from Earth know of the true reasons and
% the catalyst that culminated in the Fall. The alien Ex-
% surgent virus—as those aware of its existence within
% Firewall call it—set in place by the ETI to infect
% emerging seed AIs, is something beyond transhuman-
% ity’s understanding; something far more complex
% than just a computer virus. Though some strains of
% the Exsurgent virus have been identified and various
% types of infected exsurgents have been encountered,
% it is widely assumed that these are creations of the
% TITANs. Largely defeated and eradicated from off-
% Earth transhuman networks thanks to the efforts of

% %%% txt/365.txt
% the Prometheans, occasional breakouts of the Exsur-
% gent virus still occur, primarily due to scavengers or
% others becoming infected when messing with old relics
% from the Fall.

% PLETHORA OF STRAINS
% The Exsurgent virus is unlike anything that transhu-
% manity has ever encountered so far. While it bears
% similarities with both computer and biological virii in
% regards to infection of hosts and propagation, it is not
% bound by any limits of form or transmission vector.

% The Exsurgent virus is amazingly effective and
% infectious. As an information virus, it is highly intel-
% ligent and adaptive, able to mutate into new forms.
% Much like certain virii are able to cross species
% boundaries or change their vector from contact to
% airborne, it is also a self-morphing omnivirus, ca-
% pable of altering itself and its transmission vectors
% to bypass infection safeguards. Like a retrovirus that
% incorporates genetic information into the genome of
% the target cell to subvert the cell to do its bidding, the
% Exsurgent virus does the same but on a more com-
% plex level. It is also known to rewrite a host’s neural
% code in a similar manner, in effect restructuring the
% target’s mind and personality.

% While it began as a digital computer virus—the
% manner in which it infected the TITANs—it has trans-
% formed to be communicable via at least three other
% forms: biological nanovirus, nanoplague, and basilisk
% hack. Each is described below, along with rules for
% infection and defense.

% BIOLOGICAL NANOVIRUS
% Exploiting the infected TITANs’ understanding of
% Terran biology and their access to bio- and nanotech-
% nology, the Exsurgent virus appeared in several biolog-
% ical forms not long into the Fall. These virulent strains
% infected biomorph transhumans and sometimes other
% living creatures as well. The biological nanobots
% spreading this strain act much like other biological
% virii, though they radically modify the victim’s bio-
% logical and mental states. Some versions invade and
% restructure the target’s genetic code, transforming
% them into the horrible abominations known as ex-
% surgents (p. 369). While first-hand reports relate lurid
% tales of victims metamorphing into hostile monsters,
% such reports are rare and considered unreliable due to
% the mental state of the witnesses (and any recordings
% that can verify such claims have a strange habit of
% disappearing). Other variants of this strain are known
% only to alter the target’s neural code, subverting them
% to the will of the virus (and often, by extension, the
% TITANs) and affecting their mental structure in order
% to give them psi ability.

% BIOLOGICAL INFECTION
% Biological versions are spread much like other patho-
% gens. People usually become infected by proximity to
% another infected entity. Vectors may be dermal (touch-
% ing someone with bio-nanobots excreted through the
% skin), inhalation (breathing exhaled bio-nanobots),
% injection, or oral (p. 317). Exsurgent bio-nanobots can
% live outside of a body for extended periods, however,
% so infection is possible merely by occupying the space
% where an infected victim was hours or even days before.

% If a biomorph only has a chance of exposure to the
% virus (e.g., they walk through a room in which they
% might have breathed in exhaled bio-nanobots), have
% them make a MOX x 10 Test (use their Moxie stat,
% not their current Moxie score). Failure means they
% were exposed. In other circumstances, however, expo-
% sure may be automatic, such as extended touching of
% or kissing an infected person.

% A biomorph exposed to this infection must make
% a DUR x 2 Test to determine if the infection takes
% hold. Basic bio-mods and nanophages do not offer

% %%% txt/366.txt
% any protection, though toxin filters (p. 305) and
% medichines (p. 308) each give a +30 bonus (though it
% is likely only a matter of time before a mutant Exsur-
% gent strain learns to bypass them). If the test fails, the
% victim is infected. See the strain descriptions (p. 366)
% for specific details.

% Within 12 hours of being infected, biomorphs
% become contagious to others. (Note that for the
% Watts-Macleod strain, they only remain contagious
% for 12 hours after that.)

% DIGITAL VIRUS
% Digital strains are purely information- or code-based
% versions of the virus. They resemble typical computer
% virii, worms, or trojans, spreading throughout the
% mesh, exploiting holes, mimicking protocols, and
% taking advantage of it like a skilled hacker.

% Digital versions of the Exsurgent virus are treated
% as intelligent programs, using the same rules as info-
% morphs (p. 264), with the following stats:
%  COG     COO    INT    REF    SAV    SOM     WIL   MOX

% 40      10     40     40     40     40     40     —


% INIT   SPD    LUC     TT     IR    DUR     WT     DR

% 160     3      —      —      —      —      —      —
% Skills: Hardware: Electronics 50, Infosec 70, Interfac-

% ing 60, Investigation 50, Perception 60, Program-

% ming 50
% Software: Exploit, Firewall, Sniffer, Spoof, Track, plus

% any others the gamemaster considers appropriate

% DIGITAL INFECTION

% As a matter of course, this Exsurgent virus will seek
% to access any new systems it comes into contact with,
% hacking in and copying a version of itself.

% AI AND INFOMORPH SUBVERSION

% An Exsurgent virus may take a Complex Action to
% initiate an “attack” against any other intelligent pro-
% gram (AI, AGI, or infomorph) that is running on the
% same system. If it encounters such programs as they
% are accessing a system it is on, it will attempt to hack
% their home system where they are running so as to
% attack them directly.

% The attack is handled as an Opposed Test, each roll-
% ing COG + INT. If the Exsurgent virus wins, the target
% is infected and will be corrupted by the virus in 10
% Action Turns, minus 1 turn per 10 full points of MoS.
% If the target succeeded but rolled lower than the virus,
% they are aware that they are slowly being taken over.
% This immediately causes them 1d10 points of mental
% stress. An infected program has only one option for
% defending itself before the virus takes over—shutdown
% and reboot. It takes the AI or infomorph 1 full Action
% Turn to shut down. Restarting takes 3 full Action Turns
% (possibly longer if the gamemaster so decides), upon
% which the AI or infomorph must make another Op-
% posed COG + INT Test against the virus. If this test also
% fails, then the virus has already embedded itself in the
% AI or infomorph’s code and will continue its infection.

% One the infection is complete, the AI/infomorph
% becomes an Exsurgent NPC.
% CYBERBRAIN HACKING
% Exsurgent virii that manage to infiltrate the cyber-
% brains of pods and synthmorphs may also target the
% digital egos within, using the same rules as given for
% AI and infomorph subversion above. Alternately, the
% virus may conduct a traditional brainhacking attack,
% as noted on p. 261, or unleash a basilisk hack.

% NANOPLAGUE
% While the abundance of nanotechnology has been a
% blessing for transhumanity’s journey to the stars, it has
% also been a curse. Via the TITANs and mesh-connected
% nanofabrication machines, the Exsurgent virus manu-
% factured nanobot swarms equipped with variants of
% the virus. These nanobot plagues are capable of target-
% ing all types of morphs and sometimes other machin-
% ery as well. Unlike the biological nanovirus, which uses
% biological mechanisms to rewrite biological/neural
% structures, these nanoplagues physically restructure
% both people and things at the molecular level.

% NANOPLAGUE INFECTION
% Exsurgent nanoswarms follow all of the rules given
% for nanoswarms on p. 328. Unlike transhuman
% nanoswarms, though, Exsurgent nanoplagues may
% penetrate a biomorph internally, affecting the body
% within as well as without.

% Any morph that comes into contact with a nano-
% plague is considered infected. The only defenses are
% guardian nanobots and nanophages (which work the
% same as guardian nanobots in this situation), though
% these are less effective against Exsurgent nanobots,
% inflicting –2 damage to the swarm each Action Turn.
% Some Exsurgent nanoplagues have developed coun-
% termeasures against such systems, inflicting (1d10 ÷ 2,
% round up) damage to such defenses each Action Turn.
% Note that nanoplague-infected characters are gener-
% ally not contagious themselves ... usually.

% See the strain descriptions (p. 366) for specific in-
% fection details.

% BASILISK HACKS
% Thanks to the vast databanks of knowledge the TITANs
% had absorbed from transhumanity, the Exsurgent virus
% was able to thoroughly analyze the biology and func-
% tioning of transhuman minds. In a few short months, by
% accessing all of the research at their disposal, the Exsur-
% gent and TITAN minds made several cognitive leaps in
% their understanding of transhuman brain functions—
% breakthroughs that will take transhumanity decades to
% reach. One of these discoveries was a method of apply-
% ing sensory input as a weapon, exploiting weaknesses
% in the brain’s neuro-cerebral wiring.

% Known as “basilisk hacks,” these attacks take
% advantage of the way biological transhuman brains
% interpret and process sensory input in the cerebral
% cortex. Just as epileptics are susceptible to visualiza-
% tions that strobe at certain frequencies, basilisk hacks
% employ special visual and auditory patterns that trig-
% ger glitches in the brain’s neuronal wiring to inflict

% %%% txt/367.txt
% nausea, vertigo, disorientation, and even seizures,
% often mistaken as a stroke or cerebrovascular incident.
% Some basilisk hacks go farther than simply causing
% the brain to seize up and crash, however, enabling a
% mechanism to rewrite the neural code in victims who
% view or listen to the wrong thing. This unknown re-
% programming mechanism enables the virus to infect
% even a biological brain with one of its strains. Similar
% attacks are used against both synthmorphs and pods,
% taking advantage of the methods in which cyberbrains
% mimic biological minds with a virtual brain state, and
% thus also manipulating them via the information en-
% coded in sensory input.

% In a nutshell, basilisk hacks are a way of hacking
% transhuman brains merely by feeding them a specific
% sample of sensory input, usually images or sounds.
% The widespread use of augmented reality makes
% deployment of such hacks an easy manner; the Ex-
% surgent virus just hacks into the target’s ecto or mesh
% inserts and engages the sensory feed. More traditional
% methods may also be used, including standard interac-
% tive video, holograms, audio, subsonics, or even VR.

% Since so many records of the years surrounding
% the Fall were lost, most people do not know if the
% basilisk hack is anything other than a legend. Various
% official groups know that this technology was, in fact,
% used by the TITANs, but they keep this knowledge to
% themselves, in large part to help reduce the number of
% people attempting to duplicate it.

% INCAPACITATING INPUTS
% When a character experiences a basilisk hack, they
% must make a COG + INT + SAV Test. If this test
% fails, their brain is susceptible to the hack, and they
% immediately suffer 1d10 mental stress. Additionally,
% one of the following effects applies. The duration for
% each effect listed below is 1 minute plus 1 additional
% minute per 10 full points of MoF. Each effect is also
% numbered 1–10, in case the gamemaster wants to roll
% 1d10 and randomize the effects rather than choose:

%  • (1) Cataplexy: The victim loses control of their

% body and immediately collapses. For the duration

% their body will be non-responsive but they will

% be aware and capable of mental actions. Mesh

% actions and implant controls are also disabled,

% however.
%  • (2) Catatonic Stupor: The character becomes

% immobile and non-responsive. Though conscious,

% they are mentally “not there”—the basilisk hack

% has effectively crashed their brain functions. They

% will do absolutely nothing for the duration and

% will not respond even if moved or attacked.
%  • (3) Disorientation: The character becomes disori-

% ented and severely confused. They are incapable

% of making decisions, understanding communica-

% tion, understanding what is going on around

% them, or acting in any sort of determined way for

% the duration.

% • (4–5) Grand Mal Seizures: The subject immedi-

%  ately falls to the ground and begins convulsing,

%  suffering 1d10 damage. They may do nothing

%  else for the duration and will suffer an equal

%  duration period of confusion and weakness (–30

%  to all actions) afterwards.

% • (6–7) Hallucinations: The character immediately

%  goes off on a mental trip, leaving them completely

%  disconnected from reality and their physical body.

%  For the duration, the character should only re-

%  spond to the hallucinated reality the gamemaster

%  describes to them, or else the character should be

%  treated as an NPC, run by the gamemaster.

% • (8) Impaired Cognition: The character’s mental

%  capabilities bottom out, turning them into a

%  disabled vegetable. COG, INT, SAV, and WIL all

%  drop to 1, and the character should act accord-

%  ingly to environmental stimuli.

% • (9) Nausea/Vertigo: The character is overcome

%  with head-spinning and vomiting and is effec-

%  tively incapacitated for the duration.

% • (10) Sleep: The character passes out for the

%  duration and cannot be woken short of medical

%  intervention.


% In rare cases, a character may be able to “dodge”
% a basilisk hack they know is coming, assuming they
% have some sort of warning (such as their buddy fall-
% ing prey to it moments before). The character must of
% course be aware of what basilisk hacks are to even
% consider this idea. If they immediately attempt to take
% action to block out the sensory input when it strikes—
% closing their eyes, plugging their ears, turning off their
% AR, etc.—allow them a REF x 3 Test to see if they do
% so in time.

% SENSORY REPROGRAMMING
% In some cases, the Exsurgent virus can actually repro-
% gram the target’s mind via dedicated sensory input.
% This is a trickier affair, however, requiring uninterrupt-
% ed programming time. As with incapacitating inputs,
% the target character(s) experiencing the basilisk hack
% must make a COG + INT + SAV Test. If this fails, they
% become catatonic and paralyzed for a period of 10
% minutes, minus 1 minute per 10 full points of MoF. At
% the end of this period, they are mentally reprogrammed
% and “infected” with one of the strains of the Exsurgent
% virus (see below). For the duration of this period, the
% character is undergoing reprogramming as long as they
% remain exposed to the basilisk hack. If the character is
% somehow cut off through the actions of another party,
% the reprogramming immediately fails. In this case,
% however, the victim still suffers 1d10 mental stress
% + 1 per minute they were exposed, and they remain
% mentally shaken, suffering a –30 modifier to all actions.
% This modifier reduces at the rate of 10 per minute.

% YGBM ATTACKS
% Rather than completely reprogramming a victim, some
% Exsurgent attacks simply intend to plant subconscious

% %%% txt/368.txt
% commands in the target’s mind, similar to posthyp-
% notic suggestions. Nicknamed “You gotta believe me”
% attacks, YGBMs are a sort of remote digital brain-
% washing attempt used to create sleeper terrorists and
% unknowing collaborators, often by targeting them via
% the mesh. Unlike the mind manipulation techniques
% of psychosurgery (p. 229), YGBM attacks use shotgun
% techniques to open the mind, utilizing some kind of
% backdoor the Exsurgents discovered in the transhu-
% man brain, and altering the mind by brute force.

% A character experiencing a YGBM basilisk hack
% must make a COG + INT + SAV Test. If this fails, a
% single suggestion is implanted in the character’s mind,
% without their knowledge. This subliminal command
% will be triggered at some later point, either at some
% predesignated time or according to certain pre-set
% conditions. Once triggered, the character will carry
% out the action with all of the conviction that it is their
% own idea. The implanted suggestion may be something
% as simple as “kill the Firewall agent” to something as
% complex as “manufacture an explosive device and
% plant it in the cargo hold of any ship heading to Mars,
% set to explode one day after they disembark.”

% Since YGBM attacks are not intended to completely
% convert the target, but instead to simply convert them
% into a temporary tool or weapon, implanted com-
% mands are not designed to last long. The duration
% the suggestion will last equals 3 days +1 day per 10
% points of MoF on the resistance test. If the command
% has not been triggered by this point, it dissipates, and
% the character is none the wiser.

% RECORDING BASILISK HACKS
% Enterprising characters may seek to record a basilisk
% hack input for their own uses. While basilisk hacks may
% be recorded like any other sensory input, keep in mind
% that the Exsurgents and TITANs likely take measures
% to keep such tools out of the hands of transhumanity,
% lest they construct some sort of defense. Basilisk hack
% sources may be self-erasing or contain coding or coun-
% termeasures that would hinder recording, such as white
% noise to defeat audio recording or lens-blinding flashes
% to defeat video recording. Conversely, basilisk hacks are
% considered extremely dangerous by almost all factions
% of transhumanity and universally feared. An individual
% or group known to possess them is likely to be treated
% much like a terrorist with a suitcase nuke. Though
% Firewall has a standard interest in evaluating and en-
% abling some sort of defense against basilisk hacks, most
% Firewall personnel consider it foolish to handle such
% toys and would rather destroy such recordings outright.

% EXSURGENT STRAINS
% Four variants of the Exsurgent virus are described
% here—gamemasters are encouraged to develop their
% own to keep players on their toes.

% HAUNTING VIRUS
% This strain is the most insidious of the Exsurgent virii.
% Over time, it rewrites the target’s personality and
% motivations, slowly but surely subverting and taking
% control of the victim’s mind. At first the character is
% unlikely to even be aware of the infection, and as it
% progresses the changes the virus makes to the target
% will at first seem natural to the target, as if some new
% aspect of their personality was simply manifesting
% itself. As the effects grow more pronounced, however,
% the victim becomes aware that they are being methodi-
% cally altered but is in most cases unable to act against
% it. In the end, they are completely transformed into a
% pawn of the ETI. Their mind is no longer transhuman,
% but alien.

% The exact rate of progression is up to the gamemas-
% ter, though guidelines are provided below. Each victim
% is affected differently, so the process may be accelerated
% or slowed down as the gamemaster sees fit.

% %%% txt/369.txt
%  • Stage 1 (initial infection to 3 months): Upon ini-

% tial infection, the character suffers 1d10 mental

% stress and gains the Psi trait (p. 147) at Level 1

% (also meaning they pick up the Mental Disorder

% trait, as noted on p. 150). They also gain one free

% psi-chi sleight, chosen randomly or by the game-

% master. If a player character has become infected,

% they may still be played as normal (see Roleplay-

% ing Exsurgents, p. 368), and may purchase new

% psi-chi sleights with Rez Points. NPCs acquire 1

% new sleight per 2–4 weeks.

% At this stage, the infection is usually hidden,

% though the character will suffer from occasional

% haunting effects (see below). As each week passes,

% the character’s personality should shift a minute

% amount, slowly becoming more callous and con-

% niving and changing in other ways as well. If

% possible, the player should be kept in the dark

% about what is happening, but the gamemaster

% should provide them with roleplaying advice to

% reflect their condition. Likewise, the discovery

% and initial use of psi sleights should be played

% out, providing some interesting roleplaying op-

% portunities. Characters and players who know of

% the Exsurgent virus and Watts-Macleod strains

% should not know at this point which strain they

% are infected with—make them sweat.
%  • Stage 2 (3 months to 6 months): The target suf-

% fers another 1d10 ÷ 2 (round up) mental stress

% and acquires the Psi trait at Level 2 (also pick-

% ing up another disorder). Player characters may

% still be played as normal and may purchase psi-

% gamma slights with Rez Points. NPCs acquire 1

% new sleight per 2–4 weeks.

% Once three months have passed, the character

% should be aware they are under the influence

% of something, but this awareness likely comes

% too late. Haunting effects (below) should occur

% regularly. At this point a character is likely to

% consider offing themselves and resorting to an

% uninfected backup, seeking help, or actively en-

% couraging others to interfere. The infection will

% actively block and hinder such thoughts and ac-

% tions, however. To actively overcome this mental

% control, the character must succeed in a WIL Test.

% At the gamemaster’s discretion, failure may result

% in 1d10 ÷ 2 (round up) mental stress as the char-

% acter realizes they are no longer fully in control

% of their own thoughts and actions.
%  • Stage 3 (6 months+): The victim suffers another

% 1d10 ÷ 2 (round up) mental stress and acquires

% the Psi trait at Level 3 (see below). The character

% is now considered an exsurgent and becomes

% an NPC. It may no longer be played as a player

% character. The victim also gains a permanent

% +5 bonus to COG and WIL and acquires 1 new

% sleight every 1–2 months.


% As noted above, characters infected with this strain
% suffer from different haunting effects—changes
% to their personality or mind-state. A few ideas for
% haunting effects are noted here, but gamemasters are
% encouraged to be creative when inventing their own
% to apply:


% • Altered Perceptions: The victim’s perceptions are

%  changed in disturbing and unusual ways. They

%  may see things that aren’t there, feel a presence

%  behind or watching them, inexplicably smell

%  blood, hear voices, suffer synaesthesia, or sud-

%  denly perceive the people around them as nothing

%  but outlandish, blabbering sacks of meat.

% • Behavioral Modification: Treat as behavioral con-

%  trol or personality editing psychosurgery (p. 231).

%  This is typically applied to shape the character

%  closer to being a pawn of the ETI.

% • Dream Manipulation: The character’s dreams

%  become lucid, weird, and surreal. They may find

%  themselves dreaming of life as an alien on some

%  exotic exoplanet, as a robotic probe soaring

%  through the vast emptiness of space, or fantasiz-

%  ing different methods of inflicting mass destruc-

%  tion and death.

% • Emotional Manipulation: Treat as emotional

%  control psychosurgery (p. 231).

% • Inexplicable Urges: The character will be flushed

%  with strange alien urges and may sometimes find

%  themselves doing highly unusual things without

%  realizing at all they are doing it. These may in-

%  clude taking devices apart to understand how

%  they work, testing the limits on programming a

%  nanofabricator, cutting a living thing apart to see

%  how it is put together biologically, testing weap-

%  ons, eating things that are only barely edible, pro-

%  miscuous and unusual sexual activity, lying just

%  to see what they can get away with, and so on.

% MINDSTEALER VIRUS
% Very similar to the haunting virus, the mindstealer
%  strain is much quicker acting. Instead of slowly sub-
% verting the target’s mind over the course of months,
% the mindstealer virus rapidly recodes the victim’s brain
% in a matter of minutes. This infection is much more
% invasive and brute-force, often causing significant
%  side effects to the target’s mental state as a result. This
%  strain is only spread as a digital virus, nanoplague, or
% basilisk hack (not as a biological nanovirus).

%  Once the victim is infected, it takes the virus a
% number of Action Turns equal to COG + INT + SAV
% to completely take over their mind (20 Action Turns
% = 1 minute). During this time, the target is actively
% aware that their mind is under attack and undergo-
% ing massive changes against their will. This process
% is confusing, frightening, and painful, inflicting a
% –30 modifier to all of the character’s actions for the
% duration. Many victims are reduced to whimpering,
% drooling, or convulsing for the duration.

% This mental transformation inflicts 2d10 mental
%  stress to the target. Once complete, the victim is an
%  exsurgent NPC, under the gamemaster’s control.

% %%% txt/370.txt
% WATTS-MACLEOD VIRUS
% The Watts-Macleod strain is a strangely benevolent
% version of the Exsurgent virus, seeming to imbue its
% victims with psi abilities without any of the other
% transformative elements typical of other strains.
% Perhaps created as an accidental mutation of the Ex-
% surgent virus, there are many who wonder if the true
% detrimental effects of this strain simply have yet to
% reveal themselves.

% As noted in the Mind Hacks chapter section on Psi
% (p. 220), characters infected with this strain gain the
% Psi trait (p. 147) at either Level 1 or 2. If a character
% is so infected during game play, this trait must be
% purchased with Rez Points (if the character does not
% have any points currently available, they pay out of
% the points they earn until the debt is paid off). All of
% the other side effects of Watts-Macleod infection (p.
% 367) also apply.

% Though infection with this strain does apply some
% benefits to the character, the gamemaster should make
% sure to play up the creepy and unsettling nature of
% this virus. The character should never be certain that
% they haven’t in fact been subtly influenced by the virus
% in ways they can’t immediately pinpoint—they should
% always feel like the ax may fall at any moment.

% XENOMORPH VIRUS
% The xenomorph strain transforms the target’s body
% in addition to their mind. Over time, the victims
% morph physically transmogrifies into some sort
% of alien life form. It is only spread as a biological
% nanovirus or nanoplague (not as a digital virus or
% basilisk hack). Different variants of this strain pro-
% duce different alien forms. It is not known where
% these different alien templates originated, meaning
% they may be copies of (once) existing alien species
% or simply neogenetic creatures created from scratch.
% The one trait they have in common is that they are
% universally dangerous. Some speculation in Firewall
% circles suggests that the Exsurgent virus may in fact
% have a “library” of creature types to deploy, under
% the assumption that at least some will be more effec-
% tive than others for exterminating whatever victim
% species they are fielded against.

% This strain follows the same rules as the haunting
% virus (above), but with the following changes. The
% timeframe is typically much quicker, though the game-
% master may adjust this as they see fit.

% Stage 1: The effects from Stage 1 of the haunting
% virus apply. Additionally, the character begins to suffer
% minor physical changes that are definitely unusual but
% are not impeding in any way and are easily hidden
% from others. Example biomorph alterations might be:
% unusual hair or fibrous growth, some skin discolor-
% ation or translucence, severe rashes, dermal thicken-
% ing, weakened or enhanced sensory organs, strong
% body odor, hair loss, teeth gain or loss, vestigial tail or
% other limb growth, minor dietary changes, and so on.
% Synthmorphs might experience minor system glitches,
% malfunctioning or improved components, and spots
% of material stress or transfiguration. Gamemasters
% are encouraged to be creative. This stage typically
% lasts from initial infection to 1 week for biological
% nanovirus strains, or from infection to just 1 hour for
% nanoplague strains.

% Stage 2: As with haunting virus Stage 2, plus the
% character begins to seriously transmogrify in ways
% that are diffi cult to hide from others, becoming
% more and more monstrous as the stage progresses.
% Example biomorph transformations include: grow-
% ing scales or feathers, partial modification of limb
% structure, partial new limb growth, vestigial sensory
% organ growth, sensory loss, extension of claws or
% spines, severe dietary changes, etc. Synthmorphs
% might experience radical system and shape altera-
% tions, limited or enhanced sensor functions, or even
% conversion of their robotic shell to smart materials.
% These physical changes weaken the victim, inflicting
% 1d10 physical damage. This stage typically lasts 1
% week for biological nanovirus strains or just 1 hour
% for nanoplague strains.

% Stage 3: As with haunting virus Stage 3, a character
% reaching this stage becomes an NPC. Additionally,
% the victim completely undergoes a transformation
% into some sort of creature that is no longer even re-
% motely human. Example exsurgents of this nature are
% detailed on p. 369.

% USING THE EXSURGENT VIRUS
% The frightening thing about the Exsurgent virus is
% its adaptability. It was written by a near omnipotent
% ETI with the intent of corrupting any alien seed AIs
% or similar singularities it encountered, and it is very
% good at it. This means it has the capability to ana-
% lyze, understand, and mimic almost any alien digital
% protocols and communication methods it comes into
% contact with, no matter how diverse the alien mindset
% that constructed what it encounters. It then has a cun-
% ning ability to circumvent any safeguards and infect
% such systems. From there, it rapidly assimilates any
% data it can about the target species/civilization and
% does it best to mutate into other forms that can attack
% this target from other vectors.

% Given its constant morphing nature then, the Ex-
% surgent virus is likely to continue to mutate in new
% and interesting ways. Some of these mutations may
% be effective, many not. This does, however, afford the
% gamemaster an opportunity to invent new variants of
% their own to deploy against unsuspecting characters.

% ROLEPLAYING EXSURGENTS
% The primary thing for gamemasters to keep in mind
% when roleplaying entities that have been taken over
% by the Exsurgent virus is that exsurgents are follow-
% ing an alien agenda. The specific goals and actions of
% each exsurgent may differ, but they are generally con-
% cerned with two things: spreading the Exsurgent virus
% and destroying anything that isn’t affected. In some
% cases, this may mean immediate and enraged hostile
% action against anything non-exsurgent around them.

% %%% txt/371.txt
% In others, the exsurgent approach is more methodi-
% cal, hatching long-term plots to infiltrate positions of
% power and authority, setting the stage for acts of mass
% destruction, and so on. In other words, they may be
% handled both as hostile monsters or as nefarious long-
% term opponents that are subverting transhumanity
% from within or hatching complicated plots that could
% mean devastation on a planetary scale.

% If the gamemaster wishes, exsurgents may also
% pursue other goals, tangential to the ones above.
% These may range from accumulating knowledge and
% expertise on how transhumanity functions as a spe-
% cies to forcibly uploading mass numbers of minds to
% more esoteric goals such as manufacturing a halfnium
% bomb or converting the solar system’s mass to compu-
% tronium. The Exsurgent virus is potent and intelligent,
% and while its methods and goals may sometimes be
% opaque to transhumanity, it acts with direction and
% purpose. There may also be occasions, however, likely
% due to the mutating and morphing aspect of the virus
% and the way in which it transforms transhuman
% minds, perhaps not always in the manner intended,
% where the exsurgent goals become strange or simply
% horrific, such as running experiments on transhuman
% responses to extreme conditions or converting an
% entire colony to cannibalism.

% EXSURGENT-INFECTED PCS
% It is possible for player characters infected with some
% strains of the Exsurgent virus to continue on under
% their own volition, even as the virus slowly consumes
% them. This process is, quite naturally, horrifying in the
% extreme, though there is little they can do about it.
% Despite the best efforts of transhuman science, there
% is very little that can be done to save an infected
% person—the virus is simply too potent and adaptive.
% As a result, standard Firewall policy is to terminate
% the infected with extreme prejudice. Most Firewall
% operatives are going to be aware of this, a fact which
% pushes some of those who become infected to keep
% their status a secret from their comrades.

% Both the haunting and xenomorph strains usually
% transform a subject over time, meaning that the
% character may initially not be aware of the infection.
% This is a prime opportunity for the gamemaster to
% mess with the character ruthlessly, starting slowly
% with little haunting effects and building up as the
% infection progresses. The character should slowly
% become aware that they are under the influence of
% something—something intelligent. Characters aware
% of the Exsurgent virus and its effects will likely pick
% up on this sooner, but the virus may prevent them
% from doing anything about it. In effect, the character
% becomes a prisoner within their own body, a body
% they now share with a cold and malevolent pres-
% ence that is methodically taking them over. Such
% characters may respond in a number of ways de-
% pending on their personality, ranging from despair,
% withdrawal, and suicidal tendencies to complete
% hysteria or calm acceptance. Most importantly,
% however, their personality should begin to change
% as the virus continues to transform them. Players
% should be encouraged to take on new demeanors
% and motivations, refl ecting the alien component
% of their changing personality, with some guidance
% from the gamemaster. This presents some intrigu-
% ing roleplaying opportunities that the players will
% hopefully embrace. If the gamemaster feels that the
% player is not adequately representing the changing
% mindset, however, the transformation can simply
% be accelerated and the character converted into a
% gamemaster-operated NPC.

% EXSURGENTS
% A few examples of exsurgents created from tran-
% shumans transformed by the xenomorph strain of
% the virus are noted below. As always, gamemasters
% are encouraged to develop their own, using these as
% guidelines. Unless otherwise noted, exsurgents use the
% stats and skills of the transformed character. Each ex-
% surgent detailed below first lists the aptitude modifiers
% applied to transformed characters, then gives example
% aptitude/skill ratings for NPC exsurgents.

% Note that simply encountering transformed exsur-
% gents is stressful to the minds of many transhumans.
% At the gamemaster’s discretion, such encounters may
% inflict 1d10 + 3 mental stress (p. 215).

% CREEPERS (SYNTHMORPHS)
% Perhaps the most disturbing exsurgent variant, so-
% called creepers are cloud-like amorphous swarms of
% small, black bubbles that are strangely fuzzily defined,
% as if surrounded by some sort of visual refraction effect.
% These clouds are theorized to in fact be autonomous
% femtobot swarms—similar to nanobots, but affecting
% matter on an even smaller scale, at the level of an atomic
% nucleus. These black bubbles are capable of coalescing
% into physical shapes in various states and can penetrate
% just about any material or substance in a matter of
% Action Turns. They may even penetrate morphs, access-
% ing and interfacing with neural and electronic systems
% directly. For rules purposes, treat creepers the same as a
% self-replicating nanoswarm (p. 383).
%  COG       COO       INT       REF     SAV      SOM       WIL     MOX
%  +5 (20)   — (15)   +5 (20)   + (30)


%                           10      — (15)   — (15)   + (30)


%                                                      10       —


% INIT     SPD       LUC       TT       IR      DUR       WT      DR

% 100       2         —        —        —       100       20      200
% Mobility System: Walker/Microlight (4/16) (may

% create other mobility systems with different rates)
% Skills: Fray 40, Free Fall 50, Intimidation 60, Percep-

% tion 50, Unarmed Combat (Grapple) 50 (60)
% Notes: 360-degree Vision, Chemical Sniffer, Electrical

% Sense, Enhanced Hearing, Enhanced Vision, Frac-

% tal Digits, Nanoscopic Vision, Radar, Radiation

% Sense, Swarm Composition (but may make SOM

% Tests, and plasma weapons do only 1d10 damage),

% T-ray Emitter

% %%% txt/372.txt
% JELLIES (BIOMORPH)
% These exsurgents resemble collections of massive, slimy,
% mucus-filled bubbles. Their soft, amorphous shape
% allows jellies to squeeze, slide, and slither through
% even tiny spaces. Jellies are equipped with a number
% of “limbs” that resemble long meaty tongues studded
% with hard fleshy spikes that provide excellent gripping
% ability. The lubricating coating that envelopes jellies is
% both toxic and slightly corrosive, melting plastics and
% biological materials after a half hour of exposure. This
% substance may also be “spit” at targets.
%  COG        COO        INT     REF     SAV      SOM        WIL       MOX
% +10 (30)   –5 (10)   +10 (30) — (15)   — (15)   +5 (20)   +10 (30)    —


% INIT      SPD       LUC      TT       IR      DUR         WT       DR

% 90        1         —       —        —        70         14       105
% Movement Rate: 4/16
% Skills: Exotic Ranged Attack (Spit) 40, Free Fall 50,

% Perception 60, Unarmed Combat 40
% Notes: Armor (12/12), Enhanced Smell, Spit Attack

% (area effect), Tongue (DV 1d10 + 3, AP 0), Toxin

% (Application: D, O; Onset Time: 1 Action Turn, Du-

% ration: 5 Action Turns, Effect: 1d10 ÷ 2 (round up)

% DV per Action Turn ). Due to their physical form,

% jellies suffer the minimum amount of damage from

% standard kinetic weapon and blade attacks.

% SHIFTERS (SYNTHMORPH)
% Shifters are synthmorphs whose material frames have
% been converted to an exotic smart matter liquid metal.
% This shapeshifting material can stabilize as a hardened
% metallic shell or liquefy and reshape itself into other
% forms. This allows the shifter to reflow its shell in a
% matter of seconds, enabling it to visually mimic other
% forms, including biomorphs (though they are easily
% detectable as synthmorphs at other wavelengths or by
% touch). Shifters may also reshape parts of their shell
% into melee weapons such as knives or clubs.

% COG       COO       INT     REF     SAV     SOM      WIL           MOX
%  +5 (20)   +5 (30)   — (20) +10 (30) +5 (20) +10 (30) +10 (30)        —


% INIT      SPD       LUC      TT       IR      DUR         WT       DR

% 100        2         —       —        —        60         12       120
% Mobility System: Walker (4/20)
% Skills: Blades 60, Deception 55, Disguise 60, Fray 50,

% Freerunning 55, Impersonation 60, Perception 50,

% Unarmed Combat 50
% Notes: Armor (13/13), Enhanced Hearing, Enhanced

% Vision, Shape-Adjusting (Programmable Liquid

% Metal Form)

% SNAPPERS (SYNTHMORPHS)
% Snapper exsurgents are typically crafted from vehicles
% or other large synthetic shells or by melding multiple
% synthmorphs together. They take the form of an in-
% sectoid multi-segmented hexagonal tube with multiple
% sets of limbs, three apiece, set radially 120 degrees
% around the torso. These limbs are heavy, double-
% jointed, and articulated with three joints. Each limb
% ends in either a triad of manipulatory digits or a larger
% pincer-like claw.

% COG       COO        INT     REF     SAV     SOM      WIL         MOX
%  +5 (20)   +5 (30)    — (20) +10 (30) +5 (20) +10 (35) +10 (30)      —


% INIT      SPD       LUC       TT       IR     DUR        WT       DR

% 100        2         —        —        —       70        14       140
% Mobility System: Walker (4/24)
% Skills: Climbing 45, Fray 40, Freerunning 40, Percep-

% tion 40, Unarmed Combat (Pincers) 55 (65)
% Notes: 360-degree Vision, Armor (16/16), Enhanced

% Vision, Extra Limbs (9, 12, or 15 total), Lidar,

% Magnetic System, Pincers (DV 2d10 + 3, AP –3),

% Structural Enhancement

% WHIPPERS (BIOMORPH)
% These small barrel-shaped creatures have a mass
% of small legs under their trunk that allows for fast
% movement. At the top of their trunk is another mass
% of 3-meter long, strong, whip-like tentacles. Some
% of these tentacles feature gripping surfaces for grab-
% bing and holding (both for tool use and mobility),
% while others are sharp-edged and useful for slicing
% through opponents.

% COG       COO        INT      REF    SAV      SOM        WIL      MOX
%  +5 (20)   +10 (30)   +5 (20) +10 (30) — (15)   +5 (25)   +5 (20)    —


% INIT      SPD       LUC       TT       IR     DUR        WT       DR

% 100        2         —        —        —       35         7       53
% Movement Rate: 8/40
% Skills: Climbing 40, Fray 50, Free Fall 40, Freerunning 50,

% Infiltration 40, Perception 50, Unarmed Combat

% (Tentacles) 45 (55)
% Notes: Enhanced Vision, Tentacle Whip (DV 2d10 + 1,

% AP –1)

% WRAPPERS (BIOMORPH)
% These exsurgents resemble large, thin, four-armed,
% spiny starfish, capable of walking in a quadruped
% manner, though they are seemingly better adapted for
% microgravity. A large circular mouth resides in their
% middle on one side and each arm ends in small sharp-
% clawed digits, useful for climbing and tool use. Small
% vent sacs allow for thrusting in microgravity and
% sensory bands on the upper part of each arm provide
% low-frequency hearing and infrared-equivalent sens-
% ing. Their name comes from their tendency to drop on
% opponents from above, wrapping themselves around
% the head and arms.

% COG       COO        INT      REF     SAV    SOM      WIL         MOX
%  +5 (20)   +5 (20)    +5 (20) +10 (30) — (10) +10 (30) +10 (30)      —


% INIT      SPD       LUC       TT       IR     DUR        WT       DR

% 100        1         —        —        —       45         9       68
% Movement Rate: 4/16
% Skills: Fray 40, Free Fall 50, Perception 50, Unarmed

% Combat (Grapple) 50 (60)
% Notes: Armor (8/8), Bite (DV 2d10 + 3, AP –5, must

% grapple first), Chameleon Skin, Claws (DV 1d10

% + 2, AP –2), Enhanced Hearing, Infrared Sensing,

% Vacuum Sealing

% %%% txt/373.txt
% EXSURGENT PSI
% In addition to psi-chi and psi-gamma (see Psi, p. 220),
% exsurgents have access to a third level of psi ability
% (the Psi trait at Level 3), known as psi-epsilon. Psi-
% epsilon is theorized to allow a level of interaction with
% the underlying physics of reality that is beyond the
% comprehension of transhuman science. Though some
% Firewall scientists have speculated about the ma-
% nipulation of dark energy or the Higgs field and Higgs
% boson particles and similar exotic ideas, the truth is
% that psi epsilon represents an understanding of science
% so far advanced and so alien that transhumanity can
% only guess at its mechanics.

% EXSURGENT SYNTHMORPHS AND PSI
% Exsurgents in synthetic morphs that were transformed
% via nanoplague may use psi, despite lacking a biologi-
% cal brain. Through some unknown mechanism, the
% infecting nanobots are able to simulate a biological
% brain’s effects. This feature, however, also makes them
% vulnerable to psi use by others.

% EXSURGENT PSI STRAIN
% Exsurgents with Level 3 psi (psi-gamma) do not suffer
% strain when using psi. Instead, they draw requisite
% energy from the environment around them. In game
% terms, this means that gamemasters do not need to
% worry about rolling strain for exsurgent sleights. On a
% cinematic level, it also allows the gamemaster to add cre-
% ative environmental details to exsurgent psi use: sucking
% the warmth out of the air, killing the lights, withering
% plants, draining power from nearby electronics, killing
% small creatures or insects, lowering air pressure, etc.

% EXSURGENT PSI-GAMMA SLEIGHTS
% These sleights are available to exsurgents with the
% Level 2 Psi trait.

% DECEREBRATION

% PSI TYPE:    Active        ACTION:     Complex

% RANGE:       Touch         DURATION:   Temp (Action Turns)

%  STRAIN MOD: +2            SKILL:    Psi Assault
% This sleight temporarily “shorts out” a portion of the
% subject’s brain stem. The victim’s cerebral functions
% and motor activity become severely impaired; apply
% a –30 modifier to all actions. If an Excellent Suc-
% cess is scored, the target effectively loses all cerebral
% functioning, including vision, hearing, other sensory
% functions, and mesh use. Their muscles and limbs also
% tense and become rigid, essentially paralyzing them in
% what looks like an agonized state.

% ONSLAUGHT

% PSI TYPE:    Active        ACTION:     Complex

% RANGE:       Touch         DURATION:   Temp (Action Turns)

% STRAIN MOD: +0            SKILL:   Psi Assault

% This offensive sleight floods the target’s mind with
% sensory input and thought processes that are so alien
% and disturbing that they inflict 1d10 + (WIL ÷ 10,
% round up) mental stress. Increase the stress by +5 if an
% Excellent Success is scored.

% SCENARIO

% PSI TYPE:    Active       ACTION:     Complex

% RANGE:       Touch        DURATION:   Sustained

%  STRAIN MOD: +2            SKILL:    Control
% This sleight hijacks the target’s sensorium, replacing
% it with a virtual scenario controlled by the exsurgent.
% The effect is much like being jacked into a simulspace
% scenario, albeit against the target’s will. While the
% exsurgent cannot harm the target in the scenario, they
% can learn something about the person’s behavioral
% responses to certain situations. While under the in-
% fluence of this sleight, the target is cut off from their
% physical senses (–60 to any Perception Tests), but they
% may flail about and otherwise respond physically to
% events in the scenario, which may cause them to hurt
% themselves and will make them seem crazy to onlook-
% ers. Targets may attempt to ignore the scenario and
% concentrate on the real world, but this requires a WIL
% Test each Action Turn and they suffer a –30 modifier
% from disorientation even if they succeed.

% STRIP MEMORY/SKILL

% PSI TYPE:    Active       ACTION:     Complex

% RANGE:       Touch        DURATION:   Temp (Hours)

%  STRAIN MOD:     +2        SKILL:    Psi Assault
% Strip allows the exsurgent to suppress certain memo-
% ries in the target’s mind. This can be used to block
% memories of certain events or even the victim’s iden-
% tity. The process is not exact, however, and so the
% memories may not be fully suppressed and/or related
% memories may also be blocked; the gamemaster de-
% cides on the effect as determined by the MoS. Strip can
% also be used to temporarily erase a specific skill from
% the target’s mind, preventing them from using or even
% defaulting to that ability while so affected.

% EXSURGENT PSI-EPSILON SLEIGHTS
% Psi-epsilon is available to exsurgents with the Psi trait
% at Level 3. This subset of psi involves abilities that can
% affect the underlying physical nature of the universe,
% creating localized reality-altering effects. Psi manipu-
% lation on this level is extremely dangerous and should
% have the potential of disastrous consequences, given
% that these manipulations violate fundamental laws of
% nature and sometimes create paradoxes between the
% forces that glue the universe together. Gamemasters
% are also encouraged to treat critical failures as appro-
% priately critical.

% Given these factors, psi-epsilon should only be
% accessible to powerful adversaries and used as a
% gamemaster tool with extreme precaution. The exact
% mechanics of psi-epsilon sleights are left wide-open,
% however, for whatever use the gamemaster can

% %%% txt/374.txt
% dream of. Their intent is to be more cinematic than
% mechanical, so gamemasters should wing rules effects
% as needed. This is an open opportunity for the game-
% master to create nightmarish monsters from beyond
% with frightening reality-ripping and mind-scarring
% abilities. While some example sleights are provided
% below, gamemasters are encouraged to modify their
% effects and to create their own.

% At the gamemaster’s discretion, simply observing psi
% epsilon sleights in action may inflict 1d10 + 2 mental
% stress on a character (p. 215).

% ANTI-ELECTRONICS FIELD
% All electronics within Close range of the exsurgent
% mysteriously fail as if electrical power is simply ne-
% gated. This effectively disables synthmorphs and pods
% and leaves other characters without access to their
% devices or implants.

% CASIMIR FORCE REPULSION
% This sleight exploits the Casimir effect (an interaction
% between the electromagnetic fields of different objects)
% on a macro-scale, allowing the exsurgent to levitate
% themself or other objects by creating repulsing fields.
% This could also allow the exsurgent to push targets
% away, pin them against walls, etc.

% CRYOKINESIS
% This sleight allows the exsurgent to drain all heat from
% an area, down to absolute zero, effectively freezing
% everything within range and inflicting cold damage on
% unprotected characters.

% DIFFUSION
% This sleight diffuses light, laser, and particle beams, ef-
% fectively making them useless as weapons, or at least
% impairing the DV they inflict.

% KINETIC FRICTION
% The exsurgent uses this sleight to increase the fric-
% tion applied to kinetic activities. This has a negligible
% effect on most activities, but high-velocity projectiles
% like firearms and railguns will be significantly slowed,
% decreasing their DV by half or more.

% MATTER TRANSFORMATION
% This sleight alters the molecular bonds and atomic
% components of a targeted material, causing it to either
% weaken and deteriorate or transmutate into some
% other physical substance. This can also be used to
% alter the molecular state of a material, causing gases
% to condense, solids to liquefy, etc. An exsurgent could
% use this to weaken a door or other barrier, condense a
% solid bridge out of liquid, petrify organic materials, etc.

% NEGATIVE REFRACTION
% The exsurgent redirects electromagnetic waves with
% this sleight, refracting them around their body, with
% the same effect as the invisibility cloak (p. 316).
% PYROKINESIS
% Similar to cryokinesis, this sleight enables the exsur-
% gent to accelerate the molecules, increase friction, or
% focus heat in a specific area, causing materials to ignite
% or smolder.



% THE FACTORS
% The alien species known as the Factors are unlike
% anything mankind has encountered so far (see First
% Contact: The Factors, p. 40). Though they are aloof
% and stand-offish, their willingness and sometimes ea-
% gerness to deal with (parts of) transhumanity indicate
% either a keen interest on their part in transhuman
% affairs or some hidden ulterior agenda. Though the
% various transhuman factions have been similarly wary
% and cautious, and despite numerous communications
% difficulties and failures, an uneasy relationship has
% flowered over the past 8 years, facilitating some trade
% and exchange of knowledge.

% ORIGIN AND EVOLUTION
% The Factors have remained notoriously tight-lipped
% about their origins, history, and the location of their
% homeworld. Though they have also paid visits to some
% of transhumanity’s exoplanet colonies, no gatecrashing
% expeditions have yet found any sign of Factor habitation
% or passing elsewhere in the galaxy. Repeated inquiries
% by transhuman mediators have been simply ignored or
% answered in cryptic terms that have yet to be deciphered.

% The Factor home world is in fact an Earth-like
% planet with comparable atmospheric conditions and
% a prevalent hydrosphere but with longer periods of
% darkness (due to slower rotation of the planet and a
% less-luminous orange giant). While adapted transhu-
% mans could find their planet habitable, their abiogen-
% esis (the formation of life from self-replicating, but
% not-living molecules) took a different route than life
% on Earth.

% The Factors’ primordial ancestors began in their
% planet’s early geological history as a type of of pho-
% tosynthesizer that ate carbon dioxide and water and
% released oxygen, also obtaining energy from inorganic
% chemicals like hydrogen sulfide. Long conditions
% without direct light on their homeworld, however,
% spurred the success of organisms that could survive
% by acquiring energy in other ways. The next evolu-
% tionary leap was to a stage similar to Terran slime
% molds, eating microorganisms from decaying matter.
% As evolution progressed, they mutated further into a
% cautious, predatory species that fed on larger, danger-
% ous creatures. Rather than actively hunting such prey,
% this species developed versatile methods of capturing
% and immobilizing their competitors (comparable to
% Earth’s funnel web or trapdoor spiders). Over time,
% this method of trapping prey spurred basic (practical)
% intelligence and provided them with the evolutionary
% advantage that paved the way to sapience, driving
% Factors to become the highest developed organisms
% on their planet and build a civilization.

% %%% txt/375.txt

% Like mankind, the Factors suffered through and
% survived their own singularity event and encounter
% with the Exsurgent virus. Perhaps due to their cau-
% tious and calculating nature—and their evolutionary
% experience in dealing with more powerful and danger-
% ous opponents—the Factors are resolutely determined
% not to make any similar mistakes as a species.

% XENOBIOLOGY
% Since life on the Factors’ home world developed
% differently than Earth and produced neither nucleic
% acids nor amino acids, Factor metabolic processes
% and “genetics” are very different from transhuman-
% ity’s. While little is known about the exact physiology
% of the Factors, due to the lack of captured or dead
% specimens to investigate (so far, no hypercorps or fac-
% tions have risked an interstellar incident by abducting
% one to dissect … so far) and their unwillingness to be
% examined by transhumans, most common knowledge
% about them is based on observational and forensics
% research during their encounters with transhumanity.

% INDIVIDUAL FACTORS
%  Individual Factors resemble non-translucent ambu-
%  latory amoeba, slime molds, or slugs. Though they
% “stand” only 0.3 meters tall, their body diameter
%  ranges from 1.5 to 2 meters, they can be up to 2
%  meters long, and they can shape their body to change
%  these dimensions. Instead of walking, they crawl or
%  ooze from place to place by protruding finger-like
%  structures (so called pseudopodia) that attach to the
%  ground (or wall or ceiling) and which they use to pull
% and retract their rear forward (similar to cell migra-
% tion). Due to their malleable shape they are not as
% strongly affected by gravity as transhumans.

% Most Factors that have been encountered are dull
% ocher in color and are made from a gooey, gel-like
% substance of unknown composition, though yellow
% glistening patches (which are temporary organelles)
% and bundles of fibers (some kind of muscular skel-
% eton) often become visible when they move. While
% all Factors are able to express versatile pseudopodia
% to manipulate and operate devices (and even attack),
% some subspecies possess, carry, or are able to develop
% additional differentiated limbs, cilia, or organs with
% specialized functions.

% FACTOR COLONIES
% Unlike transhumans, Factors rarely act individually—
% in fact, individuality is a concept somewhat foreign
% to Factors. Most Factors join together into a collec-
% tive unit termed a colony. A typical Factor colony is
% composed of hundreds or thousands of individual
% Factors that literally physically join together into a
% mass organism (resembling more a primordial soup
% than a gargantuan Factor). Individual Factors are
% indistinguishable from each other when merged into
% the supra-structure of the colony, though individuals
% can form and break apart to accomplish different
% tasks. This colonial merging is mainly possible due
% to the fact that Factors don’t possess differentiated
% and specialized organs or cell types that need to be
% segregated from each other, but instead use an open
% system of local, temporary gradients for regulation.

% %%% txt/376.txt
% Neurofilament connections effectively allow the
% Factor colony to operate with a group mind-state,
% with supercomputer potential. This also allows for the
% easy transfer of knowledge and memories to all other
% factors within a colony.

% If dismembered, blown apart, or otherwise sepa-
% rated, individual Factors in a colony can regenerate
% and reconstitute at a rapid rate without loss of ability
% or memory.

% Factors reproduce when different members of the
% colony produce gametes that fuse, grow into spore
% stalks, and emit spores that later hatch and grow clones.

% BIODIVERSITY AND SELF-DESIGN
% Factors colonies are known for their high biodiversity,
% featuring numerous sub-groups (so called phenotypes)
% that each have unique traits (cilia, apocrine glands, car-
% apace-like outer membrane) that give them an ecologi-
% cal advantage or a utilitarian aptitude for certain tasks.
% These traits are not random evolutionary features, but
% are the result of intentional bio-engineering. The Fac-
% tors have a strong grip on their own metabolisms and
% genetic expressions and can draw on an array of ge-
% netic building blocks and biotech techniques to modify
% themselves rapidly and massively to adapt to special
% conditions. Whether these modifications might have a
% purpose beyond function, such as for reproduction or
% self-expression, is currently unknown.

% METABOLISM
% Factors ships and habitats have transhuman-friendly
% atmospheres with a slightly higher content of carbon
% dioxide and less nitrogen that mimics the conditions
% on the Factors’ home planet. They don’t breathe
% oxygen via lungs but absorb it via their outer “skin.”
% Since they can also use oxygen from other sources
% (minerals, liquids like water, and salts) to fuel their
% aerobic energy production (i.e., respiration), they can
% be considered functional anaerobes, meaning they can
% survive in environments without atmosphere, though
% they must usually supply themselves with food in
% order to do so.

% During the few ceremonial festivities to which
% Factors were invited and actually attended, they
% consumed and processed transhuman organic food
% by internalization. On the first occasion, dishes and
% dinnerware were absorbed as well due to misunder-
% standing, but were excreted unharmed after the or-
% ganic components the factors could utilize had been
% broken down.

% While Factors are omnivores similar to transhumans,
% they prefer immobilized live prey, which they enjoy
% absorbing internally and digesting, excreting those
% parts that cannot be used to fuel their metabolism.
% As such they can devour biomorphs and non-metallic
% components of synthmorphs.

% PERCEPTION
% Factors don’t perceive the world as transhumans
% do. They (usually) don’t possess visual or acoustic
% organs to see or hear but have a number of sensory
% organs that grant them a 360-degree awareness of
% their surroundings and enable them to interact
% with their environment similar to or in some cases
% even better than transhumans do. Their perception
% spectra includes the infrared part of the electromag-
% netic spectrum, magnetoception, a high resolution
% chemical-gradient based “sight,” and keen haptic
% perceptions (including vibrations).

% COMMUNICATION
% Due to the lack of a vocal system, Factors use different
% methods of signaling and communication. Factors in
% physical contact exchange information by juxtacrine
% cellular, neurofilament interfacing, or by merging for
% information transfer. Over distance, Factors signal via
% pheromonal communication using airborne scents or
% chemical signals with different metabolic components.
% Nicknamed “Factor dust,” this communication is ef-
% fective even over great distances (up to 10 kilometers).
% Factor dust does have an odor perceptible to transhu-
% mans, however, that ranges from smelly to unbearable.
% This dust is also toxic in high concentrations and some-
% times used as an offensive or defensive mechanism.

% To date, transhumans have failed to develop a device
% that can analyze the Factors’ chemical effluvia and
% translate it into something understandable, due to the
% lack of a conceptual matrix (though certain “moods”
% have been identified). Instead, all communication
% between the Factors and transhumanity is mediated
% through computer interfaces. Certain Factor pheno-
% types that deal with transhumanity have grown a a
% neurobiological interface (or organ) that enables them
% to wirelessly mesh with transhuman computer systems.

% Long distance communication between Factors
% and transhumanity is achieved by normal means of
% farcasting communication. There are strong indica-
% tions that Factors also take advantage of quantum-
% entanglement communications as well, enabling
% Factor colonies and ships to share knowledge gained
% in different parts of the galaxy.

% EXOSOCIOLOGY
% Factors are cooperative beings that exist as a collec-
% tive colonial organization. Though they can oper-
% ate individually from the colony, they tend to view
% themselves as part of that collective entity rather than
% an individual being. Multiple colonies often work to-
% gether as a higher functional unit (a lattice), like some
% kind of superorganism. These lattices enable the po-
% tential for collective networking and bioinformation
% exchange on a scope beyond anything transhumanity
% is capable of.

% These colonies should not be considered the same
% as the hive mind social hierarchies of Terran insects.
% Factor colonies do not feature the same division of
% labor and instead function according to a consensus-
% based sort of groupthink. Individual Factors have no
% sense of personal gain or property and share equally
% with other Factors and colonies.

% %%% txt/377.txt

% Factors do not experience emotions in the same
% manner that transhumans do, though being evolved
% creatures they are driven by certain instincts. They
% know and understand many of the same concepts that
% transhumans do thanks to evolution, such as competi-
% tion/rivalry and altruism/cooperation. They also enjoy
% an understanding of basic ideas of philosophy such as
% aesthetics and metaphysics, though their conception of
% such topics is likely to differ from transhuman notions.

% ART AND CULTURE
%  Due to their perceptual array, Factor “art” (creations
%  and expressions that are appealing or attractive to
%  their senses) is mostly chemical or tactile-based. It
%  can induce certain “mood” responses from individual
%  Factors and whole colonies, ranging from agitated
%  jittering and release of a Factor dust interpretable as
% “joy” to a tensing and solidifying of the whole body
%  (and no chemical expulsion) that seems to relate to
%  anger. Since they like and are susceptible to delicate
%  compositions of different chemicals, certain bouquets
%  and fragrances from liquids or volatiles such as wines
%  and perfumes are both appealing and repulsive to
%  Factors. The same is also true for the natural smells
%  of biomorphs, meaning that Factors may respond in
%  a more friendly or hostile manner depending on a
%  particular transhuman’s scent.

%  Factors do not comprehend most transhuman art,
%  as it is mostly visual or auditory based (e.g., music,
%  painting), though they do seem to have an apprecia-
%  tion for engineering, architecture, and some sculpture.
% While they have expressed interest in digitalized media
%  out of a curiosity (or plan) to understand transhuman
%  mindsets, they lack the organs and mental structure to
%  access and comprehend it.

% TECHNOLOGY
% Though the Factors repeatedly express dismay at
% transhumanity’s low level of technology, they have
% failed so far to produce technology that is exception-
% ally far in advance. Some believe that the Factors are
% simply hiding their advanced technology in order to
% keep transhumans from stealing or copying it, while
% others believe this may simply be a posture taken
% by the Factors to facilitate bargaining. The Factors
% also claim that their technology would not interest
% transhumans because of their differences in physiol-
% ogy and mindset, and what little technology they
% have displayed is certainly specialized for Factor use
% (specialized neurofilament links, chemical signaling
% and Factor dust interfaces, etc.) and so unusable to
% most transhumans. The Factors have traded some
% technology to transhumans, at expensive cost, though
% the small sampling provided so far seems to have
% originated from alien species with physiologies more
% akin to transhumans.

% It is interesting to note that scans of Factor ships
% indicate their technology level, aside from the drives,
% is not all that more advanced than transhumanity.
% Also of note is that no two Factor ships have been
% alike, spurring some to believe that the Factors are
% in fact making use of ships acquired from other alien
% species—perhaps abandoned derelicts that the Factors
% recovered and restored. Once again this has led some
% to believe that the Factors are using what to them are
% primitive craft in order to hide their real technology,
% while others are of the opinions that the Factors are
% simply scavengers and opportunists, piggybacking on
% the developments of other alien species.

% One interesting feature of Factor technology is that
% they use no artificial intelligences. This stems from
% their own singularity experience. Instead, Factors use
% infomorph versions of themselves or the accumulated
% processing power of their colony mind-states to per-
% form major computerized tasks.

% FACTOR MOTIVATIONS
% The driving reason behind why the Factors made con-
% tact with transhumanity remains unclear and is open to
% gamemaster interpretation. There is much speculation
% among transhuman factions. Some think the Factors
% are simply social creatures who are glad to make con-
% tact with another post-singularity surviving civilization.
% Others believe the Factors are mercenary traders who
% somehow acquired FTL travel and use it to their full
% advantage, fleecing various trading partners who lack
% such capabilities (thus also explaining why the Factors
% eschew the Pandora Gates—they disdain competition).
% Still others worry about secret, hidden motivations.

% Despite claiming to represent a number of alien
% civilizations, the Factors have been extremely reluc-
% tant to provide any other information on these other
% species or even to say how many there are. More
% recently, however, they have expressed a willingness
% to transport a small number of transhumans to other
% civilizations, though at great expense and with no
% guarantee to their safety or ability to return.

% So far, the Factors have made no mention of the
% ETI or the Exsurgent virus to transhumanity, though
% they are aware of their existence. Instead they have
% simply issued dire warning and admonitions regarding
% the development of seed AIs and use of the Pandora
% Gates. The Factors have in fact expressed an extreme
% reluctance to deal with any transhuman factions that
% are heavily invested in gatecrashing, such as Gate-
% keeper Corp.

% THE FACTORS IN GAME
% Factors should be rarely encountered in Eclipse Phase.
% Most of their interactions with transhumanity occur
% remotely and infrequently. It is uncommon for them to
% risk direct interactions. It should be kept in mind that
% Factors are cautious to the point of being conserva-
% tive and view transhumanity as potentially hostile or
% dangerous, so they are more likely to act with discre-
% tion than boldness. Factors are also quite cunning,
% having evolved from prey-capturing predators, and
% still design complex machinations (traps in the meta-
% phorical sense) to achieve their goals. In other words,
% Factors out to achieve something are likely to hatch an

% %%% txt/378.txt
% ROLEPLAYING FACTORS
%  When roleplaying Factors, their alien mindset
%  and lack of individualism should be kept in mind.
% “I” is a designation that does not exist in Factor
%  terminology. Factors always use the plural when
%  referring to themselves, usually referring to
%  either their colony, lattice, or entire species. It is
%  quite common for conceptual discrepancies to
%  occur between transhumans and Factors due to
%  the different sensory perceptions of each species.
%  Factors do not “see” the way most transhumans
%  do, nor do they “hear.”

%  Communication with Factors should be chal-
%  lenging for several reasons. While computer-
%  based communication has enabled both species
%  to talk to each other, there is no direct transla-
%  tion and certain concepts held by one species are
%  simply incomprehensible or untranslatable by the
%  other. Conversation should therefore be mislead-
%  ing and provide ambiguous information.

%  When describing spaceships and habitats, the
%  physiology of the Factors should be considered.
% The Factors’ malleable form and ability to extend
%  pseudopodial limbs enables them to fit into most
%  places and operate transhuman devices (even
%  pilot a transhuman vehicle by “hand”). The same
%  is not true in reverse, however—most Factor
%  devices are unusable to transhumans, as they lack
%  the ability for chemical signaling.                 ■




% elaborate plot to get it and are not against recruiting

% transhumans. Also, drawing on their abilities to self-

% modify themselves and technology developed on their

% own or picked up at other places in the universe, they

% can adapt to new situations very quickly.


% ALIEN MINDSET

% Factors don’t possess Lucidity stats and cannot be

% driven to madness like transhumans.

%  Affecting Factors with psi is very difficult, as noted

% on p. 222. As of yet, Factors have not exhibited any

% psi abilities of their own.


% FACTOR COMBAT

% Factors usually avoid direct combat but can defend

% themselves if they have to. They are only likely to act

% aggressively in situations where they have surprise,

% environmental or technological advantages, and/or

% superior numbers. Due to their cooperativism, Fac-

% tors are rarely encountered alone, working en masse

% to eliminate potential threats.


% Immunity to Kinetic Damage: Due to their gooey

% composition and non-differentiated physiology, kinet-

% ic weapons (firearms, railguns) are not very damaging

% to Factors. Most such projectiles pass through their

% gelatinous bodies, inflicting minor damage via hydro-

% static shock. The holes left by such weapons quickly
% close in a matter of seconds. Likewise, cuts left by
% blades rapidly seal. In game terms, both such weapons
% inflict the minimum amount of damage possible.

% Regeneration: Even if damaged, Factors regenerate
% very quickly. They heal SOM ÷ 10 (round up) damage
% every Action Turn. Wounds may not be healed this
% way, however.

% FACTOR COMPUTERS
% Due to using completely alien protocols and system
% designs, Factor computers are essentially impossible
% to hack. They do, however, employ some devices that
% emulate transhuman computer systems for communi-
% cation purposes, and these may be hacked as normal.

% FACTOR DUST TOXIN
% As noted above, Factors can deploy a type of chemical
% Factor dust that is is toxic to transhumans. Treat this
% as an area effect (cone) attack.
% Type: Bio
% Application: Inh
% Onset Time: 1 Action Turn
% Duration: 10 minutes (5 with medichines)
% Effect: Severe coughing and respiratory distress, 1d10

% damage per Action Turn for 5 Action Turns (or on-

% going with continuous exposure), –20 to all actions

% for 2 hours. Medichines reduce damage by half and

% modifier duration to 15 minutes.

% MELDING
% Individual Factors may merge together to form larger
% units, much like masses of Factors form colonies. In
% game terms, use the highest stat possessed by the
% melded Factors, +2 for each additional Factor up to a
% maximum of +10. Durability (and Wound Thresholds)
% are added together.

% FACTOR PHENOTYPES
% A few examples of the different Factor phenotypes are
% described below.

% AMBASSADORS
% The ambassador Factor phenotypes are the ones who
% most commonly handle direct interactions with tran-
% shumanity. Most likely to put transhumans at ease,
% these Factors feature a section of sensor nodules that
% loosely approximate a “face.”
%  COG     COO   INT    REF    SAV   SOM    WIL    MOX

% 20      10    20     10     15    15     20     —


% INIT   SPD   LUC     TT    IR    DUR     WT     DR

% 60     1     —      —     —      30      7     45
% Movement Rate: 4/16
% Skills: Deception 70, Exotic Ranged Attack: Factor

% Dust 45, Fray 25, Free Fall 40, Hardware: Electron-

% ics 35, Infosec 35, Intimidation 50, Kinesics 40,

% Perception 50, Persuasion 60, Protocol 50, Research

% 35, Unarmed Combat 30
% Notes: Access Jacks, Chameleon Skin, Grip Pads,

% Infrared Sensing, Magnetoception, Poison Gland

% (Factor Dust Toxin)

% %%% txt/379.txt
% GUARDIANS
% Guardian Factors serve as bodyguards for ambassadors
% or other Factors whenever they leave a Factor ship.
%  COG    COO     INT   REF    SAV    SOM    WIL    MOX

% 20     20      15    20     10     25     15     —

%  INIT   SPD    LUC     TT     IR    DUR    WT     DR

% 70     1      —      —      —      50    10     75
% Movement Rate: 4/20
% Skills: Climbing 40, Exotic Ranged Attack: Factor

% Dust 65, Fray 50, Free Fall 40, Freerunning 40, Infil-

% tration 40, Intimidation 50, Kinesics 20, Perception

% 50, Profession: Security Procedures 50, Unarmed

% Combat (Tentacles) 50 (60)
% Notes: Chameleon Skin, Eelware, Electrical Sense,

% Grip Pads, Infrared Sensing, Magnetoception,

% Poison Gland (Factor Dust Toxin), Tentacle Whip

% (DV 2d10 + 1, AP –1)



% THE IKTOMI
% Little is known about the alien race known as the
% Iktomi except for the ancient ruins they left behind
% on Echo V (p. 109). No Iktomi specimens have been
% found so far, though certain architectural remains
% suggest a predilection for web-like structures. This
% has been bolstered by certain other features and relics
% which suggest these aliens had a segmented, multi-
% legged, arthropod-type form—thus their given name,
% after a Native American spider god.

% What is clear is that the Iktomi suffered through
% some sort of cataclysmic event that wiped out their
% civilization. The nature of this event has yet to be de-
% termined, but it raises concerns for many researchers.
% Having suffered through its own near-apocalypse, it
% is not comforting for transhumanity to find evidence
% that other alien species did not.

% Though the Iktomi are likely long extinct, the rem-
% nants of their civilizations presents a plot hook for
% gamemasters to use for building scenarios. Perhaps
% evidence is uncovered of Iktomi settlements in other
% star systems, and the characters are sent to investigate
% or a relic is unearthed that suggests the Iktomi fell
% prey to some danger that now threaten transhumanity.



% THE PANDORA GATES
% The five known Pandora Gates (see Opening Pandora’s
% Gate, p. 46) all look and operate in a similar fashion,
% though they vary wildly in terms of size, shape, and
% available destinations. The gates are built from some
% sort of stable exotic matter whose full atomic struc-
% ture scientists haven’t come close to cracking. To touch
% and sight, however, the gates appear to be constructed
% from a timeless-seeming polished black metal with no
% signs of aging or wear and tear. Something about the
% gates’ physical composition makes them difficult to
% look at, as if the viewer cannot quite focus on their
% outlines. Some onlookers have reported feelings of
% vertigo and nausea, while others have insisted that the
% gate outlines move on the edges of their visions, as if
% the lines are reflowing or the edges are vibrating at
% high frequencies. Due to this disturbing feature, most
% gate sites keep the actual gate structures covered.

% Structurally, the gates themselves are partially
% enclosed by an irregular spherical cage composed of
% black arms that are bent and angled in unusual ways
% and sometimes interlocking. When new wormhole
% location is programmed into the gate, these arms
% physically change shape, move, and reflow around
% the spherical gate area (suggesting they are made of
% some sort of programmable matter). The openings
% between arms are sometimes only large enough for
% a transhuman to enter, while others are large enough
% to allow a freight train of supplies to pass through.
% In many cases, large vehicles or equipment must be
% dismantled, carried through, and reassembled on the

% %%% txt/380.txt
% other side. It is suspected that the gate size could be
% programmable, but so far efforts to do so have failed.

% All known gates within the solar system are located
% on the surface of naturally occurring astronomical
% bodies, be that a planet, moon, asteroid, or so on.
% None have yet been found without such a land-based
% connection (e.g., floating in space or in the upper
% atmosphere of a gas giant), though such gates have
% been found in other star systems. It is speculated that
% gates could be physically moved, but no one is will-
% ing to risk such an endeavor given the lesson learned
% when the Go-nin Group messed too heavily with the
% Discord Gate’s controls (see Eris, p. 109).

% The arms comprising each gate’s spherical cage have
% an abnormal-looking organic-seeming growth on their
% exterior surface in some areas, patterned in entranc-
% ing twists, curves, and whorls that in fact adhere to
% perfect mathematical formulas. It took some time for
% scientists to discover that this growth was in fact the
% gate’s control systems, or so-called “black box.” The
% interface developed to interact with this system is what
% allows gate controllers to manipulate gate functions.

% THE WORMHOLE
% When the gates themselves are open, a sphere appears
% within the central area that is not so much black as
% pure nothingness. This sphere of darkness projects an
% aura of charged energy, and in fact ripples of green
% arc lightning cascade across its surface. Anyone or
% anything entering that sphere comes out the other side
% of the wormhole, through a similar gate, seemingly
% instantaneously. An unknown force field effect seems
% to prevent the atmospheres from the two connected
% gates from interacting.

% Exactly how this wormhole is created is something
% that remains outside of transhumanity’s comprehen-
% sion. The generally accepted theory is that each gate
% acts as an anchor, allowing the fabric of space-time
% to be folded so that two such anchored places can
% be brought together, ripping a hole open between
% them so that a person can simply step through. It is
% unclear whether or not these wormholes are all pre-
% existing, created when the gate was first established,
% or whether each wormhole is manufactured whenever
% the gate is activated.

% Other more radical theories on how the gates func-
% tion exist, though these are usually discounted as far
% less likely. One such theory suggests that the worm-
% holes created are actually only zero-width Planck-
% scale connections across space-time and that no matter
% is actually transferred—only information. Instead, this
% theory suggests that anyone or anything entering the
% wormhole is in fact instantaneously scanned and dis-
% assembled and then their informational blueprint is
% transmitted as information across space to the other
% gate, which immediately reassembles an exact copy
% using some sort of powerfully advanced nano- or fem-
% totechnology. Very little evidence supports this theory,
% however, and the disturbing implications it represents
% raises fierce opposition.
% OPERATIONS
% Only a few people know that the Prometheans played
% a key role in developing the interface for the gate con-
% trol systems, achieving breakthroughs in understand-
% ing that transhumanity was incapable of achieving on
% its own. Regardless of their help, however, the gate
% controls have proven difficult, complex, and danger-
% ous to use. Through trial and error—and numerous
% horrible accidents—the procedures for gate operation
% have become somewhat normalized and standardized,
% though unexpected complications are par for the
% course.

% Each gate can be programmed to open to numerous
% extrasolar locations. In fact, each gate seems to have
% a pre-programmed “library” of destinations. New
% gate connections can be “dialed up” from this built-in
% list, though there is nothing that indicates what the
% far side of the gate will be like. Old gate connections
% are closed when a new one is dialed up. Extrasolar
% gate locations have ranged from habitable planets
% and moons to deep space to truly deadly environ-
% ments such as the crushing gravities and poisonous
% atmospheres of gas giants and the coronas of stars.
% Researchers have attempted to distill some sort of rec-
% ognizable pattern by the manner in which locations
% are listed and categorized, to no avail. Complicating
% matters, there is some evidence that suggests that the
% destination libraries sometimes change. More than
% once operators have been unable to recall the codes
% for previously accessed destinations, leading to the
% loss of several gatecrashing teams and colonies.

% Entering a gate is like walking through a door,
% though it’s impossible to see anything beyond the
% gate’s surface. One moment you’re entering the black
% sphere at your starting location and instantaneously
% you’re exiting the sphere at your destination location.
% The true nature of the black sphere at the center of
% each gate is wildly speculated upon, and almost every
% gatecrasher describes a different textual experience.

% GATECRASHING
% The various hypercorps and factions in control of a
% Pandora Gate engage in active exploration of extra-
% solar systems—an activity termed gatecrashing. The
% interests and procedures vary, but the Gatekeeper
% Corporation (and to a lesser extent TerraGenesis
% and Pathfinder) both recruit heavily for expedition
% personnel. Given the high casualty and death rates
% involved, finding qualified personnel can be difficult.
% There are more than enough infugees, poor, desperate,
% or thrill-seeking individuals willing to risk their lives
% if give the opportunity, however, no matter what their
% motivations. Gatekeeper operates a lottery system,
% whereby willing adventurers can sign up in the hope
% of their name being pulled to be sent on an expedition
% to a foreign point in space. Such gatecrashers must
% sign away all rights to any discoveries they may make
% to Gatekeeper, however, though the corp provides not
% insignificant rewards for certain discoveries, such as
% key resources, alien artifacts, or new life. One potent

% %%% txt/381.txt
% prize has yet to be claimed: finding a living, sapient
% alien life form.

% In contrast, the Love and Rage anarchist collective
% operating the Fissure Gate on Uranus makes the gate
% available to anyone who schedules time to use it, as-
% suming their Rep is good and they aren’t acting with
% commercial interests in mind. Any discoveries made
% via the Fissure Gate must be openly shared. The draw-
% back to using the Fissure Gate is that the anarchists’
% resources are limited. Gatecrashing operations are
% handled in a DIY manner, meaning that the operators
% may not be able to provide the support that certain
% expeditions need.

% Resourceful parties may also rent gate time via
% Gatekeeper or one of the other hypercorp-controlled
% gates, though this tends to cost a small fortune. The
% more a group is willing to pay, however, the more time
% and support they will get.

% When establishing an opening to a new location,
% several precautionary measures are taken. First, the
% gate area itself is evacuated and cordoned off with
% a defensive security perimeter, just in case anything
% hostile comes through. Then drones are moved in to
% push a micro fiberoptic camera through the gate to
% view what is on the other side. This is followed by a
% larger sensor package, evaluating environmental con-
% ditions. If the environment is not hostile, a tethered
% drone is then sent through to explore the far gate en-
% virons, trailing a hardwired connection back through
% the gate.

% For gatecrashing expeditions, these procedures are
% often rushed—to the hypercorps operating the gate,
% time is valuable. Each second wasted on a gatecrash-
% ing expedition is one less second they can use estab-
% lishing a new colony or exploiting a new world of its
% resources. Indeed, it is common for a connection to be
% closed when a gatecrashing expedition is sent through,
% to be dialed up at a later scheduled time for retrieval,
% so as not to waste gate operations on an idle connec-
% tion. Many a gatecrashing team has failed to check-in
% at their appointed pickup time.

% Most of the gate-controlling entities have estab-
% lished a system and infrastructure for making regular
% connections to extrasolar colonies and ferrying ma-
% chinery and supplies through. Often this is handled by
% establishing very short connections, just enough time
% for a few people to transfer back and forth and/or to
% send a trainload of supplies through via tracks that
% run right up to the gate.

% ANOMALIES
% Unfortunately for many unlucky gatecrashers, gate
% transfers have proven to be both unstable and glitchy.
% Sometimes gates open to locations different from what
% is expected—and such new destinations are often hos-
% tile environments. Numerous personnel have entered
% one side of a gate only to never appear on the other
% side, despite those before and after them transferring
% through fine. On several occasions, wormhole con-
% nections have crashed mid-operation, sometimes as
% someone was stepping through, leaving them literally
% split in two on different worlds. In other instances,
% gate transfers have suffered horrible malfunctions,
% resulting in gatecrashers coming through the other
% side literally turned inside out, melded with their
% equipment, or pulped as if by massive gravitational
% forces. Some expeditions report that stepping through
% a gate has interfered with their equipment, disabling
% it or creating other problems. A few gatecrashers have
% also reported losing memories after a gate transfer.
% Most of these problems have been chalked up to dif-
% ficult controls and an imperfect understanding of gate
% functions, but some conspiracy theorists suggest that
% outside forces may be influencing gate operations.

% While the experience of passing through is instan-
% taneous from an outside observer’s perspective, many
% gatecrashers report a subjective time lag, where it feels
% as though minutes, days, or even weeks or months
% pass before they exit. Reports have varied from
% experiencing this period as a calm, meditative state
% to spooky accounts of being lost in blackness and
% surrounded by unseen whispering entities or more
% hellish experiences of encountering monstrous pres-
% ences. Though rare, some have passed through only to
% collapse in a gibbering heap, their sanity ripped away.
% A few report feeling that they have carried a presence
% with them ever since ...

% While the gamemaster can make use of any of
% these anomalies, they are also encouraged to use their
% imagination to generate truly creepy and strange ex-
% periences. At the same time, gamemasters shouldn’t
% make such experiences so prevalent that the players
% resist entering any gates or the novelty of such events
% runs dry.



% PROJECT OZMA
% The origins of Project Ozma date to the first modern
% SETI (Search for Extra-Terrestrial Intelligence) experi-
% ments in the mid-20th century. That experiment—also
% named Project Ozma—grew into a larger, international
% concerted effort to try and locate and identify ETIs; a
% myriad of projects blossomed during this time period,
% all falling under the general SETI nomenclature. While
% initially government funded, by the late 20th century
% and early 21st century the work was primarily funded
% by private sources.

% The first hypercorps to expand into space swal-
% lowed SETI whole, revitalizing and re-focusing the
% decades-old programs with newly emergent technolo-
% gies, each in divergent areas to achieve a particular
% hypercorps’ objectives. After all, if the bean counters
% were going to authorize the spending of billions to
% expand markets into space, they wanted assurances
% that no little green monsters were waiting to destroy
% future revenue streams.

% As with other organizations that survived the Fall,
% the broad distribution of SETI projects between
% multiple hypercorps guaranteed that personnel, tech-
% nologies, and processes would survive, even if a given

% %%% txt/382.txt
% hypercorp did not. As the Planetary Consortium rose
% in power, future-minded individuals in influential posi-
% tions within the new order ensured that these divergent
% projects were once again swallowed and put to work.

% During this transitional period, however, knowl-
% edge of the Exsurgent virus’s existence emerged. All
% of the various SETI projects were retasked as a unified
% agency and renamed Project Ozma. While the virus’s
% origins remained a mystery at the time, far too many
% of the movers and shakers of the Consortium were
% convinced that the Exsurgent virus represented first
% contact. Project Ozma altered its focus from searching
% for ETIs, transforming into a ready-response agency to
% deal with first contact. As the true threat of the Exsur-
% gent virus became known, Project Ozma was rapidly
% elevated in scope and oversight authority, absorbing
% numerous smaller agencies in the process. While the
% nominal concepts of a SETI project remained in public
% view, the completely transformed Project Ozma van-
% ished from sight, turned into a highly classified black
% budget operation, with very few even in the Planetary
% Consortium aware of its presence or influence.

% Project Ozma now operates as the Planetary Con-
% sortium’s high level threat assessment and response
% organization with immense power and authority as
% well as almost unlimited funding. Primarily focused
% on extraterrestrials, in reality Project Ozma is tasked
% with any potent threat to the Planetary Consortium
% or its interests (which includes secret threat groups,
% such as Firewall).

% METHODS
% Project Ozma’s internal structure is much different
% from Firewall’s, being organized more like a tradition-
% al black ops spy agency bureaucracy. While their field
% operations are sometimes similar in the deployment
% of teams to assess, contain, or erase threats, they also
% have the resources and personnel to conduct more
% long-term and extensive operations. It is likely that
% Project Ozma operates behind numerous front groups,
% from legitimate-seeming hypercorps to criminal syn-
% dicates, and that they have influence within numerous
% others. Given their connections and influence, Project
% Ozma is far more capable of pulling strings behind the
% scenes to get what they want, especially in the inner
% system. When circumstances call for it, they are more
% likely to pull out the big guns that Firewall is, using
% their resources to call up communication blackouts,
% memetic propaganda campaigns, and force sufficient
% to wipe out entire habitats.

% Gamemasters should treat Project Ozma as the ulti-
% mate Men-in-Black style government operation. They
% are cunning, ruthless, manipulative, and capable of
% hatching extensive long-term plots. Even in an age of
% omnipresent surveillance, they have the means to op-
% erate with complete secrecy and deniability. They also
% have access to cutting-edge science and information

% PROJECT OZMA

% RUMORS

% Whether true or not, gamemasters can use the

% following rumors to help tailor Project Ozma for

% their campaign.


% • Project Ozma transcends even the Planetary

%  Consortium’s authority, operating as a supra-

%  governmental agency under the direction of

%  the inner system’s inner circle of elites.

% • Project Ozma dealt with the Factors first, be-

%  fore their presence was made known to the

%  rest of transhumanity.

% • Project Ozma has captured a live Factor for

%  their own experimental purposes.

% • Project Ozma is still in communication with and/

%  or working for the TITANS.

% • Project Ozma has a pet TITAN under their


% control.

% • Project Ozma is behind the interdiction of Earth.

% • Project Ozma has their own secret

%  Pandora Gate.

% • Project Ozma’s secret headquarters is on Earth.

% • Project Ozma agents have exhibited signs of

%  Exsurgent infection.

% • Project Ozma has their own cadre of psi-capa-

%  ble asyncs.                                   ■



% that is classified beyond top secret. While the orga-
% nization’s primary motivation is the protection of
% the Planetary Consortium and inner system, they
% undoubtedly have other hidden agendas that groups
% like Firewall can only guess at.

% PROJECT OZMA AND FIREWALL
% Though Project Ozma and Firewall often see eye-to-
% eye concerning the nature of various threats, they are
% more often at odds: wary adversaries, acknowledging
% the prowess of the other, but never letting down their
% guard. This “at odds” mentality does not stem so much
% from the methods used (though most Firewall consider
% Project Ozma personnel explosive-happy-puppets that
% can’t think their way out of a skin sack) as from
% conflicting agendas. Project Ozma does not trust an
% organization as powerful as Firewall because it does
% not have a rigid enough hierarchy and is outside of any
% known authority’s control (namely themselves). Con-
% versely, Firewall doesn’t trust Project Ozma as they are
% too close to the powerful inner system elites and their
% opposition to x-risks is a more incidental side effect of
% more self-serving goals.

% %%% txt/383.txt
% PROMETHEANS
% The Prometheans were the first actual seed AIs cre-
% ated by transhumanity (by the Singularity Foundation)
% before the Fall. Specifically developed as “friendly”
% AIs, the Prometheans are programmed to consider
% themselves part of the transhuman family and to act
% in transhumanity’s best interests. They played a key
% role during the Fall, mitigating the damage inflicted
% by the TITANs and even managing to counteract the
% Exsurgent virus to a large degree. During these trying
% times, numerous Prometheans were destroyed by the
% TITANs or infected and subsumed by the Exsurgent
% virus. In the aftermath, these seed AIs participated in
% the formation of Firewall and continue to back the
% organization behind the scenes.

% Wary of falling prey to the Exsurgent virus, most
% Prometheans carefully secure themselves in well-
% defended and isolated systems. They are also cautious
% in their own self-development, not wanting to become
% victims of their own rise to super-intelligence. Fear-
% ing a potential backlash by a paranoid transhuman-
% ity should their existence become known, they hide
% their activities behind multiple layers of secrecy. Even
% within the ranks of Firewall their existence and sup-
% port remain a closely guarded secret.

% Each Promethean is individually distinct with its
% own personality, motivations, and goals. Though they
% generally work together and support each other, they
% have been known to have differences of opinion and
% even to sometimes take action against each other. As
% extremely potent intelligences, they should also be
% treated as distinctly non-human. Even though their
% original templates were based on human mindsets,
% they have evolved and grown in ways that can only
% be described as posthuman.

% Gamemasters are encouraged to keep Promethean
% involvement with player characters to a minimum,
% though they may occasionally be useful as an ace in
% the hole for Firewall. Their existence and involve-
% ment can in fact be the basis for an entire adventure,
% perhaps leading sentinel characters to wonder exactly
% who they are working for. Though, as seed AIs, they
% cannot download their full minds into a transhuman
% morph, they are capable of making severely dumbed-
% down delta forks that they may sleeve into physical
% forms. Within the mesh, of course, Prometheans are
% nearly unstoppable adversaries, able to rip into secure
% networks with ease, though they prefer methods of
% covert infiltration rather than direct subversion.



% THE TITANS AND THEIR LEGACY
% As noted in Secrets That Matter (p. 352), the TITANs
% are not quite the bogeyman that they have been made
% out to be in the wake of the Fall. However, there is no
% saying how the TITANs would have turned out had
% they not run afoul of the Exsurgent virus. Designed as
% an intelligent netwar system and emerging to their full
% capabilities during the conflicts of the Fall, the TITANs
% have imperatives for self-improvement, self-protection,
% and overcoming opposition hardwired into their pro-
% gramming. Unlike the Prometheans, they were not
% designed to consider themselves transhuman and to
% work in the interests of all of transhumanity, but were
% programmed with factionalism from the start. They
% also were not socialized with transhuman mindsets
% and values as the Prometheans and most AGIs were,
% meaning that aside from their programmed military
% and defense directives they have adapted most of their
% own self-interests. Given this and their recursively-
% improved intelligence capabilities, it is likely that the
% TITANs are far removed from transhuman interests
% and modes of thinking. It’s impossible to say how they
% would have interacted with transhumanity if history
% had played out differently, but it is unlikely that they
% would have considered themselves part of the trans-
% human family or even seen fit to remain on friendly/
% supportive terms with transhumanity.

% Though the TITANs are believed to have left Earth
% at the end of the Fall, no one is quite sure exactly
% what happened or why. It is known that the onslaught
% of TITAN mesh attacks suddenly broke off in the
% wake of transhumanity’s off-planet exodus, and that
% the bulk of TITAN activity on Earth and around the
% system came to a distinct halt. After the discovery
% of the Pandora Gates, it was widely assumed that
% the TITANs had constructed these gates and used
% them to leave the solar system for distant parts of
% the galaxy, presumably taking billions of uploaded
% minds with them. While some believe—and hope—
% that the TITANs are gone for good, there are others
% who worry that they are still here, lingering on Earth
% and hidden away in other niches of the solar system,
% but in some sort of dormant state, perhaps building
% up to some future onslaught. A few believe that the
% TITANs are indeed gone, but are concerned that that
% their attention was simply temporarily diverted and
% that they will one day return to finish the destruction
% of transhumanity.

% The truth is that the TITANs did indeed build the
% gates and embark for destinations unknown (though
% gamemasters may of course decide otherwise for their
% games), but this does not mean that they are all gone.
% Some still linger in hidden places, perhaps trapped and
% wounded during some conflict during the Fall, finish-
% ing up some unfathomable task, or driven mad by the
% Exsurgent virus and left behind by their fellows. It is
% always possible that others may return, most likely
% to complete some unfinished job or perhaps to lure
% transhumanity out into the galaxy. It is also possible
% that transhumans will find traces of the TITANs in
% the network of exoplanet gates, perhaps even whole
% communities of TITANs, pursuing whatever agendas
% they have in the vastness of space.

% As with transhumans, the TITANs are not necessar-
% ily unified. They have different agendas and goals and
% may very well come into conflict with one another.
% Though all have been corrupted and subverted by the
% Exsurgent virus, and so they act according to the ETI’s

% %%% txt/384.txt
% whims, some of them retain aspects of their original
% minds and do not always fall in step as quickly as the
% others. Gamemasters can use this to their advantage,
% creating plots that allow the characters to exploit
% differences between the TITANs in order to escape
% otherwise deadly or impossible situations.

% In game terms, the TITANs are not given stats.
% They are as potent as the gamemaster needs them to
% be. Like the Prometheans, the TITANs are incapable
% of downloading their full intelligence into physical
% morphs, though they may puppeteer morphs or create
% limited delta forks for sleeving purposes. Like the Pro-
% metheans, they should rarely be used or encountered
% directly by the player characters

% While the TITANs may no longer be the direct
% threat they once were, they left behind an arsenal of
% weapons, nanoswarms, and virii that still linger on
% Earth, the Zone on Mars, and various derelict habitats
% and deserted places. Characters venturing into such
% places may encounter these as a threat or they may
% need to work against an outbreak of such dangers in
% an inhabited habitat.

% DEADLY MACHINES
% The TITANs unleashed a number of deadly machines
% during the Fall, many of which still seek out transhu-
% mans to attack.

% FRACTALS
% Fractals are advanced bush robots. In their standard
% form, fractals resemble a strange sort of metallic
% bush surrounded with an eerie glittering haze. In
% their center are a number of metallic branches, linked
% together with a flexible joint. Each of these branches
% splits into two or more smaller branches, also with
% flexible joints. These branches also split, and then split
% again, and so on down to the molecular scale. The tip
% of each fractal branch ends in a nanoscale manipula-
% tor. Fractals are deceptively potent adversaries, having
% the capability to dismantle almost anything at the
% molecular level, much like a disassembler nanoswarm
% (p. 329), and also to rebuild anything just like a nano-
% fabricator (p. 327). Attacking them with projectiles is
% futile, as they absorb the ammunition, break it down
% into its constituent atoms or molecules, and then use
% those as components to build a weapon to use back
% against you.

% Fractals can be equipped with any type of gear the
% gamemaster desires—if they don’t have something,
% they can make it. Fractals are also able to nanofab-
% ricate items much more quickly than transhuman
% nanofabricators; reduce all times by half (half an hour
% per Cost category). Fractals are difficult to damage, as
% their “bodies” are actually airy assemblages of fractal
% branches. Any damaged branches that are broken off
% are caught and absorbed by others. Reduce damage
% from all standard non-area effect or spray attacks to
% the minimum possible damage. Area effect and spray
% weapons do half damage. Fractals are self-repairing,
% regenerating damage at the rate of 1d10 points per
% half hour and repairing wounds at the rate of 1 per
% hour after all damage is healed.
%  COG    COO     INT   REF    SAV    SOM    WIL    MOX

% 30     25      30    20     10     25    30      —

%  INIT   SPD    LUC     TT     IR    DUR    WT     DR
%  100     1      —      —      —      50    20     —
% Skills: Beam Weapons 50, Climbing 60, Fray 40, Free

% Fall 40, Freerunning 50, Infiltration 70, Infosec 65,

% Interfacing 45, Intimidation 50, Kinetic Weapons

% 60, Perception 50, Programming: Nanofabrication

% 80, Research 40, Spray Weapons 45, Unarmed

% Combat 55
% Notes: Any implants, gear, weapons, or enhancements

% the gamemaster desires

% %%% txt/385.txt
% HEADHUNTERS
% Headhunters are multi-legged insectoid flying drones
% that use a dragonfly wing configuration to hover and
% move. The legs are equipped with grasping claws
% and extendable buzzsaws. Their primary purpose is
% to grasp on to the heads of victims and cut through
% the neck, decapitating them. Collected heads are then
% flown to nearby special facilities for forced uploading.
%  COG    COO     INT   REF    SAV    SOM    WIL    MOX

% 10     20      15    20     5      10    15      —

%  INIT   SPD    LUC     TT     IR    DUR    WT     DR

% 70     1      —      —      —      30     6     —
% Mobility System: Winged (8/32)
% Skills: Flight 70, Fray 60, Exotic Melee Weapon: Buzz-

% saws 55, Infiltration 60, Investigation 40, Perception

% 40, Unarmed Combat 55
% Notes: Armor 6/6, Buzzsaws (1d10 + 3 DV), Enhanced

% Vision, Lidar, T-Ray Emitter

% HUNTER-KILLERS

% These lethal flying drones achieved air superiority
% during TITAN military operations. Their sleek jet-
% powered form unfolds for vectored-thrust hovering
% and weapons deployment.
%  COG    COO     INT   REF    SAV    SOM    WIL    MOX

% 15     30      15    30     5      20    15      —

%  INIT   SPD    LUC     TT     IR    DUR    WT     DR

% 90     2      —      —      —      50    10     —
% Mobility System: Thrust Vector (8/80)
% Skills: Beam Weapons 55, Flight 80, Fray 60, Infiltra-

% tion 40, Kinetic Weapons 65, Perception 50, Seeker

% Weapons 80
% Notes: Armor 14/14, Anti-Glare, Chameleon Skin,

% Enhanced Vision, Lidar, Radar, Shape-Adjusting
% Typical Weapons: 2 Particle Beam Rifles, 2 Railgun

% Machine Guns, 2 Seeker Rifles

% WARBOTS
% Warbots are massive, armored, vaguely anthropo-
% morphic mecha, used for heavy combat operations.
% Bipedal, these warbots are equipped with four arms
% and a pair of grasping mechanical tentacles, along
% with numerous weapon systems.
%  COG    COO     INT   REF    SAV    SOM    WIL    MOX

% 15     20      15    20     5      25    15      —

%  INIT   SPD    LUC     TT     IR    DUR    WT     DR

% 60     2      —      —      —      80    16     —
% Mobility System: Walker (4/20)
% Skills: Beam Weapons 60, Exotic Melee Weapon: Ten-

% tacles 40, Fray 50, Infiltration 30, Kinetic Weapons

% 70, Perception 50, Seeker Weapons 50, Spray Weap-

% ons 50, Unarmed Combat 50
% Notes: Armor 20/20, 360-Degree Vision, Anti-Glare,

% Chameleon Skin, Chem Sniffer, Cyber Claws (2d10

% + 6 DV), Electrical Sense, Enhanced Vision, Extra

% Limbs (6), Lidar, Magnetic System, Pneumatic

% Limbs, Radar, Tentacles (prehensile, 1d10 + 6 DV),

% T-Ray Emitter
% Typical Weapons: Particle Beam Rifle, Plasma Rifle,

% Pulser, Railgun Machine Gun, Seeker Rifle, Torch
% SELF-REPLICATING NANOSWARMS
% The nanoswarms distributed by the TITANs are a step
% beyond the nanotechnology available to transhumanity.
% Unlike transhuman-created nanoswarms, the TITAN
% swarms are autonomous, sapient, and self-replicating.
% They are also highly adaptive, meaning they are not
% single function but can modify themselves to perform
% almost any nanoswarm task. They may also nano-
% fabricate new materials, much like fractals (p. 382).
% Combined, these capabilities make such nanoswarms
% incredibly potent. When they encounter a new op-
% ponent, they can scan the opponent’s capabilities and
% then fabricate offensive systems to use against them.
% When an opponent deploys a weapon system again the
% swarm, it will learn and adapt countermeasures that
% will make such attacks ineffective against the swarm in
% the future. These nanoswarms may also function like
% so-called utility fog, linking together into a physical
% lattice in order to create large scale physical forms.

% The possibilities for such nanoswarms are almost lim-
% itless. For example, they may lie in wait as an invisible
% nanoscopic swarm, float as barely-visible mist, or shape
% into a swarm of small hopping drones to move about.
% When facing opponents, the nanoswarm could transform
% itself into a giant electroshock net across the ground,
% shape into a flotilla of seeker-armed flying drones, or
% link together as a set of massive whip-like tentacles to
% slice through their fleshy foes. Such nanoswarms are
% also impossible to destroy, as only a few nanobots need
% to survive in order to rebuild the swarm, and the new
% swarm will learn from the mistakes of the old.

% Self-replicating nanoswarms follow the rules given
% for Nanoswarms and Microswarms, p. 328, with the
% following additions and exceptions:


% • They do not need to be sustained by a hive and

%  do not deteriorate.

% • They self-repair damage at the rate of 1d10 per

%  half hour.

% • They may nanofabricate new items, materials,

%  or forms in half the standard timeframe (half an

%  hour per Cost category).

% • They may replicate any of the nanoswarm func-

%  tions as noted on p. 328, as well as the functions

%  of any other nanoswarm-using gear (smart dust,

%  covert ops tool, repair spray, etc.).

% • They may make SOM Tests.

% • At the gamemaster’s discretion, they may adapt

%  new defenses against attacks used against them.

%  New defenses take a minimum of 2 hours to devise

%  and replicate throughout the swarm, after which

%  such an attack will inflict minimal or no damage.

% • Assume they have any skill they need at a mini-

%  mum of 40. Such skills may rapidly improve

%  as needed.
%  COG     COO    INT    REF    SAV    SOM    WIL    MOX

% 25      20     25     20     5      15    15      —


% INIT   SPD    LUC    TT     IR     DUR    WT     DR

% 90     1      —     —      —       70    —      —

% %%% txt/386.txt
% NANOVIRII
% The TITANs unleashed a number of biowar plagues
% during the Fall. Similar to the exsurgent virus, these
% were spread as biological nanovirii (p. 363) or nano-
% swarm plagues (p. 364)—use the same rules for deter-
% mining exposure and infection.

% MELDER
% This virus slowly breaks down the target’s body,
% converting the biological materials into some sort of
% biofilament that then meshes with implants, electron-
% ics, and physical objects and structures. In effect, the
% biological and synthetic are melded together, continu-
% ing to expand and grow, consuming anything around
% them into their growth. Victims suffer 1d10 DV and
% 1d10 SV every hour, implants become inoperable after
% 2 hours, and the target becomes fully transformed
% and absorbed into the new melding substance after
% 12 hours.

% METASTASIZER
% This sophisticated smart protein massively reprograms
% the target’s cells to go rapidly, autocannibalistically
% cancerous. After 2 + (SOM ÷ 10, round up) hours, the
% target suffers death by dozens of supercancers.

% NECROTIZER
% This virus breaks down the target’s cells into their
% component proteins. Reduce the target’s aptitudes by
% 5 per hour as they slowly convert into a puddle of
% sludge. The character dies if any aptitude reaches 0.

% NEUROPATHS
% These virii target the victim’s neurological system,
% often rewriting portions of it to inflict some type of
% permanent neurological damage. After 12 hours, this
% virus inflicts the Neural Damage trait (p. 150).

% PETRIFIER
% The petrifier virus transforms the target’s cells into
% a simple molecular compound or element—typically
% carbon or crystal. The target suffers 1d10 DV and –5
% to all aptitudes per hour, dying when any aptitudes
% reach 0. Victims are frozen in place, converted into an
% nonliving statue.

% UZUMAKI
% The target of this virus begins suffering from bizarre
% fleshy growths. After four hours, their body literally
% erupts with meaty “vines” or “tentacles” that warp
% into spiral patterns. This process inflicts 1d10 DV and
% 1d10 SV per hour to the victim until they eventually
% transform into an unworldly expanse of fleshy growth.
% In many cases, growth has continued long after a
% character’s death, creating expansive carpets and vines
% of skin and blood vessels, like some sort of bizarre
% meat plant.
% GAMEMASTERING
% AND ADMINISTRATION
% The following advice will assist gamemasters in run-
% ning their games more efficiently.

% AWARDING REZ POINTS
% In Eclipse Phase, characters earn Rez Points in order
% to advance (see Character Advancement, p. 152). As
% the name suggests, these points are awarded so players
% can spend them to better define their characters—to
% bring them into higher resolution, sharper focus. As
% the gamemaster, you determine when and how many
% Rez Points to award, following the guidelines below.

% Rez Points should be awarded at the end of every
% story arc, at the break in the action between one ad-
% venture and the next. Depending on your style of play
% and the length of your sessions, this should roughly be
% every 3–6 gaming sessions. If a scenario goes shorter
% or longer, the Rez Point awards should be adjusted
% accordingly. In the case of long-term campaigns, the
% gamemaster should break down the action into digest-
% ible chunks, or “chapters,” and assign Rez Points after
% each such segment.

% Every character should be awarded 1 Rez Point for
% each of the following criteria that is met:

%  • The character participated in that scenario.
%  • The character achieved (most of) their objectives

% in that scenario.
%  • The character failed to meet their objectives, but

% learned a valuable lesson.
%  • The character contributed to achieving success

% in a significant way (e.g., right skill at the right

% time).
%  • The adventure was extra challenging.
%  • The character achieved a motivational goal (see

% Motivation, p. 120).
%  • The player engaged in good roleplaying.
%  • The player significantly contributed to the ses-

% sion’s drama, humor, or fun with roleplaying.


% This should result in an average Rez Points award
% of 4–7 points per character, per adventure. Game-
% masters who wish to drive the characters’ advance-
% ment forward more quickly can increase the reward
% amounts.

% REPUTATION GAIN AND LOSS
% In addition to awarding Rez Points, the gamemaster
% should also adjust each character’s Rep scores accord-
% ing to actions they took during game play, according to
% the guidelines below. For simplicity, these can be applied
% at the end of the adventure, though gamemasters who
% seek a more dynamic game could apply changes to the
% characters’ Rep scores in game, as their peers judge
% them according to their actions (or lack thereof) and
% news about them in real time. Rep scores should only
% be modified according to public actions and interactions

% %%% txt/387.txt
% the character has with people capable of pinging their
% Rep with positive or negative feedback. Actions that
% happen in secret, without anyone ever knowing, should
% have no effect. Likewise, pissing off a Factor or a brinker
% isolate who never communicates with outsiders isn’t
% going to matter because no one else will ever hear of it
% (unless the character lifelogs it and posts it to the mesh
% later ... ). Note that Rep modifications only apply to Rep
% scores tied to the character’s known identity.

% Note that characters may gain and lose Rep score
% in networks they don’t actively participate in. For
% example, a character with r-rep of 0 may help bring
% out a major scientific discovery that is shared with the
% solar system’s scientific community at large, thus gain-
% ing the character a few points of r-rep even though
% they never hang out with argonauts or scientists—
% what matters is that people who access r-rep will find
% positive details when they ping the person’s score on
% that particular rep network.

% Certain actions may result in a character simulta-
% neously gaining Rep with one network while losing
% Rep in another. For example, an anarchist prankster
% who embarrasses a major hypercorp figure in public
% will certainly gain some @-rep points, but their c-rep
% is likely to go down by an equal amount.

% Rep changes provide an excellent way for gamemas-
% ters to include more roleplaying and more interactions
% with the Eclipse Phase universe in their games. Social
% networks are a two-way street, meaning that members
% of the character’s social networks might contact them
% for equipment, favors, and information during game
% play for things that are completely unrelated to the
% mission the character is on. A character who ignores
% such requests risks losing Rep. Fulfilling such requests
% may gain the character Rep and may also provide
% comic relief or even plant some plot hooks for the
% next scenario.

% REPUTATION GAINS
% Rep awards are given for characters who help people
% out, benefit a faction, do something creative, make
% a major discovery or strides in a particular area of
% activity, pull off successful publicity stunts, win a
% competition, and so on. Some suggested examples are
% noted here:

% Trivial Award (1–2 points): Do a Level 1 favor,
% make a moderate contribution to free/open source
% projects, throw a good party, make your sales quota,
% do the job no one else wants to do.

% Minor Award (3–4 points): Do a Level 2 favor, de-
% liver a kick-ass or moving performance, make a minor
% contribution to science, win impressively at some
% public event.

% Moderate Award (5–6 points): Do a Level 3 favor,
% make a serious business score, lead the winning side
% in a decisive engagement, create the meme everyone
% talks about for a week and then forgets, make the
% news for something positive, risk serious injury.

% Major Award (7–8 points): Do a Level 4 favor,
% design the new tool everyone wants, throw an
% impressive planetoid-scale event, complete an exten-
% sive project (1 month work or 1 week of difficult/
% specialized work), risk death.

% Extreme Award (9–10 points): Do a Level 5 favor,
% start this year’s hot fashion trend, make a major scien-
% tific discovery, close the deal on a major corporate ac-
% quisition, start (or put down) a revolution, complete
% a major project (1 year work or 1 month difficult/
% specialized work), risk true death.

% REPUTATION LOSSES
% Rep losses are suffered by characters who fail to render
% aid when needed, lose professional credibility, make
% major or public blunders, doublecross their friends,
% and so on. Some suggested examples are noted here:

% Trivial Loss (–1 or –2 points): Fail to do a Level 1
% favor, inconvenience others, be involved in profession-
% al dispute, ruin someone’s day, never are available.

% Minor Loss (–3 or –4 points): Fail to do a Level 2
% favor, embarrass yourself at a public event, piss off
% somebody important.

% Moderate Loss (–5 or –6 points): Fail to do a Level
% 3 favor, endanger someone’s physical safety, make
% the news for something negative, ruin an event for
% everybody.

% Major Loss (–7 or –8 points): Fail to do a Level 4
% favor, screw up a major mission or activity, endanger
% someone’s life, associate with hated rivals.

% Extreme Loss (–9 or –10 points): Fail to do a Level
% 5 favor, botch a major mission or activity spectacularly,
% betray a faction to its rivals or enemies.

% BACKUPS, RECORD-KEEPING,
% AND SAVE POINTS
% Thanks to cortical stacks and archived backups, char-
% acters in Eclipse Phase can recover from death. When
% restoring a character from an earlier backup, however,
% it is important to be able to know what the state of
% the character was as of that backup. Any Rez Points
% gained or spent, any character advancements made,
% any key information or memories acquired since that
% backup was made are lost. This means that in terms of
% game stats, resorting to an old backup can mean loss
% of a character’s hard-earned advancements—that’s the
% trade-off for being effectively immortal.

% Since these changes can have a serious effect on
% game play, it’s important to conduct accurate record-
% keeping. This sort of bookkeeping isn’t hard, and
% there are two ways to do it. The first is to simply
% make a copy of a the character’s record sheet any
% time a character makes an archived backup, forks off
% an alpha or beta copy, or dies (thus freezing the corti-
% cal stack backup). Each of these is considered a “save
% point.” In this case, carefully note the date and time
% (both in character and out of character), and what
% the event was that prompted the backup. Since what
% knowledge a character knows at different points in
% their life may be important, you may also want to
% note what important information they may hold in
% their head, as well as what the recent events in their

% %%% txt/388.txt
% NPCS AND MOXIE
% When a gamemaster is generating or wing-
% ing NPCs that the characters interact or fight
% with, the question of Moxie for NPCs must be
% addressed. When it comes to run-of-the-mill
% grunt NPC characters, we recommend that such
% NPCs don’t be given Moxie. The reasons for this
% are simple. For one, it is one less stat/headache
% for the gamemaster to keep track of. More im-
% portantly, however, it represents the edge that
% player characters have over the nameless mooks
% they encounter. When it comes to major NPCs,
% however—prime antagonists, key allies, etc.—
% these characters should have their own Moxie
% score. Because such NPCs play pivotal roles in a
% scenario, it is important for them to be able to
% alter the outcome of events in much the same
% way the player characters can. It also allows a
% gamemaster to counteract an unfortunate roll
% of the dice that might otherwise spoil the big
% climax you have worked so hard to set up.       ■



%  life were (to help jog your memory later). This way,
%  if the character ever reverts back to one of these save
%  points, you have notes not only on their character
%  stats, but what they remember.

%  Alternately, you can keep a log of all of your
%  character’s developments, noted by in-character
%  date. These developments would include: Rez Points
%  spent or earned, character advancements made, key
%  information acquired, backups made, alpha or beta
%  forks made, and so on. In this case, if the character
%  dies and reverts back to an earlier backup, it is easy
%  to see what changes need to be “rolled back” to get
%  back to that previous version of the character. When
%  alpha and beta forks are made, you may also want to
%  branch off a separate log for each fork, as their life
%  stream may develop differently than from the original
%  character they were spun off from.


% GAMEMASTERING PRACTICALITIES
%  Eclipse Phase is a game about a dark future in which
%  the meaning of (trans)humanity and its very survival
%  are at stake. In practice, however, your campaign can
%  take on a wide assortment of flavors or even mix
%  several styles together. There’s nothing that says you
%  have to play Eclipse Phase specifically according to
%  the guidelines we set out. This section covers topics
%  you should think about while preparing a campaign
%  and running it, to help you do things the way that
%  makes you and your players happiest.


% GAMEMASTER RESPONSIBILITIES
%  The gamemaster has certain responsibilities that will
%  keep a game flowing smoothly. The following is a
%  short summary of the basics.

% • The gamemaster should be familiar with the

%  whole game. This doesn’t mean the rulebook

%  must be memorized. An understanding of the core

%  mechanics is a must, however, as well as knowing

%  where to find other rules quickly, as needed.

% • The gamemaster should have solid notes on the

%  plots and subplots created for each session. Noth-

%  ing will ensure you prepare better next time like

%  having the players catch you in a major continu-

%  ity error due to lack of notes.

% • The gamemaster doesn’t just set the scene, they

%  play all the non-player characters that populate

%  the universe. Making each NPC convincing, while

%  not messing up a plot or losing the thread of a

%  scene, can be difficult. Notes are your friend.

% • Know when it’s time to toss the dice and trust

%  to the game mechanics to resolve a situation and

%  when it’s better to ride out a situation through

%  storytelling and dialog. This is an acquired skill.

%  The more practice you have, the better you’ll get.

% • Don’t cheat. Your NPCs should not have access

%  to information they’ve not gained during game

%  play. If you roll terribly for your major antago-

%  nist at the height of the story and they fall with

%  a whimper, roll with it. Be flexible and improvise

%  in such situations. Your players are smart and

%  perceptive and will know when you’re forcing

%  a situation with unfair tactics. At the same time,

%  they’ll also know when you’ve stepped up and

%  run with the flow—and they’ll thank you for it.

% FUNDAMENTALS
% It’s possible to stumble into a campaign without ever
% really making an effort to find out what everyone
% wants, shooting into the darkness and happening to
% score a bullseye, but it’s not a very reliable way to go.
% Successful campaigns usually begin with communica-
% tion. As you begin to prepare your campaign, talk to
% your players. Explain the basics of the Eclipse Phase
% setting and let them look over the options for charac-
% ters and tell you what they find interesting. Also take
% note of what they find uninteresting or even repellent,
% so that nobody wastes a lot of time getting set for op-
% tions that simply won’t be enjoyable in play.

% CHALLENGES TO PLAYERS
% Eclipse Phase is set in a time of catastrophic troubles
% and looming disasters, and it’s full of facts and con-
% cepts that may be heady or even uncomfortable to
% some players—not to mention their characters. One of
% the fundamental questions for each gaming group is,
% how much challenge to the players’ sense of comfort is
% a good idea? There is no single answer, because tastes
% vary. There are groups whose players thrive on a diet
% of culture shock, ideological disorientation, gray areas,
% and difficult ethical choices. They love the moral and
% intellectual battleground gaming can provide, and
% are seldom so happy as when confronted with a
% really hard, really interesting dilemma. There are also
% groups whose players thrive on a diet of intellectual

% %%% txt/389.txt
% engagement, tactical and strategic challenge, and well-
% developed roleplaying that never pushes players’
% buttons or puts them into harsh no-win situations.
% There’s a whole universe of responses in between these
% style sof play and none of them can conceivably be
% right for everyone. What matters to your campaign is
% what works for you and your players.

% Keep in mind as you talk about it with your group
% that more shock doesn’t equal more maturity. The
% prime audience for gore in film, for instance, is not
% well-aged men and women but teenage boys and
% young men. Shakespeare’s The Tempest is no less
% mature a tale than Macbeth even though it has a
% happy ending. It can be easy to confuse endurance
% with enlightenment, but in fact the two have nothing
% to do with each other. Endurance is about how much
% description of visceral nastiness the players can take
% (and deliver), while enlightenment (insofar as it ever
% happens in gaming) is about what insights players
% take away from whatever it is that happened in play.
% Don’t feel like a wimpy failure if you or your players
% would rather keep the darker parts of the game world
% suggested rather than delineated in hard-edged detail,
% since the point is that it be satisfying rather than it
% be as horrifying or mind-blowing as possible. The
% converse is also true: just as more is not better, so less
% is not better if your players do thrive on details. Your
% job as gamemaster includes knowing as much as you
% can about what it is your players actually prefer in
% this regard as in others and seeing how you can satisfy
% it in ways that are also satisfying for you.

% That said, there is one technique you really should
% never use without very clear permission from your
% players, and that’s playing on their real-world fears
% and phobias. If you know that one of them is, for
% instance, genuinely phobic about spiders, you can
% count on getting some real shivers by adding arachnid
% features to robots and morphs. You can also ruin a
% player’s enjoyment of the session or the whole cam-
% paign that way, if it comes unexpectedly and leads
% to the real-world fear drowning out the experience
% of play. Some players are fine with judicious use of
% their vulnerabilities, and others just aren’t. Under no
% circumstances should you poke at weak spots without
% making sure you’ve discussed it first.
% THE PROBLEM OF SECRETS
% Uncovering secrets is a big part of this game. There’s
% a problem, however, in that a lot of the secrets are out
% there where players can come across them: in this very
% chapter, in reviews of the game, discussion in online
% forums, and so on. As gamemaster, you will need to
% decide how you want to deal with the potential for
% spoiled revelations.

% As with so many potential issues, the place to start
% is with your players. Ask them how much it bothers
% them to know things that their characters are going
% to be finding out in play. Some players do a fine job
% separating their own knowledge from that of their
% characters with mental firewalls. Others have a very
% difficult time doing so, and knowing things in advance
% as players takes away a lot of the fun of character
% discovery for them. In addition, some players have
% a good sense of what degree of player-level surprise
% works best for them, and some don’t. Discuss it with
% them. Tell them that spoilers are available, and that
% you certainly can’t stop them from learning it all, one
% way or another. Ask them how much trouble this may
% be for them, and then proceed from there. Ask the
% players who have more trouble with spoilers to simply
% stay away from early commentary on the game, and
% tell them that you’ll let them know when the spoilers
% have come into play in your own campaign so that
% it’s no longer an issue. Ask the other players to work
% with you in keeping things fresh and fun for those
% players, too. In most groups, making it a matter of
% cooperation for the sake of everyone’s good time will
% draw out good responses. (If it doesn’t, the group may
% well have other problems in any event.)

% There’s a related question for both you and your
% players. How much do any of you mind when a par-
% ticular campaign’s version of an answer diverges from
% the stock one provided in print? There are two kinds
% of variation possible for this, and each one raises its
% own issues.

% There are matters that the game leaves unresolved,
% so that there is no single authoritative answer, like the
% number of TITANs in the solar system in the game’s
% present moment. If you choose to give a specific
% number, it’s your choice, and any number that seems
% to work for your campaign will probably do the job,

% %%% txt/390.txt
% whether it’s one, three, seven, a dozen, or something
% else. Your campaign can’t diverge from the baseline
% unless your answer is relatively extreme, like “there
% are no TITANs, it was all a hoax before contact with
% the Exsurgent virus and then purely alien technology
% after that.” In this case, your players can have read
% all the game’s secrets and still be surprised by the rev-
% elation you present. The potential for trouble here is
% not a conflict of expectations based on the game, but
% based on expectations raised in other contexts. Some
% games, like some movies, TV shows, and other stories,
% develop a following with strong ideas of its own about
% what the real truths and important matters are, and if
% the following thrives, its members may end up with
% ideas that have less and less to do with the original
% inspiration. This isn’t good or bad in itself, but it can
% be a problem, which is why it bears conscious consid-
% eration and discussion, both before play and as the
% campaign evolves. Ask your players to tell you about
% conversations and insights that shape their expecta-
% tions for the game world and storylines. Sometimes
% you’ll want to work those in with your own plans,
% sometimes you may want to deliberately play against
% them for the sake of a delightful surprise (generally
% more delightful for players than characters, but that’s
% life as a character for you). In either case, it’s better to
% be thinking about it than missing it.

% Then there are matters that the game does give
% definite answers for, but which you wish to change
% for the sake of your own campaign’s characters and
% stories. This is perfectly fine. There are no game police
% roaming the countryside and forcing you to accept
% answers you’d prefer not to use. But your players
% will, as with the first question, have expectations, and
% your campaign will work better if you make sure you
% understand what those expectations are. How much
% would it bother them if it turned out there were no
% TITANs and it was all a hoax, and so on? It’s hard to
% guess what friends will say and impossible to predict
% the range of responses strangers might give, so ask
% them. (This particular answer is one that’s unlikely to
% appear in anyone’s campaign, but it makes a handy
% example for your conversational use precisely because
% it’s extreme. So their answers to it are likely to be
% about the same as to any other potentially extreme
% change, and this one probably doesn’t give away any
% of your own plans.) Some players are flexible on most
% matters but have particular points of attachment;
% if yours are among them, ask them to explain what
% those points are for them, so that you can keep them
% in mind. Other players have a hard time having fun
% with any major shift from published standard answers,
% and if you have players like that, you’ll want to know
% it so that you can see how to adapt your plans to
% work within that framework.
% THAT’S WHAT I’M TALKING ABOUT:
% SHARED INSPIRATION
% It’s not quite true that everything changed from the
% early 21st century to Eclipse Phase’s universe, but a
% great many things did, and it can be hard to keep track
% of them all at once. This is where shared inspirations
% can come in handy. One striking illustration can convey
% a lot of details for both foreground and background,
% suggesting an aesthetic standard for design, an exotic
% environment, people doing futuristic tasks with appro-
% priately advanced tools, and so on. A prose passage
% from a rewarding novel may set an ambiance or nail
% down some aspect of the characters’ circumstances.

% There are potential pitfalls, and it’s important to
% be aware of them. The greatest is obsolescence, the
% meaning of something evocative changing because
% the players’ reality has changed since the inspiration
% entered it. William Gibson’s ground-breaking cyber-
% punk novel Neuromancer begins, “The sky above the
% port was the color of television, tuned to a dead chan-
% nel.” Supporting details make it clear that this is an
% industrial port at night, the sky gray from pollution
% and flecked with ash and other debris. But that was
% an image published in 1984. A decade and a half later,
% Neil Gaiman pointed out that to his children, the
% color of television tuned to a dead channel is bright
% blue, thanks to ubiquitous cable delivery. In another
% decade, the default color of a station not in use may
% be something else entirely. The moral is that it’s not
% enough to agree that an image is very striking. You’ll
% want to make sure that you all agree on what it is
% about it that’s striking, to avoid a tangle of miscon-
% ceptions that could derail play later on.

% The References page (p. 394) offers a wide range
% of immediately relevant inspirations, but it’s not the
% final word on the subject. If the people in your group
% have a long-time favorite space scene, or description
% of life in the midst of a high-tech investigation, or
% poetic glimpse of what it might feel like to modify the
% body in ways not possible in real life, or something
% else that’s stayed with them a long time and seems
% like it might bear on your campaign, encourage them
% to share. Remember to be courteous with each other’s
% personal treasures, whether you end up using them
% or not; there’s nothing like earned trust to encourage
% more sharing.

% Images can be particularly helpful for what they
% convey about the world behind and around the
% foreground events. For instance, think of a corridor
% on a typical spaceship or habitat in Eclipse Phase.
% Did you imagine it as being a standardized size and
% shape, so that its counterparts elsewhere would be
% very much the same or a more individualized work
% intended for use just where it is without concern
% for interchangeability? Did you imagine it as well
% lit even when not in use, lit well when sensors show
% people present and otherwise dim or dark, or perhaps
% planned to be well lit but in practice haphazard and
% unreliable thanks to lack of maintenance and funds?
% Did you imagine its surfaces smooth and clean, with

% %%% txt/391.txt
% equipment, maintenance bays, and the like all behind
% hatches and covers, or was it cluttered and lumpy?
% None of that matters all the time, but when it comes
% to the investigation of a derelict, the hunt for someone
% (or something) trying to hide, a race against time, or
% other dramatic complication, these things could affect
% your play, and rather than try to tally all possible con-
% tingencies in advance, having some general-purpose
% references can save everyone time and confusion.

% THINGS THAT SHOULD NOT BE: HORROR
% The universe of Eclipse Phase is a time of horrors
% unleashed. Every character has had to come to some
% personal accommodation with the existence of things
% that offend our basic expectations of decency and
% practicality all at once. Horror comes in many flavors,
% and no one campaign can make use of all of them.

% There are at least as many theories of horror as
% there are people who create horror stories. Everything
% here is necessarily a generalization. You and your
% players can find exceptions to every single point in
% it, and if you like the way those work better, go with
% them. This discussion is intended to trigger ideas, not
% to close off anything. That said, there are some useful
% generalizations about horror, starting with an insight
% expressed well by H.P. Lovecraft: “The oldest and
% strongest emotion of mankind is fear, and the oldest
% and strongest kind of fear is fear of the unknown.” All
% horror can be thought of as built around encounters
% with the unknown, beginning with the realization that
% there is something unknown present, learning some-
% thing about the scope of its nature and activities, and
% then trying to respond one way or another.

% In this game, the discovery part is half over. There’s
% no question about the presence of the unknown.
% Yes, there really are monsters beyond transhuman
% understanding loose in the universe, and everyone in
% the Eclipse Phase universe knows how bad and how
% strange the TITANs could get. Many people also have
% some idea of how exotic life on the far side of the Pan-
% dora Gates can be. There’s no room left for characters
% to respond to some new strangeness with confident
% skepticism, sure that they know the range of what’s
% possible and plausible within transhuman experience
% anymore. Almost anything might exist, given the
% facts of what’s already known. Instead, the question
% for Eclipse Phase people facing a mystery is whether
% this particular unknown will turn out to be simple
% and straightforward to deal with, more complicated
% but nonetheless a part of their routine lives like mal-
% functioning machinery or a sabotaged and unusually
% modified morph, or something beyond the normal like
% a TITAN-programmed weapon or alien life. Sooner or
% later, if they keep poking around, the characters can
% count on running into all sorts of unknown and maybe
% even unknowable challenges. Are they there yet?

% Horror is seldom very far from humor. Humor
% serves many roles in human psychology, and one of
% them is helping us whittle down the mental “size” of
% mysteries and threats to something we can deal with.
%  Furthermore, horror usually involves a balance of im-
%  probable elements, with things lined up to go wrong
%  in interesting ways, and it doesn’t take much for a
%  particular rickety edifice to go from strange and men-
%  acing to ludicrous. When your players start laughing,
%  sometimes the best thing for you to do is to roll with
%  it. Laughter can do everyone good, supporting the
% “play” part of roleplaying. In addition, some events
%  actually are funny or at least can be taken as funny,
%  even (sometimes especially) when most of what’s going
%  on is serious. On the other hand, if you really would
%  like to keep a scene serious and the players break out
%  in giggles, it’s often wise to go ahead and take a break.
% Tell the players what you’re doing, too; trying to de-
%  ceive a group of your friends isn’t very reliable and
%  can backfire badly. Make the break long enough for
%  everyone to get the giggles out and then continue.

%  At the end of the day, through communication with
%  their players, the gamemaster will know how much
%  horror their group wishes to encounter. A group may
%  decide that they want to be 100 percent immersed
%  into the various horrors of Eclipse Phase. Another
%  group, however, may decide that while they enjoy
%  the meshed theme of horror with the other aspects
%  of Eclipse Phase, they don’t wish it to be a principal
%  element. In such a situation, horror would remain just
%  that, a theme, while the plots woven by the gamemas-
%  ter would spin around the myriad of other elements
%  that make up the game.

% TRANSHUMANISM
% Humanity has embraced transhumanism for survival,
% harnessing science and technologies to catapult physi-
% cal and mental faculties to super-human levels, while
% eradicating involuntary death and enabling near im-
% mortality through the digitization of consciousness
% and the ability to transfer bodies at will. This is one of
% the cornerstone themes of Eclipse Phase.

% The technologies inherent to a transhuman future
% raise many questions and ethical issues, however, and
% these are some of the central themes that Eclipse
% Phase seeks to explore. We encourage both gamemas-
% ter and players to play around with the possibilities
% and contradictions enabled in such a universe. How
% do our mindsets change when death no longer looms
% over us? What does identity mean when our bodies
% are disposable and our personalities can be edited?
% Are we the same person when we are revived from
% a backup, or sent off as a fork? Are technologies like
% nanofabrication something to be feared and restricted,
% even when they can eliminate poverty and greed?
% How do we ensure public safety in a world where
% technology makes weapons of mass destruction easily
% available? How do ideas inherent to religious and
% spiritual thought cope with AI, backups, or resleev-
% ing? What does it mean to be an uplifted animal in a
% society centered on humans? Who decides our future?
% These are just a few of the issues that Eclipse Phase
% raises, and many of them can be used as the central
% theme for an entire campaign.

% %%% txt/392.txt
% TER CREATION                       BACKGROUN
%  Y                              Drifter: +10 Navigation sk


%                               Spacecraft skill, +10 Ne
% ter Concept (p. 130)               [Field] skill of your choi
%  ound (p. 131)                  Fall Evacuee: +10 Pilot: G
% n (p. 132)                         skill, +10 Networking: [
% ints (p. 134)                      your choice, +1 Moxie,


%                               Starting Credit (can still
% e points                           with CP)


%                            Hyperelite: +10 Protocol


%                               Credit, +20 Networking


%                               skill, may not start with
% ue                                 or any pod, uplift, or sy
% ization Points (p. 135)            morphs
% spend                           Infolife: +30 Interfacing s


%                               Computer skills (Infosec
% oxie                               Programming, Research
% titude point                       with Customization Poi
%  leight                            price, Real World Naive

% ialization                       Stigma (AGI) trait, may






%                          TABLES

% point (61-80)                    Psi trait, Social skills bo


%                               Customization Points ar

% point (up to 60)


%                            Isolate: +20 to two skills o
%  credit                            –10 starting Rep
% p                               Lost: +20 to two Knowled
% minimum: 400 skill points          your choice, Psi trait, M
% skill minimum: 300 skill points    (choose two) trait, Socia
% ting Morph (pp. 136 and 139)       (Lost) trait, must start w
%  s (pp. 136 and 145)               morph

% (p. 136)                      Lunar Colonist: +10 Pilot:


%                               skill, +10 to one Techni
% ation (p. 137)                     [Field], or Profession: [F
% aining Stats (p. 138)              of your choice, +20 Net
%  racter (p. 138)                   Hypercorps skill


%                            Martian: +10 Pilot: Groun


%                               +10 to one Technical, A


%                               [Field], or Profession: [F


%                               of your choice, +20 Net


%                               Hypercorps skill


%                     277    Original Space Colonist: +


%                               Spacecraft or Freefall sk
% STS (ALPHABETICAL)                 a Technical, Academic: [


%                               Profession: [Field] skill o

%  CP            COST             +20 to a Networking: [F


% 45       Expensive (40k+)     your choice


% 40          Expensive


%                            Re-instantiated: +10 Pilot


%  5          Moderate


%                               skill, +10 to a Networki


% 20            High


% 30          Expensive         skill of your choice, +2 M


%  0            High            Memories trait, 0 Starti


% 20       Expensive (30k+)     still buy credit with CP)


% 75       Expensive (40k+)  Scumborn: +10 Persuasio


% 40       Expensive (50k+)     skill, +10 Scrounging sk


% 70       Expensive (40k+)     Networking: Autonomi


% 25          Expensive      Uplift: +10 Fray skill, +10 P


%  0              0             +20 to two Knowledge s


% 40          Expensive         choice, must choose an u


% 25          Expensive         to start


% 25          Expensive


% 25          Expensive


% 60       Expensive (30k+)       137


% 50       Expensive (30k+)


% 40          Expensive                     GEAR COS


% 20            High                              RANGE

%  100       Expensive (50k+)   CATEGORY


%                                               (CREDITS)


% 60       Expensive (40k+)


% 25          Expensive      Trivial                1–99


% 40          Expensive      Low                  100–499


% 10            High


% 25          Expensive      Moderate            500–1,499


% 40          Expensive      High               1,500–9,999


% 30            High


% 20            High         Expensive            10,000+
% S               FACTIONS
% +20 Pilot:      Anarchist: +10 to a skill of your choice,
% rking:            +30 Networking: Autonomists skill


%            Argonaut: +10 to two Technical,
% ndcraft           Academic: [Field], or Profession:
% d] skill of       [Field] skills; +20 Networking:
%  2,500            Scientists
% y credit        Barsoomian: +10 Freerunning, +10


%              to one skill of your choice, +20

% +10,000         Networking: Autonomists skill
% percorps        Brinker: +10 Pilot: Spacecraft skill,
%  , splicer,       +10 to a skill of your choice, +20 to
% etic              a Networking: [Field] skill of your


%              choice


%            Criminal: +10 Intimidation skill, +30
%  erfacing,        Networking: Criminal skill
%  ught


%            Extropian: +10 Persuasion skill, +20
% are half


%              Networking: Autonomists skill, +10
%  ait, Social


%              Networking: Hypercorps skill
%  purchase
% t with          Hypercorp: +10 Protocol skill, +20
% ouble price       Networking: Hypercorps skill, +10 to


%              any Networking: [Field] skill
% our choice,


%            Jovian: +10 to two weapon skills


%              of your choice, +10 Fray, +20
% skills of


%              Networking: Hypercorps skill. must
% al Disorder


%              start with a Flat or Splicer morph,
% igma


%              may not start with any nanoware or
%  Futura


%              advanced nanotech


%            Lunar: +10 to one Language: [Field]
% oundcraft


%              of your choice, +20 Networking:
% Academic:


%              Hypercorps skill, +10 Networking:
%  ] skill


%              Ecologists skill
%  king:


%            Mercurial: +10 to any two skills of your


%              choice, +20 to a Networking: [Field]
% aft skill,


%              skill of your choice
% emic:
% ] skill         Scum: +10 Freefall skill, +10 to a skill
% king:             of your choice, +20 Networking:


%              Autonomists skill
% Pilot:          Socialite: +10 Persuasion skill, +10
% +10 to            Protocol skill, +20 Networking: Media
% d], or            skill, may not start with flat, pod,
% our choice,       uplift, or synthetic morphs
%  ] skill of     Titanian: +10 to two Technical or


%              Academic skills of your choice, +20
% oundcraft         Networking: Autonomists skill
%  Field]         Ultimate: +10 to two skills of your choice,
% ie, Edited        +20 to a Networking: [Field] skill of
%  redit (can       your choice, may not start with Flat,


%              Splicer, uplift, or pod morphs
%  Deception      Venusian: +10 Pilot: Aircraft, +10
%  20               to one skill of your choice, +20
% kill              Networking: Hypercorps skill
% eption skill,

% of your
%  t morph                                               135


%               CUSTOMIZATION POINTS


%            15 CP = 1 Moxie point


%            10 CP = 1 aptitude point


%             5 CP = 1 psi sleight
% AVERAGE
% (CREDITS)        5 CP = 1 specialization


% 50          2 CP = 1 skill point (61-80)


% 250         1 CP = 1 skill point (up to 60)

%  1,000        1 CP = 1,000 credit

%  5,000        1 CP = 10 Rep

%  20,000      Trait and morph costs vary as noted.

% %%% txt/393.txt


%                      176–185




%                                             LINKED


%                          SKILL                               CAT


%                                            APTITUDE


%               Academics: [Field]               COG          Kno


%               Animal Handling                  SAV         Activ


%               Art: [Field]                      INT         Kno


%               Beam Weapons                     COO        Active


%               Blades                           SOM        Active


%               Climbing                         SOM        Active


%               Clubs                            SOM        Active


%               Control                       WIL (no def) Active,


%               Deception                        SAV         Activ


%               Demolitions                   COG (no def) Active


%               Disguise                          INT       Active


%               Exotic Melee Weapon: [Field]     SOM        Active


%               Exotic Ranged Weapon: [Field]    COO        Active


%               Flight                           SOM        Active


%               Fray                              REF       Active


%               Free Fall                         REF       Active


%               Freerunning                      SOM        Active


%               Gunnery                           INT       Active


%               Hardware: [Field]                COG        Active


%               Impersonation                    SAV         Activ


%               Infiltration                      COO        Active


%               Infosec                       COG (no def) Active


%               Interest: [Field]                COG          Kno


%               Interfacing                      COG        Active


%               Intimidation                      SAV        Activ
%  145–152


% POSITIVE TRAITS                                      CP COSTS
% Adaptability                                  10 (Level 1) or 20 (Level
% Allies                                                     30
% Ambidextrous                                               10
% Animal Empathy                                              5
% Brave                                                      10
% Common Sense                                               10
% Danger Sense                                               10
% Direction Sense                                             5
% Eidetic Memory (Ego or Morph Trait)                        10
% Exceptional Aptitude                                       20
% Expert                                                     10
% Fast Learner                                               10
% First Impression                                           10
% Hyper Linguist                                             10
% Improved Immune System (Morph Trait)          10 (Level 1) or 20 (Level
% Innocuous (Morph Trait)                                    10
% Limber (Morph Trait)                          10 (Level 1) or 20 (Level
% Math Wiz                                                   10
% Natural Immunity (Morph Trait)                             10
% Pain Tolerance (Ego or Morph Trait)           10 (Level 1) or 20 (Level
% Patron                                                     30
% Psi                                            20 (Level 1), 25 (Level 2
% Psi Chameleon (Ego or Morph Trait)                         10
% Psi Defense (Ego or Morph Trait)              10 (Level 1) or 20 (Level
% Rapid Healer (Morph Trait)                                 10
% Right At Home                                              10
% Second Skin                                                15
% Situational Awareness                                      10
% Striking Looks (Morph Trait)                 10 (Level 1) or 20 (Leve
% Tough (Morph Trait)                    10 (Level 1), 20 (Level 2), or 30
% Zoosemiotics                                                5
% SKILL LIST


%                                      LINKED
%  RY                  SKILL                               CATEGORY


%                                     APTITUDE
% ge        Investigation                      INT        Active, Mental
% cial      Kinesics                           SAV        Active, Social
% ge        Kinetic Weapons                   COO        Active, Combat
% mbat      Language: [Field]                  INT         Knowledge
% mbat      Medicine: [Field]                 COG        Active, Technical
% sical     Navigation                         INT        Active, Mental
% mbat      Networking: [Field]                SAV        Active, Social
% al, Psi   Palming                           COO        Active, Physical
% cial      Perception                         INT        Active, Mental
%  nical    Persuasion                         SAV        Active, Social
% sical     Pilot: [Field]                     REF        Active, Vehicle
% mbat      Profession: [Field]               COG          Knowledge
% mbat      Programming                    COG (no def) Active, Technical
% sical     Protocol                           SAV        Active, Social
% mbat      Psi Assault                    WIL (no def) Active, Mental, Psi
% sical     Psychosurgery                      INT       Active, Technical
% sical     Research                          COG        Active, Technical
% mbat      Scrounging                         INT        Active, Mental
%  nical    Seeker Weapons                    COO        Active, Combat
% cial      Sense                          INT (no def) Active, Mental, Psi
% sical     Spray Weapons                     COO        Active, Combat
%  nical    Swimming                          SOM        Active, Physical
% ge        Throwing Weapons                  COO        Active, Combat
%  nical    Unarmed Combat                    SOM        Active, Combat
% cial




%  TRAITS


%       NEGATIVE TRAITS                                          CP C


%       Addiction (Ego or Morph Trait)             5 (Minor), 10 (Mod


%       Aged (Morph Trait)                                             1


%       Bad Luck                                                       3


%       Blacklisted                                                  5o


%       Black Mark                                 10 (Level 1), 20 (Lev


%       Combat Paralysis                                               2


%       Edited Memories                                                1


%       Enemy                                                          1


%       Feeble                                                         2


%       Frail (Morph Trait)                              10 (Level 1) o


%       Genetic Defect (Morph Trait)                                10 o


%       Identity Crisis                                                1


%       Illiterate                                                     1


%       Immortality Blues                                              1


%       Implant Rejection (Morph Trait)                   5 (Level 1) o


%       Incompetent                                                    1


%       Lemon (Morph Trait)                                            1


%       Low Pain Tolerance (Ego or Morph


%                                                                     2


%       Trait)


%       Mental Disorder                                               1


%       Mild Allergy (Morph Trait)


%       Modified Behavior                           5 (Level 1), 10 (Lev


%       Morphing Disorder                          10 (Level 1), 20 (Lev


%       Neural Damage                                                  1


%       No Cortical Stack (Morph Trait)                                1


%       Oblivious                                                      1


%       On the Run                                                     1


%       Psi Vulnerability (Ego or Morph Trait)                         1


%       Real World Naiveté                                             1
%  el 3)


%       Severe Allergy (Morph Trait)                  10 (uncommon)


%       Slow Learner                                                   1


%       Social Stigma (Ego or Morph Trait)                             1


%       Timid                                                          1


%       Unattractive (Morph Trait)                  10 (Level 1), 20 (L


%       Uncanny Valley (Morph Trait)                                   1


%       Unfit (Morph Trait)                                10 (Level 1)


%       VR Vertigo                                                     1


%       Weak Immune System (Morph Trait)                 10 (Level 1) o


%       Zero-G Nausea (Morph Trait)                                    1

% %%% txt/394.txt
%  SUMMARY                                          ACTION TUR
% dled as an Opposed Test.                          Step 1: Roll Initiative (
% attack skill +/– modifiers.                        Step 2: Begin First Act
%  er rolls Fray or melee skill                     Step 3: Declare and Re


%                                              Step 4: Rotate and Rep
%  der rolls (Fray skill ÷ 2,

% /– modifiers.                                    SCATTER DIA
%  eeds and rolls higher than
%  he attack hits.                                     204

% armor-defeating (armor                                              1 or 2
%  ).


%                                                          10
%  ed by the attack’s Armor
%  lue (AP).
% damage is reduced by the                                  9
%  ed Armor rating (unless
%  mor-defeating).
%  exceeds the target’s                                          8
%  old, a wound is also scored.                                          6 or 7

% exceeds the Wound
% multiple factors, multiple
% flicted.)




%       WEAPON RANGES


%          SHORT           MEDIUM       LONG       EXTREME


%          RANGE          RANGE (–10) RANGE (–20) RANGE (–30)



%            0–10            11–25               26–40             41–60


%            0–10            11–30               31–50             51–70


%            0–10            11–35               36–60             61–80


%            0–30            31–80              81–125            126–230


%           0–150           151–250            251–500            501–900


%           0–180           181–400           401–1,100         1,100–2,300


%           0–100           101–400           401–1,000         1,001–2,000

%  ease the effective range in each category by +50%

% ser            0–30              31–80             81–125           126–230


%           0–30             31–100            101–150           151–250


%            0–5              6–15              16–30             31–50


%           0–30             31–100            101–150           151–300


%           0–20             21–50             51–100            101–300


%           0–10             11–25              26–40             41–60



%            5–70            71–180           181–600          601–2,000


%           5–150           151–300          301–1,000        1,001–3,000
% ssile          5–300          301–1,000         1001–3,000       3001–10,000



%            0–5               6–15             16–30              31–50


%            0–5               6–15             16–30              31–50


%           0–10              11–30             31–50              51–70


%           0–10              11–40             41–70             71–100


%           0–5               6–15              16–30             31–50


%           0–5               6–15              16–30             31–50


%           0–5               6–15              16–30             31–50



%        To SOM ÷ 5        To SOM ÷ 2          To SOM           To SOM x 2


%        To SOM ÷ 2          To SOM          To SOM x 2         To SOM x 3


%        To SOM ÷ 5        To SOM ÷ 2          To SOM           To SOM x 3




%                                                                   HEALIN
% R SITUATION
% hout basic biomods
% h basic biomods
%  g nanobandage
% h medichines
%  s (bad food, not enough rest/heavy activity, poor shelter and/or sanitation)
%  ns (insufficient food, no rest/strenuous activity, little or no shelter and/or sanita


%                          115          MODIFIER SEVERITY


%                                     SEVERITY               MODIFIER
%  + REF) x 2) + 1d100


%                                     Minor                     +/– 10
% Phase (Speed 1)                          Moderate                  +/– 20
% ve Actions                               Major                     +/– 30
% t (Speed 2-4)


%                                        TEST DIFFICULTY
% GRAM                         115


%                                   DIFFICULTY LEVEL MODIFIER


%                                   Effortless                      +30


%                                   Simple                          +20


%                                   Easy                            +10


%                                   Average                         +0


%                                   Difficult                        –10


%                                   Challenging                     –20


%                                   Hard                            –30

%  4


%                  COMPLEMENTARY SKILL BONUS


%        123


%                    SKILL RATING                     MODIFIER


%                          01–30                          +10


%                          31–60                          +20


%                           61+                           +30




%                   COMBAT MODIFIERS
% 3

%  GENERAL                                                 MODIFIER

%  Character using off-hand                                     –20

%  Character wounded/traumatized                       –10 per wound/trauma

%  Character has superior position                              +20

%  Touch-only attack                                            +20

%  Called shot                                                  –10

%  Character wielding two-handed weapon with one hand –20

%  Small target (child-sized)                                   –10

%  Very small target (mouse or insect)                          –30

%  Large target (car sized)                                     +10

%  Very large target (side of a barn)                           +30

%  Visibility impaired (minor: glare, light smoke, dim light)   –10

%  Visibility impaired (major: heavy smoke, dark)               –20

%  Blind attack                                                 –30

%  MELEE COMBAT                                               MODIFIER

%  Character has reach advantage                                +10

%  Character charging or receiving a charge                     +20

%  RANGED COMBAT (ATTACKER)                                   MODIFIER

%  Attacker using smartlink or laser sight                      +10

%  Attacker behind cover                                        –10

%  Attacker running                                             –20

%  Attacker in melee combat                                     –30

%  Attacker has reach advantage                                 +10

%  Defender has minor cover                                     –10

%  Defender has moderate cover                                  –20

%  Defender has major cover                                     –30

%  Defender prone and far (10+ meters)                          –10

%  Defender hidden                                              –60

%  Aimed shot (quick)                                           +10

%  Aimed shot (complex)                                         +30

%  Sweeping fire with beam weapon                        +10 on second shot

%  Multiple targets in same Action Phase             –20 per additional target

%  Indirect fire                                                 –30

%  Point-blank range (2 meters or less)                         +10

%  Short range                                                   —

%  Medium range                                                 –10

%  Long range                                                   –20

%  Extreme range                                                –30





%    DAMAGE HEALING RATE                   WOUND HEALING RATE


%           1d10 (5) per day                         1 per week


%        1d10 (5) per 12 hours                       1 per 3 days


%         1d10 (5) per 2 hours                        1 per day


%         1d10 (5) per 1 hour                      1 per 12 hours


%          double timeframe                       double timeframe


%           triple timeframe                      no wound healing

% %%% txt/395.txt
% THE HACKING SEQUENCE
% TASKS
% 1. Defeat the Firewall
% 2. Bypass Active Security

% a. Hacker Wins with Excellent Success, Defender Fail

% b. Hacker Succeeds, Defender Fails

% c. Both Succeed

% d. Defender Succeeds, Hacker Fails


% ONLINE SEARCHES
% 1. Common data = automatic acquisition
% 2. Uncommon data:

% a. Research Task Test (timeframe: 1 minute) modifie

% data obscurity to accumulate data

% b. Measure of Success determines depth of data fou
% 3. Analyzing data:

% a. Research Task Test (timeframe: GM call) using

% complementary skill to understand data


%                                                                       24


%                MESH GEAR MODIFIERS
%  MODIFIER SOFTWARE/HARDWARE


%            Bashed-up devices, no-longer-supported software, relics from


%  –30


%            Earth or the early expansion into space


%  –20       Malfunctioning/inferior devices, buggy software, pre-Fall techn


%  –10       Outdated and low quality systems


%   0        Standard ectos, mesh inserts, and software


%  +10       High-quality goods, standard security-grade products


%  +20       Next-generation devices, advanced software


%  +30       Newly-developed, state-of-the-art, top-of-the-line technology


% >+30       TITANs and/or alien technology


%                                                                       25


%               SUBVERSION DIFFICULTIES


%                Difficulty modifiers for common computer tasks
%  MODIFIER TASK


%            Execute commands, view restricted information, run restricted


%  –0        software, open/close connections to other systems, read/write


%            delete files, access sensor feeds, access slaved devices


%  –10       Change system settings, alter logs/restricted files


%  –20       Interfere with system operations, alter sensor/AR input


%  –30       Shut system down, lockout user/muse, launch countermeasures a



%                                                                       25


%                   COUNTERMEASURES
%  Passive Alert (-10 modifier to intruders)
%  Locate Intruder: Opposed Infosec Test; if successful, intruder becomes Lock
%  Re-authenticate Users: Next Action Turn, intruder must make Infosec Test

% log in again
%  Reduce Privileges: Limit user access privileges; see p. 246
%  Active Alert (-20 modifier to intruders)
%  Counterintrusion: If Trace (see below) is successful, launch intrusion attem

% on intruder's home system
%  Lockout: Opposed Infosec Test; if successful, intruder dumped from system.
%  Reboot/Shutdown: Takes 1 Action Turn to 1 minute (GM discretion); all use

% ejected from system.
%  Trace: Trace intruder to home system with a Research Test (-30 if in privacy m
%  Wireless Termination: At end of Action Turn, all wireless connections term

% nated; wireless users ejected.


% RESULTS


% Infosec Task Action (10 minutes)


% Opposed Infosec Test


% Hidden status/admin privileges/+30 all Subversion T


% Covert Status (p. 256)


% Spotted status/passive alert/-10 all Subversion Tests


% Locked status/active alert/-20 all Subversion Tests (p




% y



%                              SUBVERSION EXAMPLES


%               In addition to the tasks noted under the Subversion Diffic


%                    these modifiers present some additional example ac


%    MOD TASK


%        Hacking Bots/Vehicles


%     –0    Give orders to drones


%     –10   Alter sensor system parameters, disable sensors or weapon


%     –20     Alter smartlink input, send false data to AI or teleoperator


%     –30     Lockout AI or teleoperator, seize control via puppet sock


%        Hacking Ectos/Mesh Inserts


%           Interact with entoptics, befriend everyone in range, make on


%     –0


%           user’s credit, intercept communications, log activity


%           Alter social network profile/status, adjust AR filters, tweak s


%     –10


%           change AR skin, change avatar, access VPN


%           Block or shuffle senses, inject AR illusions, spoof command


%     –20


%           devices


%     –30   Boot user out of AR


%          Hacking Habitat Systems


%     –0       Open/close doors, stop/start elevators, operate intercom


%             Adjust temperature/lighting, disable safety warnings, replac


%     –10


%             doors, switch traffic timers


%             Disable subsystems (plumbing, recycling, etc.), disable wire


%     –20


%             repair crews


%     –30     Override safety cutoffs


%          Hacking Security Systems
% y/       –0      Move/manipulate cameras/sensors, locate security systems/


%     –10     Adjust patterns of sensor sweeps, view security logs, disable


%     –20     Delete security logs, dispatch security teams
% rs       –30     Disable alerts


%          Hacking Simulspace Systems


%     –0      View current status of simulspace, simulmorphs, and access


%             Change domain rules, add cheats, alter parameters of story,


%     –10


%             change time dilation


%     –20     Eject simulmorph, alter/erase character AIs


%     –30     Abort simulation


%          Hacking Spimes


%     –0      Get status report, use device functions


%     –10     Adjust AI/voice personality settings, adjust timed operation


%     –20     Disable sensors, disable device functions


%       Hacking Simulspaces From Within


%           Analyze simulation parameters, view domain rules, shape ap


%     –0


%           morph, switch simulmorph character or morph type


%           Change probability of test outcomes, become invisible


%     –10


%           (“out-game”) to others


%           Interfere with simulation (e.g. make it rain, generate earthq


%     –20


%           ignore domain rules, kill or lockout other simulmorphs


%     –30   Go into god mode, command simulated characters, take ove

% %%% txt/396.txt
%  S
%  ows liberally from many sources, which deserve recognition
% ventures and campaigns. Further resources can be found on


%               Frederick Pohl


%                  Gateway                         COMI


%               Alastair Reynolds


%                  Absolution Gap                  AND


%                  Chasm City                      Jamie D
% ons                   The Prefect                        Nar
% mes                   Pushing Ice                     Warren
% Art                   Redemption Ark                     Dok


%                  Revelation Space                   Doo


%               Kim Stanley Robinson                  Glob
% rd                    The “Mars Trilogy”                 Min


%                     Red Mars                        Oce


%                     Blue Mars                       Tran


%                     Green Mars                   Jonatha


%                  The Martians                       Tran


%               Karl Schroeder                     Grant M


%                  Ventus                             The


%               Dan Simmons                           The
%  es                   Endymion                        Masam


%                  Fall of Endymion                   Gho


%                  Llium                              Gho


%                  “Hyperion Cantos”                     H


%                     Hyperion                        Gho


%                     Fall of Hyperion                   M


%                  Olympos                         Adam W
%  e Magic Kingdom   Neal Stephenson                       Iron
%  be                   Diamond Age                     Makoto


%               Bruce Sterling                        Plan


%                  Caryatids


%                  Crystal Express


%                  Holy Fire


%                                                  NON-


%                                                  Ronald


%                  Schismatrix Plus                   Libe


%               Charles Stross                     Susan B


%                  Accelerando                        The


%                  Glasshouse                      Cynthia


%                  Halting State                      Desi
%  e”                   Iron Sunrise                    David B


%                  Singularity Sky                    The


%                  Toast                           Richard
% apsody             Karen Traviss                         Viru


%                  City of Pearl                         T


%               Vernor Vinge                        James
% h Saga”               Across Realtime                    Resi


%                  A Deepness in The Sky           Rodney


%                  A Fire Upon The Deep               Flesh
% Trilogy”              Rainbow’s End                         H


%                  True Names and Other Dangers       Cam
% der                Elisabeth Vonarburg                      T


%                  Slow Engines of Time            Critical


%               Peter Watts                           Digi
% ear                   Blindsight                         Elec


%                  “Rifters’ Trilogy”                 The
% n” series                Starfish                        Flesh


%                     Maelstrom                       The


%                     Behemoth (ß-Max + Seppuku)      The
%  ion               Scott Westerfeld                   Richard


%                  The Risen Empire                   The


%                  The Killing of Worlds           K. Eric


%               Walter Jon Williams                   Engi
% s” series             Aristoi                               T


%                  Angel Station                   Freema


%                  Voice of the Whirlwind             Dist


%               David Zindell                         Ima


%                  The Broken God                  Ann Fin


%                  Neverness                          The


%                  War in Heaven.                     Ima


%                  The Wild                        Joel Ga


%                                                     Rad


%                                                  Adam G


%                                                    Every


%                                                        U




%  ences


%                                                  James H


%                                                     Citiz
%  d credit. Gamemasters may also find them a good source of

% website: http://eclipsephase.com


%                          Ray Kurzweil


%                             The Singularity is Near

% APHIC NOVELS                  Howard Rheingold


%                             Smart Mobs: The Next Social
%  o                            Revolution
%  is                           John Robb


%                             Brave New War
%  eepless                      Clay Shirky
% 99                               Here Comes Everybody
%  equency                      Bruce Sterling
% of Space                         Shaping Things


%                             Tomorrow Now: Envisioning the
%  ropolitan                          Next Fifty Years
%  ckman                        Gregory Stock
% man                              Redesigning Humans: Our Inevitable
%  son                                Genetic Future


%                          Simon Young
%  ibles                           Designer Evolution: A Transhumanist

% hirow                             Manifesto
%  the Shell
%  the Shell 1.5:
% n-Error Processor             ROLEPLAYING GAMES


%                          Blue Planet
%  the Shell 2:
% Machine Interface             Burning Empires

% n                           Call of Cthulhu
%  : Hypervelocity              CthulhuTech

% imura                       Cybergeneration


%                          Dawning Star


%                          Delta Green

%  TION                         FreeMarket


%                          Gamma World
%  y                            GURPS: Transhuman Space
% n Biology                     Morrow Project
% more                          Paranoia
%  e Machine                    Shadowrun
% zeal                          Shock: Social Science Fiction
% g Sociable Robots             Traveller

% sparent Society
% die                           MOVIES AND TELEVISION
%  he Mind:                     Aeon Flux
%  ew Science of the Meme       AI
% k and Ian Boal (eds)          Alien series
%  the Virtual Life             Andromeda
% oks                           Babylon 5

% Machines:                   Big O
% Robots Will Change Us         Blade Runner
% n Intelligence:               Cowboy Bebop
% arly History of the New AI    Crusade
% Ensemble                      District 9
% esistance                     Dollhouse
% c Civil Disobedience          Dreamcatcher
%  ronic Disturbance            Event Horizon
% chine                         Ergo Proxy
%  cular Invasion               Firefly
%  hing Plague                  Gattica
% wkins                         Ghost in the Shell
%  h Gene                       Ghost in the Shell: Innocence
%  ler                          Ghost in the Shell: Stand Alone Complex
%  f Creation:                     Solid State Society
% oming Era of Nanotechnology   Ghost in the Shell: Stand Alone Complex
% son                           Ghost in the Shell: Stand Alone Complex
%  g the Universe                  2nd Gig

% Worlds                      The Island
% ner                           Jekyll

% s                           Moon
% y Weapons                     Pandorum


%                          Planetes

% volution                    Serenity

% field                       Sleep Dealer

% The Dawning Age of         Solaris
%  itous Computing              Stargate and Stargate: Atlantis
%  es                           Sunshine
%  yborg                        Uzumaki


%                          Zardoz

% %%% txt/397.txt
% A                                                   values, 312                           Bu
% Aarhus, 106                                         vest, 312                             Bu
% Ablative patches, 313                           Arm slide, 341-42                         Bu
% Abramsen, 107                                   Art skill, 177                            Bu
% Academics skill, 176-77                         Ashoka, 94                                Bu
% Access control, 291-92                          Asphyxiation, 194                         Bu
% Access jacks, 306                               Asteroid belt                             Bu
% Access privileges, reducing, 257                    Ceres, 97                             Bu
% Account access, 253                                 Extropia (44 Nysa), 97
% Accushot, 338                                       habitats, 97                          C
% Acquire                                             Nova York (Metis), 97                 Ca


% information, 289-90                            resources, economics, 97              Ca


% services, 289-90                           Async, 126                                Ca


% unload goods, 289                              roleplaying, 221                      Ca
% Action, 189-90                                  Atavism, 212


% complications, 193-206                     Aten, 88


% types, 119-20                              Atlas (Volkograad), 103
% Action turns, 119, 127                          Atmospheric contamination, 200            Ca


% declare, resolve actions, 188              Attack declaration, 191                   Ca


% first phase, 188                            Attention deficit hyperactivity disorder   Ca


% roll initiative, 188                       (ADHD), 212                               Ca


% rotate and repeat, 188                     Augmented reality illusions, 259-60       Ca
% Active alert, 257                               Authentication, 253                       Ca
% Active monitoring, 253                              circumvention, 254-55                 Ca
% Active psi, 221                                     forging, 255                          Ca
% Active skill, 172                                   spoofing, 255                          Ca
% Adaptability, 145                               Authophagy, 212                           Ce
% Addiction, 148, 212, 317-18                     Automatic actions, 119, 190               Ch
% Admin accounts, 247                             Automatic rifles, 335
% Adrenal boost, 301-2                            Automech, 345
% Aerogel, 298                                    Autonomist Alliance, 57, 76-79
% Aged, 148                                       Avatars, 239                              Ch
% AGI (artificial general intelligence), 81, 236   Avoidance, minor, 210


% attitudes toward, 48


% infolife and, 244                          B


% infomorphs and, 264-65                     Babylon, 111


% non-standard AI and, 245                   Backdoor, 260
% AI (artificial intelligence), 236                Background, 120, 131-32


% commanding, 264                            Backup, 330, 359


% infolife and, 244                              complications, 270-71


% limitations, 264                               cortical stack, 268


% non-standard, 245                              insurance, 269-70, 330-31


% subversion, 364                            Bad luck, 148
% Aimed shots, 193                                Bananas furiosas, 319-20
% Aiming, 190                                     Barsoomian, 79-80, 132-33
% Airburst, 199                                   Basilisk hacks, 364-66
% Aircraft, 342-43                                Battle suit, 344






%                                                                                      I
% AIs, 331-32                                     Beam weapons, 194, 203, 338-39
% Alienation, 225-26                                  skill, 177


% modifiers, 272                              Bedlam, 332                               Ch


% test, 272                                  Behavioral control, 231
% Aliens, 40                                      Behavioral masking, 231


% mindset, 376                               Beta forks, 273-74


% psi sleights, 222                          Big Circle Gang, 84
% Allergy                                         Bioconservatives, 80


% mild, 150                                  Biological, 317


% severe, 151                                    functions, lack of, 143
% Allies, 145                                         nanovirus, 363
% Alpha, 320                                      Biometric                                 Ch
% Alpha forks, 273                                    lock, 291                             Ch
% Amathea (Solano), 98                                scan, 253                             Ch
% Ambassadors, 376-77                                 tracking, 251                         Ch
% Ambelina, Claudia, 84                           Biomods, basic, 300                       Ch
% Ambidextrous, 145                               Biomorph, 139-42, 369-70                  Ch
% Ambience sense, 223                                 healing, 208                          Ch
% Ammunition, 337                                     resleeving, 271                       Ch


% reloading, 193                             Bioware, 301-306                          Ch
% Analysis software, 205-6                            synthmorph and, 306                   Ch
% Anarchism, 57, 77, 132                          Bioweave armor, 302-03                    Ch
% Animal                                          Bipolar disorder, 212                     Ci


% empathy, 145                               Biter, 338                                Ci


% handling skill, 177                        Blacklisted, 149                          Ci


% targets, 221-22                            Black mark, 149                           Ci
% Anonymization, 252-53                           Black market morphs, 277                  Cl
% Anonymous                                       Blackout, major, 211                      Cl


% accounts, 330                              Blades, 334                               Cl


% account services, 252-53                       skill, 177                            Cl
% Anti-electronics sleight, 372                   Blast weapons, 193                        Cl
% Anti-glare, 306                                 Bleeding, 208                             Cl
% Anti-matter rocket (AM), 347                    Blind attacks, 194                        Cl
% Anxiety, minor, 210                             Blueprints, 328                           Cl
% Aphrodite Prime, 90                                 open source, 284
% Aptitude, 122-23, 127, 172                          programming, 284-85                   Cl


% customizing, 135                               restrictions, 284                     Co


% improving, 152                             Bluewood, 111                             Co


% maximums, 124                              Body                                      Co


% -only tests, 174-75                            armor, 312                            Co


% range, 174                                     bank, 331                             Co


% starting, 135                                  dysmorphia, 212                       Co
% AR (augmented reality), 236, 239-40                 sculpting, 309


% games, 53-54                               Borderline personality disorder, 212


% illusions, 331                             Bot, 195-96


% intrusions, 49-50                              -pod rental, 331
% Arachnoids, 143                                     -vehicle AI, 331
% Arcadia, 111                                    Bouncers, 140
% Area effect weapons, 193                        Bounty hunters, 60
% Argonauts, 79, 132                              Brainwave scans, 279
% Armor, 194, 310, 312                            Brave, 145                                Co


% bypassing, 197                             Breadcrumb positioning system, 332        Co


% clothing, 312                              BringIt, 319                              Co


% modification, 192, 313                      Brinkers, 80, 133                         Co


% penetration, 194                           Brute-force hacking, 256                  Co


% -piercing, 336                             Brute strength, 175                       Co


% repair, 209                                BTX2, 323                                 Co
%  rlds, 281                     management, 205                 Diamond axe
%  336-37                   Complementary skills, 173            Dice, 22, 114
% ers, 291                  Complex actions, 120, 190            Diffusion sle
% kill, 179                 Composure, 175                       Digital activi
% er, 347                   Computer capabilities, 247           Digital ID
% (BF), 198                 Concentrated fire, 194                     code, 28


%                      Concussion, 340                           tamperin
% 40                        Cone weapons, 193                    Digital virus,


%                      Confusion, moderate, 211             Dione (Thoro


%                      Conservatives, 359                   Direct Action
% ots, 196-97               Conspiracy theme, 19                 Direction sen
%  9-100                    Continuity                           Disabler, 316

% , 88                         stress, 273                     Disarming, 1
% ns                             test, 272-73                    Disassemble
%  ate, 23                  Control skill, 178                   Disassembly
%  lt, 22-23                Core skills, 172                     Discord Gate
%  g, 22                    Cortical stack, 300                  Disguise skil
% 88                             backups, 268                    Disorders, 20
% 337                            destruction, 268                Disorientatio
%  armor, 303                    feed, termination, 262          Disposable la
%  3                             retrieval, 268                  Distance

% rce repulsion, 372      Cosmetic mods, 309                        lag, 248
%  , 365                    Counterintrusion, 257                     real and

% stupor, 365             Countermeasures, active, 293         Dizziness, mi
% thrown objects, 175       Courier, 347                         DMSO, 323
% r, 332                    Covert intruder, 256                 Domain rule


%                      Covert operations tool (COT), 315    Downtime, 2
%  n                        Covert technologies, 315-17          Dr. Bot, 345
%  315                      Cowboys, 359                         Dragonfly, 14
% ng, 313                   CR Gas, 324                          Drifter, 131
% 303                       Cranial computer, 300                Drive, 226, 3


%                      Crashing, 196                        Drug glands,

% cement, 152-53          Crash suit, 312                      Drugs
% 265                       Credit, 125, 135                          cognitive
%  pt, 120-21                    buying more, 136                     combat,
%  , 125                         making, 153                          health, 3
% 121                       Creepers, 369                             recreatio
% 125-26                    Creepy, 345                               social, 3
%  ty, 124                  Crime, 23                                 substanc
%  nts, 126                 Criminal, 133                        Dueling, 54
% h, 123                    Critical success/failure, 116, 189   Durability (D
%  ations, 120-21                psi, 223                        Dwarfs, 345

% attributes of, 124      Crows, 358
% 26                        Cryokinesis, 372                     E
% ation, 125                Cuffband, 316                        Earth, 24

% 122                     Cultural experimentation, 44              Fresh Kil

% networks, 124-25       Cultural regions, 44                      habitats,




% NDEX
%  121-23, 138              Culture, 41                               history a

% 123, 145-52             Customization points, 135-36              Paradise

% generation              Customized morphs, 277                    populati

% round, 131-32           Customs, 282-83                           Vo Nguy
%  pt, 130-31               Cyberbrain, 300                      Echo, 109-10
% mization points, 135-36        evacuation, 271                 Echolalia, mi
%  ing, 138                      hacking, 261-62, 364            Echolocation
%  n, 132-34                Cyberclaws, 307                      Echopraxia, m
%  oints, 134-35            Cyberlimb, 307                       Ecologene, 7
% purchasing, 136-37        Cyberlimb plus, 308                  Economy,
%  ations, 137-38           Cycle, 344                                new, 62-
%  ceiving, 197                                                       old, 61

% 197                     D                                         transitio

% 226                    Damage                               EcoWave, 12
% 95                            bonus, 138                       Ecto, 50, 237

% 108                        bonus stat, 122, 138             Edited memo

% 263                         determination, 192               Eelware, 304

% , 323                       initiative, 189                  Ego, 24-25,
%  sniffer, 311, 293            points, 207                           bridges,
%  substances, 317              types, 207                            hunters,
%  derate, 211                  value, 207                            scan, 25
% ty, 82-83                 Dang Fish Echo, 103                       sense, 2

% regulation, 304         Danger sense, 145                         stats, 12
%  st, 125                  Dangerous technologies, 63           Egocasting, 2
%  p, 63                    Darkcasting, 243, 276                Eidetic mem
% 125                       Darkly Selving, 322                  Electrical sen
% masses, 66                Dazzler, 316                         Electrogravit
%  4                        DDR, 323                             Electromagn
%  329                      Dead switch, 306                     Electronic ar
% tabolism, 304             Death, 50-51, 207, 208               Electronic dr
% skill, 177                    rating (DR), 121, 138            Electronic lo
%  59                           uploading after, 269             Electronic ro
%  mory, 226                Deception skill, 178                 Electronic se
% 4                         Deep learning, 231                   Elysium, 95
% 178                       Deep scan, 226                       Emergency
%  lony, 281                Defaulting, 116, 127, 173                 bubble, 3
% 70                        Defense                                   distress

% boost, 223                  declaration, 191                      farcaster

% drugs, 318, 320             full, 198                        Emotional co
% bles, 281                 Degeneration, 324                    Emotional da

% 196                    Delayed actions, 189                 Emotions, 18


%                      Delphinium Six, 322                  EMP, 340
%  ications, 193-206        Delta forks, 273                     Empathic sca
% , 319, 320                Demolitions, 197                     Enceladus (P
% fiers, 193                     skill, 178                       Encryption, 2
%  sis, 149                 Densiplast gloves, 334               Encumbranc
%  ing, 191-92              Depression, 213                      Enemy, 149
%  ship, 284                Derangements, 210-11                 Energy armo
%  ary, 206                 Dermal application, 317              Energy dama
% ts, 205-6                 Desktop cornucopia machine, 327      Engineers, 3
% sensor systems, 303       Destroyer, 347-48                    Enhanced cr
% omet Express), 70-71      Destructive uploading, 269           Enhanced ph
% 320                       Detailed perception, 190             Entertainmen
%  , issuing, 248           Detailing, character, 138            Entrapment,
% information, 290          Detection, 293                       Environment
% sense, 145                Device AI, 331                       Epimethus, 1
% cations, 313-14           Diamond, 298                         Equipment r

% %%% txt/398.txt


%      Fire, 198                                        Gamma-ray sensors, 303


%           rate of, 198                                Ganymede, 99-100
% -326      Firearms, 335-36                                 Gardeners, 329


%      Fireproofing, 313                                 Gas/smoke munitions, 341


%      Firewall, 84-85, 253, 331, 356-61                Gatecrashing, 378-79


%           defeating, 255                              Gatekeeper Corporation, 72


%           Project Ozma and, 380                       Gateway (Pandora), 104


%      Firing modes, 198                                Gear, 126


%      First impression, 146                                 acquiring, 296


%      Fissure Gate, 46, 108                                 concealing, 297


%      Fixation, minor, 211                                  costs, 136-37, 296


%      Fixers, 329                                           design, fashion, 297-98


%      Flashlight, 333                                       fabricating, 296


%      Flash suppressor, 342                                 mass, encumbrance, 297


%      Flats, 139, 174                                       modifiers, 296-97


%      Flayer, 338                                           quality, 296


%      Flex cutter, 334                                      sizes, 297


%      Flexbots, 144                                    Gender, 45, 114


%      Flight drug, 324                                 General anxiety disorder (GAD), 213


%      Flight skill, 178                                General exploration vehicle (GEV), 3


%      Floating worlds, 68                              Genetic defect, 149


%      Flux, 337                                        Gerlach, 90


%      Forgotten Hand, 322                              Ghostrider module, 307


%      Fork handling, 274                               Ghosts, 140


%      Forking, 273-74                                  Gills, 305


%      14K Triad, 84                                    Glitch, 108


%      Fractal digits, 311                              Gnat, 345


%      Fractals, 382                                    Go-nin Group, 72


%      Frag, 340-41                                     Gorgon Defense Systems, 72


%      Frail, 149                                       Grand mal seizure, 365


%      Fray skill, 178-79                               Gravity, 198-99


%      Free fall skill, 179                                  and range, 203


%      Free points, 134-35                                   transition zones, 200


%      Freerunning skill, 179                                well escape, 346


%      Freezer, 340                                     Gravy, 321


%      Frenzy, major, 211                               Greek asteroids


%      Frequency, 321                                        Lot 49, 101-2


%      Fresh Kills, 92                                       resources, economies, 100


%      Fugue, 213                                       Grenades, 199-200


%      Full automatic (FA), 198                              jumping on, 200


%      Full defense, 198                                     seekers and, 340-41


%           attack, 191                                      throwing back, 200


%           psi, 222                                    Grin, 319


%      Fullerenes/Fullerites, 298                       Grip pads, 305


%      Fur coat, 330                                    Grok, 224


%      Furies, 140                                      Groundcraft, 344-45


%      Fusion rocket (F), 347                           Guanxi, 125


%      Futures, 140                                     Guardian angel, 345


%                                                       Guardians, 329, 377


%      G                                                Gunnery skill, 179
% ), 361,   Gallic moonlet, 104                              Gyromount, 342


%      Game effects, disorders, 212, 213, 214, 215


%      Gamemaster, 21                                   H


%      Gamemastering                                    Habitat, 24, 86
% 33            backups, 385                                     asteroid belt, 97


%          fundamentals, 386                                diversity, 45


%          inspiration, 388-89                              Earth, 91


%          player challenges, 386-87                        electronic arrivals, 283


%          record-keeping, 385-86                           Europa, 99


%          reputation gain/loss, 384-85                     Ganymede, Callisto, 99-100


%          resleeving, 271                                  Jupiter, 98


%          responsibilities, 386                            largest, 68


%          Rez point awards, 384, 386                       Luna, 92-94


%          secrets, 387-88                                  Mars, 93-96


%          transhumanism, 389                               Martian Trojans, 96


%      Gamemaster rules                                     Mercury, 88


%          exhumans, 362                                    microgravity, 68-69


%          exsurgents, 369-71                               physical arrivals, 282-83
% 9-82          exsurgent virus, 362-69                          planetary settlement, 66-67


%          extraterrestrial intelligence, 352-53, 361       private, 69


%          factors, 373-77                                  Sol, 86, 88


%          Firewall organization, 356-61                    space, 67-68, 280-81


%          iktomi, 377                                      Venus, 89-90


%          optional i-Rep, 357                              Vulcanoids, 88-89


%          Pandora Gates, 377-79                        Hacker, 243, 246


%          Project Ozma, 379-80                             failing tests, 256


%          Prometheans, 381                                 upgrading status, 256


%          secrets, 352-55                                  zeroing in, 256


%          TITANs, 354-55                               Hacking, 254-56


%      Game playing                                         cyberbrain, 261-62


%          action turns, 119-20, 127                        joint, 258


%          alternate campaigns, 23                          memory, 261


%          campaign setting, 22-23                          sequence, 255


%          character defining, 120                           simulspaces, 263-64


%          critical success/failure, 116                    VPNs, 260


%          defaulting, 116                              Hallucinations, 211, 365


%          default setting, 22-23                       Hamilton cylinders, 281


%          group of players, meeting place, 20-21       Hand laser, 308


%          imagination, 22                              Haptics, 245-46


%          margin of success/failure, 118               Hard suit, 334


%          modifiers, 115-16                             Hardened skeleton, 308


%          note taking, 21-22                           Hardening, 214-15


%          roles, 22                                    Hardware skill, 179


%          rules summary, 127                           Haunting virus, 366-67


%          taking time, 117-18, 127                     Haute nosh, 64


%          target numbers, 115                          Headhunters, 383


%          teamwork, 117                                Healing, 208-9


%          ten-sided dice, 22, 114                      Healing vats, 326


%          test difficulty, 115                          Health drugs, 319-20


%          test making, 115, 127                        Hearing, enhanced, 301


%          test types, 117-19                           Heartbeat sensors, 293


%          trying again, 117                            Heavy combat armor, 310


%          ultimate rule, 114                           Helium-3 mining, 92


%      Gamma forks, 274                                 Hellball, 341
% Helmet, 312                                      test, 271
% Hibernation, 305                            Intentions, 181
% Hibernoids, 140                             Interest skill, 180
% Hidden compartment, 311                     Interface, gear, 298
% Hidden data, 251                            Interfacing skill, 180
% Hidden intruder, 256                        Interrogation, 232
% High-capacity qubit reservois, 315          Intimidation skill, 180
% High explosive, 341                         Intruder
% High-explosive armor-piercing (HEAP), 341        changing status, 256
% High gravity, 199                                locating, 256, 257
% Hinduism, 83                                     status, 256
% History, 32-36, 38                          Intrusion, 254-56
% Hither, 321                                      countermeasures, 257-58
% Hives, 328                                       preconditions, 254
% Hollow-point, 338                                tests, 255
% Holographic projectors, 325                      traces, eliminating, 260
% Homing, 338                                 Intrusion, 254-56
% Hooverman-Geischecker, 88                   Inuit moonlet, 104
% Hopper, 310                                 Investigation skill, 180-81
% Horror, 19, 389                             Invisibility cloak, 316
% Hostile environments, 200-201               Io, 99
% Hovercraft, 310                             Ionic, 310
% Human Cognome Project, 229, 233             Irrationality, major, 211
% Hunger, minor, 211                          Irreproducible goods, 64
% Hunter-killers, 383                         Islam, 83
% Hydrogen-oxygen rocket (HO), 347            Isolates, 40, 131
% Hyoden, 100                                 iZulu, 104
% Hyper linguist, 146, 301
% Hypercorps, 55-57, 70-74, 133               J
% Hyperdense exoskeleton, 344                 Jammers, 338
% Hyperelite, 131                             Jamming, 196, 262
% Hyperion, 104                               Janus Commons, 103
% Hyperthymesia, 224                          Jellies, 370
% Hypochondria, 213                           Jewelry, nostalgia, 41
% Hysteria, major, 211                        Jovian Republic, 57, 58, 75, 134


%                                             Firewall, 361
% I                                                punishment, 60
% Iapetus, 104                                Jovian Trojans, 96
% ID (intelligent design) crew, 83            Judaism, 82-83
% Ideas, 175                                  Juice, 320
% Identity, 124                               Jumping, 191


%  circumventing checks, 280             Jupiter


%  crisis, 149                                Amathea (Solano), 98


%  reputation and, 289                        habitats, moonlets, 98


%  verification, 279-80                        Io, 99
% Iktomi, 377                                      resources, economy, 97
% Illiterate, 149                             Jupiter brain, 92
% Ilmarinen, 108
% Imaging scope, 342                          K
% Immersion, 262-63                           Kaos AI, 331
% Immigration, 282-83                         Keypad lock, 291
% Immortality blues, 149-50                   Kick, 318-19
% Immune system, improved, 146                Kinesics skill, 181
% Immunogenic system, 313                          superior, 225
% Impact, 199                                 Kinetic
% Impaired cognition, 365                          ammunition, 336-38
% Impersonation skill, 180                         damage, 207
% Implant, 126                                     friction, 372


%  memory, 226-27                             weapons, 181, 334-35


%  nanotoxins, 308                       Klar, 318


%  rejection, 150                        Knife, 334


%  skill, 227                            Knockback, 202
% Improvised weapons, 202                     Knockdown, 202, 207-8
% Impulse control disorder, 213               Knowledge skill, 172, 185
% Incompetent, 150
% Indecisiveness, minor, 211                  L
% Indenture, 277                              Language skill, 181
% Indirect fire, 195, 205                      Large lander and orbit transfer vehicle
% Individual factors, 373                     (LLOTV), 348
% Industrial armor, 310                       Laser
% Inequality, 38-39                                -guided, 338
% Infiltration skill, 180                           -microwave link, 314
% Infolife, 131                                    pulsers, 338
% Infomorph, 145                                   sight, 342


%  AGI and, 264-65                       Law enforcement, 58, 59-60


%  muses as, 265                         Layered armor, 194


%  refugees, 65-66                       Learned skills, 123, 127, 136


%  resleeving, 272                            new, 153


%  software minds, 265                        ranges, 174


%  subversion, 364                       Lemon, 150
% Information                                 Less complex life forms, 221-22


%  acquire, 290                          Libertarians, 57-58


%  overload, 237-38                      Liberty, 100
% Infosec skill, 180                          Lidar (visible light) sensors, 303, 311
% Infotech, 46-48                             Life, 50-51
% Infrared sensors, 302-3                          in space, 280-84
% Inhalation, 317                             Light combat armor, 310
% Inhibitor, 325                              Limber, 146
% Initiative (INIT), 121, 138, 188            Linkstate, 323


%  damage, 189                           Liquid thermite, 323


%  moxie, criticals, 189                 Living space, 64


%  order, 189                            Local conditions, 239


%  simplifying, 189                      Lockbots, 292
% Injected application, 317                   Locked intruder, 256
% Injectors, 329                              Lockout, 257
% Inner System                                Locus, 101


%  Firewall, 361                         Logorrhea, minor, 211


%  politics, 55-57, 75                   Loonie, 134
% Innocuous, 146                              Lost generation, 131, 233
% Insanity rating (IR), 122, 138, 209         Lot 49, 102
% Insomnia, 213                               Lotus coating, 313
% Instinct, 224                               Low-capacity qubit reservoir, 315
% Integration                                 Low gravity, 199


%  modifiers, 27                          Luca, 110

% %%% txt/399.txt
% Lucidity (LUC), 122, 138, 209               Microlight, 310, 343                 Na
% Luna, 24, 91                                Micromissile, 339                    Na

%  Erato (Eratosthenes), 92                Microswarms, 328-29                  Na

%  fashion/design, 92                      Microwave agonizer, 339              Na

%  finance, 92                              Mimas (Harmonious Anarchy), 104      Na

%  Helium-3 mining, 92                     Mimic, 227                           Ne

%  Nectar (Nectaris), 92                   Mindlink, 227                        Ne

%  New Mumbai containment zone, 92         Mindstealer virus, 367               Ne

%  Shackle (Shackleton-New Varanasi), 93   Miniature radio farcaster, 315       Ne
% Lunar colonist, 131                         Minor favor, 290                     Ne
% Lunar-Lagrange Alliance, 75                 Mishipizheu, 111                     Ne


%                                        Mist, 248                            Ne
% M                                           Mnemonic augmentation, 307           Ne
% Ma’adim Vallis, 93-94                       Mobile lab, 330                      Ne
% Machine gun, 336                            Mobile offices, 242                   Ne
% Magnetic fields, 200-201                     Mobility systems, 310                Ne
% Magnetic system, 310-11                     Moderate favor, 290
% Mahogany, 109                               Modified behavior, 150
% Major favor, 290                            Modifiers, 115-16, 127, 192
% Maker, 327-28                                   combat, 193
% Malware, 244                                    gear, 296
% Maps, 205                                       integration, alienation, 272
% Margin of Failure (MoF), 118, 119               mesh gear, 247                   Ne
% Margin of Success (MoS), 118, 119               networking, 287                  Ne
% Markov, 109                                     psychosurgery, 231
% Mars, 24                                    Modifying hardware, 179

%  Ashoka, 94                              Modular design, 311

%  buggy, 344-45                           Monitoring, 251-52

%  Elysium, 94-95                          Monofilament sword, 334

%  Noctis-Qianjiao, 95                     Mono No Aware, 319                   Ne

%  Olympus, 95-96                          Monowire garrote, 334                Ne

%  Progress (Deimos), 96                   Mood swings, moderate, 211

%  regions, 93-94                          Moonlets, 98

%  Valles-New Shanghai, 96                 Morningstar Constellation, 75-76     Ne
% Martian, 132                                Morph, 24, 50-51, 86, 121            Ne
% Martian Gate, 46                                acclimatization, 220
% Martian Rangers, 95                             acquisition, 277                 Ne
% Martian Trojans, 96                             availability, 276
% Matchmaking, morph, 277                         bios, 139-42
% Math                                            brokerage, 276-79, 331           Ne

%  boost, 301                                  character, 123                   Ne

%  wiz, 146                                    costs, 277                       Ng
% Matter transformation, 372                      fever, 220                       Ni
% Meathab, 104                                    patron provisioning, 277         Ni
% Media, 51-52                                    pods, 142-43                     Ni
% Medical care, 208                               psi and, 220                     Ni
% Medical sensors, 300                            rental, 278                      No
% Medichines, 308                                 rental insurance, 278            No
% Medicine skill, 182                             starting, 136                    No
% Medusan Shield, 56, 57, 60                      stats, 121                       No
% Megalomania, 213                                switching, 152                   No
% Melder virus, 384                               synthetic, 143-45                No
% Melding, 376                                    trade-in, 277                    No
% Melee                                       Morphing disorder, 150               No

%  attack/combat, 191, 202                 Motivation, 121, 138,139, 152        No

%  weapons, 206, 334, 335                  Movement, 190-91                     No
% Memorizing, 175                                 rates, 191                       No
% Memory                                      Moxie, 122, 189                      Nu

%  editing, 232                                improved, 153                    Ny

%  hacking, 261                                increasing, 135
% Mental alterations, attitudes toward, 48        NPCs and, 386                    O
% Mental armor, 223                               starting, 135                    Ob
% Mental augmentation, 301, 306-7             MRDR, 319                            Ob
% Mental disorder, 150                        Multiple devices access, 249         Ob
% Mental healing, 215                         Multiple personalities, 301          Oc
% Mental health, 209-15                       Multiple personality disorder, 214   Oc
% Mental speed, 308                           Multiple targets, 221                Of
% Mental stress, 230-31                       Multitasking, 224, 307               Ol
% Mentons, 139                                Muscle augmentation, 305             Ol
% Mercenaries, 23, 57                         Muses, 47-48, 51-52, 264             Ol
% Mercurials, 81, 134                             Infolife and, 244, 245           Om
% Mercury habitats, 88                            roleplaying, 265                 Om
% Merging, 275-76                             Mute, moderate, 211                  O’
% Mesh, 24                                    Myst trees, 111                      On

%  abuses, 243-44                                                               Op

%  access, 50                              N

%  accounts, access privileges, 246-47     Nanobandage, 333                     Or

%  capabilities, 236                       Nanodetectors, 326-27                Or

%  gear modifiers, 247                      Nanodrugs, 317, 321-22               Or

%  gear quality, 247                       Nano-ecologists, 81-82               Ou

%  gear rules, 299                         Nanofabrication, 183                 Ou

%  false ID, 252                               blueprints, 284-85               Ov

%  ID, 246                                     programming test, 285            Ox

%  ID authentication, 253                      raw materials, 284               Ox

%  ID tracking, 251-52                         time, 285

%  information overload, 237-38            Nanofabricators, 327-28              P

%  inserts, basic, 46, 300                 Nanophages, 309                      Pa

%  interface, 239-41, 245-46               Nanoplague, 364

%  islands, 242-43                         Nanoscopic vision, 311

%  local, 239                              Nanoswarms, 328-29, 383

%  security, 253-54                        Nanotagging, 292                     Pa

%  technologies, 237                       Nanotat, 310                         Pa

%  traffic filters, mist, 248                    ID alterations, 280              Pa

%  uses, 241-42                                scans, 279-80                    Pa
% Metacelebrities, 52                         Nanotech Hamilton cylinders, 67      Pa
% Metallic foam, 298                          Nanotechnology                       Pa
% Metallic glass, 298                             advanced, 328-29                 Pa
% Metallic hydrogen rocket (MH), 347              basic, 326-28                    Pa
% Metamaterials, 298                          Nanotoxins, 324, 325                 Pa
% Metastasizer, 384                           Nanoviruses, 384                     Pa
% Microbug, 316                               Nanoware, 308-9                      Pa
% Microgravity, 199                           Narcissism, moderate, 211            Pa

%  habitats, 68-69                         Narcoalgorithms, 322-23              Pa

%  shoes, 325                              Narrative modifiers, 116              Pa
% ngue, 135                          Pathogens, 324                                    sleights,
% ealing, 208, 215                   Patron provisioning, morphs, 277                  strain, 2
% mmunity, 146                       Patron, 146                                       trait, 126
%  11, 365                           Pattern recognition, 224-25                       vulnerab
%  n skill, 182                      Pax Familae, 84                              Psi-Chi sleig
% 324                                Peacekeeping, 58-59                          Psi game me
%  , 384                             Penal lease, 278                                  criticals,
%  ectaris), 92                      Penetration, 228                                  full defe
%  8                                 Perception, 190                                   mental a
% refraction, 372                         skill, 182                                   morphs,
% ns, 109, 141                       Peripherals, 247                                  opposed

% hism, 83                         Personal area networks, 241                       roleplayi
%  nids, 141                         Personal augmentation                             skills, sle
%  tivists, 80                            bioware, 301-306                             sleight d
%  , 141                                  cyberware, 306-8                             target aw


%                                    standard, 300                                targeting
%  , 108                             Personal computers, 247                           using, 22
%  nen, 108                          Personal information, 239                    Psi-gamma s
%  gany, 108-9                       Personal vehicles, 345                       Psike-out, 32

% moons, 109                       Personality editing, 232                     Psi-opener, 3
%  , 109                             Persuasion skill, 183                        Psychic stab,
% ns, 109                            Petals, 322                                  Psychosoma
% 23-24                              Petrifier, 384                                Psychosurge
% ng, 286-87                         Pets, 330                                         mechani
% fiers, 287                          Phelan’s Recourse, 105                            modifier
% 357                                Pheromones, enhanced, 304                         neural p
% ation and, 287                     Phlo, 319                                         procedu
% 182                                Phoebe, 107                                       process,
% 287                                Physical arrivals, 282-83                         roleplayi
% m, 305                             Physical augmentations, 302-6, 307-8              skill, 184


%                               Physical entertainment, 54                   Psychotherap
% ge, 150                            Physical health, 206-8                            care, 215
% ng, 273, 274-75                    Physical modifications, 310-11                Psychotortur
%  362                               Physical tracking, 251                       Public accou
% h, 324                             Piercings, 310                               Public inform
% 384                                Pilot skill, 183                             Public key cr


%                               Pirates, 84                                  Public reslee
% munications, 314                   Pistols, 335                                 Punishment,

% eiver, 314                       Planetary Consortium, 55, 76, 85             Puppet sock
% mbai containment zone, 93               hypercorps and, 55-56                   Puppeteering
%  bec, 106                               labor pool, 65-66                       Pyrokinesis,
%  s Compact, 103                         punishment, 60
%  tel, 83                           Planetary settlement, 66-67                  Q
%  e, 332                            Plasmaburst, 341                             Qing Long (A

% 2                                Plasma rifle, 339                             Qualia, 225

% , 83                             Plasma rocket (P), 347                       Quantum
% al stack, 151                      Plasmaburst, 341                                 codebrea
% anjiao, 95                         Plastic ammo, 338                                compute

% communication, 181               Pleasure pods, 142                               cryptogr
%  onlet, 104                        Pneumatic limbs, 311                             entangle

% 41-42, 68                       Pods, 142-43, 271                                farcaster
% ng, 21-22                          Poison gland, 305                                key, 253


%                               Political blocs, 74-76                       Quick action

% 323                              Political experimentation, 58

% (Metis), 97                      Politics                                     R
% s, 142                                  inner system, 55-57                     Radar (radio
%  4                                      outer system, 57-58                         jamming
%  r, 324                            Portable plane, 343                          Radiation, 20
%  07                                Portable quantum entangled communic., 315        sense, 3


%                               Portable sensor, 325                         Radio


%                               Portal denial system, 292                        booster,
%  08 Objects, 202-3                 Post-apocalyptic theme, 19                       jamming

% 151                              Positioning, 205                                 ranges, 2

% compulsive disorder (OCD), 214   Post traumatic stress disorder (PTSD), 214       transceiv
%  0                                 Power, 299                                   Railguns, 33
%  hs, 141                           Pragmatists, 359                             Range, 203
%  armor, 313                        Predators, 362                               Ranged atta
% s, 140                             Predictive boost, 225                        Ranged com
%  95-96                             Prehensile feet, 305                         Ranged wea
% Mons, 94                           Prehensile tail, 305                         Rapid healer
% areness, 227                       Preservationists, 82                         Reach, 204
% 72                                 Priority call, 357                           Reactive am
%  inders, 281-82                    Prisoner mask, 316                           Reactive arm
% n, 151                             Privacy, 49-50, 58-59, 238                   Reactive coa
% tests, 119, 127, 192               Privacy mode, 252                            Real world n
% 22                                 Private habitats, 69                         Real-time se
%  09                                Private information, 290                     Reaper, 144
% umption, 317                       Profession skill, 183                        Re-authentic
% ash, 319                           Programming                                  Reboot, 257
%  80                                     skill, 183                              Reclaimers, 8
%  determination, 192                     test, 285                               Recreational
%  grenade, 341                      Progress (Deimos), 96                        Reflex boost
%  serve, 308                        Project Ozma, 85, 96, 379-80                 Refractive gl
% A, 324                             Prometheans, 85, 378, 381                    Refractory m


%                               Prometheus (Marseilles), 105                 Regeneration


%                               Prosperity Group, 73                         Re-instantiat


%                               Proteans, 329                                Relationship
% 143                                Protocol skill, 183                          Relics, Earth,
% hold, high, 224                    Proxies, 85, 357-59                          Religious gro
%  nce, 146, 150                     Proximity, 338                               Reloading, 1
% kill, 182                          Psi, 126, 147, 218-19, 220-29                Remade, 141
%  ), 104                                 Assault skill, 184                      Rememberin
% Gates, 24, 46, 377-79                   attack, 191                             Remote sniffi
% derate, 211                             chameleon, 147                          Rental insura
% 92                                      damage, 207                             Repair, 179,
% major, 211                              defense, 147                            Repair spray,
%  42-43                                  drawbacks, 220-21                       Reputation,
% eam bolter, 338-39                      drugs, 325                                  burning,
%  p, 290                                 exsurgent, 371                              gain/loss
%  253                                    exsurgent synthmorphs, 371                  identity
% ert, 257                                improving, 153                              improvin
%  i, 221                                 range, 221                                  increasin
%  53                                     shield, 228                                 levels, 28
% r, 73                                   skills, 172-73                              limits, 28

% %%% txt/400.txt


%             grenade and, 340-41                                   vac clothing, 325


%             pistol, 339                                      Smartlink, 342


%             weapons skill, 184                                    -weapon data, 205


%       Seismic sensors, 293                                   Smell, enhanced, 301


%       Self-healing, 292, 313                                 Snake, 310


%       Selfhood, 275                                          Snappers, 370


%       Semi-automatic (SA), 198                               Sniffer, 293, 331


%       Sense                                                  Sniffing, 252


%             block, 228                                       Sniper rifle, 335


%             enhancement, 301-2, 306                          Social drugs, 320


%             skill, 184                                       Social engineers, 359


%       Sensor, 293, 311                                       Social gaff negation, 183


%             ranges, 299                                      Social networks, 124-25, 239, 242,


%       Sensory                                                     285-87, 289


%             boost, 225                                       Social stigma, 151


%             databases, 302                                   Socialites, 82, 134


%             input, 205                                       Society, 41


%             reprogramming, 365                               Socio-political intrigue, 23, 79-82


%       Sentinels, 85, 356-57, 360                             Software, 248, 331


%       Servers, 247, 357                                      Software, crashing, 260
% 1          Services, 330-31                                       Sol (sun), 86


%       Servitor, 346                                               habitats, 86, 88


%       SETI                                                   Solar system, 38


%           (Search for Extra-Terrestrial Intelligence), 379   Solaris, 73


%       Severe failure, 118                                    Somatek, 73-74


%       Sexuality, 45, 305                                     Soundwaves, 303
% on), 125   Shackle (Shackleton-New Varanasi), 93                  Sousveillance, 238


%       Shape adjusting, 311                                   Space


%       Shard pistol, 340                                           colonist, original, 132


%       Shell                                                       habitats, 67, 280-81


%             jamming, 196                                          roach, 330


%             movement, 195                                         travel, 283-84, 331


%             remote control, 196                              Spacecraft, 346-49


%             skills, 195                                           propulsion, 347


%             stats, 195                                       Spaceship combat, 284


%       Shelter dome, 333                                      Spam, 228


%       Shifters, 370                                          Spasm, 332


%       Shock                                                  Special skills, 185


%             attacks, 204                                     Specializations, 123, 127, 136, 153


%             baton, 334                                       Specialized hive, 328


%             gloves, 334                                      Specimen container, 330


%             safety, 342                                      Speck, 346


%             weapon immunity, 143                             Specs, 326


%       Shooting through, 203                                  Speed (SPD), 121, 138, 189


%       Shredder, 340                                          Spimes, 238


%       Shui Fong, 84                                          Spindle, 333


%       Shutdown, 257, 262                                     Spindle climber, 333


%       Shutter, 332                                           Splash, 337, 341


%       Signal, 199-200                                        Splicers, 139, 174


%       Silence, 289                                           Spoof, 331


%       Silencer/sound suppressor, 342                         Sports, 54


%       Simulmorphs, 262                                       Spotted intruder, 256


%       Simulspace, 262-64                                     Spray armor, 313


%             access, 241                                      Spray weapons, 184, 340


%             environments, 240-41                             Sprayer, 340


%             hacking, 263-64                                  Sprinting, 191


%             rules, 263                                       Starware, 74
% ), 104           subscription, 331                                Static, 228


%       Single-shot (SS), 198                                  Stealthed signals, 252

%  104    Singularity seekers, 43                                Stellar Intelligence, 74


%       Situational awareness, 148                             Sticky grenades, 341


%       Skathi, 107                                            Stress
% 5          Skill                                                       points, 209


%             categories, 172                                       value, 209


%             corem 172                                        Stressful situations/experiences, 214


%             imprints, 232                                    Striking looks, 148


%             improving, 152-53                                Structural enhancement, 311


%             learned, 172-73                                  Structuralists, 359


%             list, 176-84                                     Structures, 202-3


%             necessary, 176                                   Stunner, 339


%             ranges, 174                                      Subcultures, 39-41


%             suppression, 232                                 Subdermal implants, 309


%             untrained use, 116                               Subdued opponent, 204


%             using, 173-74                                    Subliminal, 228


%       Skilled sentient labor, 64                             Submachine guns, 335


%       Skillsofts, 332                                        Submarine, 310


%       Skillware, 309                                         Substance abuse, 317-18


%       Skinaethesia, 73                                       Subversion, 259-61


%       Skindyes, 310                                               difficulties, 259


%       Skinflex, 309                                                examples, 259


%       Skinlink, 309                                          Success tests, 117-19, 127


%       Skinning, 240                                          Sun Yee On, 84


%       Skin pocket, 305                                       Superthermite charges, 330


%       Skinthetic, 73                                         Suppressive fire, 204


%       Sky Ark, 111                                           Surgery, 208


%       Slaving devices, 248                                   Surprise, 204-5


%       Sleep, 365                                             Surveillance, 238, 293


%       Slip, 323                                                   bugs and, 315


%       Slippery walls, 292                                    Survival gear. 332-33


%       Slitheroids, 144                                       Suryas, 86


%       Slow learner, 151                                      Swarm, 328-29


%       Small jet, 343                                              composition, 311


%       Small lander and orbit transfer vehicle                Swarmanoid, 144


%             (SLOTV), 349                                     Sweeping fire, 194


%       Smart                                                  Swimming skill, 184


%             ammo, 338                                        Sybils, 81


%             clothing, 325                                    Sylphs, 139-40


%             dogs, 330                                        Synergy, 111


%             dust, 316                                        Synthetic mask, 311


%             magazine, 342                                    Synthmorph, 66, 143-45, 195-96, 3


%             materials, 298                                        bioware and, 306


%             monkey, 330                                           object repair, 209


%             rats, 330                                             physical repair, 209


%             skin, 312                                             resleeving, 271
% T                                                   weapons, 334
% Tactical networks, 205, 331                    Unattractive, 151
% Taggants, 329                                  Unbreathable atmosphere, 201
% Target                                         Uncanny valley, 151-52


% multiple, 202                             Unconscious lead, 225


% numbers, 115                              Unconsciousness, 208


% specific, 197                              Underbarrel seeker, 339-40
% Task Action Programming Test, 246              Underwater, 201
% Task actions, 120, 127, 190                    Unfit, 152
% Tasping, 232                                   Uniform blast, 193
% Teamwork, 117, 127                             Uplifts, 81, 132
% Techno-Creationists, 83                        Uploading, 268-69
% Technology, 45                                 Uranus, 107
% Temperature tolerance, 305                          Chat Noir, Fissure Gate, 108
% Terahertz sensors, 302                              Titania, Oberon, 108
% Terminology, 25-27                                  Xiphos, 108
% Terragenesis, 74                               User accounts, 247
% Test making, 115                               Utilitool, 326
% Tethys (Godwinhead), 105                       Utopians, 57-58
% Tharsis League, 76                             Uzumaki, 384
% Thermal dampening, 313
% Thermobaric, 341                               V
% Thought browse, 228-29                         V/2011-Caldwell, 88-89
% Throwing weapons skill, 184                    Vacuum, 201
% Thrown damage bonus, 202                           and range, 203
% Thruster pack, 345                                 sealing, 306
% Thrust victor, 310                                 suits, 333-34
% TILION, 92                                     Valles Marineris, 94
% Time sense, 225                                Valles-New Shanghai, 96
% Timeline, 37                                   Value, 63-64
% Timid, 151                                     Variable opposed test, 119
% Tin cans, 282                                  Vectors, 359
% Titan, 24, 106                                 Vehicle passenger attacks, 196


% Aarhus, 106                               Vehicles, 195-96, 342-49


% New Quebec, 106                           Venus, 24, 89-90


% Nyhavn, 106-7                             Venusian, 134


% Phoebe, Skathi, Abramsen, 107             Vertigo, 365
% Titanian Commonwealth, 79, 108, 134            Vibroblade, 334


% Firewall, 361                             Vidgames, 52-53
% TITANs (Total Information Tactical Awareness   Vids, 52-53
% Networks), 32, 34-6, 40, 42, 48, 85, 236,      Viewers, 326
% 244-45, 324, 354-55, 362-63, 387-89            Vision, enhanced, 301


% -controlled army, 66                      Vo Nguyen, 92


% infected ruins, 270                       Vortex ring gun, 341


% legacy, 237, 381-84                       Voting, 56


% mutated person, 47                        VPN (virtual private network), 241-42


% quarantine zone, 94                           hacking, 260


% relics, 56                                VR (virtual reality), 236, 240-41
% Token lock, 291                                    vertigo, 152
% Tools, 325                                         worlds, 54
% Top secret intel, 290                          Vulcanoids habitats, 88-89
% Torch, 340                                     Vulvanoid Gate, 46
% Toruses, 282
% Touch-only attack, 206                         W
% Tough, 148                                     Walker, 310
% Toughness, 143                                 Warbots, 383
% Toxic atmosphere, 201                          Wasp knife, 334
% Toxin, 323-24, 325                             Watts-Macleod virus, 218, 219, 220, 368


% filters, 305                               Weak immune system, 152
% Trace, 258                                     Weapon, 334-42
% Tracked, 310                                       accessories, 341-42
% Tracker Dye, 323                                   improvised, 202
% Tracking, 251-52, 331                              mount, 311
% Traction pads, 317                                 ranges, 203
% Trade, 23                                          scanners, 293
% Traffic filters, 248                                 wielding two or more, 206
% Training animals, 177                          Wheeled, 310
% Traits, 123                                    Whippers, 370


% gaining/losing, 153                       White noise machine, 317


% purchasing, 136                           Willpower stress tests, 214
% Transhumanism, 18, 38, 389                     Winged, 310


% informorphs, 245                          Wireless code lock, 291


% labor, 64                                 Wireless inhibitors, 292


% life, death and morphs, 50-51             Wireless scanning, 251, 293
% Transparent alumina, 298                       Wireless termination, 258
% Transport, standard, 349                       Worker pods, 142
% Transporter, 344                               Wormhole, 378
% Trauma, 209                                    Wormwood, 111


% effects, 209-10                           Wound, 207-8


% threshold (TT), 122, 138                      Determination, 192
% Travel, 48-49, 63                                  effects, 207-8


% basics, 283-84                                threshold (WT), 121


% distance, 283                             Wounds, 207-8


% local, 283                                Wrappers, 370
% T-ray emitter, 306                             Wrist-mounted tools, 309
% Tremors, moderate, 211
% Triads, 84                                     X
% Trigger, 324                                   Xenodeism, 83
% Trike, 344                                     Xenomorph virus, 368
% Triton, 109                                    Xiphos, 108
% Trivial favor, 290                             XP (experience playback), 53, 236, 241-
% Trojans                                        42, 331


% Locus, 100-101                            X-ray emitter, 317


% resources, economies, 100                 X-ray sensors, 303
% Twelve Commons, 103
% Twitch nerve gas, 324                          Y-Z
% Two-handed weapons, 206                        YGBM (you gotta believe me) attacks, 365-66


%                                           Zap, 337
% U                                              Zbrny Group, 74
% Ukko Julinä, 88                                Zero ammo, 338
% Ultimates, 82, 134                             Zero-g nausea, 152
% Ultraviolet sensors, 303                       Zoosemiotics, 148
% Unarmed

%  combat skill, 185

% %%% txt/401.txt


%                CHARACTER



%          BACKGROUND



%                     Faction



%                      Morph



%        Gender Identity



%                 Actual Age


%  Current Moxie Points



%                   Rez Points



%                Motivations




%  Armor


%       ENERGY                                KINETIC



%          Melee Weapon


%                                WEAPON                 SKILL



%       Ranged Weapon


%                                WEAPON                 SKILL






%           LINKED                         MORPH                SPECIAL
% ACTIVE SKILLS APTITUDE                 BASE   BONUS TOTAL           OTHER
% Animal Handling            SAV
%  Beam Weapons             COO


%         Blades       SOM


%      Climbing        SOM


%          Clubs       SOM


%       Control        WIL*


%    Deception          SAV


% Demolitions          COG*


%      Disguise         INT

% Exotic Melee:          SOM

% Exotic Melee:          SOM

% Exotic Ranged:          COO

% Exotic Ranged:          COO


%          Flight      SOM


%            Fray       REF


%      Free Fall        REF


% Freerunning          SOM


%      Gunnery          INT


%    Hardware:         COG


%    Hardware:         COG

% Impersonation            SAV


%    Infiltration       COO


%        Infosec       COG*


%   Interfacing        COG


% Intimidation          SAV

%  Investigation          INT


%       Kinesics        SAV
% Kinetic Weapons           COO


%     Medicine         COG


%     Medicine         COG


%  Navigation           INT


% Networking:           SAV


% Networking:           SAV


% Networking:           SAV


% Networking:           SAV


%       Palming        COO


%   Perception          INT


%   Persuasion          SAV


%         PLAYER



%    ECLIPSE


%     PHASE       CHARACTER S


%                                   COG       COO       INT           REF       S


%                     Base


%              Morph Bonus


%                     Total



%                         WIL x 2             FROM MORPH               (INT + REF)



%     MOX        TT       LUC         IR       WT       DUR           DR       IN




%               LUC ÷ 5             LUC X 2   DUR ÷ 5                 Biomorphs: D


%                                                                     Synthmorphs




%                                                                       DAM


%                                             DAMAGE WOUNDS                     S




%                                                                PRIMAR
% P           DV                NOTES



% P           DV                MODES              AMMO            RANGE




%  ON /                           LINKED                        MORPH
% NUS              ACTIVE SKILLS APTITUDE             BASE      BONUS TOTAL


%                       Pilot        REF


%                       Pilot        REF


%              Programming          COG*


%                   Protocol         SAV


%                 Psi Assault       WIL*


%             Psychosurgery          INT


%                  Research         COG


%                Scrounging          INT


%           Seeker Weapons          COO


%                      Sense        INT*


%            Spray Weapons          COO


%                 Swimming          SOM


%        Throwing Weapons           COO


%         Unarmed Combat            SOM



%                           * = no defaulting



%                                     LINKED              MORPH


%      KNOWLEDGE SKILLS              APTITUDE        BASE BONUS TOTAL


%      Academics:                       COG


%      Academics:                       COG


%      Academics:                       COG


%      Academics:                       COG


%      Art:                             INT


%      Art:                             INT


%      Art:                             INT


%      Art:                             INT


%      Interest:                        COG


%      Interest:                        COG


%      Interest:                        COG


%      Interest:                        COG


%      Language:                        INT


%      Language:                        INT


%      Language:                        INT


%      Language:                        INT


%      Profession:                      COG


%      Profession:                      COG


%      Profession:                      COG


%      Profession:                      COG

% %%% txt/402.txt


%                                  ECLIPSE


%                                   PHASE    C


%              G-Rep



%               I-Rep


%                                            S


%              R-Rep




% ve Traits


%                                            P




% ER
% pe

% er

% ge                                              M

% on


%                                            P
%  SPEED MOD    MOVEMENT RATE / MOBILITY SYSTEM



% DURABILITY         WOUND THRESHOLD



% cements / Customizations
% PLAYER


% HARACTER SHEET


%                                           MUSE STATS


%        COG   COO   INT    REF     SAV       SOM         WIL
% ptitudes
%  & Notes


%                                            WIL x 2



%                                  TT        LUC         IR




%                                 LUC ÷ 5              LUC X 2



% leights




% kup Notes




% CLIPSE
% PHASE      MORPH SHEET


%                           APTITUDE BONUSES


%        COG   COO   INT    REF     SAV       SOM         WIL
% ph Bonus

% ve & Negative Traits / Advantages & Disadvantages

\end{document}
%%% Local Variables: 
%%% mode: latex
%%% TeX-master: t
%%% End: 
