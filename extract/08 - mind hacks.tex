%%% 218
%%% 219
%%% 220
%%% 221
%%% 222
Though neuroscience has ascended to impressive 
pinnacles, allowing minds to be thoroughly scanned, 
mapped, and emulated as software, the transhuman 
brain remains a place that is complicated, not fully 
understood, and thoroughly messy. Despite a prevalence
of neural modifications, meddling with the seat
of consciousness remains a tricky and hazardous 
procedure. Nevertheless, psychosurgery—editing the 
mind as software—remains common and widespread, 
sometimes with unexpected results.

Likewise, even as the knowledge of neuroscientists 
grows on an exponential basis, some are discovering 
that minds are far more mysterious than they had ever 
imagined. During the Fall, scattered reports of anomalous
activity'' by individuals infected by one of the
numerous circulating nanoplagues were discounted 
as fear and paranoia, but subsequent investigations 
by black budget labs has proven otherwise. Now, top-level
confidential networks whisper that this infection
inflicts intricate changes in the victim's neural network 
that imbue them with strange and inexplicable abilities
The exact mechanism and nature of these abilities
remains unexplained and outside the grasp of modern 
transhuman science. Given the evidence of a new 
brainwave type and the paranormal nature of this 
phenomenon, it is loosely referred to as ``psi.''

\section{Psi}

In \textit{Eclipse Phase}, psi is considered a special cognitive 
condition resulting from infection by the mutant—and 
hopefully otherwise benign—Watts-Macleod strain 
of the Exsurgent virus (p. 367). This plague modifies 
the victim's mind, conferring special abilities. These 
abilities are inherent to the brain's architecture and 
are copied when the mind is uploaded, allowing the 
character to retain their psi abilities when changing 
from morph to morph.

\subsubsection{Prerequisites}

To wield psi, a character must acquire the Psi trait (p. 
147) during character creation. It is theoretically also 
possible to acquire the use of psi in game via infection
by the Watts-MacLeod strain; see \textit{The Exsurgent }
\textit{Virus,} p. 362.

Psi ability is considered an innate ability of the 
ego and not a biological or genetic predisposition of 
the morph. While psi researchers do not understand 
how it is possible to transfer this ability via uploads, 
backups, and farcasting, it has been speculated that 
all components of an async's ego are entangled on a 
quantum level, or that they possess the ability to entangle
themselves or form a unique conformation or
alignment as a whole even after they have been copied, 
up-, or downloaded. This speculated entanglement 
process is also thought to be the origin of the impairment
that asyncs experience when adapting to a new
morph (see below).

\subsubsection{Morphs And Psi}

Asyncs require a biological brain to draw on their 
abilities (the brains of uplifted animals count). An 
async whose ego is downloaded into an infomorph or 
fully computerized brain (synthmorphs) has no access 
to their abilities as long they remain in that morph.

Asyncs inhabiting a pod morph may use psi, but 
their abilities are restricted as pod brains are only 
partly biological. Pod-morphed asyncs suffer a –30 
modifier on all tests involving the use of psi sleights 
and the impact from using sleights would be doubled.

\subsubsection{Morph Acclimatization}

Async minds undergo extra difficulty adjusting to new 
morphs. For 1 day after the character has resleeved, 
they will suffer the effects of a single derangement 
(p. 210). The gamemaster and player should choose a 
derangement appropriate to the character and story. 
Minor derangements are recommended, but at the 
gamemaster's discretion moderate or major derangements
may be applied. No trauma is inflicted with
this derangement.

\subsubsection{Morph Fever}

Asyncs find it irritating and traumatizing to endure life 
as an infomorph, pod, or synthmorph for long periods 
of time. This phenomenon, known as \textit{morph fever}, 
might cause temporary derangements and trauma to 
the asyncs' ego, possibly even to the grade of permanent
disorders. If stored or held captive as an active
infomorph (i.e. not in virtual stasis), the async might 
go insane if not psychologically aided by some sort of 
anodyne program or supporting person during storage.

In game terms, asyncs take 1d10 ÷ 2 (round up) 
points of mental stress damage per month they stay 
in a pod, synthmorph or infomorph form without 
psychological assistance by a psychiatrist, software, 
or muse.

\subsubsection{Psi Drawbacks}

There are several drawbacks to psi ability:
\textbf{•  }The variant Exsurgent strain that endows psi ability 

rewires the character's brain. An unfortunate side 

effect to this change is that asyncs acquire a vulner-

ability to mental stress. Reduce the async's Trauma 

Threshold by 1.
\textbf{•  }The mental instability that accompanies psi infec-

tion also tends to unhinge the character's mind. 

Asyncs acquires one Mental Disorder negative trait 

(p. 150) for each level they have of the Psi trait 
%%% 223

without receiving any bonus CP. The gamemaster 

and player should agree on a disorder appropriate 

to the character. This disorder may be treated over 

time, according to normal rules (see\textit{ Mental Healing }

\textit{and Psychotherapy, }p. 215).
\textbf{•  }Characters with the Psi trait are also vulnerable to 

infection by other strains of the Exsurgent virus. 

The character suffers a –20 modifier when resisting 

Exsurgent infection (p. 362).
\textbf{•  }Critical failures when using psi tend to stress the 

async's mind. Each time a critical failure is rolled 

when making a sleight-related test, the async suffers 

a temporary brain seizure. They suffer a –30 modi-

fier and are incapable of acting until the end of the 

next Action Turn. They must also succeed in a WIL 

+ COG Test or fall down.

\subsubsection{Psi Skills And Sleights}

Transhuman psi users can manipulate their egos and 
otherwise create effects that can often be neither 
matched nor mimicked by technological means. To use 
these abilities, they train their mental processes and 
practice cognitive algorithms called sleights, which 
they can subconsciously recall and use as necessary. 
Sleights fall into two categories: psi-chi (cognitive 
enhancements, p. 223) and psi-gamma (brainwave 
reading and manipulation, p. 225). Psi-chi sleights are 
available to anyone with the Psi trait (p. 147), but psi-gamma
sleights are only available to characters with
the Psi trait at Level 2. In order to use these sleights, 
the async must be skilled in the Control (p. 178), Psi 
Assault (p. 183), and/or Sense skills (p. 184), as appropriate
to each sleight.

\subsubsection{Roleplaying Asyncs}

Any player who chooses to play an async should keep 
the origin of their abilities in mind: Watts-MacLeod 
strain infection. The character may not be aware of 
this source, but they undoubtedly know that they underwent
some sort of transformation and have talents
that no one else does. If unaware of the infection, they 
have likely learned to keep their abilities secret lest 
they be ridiculed, attacked, or whisked away to some 
secret testing program. Learning the truth about their 
nature could even be the starting point of a campaign 
and/or their introduction to Firewall. If they know the 
truth, however, the character must live with the fact 
that they are the victim of a nanoplague likely spread 
by the TITANs that may or may not lead to complications
side effects, or other unexpected revelations in
their future.

Gamemasters and players should make an effort 
to explore the nature of this infection and how the 
character perceives it. As noted previously, asyncs are 
often profoundly-changed people. The invasive and 
alien aspect of their abilities should not be lost on 
them. For example, an async might conceive of their 
psi talents as a sort of parasitic entity, living off their 
sleights, or they might feel that using these powers 
puts them in touch with some sort of fundamental 
substrate of the universe that is weird and terrifying. 
Alternately, they could feel as if their personality was 
melded with something different, something that 
doesn't belong. Each async is likely to view their situation
differently, and none of them pleasantly.

\subsection{Using Psi}

Using psi—i.e., drawing on a certain sleight to procure
some kind of effect—does not always require a
test. Each sleight description details how the power 
is used.

\subsubsection{Active Psi}

Active psi sleights must be ``activated'' to be used. 
These sleights usually require a skill test. Sleights that 
target other sentient beings or life forms are always 
Opposed Tests, while others are handled as Success 
Tests. The level of concentration required to use these 
sleights varies, and so may call for a Quick, Complex, 
or Task Action. Active sleights also cause strain (p. 
223) to the async. Most psi-gamma sleights fall into 
this category.

\subsubsection{Passive Psi}

Passive psi sleights encompasses abilities that are considered
automatically active and subconscious. They
rarely require an action to be activated and require 
no effort or strain by the psi user. Passive sleights typically
add bonuses to various activities or allow access
to certain abilities rather than calling for some kind of 
skill test. Most psi-chi sleights fall into this category.

\subsubsection{Psi Range}

Sleights have a Range of either Self, Touch, or Close.
\textbf{Self:} These sleights only affect the async.
\textbf{Touch:} Sleights with a Touch range may be used 
against other biological life, but the async must have 
physical contact with the target. If the target avoids 
being touched, this requires a successful melee attack, 
applying the touch-only +20 modifier.  This  attack 
does not cause damage, and is considered part of the 
same action as the psi use.
\textbf{Close:} Close sleights involve interaction with other 
biological life from a short distance. The optimal distance
is within 5 meters. For each meter beyond that,
apply a –10 modifier to the test.
\textbf{Psi vs. Psi:} Due to the nature of psi, sleights are 
more effective against other psi users. Sleights with 
a range of Touch may be used from a Close range 
against another async. Likewise, a sleight with a Close 
range may be used at twice the normal distance (10 
meters) when wielded on another async.

\subsubsection{Targeting}

Synthmorphs, bots, and vehicles may not be targeted 
by psi sleights, as they lack biological brains. Pods—
with brains that are half biological and half computer—are
less susceptible and receive a +30 modifier

ith

ge

se

0
%%% 224
when defending against psi use. Note that infomorphs 
may never be targeted by psi sleights as psi is not effective
within the mesh or simulspace.

\textbf{Multiple Targets:} An async may target more than 
one character with a sleight with the same action, as 
long as each of them can be targeted via the rules 
above. The psi character only rolls once, with each of 
the defending characters making their Opposed Tests 
against that roll. The psi character suffers strain (p. 
223) for each target, however, meaning that using psi 
on multiple targets can be extremely dangerous.

\textbf{Animals and Less Complex Life Forms:} Psi works 
against any living creature with a brain and/or 
nervous system. Against partially-sentient and partially-uplifted
animals, it suffers a –20 modifier and
increases strain by +1. Against non-sentient animals, 
it suffers a –30 modifier and increases strain by +3. It 
has no effect on or against less complex life forms like 
plants, algae, bacteria, etc.

\textbf{Factors and Aliens:} At the gamemaster's discretion, 
psi sleights may not work on alien creatures at all, depending
on their physiology and neurology. If it does
work, it is likely to suffer at least a –20 modifier and 
+1 strain.

\subsubsection{Opposed Tests}

Psi that is used against another character is resisted 
with an Opposed Test. Defending characters resist with 
WIL x 2. Willing characters may choose not to resist. 
Unconscious or sleeping characters cannot resist.

If the psi-wielding character succeeds and the defender
fails, the sleight affects the target. If the psi user
fails, the defender is unscathed. If both parties succeed
in their tests, compare their dice rolls. If the psi
user's roll is higher, the sleight bypasses the defender's 
mental block and affects the target; otherwise, the 
sleight fails to affect the defender's ego.

\subsubsection{Target Awareness}

The target of a psi sleight is aware they are being 
targeted any time they succeed on their half of the 
Opposed Test (regardless on whether the async 
rolls higher or not). Note that awareness does not 
necessarily mean that the target understands that psi 
abilities are being used on them, especially as most 
people in \textit{Eclipse Phase} are unaware of psi's existence. 
Instead, the target is simply likely to understand that 
some outside influence is at work, or that something 
strange is happening. They may suspect that they 
have been drugged or are under the influence of some 
strange technology.

Targets who fail their roll remain unaware.

\subsubsection{Psi Full Defense}

Like full defense in physical combat (p. 198), a defender
may spend a Complex Action to rally and concentrate
their mental defenses, gaining a +30 modifier
to their defense test against psi use until their next 
Action Phase.
%%% 225

\subsubsection{Criticals}

If the defender rolls a critical success, the character 
attempting to wield psi is temporarily locked out of 
the target's mind. The psi user may not target that 
character with sleights until an appropriate ``reset'' 
period has passed, determined by the gamemaster.

If the async rolls a critical failure, they suffer temporary
incapacitation as their mind dysfunctions in some
harsh and distressing ways (see \textit{Psi Drawbacks,} p. 221).

If a psi user rolls a critical success against a defender
or the defender rolls a critical failure, double
the potency of the sleight's effect. In the case of psi attacks
the DV can be doubled or mental armor can be
bypassed. Alternately, when using Psi Assault (p. 183), 
the targeted character may be in danger of infection 
by the Watts-Macleod strain (p. 362).

\subsubsection{Mental Armor}

The Psi Shield sleight (p. 228) provides mental armor, 
a form of neural hardening against psi-based attacks. 
Like physical armor, this mental armor reduces the 
amount of damage inflicted by a psi assault.

\subsubsection{Duration}

Psi sleights have one of four durations: \textit{constant, in-}
\textit{stant, temporary,} or \textit{sustained.}

\textbf{Constant:} Constant sleights are always ``on.''

\textbf{Instant:} Instant sleights take effect only in the 
Action Phase in which they are used.

\textbf{Temporary:} Temporary sleights last for a limited 
duration with no extra effort from the async. The 
temporary duration is determined by the async's WIL 
÷ 5 (round up) and is measured in either Action Turns 
or minutes, as noted. Strain for the sleight is applied 
immediately when used, not at the end of the duration.

\textbf{Sustained:} Sustained sleights require active effort 
to maintain for as long as the async wants to keep 
it active. Sustaining a sleight requires concentration, 
and so the async suffers a –10 modifier to all other 
skill tests while the sleight is sustained. The async 
must also stay within the range appropriate to the 
sleight, otherwise the sleight immediately ends. More 
than one sleight may be sustained at a time, with a 
cumulative modifier. Strain for the sleight is applied 
immediately when used, not at the end of the duration
At the gamemaster's discretion, sleights that are
sustained for long periods may incur additional strain.

\subsubsection{Strain}

The use of psi is physically (and sometimes psychologically
draining to a psi user. This phenomenon is
known as strain, and manifests as fatigue, exhaustion, 
pain, neural overload, cardiovascular stress, and 
adynamia (loss of vigor). Though strain has only 
rarely been known to actually kill an async, the use 
of too much active psi can be life-threatening in some 
circumstances.

In game terms, every active sleight has a Strain 
Value of 1d10 ÷ 2 (round up) DV. Every active sleight 
lists a Strain Value Modifier that modifies this amount. 
For example, a sleights with a Strain Value Modifier 
of –1 inflicts (1d10 ÷ 2) –1 DV.
If the damage points suffered from strain exceed 
the character's Wound Threshold, they may inflict a 
wound just like other damage (see \textit{Wounds,} p. 207).

\subsection{Psi-Chi Sleights}

Psi-chi sleights are async abilities that speed up cognitive
informatics (internal information processing) and
enhance the user's perception and cognition.

\subsubsection{Ambience Sense}


  

Passive

   Automatic

 Self

   Constant
This sleight provides the async with an instinctive 
sense about an area and any potential threats nearby. 
The async receives a +10 modifier to all Investigation, 
Perception, Scrounging, and Surprise Tests.

\subsubsection{Cognitive Boost}


  

Active

   Quick

 Self

  

   –1
The async can temporarily elevate their cognitive 
performance. In game terms, Cognition is raised by 5 
for the chosen duration. This boost to Cognition also 
raises the rating of skills linked to that aptitude.
%%% 226

\subsubsection{Downtime}

  

Active

   Task (min. 4 hours)
 Self

   Sustained

   0
This sleight provides the async with the ability 
to send the mind into a fugue-state regenerative 
downtime, during which the character's psyche is 
repaired. The async must enter the downtime for 
at least 4 hours; every 4 hours of downtime heals 
1 point of stress damage. Traumas, derangements, 
and disorders are unaffected by this sleight. For 
all sensory purposes, the async is catatonic during 
downtime, completely oblivious to the outside 
world. Only severe disturbances or physical shock 
(such as being wounded or hit by a shock weapon) 
will bring the async out of it.

\subsubsection{Emotion Control}

  

Passive

   Automatic
 Self

   Constant
Emotion Control gives the async tight control 
over their emotional states. Unwanted emotions 
can be blocked out and others embraced. This has 
the benefit of protecting the async from emotional 
manipulation, such as the Drive Emotion sleight or 
Intimidation skill tests. The async receives a +30 
modifier when defending against such tests.

\subsubsection{Enhanced Creativity}

  

Passive

   Automatic
 Self

   Constant
An async with Enhanced Creativity is more imaginative
and more inclined to think outside the box.
Apply a +20 modifier to any tests where creativity
plays a major role. This level of ingenuity can
sometimes seem strange and different, manifesting 
in odd or creepy ways, especially with artwork.

\subsubsection{Filter}

  

Passive

   Automatic
 Self

   Constant
Filter allows the async to filter out out distractions 
and eliminate negative situational modifiers from 
distraction, up to the gamemaster's discretion.

\subsubsection{Grok}

  

Active

   Complex
 Self

   Instant

   –1
By using the Grok sleight, the async is able to 
intuitively understand how any unfamiliar object, 
vehicle, or device is used simply by looking at and 
handling it. If the character succeeds in a COG x 2 
Test, they achieve a basic ability to use the object, 
vehicle, or device, no matter how alien or bizarre. 
This sleight does not provide any understanding of 
the principles or technologies involved—the psi user 
simply grasps how to make it work. If a test is called 
for, the psi user receives a +20 modifier to use the 
device (this bonus only applies to unfamiliar devices, 
and/or tests the character is defaulting on—it does not 
apply to devices the character is familiar with).

\subsubsection{High Pain Threshold}


  

Passive

   Automatic
 Self

   Constant
This sleight allows the async to block out, ignore, or 
otherwise isolate pain. The async reduces negative 
modifiers from wounds by 10.

\subsubsection{Hyperthymesia}


  

Passive

   Automatic
 Self

   Constant
Hyperthymesia grants the async a superior autobiographical
memory, allowing them to remember the
most trivial of events. A hyperthymestic async can be 
asked a random date and recall the day of the week 
it was, the events that occurred that day, what the 
weather was like, and many seemingly trivial details 
that most people would not be able to recall.

\subsubsection{Instinct}


  

Passive

   Automatic
 Self

   Constant
Instinct bolsters the async's subconscious ability to 
gauge a situation and make a snap judgment that 
is just as accurate as a careful, considered decision. 
For Task Actions that involve analysis or planning 
alone (typically Mental skill actions), the async may 
reduce the timeframe by 90% without suffering a 
modifier. For Task Actions that involve partial analysis/planning
they may reduce the timeframe by 30%
without penalty.

\subsubsection{Multitasking}


  

Passive

   Automatic
 Self

   Constant
The async can handle vast amounts of information 
without overload and can perform more than one 
mental task at once. The character receives an extra 
Complex Action each Action Phase that may only be 
used for mental or mesh actions.

\subsubsection{Pattern Recognition}


  

Passive

   Automatic
 Self

   Constant
The character is adept at spotting patterns and correlating
the non-random elements of a jumble—related
items jump out at them. This is useful for translating 
languages, breaking codes, or find clues hidden among 
massive amounts of data. The character must have 
%%% 227
a sufficiently large sample enough time to study, as 
determined by the gamemaster. This might range from 
a few hours of listening to a spoken transhuman language
to a few days of investigating inscriptions left
by long-dead aliens to a week or more of researching 
a lengthy cipher. Languages may be comprehended by 
reading or listening to them being spoken. Apply a 
+20 modifier to any appropriate Language, Investigation
Research, or cod-breaking Tests (note that this
does not apply to Infosec Tests made by software to 
decrypt a code). The async may also use this ability to 
more easily learn new languages, reducing the training 
time by half.

\subsubsection{Predictive Boost}


  

Passive

   Automatic

 Self

   Constant
The Bayesian probability machine features of the 
async's brain are boosted by this sleight, enhancing 
their ability to estimate and predict outcomes of 
events around them as they unfold in real-time and 
update those predictions as information changes. 
In effect, the character has a more intuitive sense 
for which outcomes are most likely. This grants the 
character a +10 bonus on any skill tests that involve 
predicting the outcome of events. It also bolsters 
the async's decision-making in combat situations by 
making the best course of action more clear, and so 
provides a +10 bonus to both Initiative and Fray Tests.

\subsubsection{Qualia}


  

Active

   Quick

 Self

  

   –1
The async can temporarily increase their intuitive 
grasp of things. In game terms, Intuition is raised by 
5 for the chosen duration. This boost to Intuition also 
raises the rating of skills linked to that aptitude.

\subsubsection{Savant Calculation}


  

Passive

   Automatic

 Self

   Constant
The character possesses an incredible facility with 
intuitive mathematics. They can do everything from 
calculate the odds exactly when gambling to predicting
precisely where a leaf falling from a tree will land
by observing the landscape and local wind currents. 
The character specializes in calculation involving the 
activity of complex chaotic systems and so can calculate
answers that even the fastest computers could not,
including things like patterns of rubble distribution 
from an explosion. However, this mathematic facility 
is largely intuitive, so the character does not know the 
equations they are solving, they merely know the solution
to the problem.
This sleight also provides a +30 modifier to any 
skill tests involving math (which the character is calculating
not a computer).

\subsubsection{Sensory Boost}


  

Active

   Quick
 Self

  

   –2
An async uses this sleight to increase their natural or 
augmented sensory perception (sight, audio, smell, 
augmented) by enhanced cerebral processing, granting
a +20 bonus modifier on sensory-based Perception
Tests.

\subsubsection{Superior Kinesics}


  

Passive

   Automatic
 Self

   Constant
The async acquires more insight into people's emotive
signals, gestures, facial expressions, and body
language when it comes time to predict the person's 
emotional state, intent, or reactions. Apply a +10 
modifier to Kinesics Skill Tests.

\subsubsection{Time Sense}


  

Active

   Automatic
 Self

  

   –1
An async with this ability can slow down his perception
of time, making everything appear to move
in slow motion or at reduced speed. In game terms, 
this sleight grants the async a Speed of +1. This extra 
Action Phase, however, can only be spent on mental 
and mesh actions.

\subsubsection{Unconscious Lead}


  

Active

   Automatic
 Self

  

   +0
This sleight allows the async to override their consciousness
and let their unconscious mind take point.
While in this state, the async's conscious mind is only 
dimly aware of what is transgressing, and any memories
of this period will be hazy at best. The advantage
is that the unconscious mind acts more quickly, and 
so the async's Speed is boosted by +1. The character 
remains aware and active, but is incapable of complex
communication or other mental actions and is
motivated by instinct and primitive urges more than 
conscious thought. Though it is recommended that 
the player retain control of their character while using 
Unconscious Lead, the gamemaster should feel free to 
direct the character's actions as they see fit.

hat

ng

to 
%%% 228

\subsection{Psi-Gamma Sleights}

Psi-gamma sleights deal with contacting (reading 
and communicating) and influencing the function of 
biological minds (egos within a biomorph, but also 
including animal life). Psi-gamma is only available to 
characters with Level 2 of the Psi trait.
Most psi-gamma use is handled as an Opposed Test 
between the async and the target (p. 222).

\subsubsection{Alienation}


  

Active

   Complex

 Touch

   T

   +0

  

Psi Assault
Alienation is an offensive sleight that create a sense of 
disconnection between an ego and its morph—similar 
to that experienced when resleeved into a new body. 
The ego finds their body cumbersome, strange, and 
alien, almost like they are a prisoner within it. If the 
async beats the target in an Opposed Test, treat the 
test as a failed Integration Test (p. 272) for the target. 
This effect lasts for the sleight's duration.

\subsubsection{Charisma}


  

Active

   Quick

 Touch

   T

   –1

  

Control
The async uses this sleight to influence the target's 
mind on a subconscious level, so that the target perceives
them to be charming, magnetic, and persuasive.
If the async beats the target in an Opposed Test, they 
gain a +30 modifier on all subsequent Social Skill 
Tests for the chosen duration.

\subsubsection{Cloud Memory}


  

Active

   Complex

 Touch

   T

   –1

  

Control
Cloud Memory allows the async to temporarily disrupt
the target's ability to form long-term memories. If
the async wins the Opposed Test, the target's memory-saving
ability is negated for the duration. The target
will retain short-term memories during this time, but 
will soon forget anything that occurred while this 
sleight was in effect.

\subsubsection{Deep Scan}


  

Active

   Complex

 Touch

   Sustained

   +1

  

Sense
Deep Scan is a more intrusive version of Thought 
Browse (p. 228), made to extract information from the 
targeted individual. If the Opposed Test succeeds, the 
async telepathically invades the target's mind and can 
probe it for information. For every 10 full points of MoS 
the async achieves on their test, they retrieve one piece 
of information. Each item takes one full Action Turn to 
retrieve, during which the sleight must be sustained. The 
target is aware of this mental probing, though they will 
not know what information the async acquired.

\subsubsection{Drive Emotion}


  

Active

   Complex

 Touch

   T

   –1

  

Control
This sleight allows the async to stimulate cortical areas 
of the target's brain related to emotion. This allows 
the async to induce, amplify, or tone down specific 
emotions, thereby manipulating the target. If the async 
beats the target in an Opposed Test, they will act in accordance
with the emotion for the duration and under
certain circumstances may suffer from certain penalties 
(up to +/–30), as determined by the gamemaster. For 
example, an async might receive a +30 Intimidation 
Test modifier against a target imbued with fear.

\subsubsection{Ego Sense}


  

Active

   Complex

 Close

   T

   –1

  

Sense
Ego Sense can be used to detect the presence and 
location of other sentient and biological life forms 
(i.e., egos) within the async's range. To detect these 
life forms, the async makes a single Sense Test, opposed
by each life form within range. The async may
suffer a modifier for detecting small animals and 
insects, similar to the modifier applied for targeting
them in ranged combat (see p. 193); likewise, a
modifier for detecting larger life forms may also be 
applied. If successful, the async has detected that the 
life form is nearby. Every 10 full points of MoS will 
ascertain another piece of information regarding the 
detected life: direction from async, approximate size, 
type of creature, distance from async, etc. The async 
will know if the target moves, if they do so during the 
sleight's duration.

\subsubsection{Empathic Scan}


  

Active

   Quick

 Close

   Sustained

   –2

  

Sense
Empathic Scan enables the async to sense the target's 
base emotions. If the async wins the Opposed Test, 
they intuitively feel the target's emotional current state 
for as long as the sleight is sustained. At the gamemaster's
discretion, this knowledge may provide a modifier
(up to +30) for certain Social skill tests.

\subsubsection{Implant Memory}


  

Active

   Complex

 Touch

   Instant

   +0

  

Control
%%% 229
An async using this sleight can implant a memory 
of up to an hour's length inside the target's mind. 
This memory very obviously does not belong to the 
target—there is no way they will confuse it for one 
of their own. The intent is not to fake memories, but 
to place one of the async's memories in the target's 
mind so that the target can access it just like any other 
memory. This can be useful for ``archiving'' important 
data with an ally, providing a literal alternate perspective
or simply making a ``data dump'' for the target
to peruse. Implant Memory requires an Opposed Test 
against unwilling participants. At the gamemaster's 
discretion, particularly traumatic memories might 
inflict mental stress on the recipient (p. 215).

\subsubsection{Implant Skill}


  

Active

   Complex

 Touch

  

   +0

  
Similar to Implant Memory, this sleight allows the 
async to impart some of their expertise and implant it 
into the target's mind. For the duration of the sleight, 
the target benefits when using that skill. If the async's 
skill is between 31 and 60, the target receives a +10 
bonus. If the async's skill is 61+, the target receives 
a +20 bonus. Implant Skill requires an Opposed Test 
against unwilling participants. In some cases, the 
target has been known to use the skill with the async's 
flair and mannerisms.

\subsubsection{Mimic}


  

Active

   Quick

 Close

   Instant

   +0

  
In a setting where changing your body and face is 
not unusual, people learn to recognize habits and 
personality quirks more often. The async must use 
this sleight on a target and succeed in a Success Test. 
If successful, the async acquires an ``imprint'' of the 
target's mind that they can take advantage of when 
impersonating that ego. The async then receives a +30 
bonus on Impersonation Tests when mimicking the 
target's behavior and social cues.

\subsubsection{Mindlink}


  

Active

   Quick

 Touch

   Sustained

  

Control
Mindlink allows two-way mental communication 
with a target. This may be used on more than one 
target simultaneously, in which case the async can act 
as a telepathic ``server,'' so that everyone mindlinked 
with the async may also telepathically communicate 
with each other (via the async, however, so they 
overhear). Language is still a factor in mindlinked 
communications, but this barrier may be overcome by 
transmitting sounds, images, emotions, and other sensations
Mindlink requires an Opposed Test against
unwilling participants.

\subsubsection{Omni Awareness}


  

Active

   Quick

 Close

  

   –1

  

Sense
An async with Omni Awareness is hypersensitive to 
other biological life that is observing them. During 
this sleight's duration, the async makes a Sense Test 
that is opposed by any life that has focused their attention
on them within the sleight's range; if successful
the async knows they are being watched, but not
by whom or what. It does, however, apply a +30 Perception
bonus to spot the observer. This sleight does
not register partial attention or fleeting attention, or 
simple perception of the async, it only notices targets 
Control
Sense
%%% 230
who are actively observing (even if they are concealing 
their observation). This sleight is effective in spotting a 
tail, as well as finding potential mates in a bar.

\subsubsection{Penetration}


  

Active

   Quick

 Touch

   Instant

   1 per AP point

  

Psi Assault
Penetration is a sleight that works in conjunction with 
any offensive sleight that involves the Psi Assault skill. 
It allows the async to penetrate the Psi Shield of an opponent
by concentrating their psi attack. Every point
of Armor Penetration applied to a psi attack inflicts 
1 point of strain. The maximum AP that may be applied
equals the async's Psi Assault skill divided by 10
(round down).

\subsubsection{Psi Shield}


  

Passive

   Automatic

 Self

   Constant
Psi Shield bolsters the async's mind to psi attack and 
manipulation. If the async is hit by a psi attack, they 
receive WIL ÷ 5 (round up) points of armor, reducing 
the amount of damage inflicted. They also receive a 
+10 modifier when resisting any other sleights.

\subsubsection{Psychic Stab}


  

Active

   Complex

 Touch

   Instant

   +0

  

Psi Assault
Psychic Stab is an offensive sleight that seeks to inflict 
physical damage on the target's brain and nervous 
system. Each successful attack inflicts 1d10 + (WIL ÷ 
10, round up) damage. Increase the damage by +5 if 
an Excellent Success is scored.

\subsubsection{Scramble}


  

Passive

   Automatic

 Self

   Constant
Scramble allows the async using the sleight to hide 
from another async using the Ego Sense or Omni 
Awareness sleights. Apply a +30 modifier to the defending
async's Opposed Test.

\subsubsection{Sense Block}


  

Active

   Complex

 Touch

  

   –1

  

Psi Assault
Sense Block disables and short circuits one of the target's
sensory cortices (chosen by the async), interfering
with and possibly negating a specific source of sensory 
input for the chosen duration. If the async beats the 
target in the Opposed Test, the target suffers a –30 
modifier to Perception Tests with that sense equal doubled
to –60 if the async scores an Excellent Success).

\subsubsection{Spam}


  

Active

   Complex

 Touch

  

   +0

  

Psi Assault
The sleight allows the async to overload and flood one 
of the target's sensory cortices (chosen by the async), 
spamming them with confusing and distracting sensory
input and thereby impairing them. If the async
wins the Opposed Test, the target suffers a –10 modifier
to all tests the duration of the sleight (doubled to
–20 if the async scores an Excellent Success).

\subsubsection{Static}


  

Active

   Complex

 Close

   Sustained

   +0

  

None
The async generates an anti-psi jamming field, impeding
any use of ranged sleights within their range. All
such ranged sleights suffer a –30 modifier. This sleight 
has no effect on self or touch-range sleights. 

\subsubsection{Subliminal}


  

Active

   Complex

 Touch

   Instant

   +2

  

Control
The Subliminal sleight allows the async to influence 
the train of thought of another person by implementing
a single post-hypnotic suggestion into the mind
of the target. If the async wins the Opposed Test, the 
recipient will carry out this suggestion as if it was 
their own idea. Implanted suggestions must be short 
and simple; as a rule of thumb, the gamemaster may 
only suggestions encompassed by a short sentence 
(for example: ``open the airlock,'' or ``hand over the 
weapon''). At the gamemaster's discretion, the target 
may receive a bonus for resisting suggestions that are 
immediately life threatening (``jump off the bridge'') 
or that violate their motivations or personal strictures. 
Suggestions do not need to be carried out immediately, 
they may be implanted with a short trigger condition 
(``when the alarm goes off, ignore it'').

\subsubsection{Thought Browse}


  

Active

   Complex

 Touch

   Sustained

   –1

  

Sense
Thought Browse is a less-intrusive form of mind 
reading which scans the target's surface thoughts for 
certain ``keywords'' like a particular word, phrase, 
sound, or image chosen by the async. Rather than digging
through the target's ego as with the Deep Scan
sleight, Thought Browse merely verifies whether a 
target has a particular person, place, event, or thing 
in mind, which can be used by a savvy investigator to 
draw conclusions without the need to invade the mind 
%%% 231
directly. Thought Browse may be sustained, allowing 
the async to continue scanning the target's thoughts 
over time. The async must beat the target in an Opposed
Test for each scanned item.

\section{Psychosurgery}

Given the reach of neuroscience in the time of 
\textit{Eclipse Phase,} it is easy to think of the mind as 
programmable software, as something that can be 
reverse-engineered, re-coded, upgraded, and patched. 
To a large degree, this is true. Aided by nanotechnology
genetics, and cognitive science, neuroscientists
have demolished numerous barriers to understanding
the mind's structure and functions, and even
made great leaps in unveiling the true nature of 
consciousness. Genetic tweaks, neuro-mods, and 
neural implants offer an assortment of options for 
improving the brain's capabilities. The transhuman 
mind has become a playground—and a battlefield. 
Nanovirii unleashed during the Fall infected millions
altering their brains in permanent ways, with
occasional outbreaks still occurring a decade later. 
Cognitive virii roam the mesh, plaguing infomorphs 
and AIs, reprogramming mind states. An ``infectious 
idea'' is now a literal term.

In truth, mind editing is not an easy, safe, and 
error-proof process—it is difficult, dangerous, and 
often flawed. Neuroscience may be light years 
ahead of where it was a century ago, but there are 
many aspects of the brain and neural functions 
that continue to confound and elude even the 
brightest experts and AIs. Technologies like nano-neural
mapping, uploading, digital mind emulation,
and artificial intelligence are also comparatively 
in their infancy, being mere decades old. Though 
transhumanity has a handle on how to make these 
processes work, it does not always fully understand 
the underlying mechanisms.

Any neurotech will tell you that mucking around 
in the mind's muddy depths is a messy business. 
Brains are organic devices, molded by millions of 
years of unplanned evolutionary development. Each 
is grown haphazardly, loaded with evolutionary 
leftovers, and randomly modified by an unlimited 
array of life events and environmental factors. Every 
mind features numerous mechanisms—cells, connections
receptors—that handle a dizzying array
of functions: memory, perception, learning, reasoning
emotion, instinct, consciousness, and more. Its
system of organization and storage is holonomic, 
diffused, and disorganized. Even the genetically-modified
and enhanced brains of transhumans are
crowded, chaotic, cross-wired places, with each 
mind storing its memories, personality, and other 
defining features in unique ways.
What this means is that though the general architecture
and topography of neural networks can
be scanned and deduced, the devil is in the details. 
Techniques used to modify, repair, or enhance one 
person's mind are not guaranteed equal success 
when applied to another's brain. For example, the 
process by which brains store knowledge, skills, and 
memories results in a strange chaining process where 
these memories are linked and associated with other 
memories, so attempts to alter one memory can 
have adverse affects on other memories. In the end, 
minds are slippery and dodgy things, and attempts 
to reshape them rarely go as planned.

\subsection{The Process Of Psychosurgery}

Psychosurgery is the process of selective, surgical alteration
of a transhuman mind. It is a separate field from
neural genetic modification (which alters genetic code), 
neuralware implantation (adding cybernetic or biotech 
inserts to the brain or nervous system), or brain hacking 
(software attacks on computer brains, neural inserts, and 
infomorphs), though they are sometimes combined.
Psychosurgery is almost always performed on a 
digital mind-state, whether that be a real-time emulation
a backup, or a fork. In most cases, the subject's
mind-state is copied via the same technology and process
as uploading or forking, and run in a simulspace.
The subject need not be willing, and in these cases 
the subject's permissions are restricted. Numerous 
psychosurgery simulspace environments are available, 
each custom-designed for facilitating specific psychosurgical
goals and programmed with a thorough
selection of psychotherapy treatment options.

p y

py

p
%%% 232

The actual process of psychosurgery breaks down 
into several stages. First is diagnosis, which can 
involve the use of several neuro-imaging techniques 
on morphed characters, mapping synaptic connections
and building a neurochemical model. It can
also involve complete psychological profiling  and 
psychometric behavioral testing, including personality
tests and simulspace scenario simulations. Digital
mind-states can be compared to records of people 
with similar symptoms in order to identify related 
information clusters. This analysis is used to plan 
the procedure.

The actual implementation of psychosurgical alteration
can involve several methods, depending on
the desired results. Applying external modules to the 
mind-state is often the best approach, as it doesn't 
meddle with complicated connections and new inputs 
are readily interpreted and assimilated. For treatments
mental health software patches compiled from
databases of healthy minds are matched, customized, 
and applied. Specialized programs may be run to 
stimulate certain mental processes for therapeutic 
purposes. Before an alteration is even applied, it may 
first be performed on a fork of the subject and run at 
accelerated speeds to evaluate the outcome. Likewise, 
multiple treatment choices may be applied to time-accelerated
forks this way, allowing the psychosurgeon
to test which is likely to work best.

Not all psychosurgery is performed for the subject's 
benefit, of course. Psychosurgery can be used to interrogate
or torture prisoners, erase memories, modify
behavior, or inflict crippling impairments. It is also 
sometimes used for legal punishment purposes, in 
an attempt to impair criminal activity. Needless to 
say, such methods are often brute-forced rather than 
fine-tuned, ignoring safety parameters and sometimes 
resulting in detrimental side effects.

\subsection{Psychosurgery Mechanics}

In game terms, psychosurgery is handled as a Task 
Action requiring an Opposed Test. The psychosurgeon 
rolls Psychosurgery skill against the target's WIL x 3. 
Apply modifiers as appropriate from the Psychosurgery
Modifiers table.

If the psychosurgeon succeeds and the subject fails, 
the psychosurgery is effective and permanent. The 
alteration becomes a permanent part of the subject's 
ego, and will be copied when uploaded (and sometimes
when forking).

If both sides succeed but the psychosurgeon rolls 
higher, the psychosurgery is effective but temporary. It 
lasts for 1 week per 10 points of MoS.

If the subject rolls higher, or if the psychosurgeon 
fails their roll, the attempt does not work.

The timeframe listed for psychosurgical procedures 
is according to the patient's subjective point of view. 
Since most subjects are treated in a simulspace, time 
acceleration may drastically reduce the amount of 
real-time such a procedure requires (see \textit{Defying Na-}
\textit{ture's Laws,} pp. 240–241).

\subsubsection{Mental Stress}

Psychosurgery is a modification to the transhuman 
mind, and sometimes to the actual person that resides 
in that mind. It is unsurprising then that psychosurgery
places stress on the subject's mental state and
sometimes even inflicts mental traumas.

Each psychosurgery option lists a Stress Value 
(SV) that is inflicted on the subject regardless of 
the tests' success or failure. If the psychosurgeon 
achieves an Excellent Success (MoS 30+), this stress 
is halved (round down). If the psychosurgeon rolls 
a Severe Failure (MoF 30+), the stress is doubled. 
Alternately, a Severe Failure could result in unintended
side effects, such as affecting other behaviors
emotions, or memories.
%%% 233
If a critical success is rolled, no stress is applied at 
all. If a critical failure is rolled, however, an automatic 
trauma is applied in addition to the normal stress.
Some psychosurgery conditions may also affect the 
SV, as noted on the Psychosurgery Modifiers table.

\subsection{Roleplaying Mind Edits}

Many of the changes incurred by psychosurgery are 
nebulous and difficult to pin down with game mechanics
Alterations to a character's personality and
mind-state are often better handled as roleplaying factors
anyway. This means that players should make a
real effort to integrate any such mental modifications 
into their character's words and actions, and gamemasters
should ensure that a character's portrayal
plays true to their mind edits. Some psychosurgical 
mods can be reflected with ego traits, while others 
might incur modifiers to certain tests or in certain 
situations. The gamemaster should carefully weigh a 
brain alteration's effects, and apply modifiers as they 
see appropriate.

\subsection{Psychosurgery Procedures}

The following alterations may be accomplished with 
psychosurgery. At the gamemaster's discretion, other 
mind-editing procedures may be attempted, using 
these as a guideline.

\subsubsection{Behavioral Control}


1 week

Limit/Boost –10; Block/Encourage –20, 

Expunge/Enforce –30

(1d10 ÷ 2, round up)
Commonly used for criminal rehabilitation, behavioral
control attempts to limit, block, or expunge a
specific behavior from the subject's psyche. For example
a murderer may be conditioned against acts of aggression
or a kleptomaniac might be restricted from
stealing. Some people seek this adjustment willingly, 
such as socialite glitterati who restrict their desire to 
eat, or an addict who cuts out their craving for a fix.
Behavioral control can also be applied as an unleashing
or reinforcement. A companion may desire
to eliminate their sexual inhibitions, for example, or 
a hypercorp exec may boost his commitment to place 
work above all else.
A character will simply feel compelled to avoid 
a behavior that is \textit{limited }(perhaps suffering a –10 
modifier), but will find it quite difficult to pursue a behavior
that is \textit{blocked }(requiring a WIL x 3 Test, and
suffering a –20 modifier). They will find themselves 
completely incapable of initiating a behavior that is 
\textit{expunged,} and if forced into the behavior will suffer 
a –30 modifier and (1d10 ÷ 2, round up) points of 
mental Stress.
Likewise, a character will feel compelled to pursue a 
behavior that is \textit{boosted,} and will find it hard to avoid 
engaging in a behavior that is \textit{encouraged }(requiring a 
WIL x 3 Test to avoid). They will have no choice but 
to engage in \textit{enforced }behaviors, and will suffer (1d10 
÷ 2, round up) points of mental Stress if prevented 
from doing so.

\subsubsection{Behavioral Masking}


1 week

–20

1d10 ÷ 2, round up
Given the ability to switch bodies, many security and 
law enforcement agencies have resorted to personality
and behavioral profiling as a means of identifying
people even when they resleeve. Though such systems 
are far from perfect, someone's unconscious habits 
and quirks could potentially give them away. Characters
who wish to elude identification in this way may
undergo behavioral masking, which seeks to alter and 
change the character's unconscious habits and social 
cues. Apply a +30 modifier when defending against 
such identification systems and Kinesics Tests.

\subsubsection{Deep Learning}


Skill Learning Time ÷ 2

+20

1
Using tutorial programs, memory reinforcement protocols
conditioning tasks, and deep brain stimulation,
the subject's learning ability is reinforced, allowing 
them to learn new skills more quickly.

\subsubsection{Emotional Control}


1 week

Limit/Boost –10; Block/Encourage –20, 

Expunge/Enforce –30

(1d10 ÷ 2, round up) + 2
Similar to behavioral control, emotional control seeks 
to modify, enhance, or restrict the subject's emotional 
responses. Some choose these modifications willingly, 
such as limiting sadness in order to be happier, or 
encouraging aggression in order to be more competitive
Mercenaries and soldiers have been known
to expunge fear. Follow the same rules as given for 
Behavioral Control.
Improper Preparatory Diagnosis

–30

+1
Safety Protocols Ignored

+20

x2
Simulspace Time Acceleration

–20

+2
Subject is an AI, AGI, or uplift             

–20

+1
%%% 234

\subsubsection{Interrogation}


Variable (gamemaster discretion; 1 week default)

+30

1d10
Psychosurgery can be used for interrogative purposes 
via the application of mental torture and manipulation
A successful Psychosurgery Test applies a +30
modifier to the Intimidation Test for interrogation.

\subsubsection{Memory Editing}


1 week (2 weeks adding/replacing)

–10 (willing) or –30 (forced)

(1d10 ÷ 2, round up)
By monitoring memory recall (forcibly invoked if necessary
psychosurgeons can identify where memories
are stored in the brain and target them for removal. 
Memory storage is complex and diffused, however, 
and often linked to other memories, so removing one 
memory may affect others (gamemaster discretion).
Adding or replacing memories is a much more complicated
operation and requires that such memories be
copied from someone who has experienced them or 
manufactured with XP software. Even when successfully
implanted, fake memories may clash with other
(real) memories unless those are also erased.

\subsubsection{Personality Editing}


1 week

Minor –10; Moderate –20, Major –30

(1d10 ÷ 2, round up) + 3
Possibly the most drastic psychosurgery procedure, 
personality editing involves altering the subject's core 
personality traits. The personality factors that may be 
modified is almost unlimited, including traits such as 
openness, conscientiousness, altruism, extroversion/
introversion, impulsiveness, curiosity, creativity, confidence
sexual orientation, and self-control, among
others. These traits may be enhanced or reduced to 
varying degrees. The effect is largely reflected by roleplaying
but the gamemaster may apply modifiers as
they see fit.

\subsubsection{Psychotorture}


Variable

+30

1d10 SV per day
Psychotorture is mental manipulation for the simple 
intention of causing pain and anguish, reflected  in 
game terms as mental stress and traumas. Prolonged 
torture can lead to serious mental disorders or worse.

\subsubsection{Psychotherapy}


Variable

+0

0
Therapeutic psychosurgery is beneficial for characters 
suffering from mental stress, traumas, and disorders. A 
successful Psychosurgery Test applies a +30 modifier 
to mental healing tests, as noted on p. 215.

\subsubsection{Skill Imprints}


1 week per +10

+0

 1 per +10
Skill imprinting is the use of psychosurgery to insert 
skill-set neural patterns in the subject's brain, temporarily
boosting their ability. Skill imprints are artificial
boosts, however, degrading at the rate of –10 per day. 
No skill may be boosted higher than 60.

\subsubsection{Skill Suppression}


1 day per –10

–10

1 per +10
Skill suppression attempts to identify where skills are 
stored in the brain and then block or remove them. 
The subject's skill is impaired and may be lost entirely.

\subsubsection{Tasping}


1 day

+10

1
Tasping is the use of deep brain stimulation techniques 
to tickle the mind's pleasure centers. Though this 
procedure is often used for therapeutic purposes for 
patients suffering from depression or other mental 
illnesses, the intent with tasping is to overload the 
subject into a prolonged state of almost unendurable 
bliss. Such stimulation is highly addictive, however, so 
character's exposed to it for any length of time (over 
1 hour, subjective) are likely to pick up the Addiction 
trait (p. 148). Some criminal organizations have been 
known to use tasping addiction and rewards as a 
means of controlling those under their thrall.
%%% 235
