\chapter{Piratages Cognitifs} \label{cha:mind-hacks} 













\section{PSI} 

\begin{quotation} 

$\triangleright $ Desdemona: Contente de te revoir. J'espère que tu as eu un agréable farcast depuis Pelion et que tu ne ressent pas trop de manque. Pendant que tu étais en ballade, un message d'Aeneas accompagné d'un précis sur les psi, extrait du backup de l'infomorph du psygénéticien Daborva(Stellint, Station de Recherche Dipôle sur ganymède), a été rerouté pour diffusion sur ton nœud Firewall. \end{quotation} 

Inventé par le biologiste Bertold P. Wlesner, "psi" était originellement un furre-tout utilisé pour décrire un grand nombre d'aptitude "pyschique" et d'autres phénomènes apranormaux supposés tels que la télépathie et la perception extra-sensorielle. Alors que le terme était largement utilisé dans le domain de la parapsychologie et dans la pop culture au vingtième siècle et au début du vingt et unième siècle, l'étude de psi était largement considéré comme une pseudoscience utilisant une méthodologie vciée et ayant graduellement perdu ses supports et ses financements. pendant la Chute, cependant, des rumeurs répétées et nombre de phénomènes inexpliqués furent rapportés aux scientifiques, aux chefs militaires et aux adeptes de la singularité. De nombreux nanovirus ont avaient étés lachés sur la transhumanité, se propageant à travers les populations et se transformant au fur et à mesure de leur propagation. Certains n'infligèrent que des changements biologiques ou psychologiques mineurs et des déficiences légères, mais un grand nombre d'entre eux étaient virulents et mortels. La variantes qui inspira le plus de crainte étaient celles que Firewall allait appelé le Virus Exsurgent - une nano-peste transformatrice qui fait muter ses victime et les asservit à sa volonté. Il a également été observé que le Virus Exsurgent modifiait radicallement les motifs neuronaux et l'état mental du sujet, affectant l'ordonnancement synaptique et allant jusqu'à moduler les courants électriques qui parcourent les synapses. Ces changements altèrent et amplifie la cognition de la victime et semble les doter d'une capacité à percevoir et même affecter les pensées des autres à une faible distance - une capacité appellée "psi" car les facteurs de causalités continuent de nous mystifier. L'existence et la nature de ce phénomène restent prudemment dissimulés et tenue à l'abri dans les habitats contrôllés, afin de ne pas déclencher de panique générale. Aprmi les anarchistes et d'autres communautées ouvertes, la connaissance du psi est plus répandue, mais les détails restent vagues et les rapports sont généralement acceuillis avec scepticisme. Le virus Exsurgent est cependant extraordinairement mutable et adaptable, et deux chercheurs argonautes qui étaient au courant et qui étudiaent le phénomène psi ont rapidement fait une découverte intéressante. Une souche particulière du virus qui dotait e sujet de capacités mental exceptionnelles s'évarait également ne pas engager le sujet dans le processus de transformation des autres souches. Bien que l'infection ait toujours d'autres inconvénients, Firewall et d'autres agents en sont venus à considérés cette souche comme "sûre" dans le sens où le sujet ne se métamorphose pas en quelque chose d'autres et que leur personnalité principale reste intacte. Intrigué par le fait que cette piste puisse mener à invalider les effets d'autres souches Exsurgente, Firewall et d'autres continuèrent d'expérimenter avec la souche et avec la coopération de sujets de test volontaires (ou, d'après certains rapport, de victimes involontaires dans le cas de certaines autoritées et hypercops). 

\subsection{La Nature du Psii} Notée la souche Watts-MacLeod d'après le nome des chercheurs qui l'ont sisolés, d'autres études ont acquis une connaissance des effetc du virus sur les cerveaux transhumains. Des analyses minutieuse de sujest infectés ont permis de dcouvrir que leurs synapses altérées génèrent une forme d'onde cérébrale modulée qui est extrêmement difficile à détecter. "Ceux qui savent" en sont venus à faire référence à ces ondes cérébrales asynchrones sou sle nom de "ondes psi," en suivant la désignation par lettre Grecque  des autres ondes cérébrales (alpha, béta, delta, gamma, théta). De la même manière, les individus affectés sont appelés "async." L'exploration des facteurs causals explicites à l'origine des ondes psi restent inconnues. Des théories à propos du processus mental extraordinaire qui incluent le changement d'état quantique ont été explorée mais reste non-concluantes. La cartographie et l'imagerie neurales ont permis aux scientifiques de détecter des strucstures internes au cerveau, de l'activité neurale, et des perturbations dans le champ bioélectrique du cerveau qui sont associée au processus psi, mais toutes les tentatives pour dupliquer ces capacités dans des cerveaux non-infectés ont aboutit à des échecs ou pire. Les tentatives d'identification des async par des motifs d'ondes psi n'ont aucune garantie de succès. De nombreuses impasses ont amenés beaucoup de chercheurs à postuler que les mécaniques sous-jacentes au psi sont simplement trop étranges et trop au-delà de la compréhension des scicences psychique transhumaine - renforçant peut-être les théories que le virus Exsurgent est de nature étrangère de fait. Une des spéculations majeures est que ce changement amenés dans l'esprit par l'infection déclenche en fait des sous-systèmes neuraux, activant une sorte de champ quantique ou créant peut-être des condensat de bose-Einstein à l'intérieur du cerveau, permettant d'effectuer des calculs quantiques ou même d'hypercalucl. Cela active les capacités mentales de l'async à un level fourni par les implants moderne et les neuro-mods - et parfois au-delà. Cela n'explique cependant pas les capacités d'autres async, particulièrement celles utilisé pour lire ou affecter d'autres esprits biologiques. Ces capacités semblent impliquer la lecture d'ondes cérébrales à courte portée ou affecter l'esprit des autres via un contact physique direct avec le champ bio-électrique de la cible. Bien entendu, je ne peux que spéculer avec les informations que Firewall a découvertes - il est fort probable que certaines des hypercoprs ou d'autres factions ait fait d'autres percées, mais gardent l'information pour eux. L'initiation et l'utilisation de talent psi est générallement comprise comme prenant place à un niveau subconscient, ce qui veut dire que l'async n'est pas activement au courant du processus fondamental qui alimente ses ondes psi. L'entraînement de certaines compétences permet cependant à un async de se concentrer sur certaines action et capacités psi. Ces capacités sont appelés "exploits:" des algorithmes cognitif ou mnémonique d'usage psi enraacinés dans l'ego de l'async. Le pourcentage estimé de la population transhumaine qui ai contractée la souche Watts-MacLeod reste statistiquement insignifiante - moins de 0,001\% de la population. la grande majorité des async ont étés recrutés par diverses agences, ont "disparus" pour études ou ont simplement été éliminiés au motif de menace potentielle. Dix ans après la Chute, Firewall et d'autres agences en sont venues à considérer l'infection par Watts-macLeod relativement sûre, bien que nous restions prudent vis à vis des effets de bords ou d'autres dangers cachés. La plupart de ceux qui sont engagés dans l'étude du phénomène considèrent mainteannt les asyncs comme un outil utile pour combattre le virus Exsurgent ou d'autres menaces en dépit des protestations de ceux qui sont convaincus que les async ne contrôllent pas leur propres esprits et que l'on ne peux pas leur faire confiance. Jusqu'à présent, nous n'avons pas encore découvert de cas d'infection par Watts-MacLeod qui ont infligés autre chose que des capacités psi, bien qu'il semble y avoir un risque accru que les async succombent à d'autres souche Exsurgente si ils les croisent. Il y a d'autres risques associée avec l'infection Watts-Macleod, tels qu'une fatigue extrrme et même un biofeedback létal résultant d'un usage intensif d'exploits psi et une tendance statistiques à développer des désordres mentaux due au stress mental accru placé sur l'esprit de l'async. 

\newpage

\subsection{Æther Jabber: Asyncs} \# Start Æther Jabber \# \\ \# Active Members: 2 \# 

\textbf{\textit{1}} Sorry to bother you, but my muse just alerted me to this excerpt that was sent around to my Firewall team. Is this for real? J'ai entendu la discussion sur les psi avant - suffisament pour être convaincu qu'il y avait quelque chose, même si on ne pouvait pas le prouver - mais ce truc à propos d'une variante de l'Infection Exsurgente, c'est juste trop. Va-t-on sérieusement travailler avec quelqu'un qui est un porteur connu? Et peux-tu dissiper quelques doutes sur la façon dont les async utilisent leur mojo? Je suis inquiet maintenant. Et puisque tu es connectés aux Médéens, je pensait tenter ma chance et te poser ces questions. 

\textbf{\textit{2}} Well, as to the Medeans ... that’s history. Je suis de retour sur le marché des freelancers actuellement. Mais aucun problème, je vais essayer d'expliquer. Je sait que ce n'est pas simple à saisir. 

\textbf{\textit{1}} Shiny. 

\textbf{\textit{2}} Yes, Srit was once infected with a strain of the Exsurgent virus, probably on Mars near the end of the Fall. Je dit "a été" car la souche Watts-Macleod semble devenir dormante peu de temps après avoir fini de recâbler le cerveau de la victime; l'infection de naonobots meurt et est évacuée hors du système, contrairement à d'autres souches Exsurgente qui reste implantée et continuent de transformer le sujet. Du moins, c'est la théorie dominante - j'ai aussi vu quelques spéculations comme quoi l'esprit des async pourrait être modifié pour qu'il puisse continuer à produire des bio-nanobots qui  s'attardent dans le cerveau, bien que leur fonction reste floue. Cepenant, l'opinion prévalente parmi nos meilleurs neuroscientifiques est que les gens comme Srit sont sûrs et non-infectieux une fois que le virus a terminé sa tâche. J'irai même un peu plus loin et dirai que l'opinion dominante est que l'on peut leur faire confiance, en partant du principe qu'ils n'attrapent pas une autre infection ... ce qu'ils semblent avoir malheureusemnet un peu tendance à avoir. Bien entendu, tout le monde n'est pas d'accord, mais nous avons une abondance de paranoïa dans nos cercles. Jusqu'à présent, nous n'avons pas vu de preuve que l'un de nos async nous ai trahi suite à cette infection initiale, et l'utilité d'avoir des psiactif dans nos rangs est simplement trop importante pour les écarter. Très bien. Je ne peux pas dire que je lui fait confiance, mais j'essayerai et lui donnerai le bénéfice du doute. Que je soit maudit si je fait confiance à un async qui ne travaille pas pour Firewall - qui sait quel horreur une corpo comme Skinthetic peut prépzrer dans leurs labos officieux. 

\textbf{\textit{2}} That seems like a wise choice. 

\textbf{\textit{1}} Maybe you can put my mind at ease by explaining to me in a bit more detail how Watts-MacLeod infection occurs. 

\textbf{\textit{2}} Well, like the other Exsurgent strains you are unfortunately familiar with, the primary transmission vector is a nanovirus, but we speculate that it may also be transmitted as a digital computer virus or possibly even as a basilisk hack. La forme d'infection physique est véhiculée par des nanobots techno-organiques et hautement avancés qui infectent une biomorph et utilisent des mécanismes de mimétismes biologique. Ces nanobots ont plusieurs longueurs d'avances sur tout ce que notre technologie peu produire, elles sont trés difficile a détecter et peuvent surpasser la plupart des contre-mesures défensive. Les esprits infectés subissent en général un recâblage, et ces changements seront copiés lorsque l'ego est uploadé. Les synthmorphs et les informophs restent immunisés à cette nano-infection, mais ils sont théoriquement vulnérables à d'autres vecteurs de transmission. 

\textbf{\textit{1}} I’ve heard that synthmorphs are effectively invulnerable to psi as well. C'est vrai? 

\textbf{\textit{2}} Yes. Tout ce que l'on peut dire, c'est que les capacités des async n'affectent que les esprits biologique - leur propre esprit ou celui des autres. Et ils peuvent lire ou affecter l'esprits des autres seulement sur une distance trés courte, nécessitant un contact physique la plupart du temps. Les esprist à moitié biologique des pods sont également vulnérable, bien que les effets soient réduits. De la même façon, les asyncs ont besoin d'un cerveau bilogique pour utiliser leurs capacités - ils ne peuvent utiliser leur psi si ils sont incarnés dans une synthmorph et ont de grosse difficultés à le faire depuis un pod. 

\textbf{\textit{1}} Interesting. Donc, je doit te redemander - tu es certaine qu'elle est sûre? J'ai entendu dire que certains de ces asyncs peuvent être de vrais cinglés. 

\textbf{\textit{2}} I’ve heard from several of these asyncs directly. En fait, l'infection recâble leur cerveau, et certains d'entre eux sortent du processuss en se sentant fondamentallement altérés. Soit il se sentent comme quelqu'un d'autre, soit ils sentent que quelque chose de nouveau fait partie d'eux - quelque chose qu'ils n'aiment pas forcément. L'un l'a décrit comme une présence, d'autre comme un vide noir qui leur murmurait des choses à l'oreille. Encore un autre l'a décrit comme si cela avait déonné une personnalité à la part d'inconscient de leur esprit, ce qui n'a fait que rendre le fossé entre parties consciente et inconsciente de l'esprit encore plus intimidante. Certains préfèrent se suicider et retourner à un backup pré-infection. Bien qu'ils aient une tendance à craquer plus facilement, je ne les ai pas entendu parler de ses capacités comme quelque chose d'incontrollable. 

1 Tout cela m'a l'air grandiose. On n'a rien d'autre sur la manière dont tout ce psi fonctionne réellement? 

\textbf{\textit{2}} Unfortunately, we don’t. Même les Prométhéens n'ont pas étés d'une grande aide. Il y a des théories bien sûr, mais rien que nous n'ayons pu vérifier via une expérimentation rigoureuse. Le fait que les factiosn qui sont au courant de l'existence du psi ne comparent aps réellement leurs notes n'aide pas non plus - ils sont tous bien trop occupé à trouver des façons de le transformer en arme et de l'utiliser les uns contre les autres, au lieu d'essayer de trouver une façon de s'en servir au bénéfice de la transhulanité. 

\textbf{\textit{1}} Of course. Les TITANs ne nous ont pas eu, mais nous sommes toujours capable de finir le travail. Le fait que nous n'ayons rien m'inquiètes. 

\textbf{\textit{2}} It’s important to keep perspective. La Transhumanité revient de loin et est parvenu à quelques accomplissements impressionant, mais nous sommes encore au tout début de la compréhension de l'univers. Ce à quoi nous pourrions être en train de faire face ici est quelque chose de concocté par une intelligence tellement différente de la notre que nous ne sommes que des insectes insignifiants en comparaison. Ils semblent avoir une prise sur l'univers qui est simplement bien au-delà de notre capacité à le comprendre. Nous ne devriosn pas être arrogant et penser que nous pouvons déchiffrer tous les mystères sur lesquels nous tombons ... nous devrions à la place être vraiment, vraiment effrayés. 

\newpage



Bien que la neuroscience est atteint des sommets, permettant aux esprit d'être scannés en profondeur, cartographiés, et émulés comme en tant que logiciel, le cerveau transhumain reste un endroit complexe, pas encore totallement compris et complètement ébouriffé. En dépit de la prévalence des modifications neurales, jouer avec le siège de la conscience reste une procédure délicate et dangereuse. Néanmoins, la psychochirurgie - l'édition de l'esprit comme un logiciel - reste commune et largement répandue, parfois avec des résultats inattendus. De al même manière, alors même que les connaissances des neuroscientifiques croissent de manière exponentielle, certains ont découvert que les esprits sont bien plus mystérieux que ce qu'ils n'avaient jamais imaginé. Pendant la Chute, des rapport épars d'"activité anormale" générées par des individus infectés par l'une des nombreuses naoninfection circulant à l'époque ont étés mise sur le compte de la peur et de la paranoïa, mais des investigatiosn ultérieures par des laboratoires financés par des caisses noires ont rpouvé le contraire. Maintenant, des réseaux confidentiels de haut niveau murmurent que cette infection inflige des changement subtils dans le réseau neuronal de la cible qui les imprègent avec des capacités étranges et inexpliquées. La nature et la mécanique exacte de ces capacités restent inexpliquées et hors de la portée de la science transhumaine moderne. Étant donné la preuve d'un nouveau type d'onde cérébrale et la nature paranormale de ce phénomène, il y est fait librement référence par "psi". 

\section{Psi} Dans Eclipse Phase, psi est considéré comme un état cognitif particulier résultant d'une infection par la souche mutante - et heureusement par ailleurs bégnine - Watts-Macleod du virus Exsurgent (p. 367). Ce fléau modifie l'esprit de la victime, lui conférrant des capacités spcéiales. Ces capacités sont inhérentes à l'architecture du cerveau et sont copiées lorsque l'esprit est uploadé, permettant aux personnages de conserver leurs capacités psi lorsqu'ils passent d'une morph à une autre. 

\subsubsection{Prérequis} 

Pour manier le psi, un personnage doit acquérir le  trait Psi (p. 147) lors de la création de personnage. Il est téhoriquement possible d'acquérir l'usage du psi en cours de jeu via l'infection par la souche Watts-macLeod; voir le chapitre Le Virus Exsurgent, p.  362. La capacité Psi est considérée comme une capacitée innée de l'ego et non pas une prédisposition biologique ou génétique de la morph. Alors que les chercheurs dans le domaine psi ne comprennent pas encore comment il est possible de transférer cette capacité via les uploads, la sauvegarde ou le farcast, il a été spéculé que tout les composants de l'ego d'un async sont liées à un niveau quantique, ou qu'ils possèdent la capacité de les lier eux-même ou de former une configuration ou un alignement unique en une seule entité même après qu'ils aient étés uploadé ou téléchargés. Ce processus d'entremèlement spéculé est aussi probablement à l'origine des déficience que lesasyncs rencontrent lorsqu'ils s'adaptent à une nouvelle morph (voir plus bas). 

\subsubsection{Morphs Et Psi} 

Les asyncs ont besoin d'un cerveau biologique pour faire usage de leurs capacités (les cerveaus d'animaux élevés comptent). Un async dont l'ego est téléchargé dans une infomorph ou un cerveau entièrement informatisé (synthmorphs) n'as pas accès à ses capacités tant qu'ils demeurent dans cette morph. Les async résidant dans un pod peuvent utiliser le psi, mais leurs capacités sont restreintes car les cerveaux des pods ne sont que partiellement biologique. Les ayncs morphés dans des pods souffrent d'un modificateur de -30 sur tous les tests impliquant l'utilisation d'exploits psi et l'impact lié à l'utilisation de ces exploit est doublé. 

\subsubsection{Acclimatation à une Morph} 

L'esprit des async subissent une difficultés supplémentaire pour s'ajuster aux nouvelles morphs. Pendant 1 journée après que le personnage se soit réincarné, il souffrira des effets d'un dérangement simple (p. 210). Le maïtre de jeu et les joueurs devraient choisir un dérangement approrpié au personnage et à l'histoire. Des dérangement mineurs sont recommandés, mais, à la discrétion du maïtre de jeu, des dérangements modérés ou majeurs peuvent être appliqués. Aucun trauma n'est infligé avec ce dérangement. 

\subsubsection{Fièvre de Morph} 

Les async trouvent irritant et traumatisant le fait de vivre comme une informoph, un pod ou une synthmorph  sur de longues périodes de temps. Ce phénomène, connu comme fièvre de morph, peut causer un dérangement temporaire et du trauma à l'ego des async, pouvant aller jusqu'au niveau des désordres mentaux permanents. Si il est stocké ou gardé captif en tant qu'infomorph active (i.e. pas en stase virtuelle), un async peut devenir fou si il ne bénéficie pas d'une aide psychologique par un programme analgésique ou du personnel de support pendant le stockage. En terme de jeu, les asyncs subissent 1d10 $\div$ 2 (arrondir au supérieur) points de dommage mental dus au stress par mois pendant lesquels  ils restent dans un pod, une synthmorph ou une infomorph sans assistance psychologique par un psychiatre, un logiciel ou une muse. 

\subsubsection{Inconvénients du Psi} 

Il y a plusieurs inconvénients à posséder la capacité psi: 

\begin{itemize} \item La variante du virus Exsurgent qui donne à un personnage la capaciét psi, recâble également son cerveau. Un effet de bord malheureux à ce changement est que les async acquièrent une vulnérabilité aux maladies mentales. Réduisez le Seuil de Trauma d'un async de 1. \item L'instabilité mentale qui accompagne l'infection psi tends également à déranger l'esprit du personnage. Les async acuiqèrent un trait négatif Dérangement (p. 150) pour chaque niveau auquel ils possèdent le trait Psi sans bénéficier de CP de bonus. Le maître de jeu et les joueurs devraient se mettre d'accord sur un dérangement approprié au personnage. Ce dérangement peut être traité avec le temps, en suivant les règles normales (voir Soins de l'Esprit et Psychothérapie, p. 215). \item Les personnages avec le trait Psi sont également vulnérables à l'infection par d'autres souches du virus Exsurgent. Le personnage souffre d'un modificateur de -20 lorsqu'il résiste à une infection Exsurgente (p. 362). \item Les échecs critiques en utilisant le psi a tendance à stresser l'esprit de l'async. À chaque fois qu'un échec critique est lancé en effectuant un test lié à un exploit psi, l'async souffre d'une attaque cérébrale. Ils souffrent d'un modificateur de -30 et sont incapable d'agir jusqu'à la fin du prochain Tour d'Action. Ils doivent aussi réussir un Test de VOL + COG ou s'effondrer. \end{itemize} 

\subsubsection{Compétences et Exploits Psi} 

Les utilisateurs transhumain de psi peuvent manipuler leur ego et par ailleurs créer des effets qui dans la majorité des cas ne peuvent être reprosuit ou imités par des moyens technologique. Pour utiliser ces capacités, ils entraïnent leur processus mental et s'exercent à l'utilisation d'algorithmes cognitifs appelés exploits psi, qu'ils peuvent de manière subconsciente rappeler et utiliser lorsque c'est nécessaire. Les exploits psi se répartissent en deux catégories: les chi-psi (améliorations cognitives, p. 223) et les gamma-psi (lecture et manipulations des ondes cérébrales, p. 225). Les exploits chi-psi sont disponibles à quiconque possède le trait Psi (p. 147), mais les exploist gamma-psi ne sont disponible qu'aux personnages disposant du trait Psi au niveau 2. Afin d'utiliser ces exploits, un async doit posséder les compétences Contrôle (p. 178), Assaut Psi  (p. 183), et/ou Divination (p. 184), en fonction de chaque exploit. 

\subsubsection{Jouer des Asyncs} 

Tout joueur qui choisit de jouer un async devrait garder l'origine de ses capacités en tête: infection par la souche Watts-MacLeod. Le personnage peuvent ne pas être conscient de cette origine, mais ils savent indubitablementqu'ils ont subit une sorte de transformation et qu'ils ont des talents que personne d'autres n'a. Si ils ne sont pas conscient de l'infection, ils ont probablement appris à garder leurs capacités secrètes à moins d'être ridiculisés, attaqués ou esacmoté dans une sorte de programme de tests. Apprendre la vérité sur leur nature peut même être le point de départ d'une campagne et/ou leur introduction à Firewall. Si ils connaissent la vérité les personnages doivent cependant vivre avec le fait qu'ils sont les victimes d'une nanopeste probablemenr répandue par les TITANs qui peut ou peut ne pas amener à des complications, des effest dsecondaires ou d'autres révélations inattendues dans le futur. Les maîtres de jeu et les joueurs devraient faire un effort pour explorer la nature de cette infection et la manière dont les personnages la perçoivent. Comme noté précédement, les asyncs sont souvent des personnes profondément changées. Les aspects invasifs et étarngers de leur capacités ne devraient pas leur échapper. Par exemple, un async pourrait concevoir ses talents psi comme une sorte d'éntité parasite, vivant de ses exploits psi, ou il pourrait sentir qu'utiliser ces pouvoir les mets en contact avec une sort de substrat fondamental étrange et terrifiant de l'univers. Sinon, ils pourraient se sentir comme si leur personnalité aurait été mélangée avec quelque chose de différent, quelque chose qui n'en ferait pas parti. Chaque async a tendance à voir leur situation différement, et aucun d'entre eux ne le perçoit de manière agréable. 

\subsection{Utiliser le Psi} 

Utiliser le psi - c.à.d. lancer un exploit particulier pour déclencher certains effetx - ne nécessite pas toujours un jet. La description de chaque exploit détaille comment le pouvoir est utilisé. 

\subsubsection{Psi Actif} 

Les exploist psi actifs doivent être "activés" pour être utilisés. Ces exploits nécessitent habituellement un test de compétence. Les tests des exploits qui ciblent un autre être conscient ou une forme de vie sont toujours des Tests Opposés, alors que les autres sont gérés commes des Tests de Succès. Le niveau de concentration requis pour utilsier ces exploits varient, et donc ils peuvent nécessiter une Action Rapide, Complète ou de Tâche. Les exploits actifs causent du drain (p. 223) à l'async. La plupart des exploits gamma-psi tombent dans cette catégorie. 

\subsubsection{Psi Passif} 

Les exploits psi passifs englobent les capacités qui sont considérés comme automatiquement active et subconscientes. Ils requièrent rarement une action pour être activé et ne nécessite ni effort ni drain de l'utilisateur du psi. Les exploits passifs ajoutent typiquement des bonus à diverses activités ou permettent l'accès à certaines capacités plutôt que de faire appel à un test de compétence. La plupart des exploits chi-psi tombent dans cette catégorie. 

\subsubsection{Portée du Psi} 

Les exploits disposent d'une Portée qui est soi Soi, Contact ou Proche. Soi: Ces exploits n'affectent que l'async. Contact: Les Exploits avec une portée Contact peuvent être utilisés contre d'autre formes de vie biologique, mais l'async doit établir un contact physique avec la cible. Si la cible évite le contact, cela requiert une attaque de mélée réussie, en appliquant le moidificateur contact-seulement de +20. Cette attaque ne cause aucun dommage, et est considérée comme faisant partie de la même action que l'utilisation de psi. Proche: Les exploits qui disposent d'une portée Proche impliquent une interaction avec les autres formes de vie biologique sur une courte distance. La distance optimale est inférieure à 5 mètres. Pour chaque mètre au-deloà de cette limite, appliquez un modificateur de -10 au test. Psi contre Psi: Enr aison de la nature du psi, les exploits sont plus efficace contre d'autres utilisateurs de psi. Les exploits avec une portée de Contact peuvent être utilisé à portée Proche contre un autre async. De la même manière, un exploit avec une portée Proche peut être utilisé à deux fois la distance normale (10 mètres) lorsqu'il est lancé sur un autre async. 

\subsubsection{Ciblage} 

Les synthmorpsh, les bots et les véhicules ne peuvent pas être ciblé par des exploits psis, car ils ne possèdent pas de cerveau biologique. Les pods - avec des cerveaux moitié biologique moitié numérique - sont moins succeptible d'être affectés par le psi et reçoivent un modificateur de +30 lorsqu'il se défendent contre une utilisation du psi. Notez que les infomorphs ne peuvent jamais être ciblées par des exploits psi car le psi n'est pas actifs dans le mesh ou dans les simulspaces. Cible multiples: Un async peut cibler plus d'un personnage avec un exploit avec la même action, tant que chacun d'entre eux peut être ciblé en respect des règles ci-dessus. Le personnage psi ne fait qu'un seul jet, et chacun des personnages en défense fait un Test Opposé contre ce jet. Le personnage psi souffre du drain (p. 223) pour chaque cible cependant, ce qui siginfie qu'utiliser le psi sur des cibles multiples est extrêmement dangereux. Animaux et autres formes de vie moins complexe: le psi fonctionne contre n'importe quelle créature vivante disposant d'un cerveau et/ou d'un système nerveux. Contre les animaux partiellement-conscient et partiellement-élevées, le psi souffre d'un modificateur de -20 et augemente le drain de +1. Contre des animaux non-conscient, le psi souffre d'un modificateur de -30 et augmente le drain de +3. Il n'a aucune effet sur ou contre les formes de vie moins complexe telles que les plantes, les algues, les bactéries, etc. Facteurs et Aliens: à la discrétion du maître de jeu, les exploits psis peuvent ne pas fonctionner du tout sur les créatures étrangères, en fonction de leur physiologie et de leur neurologie. Si cela fonctionne, l'usage de psi souffrira au moins d'un modificateur de -20 et d'une augmentation du drain de +1. 

\subsubsection{Tests Opposé} 

Le psi utilisé contre un autre personnage est résisté par un Test Opposé Les personnages se défendant résistent avec WOL x 2. Les personnages volontaires peuvent choisir de ne pas résister. Les personnages inconscients ou endormis ne peuvent pas résister. Si le personnage qui lance le psi réussit son test et que el défenseur échoue, l'exploit affecte sa cible. Si l'utilisateur de psi échoue, le défenseur n'est pas affecté. Si les deux parties réussissent leur tests, il faut comparer leurs résultats. Si l'utilisateur de psi obtient un meilleur résultat que le défenseur, l'exploit surpasse les blocage mentaux du défenseur et affectent leur cible; dans le cas contraire, l'explot échoue à affecter l'égo du défenseur. 

\subsubsection{Conscience de la Cible} 

La cible d'un exploit psi est consciente d'être ciblées dés qu'elles réussissent leur part du Test Opposé (que l'async obtienne un meilleur résultat ou pas). Notez que le fait d'avoir conscience d'être ciblé par un psi ne signifie pas nécessairement que la cible comprends que des capacités psi sont utilisées sur elle, surtout que la plupart des personnes dans Eclipse Phase ne sont aps au courant de l'existence du psi. Au lieu de ça, la cible est plus encline à comprendre qu'une influence extérieure quelconque est à l'œuvre, ou que quelque chose d'étrange est en train de se passer. Ils peuvent soupçonner avoir été drogués ou être sous l'influence d'une technologie étrange. Les cibles qui échouent leur jet restent inconsciente de la tentative de psi. 

\subsubsection{Défense Psi Totale} 

Comme la défense totale en combat physique (p. 198), un défenseur peut dépenser une Action Complexe pour se regrouper et concentrer ses défenses mentales, gagnant un modificateur de +30 à ses jets de défense contre l'utilisation de psi jusqu'à leur prochaine Phase d'Action. 

\subsubsection{Critiques} 

Si le défenseur obtient un succès critique, le personnage qui tente de l'affecter par le psi est temporairement bloqué à l'extérieur de l'esprit de la cible. L'utilisateur de psi ne peut plus cibler ce personnage avec des exploits jusqu'à ce qu'une période de "reset" appropriée se soit écoulée, déterminée par le maître de jeu. Si l'async obtient un échec critique, il souffre d'une incapacité temporaire suite aux dysfonctionnements particulièrement violents et stressants de leur esprit (voir Inconvénients du Psi p. 221). Si un utilisateur de psi obtient un succès critique contre un défenseur, ou si le défenseur obtiens un échec critique, doublez la puissance des effets de l'exploit. Dans le cadre d'attaque psi, la VD peut être doublée ou l'armure mentale peut-être ignorée. Alternativement, lors de l'utilisation d'Assaut Psi (p. 183), le personnage ciblé peut risquer de se faire infecter par le souche Watts-McLeod (p. 362). 

\subsubsection{Armure Mentale} 

L'exploit Bouclier Psi (p. 228) Procure une armurre mentale, une forme de renforcement neuronal contre les attaques basées sur le psi. Comme les armures physique, cette armure mentale réduit le total de dommage infligés par un assaut psi. 

\subsubsection{Durée} 

Les exploits psi possèdent l'une des quatres durées suivantes: permanent, instantané, temporaire ou maintenu. Permanent: Les exploits Permanents sont toujours actifs. Instantané: Les exploits Instatanés font effets pendant la Phase d'Action dans laquelle isl ont été utilisés uniquement. Temporaire: Les exploits Temporaires sont effectifs pendant une durée limitée sans effort supplémentaires de la part de l'async. La durée d'un effet Temporaire est déterminée par la VOL $\div$ 5 (arrondi au supérieur) de l'async et est mesurée soit en Tour d'Actions soit en minute, comme noté dans la description de l'exploit. Le drain pour l'exploit est appliqué immédiatement lorsqu'il est utilisé, pas à la fin de l'effet. Maintenus: Les exploits maintenus requièrent un effort de l'async pour le garder actif aussi longtemps que le désire l'async. Maintenir un exploit requiert de la concentration, et infligent donc à l'async un modificateur de -10 à toutes ses autres compétences tant que l'exploit est maintenu. L'async doit également rester à portée de sa cible, sinon l'exploit se termine immédiatement. Plus d'un exploit peut-être maintenu à la fois, avec un modificateur cumulatif. Le drain pour l'exploit est appliqué immédiatement lorsqu'il est utilisé, pas à la fin de l'effet. A la discrétion du maître de jeu, les exploits qui sont maintenus sur une longue période devraient infligés un drain supplémentaire. 

\subsubsection{Drain} 

L'utilisation de psi draine physiquement (et parfois psychologiquement)  un utilisateur psi. The phénomène est appellé le drain, et se manifeste comme de la fatigue, de l'épuisevement, de la douleur, de la surcharge neurale, du stress cardiovasculaire et de l'adynamie (perte de vigueur). Bien que la mort d'async provoquées par le drain soit un phénomène rare, l'utilisation de trop de psi actif peut mettre l'async en danger de mort dans certaines circonstances. En terme de jeu, chaque exploit actif possède une Valeur de Drain de 1d10 $\div$ 2 (arrondies au supérieur) VD. Chaque exploit actif possède une Valeur de Modificateur de Drain qui modifie ce total. Par exemple, un exploit avec une Valeur de Modificateur de Drain de -1 inflige (1d10 $\div$ 2) -1 DV. Si les points de dommages générés par le drain dépassent le Seuil de Blessure d'un personnage, ils peuvent infliger une blessure comme toute autre source de dommage (voir Blessures, p. 207). 

\begin{quotation} \textbf{Exemple} \\ Matric est en train d'enquêter sur une disparition, il décide donc d'utiliser son exploit Qualia pour booster son intuition pendant qu'il cherche des indices. Cet exploit chi-psi ne prend qu'une Action Rapdie pour se déclencher et ne nécessite pas de test. La VOL de matric est 25, donc la durée de cet exploit temporaire est de 5 Tours d'Action (25 $\div$ 5 = 5). Le modificateur de Drain de l'exploi est de -1, il doit donc encaisser (1d10 $\div$ 2) - 1 DV. Il obtient un 1 et il ne prend donc pas de drain. Un peu plus tard, Matric se retrouve prit dans une lutte à mort avec un kidnapper. Heureusement pour Matric, ils sont au corps à corps et il est donc suffisament proche pour tenter de toucher son adversaire. A sa Phase d'Action, il fait un Test de Combat à Mains nues avec un modificateur de +20 (pour une attaque de toucher uniquement) et réussit. Cela lui permet d'essayer d'utiliser son exploit Frappe Psychique Il lance sa compétence Assaut Psi de 57 contre la VOL x 2 de sa cible (32). Sa cible est incarnée dans un pod ouvrier qui est donc moins sensible au psi, il reçoit donc un modificateur de +30 (32 + 30 = 62). Matric obtient 32 et le pod ouvrier un 64 - Matric l'emporte! Pour détermeiner les dégats, il lance 1d10 + (VOL $\div$ 10). Sa VOL est 25, ce qui fait donc 1d10 + 3. Il obtient un 7 et inflige donc 10 (7 + 3° points de dommage à sa cible. Le pod ouvrier hurle de douleur, souffrant d'une blessure suite à l'assaut psychique. 

\end{quotation} 





\subsection{Exploits Chi-Psi} Les exploist chi-psi sont le capacités des aync qui accélère l'informatique cognitive (traitement interne de l'information) et améliore la perception et la cognition de l'utilisateur. 

\subsubsection{Précognition Ambiante} \textbf{Type:} Passif \\ \textbf{Action:} Automatique \\ \textbf{Portée:} Soi \\ \textbf{Durée:} Permanente \\ Cet exploit fournit à l'async une compréhension instinctive d'une zone et de toute menace potentielle environnantes. L'async reçoit un modificateur de +10 à tous les jets d'Investigation, de Perception, de Fouille et aux Test de Surprise. 

\subsubsection{Boost Cognitif} \textbf{Type:} Actif \\ \textbf{Action:} Rapide \\ \textbf{Portée:} Soi \\ \textbf{Durée:} Temp (Tour d'Actions) \\ \textbf{Mod. de Drain:} –1 \\ L'async peut élever temporairement ses performances cognitives. En terme de jeu, la Cognition est augmentée de 5 pour la durée déterminée. Ce boost à la Cognition augmente également  le niveau des compétences liées à cette aptitude. 

\subsubsection{Veille} \textbf{Type:} Actif \\ \textbf{Action:} Tâche (min. 4 heures) \\ \textbf{Portée:} Soi \\ \textbf{Durée:} Maintenue \\ \textbf{Mod. de Drain:} 0 \\ Cet exploit donne à l'async la possibilité d'envoyer un esprit dans un fugue de veille régénérative, pendant laquelle la psyche du personnage est réparée. L'async doit entrer en veille pour au moins 4 heures; chaque période de 4 heures de veille soigne 1 point de dégats de stress. Les traumas, dérangement et désordres mentaux ne sont pas affectés apr cet exploit. Dans le cadre de toute tentative de perception, l'async est cataonique durant la veille, complètement isolés du monde extérieur. Seul les plus grosses perturbations ou les chocs physiques (tels qu'être blessé ou touché par une arme à choc) sortira l'async de cet état. 

\subsubsection{Contrôle Émotionnel} \textbf{Type:} Passif \\ \textbf{Action:} Automatique \\ \textbf{Portée:} Soi \\ \textbf{Durée:} Permanente \\ Contrôle des Émotions donne à l'async un contrôle précis de son état émotionnel. Les émotions non désirées peuvent être bloquées et d'autres peuvent être adoptées. Ceci à pour bénéfice de protéger l'async des manipulatiosn émotionnelles, telles que l'exploit Déclencher une Émotion ou la compétence Intimidation. L'async reçoit un bonus de +30 lorsqu'il se défend contre de tels tests. 

\subsubsection{Créativité Améliorée} \textbf{Type:} Passif \\ \textbf{Action:} Automatique \\ \textbf{Portée:} Soi \\ \textbf{Durée:} Permanente \\ Un async avec Créativité Améliorée est plus imaginatif et plus enclin à sortir des cadres de pensées. Appliquez un modificateur de +20 à tous les tests dans lequel la créativité à un rôle majeur. Ce niveau d'ingéniosité peut parfois paraître étarnge et déifférent, se manifestant de manière bizarre et terrifiante, partciulièrement dans le cadre d'œuvres d'art. 

\subsubsection{Filtre} \textbf{Type:} Passif \\ \textbf{Action:} Automatique \\ \textbf{Portée:} Soi \\ \textbf{Durée:} Permanente \\ Filtre permet à l'async de filtrer les distractions et d'élliminer les modificateurs situationnels négatifs liés à la distraction, dans les limites déterminées par le maître de jeu. 

\subsubsection{Grok} \textbf{Type:} Actif \\ \textbf{Action:} Complexe \\ \textbf{Portée:} Soi \\ \textbf{Durée:} Instantanée \\ \textbf{Mod. de Drain:} –1 \\ En utilisant l'exploit Grok, l'async est capable de comprendre instinctivement comment n'importe quel objet, véhicule ou appareil non familier est utilisé simplement en le regardant et en le manipulant. Si le personnage réussit un test de COG x 2, il parvient à déterminer une utilisation basic de l'obejt, du véhicule ou de l'appareil, indépendamment de l'étrangeté ou de la bizzarerie du métériel en question. Cet exploit ne fournit pas une compréhension des principes technologiques mis en œuvre - l'utilisateur de psi saisit juste comment le faire fonctionner. Si un test est nécessaire, l'utilisateur de psi reçoit un modificateur de +20 pour utiliser l'appareil (ce bonus ne s'applique qu'au appareil non-familier et/ou aux tests pour lesquels le personnage se défausse - il ne s'applique pas au apapreils avec lequel le personnage est familier). 

\subsubsection{Haute Tolérance à la Douleur} \textbf{Type:} Passif \\ \textbf{Action:} Automatique \\ \textbf{Portée:} Soi \\ \textbf{Durée:} Permanente \\ Cet exploi autorise l'async à bloquer, ignorer ou d'une manière ou d'une autre à isoler la douleur. L'async réduit les modificateurs négatif liés aux blessures de 10. 

\subsubsection{Hyperthymesie} \textbf{Type:} Passif \\ \textbf{Action:} Automatique \\ \textbf{Portée:} Soi \\ \textbf{Durée:} Permanente \\ L'hyperthymésie donne à l'async une mémoire autobiographique supérieur, lui permettant de se rappeler des éveènements les plus triviaux. On peut demander à un async hyperthymésique de donner, pour une date passée choisie aléatoirement, le jour de la semaine, ce qu'il s'est passé ce jour là, quel temps faisait-il et plein d'autres détails insignifiant que la plupartd es gens nesont pas capable de retenir. 

\subsubsection{Instinct} \textbf{Type:} Passif \\ \textbf{Action:} Automatique \\ \textbf{Portée:} Soi \\ \textbf{Durée:} Permanente \\ Instinct améliore la capacité subconsciente d'un async a estimer une situation et à faire un jugement rapide qui est aussi précis qu'une décision prudente et réfléchie. Pour les Actiosn de Tâches qui n'impliquent que de l'analyse et de la plannification (typqiuemet, des actions liées aux compétences Mentales), l'async peut réduire l'intervalle de temps de 90\% sans subir de modificateur. Pour les Actions de Tâches qui n'impliquent que partiellement de l'analyse/planification, il peut réduire l'intervalle de temps de 30\% sans souffrir de pénalités. 

\subsubsection{Multitâche} \textbf{Type:} Passif \\ \textbf{Action:} Automatique \\ \textbf{Portée:} Soi \\ \textbf{Durée:} Permanente \\ L'async peut gérer un vaste total d'information sans surcharge et peut effectuer plus d'une action mental en même temps. Le personnage reçoit une Action Complexe supplémentaire à chaque Tour d'Action qui ne peut être utilisée que pour des actiosn mentales ou sur le mesh. 

\subsubsection{Reconnaissance de Motifs} \textbf{Type:} Passif \\ \textbf{Action:} Automatique \\ \textbf{Portée:} Soi \\ \textbf{Durée:} Permanente \\ Le personnage est un adepte de la détection de motifs et de la corrélation d'éléments non-aléatoire d'un magma d'information - les éléments liés lui saute aux yeux. Cette compétence est utiles pour la traduction, le cassage de code ou pour trouver des indices cachés dans une montagne de données. Le personnage doit avoir un échantillon suffisament large et assez de temps pour étudier, déterminés par le maître de jeu. Cela peut aller de quelques heures d'écoute d'un langage transhumain parlé à quelques jours d'investigation et d'études d'inscription laissées par des aliens depuis longtemps décédés juquu'à une semaine ou plus de recherche de code secret particulièrement long. Les langues peuvent être comprise par la lecture ou par l'écoute d'extrait parlés. Appliquez un modificateur de +20 à chaque Test approprié de Langue, Investigation, Recherche ou de cassage de code (notez que cela ne s'applique pas aux Tests d'Infosec fait par un logiciel pour déchiffrer un code) L'async peut également utiliser cette capacité pour apprendre plsu facilement de nouveaux Langages, réduisant de moitié le temps d'entraînement nécessaire. 

\subsubsection{Boost de Prédiction} \textbf{Type:} Passif \\ \textbf{Action:} Automatique \\ \textbf{Portée:} Soi \\ \textbf{Durée:} Permanente \\ La machine de probabilité Bayesienne caractéristique du cerveau d'un async est boostée par cet exploit, renforçant sa capacité à estimer et à prévoir l'issue d'évènements autour d'eux alors qu'ils se déroulent en temps réels et transforme ces prédictions en informations. En pratique, le personnage perçoit intuitivement les issues les plus probables. Cela donne au personnage un bonus de +10 a tout test de compétence qui impliquent de prévoir l'issue d'évènements. Cela renforce également le processus décisionnel de l'async en situation de combat en mettant en exergue la meilleure suite d'action à effectuer, et lui procure donc un modificateur de +10 aux test d'Initiative et de Défilement. 

\subsubsection{Qualia} \textbf{Type:} Actif \\ \textbf{Action:} Rapide \\ \textbf{Portée:} Soi \\ \textbf{Durée:} Temp (Tour d'Actions) \\ \textbf{Mod. de Drain:} –1 \\ L'async peut temporairement augmenter sa compréhension intuitive des choses. En terme de jeu, l'Intuition est augmentée de 5 pour la durée choisie. Ce boost à l'Intuition augmente également le niveau des compétences liées à cette aptitude. 

\subsubsection{Calcul Savant} \textbf{Type:} Passive \\ \textbf{Action:} Automatique \\ \textbf{Portée:} Soi \\ \textbf{Durée:} Permanente \\ Le personnage possède d'incroyable facilités avec les mathématiques intuitives. Il peut faire n'importe quoi du calcul exact des probabilités lors de paris à la prédiction précise de l'endroit où va atterir une feuille tombant d'un arbre en observant le paysage et les courants de vents locaux. Le personnage se spécialise en calculs impliquant l'activité de systèmes chaotiques complexes et peut donc calculer les réponses que même les ordianteurs les plus rapides ne peuvent calculer, incluant des choses comme les motifs de distribution de débris lors d'une explosion. Cependant, cette fcilité en maths et largement intuitive, le personnage ne connait pas l'équation qu'il résout, ils ne connaissent que la solution au problème. Cet exploit procure aussi un modificateur de +30 à tout test de compétences impliquant des maths (que le personnage calcule, pas un ordinateur). 

\subsubsection{Boost Sensoriel} \textbf{Type:} Actif \\ \textbf{Action:} Rapide \\ \textbf{Portée:} Soi \\ \textbf{Durée:} Temp (Tour d'Actions) \\ \textbf{Mod. de Drain:} –2 \\ Un async utilise cet exploit pour augmenter sa perception sensorielle naturelle ou augmentée (vision, audio, odeur, augmentée) en augmentant le traitement cérébral, accordant un modificateur de +20 sur les Tests de Perceptiosn basés sur les sens. 

\subsubsection{Kinésique Supérieure} \textbf{Type:} Passif \\ \textbf{Action:} Automatique \\ \textbf{Portée:} Soi \\ \textbf{Durée:} Permanente \\ L'async acquiert une plus grande perspicacité vis à vis des signaux émotionnels, des gestes, des expressions faciales et du langage corporel de ses interlocuteurs lorsqu'il faut deviner son état émotionnel, ses intentions ou ses réactions. Appliquez un modificateur de +10 aux Test de Compétence Kinésique. 

\subsubsection{Perception Temporelle} \textbf{Type:} Actif \\ \textbf{Action:} Automatique \\ \textbf{Portée:} Soi \\ \textbf{Durée:} Temp (Tour d'Actions) \\ \textbf{Mod. de Drain:} –1 \\ Un async avec cette capacité peut ralentir sa perception du temps, percevant ainsi son environnement comme si tout était en slow motion ou à une vitesse réduite. En terme de jeu, cet exploi tdonne à l'async une Vitesse de +1. Cette Phase d'Action supplémentaire ne peut cependant être dépensée que sur des actiosn mental ou sur le mesh. 

\subsubsection{Guide de l'Incosncient} \textbf{Type:} Actif \\ \textbf{Action:} Automatique \\ \textbf{Portée:} Soi \\ \textbf{Durée:} Temp (Tour d'Actions) \\ \textbf{Mod. de Drain:} 0 \\ Cet exploit autorise l'async à surcharger sa conscience et à laisser la part inconsciente de son esprit prendre el contrôle. Dans cet étât, l'esprit conscient de l'async est faiblement conscient de ce qui est transgresser, et tous les souvenirs de cette période seront, dans le meilleur des cas, flous. L'avantage et que la part insconsciente de l'esprit agit plus rapidement, et donc la Vitesse de l'async est boostée de +1. Le personnage reste éveillé et actif, mais il est incapable de communications complexe ou de toute autre actions mental et est plus motivé par ses instinct et envies primitives que dans son étât conscient. Bien qu'il soit recommandé que le joueur garde le contrôle de son personnage pendant l'utilisation de Guide de l'Inconscient, le maître de jeu devrait se sentir libre de diriger les actions du personnage comme il l'entend. 



\subsection{Exploit Gamma-Psi} Les exploits gamma psi traitent de contacter (ou lire et communiquer) et d'influencer les fonctions d'esprits biologiques (des egos dans une biomorph, mais incluant également les formes de vie animales). Les exploiits gamma psi ne sont disponibles qu'aux personnage qui disposent du trait Psi au Niveau 2. La plupart des utilisation du psi-gamma sont résolues par un Test Opposé entre l'async et la cible (p. 222). 

\subsubsection{Aliénation} \textbf{Type:} Actif \\ \textbf{Action:} Complexe \\ \textbf{Portée:} Contact \\ \textbf{Durée:} Temp (Tour d'Actions) \\ \textbf{Mod. de Drain:} 0 \\ \textbf{Compétence:} Assaut Psi \\ Aliénation est un exploit offensif qui cré un sentiment de déconnexion entre un ego et sa morph - similaire à celui vécu lors d'une réincarnation dans un nouveau corps. L'égo trouve son corps encombrant, bizarre et étranger, presque comme si il était emprisonné dans ce corps. Si l'async bat la cible dans un Test Opposé, considérez le test comme un Test d'Intégration raté (p. 272) pour la cible. Cet effet persiste pour la durée de l'exploit 

\subsubsection{Charisme} \textbf{Type:} Actif \\ \textbf{Action:} Rapide \\ \textbf{Portée:} Contact \\ \textbf{Durée:} Temp (Minute) \\ \textbf{Mod. de Drain:} –1 \\ \textbf{Compétence:} Contrôle \\ L'async utilise cet exploit pour influencer l'esprit de la cible à un niveau subconscient, de manière à ce qu'elle le perçoive comme étant charismatique, magnétique et persuasif. Si l'async bat la cible dans un Test Opposé, il gagne un modificateur de +30 sur tous les Tests de Compétences Sociales suivants pendant la durée choisie. 

\subsubsection{Mamoire Brumeuse} \textbf{Type:} Actif \\ \textbf{Action:} Complexe \\ \textbf{Portée:} Contact \\ \textbf{Durée:} Temp (Minute) \\ \textbf{Mod. de Drain:} –1 \\ \textbf{Compétence:} Contrôle \\ Mémoire Brumeuse permet à l'async d'interrompre temporairement la capacité de la cible a former une mémoire à long terme. Si l'async gagne un Test Opposé, la capacité de mémmorisation de la cible est annulée pendant la durée de l'exploit. La cible retiendra les souvenirs à court-terme pendant cette période de temps, mais elle oubliera rapidement tout ce qui s'est produit pendant que l'exploit était actif. 

\subsubsection{Analyse en Profondeur} \textbf{Type:} Actif \\ \textbf{Action:} Complexe \\ \textbf{Portée:} Contact \\ \textbf{Durée:} Maintenue \\ \textbf{Mod. de Drain:} +1 \\ \textbf{Compétence:} Divination \\ Scan en Profondeur est une version plus intrusive de Parcours des Pensées (p. 228), fait pour extraire des information de l'individu ciblé. Si le Test Opposé réussit, l'async envahi télépathiquement l'esprit de la cible et peut le sonder à la recherche d'information. Pour chaque tranche coimplète de 10 points de MdS que l'async obtient à son tets, il trouve un morceau d'information. Chaque élément prend une Action Complexe a récupérer, pendant laquelle l'exploit doit être maintenu. La cible est consciente de cette sonde mentale, bien qu'elle ne sache pas quelle information l'async a récupéré. 

\subsubsection{Déclencher une Emotion} \textbf{Type:} Actif \\ \textbf{Action:} Complexe \\ \textbf{Portée:} Contact \\ \textbf{Durée:} Temp (Tour d'Actions) \\ \textbf{Mod. de Drain:} –1 \\ \textbf{Compétence:} Contrôle \\ Cet exploit autorise l'async a stimuler les zones corticales liées au émotions du cerveau de l'a cible. Cela permet à l'async d'induire, d'amplifier ou d'atténuer des émotions spécifiques, manipulant ainsi la cible. Si l'async bat la cible dans un Test Opposé, elle agira en accord avec l'émotion pour la durée de l'exploit et sous certaines circonstances elle pourrait subir certaines pénalités (jusqu'à +/-30), déterminées par le amître de jeu. Par exemple, un async pourrait recevoir un bonus de +30 à un Test d'Intimidation contre une cible emplie de peur. 

\subsubsection{Perception Egotique} \textbf{Type:} Actif \\ \textbf{Action:} Complexe \\ \textbf{Portée:} Proche \\ \textbf{Durée:} Temp (Tour d'Actions) \\ \textbf{Mod. de Drain:} –1 \\ \textbf{Compétence:} Divination \\ Perception Egotique peut être utilisé pour détecter la présence et la localisation d'autre  forme de vie consciente et biologique (c.à.d, des égos) à portée de l'async. Pour détecter ces forme de vie, l'async fait un simple Test de Divination, opposé par chaque forme de vie à portée. L'async peu souffrir d'un modificateur pour détecter de petits animaux et insectes, similaire aux modificateurs appliqués pour les cibler en combat ) distance (voir p. 193); de la même manière, un modificateur pour détecter les formes de vie les plus grosses peut aussi être appliqué. Si le test est réussi, l'async a détecté que la forme de vie est proche. Chaque tranche de 10 points de MdS donnera une information supplémentaire en lien avec la forme de vie détectée: direction vis à vis de l'async, taille approximative, type de créatuer, distance de l'async, etc. L'asycn saura si la cible bouge, tant qu'elle el fait pendant la durée de l'exploit. 

\subsubsection{Analyse Empathique} \textbf{Type:} Actif \\ \textbf{Action:} Rapide \\ \textbf{Portée:} Proche \\ \textbf{Durée:} Maintenue \\ \textbf{Mod. de Drain:} –2 \\ \textbf{Compétence:} Divination \\ Analyse Empathique donne à l'async la possibilité de sentir les émotiosn basique de la cible. Si l'async emporte le test Opposé, ils ressentent intuitivement l'état émotionnel courant de la cible tant que l'exploit est maintenu. A la discrétion du maître de jeu, cette connaissance peut donner un modificateur (jusqu'à +30) pour certains tests de compétences Sociaux. 

\subsubsection{Implantation Mémorielle} \textbf{Type:} Actif \\ \textbf{Action:} Complexe \\ \textbf{Portée:} Contact \\ \textbf{Durée:} Instantanée \\ \textbf{Mod. de Drain:} 0 \\ \textbf{Compétence:} Contrôle \\ Un async utilisant cet exploit peu implanter un souvenir d'une durée inférieur à une heure à l'intérieur de l'esprit de la cible. Ce souvenir n'appartient pas à la cible, et ce de manière évidente - elle ne peut pas le confondre avec l'un des siens. L'intention n'est pas de faire des faux souvenirs, mais de place l'un des souvenirs de l'async dans l'esprit de la cible pour que la cible puisse y accéder comme à n'importe quel autre souvenir. Cela peut-être utile pour "archiver" des données importantes avec un allié, pour littérallement fournir un point de vue différent ou simplement pour faire un "dump de donnée"  que la cible pourra lire attentivement. Implantation Mémorielle requiert un Test Opposé contre un participant contraint. Au chois du maître de jeu, des souvenirs particulièrement traumatisant pourrait infliger du stress mental au récipiendaire (p. 215). 

\subsubsection{Implantation de Compétences} \textbf{Type:} Actif \\ \textbf{Action:} Complexe \\ \textbf{Portée:} Contact \\ \textbf{Durée:} Temp (Tour d'Actions) \\ \textbf{Mod. de Drain:} 0 \\ \textbf{Compétence:} Contrôle \\ De manière similaire à Implantation Mémorielle, cet exploit permet à l'async de transmettre une partie de son expertise dans l'esprit de sa cible. Pour la durée de l'exploit, la cible bénéficie de cette expertise lorsqu'elle utilise cette compétence. Si la compétence de l'async est comprise entre 31 et 60, la cible reçoit un bonus de +10. Si la compétence de l'async est supérieur à 61, la cible reçoit un bonus de +20. Implantation de Compétences nécessite un test Opposé contre les participants contraints. Il est arrivé que, dans certains cas, la cible utilise la compétence avec le style et les mannières de l'async. 

\subsubsection{Imitation} \textbf{Type:} Actif \\ \textbf{Action:} Rapide \\ \textbf{Portée:} Proche \\ \textbf{Durée:} Instantanée \\ \textbf{Mod. de Drain:} 0 \\ \textbf{Compétence:} Divination \\ Dans un cadre où changer de corps et de visage n'est pas unusuel, les gens apprennent à reconnaître les habitudes et la bizarrerie dans la personnalité un peu plus souvent. L'async doit utiliser cet exploit sur une cible et réussir un Test de Réussite. En cas de réussite, l'async acquiert une "empreinte" de l'esprit de la cible qu'il peut utiliser en imitant cet égo. L'async reçoit un bonus de +30 sur les Tests d'Imposture lorsqu'il imite le comportement et les signaux sociaux de la cible. 

\subsubsection{Lien Mental} \textbf{Type:} Actif \\ \textbf{Action:} Rapide \\ \textbf{Portée:} Contact \\ \textbf{Durée:} Maintenue \\ \textbf{Mod. de Drain:} +1/cible après la première \\ \textbf{Compétence:} Contrôle \\ Lien Mental permet une communication mentale à double sens avec une cible. Cela peut-être utilisé sur plsu d'une cible simultanément, auquel cas l'async peut agir comme un "server" télépathiqe, de façon à ce que quiconque lié mentalement avec l'async puisse aussi communiquer mentalement avec les autres (via l'async cependant, donc ils entendent la conversation). Le langage est toujours un facteur dans les conversataion par lien mental, mais cette barrière peut-être franchie en transmettant des sons, des images, des émotions et d'autres sensations. Lien Mental nécessite un Test Opposé contre els aprticipants contraints. 

\subsubsection{Omniscience} \textbf{Type:} Actif \\ \textbf{Action:} Rapide \\ \textbf{Portée:} Proche \\ \textbf{Durée:} Temp (Minute) \\ \textbf{Mod. de Drain:} –1 \\ \textbf{Compétence:} Divination \\ Un async avec Omniscience est hypersensible aux autres forme de vie biologique qui l'observent. Pendant toute la durée de l'exploit, l'async fait un Test de Divination qui est opposé par toute forme de vie qui concentre son attention sur lui et qui est à portée de l'exploit; si le test est réussi, l'async sait qu'il est observé, mais pas par qui ou quoi. Il peut, cependant, recevoir un bonus de +30 à la Perception pour détecter les observateurs. Cet exploit n'enregistre pas les observateurs non attentifs ou partiels, ni la perception simple de l'async. Il ne fait que noter les cibles qui observent activement (même si elles dissimulent leur observation) Cet exploit est efficace aussi bien pour percevoir un petit, ou pour trouver des ami(e)s potentiels dans un bar. 

\subsubsection{Pénétration} \textbf{Type:} Actif \\ \textbf{Action:} Rapide \\ \textbf{Portée:} Contact \\ \textbf{Durée:} Instantanée \\ \textbf{Mod. de Drain:} 1 par point de PA \\ \textbf{Compétence:} Assaut Psi \\ Pénétration est un exploit qui fonctionne en conjonction avec n'importe quel exploi offensif qui implique la compétence Assaut Psi. Il permet à l'async de pénétrer le Bouclier Psi d'un adversaire en concentrant son attaque psi. Chaque point de Pénétration d'Armure appliquée à l'attaque psi inflige un point de drain. La PA maximum qui peut-être appliquée est égale à la compétence Assaut Psi de l'async divisée par dix (arrondies à l'inférieur). 

\subsubsection{Boulcier Psi} \textbf{Type:} Passif \\ \textbf{Action:} Automatique \\ \textbf{Portée:} Soi \\ \textbf{Durée:} Permanente \\ Bouclier Psi renforce l'esprit de l'async contre les attaques et les manipulations psi. Si l'async est touché par une attaque psi, il reçoit un nombre de points d'armure égal à VOL  $\div$ 5 (arrondis au supérieur), réduisant le total de dommage infligé. Ils reçoivent également un modificateur de +10 lorsqu'il résistent à un autre exploit psi. 

\subsubsection{Frappe Psyhcique} \textbf{Type:} Actif \\ \textbf{Action:} Complexe \\ \textbf{Portée:} Contact \\ \textbf{Durée:} Instantanée \\ \textbf{Mod. de Drain:} 0 \\ \textbf{Compétence:} Assaut Psi \\ Frappe Psychique est un exploit offensif qui cherche à infliger des dommages physique sur le cerveau et le système nerveux de la cible. Each successful attack inflicts 1d10 + (WIL $\div$ 10, round up) damage. Augmentez les dommage de +5 si un Succès Excellent est obtenu. 

\subsubsection{Brouillage} \textbf{Psi :} Passif \\ \textbf{Action:} Automatique \\ \textbf{Portée:} Soi \\ \textbf{Durée:} Permanente \\ Brouillage permet à l'async utilisant l'exploit de se cacher d'un autre async utilisant les exploits Perception Egotique ou Omniscience. Appliquez un modificateur de +30 au test de défense de l'async lors du Test Opposé. 

\subsubsection{Bloquage Sensoriel} \textbf{Type:} Actif \\ \textbf{Action:} Complexe \\ \textbf{Portée:} Contact \\ \textbf{Durée:} Temp (Tour d'Actions) \\ \textbf{Mod. de Drain:} –1 \\ \textbf{Compétence:} Assaut Psi \\ Blocage Sensoriel désactive et court-circuite l'un des cortex sensoriel de la cible (choisi par l'async), ineterférant voire annulant une source spécifique d'entrée sensorielle pour la durée choisie. Si l'async bat la cible lors du Test Opposé, la cible souffre d'un modificateur de -30 à ses Tests de Perception avec ce sens (ce malus est doublé à -60 si l'async obtient un Succès Excellent). 



\subsubsection{Spam} \textbf{Type:} Actif \\ \textbf{Action:} Complexe \\ \textbf{Portée:} Contact \\ \textbf{Durée:} Temp (Tour d'Actions) \\ \textbf{Mod. de Drain:} 0 \\ \textbf{Compétence:} Assaut Psi \\ L'exploit permet à l'async de surcharger et de submerger l'un des cortex sensoriel (choisit par l'async), le noyant avec des entrées sensorielles confusante et déconcentrabte et donc les attaquants directement. Si l'async gagne le test Opposé, la cible souffre d'un modoficateur de -10 à to ses tests pendant la durée de l'exploit -doublé à -20 si l'async obtient un Succès Excellent). 

\subsubsection{Statique} \textbf{Type:} Actif \\ \textbf{Action:} Complexe \\ \textbf{Portée:} Proche \\ \textbf{Durée:} Maintenue \\ \textbf{Mod. de Drain:} 0 \\ \textbf{Compétence:} Aucune\\ L'async génère un champ de saturation anti-psi, entravant toute utilisation d'un exploit distant à portée de l'async. Tout ces exploits distants souffrent d'un modificateur de -30. Cet exploit n'a pas d'effet sur les exploits ayant une portée Soi ou Contact. 

\subsubsection{Subliminal} \textbf{Type:} Actif \\ \textbf{Action:} Complexe \\ \textbf{Portée:} Contact \\ \textbf{Durée:} Instantanée \\ \textbf{Mod. de Drain:} +2 \\ \textbf{Compétence:} Contrôle \\ L'exploit Subliminal permet à l'async d'influencer le train de pensée d'une autre personne en implémentant une seule suggestion post-hypnotique dans l'esprit de la cible. Si l'async gagne un Test Opposé, le récipiendaire va soutenir cette idée comem si il s'agissait de la leur. Les suggestins implantés doivent être courte et simple; le maître de jeu ne devrait autoriser que les suggestions tenant en une phrase simple (par exemple: "Ouvre le sas," ou "Donnes moi ton arme"). À la discrétion du maître de jey, la cible peut recevoir un  bonus pour résister aux suggestions qui menancent immédiatement leur vie ("sautes du pont") ou qui viole leur motivations ou leur éthique personnelle. Les suggestions n 'ont pas besoin d'être déclenchée immédiatement, elles peuvent  être implantée avec une rapide condition déclenchante ("quand l'alarme sonne, ignores-là"). 

\subsubsection{Parcours des Pensée} \textbf{Type:} Actif \\ \textbf{Action:} Complexe \\ \textbf{Portée:} Contact \\ \textbf{Durée:} Maintenue \\ \textbf{Mod. de Drain:} –1 \\ \textbf{Compétence:} Divination \\ Analyse des Pensées est une version moins intrusive de lecture d'esprit qui scanne les pensées de surface de la cible pour certains "mots-clés" comme un mot particulier, une phrase, un son ou une image au choix de l'async. Plutôt que de creuser à travers l'ego comme avec l'exploit Analyse en Profondeur, Analyse des Pensées vérifie plutôt si une cible a une personne, un lieu, un évènement ou une chose particulière en tête, ce qui peut être utilisé par un investigatoeur avec un peu de jugeotte pour tirer des conclusions sans avoir besoin d'envahir directement un esprit. Analyse des Pensées peut-être maintenu, permettant à l'async de continuer à scanner les pensées de sa cible au fil du temps. L'async doit battre la cible dans un Test Opposé pour chaque élément scanné. 

\section{Psychochirurgie} Étant donné les progrés réalisé par la neuroscience dans le futur d'Eclipse Phase, il est facile de concevoir que l'esprit comme un logiciel programmable, comme quelque chose qui peut être rétro-ingénieré, recodé, mis à jour et patché. C'est vrai pour la majeure partie. Assisté par la nanotechnologie, la génétique et les sciences cognitives, les neuroscientifiques ont abbatue bon nombre de barrières à la compréhension de la structure et des fonctions de l'esprit, franchissant même de grands pas dans la découverte de la véritable nature de la conscience. Les bidouilles génétiques, les neuro-mods et les implants neuronaux offrent un assortiment d'options pour améliorer les capacités du cerveau. L'esprit trnshumain est devenue un terrain de jeu - et un champ de bataille. Les nanovirus libérés pendant la Chute ont infectés des millions de personne, altérant elur cerveau de manière permanente, avec des intrusions occasionelle se produisant une décenie après. Les virus cognitive errent dans le mesh, infectant les informorphs et les IAs, reprogammant leurs états mentaux. Une "idée infectieuse" a maintenant un sens propre. En vrai, l'édition de l'esprit n'est pas un processus simple, sécurisé et sans erreur - c'est quelque chose de difficile, dangereux et souvent défectueux. La neuroscience peut être des années lumières en avance par rapport au siècle dernier, mais il y a de nombreux aspects du cerveau et des fonctiones neuronales qui continuent de perturber et de se soustraire au plus brillants experts et IAs. Des technologies comme la cartographie nanoneuronale, l'upload, l'émulation digitale d'esprit et l'intelligence articificielle en sont encore à leur début, n'ayant qu'une décennie d'ancienneté. Bien que la transhumanité ait compris la manière dont ces processus fonctionnent, elle ne comprend toujours pas complètement les mécanismes sous-jacents. N'importe quel neurotechnicien vous dira que bidouiller dans les profondeurs boueuses de l'esprit est un véritable bordel. Les cerveau sont des appareils organiques, façonnés par des millions d'années de dévelopement évolutionnaire non planifié. Chacun d'eux s'est développé par pur hasard, rempli de reste d'évolutions et modifié aléatoirement par un nombre illimité d'évènement d'une vie et de facteurs environnementaux. Chaque esprit possède de nombreux mécanismes - cellulle, connexion, récepteurs - qui manipulent un nombre vertigineux de fonctions: mémorisation, perception, apprentissage, raisonnement, émotions, instincts, conscience et tant d'autres. Son ordonnanceur et son système de stockage est holonome, diffus et désorganisé. Même les cerveaux génétiquement modifiés et amélioré des transhumains sont des endroits bondés, chaotique et maillés et chaque esprit stocke ses propres souvenirs, personnalité et d'autres caractéristiques définissantes de manière unique. Ce que cela signifie c'est que même si l'architecture générale et la topographies des réseaux de neurones peut être scannées et déuite, le diable demeure dans les détails. Des techniques utilisées pour modifier, réparer ou améliorer l'esprit d'une personne ne garantissent aps le succès lorsqu'elles sont appliquées à un autre cerveau. Par exemple, le processus par lequel le cerveau stocke la connaissance, les compétences et les souvenirs résulte en un étrange processuss de chainage dans lequel ces souvenirs sont liés et associés avec d'autres souvenirs, et donc alétrer un souvenir peut avoir des effets de bord sur d'autres. Au final, les esprits posent des problèmes glissants et épineux, et les tentatives de les modeler ne se passent que rarement comme prévu. 

\subsection{Le processuss de la Psychochirurgie} La psychochirurgie est le l'ensemble des alétrations sélective et chirurgique apportée à un esprit transhumain. C'est un domaine séparé des modificatiosn neuronales par la génétique (qui modifient le code génétique), les implantatiosn de neuroware (qui ajoutent des insert cybernétiques ou biotechnologique au cerveau ou au système nerveux) ou le hacking de cerveau (qui sont des attaques logicielle sur des cerveaux informatisés, des inserts neuraux et des infomorph) bien qu'ils soient aprfois combinés. La Psychochirurgie est toujours effectuée sur un état mental numérique, qu'ils s'agisse d'émulation en temps-réel, d'une sauvegarde ou d'un fork. Dans la plupart des cas, l'état mental du sujet est copé via la même technologie et le même procédé que par l'upload ou le fork, et s'effectue dans un simulspace. Le sujet n'as pas besoin d'être volontaire, auquel cas les permissions du sujet sont restreintes. De nombreux environnements de psychochirurgie en simulspace sont disponibles, tous conçus spécifiquement pour faciliter un but de psychochirurgie particulier et programmés avec une ample sélection d'options de taritements de psychothérapie. Le process complet de psychochirurgie se divise en différentes étapes. La première est le diagnostique, qui peut impliquer l'utilisation de différentes techniques d'imagerie neurale sur les personnages morphé, la cartographie des connexions synaptiques et la construction d'un modèle neurochimique. Elle peut aussi inclure un profil psychologique complet et des tests de comportements psychomoteurs, des tests de personnalités et des scénarios de simulations en simulspace inclus. Les états mentaux numérisés peuvent être comparés à des enregistrements de personnes ayant des symptômes similaires afin d'identifier des groupes de renseignements connexes. Cette analyse est utilisée pour planifier la procédure. L'implémentation effective des modifications psychochirurgicale impliquent souvent plusieurs méthodes, en fonction du résultat attendus. L'application de modules externes sur l'état mental est souvent la meilleure approche car elle ne nécessite pas de bidouiller des connexions complexes et les nouvelles entrées sont rapidement interprétées et assimilées. Dans le cadre de traitements, des patchs de santé mentale logiciels compilés depuis des bases de données d'esprits sains sont échantillonés, adaptés et appliqués. Des programmes spécialisés peuvent être lancés pour simuler certains processus mentaux à but thérapeutique. Avant qu'une modification soit appliquée, elle est générallement testées sur un fork du sujet et lancé en vitesse accéléré pour en déterminé l'issue. De la même manière, de multiples choix de traitements peuvent être appliqués à des forks temporellement accélérés de cette manière, permettant au psychochirurgien de tester ce qui fonctionneras le mieux. Bien entendu, toutes les opérations de psychochirurgie ne se font pas forcément au béénfice du sujet. La psychochirurgie peut être utilisée pour interroger ou torturer des prisonniers, effacer des souvenirs, modifier un comportement ou infliger des faiblesses handicapantes. Elle est également aprfois utilisé dans un but de punition légale, dans une tentative de ralentir les activités criminelles. Cela va sans dire mais ces méthodes sont générallement appliqéue en force plutôt qu'ajustée en finesse, ignorant les apramètres de sécurité et ayant parfois des effets secondaires préjudiciables. 



\begin{quotation} \textbf{Recherche dans Solarchive : "Le Projet du Cognome Humain" } \\ Le Projet du Cognome Humain était un projet de recherches académiques ayant pour but de rétroingénierer le cerveau humain, ayant de nombreux parrallèlle avec le Projet du Génome Humain et ses succès dans le déhiffrage du génome humain. Le PCH était une entreprise multidisciplinaire, relevant de la biologie, des neurosciences, de la psychologie, des sciences cognitives, de l'intelligence artificelle et de la philosophie de l'esprit. Financé et supporté par des scientifiques et des entrepreneurs corporatistes ainsi que par les premiers groupusculkes transhumains, le PCH développa les bases de la numérisation d'un ego et joua un rôle majeur dans le développement de l'élévation de l'intelligence et des capacités cérébrales. Le PCH a également été instrumentalisé pour cataloguer les esprits transhumain et développer des bases de données de "patchs cognitifs" basés sur les états mentaux d'individus sains pour traiter les maladies et dommages mentaux. Bien que la plupart des données du PCH soient mise à disposition du public, quelques argonautes prétendent que certaines données son retenues en otage par certaines hypercorp, potentiellement pour développer des technologies d'altération cognitoves propriétaires. Après la Chute, les restes de ce projet ont été acquis par le Consortium Planétaire. \end{quotation} 

\subsection{Mécanismes de Psychochirurgie} En terme de jeu, la psychochirurgie est une Action de Tâche nécessitant un Test Opposé. Le psychochirurgien fait un jet de sa compétence Psychochirurgie conte la VOL x 3 de la cible. Appliquez les modificateurs appropriées, listés dans la table des Modificateurs de Psychochirurgie. Si le psychochirurgien réussit et que le sujet rate, la psychochirurgie est fonctionnelle et permanente. L'altération devient une part permanente de l'ego du sujet, et sera copié lorsqu'il s'uploadera (et parfois en forkant). Si les deux parties réussissent mais que le spychochirurgien obtient le plus haut résultat, la psychochirurgie est fonctionnelle mais temporaire. Elle persiste pour une semaine par tranche de 10 points de MdS. Si le sujet obtiens un meilleur résultat ou si le psychochirurgien rate son jet, la tentative ne fonctionne pas. L'intervalle de temps listé pour les procédures de psychochirurgie dépend du point de vue subjectif du patient. Étant donné que la plupart des sujets sont traités en simulspace, l'accélération temporelle peut réduire drastiquement le total de temps réel qu'une telle procédure requiert (voir Défier les Lois de la Nature, pp. 240–241). 



\subsubsection{Stress Mental} La psychochirurgie est une modification de l'esprit transhumain et parfois à la personne réisdant dans cet esprit. Il n'est pas surprenant que la psychochirurgie place donc du stress sur l'état mental du sujet infligeant même parfois des traumas. Chaque option de psychochirurgie liste une Valeur de Stress (VS) qui est infligée au sujet indépendament de la réussite ou de l'échec du test. Si le psychochirurgien obtient un Succès Excellent (MdS 30+), ce stress est réduit de moitié (arrondi à l'inférieur). Si le psychochirurgien obtient un Echec Dramatique (MdE 30+), le stress est doublé. Alternativement, un Echec Dramatique pourrait amener à des effets secondaires inattendus, tels que modifier d'autre comportement, émotions ou souvenirs. Si un succès critique est obtenu, aucun stress n'est appliqué. Cependant, si un échec critique est obtenu un trauma est appliqué automatiquement en plus du stress normal. Certaines circonstances peuvent aussi affecter le SV de la psychochirurgie, telles que notées dans la table des Modificateurs de Psychochirurgie. 



\subsection{Interpréter les Editions Cognititives} Beaucoup de changement provoqués par la psychochirurgie sont nébuleuses et difficiles à cerner par des mécanismes de jeu. Les alterations apportées à la personnalités d'un personnage et à son état mental sont de toutes façon souvent mieux gérés par des facteurs d'interprétation. Cela signifie que les joueurs devraient faire un réel effort pour intégrer de telles modifications mentales dans les faits et gestes de leur personnage et les maîtres de jeu devraient s'assurer que la représentation du personnage sonne juste par rapport à leurs éditions cognitives. Certaines modifications psychochirurgicale peuvent être reflétées par des traits d'ego alors que d'autres devrait ajouter des modificateurs à certains test ou dans certaines situations. Le maître de jeu devrait précuationneusement estimé les effets de l'altération d'un cerveau et appliquer les modificateurs comme ils le pensent. 



\subsection{Procédures Psychochirurgicale} Les altératiosn suivantes peuvent être obtenues avec la psychochirurgie. À la discrétion du maœtre de jeu, d'autres proécdures d'éditions cognitives peuvent être tentée, en utilisant celles-ci comme guide. 



\begin{table} \begin{tabular}{|l|r|r|} \hline

\multicolumn{3}{|c|}{\textbf{Modificateurs de Psychochirurgie}} \\ \hline

Situation &Modificateur au Test de Psychochirurgie &Modificateur de VS\\ \hline

Diagnostiques Préparatoires Incorrects &–30 &+1 \\ \hline

Protocoles de Sécurité ignorés &+20 &x2 \\ \hline

Accélération Temporelle du Simulspace &–20 &+2 \\ \hline

Le sujet est une IA, une IAG ou un élevé &–20 &+1 \\ \hline

\end{tabular} \label{tab:psychosurgery-modifiers} \end{table} 



\subsubsection{Contrôle Comportemental} \textbf{Intervalle:} 1 sem. \\ \textbf{MP:} Limitation/Amplification –10; Bloquage/Incitation –20, Effacement/Forçage–30 \\ \textbf{SV:} (1d10 $\div$ 2, round up) \\ Communément utilisé pour réhabiliter les criminels, le contrôle comportemental tente de limiter, bloquer ou effacer un comportement spécifique de la psyché du sujet. Par exemple, un meurtrier pourrait être conditionné contre les actes d'agression et un kleptomane pourrait subir une restriction pour le vol. Certaines personnes cherche cet ajustement de manière volontaire, tel que les personnalités célèbres qui restreignent leurs désirs de manger ou les accrocs qui coupent leur envie d'un fix. Le contrôle comportemental peu aussi être appliqué comme un débridage ou un renforcement. Un compagnon désirant éliminer ses inhibitions sexuelles, par exemple, ou un exécutif hypercop améliorant son dévouement pour placer le travail au-dessus du reste. Un personnage se sentira simplement contraint d'éviter un comportement qui est limité (en étant sanctionnant d'un modificateur de -10 par exemple), mais il trouvera plutôt difficile de suivre un comportement qui est bloqué (nécessitant un test de VOL x 3 et souffrant d'un modificateur de -20). They will find themselves completely incapable of initiating a behavior that is expunged, and if forced into the behavior will suffer a –30 modifier and (1d10 $\div$ 2, round up) points of mental Stress. De la même manière, un personnage se sentira contarint de suivre un comportement qui a été amplifié et trouvera difficile d'éviter de suivre un comporter qui est encouragé (nécessitant un test de VOL x 3 pour l'éviter). They will have no choice but to engage in enforced behaviors, and will suffer (1d10 $\div$ 2, round up) points of mental Stress if prevented from doing so. 

\subsubsection{Camouflage Comportemental} \textbf{Intervalle:} 1 sem. \\ \textbf{MP:} –20 \\ \textbf{SV:} 1d10 $\div$ 2, round up \\ Étant donné la possibilité de changer de corps, beaucoup d'agence de sécurité et de forces de l'ordre ont recours à l'analyse de personnalité et au profilage comportemental comem moyen d'identifier les personnes, même lorsqu'elles se réincarnent. Bien que ces systèmes soient loin d'être parfait, les habitudes incosncientes et les tcis de quelqu'un peuvent potentiellement le démasquer. Les personnages qui cherchent a éviter l'identification de cette manière peuvent faire appel au camouflage comportemental, qui cerche à atteindre et à changer les habitudes incosncientes et les marqueurs sociaux. Appliquez un modificateur de +30 en défense contre ce type d'identification et au Tests de Kinésique. 

\subsubsection{Deep Learning} \textbf{Timeframe:} Skill Learning Time $\div$ 2 \\ \textbf{MP:} +20 \\ \textbf{VS:} 1 \\ En utilisant des programme de tutoriaux, des protocoles de renforcements mémoriels, des tâches de conditionnements et des stimulatiosn cérébrales en profondeur, les capacités d'apprentissage du sujet sont renforcées, lui permettant d'apprendre de nouvelles compétences plus rapidement. 

\subsubsection{Contrôle Émotionnel} \textbf{Intervalle:} 1 sem. \\ \textbf{MP:} Limitation/Amplification –10; Bloquage/Incitation –20, Effacement/Forçage–30 \\ \textbf{SV:} (1d10 $\div$ 2, round up) + 2 \\ De manière similaire au contrôle comportemental, le contrôle émotionnel cherche à modifier, améliorer eou restreindre les réponses émotionnelles du sujet. Certains choisissent cette modification de manière volontaires comem le limitage de la tristesse pour être plus heureux ou l'encouagement de l'agressivité pour être plsu compétitif. Les mercenaires et les soldats sont connus pour supprimer la peur. Suivez les mêmes règles que celles-données pour le COntrôle Comportemental. 



\subsubsection{Interrogatoire} \textbf{Intervalle:} Variable (à la discrétion du maître de jeu; 1 semaine par défaut) \\ \textbf{MP:} +20 \\ \textbf{VS:} 1d10 \\ La psychochirurgie peut-être utilisée dans le but de mener un interrogatoire via l'application de torture et de manipulation mentale. Un test réussi de Psychochirurgie apporte un modificateur de +30 au Testd 'Initimidation pour l'interrogatoire. 

\subsubsection{Edition Mémorielle} \textbf{Intervalle:} 1 sem. (2 semaine pour ajouter/remplacer) \\ \textbf{MP:} –10 (volontaire) or –30 (forcé) \\ \textbf{SV:} (1d10 $\div$ 2, round up) \\ En surveilalnt les rappels mémoriels (invoqués de manière forcés si nécessaire), les psychochirurgiens peuvent identifier où les souvenirs sont stockés dans el cerveau et les cibler pour suppression. Le stockage de la mémoire est complexe et diffus cependant et souvent lié à d'autres souvenirs, supprimer l'un d'entres eux peut donc en affecter d'autres (à la discrétion du maître de jeu). AJouter ou rempalcer des souvenir est une opération bien plus complexe et requiert que de tels souvenirs soient copié depuis quelqu'un qui les as vécus ou créé avec des logiciels XP. Même lorsqu'ils sont implantés avec succès, de faux souvenirs peuvent se heurter avec d'autres (vrais) souvenirs à moins qu'ils n'aient également étés effacés. 

\subsubsection{Édition de Personnalité} \textbf{Intervalle:} 1 sem.\\ \textbf{MP:} Mineur –10; Modéré –20, Majeur –30 \\ \textbf{SV:} (1d10 $\div$ 2, round up) + 3 \\ Probablement la procédure psychochirurgicale la plus radicale, l'édition de personnalité implique d'altérer les traits au cœur de la personnalité du sujet. Les facteurs de personnalité qui peuvent être modifiés sont quasiment illimités et incluent des traits tels que l'ouverture d'esprit, la conscience, l'altruisme, l'extraversion/l'introversion, l'impulsivité, la curisosité, la confiance, l'orientation sexuelle et le contrôle de soi, parmi tant d'autres. Ces traits peuvent être augmentés ou atténués à différents degrés. L'effet est largement reflété par le roleplay mais le maître de jeu peut appliquer des modificateurs de manière qu'il pense appropriée. 

\subsubsection{Psychotorture} \textbf{Intervalle:} Variable \\ \textbf{MP:} +20 \\ \textbf{VS:} 1d10 de VS par jour\\ La psychotorture est une manipulation mentale dont le seul but est de causer douleur et anxiété, réflétés en terme de jeu par du stress mental et des traumas. La torture prolongée peut amener à de sérieux dérangements mentaux ou pire. 

\subsubsection{Psychothérapie} \textbf{Intervalle:} Variable \\ \textbf{MP:} +0 \\ \textbf{VS:} 0 \\ La psychochirurgie thérapeutique est bénéfique pour un personnage souffrant de stress mental, de traumatisme ou de désordre. Un test réussi de Psychochirurgie apporte un modificateur de +30 aux Test de Guérison Mentale, telles que notée p. 215. 

\subsubsection{Impression de Compétence} \textbf{intervalle:} 1 sem. par +10 \\ \textbf{MP:} +0 \\ \textbf{VS:} 1 par +10\\ Impression de compétences est l'utilisation de la psychochirurgie pour insérer des un ensemble de schémas neuronaux de compétence dans le cerveau du sujet, améliorant temporairement ses capacités. Cepedant, les impressions de compétences sont des amélioration artificielles et se dégradent au rythme de -10 par jour. Aucune compétence ne peut être améliorées au-delà de 60. 

\subsubsection{Suppression de Compétence} \textbf{Intervalle:} 1 jour par –10 \\ \textbf{MP:} –10 \\ \textbf{VS:} 1 par +10\\ La suppression de compétence tente d'identifier la zone du cerveau dans laquelle sont stockée les compétence et tente ensuite de les bloquer ou de les supprimer. La compétence du sujet est affaibliées et peut même être complètement perdue. 

\subsubsection{Excitation} \textbf{Intervalle:} 1 jour \\ \textbf{MP:} +10 \\ \textbf{VS:} 1 \\ Excitation est l'utilisation de technique de stimulation des cortex cérébraux les plus profonds afin de chatouiller les centres cognitifs du plaisir. Bien que cette procédure soit souvent utilisée à but thérapeutique pour les patients atteint de dpéression ou d'autres maladies mentales, le but de l'excitation est de surcharger le sujet dans un état prolongé de bétatitude à peine supportable. Une telle stimulation est cependant hautement addictive, un personnage qui y serait exposé pour n'importe quelle durée (au-delà d'1 heure subejctive) devraient développer le trait Addiction (p. 148). Quelques organisations criminelles sont connues pour utiliser l'excitation, aussi bien dasn le but de rendre accroc que comme récompense, comem moyen de contrôller ceux qui sont sous leur autorité. 



\subsection{Les Égarés} 

\begin{quotation} $\prec$ begin excerpt $\succ$ \\ Projet PSICLONE - Réunion Trimestrielle du Comité\\ 2° Trimestre 8 AF\\ Conclusion du Projet FUTURA— \\ Rapport sommaire\\ Préparé par le Dr. Amelia Sheppard	\\ Suite aux requètes reçues, j'ai compilé une revue du projet Futura et de ses retombées, 5 ans après que l'attention de dénionciateurs et des médias ne nous aient forcés à terminer le projet et à libérer les sujets survivants (appelées "les Égarés" par les médias populaires). 

Futura était une initiative commune dont le fer de lance était Hanto Génomics et fortement soutenu par Cognite, ainsi que d'autres partenaires (liste complète). Le projet avait été initiallement proposé par mon mentor, le Dr. Antonio Pascal, dont l'équipe a prouvé la faisabilité de l'Apprentissage par Expérience de Vie Accélérée (ÆVA) après une série d'études pilotes sur deux échantillons réduits (N $ \prec $ 1000). Même si il est vrai ue ces études pilotes préliminaires ont utilisé à la fois des sujets plus vieux et une dilatation temporelle moindre, le raisonnement derrière le programme ambitieux du Projet Futura était justifié par une décroissance remarquable de la population transhumaine suite à la Chute, par un taux de croissance ed la popuplatioàn stagant sur l'ensemble du système solaire (imputé à divers facteurs, incluant une longévité augmentée, une contraception disponible et la hausse du désespoir pendant des temps troublés), aussi bien que par le désir de se positionner de manière agressive sur de nouveaux secteurs technologiques dans l'espoir d'obtenir des avantages compétitifs. 

Futura démarra immédiatement au début de la Chute avec un échantillon de populatio initial de *** sujets de test extrait de matériau génétique existant et mis en gestation pendant une durée comprise entre 1 semaine et 6 mois après la naissance. Moins de 10\% d'entres eux sont nés de soi d'un remplaçant ou d'une mère génétique qui parit durant la Chute. La majorité venait de nos laboratoires Lunaire et Martien et furent menés à terme avec un exutérus. 

Après que l'échantillon fut sélectionné, les sujets ont été incarné dans des biomorph à croissance rapide de type fututra et installés dans des environnement personnalisé d'apprentissage accéléré en simulspace. Le projet fit un usage intensif de technologie émergente et de techniques extraites de l'aboratoires repris aux TITAN, inclauant les traits néogénétique des morphs futura ainsi que les applications de distortion temporelle pour des populations captives de simulspaces. Futura fonctionna en concurrence sur trois stations de recherche différente avec un encadrement cumulé de 2 211 chercheurs et personnel d'encadrement ainsi que 45 IAGs programmés spécifiquement pour l'expertise en développement pédiatrique. Les objectifs du projet étaient d'amener chaque enfant à une expérience de 18 ans en temps subjectif, en une durée de 3 ans de temps objectif. 

En dépit d'observation omniprésente et d'ajustement en temps-réel du simulspace et des programmes éducatifs pour une normalité optimale, quelque chose dans le cours du projet subit une panne au niveau de la gestion du de la qualité et du suivi des paramètres qui amena à un échec presque total de la modélisation empathique. Nous avosn observé ces effets pour la première fois 11 mois après le début du projet lorsque l'on des sujets avait vieilli jusqu'à un âge approximatif de six ans. Des incidents de cruauté animal et d'expression ont surgi, bien que à ce moment, ils restaient dans des standards acceptables. Dans les mois suivants, cette tendance continua et le Dr. Pascal autorisa l'usage dun modèle "éducatifs" plus agressif pour tenter de corriger les comportements limite sociopathique qui était exposés par 23,19\% des sujets à 18 mois (agés de neuf ans). 

Nous savons maintenant que ces changements ont eu des conséquence innatendue de supprimer les étalages francs de cruauté et de violence et d'apprendre à quasiment tous les sujets comment dissimuler leur psychoses. C'est également à ce moment que les premières morts se produisirent. Nous pensâmes à la vague initiales comme étant des accidents et les victimes aussi bien que les auteurs étaient généralement récupérés d'une sauvegarde datant d'une semaine en temps subjectif. Une analyse post-projet montre maintenant que 43,87\% de nos sujets étaient impliquée dans au moins un acte de meurtre prémédité à 24 mois (12 ans d'âge) et les protocoles de consultations n'as fait que leur apprendre comment mentir plus efficacement. 

Ce fut à ce moment que le Dr Aaron Bharani et moi-même avons commencer à vouloir arréter le projet et à amener les sujets en temps réel et en consultations intense. Le Dr. Pascal a opposé son veto à nos inquiétudes sans même les amener jusqu'au panel de direction du projet. Alors que le projet spiralait vers sa conclusion, un fork du Dr. Bharani révéla le projet au public au bout de 34 mois, déclenchant une tempête de controverse. Alors que le Dr. Pascal réussissait à tenir ses enquèteurs, espérant voir le projet arriver à sa conclusion, l'incident à notre station de recherche Legacy se produidit. L'enquète initiale conclut qu'au moins un des sujets s'était échappé du programme et était en fait responsable des pannes du système environnemental de l'habitat et des milliers de morts qui découlèrent de l'incident. 

Faisant face à des examens public et privés minutieux, beaucoup des partenaires impliqués dans le projet tentèrent de s'en défaire  et même de détruire toute trace de leur implication. Dans le chaos résultant, un nombre de sujets estimés à *** ont été silencieusement libéré dans la population générale du système. Ce ne fût qu'àprès cet incident que tous les sujets ont étés identifiés comme ayant été infectés avec la souche Watts-MacLeod du virus Exsurgent, bien que le moment et la manière dont cette infection eut lieu reste trouble et incertaine. Bien que des ordres tardifs amenèrent tous les sujets restants à être euthanasiés et/ou sauvegardés en stockage froid, seul *** des sujets libérés furent recapturés. Dans la population restante, *** ont cherché un refuge chez des autorités sympathisantes, *** se sont révélées au public et se sont soumit à une psychothérapie intense, *** furent tués dans des incidents violents et n'ont pas rescussités et le reste est supposé se cacher quelque part. 

$\prec$ end excerpt $\succ$ \end{quotation} 